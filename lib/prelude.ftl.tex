\documentclass{article}

\usepackage[utf8]{inputenc}
\usepackage[english]{babel}
\usepackage{../lib/tex/naproche}

\title{Prelude}
\author{}
\date{}

\begin{document}
  \pagenumbering{gobble}

  \maketitle

  This module implements a "Prelude" for Naproche: Those notions that were built-in
  in the last release and need to be specified in Naproche texts now.
  It is a shortened version of the Morse Kelley file in the same directory,
  since the longer version adds many useful notions that make it harder
  for provers to prove the tasks that they could prove before.

  This module implements the Morse-Kelley (MK) set theory,
  an extension of ZFC. Any lemma that can be proven in ZFC
  also can be proven in MK, but unlike ZFC, MK is finitely axiomatizable,
  that is, we can describe it without using schemata over arbitrary formulas.
  The axiom schema of class formation has been omitted in this presentation
  because it is built into the compiler.

  \begin{forthel}
    [synonym set/-s]
    [synonym relation/-s]
    [synonym function/-s]

    \begin{signature}
      A set is a notion.
    \end{signature}

    Let $B, C, D$ denote classes.
    Let $s, t, u$ denote sets.
    Let $w, x, y, z$ denote objects.

    \begin{axiom}
      $s$ is a class.
    \end{axiom}
    
    \begin{axiom}
      $s$ is an object.
    \end{axiom}

    \begin{definition}
      $x$ is in $B$ iff $x$ is an element of $B$.
    \end{definition}

    \begin{axiom}[SubclassOfSet]
      Assume for any object x such that x is an element of $B$ x is an element of $s$. Then $B$ is a set.
    \end{axiom}

    \begin{definition}
      $x \in B$ iff $x$ is in $B$.
    \end{definition}

    \begin{definition}
      $x \notin B$ iff $x$ is not in $B$.
    \end{definition}

    %This axiom is implicitly used in the compiler.
    \begin{axiom}[Ext]
      If for every object $x$ $x$ is an element of $C$ iff $x$ is an element of $B$ then $C = B$.
    \end{axiom}

    \begin{signature}
      The empty set is a set.
    \end{signature}

    \begin{axiom}[Empty]
      $x$ is not an element of the empty set.
    \end{axiom}

    \begin{signature}
      The pair of x and y is a set.
    \end{signature}

    \begin{axiom}[Pair]
      x is in the pair of y and z iff x = y or x = z.
    \end{axiom}

    \begin{lemma}
      The pair of x and y is { v | v = x or v = y }.
    \end{lemma}

    \begin{definition}
      The singleton of x is the pair of x and x.
    \end{definition}

    \begin{definition}
      The ordered pair of x and y is the pair of (the singleton of x) and the pair of x and y.
    \end{definition}

    \begin{definition}
      (x, y) is the ordered pair of x and y.
    \end{definition}

    \begin{lemma}[OrdPair]
      If (x, y) = (z, w) then x = z and y = w.
    \end{lemma}

    \begin{definition}
      The product of C and B is { (u, v) | u is a object and v is a object and u is in C and v is in B }.
    \end{definition}

    \begin{signature}
      A relation is a class.
    \end{signature}

    \begin{axiom}[RelationIntro]
      C is a relation iff for every x such that x is in C
        there is a y such that there is a z such that x is the ordered pair of y and z.
    \end{axiom}

    Let R, S denote relations.

    \begin{definition} 
      The domain of R is { object u | there is a object v such that (u, v) is in R }.
    \end{definition}

    \begin{definition} 
      \Dom{R} is the domain of R.
    \end{definition}
    
    \begin{signature}
      A function is a relation.
    \end{signature}

    \begin{axiom}[FunctionIntro]
      R is a function iff for all x, y, z such that (x, y) is in R and (x, z) is in R y = z.
    \end{axiom}

    Let $F, G$ denote functions.

    \begin{signature}
      The value of F at x is an object.
    \end{signature}

    \begin{axiom}
      Assume $x \in \Dom{F}$.
      The value of F at x is a object y such that (x, y) is in F.
    \end{axiom}

    \begin{definition}
      Assume $x \in \Dom{F}$.
      F(x) is the value of F at x.
    \end{definition}
  \end{forthel}
\end{document}