\documentclass{article}
\usepackage{../lang/naproche}
\usepackage{xstring}
\usepackage{hyperref}

\hypersetup{
  colorlinks=true,
  linkcolor=blue,
  urlcolor=blue
}

\newcommand\printlicense[3][CcZero]{\section*{License}
  \textcopyright\,#2, #3. This work is licensed under a
  \IfStrEqCase{#1}{
    {CcZero}{\href
      {https://creativecommons.org/publicdomain/zero/1.0/}
      {Creative Commons ``CC0 1.0 Universal''}
      (CC0 1.0)
    }
    {CcBy}{\href
      {https://creativecommons.org/licenses/by/4.0/}
      {Creative Commons ``Attribution 4.0 International''}
      (CC BY 4.0)
    }
    {CcBySa}{\href
      {https://creativecommons.org/licenses/by-sa/4.0/}
      {Creative Commons ``Attribution-ShareAlike 4.0 International''}
      (CC BY-SA 4.0)
    }
    {CcByNc}{\href
      {https://creativecommons.org/licenses/by-nc/4.0/}
      {Creative Commons ``Attribution-NonCommercial 4.0 International''}
      (CC BY-NC 4.0)
    }
    {CcByNcSa}{\href
      {https://creativecommons.org/licenses/by-nc-sa/4.0/deed.en}
      {Creative Commons ``Attribution-NonCommercial-ShareAlike 4.0 International''}
      (CC BY-NC-SA 4.0)
    }
    {CcByNd}{\href
      {https://creativecommons.org/licenses/by-nd/4.0/}
      {Creative Commons ``Attribution-NoDerivatives 4.0 International''}
      (CC BY-ND 4.0)
    }
    {CcByNcNd}{\href
      {https://creativecommons.org/licenses/by-nc-nd/4.0/}
      {Creative Commons ``Attribution-NonCommercial-NoDerivatives 4.0 International''}
      (CC BY-NC-ND 4.0)
    }
  }
  license.
}
\newcommand\printcopyright[2]{\section*{Copyright}
  \textcopyright\,#1, #2.
}

\addbibresource{../../latex/lib/bibliography}

\title{The Barber Paradox}
\author{\Naproche formalization:\\[0.5em]Marcel Schütz}
\date{2024}

\begin{document}
  \pagenumbering{gobble}
  \maketitle

  The \emph{Barber Paradox} is a puzzle illustrating Russell's
  Paradox, formulated by Bertrand Russell in 1919 \cite[p. 355]{Russell1919}:

  \begin{quotation}
    \noindent You can define the barber as ``one who shaves all those, and 
    those only, who do not shave themselves''.
    The question is, does the barber shave himself?
  \end{quotation}
  
  \noindent In \Naproche we can formalize this paradox as follows.

  \begin{forthel}
    [synonym person/-s]

    \begin{signature*}
      A person is an object.
    \end{signature*}
  
    [synonym shave/-s]

    \begin{signature*}
      Let $P, P'$ be persons.
      $P$ shaves $P'$ is a relation.
      Let $P$ shaves himself stand for $P$ shaves $P$.
    \end{signature*}

    \begin{signature*}
      The barber is a person such that for any person $P$ the barber shaves $P$ iff $P$ does not shave himself.
    \end{signature*}
    
    \begin{theorem*}[Barber Paradox]\label{barber_paradox}
      The barber shaves himself iff the barber does not shave himself.
    \end{theorem*}
  \end{forthel}

  \printbibliography
  \printlicense[CcByNcSa]{2024}{Marcel Schütz}
\end{document}
