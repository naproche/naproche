\documentclass{article}
\usepackage{../lang/naproche}
\usepackage{xstring}
\newcommand\printlicense[3][CcZero]{\section*{License}
  \textcopyright\,#2, #3. This work is licensed under a
  \IfStrEqCase{#1}{
    {CcZero}{\href
      {https://creativecommons.org/publicdomain/zero/1.0/}
      {Creative Commons ``CC0 1.0 Universal''}
      (CC0 1.0)
    }
    {CcBy}{\href
      {https://creativecommons.org/licenses/by/4.0/}
      {Creative Commons ``Attribution 4.0 International''}
      (CC BY 4.0)
    }
    {CcBySa}{\href
      {https://creativecommons.org/licenses/by-sa/4.0/}
      {Creative Commons ``Attribution-ShareAlike 4.0 International''}
      (CC BY-SA 4.0)
    }
    {CcByNc}{\href
      {https://creativecommons.org/licenses/by-nc/4.0/}
      {Creative Commons ``Attribution-NonCommercial 4.0 International''}
      (CC BY-NC 4.0)
    }
    {CcByNcSa}{\href
      {https://creativecommons.org/licenses/by-nc-sa/4.0/deed.en}
      {Creative Commons ``Attribution-NonCommercial-ShareAlike 4.0 International''}
      (CC BY-NC-SA 4.0)
    }
    {CcByNd}{\href
      {https://creativecommons.org/licenses/by-nd/4.0/}
      {Creative Commons ``Attribution-NoDerivatives 4.0 International''}
      (CC BY-ND 4.0)
    }
    {CcByNcNd}{\href
      {https://creativecommons.org/licenses/by-nc-nd/4.0/}
      {Creative Commons ``Attribution-NonCommercial-NoDerivatives 4.0 International''}
      (CC BY-NC-ND 4.0)
    }
  }
  license.
}
\addbibresource{../../latex/lib/bibliography}

\newcommand{\varzero}{x_0}
\newcommand{\abs}[2]{\lambda#1.\ #2}
\newcommand{\app}[2]{(#1)(#2)}
\newcommand{\fix}{\texttt{fix}}

\title{Curry's Paradox}
\author{\Naproche formalization:\\[0.5em]Marcel Schütz}
\date{2024}

\begin{document}
  \maketitle

  \noindent \emph{Curry's Paradox} is a paradox described by Haskell Curry in 
  1942 \cite{Curry1942}.
  It allows the derivation of an arbitrary statement from a self-referential 
  expression that presupposes its own validity.

  \section*{Untyped $\lambda$-Calculus}

  \noindent Our formalization of Curry's Paradox in \Naproche is based on an 
  untyped $\lambda$-calculus:

  \begin{forthel}
    [synonym expression/-s]

    \begin{signature*}
      An expression is a notion.
    \end{signature*}
  
    Let $E, E'$ denote expressions.
  
    \begin{signature*}
      A variable is an expression.
    \end{signature*}
  
    \begin{signature*}
      $\varzero$ is a variable.
    \end{signature*}
  
    \begin{signature*}[Abstraction]\label{abstraction}
      Let $x$ be a variable.
      $\abs{x}{E}$ is an expression.
    \end{signature*}
  
    \begin{signature*}[Application]\label{application}
      $\app{E}{E'}$ is an expression.
    \end{signature*}
  
    \begin{signature*}[Fixed-point combinator]\label{fixed_point_combinator}
      $\fix$ is an expression such that
      \[\app{\fix}{E} = \app{E}{\app{\fix}{E}}\]
      for every expression $E$.
    \end{signature*}
  \end{forthel}


  \section*{Propositional Logic}

  \noindent We extend this $\lambda$-calculus by a fragment of propositional
  logic:

  \begin{forthel}  
    Let $E, E'$ denote expressions.
    Let $x$ denote a variable.

    \begin{signature*}[Implication]\label{implication}
      $E \rightarrow E'$ is an expression.
    \end{signature*}
  
    \begin{signature*}[Truth]\label{truth}
      $E$ is true is a relation.
    \end{signature*}
  
    \begin{axiom*}[$\beta$-reduction]\label{beta_reduction}
      $\app{\abs{x}{x \rightarrow E'}}{E} = E \rightarrow E'$.
    \end{axiom*}
  
    \begin{axiom*}[Reflexivity]\label{reflexivity}
      $E \rightarrow E$ is true.
    \end{axiom*}
  
    \begin{axiom*}[Modus Ponens]\label{modus_ponens}
      If $E$ is true and $E \rightarrow E'$ is true then $E'$ is true.
    \end{axiom*}
  
    \begin{axiom*}[Strengthening]\label{strengthening}
      If $E \rightarrow (E \rightarrow E')$ is true then $E \rightarrow E'$ is true.
    \end{axiom*}
  \end{forthel}


  \section*{Curry's paradox}

  \noindent Using the fixpoint combinator from above we can formulate a 
  self-referential expression $X$ of the form ``If $X$ is true then $E$ is
  true'' for any arbitrary expression $E$ by defining
  $X = \app{\fix}{\abs{\varzero}{\varzero \rightarrow E}}$.
  From the existence of such an expression $X$ together with the above axioms 
  we can then derive that any expression $E$ is true.

  \begin{forthel}
    \begin{theorem*}[Curry's paradox]\label{curry_paradox}
      Every expression is true.
    \end{theorem*}
    \begin{proof}
      Let $E$ be an expression.
      Take $N = \abs{\varzero}{\varzero \rightarrow E}$ and $X = \app{\fix}{N}.$
  
      (1) Then $X = \app{N}{X} = X \rightarrow E$ (by \nameref{beta_reduction}).
  
      Hence $X \rightarrow (X \rightarrow E)$ is true (by 1, \nameref{reflexivity}).
  
      (2) Thus $X \rightarrow E$ is true (by \nameref{strengthening}).
  
      (3) Therefore $X$ is true (by 1, 2).
  
      Consequently $E$ is true (by \nameref{modus_ponens}, 2, 3).
    \end{proof}
  \end{forthel}

  \printbibliography
  \printlicense[CcByNcSa]{2024}{Marcel Schütz}
\end{document}
