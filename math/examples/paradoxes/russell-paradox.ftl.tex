\documentclass{article}
\usepackage{naproche}
\libinput[latex]{example-preamble}
\libinput[libraries/foundations]{preamble}

\title{Russell's Paradox}
\author{\Naproche formalization:\\[0.5em]Marcel Schütz}
\date{2024}

\begin{document}
\begin{smodule}{russell-paradox.ftl}
  \pagenumbering{gobble}
  \maketitle
  \hidetlsnumbers

  \noindent \emph{Russell's Paradox} is a set-theoretical paradox discovered by
  Bertrand Russell around 1902 \cite[chapter XV]{Frege1980} which shows that 
  there exist statements $\varphi$ whose extension
  $\{x\mid\varphi(x)\}$ cannot constitute a set -- or in other words:
  Not every class is a set.

  \inputref[libraries/foundations]{russell-paradox.ftl.tex}

  \printbibliography
  \printlicense[CcByNcSa]{2024}{Marcel Schütz}
\end{smodule}
\end{document}
