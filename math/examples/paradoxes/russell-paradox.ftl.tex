\documentclass{article}
\usepackage{../lang/naproche}
\usepackage{xstring}
\newcommand\printlicense[3][CcZero]{\section*{License}
  \textcopyright\,#2, #3. This work is licensed under a
  \IfStrEqCase{#1}{
    {CcZero}{\href
      {https://creativecommons.org/publicdomain/zero/1.0/}
      {Creative Commons ``CC0 1.0 Universal''}
      (CC0 1.0)
    }
    {CcBy}{\href
      {https://creativecommons.org/licenses/by/4.0/}
      {Creative Commons ``Attribution 4.0 International''}
      (CC BY 4.0)
    }
    {CcBySa}{\href
      {https://creativecommons.org/licenses/by-sa/4.0/}
      {Creative Commons ``Attribution-ShareAlike 4.0 International''}
      (CC BY-SA 4.0)
    }
    {CcByNc}{\href
      {https://creativecommons.org/licenses/by-nc/4.0/}
      {Creative Commons ``Attribution-NonCommercial 4.0 International''}
      (CC BY-NC 4.0)
    }
    {CcByNcSa}{\href
      {https://creativecommons.org/licenses/by-nc-sa/4.0/deed.en}
      {Creative Commons ``Attribution-NonCommercial-ShareAlike 4.0 International''}
      (CC BY-NC-SA 4.0)
    }
    {CcByNd}{\href
      {https://creativecommons.org/licenses/by-nd/4.0/}
      {Creative Commons ``Attribution-NoDerivatives 4.0 International''}
      (CC BY-ND 4.0)
    }
    {CcByNcNd}{\href
      {https://creativecommons.org/licenses/by-nc-nd/4.0/}
      {Creative Commons ``Attribution-NonCommercial-NoDerivatives 4.0 International''}
      (CC BY-NC-ND 4.0)
    }
  }
  license.
}
\addbibresource{../../latex/lib/bibliography}

\title{Russell's Paradox}
\author{\Naproche formalization:\\[0.5em]Marcel Schütz}
\date{2024}

\begin{document}
\pagenumbering{gobble}
\maketitle

\noindent \emph{Russell's Paradox} is a set-theoretical paradox discovered by
Bertrand Russell around 1902 \cite[chapter XV]{Frege1980} which shows that 
there exist statements $\varphi$ whose extension
$\{x\mid\varphi(x)\}$ cannot constitute a set -- or in other words:
Not every class is a set.

\begin{forthel}
  \begin{theorem}[Russell's Paradox]
    There exists a class that is not a set.
  \end{theorem}
  \begin{proof}[forthel]
    Assume the contrary.
    Define $R = \{x \mid x \text{ is a set such that } x \notin x\}$.
    Then $R$ is a set.
    Hence $R \in R$ iff $R \notin R$.
    Contradiction.
  \end{proof}
\end{forthel}

\printbibliography
\printlicense[CcByNcSa]{2024}{Marcel Schütz}
\end{document}
