\documentclass{article}
\usepackage{../lang/naproche}
\usepackage{xstring}
\newcommand\printlicense[3][CcZero]{\section*{License}
  \textcopyright\,#2, #3. This work is licensed under a
  \IfStrEqCase{#1}{
    {CcZero}{\href
      {https://creativecommons.org/publicdomain/zero/1.0/}
      {Creative Commons ``CC0 1.0 Universal''}
      (CC0 1.0)
    }
    {CcBy}{\href
      {https://creativecommons.org/licenses/by/4.0/}
      {Creative Commons ``Attribution 4.0 International''}
      (CC BY 4.0)
    }
    {CcBySa}{\href
      {https://creativecommons.org/licenses/by-sa/4.0/}
      {Creative Commons ``Attribution-ShareAlike 4.0 International''}
      (CC BY-SA 4.0)
    }
    {CcByNc}{\href
      {https://creativecommons.org/licenses/by-nc/4.0/}
      {Creative Commons ``Attribution-NonCommercial 4.0 International''}
      (CC BY-NC 4.0)
    }
    {CcByNcSa}{\href
      {https://creativecommons.org/licenses/by-nc-sa/4.0/deed.en}
      {Creative Commons ``Attribution-NonCommercial-ShareAlike 4.0 International''}
      (CC BY-NC-SA 4.0)
    }
    {CcByNd}{\href
      {https://creativecommons.org/licenses/by-nd/4.0/}
      {Creative Commons ``Attribution-NoDerivatives 4.0 International''}
      (CC BY-ND 4.0)
    }
    {CcByNcNd}{\href
      {https://creativecommons.org/licenses/by-nc-nd/4.0/}
      {Creative Commons ``Attribution-NonCommercial-NoDerivatives 4.0 International''}
      (CC BY-NC-ND 4.0)
    }
  }
  license.
}
\addbibresource{../../latex/lib/bibliography}

\title{The Drinker Paradox}
\author{\Naproche formalization:\\[0.5em]Marcel Schütz}
\date{2024}

\begin{document}
  \pagenumbering{gobble}
  \maketitle

  \begin{forthel}
    [read \path{examples/lang/everyday-ontology.ftl.tex}]
  \end{forthel}

  \noindent The \emph{Drinker Paradox} is a principle of classical predicate 
  logic popularised by the logician Raymond Smullyan in his 1978 book
  \textit{What Is the Name of this Book?} \cite{Smullyan1978} which can be 
  stated as:

  \begin{quotation}
    \noindent There is someone in the pub such that, if he is drinking then 
    everyone in the pub is drinking.
  \end{quotation}

  \begin{forthel}
    \begin{signature*}
      The pub is an object.
    \end{signature*}
    
    \begin{theorem*}[Drinker Paradox]\label{drinker_paradox}
      Assume that there is a person inside the pub.
      Then there is a person $P$ inside the pub such that if $P$ is drinking then every person inside the pub is drinking.
    \end{theorem*}
    \begin{proof}[by case analysis]
      Case every person inside the pub is drinking.
        Choose a person $P$ inside the pub.
        Then $P$ is drinking and every person inside the pub is drinking.
        Hence if $P$ is drinking then every person inside the pub is drinking.
      End.
    
      Case there is a person inside the pub that is not drinking.
        Consider a person $P$ inside the pub that is not drinking.
        Then if $P$ is drinking then every person inside the pub is drinking.
      End.
    \end{proof}
  \end{forthel}

  \printbibliography
  \printlicense[CcByNcSa]{2024}{Marcel Schütz}
\end{document}
