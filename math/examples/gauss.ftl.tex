\documentclass{article}
\usepackage{naproche}
\libinput[latex]{example-preamble}
\libinput[libraries/foundations]{preamble}
\libinput[libraries/arithmetics]{preamble}

\newcommand{\gauss}{\mathcal{G}}

\title{Little Gauß' Theorem}
\author{\Naproche formalization:\\[0.5em]Christian Schöner, Marcel Schütz}
\date{2024}

\begin{document}
  \pagenumbering{gobble}
  \maketitle

  \usemodule[libraries/arithmetics]{multiplication.ftl}

  \noindent This is a formalization of ``Little Gauß' Theorem'', i.e. of
  the assertion that
  \[\sum_{k=\NATzero}^n k = \frac{k(k \NATplus \NATone)}\NATtwo\]
  for all $n \in \Nat$.
  In \Naproche we can define the sum $\sum_{n=\NATzero}^k n$ by the function
  $\gauss : \Nat \to \Nat$ with $\gauss(\NATzero) = \NATzero$ and
  $\gauss(n \NATplus \NATone) = \gauss(n) \NATplus (n \NATplus \NATone)$:

  \begin{forthel}
    \begin{signature*}
      Let $n$ be a natural number.
      $\gauss(n)$ is a natural number.
    \end{signature*}

    \begin{axiom*}
      $\gauss(\NATzero) = \NATzero$.
    \end{axiom*}

    \begin{axiom*}
      Let $n$ be a natural number.
      Then $\gauss(n \NATplus \NATone) = \gauss(n) \NATplus (n \NATplus \NATone)$.
    \end{axiom*}
  \end{forthel}

  \noindent Then the theorem can be stated as follows.

  \begin{forthel}
    \begin{theorem*}[title=Little Gauß,id=little_gauss]
      For all natural numbers $n$ we have
      \[\NATtwo \NATmul \gauss(n) = n \NATmul (n \NATplus \NATone).\]
    \end{theorem*}
    \begin{proof}
      Define $\Phi = \{n \in \Nat \mid \NATtwo \NATmul \gauss(n) = n \NATmul (n \NATplus \NATone)\}$.

      (1) $\NATzero \in \Phi$.

      (2) For all $n \in \Phi$ we have $n \NATplus \NATone \in \Phi$.
      \begin{proof}
        Let $n \in \Phi$.
        Then $\NATtwo \NATmul \gauss(n) = n \NATmul (n \NATplus \NATone)$.
        Hence $\NATtwo \NATmul \gauss(n \NATplus \NATone)
          = \NATtwo \NATmul (\gauss(n) \NATplus (n \NATplus \NATone))
          = (\NATtwo \NATmul \gauss(n)) \NATplus (\NATtwo \NATmul (n \NATplus \NATone))
          = (n \NATmul (n \NATplus \NATone)) \NATplus (\NATtwo \NATmul (n \NATplus \NATone))
          = ((n \NATplus \NATone) \NATmul n) \NATplus ((n \NATplus \NATone) \NATmul \NATtwo)
          = (n \NATplus \NATone) \NATmul (n \NATplus \NATtwo)
          = (n \NATplus \NATone) \NATmul ((n \NATplus \NATone) \NATplus \NATone)$.
        Therefore $n \NATplus \NATone \in \Phi$.
      \end{proof}

      Thus $\Phi$ contains every natural number.
      Consequently $\NATtwo \NATmul \gauss(n) = n \NATmul (n \NATplus \NATone)$ for every $n \in \Nat$.
    \end{proof}
  \end{forthel}
  \printlicense[CcByNcSa]{2024}{Christian Schöner, Marcel Schütz}
\end{document}
