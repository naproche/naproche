\documentclass[english]{article}
\usepackage{lang/naproche}
\providecommand\libarchive{}
\renewcommand\libarchive{Set Theory}

\usepackage{amssymb}

\newcommand{\gauss}{\mathcal{G}}

\title{Little Gauß' Theorem}
\author{\Naproche formalization:\\[0.5em]Christian Schöner, Marcel Schütz}
\date{2024}

\begin{document}
\pagenumbering{gobble}
\maketitle

\noindent This is a formalization of ``Little Gauß' Theorem'', i.e. of
the assertion that
\[\sum_{k = 0}^n k = \frac{k(k + 1)}2\]
for all $n \in \mathbb{N}$.

\section{Natural Numbers}

\begin{forthel}
  [read \path{examples/preliminaries.ftl.tex}]

  \begin{signature}
    A natural number is an object.

    Let $n, m, k$ denote natural numbers.
  \end{signature}

  \begin{definition}
    $\mathbb{N}$ is the class of all natural numbers.
  \end{definition}

  \begin{axiom}[Infinity Axiom]
    $\mathbb{N}$ is a set.
  \end{axiom}

  \begin{signature}
    $0$ is a natural number.

    Let $n$ is nonzero stand for $n \neq 0$.
  \end{signature}

  \begin{signature}
    $1$ is a natural number.
  \end{signature}
\end{forthel}


\subsection{Addition}

\begin{forthel}
  \begin{signature}
    $n + m$ is a natural number.
  \end{signature}

  \begin{axiom}[Addition Axiom I]
    $n + 0 = n$.
  \end{axiom}

  \begin{axiom}[Addition Axiom II]
    $n + (m + 1) = (n + m) + 1$.
  \end{axiom}

  \begin{axiom}[Commutativity of Addition]
    $n + m = m + n$.
  \end{axiom}

  \begin{axiom}[Associativity of Addition]
    $n + (m + k) = (n + m) + k$.
  \end{axiom}
\end{forthel}


\subsection{Multiplication}

\begin{forthel}
  \begin{signature}
    $n \cdot m$ is a natural number.
  \end{signature}

  \begin{axiom}
    $n \cdot 0 = 0$.
  \end{axiom}

  \begin{axiom}
    $n \cdot (m + 1) = (n \cdot m) + n$.
  \end{axiom}

  \begin{axiom}[Commutativity of Multiplication]
    $n \cdot m = m \cdot n$.
  \end{axiom}

  \begin{axiom}[Associativity of Multiplication]
    $n \cdot (m \cdot k) = (n \cdot m) \cdot k$.
  \end{axiom}

  \begin{axiom}[Left Distributivity]
    $n \cdot (m + k) = (n \cdot m) + (n \cdot k)$.
  \end{axiom}

  \begin{axiom}[Right Distributivity]
    $(n + m) \cdot k = (n \cdot k) + (m \cdot k)$.
  \end{axiom}
\end{forthel}


\subsection{The Peano Axioms}

\begin{forthel}
  \begin{axiom}[Peano Axiom I]
    If $n + 1 =  m + 1$ then $n = m$.
  \end{axiom}

  \begin{axiom}[Peano Axiom II]
    There exists no natural number $n$ such that $n + 1 = 0$.
  \end{axiom}

  \begin{axiom}[Peano Axiom III]
    $n = 0$ or $n = m + 1$ for some natural number $m$.
  \end{axiom}

  \begin{axiom}[Induction Axiom]
    $n \prec n + 1$.
  \end{axiom}
\end{forthel}


\subsection{Additional Numbers}

\begin{forthel}
  \begin{definition}
    $2 = 1 + 1$.
  \end{definition}
\end{forthel}


\section{Little Gauß' Theorem}

In \Naproche we can define the sum $\sum_{n = 0}^k n$ by the function
$\gauss : \mathbb{N} \to \mathbb{N}$ with $\gauss(0) = 0$ and
$\gauss(n + 1) = \gauss(n) + (n + 1)$:

\begin{forthel}
  Let $n$ denote a natural number.

  \begin{signature*}
    $\gauss(n)$ is a natural number.
  \end{signature*}

  \begin{axiom*}
    $\gauss(0) = 0$.
  \end{axiom*}

  \begin{axiom*}
    $\gauss(n + 1) = \gauss(n) + (n + 1)$.
  \end{axiom*}
\end{forthel}

\noindent Then the theorem can be stated as follows.

\begin{forthel}
  \begin{theorem*}[Little Gauß]\label{little_gauss}
    For all natural numbers $n$ we have
    \[2 \cdot \gauss(n) = n \cdot (n + 1).\]
  \end{theorem*}
  \begin{proof}[by induction on $n$]
    Let $n$ be a natural number.

    Case $n = 0$. Obvious.

    Case $n \neq 0$.
      Take a $m \in \mathbb{N}$ such that $n = m + 1$.
      Then $2 \cdot \gauss(m) = m \cdot n$.
      Hence
      \begin{align*}
            2 \cdot \gauss(n)
        & = 2 \cdot (\gauss(m) + n)           \\
        & = (2 \cdot \gauss(m)) + (2 \cdot n) \\
        & = (m \cdot n) + (2 \cdot n)         \\
        & = (m + 2) \cdot n                   \\
        & = n \cdot (n + 1)
      \end{align*}.
    End.
  \end{proof}
\end{forthel}

\printlicense[CcByNcSa]{2024}{Christian Schöner, Marcel Schütz}
\end{document}
