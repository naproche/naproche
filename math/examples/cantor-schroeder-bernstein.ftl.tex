\documentclass{article}
\usepackage{naproche}
\libinput[meta-inf]{example-preamble}
\libinput[libraries/set-theory]{preamble}

\title{The Knaster-Tarski fixed point theorem and \\
the Cantor-Schröder-Bernstein Theorem}
\author{\Naproche formalization:\\[0.5em]Alexander Holz, Marcel Schütz}
\date{2018}

\begin{document}
\begin{smodule}{cantor-schroeder-bernstein.ftl}
  \maketitle

  \usemodule[libraries/foundations]{equinumerosity.ftl}
  \usemodule[libraries/foundations]{maps-and-systems-of-sets.ftl}
  \usemodule[libraries/set-theory]{zfc.ftl}

  \noindent This is a formalization of the \textit{Knaster-Tarski Fixed Point
  Theorem} (i.e. of the assertion that every subset-preserving map has a fixed
  point) and of the \textit{Cantor-Schröder-Bernstein Theorem} (i.e. of the
  fact that two sets are equinumerous iff they can be embedded into each other),
  where the proof of the latter theorem is based on the former one, as in
  \cite{Schroeder2012}.

  \section*{The Knaster-Tarski Fixed Point Theorem}
  \inputref[libraries/set-theory]{knaster-tarski.ftl.tex}

  \section*{The Cantor-Schröder-Bernstein Theorem}
  \inputref[libraries/set-theory]{cantor-schroeder-bernstein.ftl.tex}

  \printbibliography
  \printlicense[CcByNcSa]{2018}{Alexander Holz, Marcel Schütz}
\end{smodule}
\end{document}
