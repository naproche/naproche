\documentclass[english]{article}

% Switch between sTeX and simple LaTeX
% sTeX:
%\usepackage{../latex/lib/naproche}
% Simple LaTeX
\usepackage{aux/naproche}

\renewcommand{\mod}{\text{mod }}
\newcommand{\Int}{\mathbb{Z}}

\title{The Chinese remainder theorem}
\author{Andrei Paskevich et. al.}
\date{2007 - 2021}

\begin{document}
  \maketitle

  The Chinese remainder theorem is a number theoretical result
  about the solution of simultaneous congruences in the case of
  coprime modules. The earliest known formulation of the theorem
  dates back to the Chinese mathematician Sun-tzu in the third
  century. In the following we present a formalization of a
  generalization of the theorem in terms of ideals in an
  integral domain. Checking
  the formalization takes about 3 minutes on a modest laptop.


  \section{Integral domain axioms}

  We assume that our universe is a fixed integral domain. We call elements of
  our universe simply ``elements''. In particular, we have two special elements,
  $0$ and $1$. Moreover, there is a unary operation, $-$, and two binary
  operations, $+$ and $\cdot$.

  \begin{forthel}
    [read \path{examples/aux/vocabulary.ftl.tex}]

    Let $a,b,c,x,y,z,u,v,w$ denote elements.

    \begin{signature}[id=SortsC]
      $0$ is an element.
    \end{signature}

    \begin{signature}[id=SortsC]
      $1$ is an element.
    \end{signature}

    \begin{signature}[id=Sortsu]
      $-x$ is an element.
    \end{signature}

    \begin{signature}[id=SortsB]
      $x + y$ is an element.
    \end{signature}

    \begin{signature}[id=SortsB]
      $x \cdot y$ is an element.
    \end{signature}

    Let $x$ is nonzero stand for $x \neq 0$.
    Let $x - y$ stand for $x + (-y)$.
  \end{forthel}

  To ensure that our operations form a commutative ring we have to state the
  appropriate axioms. First we make sure that the addition yields an abelian
  group.

  \begin{forthel}
    \begin{axiom}[id=AddComm]
      $x + y = y + x$.
    \end{axiom}

    \begin{axiom}[id=AddAsso]
      $(x + y) + z = x + (y + z)$.
    \end{axiom}

    \begin{axiom}[id=AddBubble]
      $x + (y + z) = y + (x + z)$.
    \end{axiom}

    \begin{axiom}[id=AddZero]
      $x + 0 = x = 0 + x$.
    \end{axiom}

    \begin{axiom}[id=AddInvr]
      $x + (-x) = 0 = -x + x$.
    \end{axiom}
  \end{forthel}

  In fact axiom \textit{AddBubble} is redundant. We can easily prove it from
  \textit{AddComm} and \textit{AddAsso}:

  \[ x + (y + z)
  \overset{AddComm}{=} (y + z) + x
  \overset{AddAsso}{=} y + (z + x)
  \overset{AddComm}{=} y + (x + z). \]

  Let us continue with the axioms that ensure that the multiplication yields a
  commutative monoid.

  \begin{forthel}
    \begin{axiom}[id=MulComm]
      $x \cdot y = y \cdot x$.
    \end{axiom}

    \begin{axiom}[id=MulAsso]
      $x \cdot (y \cdot z) = (x \cdot y) \cdot z$.

      % NOTE: In the original ftl file the lhs and the rhs of the equation are
      % swapped. But due to a bug in the parser the original version is not
      % accepted by Naproche.
    \end{axiom}

    \begin{axiom}[id=MulBubble]
      $x \cdot (y \cdot z) = y \cdot (x \cdot z)$.
    \end{axiom}

    \begin{axiom}[id=MulUnit]
      $x \cdot 1 = x = 1 \cdot x$.
    \end{axiom}
  \end{forthel}

  As above we can prove \textit{MulBubble} from \textit{MulComm} and
  \textit{MulAsso}. Now we ensure that the distribution laws hold.

  \begin{forthel}
    \begin{axiom}[id=AMDistr1]
      $x \cdot (y + z) = (x \cdot y) + (x \cdot z)$.
    \end{axiom}

    \begin{axiom}[id=AMDistr2]
      $(y + z) \cdot x = (y \cdot x) + (z \cdot x)$.
    \end{axiom}
  \end{forthel}

  The next two statements are some simple computation rules. The first one
  concerning multiplication with $-1$ can be derived from our previous laws
  together with \textit{MulZero}, even if we state it as an axiom here. We
  leave the proof of this claim as an exercise for the reader.

  \begin{forthel}
    \begin{axiom}[id=MulMnOne]
      $(-1) \cdot x = -x = x \cdot (-1)$.
    \end{axiom}

    \begin{lemma}[id=MulZero]
      $x \cdot 0 = 0 = 0 \cdot x$.
    \end{lemma}
    \begin{proof}
      Let us show that $x \cdot 0 = 0$.
        $x \cdot 0
        .= x \cdot (0 + 0)$ (by AddZero)
        $.= (x \cdot 0) + (x \cdot 0)$ (by AMDistr1).
      End.

      Let us show that $0 \cdot x = 0$.
        $0 \cdot x
        .= (0 + 0) \cdot x$ (by AddZero)
        $.= (0 \cdot x) + (0 \cdot x)$ (by AMDistr2).
      End.
    \end{proof}
  \end{forthel}

  There are two axioms remaining to ensure that our universe is not just a
  commutative ring but an integral domain: There must be no non-trivial
  zero-divisors and our ring must not be trivial.

  \begin{forthel}
    \begin{axiom}[id=Cancel]
      $x \neq 0 \wedge y \neq 0 \implies x \cdot y \neq 0$.
    \end{axiom}

    \begin{axiom}[id=UnNeZr]
      $1 \neq 0$.
    \end{axiom}
  \end{forthel}


  \section{Sets}

  Next we consider subsets of our universe. To keep our notion of sets as easy
  as possible we state that \textit{every} set is a subset of our universe.

  \begin{forthel}

    Let $X,Y,Z,U,V,W$ denote sets.

    \begin{axiom}
      Every element of $X$ is an object.
    \end{axiom}

    Let $x$ belongs to $W$ denote $x$ is an element of $W$.

    \begin{axiom}[id=SetEq]
      If every element of $X$ belongs to $Y$ and every element of $Y$ belongs to
      $X$ then $X = Y$.
    \end{axiom}

    \begin{definition}[id=DefSum]
      $X \oplus Y$ is a set such that for every element $z$ ($z \in X \oplus Y$)
      iff there exist $x \in X, y \in Y$ such that $z = x + y$.
    \end{definition}

    \begin{definition}[id=DefSInt]
      $X \cap Y$ is a set such that for every element $z$ ($z \in X \cap Y$) iff
      $z \in X$ and $z \in Y$.
    \end{definition}
  \end{forthel}


  \section{Ideals and the Chinese Remainder Theorem}

  Now we can define ideals as sets which are closed under certain operations.

  \begin{forthel}
    Let $a,b,c,x,y,z,u,v,w$ denote elements.

    \begin{definition}[id=DefIdeal]
      An ideal is a set $X$ such that for every $x \in X$ we have
        $\forall y \in X (x + y \in X)$ and
        $\forall z (z \cdot x \in X)$.
    \end{definition}

    Let $I,J$ denote ideals.
  \end{forthel}

  We can show that the sum and the intersection of two ideals is again an ideal.

  \begin{forthel}
    \begin{lemma}[id=IdeSum]
      $I \oplus J$ is an ideal.
    \end{lemma}
    \begin{proof}
      Let $x$ belong to ($I \oplus J$).

      $\forall y \in (I \oplus J) (x + y) \in (I \oplus J)$.
      \begin{proof}
        Let $y \in (I \oplus J)$.
        (1) Take $k \in I$ and $l \in J$ such that $x = k + l$.
        (2) Take $m \in I$ and $n \in J$ such that $y = m + n$.
        $k + m$ belongs to $I$ and $l + n$ belongs to $J$.
        $x + y .= (k + m) + (l + n)$ (by 1, 2, AddComm,AddAsso,AddBubble).
        Therefore the thesis.
      \end{proof}

      For every element $z$ $(z \cdot x) \in (I \oplus J)$.
      \begin{proof}
        Let $z$ be an element.
        (1) Take $k \in I$ and $l \in J$ such that $x = k + l$.
        $z \cdot k$ belongs to $I$ and $z \cdot l$ belongs to $J$.
        $z \cdot x .= (z \cdot k) + (z \cdot l)$ (by AMDistr1, 1).
        Therefore the thesis.
      \end{proof}
    \end{proof}


    \begin{lemma}[id=IdeInt]
      $I \cap J$ is an ideal (by DefIdeal).
    \end{lemma}
    \begin{proof}
      Let x belong to $I \cap J$.
      $\forall y \in (I \cap J) (x + y) \in (I \cap J)$.
      For every element $z$ $(z \cdot x) \in (I \cap J)$.
    \end{proof}
  \end{forthel}

  Now we can state the Chinese remainder theorem in terms of congruence modulo
  some ideal.

  \begin{forthel}
    \begin{definition}[id=DefMod]
      $x = y ~(\mod I)$ iff $x - y \in I$.
    \end{definition}

    \begin{theorem}[id=ChineseRemainder]
      Suppose that every element belongs to $I \oplus J$.
      Let $x, y$ be elements.
      There exists an element $w$ such that $w = x ~(\mod I)$ and
      $w = y ~(\mod J)$.
    \end{theorem}
    \begin{proof}
      Take $a \in I$ and $b \in J$ such that $a + b = 1$ (by DefSum).
      (1) Take $w = (y \cdot a) + (x \cdot b)$.

      Let us show that $w = x ~(\mod I)$ and $w = y ~(\mod J)$.

        $w - x$ belongs to $I$.
        \begin{proof}
          $w - x = (y \cdot a) + ((x \cdot b) - x)$.
          $x \cdot (b - 1)$ belongs to $I$.
          $x \cdot (b - 1) = (x \cdot b) - x$.
        \end{proof}

        $w - y$ belongs to $J$.
        \begin{proof}
          $w - y = (x \cdot b) + ((y \cdot a) - y)$.
          $y \cdot (a - 1)$ belongs to $J$.
          $y \cdot (a - 1) = (y \cdot a) - y$.
        \end{proof}
      End.
    \end{proof}
  \end{forthel}


  \section{Greatest common divisors and principal ideals}

  In this section we extend our integral domain to a Euclidean domain. To be
  able to do this we first have to establish a notion of natural numbers.

  \begin{forthel}
    Let $a,b,c,x,y,z,u,v,w$ denote elements.

    \begin{signature}[id=NatSort]
      A natural number is an object.
    \end{signature}

  \end{forthel}

  Now we can equip our domain with a Euclidean function $|\cdot|$.

  \begin{forthel}
    \begin{signature}[id=EucSort]
      Let $x$ be a nonzero element. $|x|$ is a natural number.
    \end{signature}

    \begin{axiom}[id=Division]
      Let $x,y$ be elements and $y \neq 0$.
      There exist elements $q,r$ such that $x = (q \cdot y) + r$ and
      $(r \neq 0 \implies |r| \prec |y|)$.
    \end{axiom}
  \end{forthel}

  The \textit{Division} axiom makes use of {$\mathbb{N}$aproche's} built-in
  induction scheme: For any statement $\varphi(x)$ (with one free variable $x$)
  and any element $r$ the following is true:
  \[ (\forall r' (|r'| \prec |r| \rightarrow \varphi(r'))) \rightarrow \varphi(r) \]
  This allows us to prove certain statements about $r$ by induction on $|r|$.

  \paragraph{} Next let us have a look at the notion of \textit{divisors} and,
  in particular, \textit{greatest common divisors} (\textit{gcd}s).

  \begin{forthel}
    \begin{definition}[id=DefDiv]
      $x$ divides $y$ iff for some $z$ $(x \cdot z = y)$.
    \end{definition}

    Let $x \mid y$ stand for $x$ divides $y$.
    Let $x$ is divided by $y$ stand for $y \mid x$.

    \begin{definition}[id=DefDvs]
      A divisor of $x$ is an element that divides $x$.
    \end{definition}

    \begin{definition}[id=DefGCD]
      A gcd of $x$ and $y$ is a common divisor $c$ of $x$ and $y$ such that any
      common divisor of $x$ and $y$ divides $c$.
    \end{definition}

    \begin{definition}[id=DefRel]
      $x,y$ are relatively prime iff $1$ is a gcd of $x$ and $y$.
    \end{definition}
  \end{forthel}

  If we have two elements, say $a$ and $b$, we will see that the ideal
  \textit{generated} by $a$ and $b$ also contains the gcd of $a$ and $b$ (as
  long as $a$ or $b$ is non-zero). An ideal which is generated by a single
  element, a so-called \textit{principal ideal}, is defined as follows.

  \begin{forthel}
    \begin{definition}[id=DefPrIdeal]
      $\langle c \rangle$ is a set such that for every $z$ $z$ is an element of
      $\langle c \rangle$ iff there exists an element $x$ such that
      $z = c \cdot x$.
    \end{definition}

    \begin{lemma}[id=PrIdeal]
      $\langle c \rangle$ is an ideal.
    \end{lemma}
    \begin{proof}
      Let $x$ belong to $\langle c \rangle$.

      $\forall y \in \langle c \rangle x + y \in \langle c \rangle$.
      \begin{proof}
        Let $y \in \langle c \rangle$.
        (1) Take an element $u$ such that $c \cdot u = x$.
        (2) Take an element $v$ such that $c \cdot v = y$.
        $x + y .= c \cdot (u + v)$ (by 1, 2, AMDistr1).
        Therefore the thesis.
      \end{proof}

      $\forall z z \cdot x \in \langle c \rangle$.
      \begin{proof}
        Let $z$ be an element.
        (1) Take an element $u$ such that $c \cdot u = x$.
        $z \cdot x .= c \cdot (u \cdot z)$ (by 1, MulComm, MulAsso, MulBubble).
        Therefore the thesis.
      \end{proof}
    \end{proof}
  \end{forthel}

  The notion of a principal ideal allows us write the ideal which is generated
  by two elements $a$ and $b$ as $\langle a \rangle \oplus \langle b \rangle$.
  As mentioned before if not both $a$ and $b$ are zero,
  $\langle a \rangle \oplus \langle b \rangle$ contains the gcd of $a$ and $b$.
  That means that if $c$ is the gcd of $a$ and $b$ then $c$ is of the form
  $x \cdot a + y \cdot b$ for certain elements $x$ and $y$. For example if we
  take $\Int$ as our Euclidean domain we get \textit{Bézout's identity}: For two
  integers $n,m$ with a gcd $d$ there exist integers $x,y$ such that
  $d = x \cdot n + y \cdot m$. For instance
  \[ \gcd(8,14) = 2 = 2 \cdot 8 + (-1) \cdot 14 \]
  and
  \[ \gcd(9,25) = 1 = -11 \cdot 9 + 4 \cdot 25. \]

  \begin{forthel}
    \begin{theorem}[id=GCDin]
      Let $a,b$ be elements.
      Assume that $a$ is nonzero or $b$ is nonzero.
      Let $c$ be a gcd of $a$ and $b$.
      Then $c$ belongs to $\langle a \rangle \oplus \langle b \rangle$.
    \end{theorem}
    \begin{proof}
      Take an ideal $I$ equal to $\langle a \rangle \oplus \langle b \rangle$.
      We have $0,a \in \langle a \rangle$ and $0,b \in \langle b \rangle$
      (by MulZero, MulUnit).
      Hence there exists a nonzero element of
      $\langle a \rangle \oplus \langle b \rangle$.
      Indeed $a \in \langle a \rangle \oplus \langle b \rangle$ and
      $b \in \langle a \rangle \oplus \langle b \rangle$ (by AddZero).

      Take a nonzero $u \in I$ such that for no nonzero
      $v \in I$ $(|v| \prec |u|)$.
      Indeed we can show by induction on $|w|$ that for every nonzero $w \in I$
      there exists nonzero $u \in I$ such that for no nonzero
      $v \in I$ $(|v| \prec |u|)$.
      Obvious.

      $u$ is a common divisor of $a$ and $b$.
      \begin{proof}[method=contradiction]
        Assume the contrary.

        For some elements $x,y$ $u = (a \cdot x) + (b \cdot y)$.
        \begin{proof}
          Take $k \in \langle a \rangle$ and $l \in \langle b \rangle$ such that
          $u = k + l$.
          Take elements $x,y$ such that ($k = a \cdot x$ and $l = b \cdot y$).
          Hence the thesis.
        \end{proof}

        \begin{case}{$u$ does not divide $a$.}
          Take elements $q,r$ such that $a = (q \cdot u) + r$ and
          ($r = 0 \vee |r| \prec |u|$) (by Division).
          $r$ is nonzero.
          $-(q \cdot u)$ belongs to $I$.
          $a$ belongs to $I$ (by AddZero).
          $r = - (q \cdot u) + a$.
          Hence $r$ belongs to $I$ (by DefIdeal).
        \end{case}

        \begin{case}{$u$ does not divide $b$.}
          Take elements $q,r$ such that $b = (q \cdot u) + r$ and
          ($r = 0 \vee |r| \prec |u|$) (by Division).
          $r$ is nonzero.
          $-(q \cdot u)$ belongs to $I$.
          $b$ belongs to $I$ (by AddZero).
          $r = - (q \cdot u) + b$.
          Hence $r$ belongs to $I$ (by DefIdeal).
        \end{case}
      \end{proof}

      Hence $u$ divides $c$.

      Hence the thesis.
      \begin{proof}
        Take an element $z$ such that $c = z \cdot u$.
        Then $c \in I$ (by DefIdeal).
      \end{proof}
    \end{proof}
  \end{forthel}

  \textit{Bézout's identity} ensures that for any two coprime integers $n,m$ we
  have $n \Int \oplus m \Int = \Int$. Because we can take integers $x,y$ such
  that $x \cdot n + y \cdot m = 1$ and thus for every integer $z$ we have
  $zx \cdot n + zy \cdot m = z$, hence $z \in n \Int \oplus m \Int$. So as a
  special case of the Chinese remainder theorem if $n$ and $m$ are coprime then
  for all integers $x,y$ the simultaneous congruence
  \[ w = x ~(\mod n) \]
  \[ w = y ~(\mod m) \]
  has a solution.
\end{document}
