\documentclass[english]{article}
\usepackage{../latex/lib/naproche}
\addbibresource{../latex/lib/bibliography}

\begin{document}

\title{Numbers for Rudin}

\author{\Naproche{} Formalization: Peter Koepke}

\maketitle

The \Naproche{} system (Natural Proof Checking) \cite{DeLonKoepkeLorenzenMartiSchuetzWenzel2021, NaprocheWeb} checks the logical correctness of texts
written in an input language which ideally reads
like common mathematical language. Proof structures should resemble the
style of undergraduate textbooks. To test this
idea we have formalized number systems that are
axiomatically assumed or developed in the first chapter of the
\textit{Principles of Mathematical Analysis} by Walter Rudin
\cite{Rudin1953}. The formalization is written in the {\LaTeX} dialect 
of ForTheL (Formula Theory Language) which is the input language
of \Naproche{}.

Our approach parallels material
on pages 1 to 9 of \cite{Rudin1953}.
The ordered field $\mathbb{R}$ of real numbers
is postulated axiomatically. We construe the structures of integer and
rational numbers as substructures of $\mathbb{R}$:
\[ \mathbb{R} \supseteq \mathbb{Q} \supseteq \mathbb{Z}
\supseteq \mathbb{N}. \]
This offers the advantage that addition and multiplication of real numbers
can be used on those substructures.

Our axioms parallel axioms and the explicit construction of
$\mathbb{R}$ from $\mathbb{Q}$ in the Appendix of Chapter 1 of \cite{Rudin1953}.
A large part of the
material of Rudin appears in our formalization in some form, but
we have restructured the original to simplify proof
checking. E.g., we have restricted axioms about general
linear orders and fields to axioms about real numbers.
Labels of definitions and theorems
refer to similar labels in the original.
Checking the formalization
with a 3 second time out for the external ATP E \cite{Eprover}
takes less than 2 minutes on a modest laptop.


\subsection{Importing Set-Theoretic Preliminaries}

\begin{forthel}

[read examples/preliminaries.ftl.tex]

Let $A,B$ stand for sets.
Let $x$ is in $A$ denote $x$ is an element of $A$.

\begin{definition}
Let $x$ be an object.
$\{x\} = \{y \mid y = x\}$.
\end{definition}

\begin{definition}[id=1_3,title=1 3]
$A$ is nonempty iff $A$ has an element.
\end{definition}

\end{forthel}


\section{The Real Field}

\begin{forthel}
\begin{signature}
A real number is a mathematical object.
\end{signature}

\begin{definition}
$\mathbb{R}$ is the collection of real numbers.
\end{definition}
Let $x,y,z$ denote real numbers.

\begin{axiom}
$\mathbb{R}$ is a set.
\end{axiom}


\begin{signature}[id=1_12_A1,title=1 12 A1]
$x + y$ is a real number.
\end{signature}
Let the sum of $x$ and $y$ stand for $x + y$.

\begin{signature}[id=1_12_M1,title=1 12 M1]
$x \cdot y$ is a real number.
\end{signature}
Let the product of $x$ and $y$ denote $x \cdot y$.

\begin{signature}[id=1_5,title=1 5]
$x < y$ is an atom.
\end{signature}
Let $x > y$ stand for $y < x$.
Let $x \leq y$ stand for $x < y \vee x = y$.
Let $x \geq y$ stand for $y \leq x$.

\begin{axiom}[id=1_5_i,title=1 5 i]
$(x < y \wedge x \neq y) \wedge not y < x$
or $not x < y \wedge x = y \wedge not y < x$
or $not x < y \wedge x \neq y \wedge y < x$.
\end{axiom}

\begin{axiom}[id=1_5_ii,title=1 5 ii]
If $x < y$ and $y < z$ then $x < z$.
\end{axiom}

\begin{proposition}
$x \leq y$ iff not $x > y$.
\end{proposition}

\begin{axiom}[id=1_12_A2,title=1 12 A2]
$x + y = y + x$.
\end{axiom}

\begin{axiom}[id=1_12_A3,title=1 12 A3]
$(x + y) + z = x + (y + z)$.
\end{axiom}

\begin{signature}[id=1_12_A4,title=1 12 A4]
$0$ is a real number such that
for every real number $x$ $x + 0 = x$.
\end{signature}

\begin{signature}[id=1_12_A5,title=1 12 A5]
$-x$ is a real number such that $x + (-x) = 0$.
\end{signature}

\begin{axiom}[id=1_12_M2,title=1 12 M2]
$x \cdot y = y \cdot x$.
\end{axiom}

\begin{axiom}[id=1_12_M3,title=1 12 M3]
$((x \cdot y)) \cdot z = x \cdot (y \cdot z)$.
\end{axiom}

\begin{signature}[id=1_12_M4,title=1 12 M4]
$1$ is a real number such that $1 \neq 0$ and
for every real number $x$ $1 \cdot x = x$.
\end{signature}

\begin{signature}[id=1_12_M5,title=1 12 M5]
Assume $x \neq 0$. $1/x$ is a real number
such that $x \cdot (1/x) = 1$.
\end{signature}

\begin{axiom}[id=1_12_D,title=1 12 D]
$x \cdot (y + z) = (x \cdot y) + (x \cdot z)$.
\end{axiom}

\begin{proposition}[id=Dist_1,title=Dist 1]
$((y \cdot x)) + (z \cdot x) = (y + z) \cdot x$.
\end{proposition}

\begin{proposition}[id=1_14_a,title=1 14 a]
If $x + y = x + z$ then $y = z$.
\end{proposition}
\begin{proof}
Assume $x + y = x + z$.
Then \[ y = (-x+x) + y = -x + (x+y) = -x + (x+z) = (-x+x) + z = z. \]
\end{proof}

\begin{proposition}[id=1_14_b,title=1 14 b]
If $x + y = x$ then $y = 0$.
\end{proposition}

\begin{proposition}[id=1_14_c,title=1 14 c]
If $x + y = 0$ then $y = -x$.
\end{proposition}

\begin{proposition}[id=1_14_d,title=1 14 d]
$-(-x) = x$.
\end{proposition}


\begin{proposition}[id=1_15_a,title=1 15 a]
If $x \neq 0$ and $x \cdot y = x \cdot z$
then $y = z$.
\end{proposition}
\begin{proof}
Let $x \neq 0$ and $x \cdot y = x \cdot z$.
\[ y = 1 \cdot y = ((1/x) \cdot x) \cdot y = (1/x) \cdot (x \cdot y) =
(1/x) \cdot (x \cdot z) = ((1/x) \cdot x) \cdot z = 1 \cdot z = z. \]
\end{proof}

\begin{proposition}[id=1_15_b,title=1 15 b]
If $x \neq 0$ and $x \cdot y = x$ then $y = 1$.
\end{proposition}

\begin{proposition}[id=1_15_c,title=1 15 c]
If $x \neq 0$ and $x \cdot y = 1$ then $y = 1/x$.
\end{proposition}

\begin{proposition}[id=1_15_d,title=1 15 d]
If $x \neq 0$ then $1/(1/x) = x$.
\end{proposition}

\begin{proposition}[id=1_16_a,title=1 16 a]
$0 \cdot x = 0$.
\end{proposition}

\begin{proposition}[id=1_16_b,title=1 16 b]
If $x \neq 0$ and $y \neq 0$ then $x \cdot y \neq 0$.
\end{proposition}

\begin{proposition}[id=1_16_c,title=1 16 c]
$(-x) \cdot y = -(x \cdot y)$.
\end{proposition}
\begin{proof}
$((x \cdot y)) + ((-x \cdot y)) = (x + (-x)) \cdot y
= 0 \cdot y = 0$.
\end{proof}

\begin{proposition}
$-x = -1 \cdot x$.
\end{proposition}

\begin{proposition}[id=1_16d,title=1 16d]
$(-x) \cdot (-y) = x \cdot y$.
\end{proposition}
\begin{proof}
$(-x)\cdot (-y) = -(x\cdot(-y)) = -((-y)\cdot x) =
-(-(y\cdot x)) = y\cdot x = x\cdot y$.
\end{proof}

Let $x - y$ stand for $x + (-y)$.
Let $\frac{x}{y}$ stand for $x \cdot (1/y)$.

\end{forthel}


\section{The Real Ordered Field}

\begin{forthel}
Let $x,y,z$ denote real numbers.

\begin{axiom}[id=1_17_i,title=1 17 i]
If $y < z$ then $x + y < x + z$ and $y + x < z + x$.
\end{axiom}

\begin{axiom}[id=1_17_ii,title=1 17 ii]
If $x > 0$ and $y > 0$ then $x \cdot y > 0$.
\end{axiom}

\begin{definition}
$x$ is positive iff $x > 0$.
\end{definition}

\begin{definition}
$x$ is negative iff $x < 0$.
\end{definition}


\begin{proposition}[id=1_18_a,title=1 18 a]
$x > 0$ iff $-x < 0$.
\end{proposition}

\begin{proposition}[id=1_18_b,title=1 18 b]
If $x > 0$ and $y < z$ then
$x \cdot y < x \cdot z$.
\end{proposition}
\begin{proof}
Let $x > 0$ and $y < z$.
$z - y > y - y = 0$.
$x \cdot (z - y) > 0$.
$x \cdot z = (x \cdot (z - y)) + (x \cdot y)$.
$((x \cdot (z - y))) + (x \cdot y)  > 0 + (x \cdot y)$.
$0 + (x \cdot y) = x \cdot y$.
\end{proof}

\begin{proposition}[id=1_18_bb,title=1 18 bb]
If $x > 0$ and $y < z$ then
$y \cdot x < z \cdot x$.
\end{proposition}


\begin{proposition}[id=1_18_d,title=1 18 d]
If $x \neq 0$ then $x \cdot x > 0$.
\end{proposition}
\begin{proof}
Let $x \neq 0$.
Case $x > 0$. Then the thesis. end.
Case $x < 0$. Then $ -x > 0$. end.
\end{proof}

\begin{proposition}[id=1_18_dd,title=1 18 dd]
$1 > 0$.
\end{proposition}

\begin{proposition}
$x < y$ iff $-x > -y$.
\end{proposition}
\begin{proof}

$x < y \iff x - y < 0$.

$x - y < 0 \iff (-y) + x < 0$.

$(-y) + x < 0 \iff (-y)+(-(-x)) < 0$.

$(-y)+(-(-x)) < 0 \iff (-y)-(-x) < 0$.

$(-y)-(-x) < 0 \iff -y < -x$.
\end{proof}

\begin{proposition}[id=1_18_c,title=1 18 c]
If $x < 0$ and $y < z$ then
$x \cdot y > x \cdot z$.
\end{proposition}
\begin{proof}
Let $x < 0$ and $y < z$.
$-x > 0$.
$(-x)\cdot y < (-x)\cdot z$.
$-(x\cdot y) < -(x\cdot z)$.
\end{proof}

\begin{proposition}[id=1_18_cc,title=1 18 cc]
If $x < 0$ and $y < z$ then
$y \cdot x > z \cdot x$.
\end{proposition}

\begin{proposition}
$x + 1 > x$.
\end{proposition}

\begin{proposition}
$x - 1 < x$.
\end{proposition}

\begin{proposition}[id=1_18_e,title=1 18 e]
If $0 < y$ then $0 < 1/y$.
\end{proposition}

\begin{proposition}[id=1_18_ee,title=1 18 ee]
Assume $0 < x < y$.
Then $1/y < 1/x$.
\end{proposition}
\begin{proof}
Consider $w = 1/x$. Consider $v = 1/y$.

Case $1/x < 1/y$.
Then
\[ 1 = x \cdot (1/x) = (1/x) \cdot x < (1/x) \cdot y =
y \cdot (1/x) < y \cdot (1/y) = 1. \]
Contradiction. end.

Case $1/x = 1/y$. Then
\[ 1 = x \cdot (1/x) < y \cdot (1/y) = 1. \]
Contradiction. end.

Hence not $1/x \leq 1/y$ and $1/y < 1/x$.
\end{proof}

\end{forthel}


\section{Upper and lower bounds}

\begin{forthel}
Let $x,y,z$ denote real numbers.
Let $E$ denote a subset of $\mathbb{R}$.

\begin{definition}[id=1_7,title=1 7]
An upper bound of $E$ is a
real number $b$ such that for all elements $x$ of $E$ $x \leq b$.
\end{definition}

\begin{definition}[id=1_7a,title=1 7a]
$E$ is bounded above iff
$E$ has an upper bound.
\end{definition}

\begin{definition}[id=1_7b,title=1 7b]
A lower bound of $E$ is a
real number $b$ such that for all elements $x$ of $E$ $x \geq b$.
\end{definition}

\begin{definition}[id=1_7c,title=1 7c]
$E$ is bounded below iff
$E$ has a lower bound.
\end{definition}

\begin{definition}[id=1_8,title=1 8]
A least upper bound of $E$ is a real number $a$ such that
$a$ is an upper bound of $E$ and for all $x$ if $x < a$ then $x$
is not an upper bound of $E$.
\end{definition}

\begin{definition}[id=1_8a,title=1 8a]
Let $E$ be bounded below.
A greatest lower bound of $E$ is a real number $a$ such that
$a$ is a lower bound of $E$ and for all $x$ if $x > a$ then $x$ is
not a lower bound of $E$.
\end{definition}

\begin{axiom}[id=1_19,title=1 19]
Assume that $E$ is nonempty and bounded above.
Then $E$ has a least upper bound.
\end{axiom}

\begin{definition}
$E^- = \{-x \in \mathbb{R} \mid x \in E\}$.
\end{definition}

\begin{lemma}
$E^-$ is a subset of $\mathbb{R}$.
\end{lemma}

\begin{lemma}
$x$ is an upper bound of $E$ iff $-x$ is a lower bound of $E^-$.
\end{lemma}

\begin{theorem}[id=1_11,title=1 11]
Assume that $E$ is a nonempty subset of $\mathbb{R}$
such that $E$ is bounded below.
Then $E$ has a greatest lower bound.\end{theorem}
\begin{proof}
Take a lower bound $a$ of $E$.
$-a$ is an upper bound of $E^-$.
Take a least upper bound $b$ of $E^-$.
Let us show that $-b$ is a greatest lower bound of $E$.
$-b$ is a lower bound of $E$. Let $c$ be a lower bound of $E$.
Then $-c$ is an upper bound of $E^-$.
end.
\end{proof}

\end{forthel}


\section{The rational numbers}

\begin{forthel}
Let $x$ denote a real number.
Let $A$ denote a set.

\begin{signature}[id=1_19a,title=1 19a]
A rational number is a real number.
\end{signature}

Let $p,q,r$ denote rational numbers.

\begin{definition}
$\mathbb{Q}$ is the collection of rational numbers.
\end{definition}

\begin{theorem}
$\mathbb{Q}$ is a set.
\end{theorem}
\begin{proof}
$\mathbb{Q}$ is a subset of $\mathbb{R}$.
\end{proof}

\end{forthel}


Theorem 1.19 of \cite{Rudin1953} states that $\mathbb{Q}$ is a
subfield of $\mathbb{R}$. We require this axiomatically.


\begin{forthel}

\begin{lemma}
$\mathbb{Q} \subseteq \mathbb{R}$.
\end{lemma}

\begin{axiom}
$p + q$, $p \cdot q$, $0$, $-p$, $1$ are rational numbers.
\end{axiom}

\begin{axiom}
Assume $p \neq 0$.
$1/p$ is a rational number.
\end{axiom}
\end{forthel}

We assume axiomatically that $\mathbb{R}$ is the completion of $\mathbb{Q}$.

\begin{forthel}
\begin{axiom} Let $y$ be a real number.
Then there exists a subset $A$ of $\mathbb{Q}$
such that $A$ is bounded above and
$y$ is a least upper bound of $A$.
\end{axiom}

\begin{theorem}
$\mathbb{R} = \{x \in \mathbb{R} \mid$ there exists $
A \subseteq \mathbb{Q}$ such that $
A$ is bounded above and $x$ is a least upper
bound of $A\}$.
\end{theorem}

\end{forthel}


\section{Integers}

\begin{forthel}
Let $p$ denote a rational number.

\begin{signature}
An integer is a rational number.
\end{signature}
Let $a,b,i$ stand for integers.

\begin{definition}
$\mathbb{Z}$ is the collection of integers.
\end{definition}


\begin{theorem}
$\mathbb{Z}$ is a subset of $\mathbb{R}$.
\end{theorem}

\end{forthel}


$\mathbb{Z}$ is a discrete subring of $\mathbb{Q}$:


\begin{forthel}

\begin{axiom}
$a + b$, $a \cdot b$, $0$, $-a$, $1$ are integers.
\end{axiom}

\begin{axiom}
There is no integer $a$ such that $0 < a < 1$.
\end{axiom}

\begin{axiom}
There exist $a$ and $b$ such that
$a \neq 0 \wedge p = \frac{b}{a}$.
\end{axiom}

\begin{theorem}[id=Archimedes_1,title=Archimedes 1]
$\mathbb{Z}$ is not bounded above.
\end{theorem}
\begin{proof}
Assume the contrary.
$\mathbb{Z}$ is nonempty. Indeed $0$ is an integer.
Take a least upper bound
$b$ of $\mathbb{Z}$.
Let us show that $b - 1$ is an upper bound of $\mathbb{Z}$.
Let $x \in \mathbb{Z}$. $x + 1 \in \mathbb{Z}$.
$x + 1 \leq b$.
$x = (x + 1) - 1 \leq b - 1$.
end.
\end{proof}

\begin{theorem}
$\mathbb{Z}$ is not bounded below.
\end{theorem}
\begin{proof}
Assume the contrary.
Then $\mathbb{Z}$ is bounded below.
Take an real number $m$ that is a lower bound of $\mathbb{Z}$.
Then $-m$ is an upper bound of $\mathbb{Z}$.
Contradiction.
\end{proof}

\begin{theorem}[id=Archimedes_2,title=Archimedes 2]
Let $x$ be a real number.
Then there is an integer $a$
such that $x < a$.
\end{theorem}
\begin{proof}[method=contradiction]
Assume the contrary.
Then $x$ is an upper bound of $\mathbb{Z}$.
Contradiction.
\end{proof}

\end{forthel}


\section{The natural numbers}

\begin{forthel}
Let $x$ denote a real number.

\begin{definition}
$\mathbb{N}$ is the collection
of positive integers.
\end{definition}
Let $m,n$ stand for positive integers.

\begin{lemma}
$\mathbb{N}$ is a subset of $\mathbb{R}$.
\end{lemma}

\begin{lemma}
$\mathbb{Z} = (\mathbb{N}^- \cup \{0\}) \cup \mathbb{N}$.
\end{lemma}

\begin{theorem}[id=Induction_Theorem,title=Induction Theorem]
Assume $A \subseteq \mathbb{N}$
and $1 \in A$ and for all $n \in A$ $n + 1 \in A$.
Then $A = \mathbb{N}$.
\end{theorem}

\begin{proof}
Let us show that every element of $\mathbb{N}$ is an element of $A$.
  Let $n \in \mathbb{N}$.
  Assume the contrary.
  Define $F = \{  j \in \mathbb{N} \mid j \notin A\}$.
  $F$ is nonempty. $F$ is bounded below.
  Take a greatest lower bound $a$ of $F$.
  Let us show that $a+1$ is a lower bound of $F$.
    Let $x \in F$. $x - 1 \in \mathbb{Z}$.

    Case $x - 1 < 0$. Then $0 < x < 1$. Contradiction. end.

    Case $x - 1 = 0$. Then $x = 1$ and $1 \notin A$. Contradiction. end.

    Case $x - 1 > 0$. Then $x - 1 \in \mathbb{N}$.
      $x - 1 \notin A$. Indeed $(x - 1) + 1 = x \notin A$. $x - 1 \in F$.
      $a \leq x - 1$.
      $a + 1 \leq (x - 1) + 1 = x$.
    end.
  end.

  Then $a+1 > a$.
  Contradiction.
end.
\end{proof}

\begin{lemma}
There is an integer $m$ such that
$m -1 \leq x < m$.
\end{lemma}
\begin{proof}
Assume the contrary.
Then for all $m$ such that $x \geq m-1 $ we have $x \geq m$.
Take an integer $n$ such that $n < x$.
Define
\[ A = \{i \in \mathbb{N} \mid n + (i - 1) \leq x\}. \]
(1) $A = \mathbb{N}$.

Proof.
$A \subseteq \mathbb{N}$.
$1 \in A$.

For all $i \in A$ $i + 1 \in A$.

  Proof. Let $i \in A$. Then
  $n + (i -1) = (n + i) - 1 \leq x$ and
  $n + ((i + 1) - 1) = n + i \leq x$.
  Hence $i + 1 \in A$.
  qed.

qed.

(2) $x + 1$ is an upper bound of $\mathbb{Z}$.

Proof.
Let $j$ be an integer. Let us show that $j \leq x + 1$.

Case $j \leq n$. Trivial.

Then $j > n$. Take a positive integer $i$
such that $j = n + i$. $i \in A$.
Hence $n + (i-1) \leq x$ and $j = n+i \leq x + 1$.
qed.

qed.

Contradiction.
\end{proof}

\end{forthel}


\section{Archimedian properties}

\begin{forthel}
Let $x,y$ denote a real number.

\begin{theorem}[id=1_20_a,title=1 20 a]
Let $x > 0$.
Then there is a
positive integer $n$ such that \[ n \cdot x > y. \]
\end{theorem}
\begin{proof}
Take an integer $a$ such that $a > \frac{y}{x}$.
Take a positive integer $n$ such that $n > a$.
$n > \frac{y}{x}$ and $n \cdot x > (\frac{y}{x}) \cdot x = y$.
\end{proof}

\begin{theorem}[id=1_20_b,title=1 20 b]
Let $x < y$. Then there exists
$p \in \mathbb{Q}$ such that $x < p < y$.
\end{theorem}
\begin{proof}
We have $y - x > 0$.

Take a positive integer $n$ such that
$n\cdot (y-x) > 1$. Indeed consider $z = y - x$.

Take an integer $m$ such that
\[ m -1 \leq n \cdot x < m. \] Indeed consider $z = n \cdot x$.

Then
\[ n \cdot x < m = (m - 1) + 1 \leq (n\cdot x) + 1  < \]

\[((n\cdot x)) + (n\cdot (y-x)) = n \cdot (x + (y - x)) = n \cdot y. \]
Indeed consider $z = y - x$.
$n > 0$ and
\[ x = \frac{n\cdot x}{n} < \frac{m}{n} < \frac{n\cdot y}{n} = y. \]
Take $p = \frac{m}{n}$. Then $p \in \mathbb{Q}$ and $x < p < y$.
\end{proof}

\end{forthel}

\printbibliography

\end{document}
