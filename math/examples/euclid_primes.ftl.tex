\documentclass[11pt]{article}
\usepackage{latex/naproche}
\usepackage[english]{babel}

\author{\Naproche{} Formalization: Peter Koepke\\
University of Bonn}

\title{Euclid's Proof of the Infinitude of Primes}

\date{2021}

\begin{document}

\newcommand{\Prod}[3]{#1_{#2} \cdots #1_{#3}}
\newcommand{\Seq}[2]{\{#1,\dots,#2\}}
\newcommand{\FinSet}[3]{\{#1_{#2},\dots,#1_{#3}\}}
\newcommand{\Primes}{\mathbb{P}}

\maketitle

\noindent
This paper contains a standard proof of Euclid's theorem
that there are infinitely many prime numbers,
following the first proof given in \emph{Proofs from THE BOOK}
by Martin Aigner and Günter M. Ziegler, Springer Verlag.
Before the proof we
set up a language and axioms for natural number arithmetic,
define divisibility and prime numbers,
introduce some set theoretic background and define
finite sets, sequences and products.
Checking the formalization takes about one minute on a modest
laptop.


\section{Importing Standard Preliminaries}

\begin{forthel}
  [read examples/preliminaries.ftl.tex]
\end{forthel}


\section{Natural Numbers}

\begin{forthel}

\begin{signature}
A natural number is a mathematical object.
\end{signature}

Let $i,k,l,m,n,p,q,r$ denote natural numbers.

\begin{definition}
$\mathbb{N}$ is the collection of natural numbers.
\end{definition}

\begin{signature}
$0$ is a natural number.
\end{signature}

Let $x$ is nonzero stand for $x \neq 0$.

\begin{signature}
$1$ is a nonzero natural number.
\end{signature}

\begin{signature}
$m + n$ is a natural number.
\end{signature}

\begin{signature}
$m * n$ is a natural number.
\end{signature}

\begin{axiom}
$m + n = n + m$.
\end{axiom}

\begin{axiom}
$(m + n) + l = m + (n + l)$.
\end{axiom}

\begin{axiom}
$m + 0 = m = 0 + m$.
\end{axiom}

\begin{axiom}
$m * n = n * m$.
\end{axiom}

\begin{axiom}
$(m * n) * l = m * (n * l)$.
\end{axiom}

\begin{axiom}
$m * 1 = m = 1 * m$.
\end{axiom}

\begin{axiom}
$m * 0 = 0 = 0 * m$.
\end{axiom}

\begin{axiom}
$m * (n + l) = (m * n) + (m * l)$ and
$(n + l) * m = (n * m) + (l * m)$.
\end{axiom}

\begin{axiom}
If $l + m = l + n$ or $m + l = n + l$ then $m = n$.
\end{axiom}

\begin{axiom}
Assume that $l$ is nonzero.
If $l * m = l * n$ or $m * l = n * l$ then $m = n$.
\end{axiom}

\begin{axiom}
If $m + n = 0$ then $m = 0$ and $n = 0$.
\end{axiom}

\end{forthel}


\section{The Natural Order}

\begin{forthel}
Let $l,m,n$ denote natural numbers.

\begin{definition}
$m \leq n$ iff
there exists a natural number $l$ such that $m + l = n$.
\end{definition}

Let $m < n$ stand for $m \leq n$ and $m \neq n$.

\begin{definition}
Assume that $n \leq m$.
$m - n$ is a natural number $l$ such that $n + l  = m$.
\end{definition}

\end{forthel}

The following three lemmas show that $\leq$ is a partial order:

\begin{forthel}

\begin{lemma}
$m \leq m$.
\end{lemma}

\begin{lemma}
If $m \leq n \leq m$ then $m = n$.
\end{lemma}
\begin{proof}
Let $m \leq n \leq m$.
Take a natural numbers $k,l$ such that
$n = m + k$ and $m = n + l$.
Then $m = m + (k + l)$ and $k + l = 0$ and $k = 0$.
Hence $m = n$.
\end{proof}

\begin{lemma}
If $m \leq n \leq l$ then  $m \leq l$.
\end{lemma}

\begin{axiom}
$m \leq n$ or $n < m$.
\end{axiom}

\begin{lemma}
Assume that $l < n$.
Then $m + l < m + n$ and $l + m < n + m$.
\end{lemma}

\begin{lemma}
Assume that $m$ is nonzero and $l < n$.
Then $m * l < m * n$ and $l * m < n * m$.
\end{lemma}

\end{forthel}


\section{Induction}

\Naproche provides a special binary relation
symbol $\prec$ for a universal inductive relation: if at any
point $m$ property $P$ is inherited at $m$ provided all
$\prec$-predecessors of $m$ satisfy $P$, then $P$ holds everywhere.
Induction along $<$ is ensured by:

\begin{forthel}
Let $l,m,n$ denote natural numbers.

\begin{axiom}
If $n < m$ then $n \prec m$.
\end{axiom}

\begin{lemma}
For every natural number $n$ $n = 0$ or $1 \leq n$.
\end{lemma}
\begin{proof}[by induction]
\end{proof}

\begin{lemma}
Let $m \neq 0$. Then $n \leq n * m$.
\end{lemma}
\begin{proof}
$1 \leq m$.
\end{proof}

\end{forthel}


\section{Division}

\begin{forthel}
Let $l,m,n,p,q,r$ denote natural numbers.

\begin{definition}
$n$ divides $m$ iff for some $l$ $m = n * l$.
\end{definition}

Let $x | y$ denote $x$ divides $y$.
Let a divisor of $x$ denote a natural number that divides $x$.

\begin{lemma}
Assume $l | m | n$.
Then $l | n$.
\end{lemma}

\begin{lemma}
Let $l | m$ and $l | m + n$.
Then $l | n$.
\end{lemma}
\begin{proof}
Assume that $l$ is nonzero.
Take $p$ such that $m = l * p$.
Take $q$ such that $m + n = l * q$.

Let us show that
$p \leq q$.
Proof by contradiction.
Assume the contrary. Then $q < p$.
$m+n = l * q < l * p = m$.
Contradiction. qed.

Take $r = q - p$.
We have $(l * p) + (l * r) = l * q = m + n = (l * p) + n$.
Hence $n = l * r$.
\end{proof}

\begin{lemma}
Let $m | n \neq 0$.
Then $m \leq n$.
\end{lemma}

\end{forthel}


\section{Primes}

\begin{forthel}
Let $i,k,l,m,n,p,q,r$ denote natural numbers.

Let $x$ is nontrivial stand for $x \neq 0$ and $x \neq 1$.

\begin{definition}
$n$ is prime iff $n$ is nontrivial and
for every divisor $m$ of $n$ $m = 1$ or $m = n$.
\end{definition}

\begin{lemma}
Every nontrivial $m$ has a prime divisor.
\end{lemma}
\begin{proof}[by induction on $m$]
\end{proof}

\end{forthel}


\section{Finite Sequences and Products}

\begin{forthel}
Let $m,n$ denote natural numbers.

\begin{definition}
$\Seq{m}{n}$ is the class of
natural numbers $i$ such that $m \leq i \leq n$.
\end{definition}

\begin{axiom}
$\Seq{m}{n}$ is a set.
\end{axiom}

\begin{axiom}
Assume $F$ is a function and $x \in \dom(F)$.
Then $F(x)$ is an object.
\end{axiom}

\begin{definition}
A sequence of length $n$ is a
function $F$ such that $\dom(F) = \Seq{1}{n}$.
\end{definition}

Let $F_{i}$ stand for $F(i)$.

\begin{definition}
Let $F$ be a sequence of length $n$.
$\FinSet{F}{1}{n} = \{ F_{i} | i \in \dom(F)\}$.
\end{definition}

\begin{signature}
Let $F$ be a sequence of length $n$
such that $\FinSet{F}{1}{n} \subseteq \mathbb{N}$.
$\Prod{F}{1}{n}$ is a natural number.
\end{signature}

\begin{axiom}\label{Factorproperty}
Let $F$ be a sequence of length $n$
such that $F(i)$ is a nonzero natural number for every $i \in \dom(F)$.
Then $\Prod{F}{1}{n}$ is nonzero and
$F(i)$ divides $\Prod{F}{1}{n}$ for every $i \in \dom(F)$.
\end{axiom}

\end{forthel}


\section{Finite and Infinite Sets}

\begin{forthel}

Let $S$ denote a class.

\begin{definition}
$S$ is finite iff
$S = \FinSet{s}{1}{n}$ for some natural number $n$ and some function $s$ that is
a sequence of length $n$.
\end{definition}

\begin{definition}
$S$ is infinite iff $S$ is not finite.
\end{definition}

\end{forthel}


\section{Euclid's Theorem}

\begin{forthel}
Let $i$ denote natural numbers.

\begin{signature}
$\Primes$ is the collection of prime natural numbers.
\end{signature}

\begin{theorem}[Euclid]
$\Primes$ is infinite.
\end{theorem}
\begin{proof}
Assume that $r$ is a natural number and
$p$ is a sequence of length $r$ and
$\FinSet{p}{1}{r}$ is a subclass of $\Primes$.
$p_{i}$ is a nonzero natural number for every
$i \in  \dom(p)$.
Consider $n=\Prod{p}{1}{r}+1$.
$\Prod{p}{1}{r}$ is nonzero (by Factorproperty).
$n$ is nontrivial.
Take a prime divisor $q$ of $n$.

Let us show that $q \neq p_{i}$ for all $i$ such that
$1 \leq i \leq r$.

Proof by contradiction.
Assume that $q=p_{i}$ for some natural number $i$ such that
$1 \leq i \leq r$.
$q$ is a divisor of $n$.
$q$ is a divisor of $\Prod{p}{1}{r}$
(by Factorproperty).
Thus $q$ divides $1$.
Contradiction. qed.

Hence $\FinSet{p}{1}{r}$ is not the class of prime natural numbers.
\end{proof}

\end{forthel}

\end{document}
