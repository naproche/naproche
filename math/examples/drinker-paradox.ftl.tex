\documentclass{article}
\usepackage{naproche}
\libinput[latex]{example-preamble}
\libinput[libraries/foundations]{preamble}

\title{The Drinker Paradox}
\author{\Naproche formalization:\\[0.5em]Marcel Schütz}
\date{2024}

\begin{document}
  \pagenumbering{gobble}
  \maketitle

  \begin{forthel}
    [read \path{examples/lang/everyday-ontology.ftl.tex}]
  \end{forthel}

  \noindent The \emph{Drinker Paradox} is a principle of classical predicate 
  logic popularised by the logician Raymond Smullyan in his 1978 book
  \textit{What Is the Name of this Book?} \cite{Smullyan1978} which can be 
  stated as:

  \begin{quotation}
    \noindent There is someone in the pub such that, if he is drinking then 
    everyone in the pub is drinking.
  \end{quotation}

  \begin{forthel}
    \begin{signature*}
      The pub is an object.
    \end{signature*}
    
    \begin{theorem*}[title=Drinker Paradox,id=drinker_paradox]
      Assume that there is a person inside the pub.
      Then there is a person $P$ inside the pub such that if $P$ is drinking then every person inside the pub is drinking.
    \end{theorem*}
    \begin{proof}[method=case analysis]
      \begin{case}{every person inside the pub is drinking.}
        Choose a person $P$ inside the pub.
        Then $P$ is drinking and every person inside the pub is drinking.
        Hence if $P$ is drinking then every person inside the pub is drinking.
      \end{case}
    
      \begin{case}{there is a person inside the pub that is not drinking.}
        Consider a person $P$ inside the pub that is not drinking.
        Then if $P$ is drinking then every person inside the pub is drinking.
      \end{case}
    \end{proof}
  \end{forthel}

  \printbibliography
  \printlicense[CcByNcSa]{2024}{Marcel Schütz}
\end{document}
