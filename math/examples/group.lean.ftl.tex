\documentclass{article}
\usepackage{naproche}
\libinput[meta-inf]{example-preamble}

\title{group.lean in \Naproche}
\author{\Naproche Formalization: Peter Koepke}
\date{2021}

\begin{document}
\pagenumbering{gobble}

\maketitle

\usemodule[meta-inf]{vocabulary.ftl}

\noindent We give a ForTheL version of the ``multiplicative half'' of
\path{https://github.com/leanprover/lean/blob/master/library/init/
algebra/group.lean} on basic concepts in group theory
by J. Avigad and L. de Moura.
Whereas the Lean formalization takes place within a sophisticated type system
we suggest a simplistic approach in which types are naively viewed as sets.

The purpose of this text is a comparison of the technical approach in Lean
with the rather natural input language ForTheL of \Naproche{}.
\Naproche{} allows to introduce new notions like ``type with multiplication''
or ``group'' for which further structural components like a multiplication can
be introduced and equipped with axioms. Since the mathematics of this
text is very basic, \Naproche{} can prove most lemmas automatically without
explicit proofs. The formalization given here motivates to further
use and adapt \Naproche{}'s soft types for abstract algebraic structures.


\section{Defining Groups}

\begin{forthel}

\begin{signature}
A type is a set.
\end{signature}

Let $\alpha$ stand for a type.
Let $a : t$ stand for $a$ is an element of $t$.

\begin{signature}
A type with multiplication is a type.
\end{signature}

\begin{signature}
Let $\alpha$ be a type with multiplication and $a,b : \alpha$.
$a *^{\alpha} b$ is an element of $\alpha$.
\end{signature}

\begin{definition}
A semigroup is a type with multiplication $\alpha$
such that for all $a,b,c : \alpha$
$(a *^{\alpha} b) *^{\alpha} c = a *^{\alpha} (b *^{\alpha} c)$.
\end{definition}

\begin{definition}
A commutative semigroup is a semigroup $\alpha$
such that for all $a,b : \alpha$
$a *^{\alpha} b = b *^{\alpha} a$.
\end{definition}

\begin{definition}
A semigroup with left cancellation is a semigroup
$\alpha$ such that for all $a,b,c : \alpha$
$a *^{\alpha} b = a *^{\alpha} c => b = c$.
\end{definition}

\begin{definition}
A semigroup with right cancellation is a semigroup
$\alpha$ such that for all $a,b,c : \alpha$
$a *^{\alpha} b = c *^{\alpha} b => a = c$.
\end{definition}

\begin{signature}
A type with one is a type.
\end{signature}

\begin{signature}
Assume $\alpha$ is a type with one. $1^{\alpha}$ is an
element of $\alpha$.
\end{signature}

\begin{definition}
A monoid is a semigroup $\alpha$ such that $\alpha$ is a type
with one and
$\forall a : \alpha 1^{\alpha} *^{\alpha} a = a$ and
$\forall a : \alpha a *^{\alpha} 1^{\alpha} = a$.
\end{definition}

\begin{definition}
A commutative monoid is a monoid that
is a commutative semigroup.
\end{definition}

\begin{signature}
A type with inverses is a type.
\end{signature}

\begin{signature}
Assume $\alpha$ is a type with inverses and $a : \alpha$.
$a^{-1,\alpha}$ is an element of $\alpha$.
\end{signature}

\begin{definition}
A group is a monoid $\alpha$ such that $\alpha$ is a type
with inverses and
for all $a$ : $\alpha a^{-1,\alpha} *^{\alpha} a = 1^{\alpha}$.
\end{definition}

\begin{definition}
A commutative group is a group that is a commutative
monoid.
\end{definition}

\end{forthel}


\section{Term Identities in Groups}

We prove a number of simple algebraic consequences of the the
group axioms. The formalizations would read even more natural
if the dependence on a group $\alpha$ could be made implicit
and one could write $a * a^{-1} = 1$ instead of
$a *^{\alpha} a^{-1,\alpha} = 1^{\alpha}$.
Lean's elaboration mechanism provides this important feature
which we would like to implement in \Naproche{} as well.


\begin{forthel}

\begin{lemma}[title=mul left comm]
Let $\alpha$ be a commutative semigroup.
Then for all $a,b,c : \alpha$
$a *^{\alpha} (b *^{\alpha} c) = b *^{\alpha} (a *^{\alpha} c)$.
\end{lemma}

\begin{lemma}[title=mul right comm]
Let $\alpha$ be a commutative semigroup.
Then for all $a,b,c : \alpha$
$a *^{\alpha} (b *^{\alpha} c) = a *^{\alpha} (c *^{\alpha} b)$.
\end{lemma}

\begin{lemma}[title=mul left cancel iff]
Let $\alpha$ be a semigroup with left cancellation.
Then for all $a,b,c : \alpha$
$a *^{\alpha} b = a *^{\alpha} c <=> b = c$.
\end{lemma}

\begin{lemma}[title=mul right cancel iff]
Let $\alpha$ be a semigroup with right cancellation.
Then for all $a,b,c : \alpha$
$b *^{\alpha} a = c *^{\alpha} a <=> b = c$.
\end{lemma}

Let $\alpha$ denote a group.

\begin{lemma}[title=inv mul cancel left]
For all $a,b : \alpha$
$a^{-1,\alpha} *^{\alpha} (a *^{\alpha} b) = b$.
\end{lemma}

\begin{lemma}[title=inv mul cancel right]
For all $a,b : \alpha$
$a *^{\alpha} (b^{-1,\alpha} *^{\alpha} b) = a$.
\end{lemma}

\begin{lemma}[title=inv eq of mul eq one]
Let $a, b : \alpha$ and $a *^{\alpha} b = 1^{\alpha}$.
Then $a^{-1,\alpha} = b$.
\end{lemma}

\begin{lemma}[title=one inv]
$(1^{\alpha})^{-1,\alpha} = 1^{\alpha}$.
\end{lemma}

\begin{lemma}[title=inv inv]
Let $a : \alpha$.
Then $(a^{-1,\alpha})^{-1,\alpha} = a$.
\end{lemma}

\begin{lemma}[title=mul right inv]
Let $a : \alpha$.
Then $a *^{\alpha} a^{-1,\alpha} = 1^{\alpha}$.
\end{lemma}

\begin{lemma}[title=inv inj]
Let $a,b : \alpha$ and $a^{-1,\alpha} = b^{-1,\alpha}$.
Then $a = b$.
\end{lemma}

\begin{lemma}[title=group mul left cancel]
Let $a,b,c : \alpha$ and $a *^{\alpha} b = a *^{\alpha} c$.
Then $b = c$.
\end{lemma}

\begin{lemma}[title=group mul right cancel]
Let $a,b,c : \alpha$ and $a *^{\alpha} b = c *^{\alpha} b$.
Then $a = c$.
\end{lemma}
\begin{proof}
$a = (a *^{\alpha} b) *^{\alpha} b^{-1,\alpha}
= (c *^{\alpha} b) *^{\alpha} b^{-1,\alpha} = c$.
\end{proof}

\begin{lemma}[title=mul inv cancel left]
Let $a,b : \alpha$.
Then $a *^{\alpha} (a^{-1,\alpha} *^{\alpha} b) = b$.
\end{lemma}

\begin{lemma}[title=mul inv cancel right]
Let $a,b : \alpha$.
Then $(a *^{\alpha} b) *^{\alpha} b^{-1,\alpha} = a$.
\end{lemma}


\begin{lemma}[title=mul inv rev]
Let $a,b : \alpha$.
Then $(a *^{\alpha} b)^{-1,\alpha} = b^{-1,\alpha} *^{\alpha} a^{-1,\alpha}$.
\end{lemma}
\begin{proof}
$(a *^{\alpha} b) *^{\alpha} (b^{-1,\alpha} *^{\alpha} a^{-1,\alpha})
= 1^{\alpha}$.
\end{proof}

\begin{lemma}[title=eq inv of eq inv]
Let $a,b : \alpha$ and $a = b^{-1,\alpha}$.
Then $b = a^{-1,\alpha}$.
\end{lemma}

\begin{lemma}[title=eq inv of mul eq one]
Let $a,b : \alpha$ and $a *^{\alpha} b = 1^{\alpha}$.
Then $a = b^{-1,\alpha}$.
\end{lemma}

\begin{lemma}[title=eq mul inv of mul eq]
Let $a,b,c : \alpha$ and $a *^{\alpha} c = b$.
Then $a = b *^{\alpha} c^{-1,\alpha}$.
\end{lemma}

\begin{lemma}[title=eq inv mul of mul eq]
Let $a,b,c : \alpha$ and $b *^{\alpha} a = c$.
Then $a = b^{-1,\alpha} *^{\alpha} c$.
\end{lemma}

\begin{lemma}[title=inv mul eq of eq mul]
Let $a,b,c : \alpha$ and $b = a *^{\alpha} c$.
Then $a^{-1,\alpha} *^{\alpha} b = c$.
\end{lemma}

\begin{lemma}[title=mul inv eq of eq mul]
Let $a,b,c : \alpha$ and $a = c *^{\alpha} b$.
Then $a *^{\alpha} b^{-1,\alpha} = c$.
\end{lemma}

\begin{lemma}[title=eq mul of mul inv eq]
Let $a,b,c : \alpha$ and $a *^{\alpha} c^{-1,\alpha} = b$.
Then $a = b *^{\alpha} c$.
\end{lemma}

\begin{lemma}[title=eq mul of inv mul eq]
Let $a,b,c : \alpha$ and $b^{-1,\alpha} *^{\alpha} a = c$.
Then $a = b *^{\alpha} c$.
\end{lemma}

\begin{lemma}[title=mul eq of eq inv mul]
Let $a,b,c : \alpha$ and $b = a^{-1,\alpha} *^{\alpha} c$.
Then $a *^{\alpha} b = c$.
\end{lemma}

\begin{lemma}[title=mul eq of eq mul inv]
Let $a,b,c : \alpha$ and $a = c *^{\alpha} b^{-1,\alpha}$.
Then $a *^{\alpha} b = c$.
\end{lemma}

\begin{lemma}[title=mul inv]
Let $\alpha$ be a commutative group.
Let $a,b : \alpha$.
Then $(a *^{\alpha} b)^{-1,\alpha} = a^{-1,\alpha} *^{\alpha} b^{-1,\alpha}$.
\end{lemma}

\end{forthel}

\end{document}
