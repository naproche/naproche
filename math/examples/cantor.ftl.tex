\documentclass{article}
\usepackage{naproche}
\libinput[latex]{example-preamble}
\libinput[libraries/foundations]{preamble}
\libinput[libraries/set-theory]{preamble}

\title{Cantor's Theorem}
\author{\Naproche formalization: \vspace{0.5em} \\
Marcel Schütz}
\date{2024}

\begin{document}
\begin{smodule}{cantor.ftl}
  \maketitle
  \hidetlsnumbers

  \usemodule[libraries/foundations]{sub-and-supersets.ftl}
  \usemodule[libraries/set-theory]{cardinals.ftl}

  \noindent This is a formalization of \emph{Cantor's Theorem}
  \cite{Cantor1891}, i.e. of the assertion that any set $x$ is strictly smaller
  than its powerset $\SETpow(x)$
  In an elementary version, this ``strictly smaller than'' relation can be
  formalized via the notion of surjective maps:
  
  \begin{forthel}
    \inputref[libraries/foundations]{cantor.ftl.tex}
  \end{forthel}

  Using the notion of the cardinality $\SETcard{x}$ of a set $x$, the
  ``strictly smaller than'' relation can alternatively be formalized as follows:

  \begin{forthel}
    \inputref[libraries/set-theory]{cantor.ftl.tex}
  \end{forthel}

  \printbibliography
  \printlicense[CcByNcSa]{2024}{Marcel Schütz}
\end{smodule}
\end{document}
