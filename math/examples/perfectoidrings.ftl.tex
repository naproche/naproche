\documentclass[english,11pt]{article}
\usepackage{../latex/lib/naproche}
\addbibresource{../latex/lib/bibliography}

\usepackage{ifthen}
\usepackage{xstring}

\title{A Natural Language Formalization of Perfectoid Rings in \Naproche}
\author{Peter Koepke\\
University of Bonn}
\date{2025}


\newcommand{\val}[2]{#1_{#2}}
\newcommand{\card}[1]{\operatorname{card}(#1)}
\newcommand{\powers}[2]{\{{#1}^{n,#2} \mid n \in \mathbb{N}\}}
\newcommand{\poww}[3]{\{{#1}^{i} \cdot {#2}^{j} \mid i,j \in \mathbb{N}\}}
\newcommand{\Powers}[2]{\{{#1}^{n,#2} \mid n \in \mathbb{N}\}}
\newcommand{\neigh}[2]{N^{#2}_{#1}}
\newcommand{\Seq}[2]{\{#1,\dots,#2\}}
\newcommand{\FinSet}[3]{\{#1_{#2},\dots,#1_{#3}\}}
\newcommand{\Prod}[3]{\prod^{#3}_{i=#2} {#1}_i}

\begin{document}

\maketitle

\abstract{The formalization of P. Scholze's {\em perfectoid spaces} 
\cite{Scholze2012}
in the Lean proof assistant \cite{Lean}
by K. Buzzard, J. Commelin, and P. Massot \cite{BuzzardCommelinMassot2019} was a milestone in bringing 
formal mathematics closer towards leading research-level mathematics. 
The Lean formalization, however, is not ``readable'' by mathematicians, 
since Lean resembles - and indeed is - a programming language 
for building proofs from proof commands.

The \Naproche{} proof system
\cite{DeLonKoepkeLorenzenMartiSchuetzWenzel2021}, in contrast, accepts and checks 
readable texts written in a 
(controlled) natural mathematical language, with familiar proof structurings. 
In this article we present a \Naproche{} formalization of perfectoid {\it rings} 
which are main components of perfectoid spaces. We partly follow mentioned Lean formalization. Our formalization can be loaded 
within Isabelle 2025 
(see \cite{Isabelle}) which then automatically calls the \Naproche{} proof checking 
component of the Isabelle distribution.}

\newpage
\tableofcontents
\newpage

\section*{Introduction}

We define \textit{perfectoid rings} in the proof assistant \Naproche{} \cite{DeLonKoepkeLorenzenMartiSchuetzWenzel2021} 
with the intention to closely approximate the mathematical 
\textit{language} 
of Definition 3.1 of
P. Scholze's \cite{Scholze2012}:
\begin{quote}
{\bf Definition 3.1} A Tate ring $R$ is perfectoid if $R$ is complete, uniform,
i.e. $R^o \subset R$ is bounded, and there exists a
pseudo-uniformizer $\varpi \in R$ such that $\varpi^p | p$ in $R^o$ and the
Frobenius map
$$\Phi: R^o/\varpi \rightarrow R^o/\varpi^p : x \mapsto x^p$$
is an isomorphism.
\end{quote}

\Naproche{} (for Natural Proof Checking) uses the
controlled natural language ForTheL (for Formula Theory Language)
as its
input language (see also \cite{Paskevich2007})
which is a subset of \LaTeX{}.
\Naproche{} can be viewed as a system that
transforms ForTheL statements into proof tasks that are sent out
to an external automated theorem prover (ATP). We mostly use the E first-order prover \cite{Eprover} but also Vampire \cite{Vampire}.
A ForTheL text is correct if
all generated proof tasks can be discharged automatically.

In our formalization, the (\LaTeX{} rendering of the) definition of perfectoid ring reads:
%
\begin{quote}
{\bf Definition 291.}
$R$ is perfectoid iff $R$ is complete and uniform and there
exists a pseudouniformizer $\varpi$ of $R$ such that    
$\varpi^{p,R} | p^{[R]}$ in $R^o$ within $R$
and 
$$\Phi^{R} : R^o / \varpi \cong R^o / \varpi^{p,R}.$$
\end{quote}

Such a definition requires a long chain of intermediate definitions
and propositions, starting from first principles.
The formalization is logically self-contained; theories 
about sets, functions, natural numbers, groups, rings, and topology are
developed as far as required for the intermediate and final definitions. 
Ideally foundational theories would be provided by some library
but the existing libraries and library mechanisms in \Naproche{} 
are still rudimentary.

We make
several simplifying assumptions:
all groups will be commutative and written additively;
rings will be commutative containing a one.
We circumvent some "higher" constructions like quotients of rings,
and express a property like
$$\Phi : R^o / \varpi \cong R^o / \varpi^{p}$$
via congruences mod $\varpi$ and mod $\varpi^{p}$, respectivly.


This text is written as a file 
\texttt{perfectoidring.ftl.tex}
in the \LaTeX{} dialect of \Naproche{} and
typeset by pdf-\LaTeX. The formal parts are printed on a grey background. 
Everything else
is interpreted as "literate" commentary that is not subject to the 
logical checking by the system.
\texttt{perfectoidring.ftl.tex}
can be checked in the \Naproche{} program by opening the file in the Isabelle generic 
proof assistant
\cite{Isabelle}.
\Naproche{} proof checking constitutes a major computational task: each statement generates several
first-order proof tasks each of which can take several seconds of
ATP proving. To check the whole document may require half an hour, 
depending on the available processing power.

\part{Preliminaries}

Our formalization takes place in a familiar mathematical environment. 
\Naproche{} supports standard proof arguments whilst avoiding explicit and tedious foundational
subleties. Nevertheless, \Naproche{} is a formal system and the logical foundations
have to be set up unambiguously and consistently.

Initially the environment contains 
mathematical \textit{objects} and collections of objects, called \textit{classes}. Classes can
be formed as abstraction terms $\{x\mid\phi\}$ where $\phi$ is a mathematical
statement about the variable $x$. Classes which are objects themselves are 
\textit{sets}. Similarly, \textit{maps} send object to objects, and a \textit{function}
is a map which is an object.
Some basic theory of sets and functions is built into \Naproche{}. Further properties
are postulated axiomatically in the next section. Later we shall introduce more types 
of objects like numbers or structures 
by \textbf{Signature}, \textbf{Definition} and \textbf{Axiom} commands.

We do not require all mathematical objects to be sets. Natural numbers, e.g., are introduced as objects
of type ``natural number'' and characterized by arithmetical axioms. This correponds to
widespread intuitions that numbers are ``atomic'' objects.
This focusses proof search on the domains under
consideration.

For convenience we also state some axioms which with some effort could be proved as lemmas. 
Other axioms are used to define classes of structures like groups or 
topological spaces. Finally we use axioms for inductive definitions
of data types and functions, since we do not have
definition mechanisms by recursion.

We start now by expanding our mathematical language by some
singular/plural pairs (vocabulary) and some alternative phrases 
for mathematical properties (macros). These are imported from
library files.


\begin{forthel}

[read examples/aux/vocabulary.ftl.tex]

[read examples/aux/macros.ftl.tex]

[memorylimit 8000]

%%%%%%%%%[prove off][check off]
\end{forthel}
%
The "memorylimit" instruction lets \Naproche{} use up to 8000 Megabytes of working memory per prover
call, the default being 2048.
The grey background marks the strict formalization
embedded in the text and is produced by
a
\verb+\+\verb+begin{forthel}+ ...
\verb+\+\verb+end{forthel}+-environment.


\section{Sets and Functions}
Our foundations correspond to a system of Kelley-Morse set theory with atoms 
but without the axiom of choice 
(see \cite{Kelley1975}). The axiom of infinity will be stated after
the introduction of natural numbers.
New notions and axioms will be \textit{conservative} over classical Kelley-Morse set theory and 
could in principle be reduced to definitions and propositions within
Kelley-Morse set theory, as is usual when developing mathematics in set theories.

\subsection{Axioms}
%
The following axioms express that certain "small" classes of axioms are sets.

\newcommand{\sing}[1]{\{#1\}}
\newcommand{\unp}[2]{\{#1,#2\}}

\begin{forthel}

\begin{proposition} Every set is a class. \end{proposition}

Let $u,v,w$ denote mathematical objects.

\begin{definition} $\emptyset = \{u | u \neq u\}$.
\end{definition}
Let the empty set denote $\emptyset$.

\begin{axiom}[title = Set Existence] $\emptyset$ is a set. \end{axiom}

\begin{definition}
$\sing{u} = \{v | v = u\}$.
\end{definition}
Let the singleton of $u$ denote $\sing{u}$.


\begin{definition}
$\unp{u}{v} = \{w | w = u$ or $w = v\}$.
\end{definition}
Let the unordered pair of $u$ and $v$ denote $\unp{u}{v}$.

\begin{axiom}[title = Pairs]
$\unp{u}{v}$ is a set.
\end{axiom}

\begin{lemma}
$\sing{u}$ is a set.
\end{lemma}

Let $X,Y,Z$ denote sets.

\begin{definition} Assume that every element of $Z$ is a set.
$\bigcup Z = \{x | x \in z$ for some $z \in Z\}$.
\end{definition}
Let the union of $Z$ denote $\bigcup Z$.

\begin{axiom}[title = Union] 
Let $Z$ be a set such that every element of $Z$ is a set.
Then $\bigcup Z$ is a set.
\end{axiom}

\begin{definition} Let $C$ be a class. A subset of $C$ is a set $X$ such
that every element of $X$ is an element of $C$.
\end{definition}

Let $X \subseteq C$ stand for $X$ is a subset of $C$. 

\begin{axiom}[title = Powerset] Let $X$ be a set. Let $Y$ be a class such that
every element of $Y$ is a subset of $X$. Then $Y$ is a set.
\end{axiom}

\begin{axiom}[title = Separation]
Let $X$ be a set. Let $Y$ be a class such that every element
of $Y$ is an element of $X$.
Then $Y$ is a set.
\end{axiom}

\end{forthel}
Further strength is gained using maps and functions.

\begin{forthel}

\begin{proposition}
Assume $F$ is a map and $x \in \dom(F)$.
Then $F(x)$ is an object.
\end{proposition}

\begin{axiom}
Assume that $F$ is a function. Then $\dom(F)$ is a set.
\end{axiom}
Let the domain of $F$ denote $\dom(F)$.

\begin{axiom}
Assume that $F$ is a map and $\dom(F)$ is a set.
Then $F$ is a function.
\end{axiom}

\begin{axiom}[title = Replacement]
Assume that $F$ is a map and $X$ is a set such that $X \subseteq \dom(F)$.
Then there is a set $Y$ such that $Y = \{F(x) \mid x \in X\}$.
\end{axiom}

\end{forthel}
\subsection{Miscellanous Notions and Properties}
\begin{forthel}

\begin{lemma} $\emptyset$ and $X$  are subsets of  $X$. \end{lemma}

\begin{lemma} Let $Z$ be a set such that every element of $Z$ is a set.
Let $z \in Z$. Then $z \subseteq \bigcup Z$.
\end{lemma}
\begin{proof}
Let $x \in z$. Then $x \in \bigcup Z$.
\end{proof}

\begin{lemma} Let $Y$, $Z$ be sets such that every element of $Z$ 
is a subset of $Y$. Then $\bigcup Z$ is a subset of $Y$.
\end{lemma}

\begin{definition}
$X \cup Y = \{x | x \in X$ or $x \in Y\}$. 
\end{definition}
Let the union of $X$ and $Y$ denote $X \cup Y$.

\begin{lemma}
Let $X,Y \subseteq Z$. Then $X \cup Y \subseteq Z$.
\end{lemma}

\begin{definition} Let $Z$ be a set such that every element of $Z$ is a set.
$\bigcap Z = \{x | x \in z$ for all $z \in Z\}$.
\end{definition}
Let the intersection of $Z$ stand for $\bigcap Z$. 

\begin{definition} $X \cap Y = \{x \in X | x \in Y\}$. \end{definition}
Let the intersection of $X$ and $Y$ denote $X \cap Y$. 

\begin{lemma} $X \cap Y$ is a set. \end{lemma}

\begin{definition} $X$ is nonempty iff $X \neq \emptyset$.
\end{definition}

\begin{lemma} Let $Z$ be a nonempty set such that every element of $Z$ is a set.
Then $\bigcap Z$ is a set.
\end{lemma}

\begin{lemma} Let $Z$ be a nonempty set such that every element of $Z$
is a set. Assume that $Y$ is a subset
of every element of $Z$. Then $Y$ is a subset of $\bigcap Z$.
\end{lemma}
\begin{proof}
Let $y$ be an element of $Y$.
$y$ is an element of every element of $Z$.
Thus $y$ is an element of $\bigcap Z$.
\end{proof}
\end{forthel}

\section{Natural Numbers}

We introduce the notion (or type) of natural numbers. Together with
an induction axiom to be stated later, the natural numbers can be understood as the inductive type
generated by $0$ and $+ 1$.

In this chapter we proceed towards prime numbers and divisibility properties of 
factorials and binomial coefficients.   

\subsection{Axioms}

\begin{forthel}

\begin{signature}
A natural number is a mathematical object.
\end{signature}

Let $n,m,k,l,i,j$ denote natural numbers.

\begin{definition}
$\mathbb{N}$ is the collection of natural numbers.
\end{definition}

\begin{axiom}[title = Axiom of Infinity]
$\mathbb{N}$ is a set.
\end{axiom}

\begin{signature}
$0$ is a natural number.
\end{signature}

Let $x$ is nonzero stand for $x \neq 0$.

\begin{signature}
$1$ is a nonzero natural number.
\end{signature}

\begin{signature}
$m + n$ is a natural number.
\end{signature}

\begin{axiom} If $n$ is a nonzero natural number then
$n = m + 1$ for some natural number $m$.
\end{axiom}

\end{forthel}

We postulate basic arithmetic properties of $\mathbb{N}$ axiomatically,
although they could also be proved inductively. 

\begin{forthel}

\begin{signature}
$m * n$ is a natural number.
\end{signature}

\begin{axiom}
$m + n = n + m$.
\end{axiom}

\begin{axiom}
$(m + n) + l = m + (n + l)$.
\end{axiom}

\begin{axiom}
$m + 0 = m = 0 + m$.
\end{axiom}

\begin{axiom}
$m * n = n * m$.
\end{axiom}

\begin{axiom}
$(m * n) * l = m * (n * l)$.
\end{axiom}

\begin{axiom}
$m * 1 = m = 1 * m$.
\end{axiom}

\begin{axiom}
$m * 0 = 0 = 0 * m$.
\end{axiom}

\begin{axiom}
$m * (n + l) = (m * n) + (m * l)$ and
$(n + l) * m = (n * m) + (l * m)$.
\end{axiom}

\begin{axiom}
If $l + m = l + n$ or $m + l = n + l$ then $m = n$.
\end{axiom}

\begin{axiom}
Assume that $l$ is nonzero.
If $l * m = l * n$ or $m * l = n * l$ then $m = n$.
\end{axiom}

\begin{axiom}
If $m + n = 0$ then $m = 0$ and $n = 0$.
\end{axiom}

\end{forthel}

We name two more natural number:

\begin{forthel}

\begin{definition}
$2 = 1 + 1$.
\end{definition}

\begin{definition}
$3 = 2 + 1$.
\end{definition}

\end{forthel}

\subsubsection{Some Equalities}

Since \Naproche{} is weak on algebraic manipulations, we 
state two complex equalities for later use.

\begin{forthel}

\begin{lemma}
Let $u,v,w,x,y$ be natural numbers. 
Then $y + (u * (v * (w * x))) = y + (v * ((u * w) * x))$.
\end{lemma}
\begin{proof}
$(u * (v * (w * x))) = (v * ((u * w) * x))$.
\end{proof}

\begin{lemma}
Let $u,v,w,x,y$ be natural numbers. Then  $(u * (v * (w * x))) + y = (v * (w * (u * x))) + y$.
\end{lemma}


\end{forthel}


\subsection{The Natural Order}

\begin{forthel}

\begin{definition}
$m \leq n$ iff
there exists a natural number $l$ such that $m + l = n$.
\end{definition}

Let $m < n$ stand for $m \leq n$ and $m \neq n$.
Let $n > m$ stand for $m < n$.
Let $n \geq m$ stand for $m \leq n$.


\begin{definition}
Assume that $n \leq m$.
$m - n$ is a natural number $l$ such that $n + l  = m$.
\end{definition}

\end{forthel}

The following three lemmas show that $\leq$ is a partial order:

\begin{forthel}

\begin{lemma}
$m \leq m$.
\end{lemma}

\begin{lemma}
If $m \leq n \leq m$ then $m = n$.
\end{lemma}
\begin{proof}
Let $m \leq n \leq m$.
Take natural numbers $k,l$ such that
$n = m + k$ and $m = n + l$.
Then $m = m + (k + l)$ and $k + l = 0$ and $k = 0$.
Hence $m = n$.
\end{proof}

\begin{lemma}
If $m \leq n \leq l$ then  $m \leq l$.
\end{lemma}
\end{forthel}
We axiomatically postulate monotonicity properties for the arithmetical operations.
\begin{forthel}
\begin{axiom}
$m \leq n$ or $n < m$.
\end{axiom}

\begin{lemma}
Assume that $l < n$.
Then $m + l < m + n$ and $l + m < n + m$.
\end{lemma}

\begin{lemma}
Assume that $m$ is nonzero and $l < n$.
Then $m * l < m * n$ and $l * m < n * m$.
\end{lemma}
\end{forthel}


\subsection{Induction}

\Naproche{} provides an in-built binary relation
symbol $\prec$ as a universal inductive relation: if 

\begin{quote}
inheritance property: at any
point $m$ property $P$ holds at $m$ provided all
$\prec$-predecessors of $m$ satisfy $P$ 
\end{quote} 
then 
\begin{quote}
$P$ holds everywhere.
\end{quote}

\Naproche{} has a proof tactic "by induction [on ...]", which 
reduces the inductive proof goal "$P$ holds everywhere" 
to proving the inheritance property for $P$.

Initially, there is no specification of $\prec$.
The induction proof method for some concrete relation 
is made available by embedding that relation
into $\prec$. Therefore we axiomatically embed the natural
order into $\prec$.

\begin{forthel}
\begin{axiom}
If $m < n$ then $m \prec n$.
\end{axiom}

Let $m$ is inductively smaller than $n$ stand for $m \prec n$.

\end{forthel}

\begin{forthel}

\begin{lemma}
For every natural number $n$: $n = 0$ or $1 \leq n$.
\end{lemma}
\begin{proof}[method = induction]
Let $n$ be a natural number.
Case $n = 0$. Trivial.

Take $n' = n - 1$. 
\end{proof}

\begin{lemma}
If $m \leq n+1$ then $m \leq n$ or $m = n+1$.
\end{lemma}


\begin{lemma}
Let $m \neq 0$. Then $n \leq n * m$.
\end{lemma}
\begin{proof}
$1 \leq m$.
\end{proof}
\end{forthel}

Here are some intuitive facts about the numbers $0,1,2,3$:

\begin{forthel}
\begin{lemma}
If $m \leq 0$ then $m = 0$.
\end{lemma}

\begin{lemma}
If $m \leq 1$ then $m = 0$ or $m = 1$.
\end{lemma}

\begin{lemma}
If $m \leq 2$ then $m = 0$ or $m = 1$ or $m = 2$.
\end{lemma}

\begin{lemma}
$0 < 1 < 2 < 3$.
\end{lemma}

\end{forthel}


\subsection{Division}

\begin{forthel}

\begin{definition}
$n$ divides $m$ iff for some $l$: $m = n * l$.
\end{definition}

Let $x | y$ denote $x$ divides $y$.
Let a divisor of $x$ denote a natural number that divides $x$.

\begin{lemma}
Assume $l | m | n$.
Then $l | n$.
\end{lemma}

\begin{lemma}
Let $l | m$ and $l | m + n$.
Then $l | n$.
\end{lemma}
\begin{proof}
Assume that $l$ is nonzero.
Take a natural number $q$ such that $m = l * q$.
Take a natural number $r$ such that $m + n = l * r$.

Let us show that
$q \leq r$.

Proof by contradiction.
Assume the contrary. Then $r < q$.
$m+n = l * r < l * q = m$.
Contradiction. qed.

Take $s = r - q$.
We have $(l * q) + (l * s) = l * r = m + n = (l * q) + n$.
Hence $n = l * s$.
\end{proof}

\begin{lemma}
Let $m | n \neq 0$.
Then $m \leq n$.
\end{lemma}

\begin{definition} $n$ is even iff $2$ divides $n$.
\end{definition}

Let $n$ is odd stand for $n$ is not even.

\begin{lemma}
For all natural numbers $n$ $n$ is even or $n = (2 * m) + 1$ for some $m$.
\end{lemma}
\begin{proof}[method = induction]
Let $n$ be a natural number.

Case $n = 0$. Trivial.

Take $n' = n - 1$.

Case $n'$ is even. Take a natural number $m'$ such that $n' = 2 * m'$.
Then $n = (2 * m) + 1$ for some $m$. qed.

Take a natural number $m'$ such that $n' = (2 * m') + 1$.
Then $n = ((2 * m') + 1) + 1 = 2 * (m' + 1)$.
Hence $n$ is even.

\end{proof}

\end{forthel}


\subsection{Primes}

\begin{forthel}

Let $p,d$ denote natural numbers.

Let $n$ is nontrivial stand for $n \neq 0$ and $n \neq 1$.

\begin{definition}
$p$ is prime iff $p$ is nontrivial and
for every divisor $d$ of $p$ $d = 1$ or $d = p$.
\end{definition}
Let a prime number stand for a natural number that is prime.

\begin{lemma} $2$ is prime.
\end{lemma}

\begin{lemma}
Every even prime number is equal to $2$.
\end{lemma}

\begin{lemma} $3$ is prime.
\end{lemma}

\begin{lemma}
Every nontrivial $n$ has a prime divisor.
\end{lemma}
\begin{proof}[method = induction]
Let $n$ be a nontrivial natural number.
Assume that $n$ is not prime.
Take a divisor $m$ of $n$ such that $m \neq 1$ and $m \neq n$.
$m$ is inductively smaller than $n$.
Every prime divisor of $m$ is a prime divisor of $n$.
\end{proof}
\end{forthel}

\subsection{Euclid's Lemma}

We need that prime numbers are prime 
elements in the ring of integers, or the halfring of natural numbers.
The following argument is taken over almost verbatim from 
the Wikipedia article on Euclid's Lemma
\cite{Wikipedia_EuclidsLemma}.


\begin{forthel}
\begin{definition}
$m$ and $n$ are coprime iff every common divisor of $m$ and $n$ is equal to $1$. 
\end{definition}

\begin{lemma}
If $m$ and $m$ are coprime then $m = 1$.
\end{lemma}

Let $a,b$ denote natural numbers.

\begin{lemma}
For all nonzero natural numbers $n,a,b$
if $n | a * b$ and $n$ and $a$ are coprime then $n$ divides $b$.
\end{lemma}

\begin{proof}[method = induction on $a * b$]

Let $n,a,b$ be nonzero natural numbers such that $n | a * b$ and $n$ and $a$ are coprime.
Take a natural number $q$ such that $n * q = a * b$.


Case $n = a$. Then $n = 1$ and $n | b$. qed.

Case $a > n$. Then $q \geq b$.

$$n * (q - b) = (n * q) - (n * b) = (a * b) - (n * b) = (a - n) * b.$$
%
Thus $n$ divides $(a - n) * b$.
$n$ and $a - n$ are coprime.
$(a - n) * b <  a * b$.
$(a - n) * b$ is inductively smaller than $a * b$.
Thus $n$ divides $b$.
qed.

Hence $n > a$ and $b \geq q$.
$$(n - a) * q = (n * q) - (a * q) = (a * b) - (a * q) = a * (b - q).$$
$n - a$ divides $a * (b - q)$.
$n - a$ and $a$ are coprime.
$a * (b - q) < a * b$.
$a * (b - q)$ is inductively smaller than $a * b$.
Thus $n - a$ divides $b - q$.

Take a natural number $r$ such that $b - q = r * (n - a)$.

$(n - a) * q = (n * q) - (a * q) = (a * b) - (a * q)
= a * (b - q) = a * (r * (n - a)) = (a * r) * (n - a)$.

$n - a \neq 0$ and $q = a * r$.

$a * (n * r) = n * (a * r) = n * q = a * b$.

Then $n * r = b$ and $n$ divides $b$.
\end{proof}

\begin{theorem}[title = Euclids Lemma]
Let $p$ be a prime number and $p | m * n$. Then $p | m$ or $p | n$.
\end{theorem}

\end{forthel}

\subsection{Binomial Coefficients}

\newcommand{\cho}[2]{{#1 \choose #2}}

We introduce factorials $n!$ and binomial coefficients
$\cho{n}{i}$ together with some divisibility properties
which will be used later in connection with
Frobenius homomorphisms.

\newcommand{\fac}[1]{#1 !}

\begin{forthel}

\begin{signature}
$\fac{n}$ is a natural number.
\end{signature}

\begin{axiom}
$\fac{0} = 1$.
\end{axiom}

\begin{axiom}
$\fac{(n+1)} = (n+1) * \fac{n}$.
\end{axiom}

\begin{lemma} Let $n \neq 0$. Then $n | \fac{n}$.
\end{lemma}
\begin{proof}
Take $m = n - 1$.
Then $\fac{n} = n * \fac{m}$.
\end{proof}

\begin{lemma}
Let $p$ be a prime number.
For all $k$ if $k < p$ then $p$ does not divide $\fac{k}$.
\end{lemma}
\begin{proof}[method = induction on $k$]
Let $k$ be a natural number.
Case $k = 0$. Trivial.
Let $k < p$.
Take $l = k - 1$.
Then $p$ does not divide $k$. $p$ does not divide $\fac{l}$.
$p$ does not divide $k * \fac{l}$.
\end{proof}

\begin{signature}
$\cho{n}{i}$ is a natural number.
\end{signature}

\begin{axiom}
$\cho{n}{0} = 1$.
\end{axiom}

\begin{axiom}
$\cho{0}{i} = 0$ for all $i$ such that $i \geq 1$.
\end{axiom}

\begin{axiom} Let $i \geq 1$.
$\cho{n+1}{i} = \cho{n}{i} + \cho{n}{i-1}$.
\end{axiom}

\begin{theorem} For all natural numbers $n$ and 
all natural numbers $i$
if $i \geq n + 1$ then $\cho{n}{i} = 0$.
\end{theorem}
\begin{proof}[method = induction on n]
Let $n$ be a natural number.
Case $n = 0$. Trivial.

Take a natural number $n'$ such that $n = n' + 1$.
Let $i$ be a natural number such that $i \geq n + 1$.
Then $\cho{n}{i} = \cho{n'}{i} + \cho{n'}{i-1} = 0 + 0 = 0$.
\end{proof}


\begin{theorem} For all $n$
$\cho{n}{n} = 1$.
\end{theorem}
\begin{proof}[method = induction]
Let $n$ be a natural number.
Case $n = 0$. Trivial.

Take a natural number $m$ such that $n = m + 1$.
Then $\cho{n}{n} = \cho{m}{n} + \cho{m}{m} = 0 + 1 = 1$.

\end{proof}
\end{forthel}
%
The next property would normally be expressed as
%
$$\cho{n}{i} = \frac{\fac{n}}{\fac{i} * \fac{(n-i)}}.$$
%
Since we don't have rational numbers available we
reformulate the property within the natural numbers.
%
\begin{forthel}
\begin{lemma} For all natural numbers $n$ and all natural numbers $i$ 
such that  $0 \leq i \leq n$
$$\fac{n} = \cho{n}{i} * (\fac{i} * \fac{(n-i)}).$$
\end{lemma}

\begin{proof}[method = induction on $n$]
Let $n$ be a natural number.
Let $i$ be a natural number such that $0 \leq i \leq n$.

Case $n = 0$. Trivial.

Take $m = n - 1$.  

For all natural numbers $j$ such that $0 \leq j \leq m$ $\fac{m} = 
\cho{m}{j} * (\fac{j} * \fac{(m-j)})$.
Indeed $m$ is inductively smaller than $n$.

Case $i = 0$. Trivial.

Case $i = n$. Trivial.

$i - 1 \leq m$ and $m - (i-1) = n - i$. $m - i$ is a natural number. 
$n - i = (m - i) + 1$.

Then

$\fac{n} = n * \fac{m} = ((n-i) + i) * \fac{m} = ((n-i) * \fac{m}) + (i * \fac{m})$

$= ((n-i) * \fac{m}) + (i * (\cho{m}{i-1} * (\fac{(i-1)} * \fac{(m - (i-1))})))$

$= ((n-i) * \fac{m}) + (\cho{m}{i-1} * ((i * \fac{(i-1)}) * \fac{(m - (i-1))}))$

$=  ((n-i) * \fac{m}) + (\cho{m}{i-1} * (\fac{i} * \fac{(m - (i-1))}))$

$=  ((n-i) * \fac{m}) + (\cho{m}{i-1} * (\fac{i} * \fac{(n-i)}))$

$= ((n-i) * (\cho{m}{i} * (\fac{i} * \fac{(m-i)}))) + (\cho{m}{i-1} * (\fac{i} * \fac{(n-i)}))$

$= ((\cho{m}{i} * (\fac{i} * ((n-i) * \fac{(m-i)})))) + (\cho{m}{i-1} * (\fac{i} * \fac{(n-i)}))$

$= ((\cho{m}{i} * (\fac{i} * \fac{(n-i)}))) + (\cho{m}{i-1} * (\fac{i} * \fac{(n-i)}))$

$= (\cho{m}{i} + \cho{m}{i-1}) * (\fac{i} * \fac{(n-i)}) $

$=  \cho{n}{i} * (\fac{i} * \fac{(n-i)})$.

Indeed $\cho{n}{i} = \cho{m}{i} + \cho{m}{i-1}$.

\end{proof}


\begin{lemma}
Let $p$ be a prime number.
Let $0 < k < p$. Then
$p$ divides $\cho{p}{k}$.
\end{lemma}
\begin{proof}
$\fac{p} = \cho{p}{k} * (\fac{k} * \fac{(p-k)})$.
$p$ divides $\fac{p}$.
$p$ does not divide $\fac{k}$.
$p$ does not divide $\fac{(p-k)}$.
$p$ does not divide $\fac{k} * \fac{(p-k)}$.
Hence 
$p$ divides $\cho{p}{k}$.
\end{proof}

\end{forthel}


\section{Finite Sequences and Sets}

We can get standard notations in the pretty-printed
text by employing \LaTeX{} macros as ForTheL terms.
The notation $\Seq{m}{n}$ can be generated 
from the ForTheL function pattern
\verb+\Seq{m}{n}+ 
by the 
macro:
\begin{verbatim}
\newcommand{\Seq}[2]{\{#1,\dots,#2\}}
\end{verbatim}.

\begin{forthel}

Let $m,n,i$ denote natural numbers.

\begin{definition}
$\Seq{m}{n}$ is the collection of
natural numbers $i$ such that $m \leq i \leq n$.
\end{definition}

\begin{lemma}
$\Seq{m}{n}$ is a set.
\end{lemma}

\begin{lemma}
$\Seq{1}{0} = \emptyset$.
\end{lemma}

\begin{lemma}
Let $m \leq n$. Then $\Seq{1}{m} \subseteq \Seq{1}{n}$.
\end{lemma}

\begin{lemma}
$\Seq{1}{n} \cup \sing{n+1} = \Seq{1}{n+1}$.
\end{lemma}
\begin{proof}

(1) $\Seq{1}{n} \cup \sing{n+1} \subseteq \Seq{1}{n+1}$.

Proof. $\Seq{1}{n} \subseteq \Seq{1}{n+1}$.
$\sing{n+1} \subseteq \Seq{1}{n+1}$.
Qed.

(2) $\Seq{1}{n+1} \subseteq \Seq{1}{n} \cup \sing{n+1}$.

Proof.
Let $m \in \Seq{1}{n+1}$. $1 \leq m \leq n+1$. Let us show that 
$m \in \Seq{1}{n} \cup \sing{n+1}$.

Case $m \leq n$. Trivial.

$m = n+1$. Trivial.

Qed.

\end{proof}

\begin{definition}
A sequence of length $n$ is a
map $a$ such that $\dom(a) = \Seq{1}{n}$.
\end{definition}

Let $a$ denote a map.
Let $a_{i}$ stand for $a(i)$.

\begin{definition}
Let $\Seq{1}{n} \subseteq \dom(a)$.
$$\FinSet{a}{1}{n} = \{ a_{i} \mid i \in \Seq{1}{n}\}.$$
\end{definition}


\begin{lemma}
Assume that $\Seq{1}{n} \subseteq \dom(a)$.
Then $\FinSet{a}{1}{n}$ is a set.
\end{lemma}
\begin{proof}
Consider $X = \Seq{1}{n}$. $X$ is a set such that $X \subseteq \dom(a)$.
[timelimit 10]
Take a set $Y$ such that $Y = \{ a_{i} | i \in X\}$.
[timelimit 3]
$Y = \FinSet{a}{1}{n}$.
\end{proof}


\begin{lemma}
Assume that $\Seq{1}{n} \subseteq \dom(a)$.
Let $m \leq n$. Then $\Seq{1}{m} \subseteq \dom(a)$ and
$\FinSet{a}{1}{m} \subseteq \FinSet{a}{1}{n}$.
\end{lemma}

\begin{lemma}
$\FinSet{a}{1}{0} = \emptyset$.
\end{lemma}

\begin{lemma}
Assume that $\Seq{1}{n+1} \subseteq \dom(a)$. 

Then
$\FinSet{a}{1}{n}$, $\sing{a_{n+1}}$ are sets and
$$\FinSet{a}{1}{n} \cup \sing{a_{n+1}} = \FinSet{a}{1}{n+1}.$$
\end{lemma}
\begin{proof}
$n+1 \in \dom(a)$.
$\sing{a_{n+1}}$ is a set and $\FinSet{a}{1}{n}$ is a set.

(1) $\FinSet{a}{1}{n} \cup \sing{a_{n+1}} \subseteq \FinSet{a}{1}{n+1}$.

Proof.
$\FinSet{a}{1}{n} \subseteq \FinSet{a}{1}{n+1}$.
$\sing{a_{n+1}}  \subseteq \FinSet{a}{1}{n+1}$.
Qed.

(2) $\FinSet{a}{1}{n+1} \subseteq \FinSet{a}{1}{n} \cup \sing{a_{n+1}}$.

Proof.
Let $x \in \FinSet{a}{1}{n+1}$. Take $m \in \Seq{1}{n+1}$ such that
$x = a_{m}$. Let us show that $x \in \FinSet{a}{1}{n} \cup \sing{a_{n+1}}$.

$m \in \Seq{1}{n} \cup \sing{n+1}$.

Case $m \in \Seq{1}{n}$. Trivial.

$m \in \sing{n+1}$. Then $m = n+1$. Trivial.
Qed.

\end{proof}

\begin{definition} Let $A$ be a class.
$A$ is finite iff
$A = \FinSet{a}{1}{n}$ for some natural number $n$ and some map $a$ such 
that $\Seq{1}{n} \subseteq \dom(a)$.
\end{definition}

\begin{lemma}
$\emptyset$ is finite.
\end{lemma}
\begin{proof}
Define $a(x) = x$ for $x \in \Seq{1}{0}$.
Then $\FinSet{a}{1}{0} = \emptyset$.
\end{proof}

\begin{lemma}
Let $u$ be an object. Then
$\sing{u}$ is finite.
\end{lemma}
\begin{proof}
Define $a(x) = u$ for $x \in \Seq{1}{1}$.
Then $\FinSet{a}{1}{1} = \sing{u}$. 
\end{proof}

\begin{lemma}
For every object $u$:
$u \in \Seq{1}{2}$ iff $u = 1$ or $u = 2$.
\end{lemma}

\begin{lemma}
Let $u,v$ be objects. Then
$\unp{u}{v}$ is finite.
\end{lemma}

\begin{proof}
Define
$  a(x) =
    \begin{cases}
    u  & : x = 1 \\
    v  & : x = 2
    \end{cases} $
  for $x \in \Seq{1}{2}$.

Then $\FinSet{a}{1}{2} = \unp{u}{v}$. 
\end{proof}

\begin{definition} Let $C$ be a class.
$C$ is infinite iff $C$ is not finite.
\end{definition}

\end{forthel}

\section{(Additive) Groups}

It is common to define structures as an underlying set with further
structural elements that "belong" to that set: "a group is a set with
\dots". This naive approach to structures is employed here.
It works alright as long as one
does not consider two different structures of the same kind on the
same underlying set. In general one will need stronger structure mechanisms as 
provided in other proof systems.
  
\subsection{Axioms}

\begin{forthel}
%[prove off][check off]
[synonym group/-s]
\begin{signature} A group is a set. \end{signature}

Let an additive group stand for group.

Let $G$ denote a group.

\begin{signature} $0^{G}$ is an element of $G$.
\end{signature}

\begin{signature} Let $x,y \in G$. $x +^{G} y$ is an element of $G$. \end{signature}

\begin{signature} Let $x \in G$. $-^{G} x$ is an element of $G$. \end{signature}

\begin{axiom}[id = Associativity, title = Associativity] $(x +^{G} y) +^{G} z = x +^{G} (y +^{G} z)$ for all $x,y,z \in G$.
\end{axiom} 
\begin{axiom} $x +^{G} 0^{G} = x$ for all $x \in G$.
\end{axiom}
\begin{axiom} $x +^{G} (-^{G} x) = 0^{G}$ for all $x \in G$.
\end{axiom}
\begin{axiom} $x +^{G} y = y +^{G} x$ for all $x,y \in G$.
\end{axiom}

Let $x -^{G} y$ stand for $x +^{G} (-^{G} y)$.

\begin{lemma}
$ -^{G} (-^{G} x) = x$ for all $x \in G$.
\end{lemma}
\begin{proof}
Let $x \in G$.
$ -^{G} (-^{G} x) =  -^{G} (-^{G} x) +^{G} (-^{G} x +^{G} x)
=  (-^{G} (-^{G} x) +^{G} -^{G} x) +^{G} x = x$.
\end{proof}
\end{forthel}
Since \Naproche{} does not provide automatic type elaboration the obvious 
group parameter $G$ has to be carried along in the \LaTeX{} source.
In the \LaTeX{} \textit{output}, however, we can hide it by 
some \LaTeX{} trickery:
\begin{verbatim}
\catcode`^\active
\def^#1{\ifthenelse{\equal{#1}{G}} {\unskip} {\sp{#1}}}
\end{verbatim}

The first command makes the symbol \verb|^|
active as a possible macro name.
The second line defines the macro
\verb|^{...}|: if the argument is $G$, it is
discarded. Otherwise we treat it as a superscript. Since \verb|^| is 
used as a macro symbol, we generate the superscript instead by 
the \LaTeX{} command \verb|\sp{ }|.
The definition of the macro requires to \verb|\usepackage{ifthen}|.
\catcode`^\active
\def^#1{\ifthenelse{\equal{#1}{G}} {\unskip} {\sp{#1}}}
The previous lemma and proof then print out more concisely as:

\begin{forthel}
\begin{lemma}
$ -^{G} (-^{G} x) = x$ for all $x \in G$.
\end{lemma}
\begin{proof}
Let $x \in G$.
$ -^{G} (-^{G} x) =  -^{G} (-^{G} x) +^{G} (-^{G} x +^{G} x)
=  (-^{G} (-^{G} x) +^{G} -^{G} x) +^{G} x = x$.
\end{proof}

\begin{lemma}
Let $x,y \in G$. Then $-^{G} (x +^{G} y) =-^{G} x -^{G} y$.
\end{lemma}
\begin{proof}
$(x +^{G} y) +^{G} (-^{G} x -^{G} y) = 0^{G}$.
\end{proof}

\end{forthel}



\subsection{Subgroups}

We define subgroups and the subgroup generated by a set of generators.

\begin{forthel}

Let $G$ denote a group. 

[synonym subgroup/-s]

\begin{definition} A subgroup of $G$ is a subset $H$ of $G$ such that
$0^{G} \in H$ and for all $x,y \in H$ 

$x +^{G} y \in H$ and $-^{G} x \in H$.
\end{definition}

\end{forthel}
Since we shall later consider subgroups of rings where we also have a multiplication,
we also use the adjective "additive":
\begin{forthel}

Let an additive subgroup of $G$ stand for a subgroup of $G$.

\begin{lemma} $G$ is a subgroup of $G$. \end{lemma}

\begin{lemma} $\sing{0^{G}}$ is a subgroup of $G$.
\end{lemma}

\begin{proof}
Let $x,y \in \sing{0^{G}}$. Then $x,y = 0^{G}$.
$x +^{G} y = 0^{G} \in \sing{0^{G}}$ and $-^{G} x = 0^{G} \in \sing{0^{G}}$.
\end{proof}

\begin{lemma} Let $F$ be a nonempty set such that every element of $F$ is a subgroup of $G$.
Then $\bigcap F$ is a subgroup of $G$.
\end{lemma}

\begin{definition}
Let $A$ be a subset of $G$.
$[A]^{G} = \{x \in G \mid x \in H$ for all subgroups $H$ of $G$ 
such that $A \subseteq H\}$.
\end{definition}

\begin{lemma}
Let $A$ be a subset of $G$. Then $[A]^{G}$ is a subgroup of $G$ such
that $A \subseteq [A]^{G}$.
\end{lemma}
\begin{proof}

(1) $[A]^{G}$ is a subset of $G$.

(2) $0^{G} \in [A]^{G}$.

(3) $x +^{G} y \in [A]^{G}$ for all $x,y \in [A]^{G}$.

(4) $-^{G} x \in [A]^{G}$ for all $x \in [A]^{G}$.

Hence $[A]^{G}$ is a subgroup of $G$.

Let $x \in A$. 
Then $x \in [A]^{G}$.
\end{proof}

Let the additive closure of $A$ in $G$ stand for $[A]^{G}$.

\begin{definition}
Let $C \subseteq G$ and $c \in G$.
$c \oplus^{G} C = \{c +^{G} d \mid d \in C\}$.
\end{definition}
\end{forthel}

\section{Rings}
\subsection{Axioms}

We shall only consider commutative rings with $1$.
After defining a group as a {\it set} with further structure, we can now
define a ring as a {\it group} together with multiplication and a $1$.


\begin{forthel}

\begin{signature} A ring is an additive group. \end{signature}
Let $R$ denote a ring.

\begin{signature} $1^{R}$ is an element of $R$ such that $1^{R} \neq 0^{R}$.
\end{signature}

\begin{signature} Let $x,y \in R$. $x \cdot^{R} y$ is an element of $R$. \end{signature}

\begin{axiom} $(x \cdot^{R} y) \cdot^{R} z = x \cdot^{R} (y \cdot^{R} z)$ 
for all $x,y,z \in R$.
\end{axiom} 
\begin{axiom} $x \cdot^{R} 1^{R} = x$ for all $x \in R$.
\end{axiom}
\begin{axiom}[title = Commutativity]
$x \cdot^{R} y = y \cdot^{R} x$ for all $x,y \in R$.
\end{axiom}
\begin{axiom}[title = Distributivity]  Let $x,y,z \in R$.
$(x +^{R} y) \cdot^{R} z = (x \cdot^{R} z) +^{R} (y \cdot^{R} z)$.
\end{axiom}
\end{forthel}
Again readability is improved if we hide the recurring superscript $^R$
by the above method.
\catcode`^\active
\def^#1{\ifthenelse{\equal{#1}{R}} {\unskip} {\sp{#1}}}


\begin{forthel}

[timelimit 5]
\begin{lemma} 
$z \cdot^{R} (x +^{R} y) = (z \cdot^{R} x) +^{R} (z \cdot^{R} y)$ 
for all $x,y,z \in R$.
\end{lemma}
\begin{proof}
Let $x,y,z \in R$.
$z \cdot^{R} (x +^{R} y) = 
(x +^{R} y) \cdot^{R} z = (x \cdot^{R} z) +^{R} (y \cdot^{R} z)$
$= (z \cdot^{R} x) +^{R} (z \cdot^{R} y)$. 
Indeed consider $v = x \cdot^{R} z$.
\end{proof}

\begin{lemma} $ 0^{R} \cdot^{R} x = 0^{R}$ for all $x \in R$. \end{lemma}
\begin{proof} Let $x \in R$.
$ (0^{R} \cdot^{R} x) +^{R} x = (0^{R} \cdot^{R} x) +^{R} (1^{R} \cdot^{R} x) 
= (0^{R} +^{R} 1^{R}) \cdot^{R} x =  1^{R} \cdot^{R} x = x $.
$ 0^{R} \cdot^{R} x = (0^{R} \cdot^{R} x) +^{R} ((-^{R}x) +^{R} x) = 0^{R}$.
\end{proof}
[timelimit 10]
\begin{lemma}
$ (-^{R} 1^{R}) \cdot^{R} x = -^{R} x$ for all $x \in R$. 
\end{lemma}
\begin{proof}
Let $x \in R$.
$ (-^{R} 1^{R}) \cdot^{R} x = 
((-^{R} 1^{R}) \cdot^{R} x) +^{R} (x +^{R} (-^{R} x)) =
(((-^{R} 1^{R}) \cdot^{R} x) +^{R} x) +^{R} (-^{R} x) =
(((-^{R} 1^{R}) \cdot^{R} x) +^{R} (1^{R} \cdot^{R} x)) +^{R} (-^{R} x)$.
\end{proof}
[timelimit 3]

\begin{lemma}
$-^{R}( x \cdot^{R} y) = (-^{R}  x) \cdot^{R} y$ for all $x,y \in R$.
\end{lemma}
\begin{proof} Let $x,y \in R$.
$-^{R}( x \cdot^{R} y) = (-^{R}1^{R}) \cdot^{R} ( x \cdot^{R} y)$.
\end{proof}

\begin{lemma}
$ (-^{R} 1^{R}) \cdot^{R} (-^{R} 1^{R}) = 1^{R}$.
\end{lemma}

\begin{lemma}
$x -^{R} (x -^{R} y) = y$ for all $x,y \in R$.
\end{lemma}

[timelimit 8]
\begin{lemma}  Let $x,y,z \in R$.
$(x -^{R} y) \cdot^{R} z = (x \cdot^{R} z) -^{R} (y \cdot^{R} z)$.
\end{lemma}

\begin{proof} 
Take $u = -^{R} y$. $u \in R$.

$(x +^{R} (-^{R} y)) \cdot^{R} z $

$= (x +^{R} u) \cdot^{R} z $.

$(x +^{R} u) \cdot^{R} z $


$= (x \cdot^{R} z) +^{R} (u \cdot^{R} z)$


$= (x \cdot^{R} z) +^{R} ((-^{R} y) \cdot^{R} z)$.


\end{proof}
[timelimit 3]
\end{forthel}

\subsection{Subrings}
\begin{forthel}

Let $R$ denote a ring. 

\begin{definition} Let $A$ be a subset of $R$. $A$ is closed under
multiplication in $R$ iff $x \cdot^{R} y \in A$ for all $x,y \in A$.
\end{definition}


\begin{definition} A subring of $R$ is an additive subgroup $S$ of $R$ such that
$1^{R} \in S$ and $S$ is closed under multiplication in $R$.
\end{definition}

\begin{lemma} $R$ is a subring of $R$. \end{lemma}

\end{forthel}

\subsection{Operations on Subsets of Rings}
\renewcommand{\ast}{\cdot}
\renewcommand{\star}{\ }


\begin{forthel}

Let $R$ denote a ring.

\begin{definition}
Let $B,C \subseteq R$.
$B \star^{R} C = \{b \cdot^{R} c \in R | b \in B$ and $c \in C\}$.
\end{definition} 

\begin{definition}
Let $B,C \subseteq R$. 
$B \ast^{R} C = [B \star^{R} C]^{R}$.
\end{definition}

\end{forthel}
We iterate the formation of additive closures of pointwise products:
\begin{forthel}

Let $n$ denote natural numbers.

\begin{signature}
Let $B \subseteq R$ and $n \geq 1$.
$B^{n,R,subset}$ is a subset of $R$.
\end{signature}

\begin{axiom}
Let $B \subseteq R$. $B^{1,R,subset} = B$.
\end{axiom}

\begin{axiom}
Let $B \subseteq R$ and $n \geq 1$.
$B^{n+1,R,subset} = B^{n,R,subset} \ast^{R} B$. 
\end{axiom}

\begin{lemma}
Let $B \subseteq R$. Then
$B^{n,R,subset}$ is a subgroup of $R$ 
where $n$ is a natural number and $n > 1$.
\end{lemma}
\begin{proof}[method = induction on $n$]

Let $n$ be a natural number.

CASE $n \leq 1$. trivial.

Take a natural number $m$ such that $n = m + 1$.
Then $B^{n,R,subset} = B^{m,R,subset} \ast^{R} B$ and
$B^{n,R,subset}$ is a subgroup of $R$.
\end{proof}


\begin{definition} Let $B \subseteq R$.
$\Powers{B}{R} = \{B^{n,R,subset} | n \in \mathbb{N}$ and $n \geq 1\} $.
\end{definition}

\begin{lemma}
Let $B \subseteq R$. Then $\Powers{B}{R}$ is a set.
\end{lemma}
\begin{proof} $\mathbb{N}$ is a set.
\end{proof}

\end{forthel}

\subsection{Divisibility}

\begin{forthel}

Let $R$ denote a ring.

\begin{definition}
A unit in $R$ is an element $x$ of $R$ such that $x \cdot^{R} y = 1^{R}$
for some element $y$ of $R$. 
\end{definition}

\begin{definition} Let $X$ be a subset of $R$. Let $x,y \in R$. 
$x$ divides $y$ in $X$ within $R$ iff 
$x \cdot^{R} z = y$ for some element $z$ of $X$.
\end{definition}

Let $x | y$ in $X$ within $R$ stand for $x$ divides $y$ in $X$ within $R$.

\begin{lemma} Let $S$ be a subring of $R$.
Let $x,y,z \in R$. Let $x | y$ in $S$ within $R$ and $y | z$ in $S$ within $R$.
Then $x | z$ in $S$ within $R$.
\end{lemma}

\begin{lemma} Let $S$ be a subring of $R$.
Let $x,y,z \in R$. Let $x | y$ in $S$ within $R$ and $x | z$ in $S$ within $R$.
Then $x | (y +^{R} z)$ in $S$ within $R$.
\end{lemma}
\begin{proof}
Take an element $u$ of $S$ such that $x \cdot^{R} u = y$.
Take an element $v$ of $S$ such that $x \cdot^{R} v = z$.
Then $x \cdot^{R} (u +^{R} v) = y +^{R} z$ and $u +^{R} v \in S$. 
\end{proof}

\begin{lemma} Let $S$ be a subring of $R$.
Let $x,y\in R$. Let $x | y$ in $S$ within $R$.
Then $x | (-^{R} y)$ in $S$ within $R$.
\end{lemma}

\begin{lemma} Let $S$ be a subring of $R$.
Let $x,y,z \in R$. Let $x | y$ in $S$ within $R$ and $x | z$ in $S$ within $R$.
Then $x | (y -^{R} z)$ in $S$ within $R$.
\end{lemma}
\begin{proof}
$x | (-^{R} z)$ in $S$ within $R$.
\end{proof}

\end{forthel}
\subsection{Congruences}
\begin{forthel}
Let $x \equiv_{S} y$ mod $a$ within $R$ stand for $a$ divides $x-^{R} y$ in $S$ within $R$.

\end{forthel}
We show that for $S$ being a subring of $R$ and $a \in S$, $\equiv_{S}$ mod $a$ is an equivalence relation and
a congruence relation on $S$.

\begin{forthel}

\begin{lemma}
Let $S$ be a subring of $R$ and $a \in S$.
$x \equiv_{S} x$ mod $a$ within $R$ for all $x \in S$.
\end{lemma}

\begin{lemma}
Let $S$ be a subring of $R$ and $a \in S$.
Let $x,y$ be elements of $S$ such that
$x \equiv_{S} y$ mod $a$ within $R$. 
Then $y \equiv_{S} x$ mod $a$ within $R$.
\end{lemma}
\begin{proof}
$a$ divides  $x-^{R} y$ in $S$ within $R$.
$a$ divides  $-^{R}(x-^{R} y)$ in $S$ within $R$.
$y -^{R} x = -^{R}(x -^{R} y)$.
\end{proof}

\begin{lemma}
Let $S$ be a subring of $R$ and $a \in S$.
Let $x,y,z$ be elements of $S$ such that
$x \equiv_{S} y$ mod $a$ within $R$
and $y \equiv_{S} z$ mod $a$ within $R$. 
Then $x \equiv_{S} z$ mod $a$ within $R$.
\end{lemma}
\begin{proof}
Take $u \in S$ such that $x -^{R} y = a \cdot^{R} u$.
Take $v \in S$ such that $y -^{R} z = a \cdot^{R} v$.
Then
 $x -^{R} z =  
((x -^{R} y) +^{R} y) -^{R} z =
(x -^{R} y) +^{R} (y -^{R} z) =
(a \cdot^{R} u) +^{R} (a \cdot^{R} v) =
a \cdot^{R} (u +^{R} v)
$.
\end{proof}

[timelimit 6]
\begin{lemma}
Let $S$ be a subring of $R$ and $a \in S$.
Let $x,x',y,y'$ be elements of $S$ such that
$x \equiv_{S} x'$ mod $a$ within $R$
and $y \equiv_{S} y'$ mod $a$ within $R$. 
Then $x +^{R} y \equiv_{S} x' +^{R} y'$ mod $a$ within $R$.
\end{lemma}
\begin{proof}
Take $u \in S$ such that $x -^{R} x' = a \cdot^{R} u$.
Take $v \in S$ such that $y -^{R} y' = a \cdot^{R} v$.
Then
$(x +^{R} y) -^{R} (x' +^{R} y')  =  (x +^{R} y) +^{R} (-^{R} x' +^{R} -^{R} y')
= ((x +^{R} y) -^{R} x') -^{R} y' 
= (x +^{R} (y -^{R} x')) -^{R} y'
= ((x -^{R} x') +^{R} y) -^{R} y'
= (x -^{R} x') +^{R} (y -^{R} y')
= (a \cdot^{R} u) +^{R} (a \cdot^{R} v)$.

\end{proof}
[timelimit 3]

\begin{lemma}
Let $S$ be a subring of $R$ and $a \in S$.
Let $x,x'$ be elements of $S$ such that
$x \equiv_{S} x'$ mod $a$ within $R$. 
Then $(-^{R}x) \equiv_{S} (-^{R}x')$ mod $a$ within $R$.
\end{lemma}
\begin{proof}
Take $z \in S$ such that $x -^{R} x' = a \cdot^{R} z$.

$(-^{R} 1^{R}) \cdot^{R} (x -^{R} x') = (-^{R} 1^{R}) \cdot^{R} (a \cdot^{R} z)$.
$(-^{R}x) -^{R} (-^{R}x') =
((-^{R} 1^{R}) \cdot^{R} x) -^{R} ((-^{R} 1^{R}) \cdot^{R} x') =
(-^{R} 1^{R}) \cdot^{R} (a \cdot^{R} z)$.
\end{proof}

\begin{lemma}
Let $S$ be a subring of $R$ and $a \in S$.
Let $x,x',y,y'$ be elements of $S$ such that
$x \equiv_{S} x'$ mod $a$ within $R$
and $y \equiv_{S} y'$ mod $a$ within $R$. 
Then $x \cdot^{R} y \equiv_{S} x' \cdot^{R} y'$ mod $a$ within $R$.
\end{lemma}
\begin{proof}
Take $u \in S$ such that $x -^{R} x' = a \cdot^{R} u$.
Take $v \in S$ such that $y -^{R} y' = a \cdot^{R} v$.

$x \cdot^{R} y = (x \cdot^{R} y) +^{R} 0^{R}$

$= (x \cdot^{R} y) +^{R} (-^{R} (x' \cdot^{R} y) +^{R} (x' \cdot^{R} y))$

$= ((x \cdot^{R} y) -^{R} (x' \cdot^{R} y)) +^{R} (x' \cdot^{R} y)$

$= ((x -^{R} x') \cdot^{R} y) +^{R} (x' \cdot^{R} y)$.


$(x' \cdot^{R} y) -^{R} (x' \cdot^{R} y') = x' \cdot^{R} (y -^{R} y')$.

[timelimit 10]

Hence

$(x \cdot^{R} y) -^{R} (x' \cdot^{R} y') = $
 
$ (((x -^{R} x') \cdot^{R} y) +^{R} (x' \cdot^{R} y)) -^{R} (x' \cdot^{R} y')$ 

$ = ((x -^{R} x') \cdot^{R} y) +^{R} ((x' \cdot^{R} y) -^{R} (x' \cdot^{R} y'))$

(by Associativity).

$(x \cdot^{R} y) -^{R} (x' \cdot^{R} y') = $

$((x -^{R} x') \cdot^{R} y) +^{R} ((x' \cdot^{R} y) -^{R} (x' \cdot^{R} y'))$ 

$ = ((x -^{R} x') \cdot^{R} y) +^{R} (x' \cdot^{R} (y -^{R} y'))$

$ = ((a \cdot^{R} u) \cdot^{R} y) +^{R} (x' \cdot^{R} (a \cdot^{R} v))$

$ = ((a \cdot^{R} u) \cdot^{R} y) +^{R} ((x' \cdot^{R} a) \cdot^{R} v)$

$ = (a \cdot^{R} (u \cdot^{R} y)) +^{R} ((a \cdot^{R} x') \cdot^{R} v)$

$ = (a \cdot^{R} (u \cdot^{R} y)) +^{R} (a \cdot^{R} (x' \cdot^{R} v))$

$ = a \cdot^{R} ((u \cdot^{R} y) +^{R} (x' \cdot^{R} v))$.

[timelimit 3]

$a$ divides $(x \cdot^{R} y) -^{R} (x' \cdot^{R} y')$
in $S$ within $R$. Indeed
$(u \cdot^{R} y) +^{R} (x' \cdot^{R} v) \in S$.
\end{proof}

\end{forthel}
We could now define equivalence sets 
%
$$[x]=\{x' \in S \mid x' \equiv_{S} x \ mod \ a\}$$
%
and form a quotient
%
$$ S/a = \{[x]\mid x \in S\}.$$
%
But as we shall not continue working with quotients in this article
we will only consider congruences mod $a$. To talk about 
quotient rings that have distinguished elements $0,1$ with 
$0 \neq 1$, we show:
\begin{forthel}
\begin{lemma}
Let $S$ be a subring of $R$ and $a \in S$.
Assume that $a$ does not divide $1^{R}$ in $S$ within $R$.
Then not $0^{R} \equiv_{S} 1^{R}$ mod $a$ within $R$.
\end{lemma}


\end{forthel}

\subsection{Embedding Natural Numbers into a Ring}

Natural numbers are represented in a ring $R$ as sums of $1^{R}$'s.
As \Naproche{} does not yet provide recursive function definitions,
we work axiomatically: the notation or function symbol $n^{[R]}$ is
introduced into the language, and then two axioms postulate the initial case
and the successor case of the recursion.

Finite powers of ring elements are introduced in a similar way.

\begin{forthel}

Let $m$ denote natural numbers.

Let $R$ denote a ring.

\begin{signature} Let $n$ be a natural number. $n^{[R]}$ is an element of $R$.
\end{signature}

\begin{axiom} $0^{[R]} = 0^{R}$.
\end{axiom}

\begin{axiom} $(n+1)^{[R]} = n^{[R]} +^{R} 1^{R}$.
\end{axiom}

\begin{lemma}
$1^{[R]} = 1^{R}$.
\end{lemma}

\begin{lemma}
Let $S$ be a subring of $R$. Then $n^{[R]} \in S$ for all $n$.
\end{lemma}
\begin{proof}[method = induction]
Let $n$ be a natural number.

Case $n = 0$. Trivial.

Take $m = n - 1$. Then $n^{[R]} = m^{[R]} +^{R} 1^{R} \in S$.
\end{proof}

\begin{lemma}
For all $n$ we have
$(m + n)^{[R]} = m^{[R]} +^{R} n^{[R]}$.
\end{lemma}
\begin{proof}[method = induction]
Let $n$ be a natural number.

Case $n = 0$. Trivial.

Take $l = n - 1$.

$(m + n)^{[R]} = 
((m + l) + 1)^{[R]} =
(m + l)^{[R]} +^{R} 1^{R} =
(m^{[R]} +^{R} l^{[R]}) +^{R} 1^{R} =
m^{[R]} +^{R} n^{[R]}$.
\end{proof}

\begin{lemma}
For all $n$ we have
$(m * n)^{[R]} = m^{[R]} \cdot^{R} n^{[R]}$.
\end{lemma}
\begin{proof}[method = induction]
Let $n$ be a natural number.

Case $n = 0$. Trivial.

Take $l = n - 1$.

$(m * n)^{[R]} = 
((m * l) + m)^{[R]} =
(m * l)^{[R]} +^{R} m^{[R]} =
(m^{[R]} \cdot^{R} l^{[R]}) +^{R} m^{[R]} =
(m^{[R]} \cdot^{R} l^{[R]}) +^{R} (m^{[R]} \cdot^{R} 1^{R}) =
m^{[R]} \cdot^{R} (l + 1)^{[R]}$.
\end{proof}

\begin{lemma}
Let $S$ be a subring of $R$. Let $m$ divide $n$. Then
$m^{[R]}$ divides $n^{[R]}$ in $S$ within $R$.
\end{lemma}

\begin{lemma}
Let $S$ be a subring of $R$ and $x \in S$.
Then $x \equiv_{S} -^{R}x$ mod $(2^{[R]})$ within $R$.
\end{lemma}
\begin{proof}
$x -^{R} (-^{R}x) = x +^{R} x = (1^{R} +^{R} 1^{R}) \cdot^{R} x = 2^{[R]} \cdot^{R} x$.
\end{proof}
%[prove off][check off]
\end{forthel}

\subsection{Exponentiation}

Exponentiation in $R$ is defined by recursive axioms that specify the
new function symbol \verb|x^{n,R}| which prints as $x^{n,R}$. 
\begin{forthel}

\begin{signature} Let $x \in R$. $x^{n,R}$ is an element of $R$.
\end{signature}

\begin{axiom} Let $x \in R$. Then $x^{0,R} = 1^{R}$.
\end{axiom}

\begin{axiom} Let $x \in R$. Then $x^{n+1,R} = x^{n,R} \cdot^{R} x$.
\end{axiom}
\end{forthel}
If the parameter $R$ is understood from the context one would like to turn
$R$ into an implicit parameter and hide \verb|,R|. Since we also want to
remove superscripts \verb|^{R}| we redefine the "macro" \verb|^|.

\begin{verbatim}
\catcode`^\active

\def^#1{\ifthenelse{\equal{#1}{R}} {\unskip} 
{{\IfEndWith{#1}{,R}{\sp{\StrBefore{#1}{,R}}}{\sp{#1}}}}}
\end{verbatim}

\catcode`^\active

\def^#1{\ifthenelse{\equal{#1}{R}} {\unskip} 
       {{\IfEndWith{#1}{,R}{\sp{\StrBefore{#1}{,R}}}{\sp{#1}}}}}

\noindent The string operations in this macro require to 
\verb|\usepackage{xstring}|.

\begin{forthel}
\begin{lemma} Assume that $n \neq 0$. Then $0^{R}^{n,R} = 0^{R}$. \end{lemma}
\begin{proof}
Take $m$ such that
$n = m + 1$. Then $0^{R}^{m+1,R} = 0^{R}^{m,R} \cdot^{R} 0^{R}
= 0^{R}$.
\end{proof}
\end{forthel}
%
The statement of the next lemma requires brackets $(\dots)$ for
disambiguation. It should be possible to extend the ForTheL grammar in
order to eliminate those brackets.
%
\begin{forthel}
\begin{lemma}
Assume that $x \in R$. Then for all natural numbers $n$

(if $n$ is even then $(-^{R}x)^{n,R} = x^{n,R}$)
and 

(if $n$ is odd then $(-^{R}x)^{n,R} = -^{R}(x^{n,R})$).
\end{lemma}
\begin{proof}[method = induction]
Let $n$ be a natural number.

(1) Case $n = 0$. Trivial.

Take $m = n - 1$.

(2a) Case $n$ is even. 

$m$ is inductively smaller than $n$ and odd.
$(-^{R}x)^{m,R} = -^{R}(x^{m,R})$.
$(-^{R}x)^{n,R} = (-^{R}x)^{m,R} \cdot^{R} (-^{R}x) = (-^{R}(x^{m,R})) \cdot^{R} (-^{R}x)
= (x^{m,R}) \cdot^{R} x = x^{n,R}$.
End.

(2b) Case $n$ is odd. 

$m$ is inductively smaller than $n$ and even.
$(-^{R}x)^{m,R} = x^{m,R}$.
$(-^{R}x)^{n,R} = (-^{R}x)^{m,R} \cdot^{R} (-^{R}x) = (x^{m,R}) \cdot^{R} (-^{R}x)
= -^{R} ((x^{m,R}) \cdot^{R} x) = -^{R}(x^{n,R})$.
End.

\end{proof}

\begin{lemma} For all natural numbers $n$ we have $1^{R}^{n,R} = 1^{R}$. \end{lemma}
\begin{proof}[method = induction on $n$]
\end{proof}

\begin{lemma} Let $n$ be a natural numbers. Then $(-^{R} 1^{R})^{n,R} = 1^{R}$
or $(-^{R} 1^{R})^{n,R} = -^{R} 1^{R}$. 
\end{lemma}
\begin{proof}
\end{proof}

\begin{lemma} Let $x \in R$ and $m$ be a natural number.
For all natural numbers $n$
we have
$x^{m,R} \cdot^{R} x^{n,R} = x^{m + n,R}$.
\end{lemma}
\begin{proof}[method = induction on $n$]
Let $n$ be a natural number.

CASE $n = 0$. Trivial.

Take a natural number $k$ such that $n = k + 1$.
Then

$x^{m,R} \cdot^{R} x^{n,R} = $

$x^{m,R} \cdot^{R} x^{k+1,R} = $

$x^{m,R} \cdot^{R} (x^{k,R} \cdot^{R} x) = $

$(x^{m,R} \cdot^{R} x^{k,R}) \cdot^{R} x = $

$x^{m + k,R} \cdot^{R} x = $

$x^{(m + k)+1,R} = x^{m+n,R}.$
\end{proof}

\begin{lemma} Let $x \in R$ and $m$ be a natural number.
For all natural numbers $n$
we have
$(x^{m,R})^{n,R} = x^{m * n,R}$.
\end{lemma}
\begin{proof}[method = induction on $n$]
Let $n$ be a natural number.

CASE $n = 0$. Trivial.

Take a natural number $k$ such that $n = k + 1$. Then
$(x^{m,R})^{n,R} =
(x^{m,R})^{k+1,R} =
(x^{m,R})^{k,R} \cdot^{R} (x^{m,R})$.
\end{proof}

\begin{lemma} Let $x,y \in R$. For all natural numbers
$n$ we have
$(x \cdot^{R} y)^{n,R} = (x^{n,R} \cdot^{R} y^{n,R})$.
\end{lemma}
\begin{proof}[method = induction on $n$]
Let $n$ be a natural number.

CASE $n = 0$. Trivial.

Take $m$ such that $n = m + 1$.
$(x \cdot^{R} y)^{n,R} =
(x \cdot^{R} y)^{m+1,R} =
(x \cdot^{R} y)^{m,R} \cdot^{R} (x \cdot^{R} y).$

$(x^{m,R} \cdot^{R} y^{m,R}) \cdot^{R} (x \cdot^{R} y) =
(x^{m,R} \cdot^{R} x) \cdot^{R} ( y^{m,R} \cdot^{R} y)$.
\end{proof}

\begin{definition} Let $x \in R$. $\powers{x}{R} = \{x^{n,R} | n \in \mathbb{N} \} $.
\end{definition}

\begin{lemma} $\powers{x}{R}$ is a set for every $x \in R$.
\end{lemma}

[timelimit 11]
\begin{lemma} Let $x \in R$. Then
$\powers{x}{R}$ is a subset of $R$ that is closed under multiplication in $R$.
\end{lemma}
[timelimit 3]

\begin{lemma}
Let $A$ be a subset of $R$ that is closed under multiplication in $R$.
Then the additive closure of $A$ in $R$ is closed under multiplication in $R$.
\end{lemma}

\begin{proof}
Define $B = \{b \in R \mid b \cdot^{R} a \in [A]^{R}$ for all $a \in A\}$.

(1) $B$ is a subset of $R$.

(2) $0^{R} \in B$.

(3) $x +^{R} y \in B$ for all $x,y \in B$.

Proof.
Let $x,y \in B$.
Let $a \in A$. Then $x \cdot^{R} a, y \cdot^{R} a \in [A]^{R}$.
Then $(x +^{R} y) \cdot^{R} a = (x \cdot^{R} a) +^{R} (y \cdot^{R} a) \in [A]^{R}$.
Qed.

(4) $-^{R} x \in B$ for all $x \in B$.

Proof.
Let $x \in B$.
Let $a \in A$. Then $x \cdot^{R} a \in [A]^{R}$.
Then $(-^{R} x) \cdot^{R} a = -^{R} (x\cdot^{R} a) \in [A]^{R}$.
Qed.

(5) $B$ is a subgroup of $R$.

(6) $A \subseteq B$. Indeed $A$ is closed under multiplication in $R$.

(7) $[A]^{R} \subseteq B$.

(8) For all $b \in [A]^{R}$ and all $a \in A$ $b \cdot^{R} a \in [A]^{R}$.

(9) For all $a \in A$ and all $b \in [A]^{R}$  $a \cdot^{R} b \in [A]^{R}$.

Define $C = \{c \in R \mid c \cdot^{R} a \in [A]^{R}$ for all $a \in [A]^{R}\}$.

(10) $C$ is a subset of $R$.

(11) $0^{R} \in C$.

(12) $x +^{R} y \in C$ for all $x,y \in C$.

Proof.
Let $x,y \in C$.
Let $a \in [A]^{R}$. Then $x \cdot^{R} a, y \cdot^{R} a \in [A]^{R}$.
Then $(x +^{R} y) \cdot^{R} a = (x \cdot^{R} a) +^{R} (y \cdot^{R} a) \in [A]^{R}$.
Qed.

(13) $-^{R} x \in C$ for all $x \in C$.

Proof.
Let $x \in C$.
Let $a \in [A]^{R}$. Then $x \cdot^{R} a \in [A]^{R}$.
Then $(-^{R} x) \cdot^{R} a = -^{R} (x\cdot^{R} a) \in [A]^{R}$.
Qed.

(14) $C$ is a subgroup of $R$.

(15) $A \subseteq C$ (by 9). 

(16) $[A]^{R} \subseteq C$.

(17) For all $b \in [A]^{R}$ and all $a \in [A]^{R}$ $b \cdot^{R} a \in [A]^{R}$.

\end{proof}
\end{forthel}

\section{Binomial Properties in Rings}

\subsection{Binomial Coefficients}

As natural numbers, the binomial coefficients can be 
canonically embedded 
into a ring $R$ together with their recursive properties.

\begin{forthel}

Let $R$ denote a ring.
Let $n,m,k,l,i,j$ denote natural numbers.

\begin{lemma}
$\cho{n}{0}^{[R]} = 1^{R}$.
\end{lemma}

\begin{lemma}
$\cho{0}{i}^{[R]} = 0^{R}$ for all $i$ such that $i \geq 1$.
\end{lemma}

\begin{lemma} Let $i \geq 1$.
$\cho{n+1}{i}^{[R]} = \cho{n}{i}^{[R]} +^{R} \cho{n}{i-1}^{[R]}$.
\end{lemma}

\begin{lemma}
If $i \geq n + 1$ then $\cho{n}{i}^{[R]} = 0^{R}$.
\end{lemma}

\begin{theorem} 
$\cho{n}{n}^{[R]} = 1^{R}$.
\end{theorem}

\begin{lemma} Let $S$ be a subring of $R$.
Let $p$ be a prime number.
Let $0 < i < p$. Then
$p^{[R]}$ divides $\cho{p}{i}^{[R]}$ in $S$ within $R$.
\end{lemma}

\end{forthel}

\subsection{Binomial Sums}

We shall deal with "partial" binomial sums of the form
%
$$\sum_{i=0}^m {n \choose i} x^{n-i} y^{i}.$$
%
We view the summation as a first-order function symbol in the 
arguments $m,n,x,y,R$ and specify the function axiomatically.

\newcommand{\binsum}[5]{{\sum}_{i = 0}^{{#1}} {#2 \choose i} #3^{#2 - i} #4^{i}}

\begin{forthel}

\begin{signature} Let $x,y \in R$.
$$\binsum{k}{m}{x}{y}{R}$$ is an element of $R$.
\end{signature}

\begin{axiom} Let $m$ be a natural number and $x,y \in R$.
Then 
$$\binsum{0}{m}{x}{y}{R} = x^{m,R}.$$
\end{axiom}

\begin{axiom}[id = Summation recursion, title = Summation recursion]
Let $1 \leq k \leq m$ and $x,y \in R$. 
Then 

$\binsum{k}{m}{x}{y}{R} = $

$(\binsum{k-1}{m}{x}{y}{R}) +^{R} 
(\cho{m}{k}^{[R]} \cdot^{R} (x^{m - k,R} \cdot^{R} y^{k,R}))$. 
\end{axiom}
\end{forthel}
The following proofs in this section will be tedious since 
\Naproche{} so far does not have efficient term rewriting or an inbuilt ring theory.
\begin{forthel}

\begin{lemma}
Let $x,y \in R$. For all natural numbers $m$ if $m \leq n$ then

$(\binsum{m}{n}{x}{y}{R}) \cdot^{R} (x +^{R} y) =$

$(\binsum{m}{n+1}{x}{y}{R}) +^{R} 
(\cho{n}{m}^{[R]} \cdot^{R} (x^{n-m,R} \cdot^{R} y^{m+1,R})).$
\end{lemma}

\begin{proof}[method = induction]
Let $m$ be a natural number such that $m \leq n$.

Case $m = 0$. 

$(\binsum{0}{n}{x}{y}{R}) \cdot^{R} (x +^{R} y) =$

$x^{n,R} \cdot^{R} (x +^{R} y) =$

$(x^{n,R} \cdot^{R} x) +^{R} (x^{n,R} \cdot^{R} y) =$

$x^{n+1,R} +^{R} (x^{n,R} \cdot^{R} y) =$

$(\binsum{0}{n+1}{x}{y}{R}) +^{R} (1^{R} \cdot^{R}(x^{n,R} \cdot^{R} y)) =$

$(\binsum{0}{n+1}{x}{y}{R}) +^{R} (\cho{n}{0}^{[R]} \cdot^{R}(x^{n-0,R} \cdot^{R} y)) =$

$(\binsum{0}{n+1}{x}{y}{R}) +^{R} (\cho{n}{0}^{[R]} \cdot^{R}(x^{n-0,R} \cdot^{R} y^{0+1,R}))$.

Indeed $y = 1^{R} \cdot^{R} y$.

End.

Take $m' = m-1$.

[timelimit 30]

$(\binsum{m}{n}{x}{y}{R}) \cdot^{R} (x +^{R} y) = $

$((\binsum{m-1}{n}{x}{y}{R}) +^{R} 
(\cho{n}{m}^{[R]} \cdot^{R} (x^{n - m,R} \cdot^{R} y^{m,R}))) \cdot^{R} (x +^{R} y)$.

[timelimit 3]

$((\binsum{m-1}{n}{x}{y}{R}) +^{R} 
(\cho{n}{m}^{[R]} \cdot^{R} (x^{n - m,R} \cdot^{R} y^{m,R}))) \cdot^{R} (x +^{R} y) =$

$ ((\binsum{m'}{n}{x}{y}{R}) \cdot^{R} (x +^{R} y)) +^{R} 
((\cho{n}{m}^{[R]} \cdot^{R} (x^{n - m,R} \cdot^{R} y^{m,R})) \cdot^{R} (x +^{R} y)) =$

$ ((\binsum{m'}{n+1}{x}{y}{R}) +^{R} (\cho{n}{m'}^{[R]} \cdot^{R} (x^{n-m',R} \cdot^{R} y^{m'+1,R}))) $
$ +^{R} ((\cho{n}{m}^{[R]} \cdot^{R} (x^{n - m,R} \cdot^{R} y^{m,R})) \cdot^{R} (x +^{R} y)) =$

$ ((\binsum{m'}{n+1}{x}{y}{R}) +^{R} (\cho{n}{m'}^{[R]} \cdot^{R} (x^{n-m',R} \cdot^{R} y^{m'+1,R}))) $
$ +^{R} 
(((\cho{n}{m}^{[R]} \cdot^{R} (x^{n - m,R} \cdot^{R} y^{m,R})) \cdot^{R} x)
   +^{R} ((\cho{n}{m}^{[R]} \cdot^{R} (x^{n - m,R} \cdot^{R} y^{m,R})) \cdot^{R} y)) $.

[timelimit 20]

$ ((\binsum{m'}{n+1}{x}{y}{R}) +^{R} (\cho{n}{m'}^{[R]} \cdot^{R} (x^{n-m',R} \cdot^{R} y^{m'+1,R}))) $
$ +^{R} 
(((\cho{n}{m}^{[R]} \cdot^{R} (x^{n - m,R} \cdot^{R} y^{m,R})) \cdot^{R} x)
   +^{R} ((\cho{n}{m}^{[R]} \cdot^{R} (x^{n - m,R} \cdot^{R} y^{m,R})) \cdot^{R} y)) =$

$ ((\binsum{m'}{n+1}{x}{y}{R}) +^{R} ((\cho{n}{m'}^{[R]} \cdot^{R} (x^{n-m',R} \cdot^{R} y^{m'+1,R})) $
$ +^{R} 
((\cho{n}{m}^{[R]} \cdot^{R} (x^{n - m,R} \cdot^{R} y^{m,R})) \cdot^{R} x)))
   +^{R} ((\cho{n}{m}^{[R]} \cdot^{R} (x^{n - m,R} \cdot^{R} y^{m,R})) \cdot^{R} y) $.

$(\cho{n}{m}^{[R]} \cdot^{R} (x^{n - m,R} \cdot^{R} y^{m,R})) \cdot^{R} y =
\cho{n}{m}^{[R]} \cdot^{R} ((x^{n - m,R} \cdot^{R} y^{m,R}) \cdot^{R} y) =
\cho{n}{m}^{[R]} \cdot^{R} (x^{n - m,R} \cdot^{R} y^{m+1,R})$.


$ ((\binsum{m'}{n+1}{x}{y}{R}) +^{R} ((\cho{n}{m'}^{[R]} \cdot^{R} (x^{n-m',R} \cdot^{R} y^{m'+1,R})) $
$ +^{R} 
((\cho{n}{m}^{[R]} \cdot^{R} (x^{n - m,R} \cdot^{R} y^{m,R})) \cdot^{R} x)))
   +^{R} ((\cho{n}{m}^{[R]} \cdot^{R} (x^{n - m,R} \cdot^{R} y^{m,R})) \cdot^{R} y) =$

$ ((\binsum{m'}{n+1}{x}{y}{R}) +^{R} ((\cho{n}{m'}^{[R]} \cdot^{R} (x^{n-m',R} \cdot^{R} y^{m'+1,R})) $
$ +^{R} 
((\cho{n}{m}^{[R]} \cdot^{R} (x^{n - m,R} \cdot^{R} y^{m,R})) \cdot^{R} x)))
   +^{R} (\cho{n}{m}^{[R]} \cdot^{R} (x^{n - m,R} \cdot^{R} y^{m+1,R})) $.

$x^{n-m',R} \cdot^{R} y^{m'+1,R} = x^{(n+1)-m,R} \cdot^{R} y^{m,R}.$

$\cho{n}{m'}^{[R]} \cdot^{R} (x^{n-m',R} \cdot^{R} y^{m'+1,R}) = 
\cho{n}{m'}^{[R]} \cdot^{R} (x^{(n+1)-m,R} \cdot^{R} y^{m,R})$.

$(\cho{n}{m}^{[R]} \cdot^{R} (x^{n - m,R} \cdot^{R} y^{m,R})) \cdot^{R} x = $

$\cho{n}{m}^{[R]} \cdot^{R} ((x^{n - m,R} \cdot^{R} y^{m,R}) \cdot^{R} x)$.

$(x^{n - m,R} \cdot^{R} y^{m,R}) \cdot^{R} x = 
x^{n - m,R} \cdot^{R} (y^{m,R} \cdot^{R} x) =
x^{n - m,R} \cdot^{R} (x \cdot^{R} y^{m,R}) =
(x^{n - m,R} \cdot^{R} x) \cdot^{R} y^{m,R}$.

$(n - m) + 1 = (n + 1) - m$.

$(x^{n - m,R} \cdot^{R} x) \cdot^{R} y^{m,R} =$
$x^{(n+1)-m,R} \cdot^{R} y^{m,R}$.

$(\cho{n}{m}^{[R]} \cdot^{R} (x^{n - m,R} \cdot^{R} y^{m,R})) \cdot^{R} x = $

$\cho{n}{m}^{[R]} \cdot^{R} (x^{(n+1) - m,R} \cdot^{R} y^{m,R})$.

$x^{n-m',R} \cdot^{R} y^{m'+1,R} = x^{(n+1)-m,R} \cdot^{R} y^{m,R}.$

$ ((\binsum{m'}{n+1}{x}{y}{R}) +^{R} ((\cho{n}{m'}^{[R]} \cdot^{R} (x^{n-m',R} \cdot^{R} y^{m'+1,R})) $
$ +^{R} 
((\cho{n}{m}^{[R]} \cdot^{R} (x^{n - m,R} \cdot^{R} y^{m,R})) \cdot^{R} x)))
   +^{R} (\cho{n}{m}^{[R]} \cdot^{R} (x^{n - m,R} \cdot^{R} y^{m+1,R})) = $

$ ((\binsum{m'}{n+1}{x}{y}{R}) +^{R} ((\cho{n}{m'}^{[R]} \cdot^{R} (x^{(n+1)-m,R} \cdot^{R} y^{m,R})) $
$ +^{R} 
(\cho{n}{m}^{[R]} \cdot^{R} (x^{(n+1) - m,R} \cdot^{R} y^{m,R}))))
   +^{R} (\cho{n}{m}^{[R]} \cdot^{R} (x^{n - m,R} \cdot^{R} y^{m+1,R})) =$

$ ((\binsum{m'}{n+1}{x}{y}{R}) +^{R} ((\cho{n}{m'}^{[R]} +^{R}  \cho{n}{m}^{[R]} )\cdot^{R} (x^{(n+1)-m,R} \cdot^{R} y^{m,R}) $
$  ))
   +^{R} (\cho{n}{m}^{[R]} \cdot^{R} (x^{n - m,R} \cdot^{R} y^{m+1,R})) $.

$\cho{n}{m'}^{[R]} +^{R}  \cho{n}{m}^{[R]} = \cho{n+1}{m}^{[R]}$.



$ \binsum{m}{n+1}{x}{y}{R} =$

$(\binsum{m'}{n+1}{x}{y}{R}) +^{R} (\cho{n+1}{m}^{[R]} \cdot^{R} (x^{(n+1)-m,R} \cdot^{R} y^{m,R}))$.



$ ((\binsum{m'}{n+1}{x}{y}{R}) +^{R} ((\cho{n}{m'}^{[R]} +^{R}  \cho{n}{m}^{[R]} )\cdot^{R} (x^{(n+1)-m,R} \cdot^{R} y^{m,R}) $
$  ))
   +^{R} (\cho{n}{m}^{[R]} \cdot^{R} (x^{n - m,R} \cdot^{R} y^{m+1,R})) =$


$ ((\binsum{m'}{n+1}{x}{y}{R}) +^{R} (\cho{n+1}{m}^{[R]} \cdot^{R} (x^{(n+1)-m,R} \cdot^{R} y^{m,R}) $
$  ))
   +^{R} (\cho{n}{m}^{[R]} \cdot^{R} (x^{n - m,R} \cdot^{R} y^{m+1,R})) =$

$ (\binsum{m}{n+1}{x}{y}{R}) $
$  +^{R} (\cho{n}{m}^{[R]} \cdot^{R} (x^{n - m,R} \cdot^{R} y^{m+1,R})) $.

\end{proof}
\end{forthel}
%
This leads to the standard binomial formula:
%
\begin{forthel}
\begin{theorem}
Let $x,y \in R$. For all natural numbers $n$ we have
$$(x +^{R} y)^{n,R} = \binsum{n}{n}{x}{y}{R}.$$
\end{theorem}
\begin{proof}[method = induction]
Let $n$ be a natural number.

Case $n = 0$. Trivial.

Take $n' = n-1$.

$(x +^{R} y)^{n,R} =$

$(x +^{R} y)^{n',R} \cdot^{R} (x +^{R} y) =$

$\binsum{n'}{n'}{x}{y}{R} \cdot^{R} (x +^{R} y) =$

$(\binsum{n'}{n'+1}{x}{y}{R}) +^{R} (\cho{n'}{n'}^{[R]} \cdot^{R} (x^{n'-n',R} \cdot^{R} y^{n'+1,R})) =$

$(\binsum{n-1}{n}{x}{y}{R}) +^{R} (\cho{n}{n}^{[R]} \cdot^{R} (x^{n-n,R} \cdot^{R} y^{n,R})).$

[timelimit 8]


$(\binsum{n-1}{n}{x}{y}{R}) +^{R} (\cho{n}{n}^{[R]} \cdot^{R} (x^{n-n,R} \cdot^{R} y^{n,R})) =$

$\binsum{n}{n}{x}{y}{R}$ (by summation recursion).

[timelimit 3]

\end{proof}

\end{forthel}

\subsection{Divisibility} 

\begin{forthel}

\begin{lemma} Let $S$ be a subring of $R$.
Let $x,y,z \in S$. Let $x | y$ in $S$ within $R$.
Then $x | (y \cdot^{R} z)$ in $S$ within $R$.
\end{lemma}

\begin{lemma} Let $S$ be a subring of $R$.
Let $x \in S$. For all natural numbers $n$
$x^{n,R} \in S$.
\end{lemma}
\begin{proof}[method = induction]
Let $n$ be a natural number.

Case $n=0$. Trivial.

Take $n'=n-1$.
$x^{n',R} \in S$.
$x^{n,R} = x^{n',R} \cdot^{R} x \in S$.
\end{proof}


\begin{lemma} Let $p$ be a prime number.
Let $S$ be a subring of $R$ and $x,y \in S$. 
Then for all natural numbers $m$ if $m < p$ then
$\binsum{m}{p}{x}{y}{R} \equiv_{S} x^{p,R}$ mod $p^{[R]}$ within $R$.
\end{lemma}

\begin{proof}[method = induction]
Let $m$ be a natural number such that $m < p$.

Case $m = 0$. Trivial.

Take $m' = m - 1$.

(f1) $\binsum{m'}{p}{x}{y}{R} \equiv_{S} x^{p,R}$ mod $p^{[R]}$ within $R$.


$p^{[R]}$ divides $\cho{p}{m}^{[R]}$ in $S$ within
$R$.

$x^{p - m,R} \in S$.

$p^{[R]}$ divides $(\cho{p}{m}^{[R]} \cdot^{R} (x^{p - m,R} \cdot^{R} y^{m,R}))$ in $S$ within
$R$.

(f2) $(\cho{p}{m}^{[R]} \cdot^{R} (x^{p - m,R} \cdot^{R} y^{m,R})) \equiv_{S} 0^{R}$ mod $p^{[R]}$ within $R$.




$(\binsum{m-1}{p}{x}{y}{R}) +^{R} (\cho{p}{m}^{[R]} \cdot^{R} (x^{p - m,R} \cdot^{R} y^{m,R}))  
\equiv_{S} x^{p,R}$ mod $p^{[R]}$ within $R$.

Proof.
Take $a = \binsum{m-1}{p}{x}{y}{R}$. 
Take $b = (\cho{p}{m}^{[R]} \cdot^{R} (x^{p - m,R} \cdot^{R} y^{m,R}))$.
Take $c = x^{p,R}$.
$a \equiv_{S} c$ mod $p^{[R]}$ within $R$.
$b \equiv_{S} 0^{R}$ mod $p^{[R]}$ within $R$.
$b \in S$.
$a +^{R} b \equiv_{S} c $ mod $p^{[R]}$ within $R$.
Qed.

%(f3) $(\binsum{m-1}{p}{x}{y}{R}) +^{R} (\cho{p}{m}^{[R]} \cdot^{R} (x^{p - m,R} \cdot^{R} y^{m,R}))  
%\equiv_{S} x^{p,R}$ mod $p^{[R]}$ within $R$.


$\binsum{m}{p}{x}{y}{R}  
\equiv_{S} x^{p,R} $ mod $p^{[R]}$ within $R$.

Proof. $m \leq p$. 

[timelimit 30]

$\binsum{m}{p}{x}{y}{R} = 
(\binsum{m-1}{p}{x}{y}{R}) +^{R} (\cho{p}{m}^{[R]} \cdot^{R} (x^{p - m,R} \cdot^{R} y^{m,R}))$.
[timelimit 3]

QED.

[timelimit 3]

\end{proof}
\end{forthel}
The following lemma will be used later to show that the Frobenius
map is an additive homomorphism.
\begin{forthel}
\begin{lemma} Let $p$ be a prime number.
Let $S$ be a subring of $R$ and $x,y \in S$. 
Then $(x +^{R} y)^{p,R} \equiv_{S} x^{p,R} +^{R} y^{p,R}$ mod $p^{[R]}$ within $R$.
\end{lemma}

\begin{proof} $p \geq 1$.

[timelimit 30]

$(x +^{R} y)^{p,R} =  \binsum{p}{p}{x}{y}{R} =$

$ (\binsum{p-1}{p}{x}{y}{R}) +^{R} (\cho{p}{p}^{[R]} \cdot^{R} (x^{p - p,R} \cdot^{R} y^{p,R})) =$

$ (\binsum{p-1}{p}{x}{y}{R}) +^{R} y^{p,R}$.

Indeed $\cho{p}{p}^{[R]} \cdot^{R} (x^{p - p,R} \cdot^{R} y^{p,R}) = y^{p,R}$.

$ \binsum{p-1}{p}{x}{y}{R} \equiv_{S} x^{p,R}$ mod $p^{[R]}$ within $R$.

$ \binsum{p-1}{p}{x}{y}{R} +^{R} y^{p,R} \equiv_{S} x^{p,R} +^{R} y^{p,R}$ mod $p^{[R]}$ within $R$.

Proof.
Take $a = \binsum{p-1}{p}{x}{y}{R}$.
Take $b = x^{p,R}$. 

$a \equiv_{S} b $ mod $p^{[R]}$ within $R$.

Take $c = y^{p,R}$.

$a +^{R} y^{p,R} \equiv_{S} b +^{R} y^{p,R} $ mod $p^{[R]}$ within $R$.
Qed.
[timelimit 3]
\end{proof}

\begin{lemma} Let $p$ be a prime number.
Let $S$ be a subring of $R$ and $x,y \in S$. 
Then $(x -^{R} y)^{p,R} \equiv_{S} x^{p,R} -^{R} y^{p,R}$ mod $p^{[R]}$ within $R$.
\end{lemma}
\begin{proof}

Case $p$ is even. Then $p$ is equal to $2$.

$(x -^{R} y)^{2,R} \equiv_{S} x^{2,R} +^{R} (-^{R}y)^{2,R}$ mod $2^{[R]}$ within $R$.

$(x -^{R} y)^{2,R} \equiv_{S} x^{2,R} +^{R} y^{2,R}$ mod $2^{[R]}$ within $R$.

$x^{2,R} +^{R} y^{2,R} \equiv_{S} x^{2,R} -^{R} y^{2,R}$ mod $2^{[R]}$ within $R$.
Indeed $y^{2,R} \equiv_{S} -^{R}(y^{2,R})$ mod $2^{[R]}$ within $R$.

$(x -^{R} y)^{2,R}$, $x^{2,R} +^{R} y^{2,R}$, 
$x^{2,R} -^{R} y^{2,R} \in S$.

Hence $(x -^{R} y)^{2,R} \equiv_{S} x^{2,R} -^{R} y^{2,R}$ mod $2^{[R]}$ within $R$.

End. 

$p$ is odd.

$(x -^{R} y)^{p,R} \equiv_{S} x^{p,R} +^{R} (-^{R}y)^{p,R}$ mod $p^{[R]}$ within $R$.

$(x -^{R} y)^{p,R} \equiv_{S} x^{p,R} -^{R} y^{p,R}$ mod $p^{[R]}$ within $R$.
\end{proof}
%[prove off][check off]
\end{forthel}

\section{Topological Spaces}

We formalize some beginnings of set-theoretic topology. For
simplicity, we do not work with a set-theoretic family of open sets, 
but instead use a predicate ``is open in $T$''. We partially
follow the presentation by Wedhorn 
\cite{Wedhhorn2019}, 5.4.

\subsection{Topological Axioms}

\begin{forthel}

\begin{signature}
A topological space is a set.
\end{signature}

Let $T$ denote a topological space.

\begin{signature} Assume that $T$ is a topological space and $X$ is a subset of $T$.
$X$ is open in $T$ is an atom.
\end{signature}

\begin{definition}
An open subset of $T$ is a subset of $T$ that is open in $T$.
\end{definition}

\begin{axiom} $T$ is open in $T$. \end{axiom}

\begin{axiom} Let $X$ be an open subsets of $T$ and $Y$ be an open subset of $T$.
Then $X \cap Y$ is open in $T$.
\end{axiom}

\begin{axiom} Let $Z$ be a set such that every element of $Z$
is an open subset of $T$. Then $\bigcup Z$ is open in $T$.
\end{axiom}

\begin{lemma} $\emptyset$ is open in $T$. \end{lemma}
\begin{proof} $\emptyset = \bigcup \emptyset$. \end{proof}

\end{forthel}

\subsection{Convergence}

\begin{forthel}

[synonym neighborhood/-s]

\begin{definition} Let $x \in T$. A neighborhood of $x$ in $T$
is a subset $V$ of $T$ such that there is a subset $U$ of $V$  
such that $U$ is open in $T$ and $x \in U$.
\end{definition}

\begin{lemma} Let $x \in T$. Let $V,V'$ be neighborhoods of $x$ in $T$.
Then $V \cap V'$ is a neighborhood of $x$ in $T$.
\end{lemma}
\begin{proof} Take a subset $U$ of $V$ such that $U$ is open
in $T$ and $x \in U$.
Take a subset $U'$ of $V'$ such that $U'$ is open
in $T$ and $x \in U'$.
Then $U \cap U'$ is open in $T$.
$x \in U \cap U' \subseteq V \cap V'$.
\end{proof}

\begin{lemma}
Let $x$ be an element of $T$. Let $U,V$ be subsets of $T$
such that $U$ is a neighborhood of
$x$ in $T$ and $U \subseteq V$.
Then $V$ is a neighborhood of $x$ in $T$.
\end{lemma}

\end{forthel}
A topology is determined by its neighborhoods.
\begin{forthel}

\begin{lemma} Let $A \subseteq T$. Then $A$ is open in $T$ iff
for every $x \in A$ there exists a neighborhood $V$ of
$x$ in $T$ such that $V \subseteq A$.
\end{lemma}

\begin{proof}
Case $A$ is open in $T$. Obvious.

Case For every $x \in A$ there exists a neighborhood $V$ of
$x$ in $T$ such that $V \subseteq A$.
For every $x \in A$ there exists a subset $U$ of $A$ such
that $U$ is open in $T$ and $x \in U$.
Define $Z = \{U \mid U \subseteq A$ and $U$ is open in $T\}$.
$Z$ is a set.
Every element of $Z$ is a subset
of $T$ that is open in $T$.
$\bigcup Z \subseteq A$.

$A = \bigcup Z$. 

Proof. Let $x$ be an element of $A$. Take
an element $U$ of $Z$ such that
$x \in U$. $U \subseteq \bigcup Z$.
qed.

Hence $A$ is open in $T$. Indeed every element of $Z$ is an open subset of $T$.
qed.
\end{proof}

\begin{definition} $T$ is Hausdorff iff for all distinct elements
$x,y$ of $T$ 
there are sets $A$, $B$ such that $A$ is a neighborhood of $x$ in $T$ 
and $B$ is a neighborhood of $y$ in $T$ and
$A \cap B = \emptyset$.
\end{definition}

\begin{definition}
A filter basis on $T$ is a nonempty set $B$ such that
every element of $B$ is a nonempty subset of $T$ and
for all elements $X,Y$ of $B$ there exists an element $Z$ of $B$
such that  $Z \subseteq X \cap Y$.
\end{definition}

\begin{definition}
Let $B$ be a filter basis on $T$ and $x \in T$.
$B$ converges to $x$ in $T$ iff for every 
neighborhood $U$ of $x$ in $T$ 
there exists an element $X$ of $B$ such that $X \subseteq U$.
\end{definition}

\begin{definition}
Let $B$ be a filter basis on $T$.
$B$ converges in $T$ iff 
$B$ converges to some element of $T$ in $T$.
\end{definition}

\begin{definition} Let $x$ be an element of $T$.
The neighborhood filter of $x$ in $T$ is the collection of neighborhoods of $x$ in $T$.
\end{definition}

Let $\neigh{x}{T}$ denote the neighborhood filter of $x$ in $T$.

\begin{lemma}
$\neigh{x}{T}$ is a set for all $x \in T$.
\end{lemma}

\begin{lemma}
Let $x$ be an element of $T$. $\neigh{x}{T}$ is a filter basis on $T$ that converges
to $x$ in $T$.
\end{lemma}

\end{forthel}

\section{Topological Groups}

A topological group requires its operations to be topologically continuous. 
%
\begin{forthel}
Let $G$ denote a group that is a topological space.

\begin{definition}
Let $x,x' \in G$. $G$ is additively continuous at $x$ and $x'$ iff
for every neighborhood $U$ of $x +^{G} x'$ in $G$
there are subsets $V,V'$ of $G$ such that 
$V$ is a neighborhood of $x$ in $G$ and $V'$ is a neighborhood
of $x'$ in $G$ and for all $v \in V$ for all $v' \in V'$ 
$v +^{G} v' \in U$.
\end{definition}

\begin{definition}
$G$ is additively continuous iff $G$ is additively continuous at
$x$ and $x'$ for all $x,x' \in G$.
\end{definition}

\begin{definition}
Let $x \in G$. $G$ is negation continuous at $x$ iff
for every neighborhood $U$ of $-^{G} x$ in $G$
there exists a neighborhood $V$ of $x$ in $G$ such that 
$-^{G} v \in U$ for all $v \in V$.
\end{definition}

\begin{definition}
$G$ is negation continuous iff $G$ is negation continuous at $x$
for every $x \in G$.
\end{definition}

\begin{definition}[title = Topological group]
A topological group is a group $G$ such that $G$ is a topological space
and $G$ is additively continuous and negation continuous.
\end{definition}

Let $G$ denote a topological group.

\begin{definition}
A fundamental system of neighborhoods of $G$ is a subset $F$ of $\neigh{0^{G}}{G}$ such
that for every neighborhood $U$ of $0^{G}$ in $G$ there exists $V \in F$ such that
$V \subseteq U$.
\end{definition}

\begin{definition}
A fundamental system of open neighborhoods of $G$ is 
a fundamental system of neighborhoods $F$ of $G$ such
that every element of $F$ is open in $G$.
\end{definition}


\begin{proposition}
$\neigh{0^{G}}{G}$ is a fundamental system of neighborhoods of $G$.
\end{proposition}

\begin{definition}
$G$ is nonarchimedean iff every neighborhood $U$ of $0^{G}$ in $G$
has a subset $S$ that is a subgroup of $G$ and open in $G$.
\end{definition}

\begin{lemma}
Let $B$ be a subgroup of $G$ that is a neighborhood of $0^{G}$ in $G$.
Then $B$ is open in $G$.
\end{lemma}
\begin{proof} 
Let $x \in B$. 
$G$ is additively continuous at $x$ and $-^{G} x$.
$B$ is a neighborhood of 
$x +^{G} (-^{G} x)$ in $G$.
[timelimit 10]
Take subsets $V,V'$ of $G$ such that 
$V$ is a neighborhood of $x$ in $G$ and $V'$ is a neighborhood
of $-^{G} x$ in $G$ and for all $v \in V$, all $v' \in V'$ 
$v +^{G} v' \in B$.
[timelimit 3]

(1) $V \subseteq B$.
Proof. Let $v \in V$.
Then $v +^{G} (-^{G} x) \in B$.
$v = (v +^{G} (-^{G} x)) +^{G} x \in B$.
end.
\end{proof}


\end{forthel}

\subsection{Completeness}

\begin{forthel}
Let $G$ denote a topological group.

\begin{definition}
Let $B$ be a filter basis on $G$. $B$ is Cauchy in $G$ iff
for every neighborhood $U$ of $0^{G}$ in $G$ there exists
$A \in B$ such that $x -^{G} y \in U$ 
for all $x,y \in A$.
\end{definition}

\begin{definition}
$G$ is complete iff $G$ is Hausdorff and every filter basis on $G$ 
that is Cauchy in $G$ converges in $G$.
\end{definition}

\end{forthel}

\section{Topological Rings}

A topological ring has continuous multiplication.

\begin{forthel}
Let $R$ denote a ring that is a topological space.
\end{forthel}
%
We hide \verb|^{R}| and \verb|,R}| as before.
\catcode`^\active
\def^#1{\ifthenelse{\equal{#1}{R}} {\unskip} 
       {{\IfEndWith{#1}{,R}{\sp{\StrBefore{#1}{,R}}}{\sp{#1}}}}}

\begin{forthel}

\begin{definition}
Let $x,x' \in R$. $R$ is multiplicatively continuous at $x$ and $x'$ iff
for every neighborhood $U$ of $x \cdot^{R} x'$ in $R$
there are subsets $V,V'$ of $R$ such that 
$V$ is a neighborhood of $x$ in $R$ and $V'$ is a neighborhood
of $x'$ in $R$ and for all $v \in V$ for all $v' \in V'$ 
$v \cdot^{R} v' \in U$.
\end{definition}

\begin{definition}
$R$ is multiplicatively continuous iff $R$ is multiplicatively continuous at
$x$ and $x'$ for all $x,x' \in R$.
\end{definition}

\begin{definition}[title = Topological ring]
A topological ring is a ring $R$ such that $R$ is a topological group
and $R$ is multiplicatively continuous.
\end{definition}

\end{forthel}

\part{Perfectoid Rings}

\section{Boundedness}

This chapter corrresponds to parts of the file \verb+power_bounded.lean+
in the Lean formalization of perfectoid spaces \cite{Lean} which contains
the theory of topologically nilpotent, bounded, and power-bounded
elements and subsets of topological rings. The line numbers "L ..." after
some definitions and theorems refer to that file.
We also use material from \cite{Wedhhorn2019}.
We omit those parts that depend on an adic topology on $R$, since this
will not be needed in our definition of perfectoid rings.
Initially we do not require a connection between the
algebraic and the topological structure on $R$.

\begin{forthel}
Let $R$ denote a ring that is a topological space.

%42
\begin{definition}[title = L 42] Assume that $B$ is a subset of $R$.
$B$ is bounded in $R$ iff for all neighborhoods $U$ of $0^{R}$
in $R$ there exists a neighborhood $V$ of $0^{R}$ in $R$ such that
$v \cdot^{R} b \in U$ where $v \in V$ and $b \in B$.
\end{definition}

\end{forthel}
The last condition can be expressed by products of sets:
\begin{forthel}
%48
\begin{lemma}[title = L 48] Assume that $B$ is a subset of $R$.
$B$ is bounded in $R$ if and only if for all neighborhoods $U$ of $0^{R}$
in $R$ there exists a neighborhood $V$ of $0^{R}$ in $R$ such that
$V \star^{R} B \subseteq U$.
\end{lemma}
\begin{proof}
CASE $B$ is bounded in $R$.
Let $U$ be a neighborhood of $0^{R}$ in $R$.
Take a neighborhood $V$ of $0^{R}$ in $R$ such that
$v \cdot^{R} b \in U$ where $v \in V$ and $b \in B$.
Then $V \star^{R} B \subseteq U$.
end.

CASE For every neighborhood $U$ of $0^{R}$
in $R$ there exists a neighborhood $V$ of $0^{R}$ in $R$ such that
$V \star^{R} B \subseteq U$.

Let us show that $B$ is bounded in $R$. 
Let $U$ be a neighborhood of $0^{R}$ in $R$.
Take a neighborhood $V$ of $0^{R}$ in $R$ such that
$V \star^{R} B \subseteq U$.
Then $v \cdot^{R} b \in U$ where $v \in V$ and $b \in B$.
end.
end.
\end{proof}

\end{forthel}
Now we strengthen our assumptions on $R$ to:
\begin{forthel}
%58
Let $R$ denote a topological ring.

%91
\begin{lemma}[title = L 91] Let $R$ be nonarchimedean and $B \subseteq R$.
Then $B$ is bounded in $R$ iff for all neighborhoods
$U$ of $0^{R}$ in $R$ there exists a subgroup $V$ of $R$ such
that $V$ is open in $R$ and the additive closure of
$V \star^{R} B$ in $R$ is a subset of $U$.
\end{lemma}

\begin{proof}
CASE $B$ is bounded in $R$.
Let $U$ be a neighborhood of $0^{R}$ in $R$.
Take a subset $U'$ of $U$ such that $U'$ is a subgroup of $R$ and open in $R$.
$U'$ is a neighborhood of $0^{R}$ in $R$.
Take a neighborhood $V'$ of $0^{R}$ in $R$ such that $V' \star^{R} B \subseteq U'$.
Take a subset $V$ of $V'$ such that $V$ is a subgroup of $R$ and open in $R$.
Then $V \star^{R} B \subseteq U'$.
The additive closure of $V \star^{R} B$ in $R$ is a subset of $U'$.
end.

CASE For all neighborhoods
$U$ of $0^{R}$ in $R$ there exists a subgroup $V$ of $R$ such
that $V$ is open in $R$ and the additive closure of
$V \star^{R} B$ in $R$ is a subset of $U$.
Let $U$ be a neighborhood of $0^{R}$ in $R$.
Take a subgroup $V$ of $R$ such
that $V$ is open in $R$ and the additive closure of
$V \star^{R} B$ in $R$ is a subset of $U$.
$V$ is a neighborhood of $0^{R}$ in $R$.
$V \star^{R} B \subseteq R$.
$V \star^{R} B$ is a subset of the additive closure of
$V \star^{R} B$ in $R$.
Then $V \star^{R} B \subseteq U$.
end.
\end{proof}

\begin{definition} A bounded subset of $R$ is a subset of $R$ that is bounded in $R$.
\end{definition}

\begin{lemma}
$\emptyset$ is a bounded subset of $R$.
\end{lemma}

\begin{lemma}[title = Boundedness of Singletons]
Let $r \in R$. Then $\sing{r}$ is bounded in $R$.
\end{lemma}
\begin{proof}
Let $U$ be a neighborhood of $0^{R} \cdot^{R} r$
in $R$.
$R$ is multiplicatively continuous at $0^{R}$ and $r$.
[timelimit 10]
Take $V,V' \subseteq R$ such that 
$V$ is a neighborhood of $0^{R}$ in $R$ and $V'$ is a neighborhood
of $r$ in $R$ and for all $v \in V$ for all $v' \in V'$ 
$v \cdot^{R} v' \in U$.
[timelimit 3]
Then $V \star^{R} \sing{r} \subseteq U$.
\end{proof}

\begin{lemma}
Let $B,B'$ be subsets of $R$ that are bounded in $R$.
Then $B \cup B'$ is bounded in $R$.
\end{lemma}
\begin{proof}
Let $U$ be a neighborhood of $0^{R}$ in $R$.
Take a neighborhood $V$ of $0^{R}$ in $R$ such that
$v \cdot^{R} b \in U$ where $v \in V$ and $b \in B$.
Take a neighborhood $V'$ of $0^{R}$ in $R$ such that
$v \cdot^{R} b \in U$ where $v \in V'$ and $b \in B'$.
$V \cap V'$ is a neighborhood of $0^{R}$ in $R$.
$v \cdot^{R} b \in U$ where $v \in V \cap V'$ and 
$b \in B \cup B'$.
\end{proof}

\begin{lemma} For every natural number $n$
for every map $s$ such that 
$\Seq{1}{n} \subseteq \dom(s)$ and $\FinSet{s}{1}{n} \subseteq R$
$\FinSet{s}{1}{n}$ is bounded in $R$.
\end{lemma}
\begin{proof}[method = induction]
Let $n$ be a natural number.
Let $s$ be a map such that 
$\Seq{1}{n} \subseteq \dom(s)$ and $\FinSet{s}{1}{n} \subseteq R$.

CASE $n = 0$. trivial.

Take a natural number $m$ such that $n = m +1$. $m \prec n$.
$\Seq{1}{m} \subseteq \dom(s)$. $\FinSet{s}{1}{m} \subseteq R$.
$\FinSet{s}{1}{m}$ is bounded in $R$.

$\FinSet{s}{1}{m+1} = \FinSet{s}{1}{m} \cup \sing{s_{m+1}}$.
$\sing{s_{m+1}}$ is a bounded subset of $R$.
Hence $\FinSet{s}{1}{m+1}$ is bounded in $R$.
\end{proof}

\begin{lemma}[title = 5 28 1]
Every finite subset of $R$ is bounded in $R$.
\end{lemma}

\begin{lemma}
Let $r,s \in R$. Then $\unp{r}{s}$ is bounded in $R$.
\end{lemma}
\begin{proof}
$\unp{r}{s}$ is a finite subset of $R$.
\end{proof}

%136
\begin{lemma}[title = L 136]
Every subset of every bounded subset of $R$ is a bounded subset of $R$.
\end{lemma}
\begin{proof}
Let $B$ be a bounded subset of $R$.
Let $A \subseteq B$.
Let $U$ be a neighborhood of $0^{R}$ in $R$.
Take a neighborhood $V$ of $0^{R}$ in $R$ such that $V \star^{R} B \subseteq U$.
Then $V \star^{R} A \subseteq V \star^{R} B \subseteq U$.
$V \star^{R} A \subseteq U$.
\end{proof}
\end{forthel}

The corresponding lemma and its proof in \verb+power_bounded.lean+ read:
\begin{verbatim}
lemma subset {S1 S2 : set R} 
(h : S1 \subset S2) (H : is_bounded S2) : 
is_bounded S1 :=
begin
  intros U hU,
  rcases H U hU with <V, hV1, hV2>,
  use [V, hV1],
  intros v hv b hb,
  exact hV2 _ hv _ (h hb),
end
\end{verbatim} 

\begin{forthel}

% Following was proved in Lean by induction on the build-up of the
% additive closure, which is not available to us.

%\begin{lemma}
%Assume that $R$ be nonarchimedean and $T$ is a bounded subset of $R$.
%Then the additive closure of $T$ in $R$ is bounded in $R$.
%\end{lemma}

%179
\begin{definition}[title = L 179]
Let $r$ be an element of $R$. $r$ is powerbounded in $R$ iff $\powers{r}{R}$ is 
bounded in $R$.
\end{definition}

\begin{lemma}
Let $r$ be an element of $R$. Then $r$ is powerbounded in $R$
iff for all neighborhoods $U$ of $0^{R}$ in $R$ there exists
a neighborhood $V$ of $0^{R}$ in $R$ such that
$v \cdot^{R} r^{n,R} \in U$ where $v \in V$ and $n$ is
a natural number.
\end{lemma}

\begin{proof}
CASE $r$ is powerbounded in $R$.
Let $U$ be a neighborhood of $0^{R}$ in $R$.
Take a neighborhood $V$ of $0^{R}$ in $R$ such that
$V \star^{R} \powers{r}{R} \subseteq U$.
Indeed $\powers{r}{R}$ is bounded in $R$.
Assume $w \in V$. Let $m$ be a natural number.
Then $r^{m,R} \in \powers{r}{R}$.
$w \cdot^{R} r^{m,R} \in V \star^{R} \powers{r}{R} \subseteq U$.

end.

CASE For all neighborhoods $U$ of $0^{R}$ in $R$ there exists
a neighborhood $V$ of $0^{R}$ in $R$ such that
$v \cdot^{R} r^{n,R} \in U$ where $v \in V$ and $n$ is
a natural number.

(1) $\powers{r}{R}$ is bounded in $R$.

Proof.
Let $U$ be a neighborhood of $0^{R}$ in $R$.
Take a neighborhood $V$ of $0^{R}$ in $R$ such that
$v \cdot^{R} r^{n,R} \in U$ where $v \in V$ and $n$ is a natural number.
qed.
end.
\end{proof}

%189
\begin{lemma}[title = L 189]
$0^{R}$ is powerbounded in $R$.
\end{lemma}
%199
\begin{lemma}[title = L 199]
$1^{R}$ is powerbounded in $R$.
\end{lemma}

\begin{lemma}
$-^{R} 1^{R}$ is powerbounded in $R$.
\end{lemma}
\begin{proof} 
$\unp{-^{R} 1^{R}}{1^{R}}$ is bounded in $R$.
$\powers{-^{R} 1^{R}}{R} \subseteq \unp{-^{R} 1^{R}}{1^{R}}$.
[timelimit 20]
Hence $\powers{-^{R} 1^{R}}{R}$ is bounded in $R$.
[timelimit 3]
\end{proof}

%248
\begin{lemma}[title = L 248]
Let $a,b$ be elements of $R$ that are powerbounded in $R$.
Then $a\cdot^{R} b$ is powerbounded in $R$.
\end{lemma}
\begin{proof}
Let $U$ be a neighborhood of $0^{R}$ in $R$.
[timelimit 10]
Take a neighborhood $V$ of $0^{R}$ in $R$ such that
$v \cdot^{R} b^{n,R} \in U$ where $v \in V$ and $n$ is a natural number.
[timelimit 15]
Take a neighborhood $W$ of $0^{R}$ in $R$ such that
$w \cdot^{R} a^{n,R} \in V$ where $w \in W$ and $n$ is a natural number.
[timelimit 3]

(1) $w \cdot^{R} (a \cdot^{R} b)^{n,R} \in U$ where 
$w \in W$ and $n$ is a natural number.

Proof. Let $w \in W$ and $n$ be a natural number.
$w \cdot^{R} a^{n,R} \in V$.

$w \cdot^{R} (a \cdot^{R} b)^{n,R} =
%
w  \cdot^{R} (a^{n,R} \cdot^{R} b^{n,R})= 
%
(w \cdot^{R} a^{n,R}) \cdot^{R} b^{n,R} \in U$. 

qed. 
\end{proof}

\begin{lemma}[title = L 290]
Let $R$ be nonarchimedean.
Let $a,b$ be elements of $R$ that are powerbounded in $R$.
Then $a +^{R} b$ is powerbounded in $R$.
\end{lemma}
\begin{proof}
Let $U$ be a neighborhood of $0^{R}$ in $R$.
Take a subset $U'$ of $U$ that is a subgroup of $R$ and open in $R$.
$U'$ is a neighborhood of $0^{R}$ in $R$.
[timelimit 30]
Take a neighborhood $V$ of $0^{R}$ in $R$ such that
$v \cdot^{R} b^{n,R} \in U'$ where $v \in V$ and $n$ is a natural number.
[timelimit 30]
Take a neighborhood $W$ of $0^{R}$ in $R$ such that
$w \cdot^{R} a^{m,R} \in V$ where $w \in W$ and $m$ is a natural number.
[timelimit 3]
(1) $w \cdot^{R} (a^{m,R} \cdot^{R} b^{n,R}) \in U'$ where 
$w \in W$ and $m,n$ are natural numbers.

Proof. Let $w \in W$ and $m,n$ be natural numbers.

$w \cdot^{R} a^{m,R} \in V$ and
$w \cdot^{R} (a^{m,R} \cdot^{R} b^{n,R}) =
(w \cdot^{R} a^{m,R}) \cdot^{R} b^{n,R} \in U$.
 
qed.

Define $G = \{x \in R \mid w \cdot^{R} x \in U'$ for all $w \in W\}$.

(3) $G$ is a subgroup of $R$. 

Proof.
 
(a) $0^{R} \in G$.

(b) For all $x,y \in G$ $x +^{R} y \in G$.

Proof. Let $x,y \in G$.
Let $w \in W$. Then
$w \cdot^{R} (x +^{R} y)
= (w \cdot^{R} x) +^{R} (w \cdot^{R} y) \in U'$.

qed.

(c) For all $x \in G$ $-^{R} x \in G$.

Proof. Let $x \in G$. Let $w \in W$.

$w \cdot^{R} (-^{R} x) = 
w \cdot^{R} ((-^{R} 1^{R}) \cdot^{R} x) =
(-^{R} 1^{R}) \cdot^{R} (w \cdot^{R} x) =
-^{R} (w \cdot^{R} x)
$.

qed.

qed.

Define $Z = \{a^{m,R} \cdot^{R} b^{n,R} \mid m,n \in \mathbb{N}\}$.

(2) $Z$ is a subset of $G$ that is closed under multiplication in $R$.

Proof.
Let $u,v \in Z$.
Take $m,n \in \mathbb{N}$ such that
$u = a^{m,R} \cdot^{R} b^{n,R}$.
Take $m',n' \in \mathbb{N}$ such that
$v = a^{m',R} \cdot^{R} b^{n',R}$.
Then
$u \cdot^{R} v = 
(a^{m,R} \cdot^{R} b^{n,R}) \cdot^{R} (a^{m',R} \cdot^{R} b^{n',R}) =$

$(a^{m,R} \cdot^{R} a^{m',R}) \cdot^{R} (b^{n,R} \cdot^{R} b^{n',R}) =$

$a^{m+m',R} \cdot^{R} b^{n+n',R} \in Z$.

Hence $u \cdot^{R} v \in Z$.
Qed.

(3) $1^{R},a,b \in Z$.
Proof.
$1^{R} = a^{0,R} \cdot^{R} b^{0,R}$.
$a = a^{1,R} \cdot^{R} b^{0,R}$.
$b = a^{0,R} \cdot^{R} b^{1,R}$.
Qed.

(4) $Z \subseteq G$.

(5) $[Z]^{R} \subseteq G$.

(6) $[Z]^{R}$ is closed under multiplication in $R$.

(7) For all natural numbers $n$ $(a +^{R} b)^{n,R} \in [Z]^{R}$.
Proof by induction. Let $n$ be a natural number.
Case $n = 0$. Trivial.

$a +^{R} b \in [Z]^{R}$.

Let $m = n - 1$. Then $(a +^{R} b)^{m,R} \in [Z]^{R}$.

$(a +^{R} b)^{n,R} = (a +^{R} b)^{m+1,R}
= (a +^{R} b)^{m,R} \cdot^{R} (a +^{R} b) \in [Z]^{R}$.
Qed.

(8) For all natural numbers $n$ $(a +^{R} b)^{n,R} \in G$.

(9) For all natural numbers $n$ and all $w \in W$ 
$w \cdot^{R} (a +^{R} b)^{n,R} \in U'$ .

\end{proof}

%342
\begin{lemma}[title = L 342]
Let $a$ be an element of $R$ that is powerbounded in $R$.
Then $-^{R} a$ is powerbounded in $R$.
\end{lemma}
\begin{proof}
$-^{R} a = (-^{R} 1^{R}) \cdot^{R} a$.
\end{proof}

%310
\begin{definition}[title = L 310]  $R^o = \{x \in R \mid  x$ is powerbounded in $R\}$. 
\end{definition}

\begin{lemma} $R^o$ is a subset of $R$. \end{lemma}

%320
\begin{lemma}[title = L 320]
$0^{R} \in R^o$.
\end{lemma}

%322
\begin{lemma}[title = L 322] 
$1^{R} \in R^o$.
\end{lemma}

%324
\begin{lemma}[title = L 324]
Let $a,b \in R^o$. Then $a \cdot^{R} b \in R^o$.
\end{lemma}

\begin{lemma}[title = L 324]
Let $a \in R^o$. Then $a^{n,R} \in R^o$ for all $n \in \mathbb{N}$.
\end{lemma}

\begin{proof}[method = induction]
Let $n \in \mathbb{N}$.

Case $n = 0$. Trivial.

Take $m = n - 1$. $m$ is inductively smaller than $n$ and $a^{m,R} \in R^o$.
Hence $a^{n,R} = a^{m,R} \cdot^{R} a \in R^o$.
\end{proof}

%337
\begin{lemma}[title = L 337] Let $R$ be nonarchimedean.
Let $a,b \in R^o$. Then $a +^{R} b \in R^o$.
\end{lemma}

%342
\begin{lemma}[title = L 342]
Let $a \in R^o$. Then $-^{R} a \in R^o$.
\end{lemma}

%365
\begin{lemma}[title = L 365] Let $R$ be nonarchimedean. Then
$R^o$ is a subgroup of $R$.
\end{lemma}

%371
\begin{lemma}[title = L 371] Let $R$ be nonarchimedean. Then
$R^o$ is a subring of $R$.
\end{lemma}


%380
\begin{definition}[title =  380]
$R$ is uniform iff $R^o$ is a bounded subset of $R$.
\end{definition}

\end{forthel}
\subsection{Topological Nilpotency}
\begin{forthel}


%30
\begin{definition}[title = L 30] Let $r$ be an element of $R$. $r$ is topologically
nilpotent in $R$ iff for all neighborhoods $U$ of $0^{R}$ in $R$
there exists a natural number $N$ such that $r^{n,R} \in U$ 
for all natural numbers $n$
such that $n > N$.
\end{definition}

\end{forthel}
For example:
\begin{forthel}

\begin{lemma}
$0^{R}$ is topologically nilpotent in $R$.
\end{lemma}

[timelimit 20]
\begin{lemma} Assume that $R$ is Hausdorff.
$1^{R}$ is not topologically nilpotent in $R$.
\end{lemma}
\begin{proof}
Take sets $A,B$ such that $A$ is a neighborhood of $0^{R}$ in $R$
and $B$ is a neighborhood of $1^{R}$ in $R$ and
$A \cap B = \emptyset$.
\end{proof}
[timelimit 3]

%%[prove on][check on]

\begin{theorem}
Let $r$ be an element of $R$ that is topologically nilpotent in $R$.
Then $r$ is powerbounded in $R$.
\end{theorem}
\begin{proof}

Let $U$ be a neighborhood of $0^{R}$ in $R$.

Take $h = 0^{R}$. Take $h' = 0^{R}$.
$U$ is a neighborhood of $0^{R} \cdot^{R} 0^{R}$ in $R$.
$R$ is multiplicatively continuous at $h$ and $h'$ and
$U$ is a neighborhood of $h \cdot^{R} h'$ in $R$.
[prover vampire]
[timelimit 30]
Take subsets $V,V'$ of $R$ such that
$V$ is a neighborhood of $0^{R}$ in $R$ and
$V'$ is a neighborhood of $0^{R}$ in $R$ and
for all $v \in V$ for all $v' \in V'$ 
$v \cdot^{R} v' \in U$.  
[timelimit 3]
[prover eprover]

Take a natural number $N$ such that $r^{n,R} \in V$ 
for all natural numbers $n$ such that $n > N$.
Define $X = \{r^{n,R} \mid n \in \mathbb{N}$ and $1 \leq n \leq N\}$.

(1) $X$ is finite.

Proof. Define $s(n)= r^{n,R}$ for $n \in \Seq{1}{N}$.
[timelimit 5]
$X = \FinSet{s}{1}{N}$.
[timelimit 3]
qed.
[timelimit 10]
$X$ is bounded in $R$.
Take a neighborhood $V''$ of $0^{R}$ in $R$ such that
$v'' \cdot^{R} x \in U$ where $v'' \in V''$ and $x \in X$.
Let $W = U \cap (V' \cap V'')$. 
[timelimit 5]$W$ is a neighborhood
of $0^{R}$ in $R$. Indeed $V' \cap V''$ is a neighborhood
of $0^{R}$ in $R$.
[timelimit 3]

(2) $w \cdot^{R} x \in U$ where $w \in W$ and $x \in \powers{r}{R}$.
    
\end{proof}

\begin{lemma}
Let $a,b$ be elements of $R$ such that $a$ is topologically nilpotent in $R$ and
$b$ is powerbounded in $R$.
Then $a\cdot^{R} b$ is topologically nilpotent in $R$.
\end{lemma}
\begin{proof}
Let $U$ be a neighborhood of $0^{R}$ in $R$.
[timelimit 10]
Take a neighborhood $V$ of $0^{R}$ in $R$ such that
$v \cdot^{R} b^{n,R} \in U$ where $v \in V$ and $n$ is a natural number.
[timelimit 15]
Take a natural number $N$ such that
$a^{n,R} \in V$ for all natural numbers $n$ such that $n > N$.
[timelimit 3]

(1) $(a \cdot^{R} b)^{n,R} \in U$ for all natural numbers $n$ such that $n > N$.

Proof. Let $n$ be a natural number such that $n > N$.

$(a \cdot^{R} b)^{n,R} = a^{n,R} \cdot^{R} b^{n,R} \in U$. 

qed. 
\end{proof}

\begin{lemma} Let $R$ be Hausdorff.
Let $a$ be an element of $R$ that is topologically nilpotent in $R$.
Then $a$ does not divide $1^{R}$ in $R^o$ within $R$.
\end{lemma}
\begin{proof}
Assume the contrary. 
Take $b \in R^o$ such that $a \cdot^{R} b = 1^{R}$.
Then  $a \cdot^{R} b$ is topologically nilpotent in $R$.
Contradiction.
\end{proof}


%[prove off][check off]
\end{forthel}

\section{Huber rings}
\begin{forthel}

Let $R$ denote a topological ring.
\end{forthel}
The following is the original definition of f-adic ring by R. Huber \cite{Huber1993}:
\begin{forthel}

\begin{definition}
A Huber ring is a topological ring $R$ such that for some subset $U$ of $R$
and some finite subset $T$ of $U$ 
$\Powers{U}{R}$ is a fundamental system of neighborhoods of $R$ and
$T \ast^{R} U = U \ast^{R} U \subseteq U$. 
\end{definition}

\begin{lemma}
Let $R$ be a Huber ring. Then $R$ is nonarchimedean.
\end{lemma}
\begin{proof}
Take a subset $U$ of $R$
such that $U \ast^{R} U \subseteq U$ and 
$\Powers{U}{R}$ is a fundamental system of neighborhoods of $R$.
Let $V$ be a neighborhood of $0^{R}$ in $R$.
Take $B \in \Powers{U}{R}$ such that $B \subseteq V$.
$B$ is a neighborhood of $0^{R}$ in $R$.
Take $n \in \mathbb{N}$ such that $n \geq 1$ and $B = U^{n,R,subset}$.

Case $n > 1$.
Then $B$ is a subgroup of $R$ that is open in $R$.
end.

$n = 1$. $B = U$. $U \ast^{R} U $ is a subgroup
of $R$ that is open in $R$.
$U \ast^{R} U \subseteq U = B \subseteq V$.
\end{proof}

\begin{lemma}
Let $R$ be a Huber ring. Then $R^o$ is a subring of $R$.
\end{lemma}
\end{forthel}

\section{Tate Rings}

\begin{forthel}

Let $R$ denote a topological ring.

%23
\begin{definition}
A pseudouniformizer of $R$ is
a unit in $R$ that is topologically nilpotent in $R$.
\end{definition}

%45

[timelimit 10]
\begin{lemma}
Let $\varpi$ be a pseudouniformizer of $R$.
Then $\varpi$ is powerbounded in $R$.
\end{lemma}
[timelimit 3]
%76
\begin{definition}
A Tate ring is a Huber ring that has a pseudouniformizer.
\end{definition} 

\end{forthel} 
\section{Frobenius maps}
Normally we would use (set-theoretic) quotients by principal ideals in $R^o$
for the definition of perfectoid rings, and then Frobenius-style maps on
those quotient rings.
Instead we define a global map $\Phi^R$, $x \mapsto x^{p,R}$ and consider
whether it induces maps between such quotients. This
can be expressed by calculations modulo the generators of the ideals. 
A complete formalization would have to prove certain well-definednesses
which we assume for simplicity.
We treat the Frobenius Map as a function symbol of the logic, dependent
on the Ring and a fixed prime number $p$.

\begin{forthel}

Let $R$ denote a Huber ring.

%24
\begin{signature} $p$ is a prime number. 
\end{signature}

\begin{definition}
Let $x \in R$. $\Phi^{R} (x) = x^{p,R}$. 
\end{definition}

\begin{lemma}
Let $x \in R^o$. Then $\Phi^{R}(x) \in R^o$.
\end{lemma}

\end{forthel}

$\Phi^{R}$ maps the subring of power-bounded elements into itself.
To show further structural properties we recall the following

\begin{lemma*}
Let $S$ be a subring of $R$ and $x,y$ be elements of $S$.
Then 

$(x +^{R} y)^{p,R} \equiv_{S} 
x^{p,R} +^{R} y^{p,R}$ mod $p^{[R]}$ within $R$ 

and

$(x -^{R} y)^{p,R} \equiv_{S} 
x^{p,R} -^{R} y^{p,R}$ mod $p^{[R]}$ within $R$ 
\end{lemma*}

\begin{forthel}

\begin{lemma}Let $\varpi$ be an element of $R^o$ such that
$(\varpi^{p,R}) | p^{[R]}$ in $R^o$ within $R$.
Let $x,y$ be elements of $R^o$ such that

$x \equiv_{R^o} y$ mod $\varpi$ within $R$.

Then 

$\Phi^{R}(x) \equiv_{R^o} \Phi^{R}(y)$ mod $\varpi^{p,R}$ within $R$.
\end{lemma}
\begin{proof}
$\varpi$ divides $x -^{R} y$ in $R^o$ within $R$.
$\varpi^{p,R}$ divides $(x -^{R} y)^{p,R}$ in $R^o$ within $R$.
$p^{[R]}$ divides $(x -^{R} y)^{p,R} -^{R} (x^{p,R} -^{R} y^{p,R})$ in $R^o$ within $R$.
Let $t = (x -^{R} y)^{p,R} -^{R} (x^{p,R} -^{R} y^{p,R})$.
Then $\varpi^{p,R}$ divides $-^{R} t$ in $R^o$ within $R$.
$\varpi^{p,R}$ divides $(x -^{R} y)^{p,R} -^{R} t$ in $R^o$ within $R$.
$\varpi^{p,R}$ divides $x^{p,R} -^{R} y^{p,R}$ in $R^o$ within $R$.
Indeed $(x -^{R} y)^{p,R} -^{R} t = x^{p,R} -^{R} y^{p,R}$.
\end{proof}

\end{forthel}

This map also respects the ring operations modulo 
$\varpi$ and $\varpi^{p,R}$.

\begin{forthel}

\begin{lemma}
$\Phi^{R}(0^{R}) = 0^{R}$.
\end{lemma}

\begin{lemma}
$\Phi^{R}(1^{R}) = 1^{R}$.
\end{lemma}

\begin{lemma} 
Let $\varpi$ be an element of $R^o$ such that
$(\varpi^{p,R}) | p^{[R]}$ in $R^o$ within $R$.
Let $x,y$ be elements of $R^o$.
Then 

$\Phi^{R}(x +^{R} y) \equiv_{R^o} 
\Phi^{R}(x) +^{R} \Phi^{R}(y)$ mod $\varpi^{p,R}$ within $R$.
\end{lemma}
\begin{proof}
$p^{[R]}$ divides $(x +^{R} y)^{p,R} -^{R} 
(x^{p,R} +^{R} y^{p,R})$ in $R^o$ within $R$.
Then
$\varpi^{p,R}$ divides $(x +^{R} y)^{p,R} -^{R} 
(x^{p,R} +^{R} y^{p,R})$ in $R^o$ within $R$.

\end{proof}

%[prove on][check on]

\begin{lemma} 
Let $\varpi$ be an element of $R^o$ such that
$(\varpi^{p,R}) | p^{[R]}$ in $R^o$ within $R$.
Let $x$ be an element of $R^o$.
Then 

$\Phi^{R}(-^{R} x) \equiv_{R^o} 
-^{R} \Phi^{R}(x)$ mod $\varpi^{p,R}$ within $R$.
\end{lemma}
\begin{proof}

$\Phi^{R}(x +^{R} (-^{R}x)) \equiv_{R^o} 
\Phi^{R}(x) +^{R} \Phi^{R}(-^{R}x)$ mod $\varpi^{p,R}$ within $R$.

$\varpi^{p,R}$ divides $0^{R} -^{R} (\Phi^{R}(x) +^{R} \Phi^{R}(-^{R}x))$ in $R^o$ within $R$.
[timelimit 10]
$\varpi^{p,R}$ divides $-^{R} \Phi^{R}(x) -^{R} \Phi^{R}(-^{R}x)$ in $R^o$ within $R$.
[timelimit 3]
\end{proof}

\begin{lemma} Let $\varpi$ be an element of $R^o$ such that
$(\varpi^{p,R}) | p^{[R]}$ in $R^o$ within $R$.
Let $x,y$ be elements of $R^o$.
Then 

$\Phi^{R}(x \cdot^{R} y) \equiv_{R^o} 
\Phi^{R}(x) \cdot^{R} \Phi^{R}(y)$ mod $\varpi^{p,R}$ within $R$.
\end{lemma}
\begin{proof}
$\Phi^{R}(x \cdot^{R} y) =
(x \cdot^{R} y)^{p,R} =
x^{p,R} \cdot^{R} y^{p,R} =
\Phi^{R}(x) \cdot^{R} \Phi^{R}(y)$.
\end{proof}

\end{forthel}
A perfectoid ring requires the Frobenius map to induce an isomorphism
of certain quotients.
So far we have established that it is a homomorphism.
To express the crucial isomorphism  property one would ordinarily 
apply a general predicate for ring congruence to the rings
$R^o/a$ and $R^o/b$. To cut things short, we (miss-)use the notation
$\Phi^{R} : S / a \cong T / b$ with 
its \LaTeX{} source 
\begin{quotation}
\verb|$\Phi^{R} : S / a \cong T / b$|
\end{quotation}
by defining its meaning in terms of congruences using the 
parameters $S,a,T,b$.
\begin{forthel}

\begin{definition} Let $S,T \subseteq R$. Let $a \in S$ and $b \in T$. 
$\Phi^{R} : S / a \cong T / b$
iff
(for every $x,y \in S$ if 
$\Phi^{R}(x) \equiv_{T} \Phi^{R}(y)$ mod $b$ within $R$ 
then $x \equiv_{S} y$ mod $a$ within $R$)
and
(for every $z \in T$ there exists $w \in S$ such that 
$z \equiv_{T} \Phi^{R}(w)$ mod $b$ within $R$).
\end{definition}

\end{forthel}

\section{Perfectoid rings}

Now all ingredients are prepared for defining
perfectoid rings in \Naproche{}:

\begin{forthel}

%[prove off][check off]

Let $R$ denote a Tate ring.

\begin{lemma}
Let $R$ be complete and $\varpi$ be a pseudouniformizer of $R$.
Then $\varpi, \varpi^{p,R}$ do not divide $1^{R}$ in $R^o$ within $R$.
\end{lemma}
\begin{proof}
$\varpi$ does not divide $1^{R}$ in $R^o$ within $R$.
Assume that $\varpi^{p,R}$ divides $1^{R}$ in $R^o$ within $R$.
Take $b \in R^o$ such that $\varpi^{p,R} \cdot^{R} b = 1^{R}$.
Let $q = p -1$. $p = 1 + q$. Then  $\varpi^{p,R} = 
\varpi^{1,R} \cdot^{R} \varpi^{q,R} = \varpi \cdot^{R} \varpi^{q,R}$.
$ \varpi \cdot^{R} (\varpi^{q,R} \cdot^{R} b) =
(\varpi \cdot^{R} \varpi^{q,R}) \cdot^{R} b =
1^{R}$.
[timelimit 6] 
Then $\varpi$ divides $1^{R}$ in $R^o$ within $R$. Indeed 
$\varpi^{q,R} \in R^o$.
[timelimit 3]
\end{proof}
\end{forthel}

In this case the quotients $R^o/\varpi$ and $R^o/\varpi^{p,R}$ are well-defined
rings, and one can finally define:
\begin{forthel}
\begin{definition}
$R$ is perfectoid iff $R$ is complete and uniform and there
exists a pseudouniformizer $\varpi$ of $R$ such that    
$\varpi^{p,R} | p^{[R]}$ in $R^o$ within $R$
and 
$$\Phi^{R} : R^o / \varpi \cong R^o / \varpi^{p,R}.$$
\end{definition}

\end{forthel}

The present formalization has solely been directed towards 
the definition of perfectoid rings in a readable 
and proof-checked mathematical language. 
We do not pursue the theory of perfectoid rings any further
and we do not consider examples. If one wanted to do so one would 
have to refine and considerably expand the previous developments.

\printbibliography

\end{document}

