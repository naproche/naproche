\documentclass[english]{article}
\usepackage{../latex/lib/naproche}
\usepackage{mathtools} % for \bigtimes
\addbibresource{../latex/lib/bibliography}

\title{König's Theorem}
\author{Steffen Frerix and Peter Koepke\\
University of Bonn}
\date{2018 - 2021}

\newcommand{\card}[1]{\operatorname{card}(#1)}
\newcommand{\SumSet}[2]{\bigsqcup_{i \in #2} #1_{i}}
\newcommand{\Sum}[2]{\sum_{i \in #2} #1_{i}}
\newcommand{\ProdSet}[2]{\bigtimes_{i \in #2} #1_{i}}
\newcommand{\Prod}[2]{\prod_{i \in #2} #1_{i}}


\begin{document}

\maketitle

König's Theorem is an important set-theoretical result about the arithmetic of
cardinals.
It was proved by Julius König in \cite{Koenig1905}.
The proof is reminiscent of Cantor's diagonal argument for proving that $\kappa
< 2^\kappa$.

The formalization begins with a minimal axiomatic setup on sets, functions and
cardinals, sufficient for the proof of the theorem.

On an older laptop with an Intel Pentium N3710 processor \Naproche takes around
35 seconds to check this text.


\section{ForTheL Setup}

\begin{forthel}
  [read \path{examples/preliminaries.ftl.tex}]

  Let $f_{i}$ stand for $f(i)$.
\end{forthel}


\section{Cardinals}

\begin{forthel}
  Let $M, N$ denote sets.
  Let $f$ denote a function.

  \begin{signature}
    A cardinal is a set.
  \end{signature}

  Let $A, B, C$ denote cardinals.

  \begin{signature}
    $A < B$ is a relation.
  \end{signature}

  Let $A$ is less than $B$ stand for $A < B$.
  Let $A \leq B$ stand for $A = B$ or $A < B$.

  \begin{axiom}
    $A < B < C implies A < C$.
  \end{axiom}

  \begin{axiom}
    $A < B$ or $B < A$ or $B = A$.
  \end{axiom}

  \begin{signature}
    The cardinality of $M$ is a cardinal.
  \end{signature}

  Let $\card{M}$ denote the cardinality of $M$.

  \begin{axiom}
    $\card{A} = A$.
  \end{axiom}

  \begin{axiom}
    Assume $M$ is a subset of $\dom(f)$.
    $\card{f[M]} \leq \card{M}$.
  \end{axiom}

  \begin{axiom}
    Assume the cardinality of $N$ is less than the cardinality of $M$.
    Then $M \setminus N$ has an element.
  \end{axiom}

  \begin{axiom}[id=function_existence]
    Assume $\card{M} \leq \card{N}$.
    Assume $M$ has an element.
    There exists a function $f$ such that $N$ is the domain of $f$ and $M$ is
    the image of $f$.
  \end{axiom}
\end{forthel}


\section{Sums and Products of cardinals}

\begin{forthel}
  Let $D$ denote a set.

  \begin{definition}
    A sequence of cardinals on $D$ is a function $\kappa$ such that
    $\dom(\kappa) = D$ and $\kappa(i)$ is a cardinal for every element $i$ of
    $D$.
  \end{definition}

  \begin{definition}
    Let $\kappa$ be a sequence of cardinals on $D$.
    \[ \SumSet{\kappa}{D} = \class{(n,i) | \text{$i$ is an element of $D$ and
    $n$ is an element of $\kappa_{i}$}}. \]
  \end{definition}

  \begin{axiom}
    Let $\kappa$ be a sequence of cardinals on $D$.
    Then $\SumSet{\kappa}{D}$ is a set.
  \end{axiom}

  \begin{definition}
    Let $\kappa$ be a sequence of cardinals on $D$.
    $\Sum{\kappa}{D} = \card{\SumSet{\kappa}{D}}$.
  \end{definition}

  \begin{definition}
    Let $\kappa$ be a sequence of cardinals on $D$.
    \[ \ProdSet{\kappa}{D} = \class{f | \classtext{$f$ is a function and
    $\dom(f) = D$ and $f(i)$ is an element of $\kappa_{i}$ for every element
    $i$ of $D$}}. \]
  \end{definition}

  \begin{axiom}
    Let $\kappa$ be a sequence of cardinals on $D$.
    Then $\ProdSet{\kappa}{D}$ is a set.
  \end{axiom}

  \begin{definition}
    Let $\kappa$ be a sequence of cardinals on $D$.
    $\Prod{\kappa}{D} = \card{\ProdSet{\kappa}{D}}$.
  \end{definition}
\end{forthel}

König's Theorem requires some form of the axiom of choice.
Currently choice is built into Naproche by the \emph{choose} construct in
function definitions.

The axiom of choice is also required to show that products of non-empty factors
are themselves non-empty:

\begin{forthel}
  \begin{lemma}[title=Choice]
    Let $\lambda$ be a sequence of cardinals on $D$.
    Assume that $\lambda_{i}$ has an element for every element $i$ of $D$.
    Then $\ProdSet{\lambda}{D}$ has an element.
  \end{lemma}
  \begin{proof}
    Define $f(i) = \text{choose an element $v$ of $\lambda_{i}$ in $v$}$ for $i$
    in $D$.
    Then $f$ is an element of $\ProdSet{\lambda}{D}$.
  \end{proof}
\end{forthel}


\section{König's theorem}

\begin{forthel}
  Let $D$ denote a set.

  \begin{theorem}
    Let $\kappa, \lambda$ be sequences of cardinals on $D$.
    Assume that for every element $i$ of $D$ $\kappa_{i} < \lambda_{i}$.
    Then \[ \Sum{\kappa}{D} < \Prod{\lambda}{D}. \]
  \end{theorem}
  \begin{proof}[method=contradiction]
    Assume the contrary.
    Then \[ \Prod{\lambda}{D} \leq \Sum{\kappa}{D}. \]
    [unfoldlow off][prover vampire]
    Take a function $G$ such that $\SumSet{\kappa}{D}$ is the domain of $G$ and
    $\ProdSet{\lambda}{D}$ is the image of $G$.
    Indeed $\ProdSet{\lambda}{D}$ and $\ProdSet{\lambda}{D}$ are sets and
    $\ProdSet{\lambda}{D}$ has an element.
    [unfoldlow on][prover eprover]

    Define $\Delta(i) = \class{G(n,i)(i) | \text{$n$ is an element of
    $\kappa_{i}$}}$ for $i$ in $D$.

    For every element $f$ of $\ProdSet{\lambda}{D}$ and every element $i$ of $D$
    $f(i)$ is an element of $\lambda_{i}$.
    For every element $i$ of $D$ $\lambda_{i}$ is a set.
    For every element $i$ of $D$ and every element $d$ of $\Delta(i)$ we have
    $d\in \lambda_{i}$.
    For every element $i$ of $D$ $\Delta(i)$ is a set.

    For every element $i$ of $D$ $\card{\Delta(i)} < \lambda_{i}$. \newline
    Proof.
      Let $i$ be an element of $D$.
      Define $F(n) = G(n,i)(i)$ for $n$ in $\kappa_{i}$.
      Then $F[\kappa(i)] = \Delta(i)$.
    End.

    Define $f(i) = \text{choose an element $v$ of $\lambda_{i} \setminus
    \Delta(i)$ in $v$}$ for $i$ in $D$.
    Then $f$ is an element of $\ProdSet{\lambda}{D}$.
    Take an element $j$ of $D$ and an element $m$ of $\kappa_{j}$ such that
    $G(m,j) = f$.
    $G(m,j)(j)$ is an element of $\Delta(j)$ and $f(j)$ is not an element of
    $\Delta(j)$.
    Contradiction.
  \end{proof}
\end{forthel}

\printbibliography
\end{document}
