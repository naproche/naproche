\documentclass{article}
\usepackage{naproche}
\libinput[meta-inf]{example-preamble}
\libinput[libraries/foundations]{preamble}
\libinput[libraries/set-theory]{preamble}

\title{Burali-Forti's Paradox}
\author{\Naproche formalization:\\[0.5em]Marcel Schütz}
\date{2024}

\begin{document}
\begin{smodule}{burali-forti-paradox.ftl}
  \pagenumbering{gobble}
  \maketitle

  \usemodule[libraries/set-theory]{ordinals.ftl}
  \usemodule[libraries/foundations]{proper-classes.ftl}
  
  \noindent \emph{Burali-Forti's Paradox}, named after Cesare Burali-Forti, 
  demonstates that the collection $\Ord$ of all ordinal numbers is a
  proper class \cite{BuraliForti1897}.

  \inputref[libraries/set-theory]{the-ordinals-form-a-proper-class.ftl.tex}

  \printbibliography
  \printlicense[CcByNcSa]{2024}{Marcel Schütz}
\end{smodule}
\end{document}
