\documentclass[../solow_main.tex]{subfiles}

\begin{document}

\chapter{Differentiation}\label{chapter:differentiation}


\begin{forthel}
    [read \path{examples/solow/sections/2continuity.ftl.tex}]
\end{forthel}

In this chapter we want to define differentiability and prove some common derivation rules: \\
The derivative of a linear function is its slope,
$\fplus{f}{g}'(x) = f'(x) + g'(x)$, $\fmult{f}{g}'(x) = \fmult{f'}{g}(x) + \fmult{f}{g'}(x)$, and
the chain rule $(g \circ f)'(x) = g'(f(x)) \cdot f'(x)$.

In regards of notation $\Diffquot{f}{a}(x)$ means $\frac{f(x) - f(a)}{x - a}$, $\derivin{f}(x)$ is a real number 
that is the slope of $f$ in $x$, and $\deriv{f}$ is a function, names $f$'s derivative.

\section{Definition of Differentiability}

\begin{forthel}
Let $a$ denote a real number.
Let $f$ denote a real map.

\begin{definition}[def diffquot]
    $\Diffquot{f}{a}$ is a real map such that 
    $\Diffquot{f}{a}(x) = \frac{f(x) - f(a)}{x - a}$ for all $x \in \mathbb{R} \setminus \sset{a}{\mathbb{R}}$.
\end{definition}

\begin{definition}
    $f$ is differentiable in $a$ iff there exists a real number $y$ such that
    $\limS{\Diffquot{f}{a}}{a} = y$.
\end{definition}

\begin{definition}
    $f$ is differentiable iff for every real number $a$ $f$ is differentiable in $a$.
\end{definition}

\begin{definition}
    Assume $f$ is differentiable in $a$.
    The derivative of $f$ in $a$ is a real number $y$ such that $\limS{\Diffquot{f}{a}}{a} = y$.
\end{definition}
Let $\derivin{f}(x)$ stand for the derivative of $f$ in $x$.

\begin{definition}
    Let $f$ be differentiable.
    The derivative of $f$ is a real map $d$ such that for every real number $x$ $d(x)$ is equal to the derivative of $f$ in $x$.
\end{definition}
Let $\deriv{f}$ stand for the derivative of $f$.

\begin{lemma}[deriv and derivin]
    Let $f$ be differentiable.
    $\deriv{f}(a) = \derivin{f}(a)$.
\end{lemma}

\end{forthel}



\section{Equivalent Definition of Differentiability}

Although the above definition of differentiability is standard, we give here an equivalent definition and also prove 
the equivalence, because this one will be more useful to prove the following theorems. 
We define a function as the difference quotient of a function $f$; i.e. $\varphi(x) = \frac{f(x) - f(a)}{x - a}$ for 
all $x \neq a$. If we can find such a function $\varphi$ that is in addition continuous in $a$, then $f$ is 
differentiable in $a$. This map we call difference quotient map.

\begin{forthel}
Let $a$ denote a real number.
Let $f$ denote a real map. [timelimit 50]

\begin{definition}
    The difference quotient map of $f$ in $a$ is a real map $\varphi$ such that $\varphi$ is continuous in $a$ and 
    for all real numbers $x$ $f(x) = ((x - a) \cdot \varphi(x)) + f(a)$.
\end{definition}

\begin{proposition}[prop difference quotient map]
    Let $\varphi$ be the difference quotient map of $f$ in $a$.
    $f$ is differentiable in $a$ and $\derivin{f}(a) = \varphi(a)$
    and for every $x \in \mathbb{R} \setminus \sset{a}{\mathbb{R}}$ $\varphi(x) = \Diffquot{f}{a}(x)$.
\end{proposition}
\begin{proof}
    Let us show that $\varphi(x) = \Diffquot{f}{a}(x)$ for every $x \in \mathbb{R} \setminus \sset{a}{\mathbb{R}}$.
        Let $x \in \mathbb{R} \setminus \sset{a}{\mathbb{R}}$. Then $x - a \neq 0$. 
        $f(x) - f(a) = (((x - a) \cdot \varphi(x)) + f(a)) - f(a) 
                = ((x - a) \cdot \varphi(x)) + (f(a) - f(a)) = (x - a) \cdot \varphi(x)$.
        $\Diffquot{f}{a}(x) = \frac{f(x) - f(a)}{x - a} = \frac{(x - a) \cdot \varphi(x)}{x - a} = \varphi(x)$.
    End. 

    $\varphi(a)$ is a real number.

    $\limS{\Diffquot{f}{a}}{a} = \varphi(a)$. Proof.
        Let $\varepsilon$ be a positive real number.

        Take a positive real number $\delta$ such that $\delta$ is a domain restrictor for $\varepsilon$ and $\varphi$ around $a$ and $\varphi(a)$.

        Then $\delta$ is a domain restrictor for $\varepsilon$ and $\Diffquot{f}{a}$ around $a$ and $\varphi(a)$.
        Indeed for every $x \in \mathbb{R} \setminus \sset{a}{\mathbb{R}}$ if $\Rmetric(x,a) < \delta$ then
                $\Diffquot{f}{a}(x) = \varphi(x)$ and 
                $\Rmetric(\Diffquot{f}{a}(x),\varphi(a)) = \Rmetric(\varphi(x),\varphi(a)) < \varepsilon$.
    End.

    $\derivin{f}(a) = \limS{\Diffquot{f}{a}}{a} = \varphi(a)$.
\end{proof}

\begin{theorem}
    $f$ is differentiable in $a$ iff there exists a difference quotient map of $f$ in $a$.
\end{theorem}
\begin{proof}
    Case $f$ is differentiable in $a$.
    Define $$\varphi(x) = \begin{cases}
        \Diffquot{f}{a}(x) & : x \neq a \\
        \derivin{f}(a)  & : x = a
    \end{cases}$$
    for $x \in \mathbb{R}$. $\varphi$ is a real map. Indeed $\range(\varphi) \subseteq \mathbb{R}$.

    Let us show that $\varphi$ is continuous in $a$.
        Let $\varepsilon$ be a positive real number.
        Take a positive real number $\delta$ such that $\delta$ is a domain restrictor for $\varepsilon$ and $\Diffquot{f}{a}$ around $a$ and $\derivin{f}(a)$. 
            Indeed for every positive real number $\epsilon$ there exists a domain restrictor for $\epsilon$ and $\Diffquot{f}{a}$ around $a$ and $\derivin{f}(a)$. 

        Then $\delta$ is a domain restrictor for $\varepsilon$ and $\varphi$ around $a$ and $\derivin{f}(a)$.
        Indeed for every $x \in \mathbb{R} \setminus \sset{a}{\mathbb{R}}$ if $\Rmetric(x,a) < \delta$ then
                $\varphi(x) = \Diffquot{f}{a}(x)$ and 
                $\Rmetric(\varphi(x),\derivin{f}(a)) = \Rmetric\left(\Diffquot{f}{a}(x),\derivin{f}(a)\right) < \varepsilon$.
    End.

    Let us show that for all real numbers $x$ $f(x) = ((x - a) \cdot \varphi(x)) + f(a)$.
        Let $x$ be a real number.

        Case $x = a$. Trivial.

        Case $x \neq a$.
            $((x - a) \cdot \varphi(x)) + f(a) 
            = \left((x - a) \cdot \Diffquot{f}{a}(x)\right) + f(a)
            = \left((x - a) \cdot \frac{f(x) - f(a)}{x - a}\right) + f(a)
            = \left(\frac{x - a}{x - a} \cdot (f(x) - f(a))\right) + f(a)
            = (f(x) - f(a)) + f(a) 
            = f(x)$.
        End.
    End.
    
    Therefore $\varphi$ is a difference quotient map of $f$ in $a$.
    End. 

    Case there exists a difference quotient map of $f$ in $a$. Trivial.
\end{proof}

[timelimit 10]
\begin{theorem}
    Let $f$ be differentiable in $a$.
    Then $f$ is continuous in $a$.
\end{theorem}
\begin{proof}
    Take the difference quotient map $\varphi$ of $f$ in $a$.
    For all $x \in \dom(f)$ 
    
    $f(x) = ((x - a) \cdot \varphi(x)) + f(a) 
            = (\flin{1}{-a}(x) \cdot \varphi(x)) + f(a) 
            = (\flin{1}{-a}(x) \cdot \varphi(x)) + \flin{0}{f(a)}(x)
            = \fplus{\fmult{\flin{1}{-a}}{\varphi}}{\flin{0}{f(a)}}(x)$. 
            
            [prover vampire]
    Therefore $f = \fplus{\fmult{\flin{1}{-a}}{\varphi}}{\flin{0}{f(a)}}$. [prover eprover]

    $\fplus{\fmult{\flin{1}{-a}}{\varphi}}{\flin{0}{f(a)}}$ is continuous in $a$.
    Indeed $\fmult{\flin{1}{-a}}{\varphi}$ is continuous in $a$. 
\end{proof}

\begin{corollary}
    Let $f$ be a differentiable real map.
    Then $f$ is continuous.
\end{corollary}[timelimit 3]
\end{forthel}



\section{Differentiation Rules}

\subsection{Derivative of a Linear Function}
\begin{forthel}
Let $a,b$ denote real numbers.

\begin{theorem}[derivative of linear function]
    $\flin{a}{b}$ is differentiable and $\deriv{\flin{a}{b}} = \flin{0}{a}$.
\end{theorem}
\begin{proof}
    Take $f = \flin{a}{b}$.
    Let $\xi$ be a real number. [timelimit 30]

    Let us show that $\derivin{f}(\xi) = a$.
        Let $\varepsilon$ be a positive real number.
    
        We can show that 1 is a domain restrictor for $\varepsilon$ and $\Diffquot{f}{\xi}$ around $\xi$ and $a$.
            Let $x \in \mathbb{R} \setminus \sset{\xi}{\mathbb{R}}$. Thus $x - \xi \neq 0$.
            
            Then $f(x) - f(\xi)= a \cdot (x - \xi)$. Indeed 
                $((a \cdot x) + b) - ((a \cdot \xi) + b) .= ((a \cdot x) + b) + (-b - (a \cdot \xi)) (by eq lemma 15, add comm)
                .= ((a \cdot x) + (b - b)) - (a \cdot \xi) (by add ass)
                .= (a \cdot x) - (a \cdot \xi) (by additive inverse, existence zero)
                .= (a \cdot x) + (a \cdot (-\xi)) (by mult comm, eq lemma 11)
                .= a \cdot (x - \xi)$ (by distributive law).

            $\Diffquot{f}{\xi}(x) = \frac{f(x) - f(\xi)}{x - \xi} = \frac{a \cdot (x - \xi)}{x - \xi} = a$. [timelimit 3]
            $\Rmetric(\Diffquot{f}{\xi}(x),a) = \Rmetric(a,a) = 0 < \varepsilon$.
        End.
    End.

    Therefore $\deriv{f}(\xi) = a = \flin{0}{a}(\xi)$ (by linear lem 1).
\end{proof}

\end{forthel}


\subsection{Derivative Under Addition}

\begin{forthel}
Let $f,g$ denote real maps.

\begin{theorem}[derivative addition rule]
    Let $f,g$ be differentiable in $a$.
    $\fplus{f}{g}$ is differentiable in $a$ and $\derivin{\fplus{f}{g}}(a) = \derivin{f}(a) + \derivin{g}(a)$.
\end{theorem}
\begin{proof}
    Take difference quotient map $\varphi$ of $f$ in $a$.
    Take difference quotient map $\gamma$ of $g$ in $a$.
    Take $\psi = \fplus{\varphi}{\gamma}$.

    Let us show that $\psi$ is the difference quotient map of $\fplus{f}{g}$ in $a$.
        $\varphi,\gamma$ are continuous in $a$. Thus $\psi$ is continuous in $a$.
        
        We can show that for all real numbers $x$ $\fplus{f}{g}(x) = ((x - a) \cdot \psi(x)) + \fplus{f}{g}(a)$. [timelimit 10]
            Let $x$ be a real number.

            $\fplus{f}{g}(x) = f(x) + g(x)$ 

            $= (((x - a) \cdot \varphi(x)) + f(a)) + (((x - a) \cdot \gamma(x)) + g(a))$

            $= (((x - a) \cdot \varphi(x)) + (((x - a) \cdot \gamma(x)) + f(a))) + g(a)$

            $= (((x - a) \cdot \varphi(x)) + ((x - a) \cdot \gamma(x))) + (f(a) + g(a))$

            $= ((x - a) \cdot (\varphi(x) + \gamma(x))) + \fplus{f}{g}(a)$

            $= ((x - a) \cdot \psi(x)) + \fplus{f}{g}(a)$. 
        End.
    End.

    $\derivin{\fplus{f}{g}}(a) = \psi(a) = \varphi(a) + \gamma(a) = \derivin{f}(a) + \derivin{g}(a)$.
\end{proof}

\begin{corollary}[derivative addition rule 2]
    Let $f,g$ be differentiable real maps.
    $\fplus{f}{g}$ is differentiable and $\deriv{\fplus{f}{g}} = \fplus{\deriv{f}}{\deriv{g}}$.
\end{corollary}
\begin{proof}
    For all real numbers $x$

        $\deriv{\fplus{f}{g}}(x) = \derivin{\fplus{f}{g}}(x) = \derivin{f}(x) + \derivin{g}(x)
        = \deriv{f}(x) + \deriv{g}(x) = \fplus{\deriv{f}}{\deriv{g}}(x)$.
\end{proof}

\end{forthel}



\subsection{Derivative Under Multiplication}

\begin{forthel}

\begin{theorem}[derivative multiplication rule]
    Let $f,g$ be differentiable in $a$.
    $\fmult{f}{g}$ is differentiable in $a$ and $\derivin{\fmult{f}{g}}(a) = (\derivin{f}(a) \cdot g(a)) + (f(a) \cdot \derivin{g}(a))$.
\end{theorem}
\begin{proof}
    Take difference quotient map $\varphi$ of $f$ in $a$.
    Take difference quotient map $\gamma$ of $g$ in $a$.
    Take $\psi = \fplus{\fmult{\varphi}{g}}{\fmult{\flin{0}{f(a)}}{\gamma}}$.
    $\psi$ is continuous in $a$. Indeed $\fmult{\varphi}{g}, \fmult{\gamma}{\flin{0}{f(a)}}$ are continuous in $a$. [timelimit 30]

    (0) $\psi(x) = (\varphi(x) \cdot g(x)) + (f(a) \cdot \gamma(x))$ for any real number $x$. Indeed for every real number $x$
    $(\fplus{\fmult{\varphi}{g}}{\fmult{\flin{0}{f(a)}}{\gamma}})(x) = (\varphi(x) \cdot g(x)) + (f(a) \cdot \gamma(x))$ (by def mult func, def add func, linear lem 1).
    
    Let us show that $\psi$ is the difference quotient map of $\fmult{f}{g}$ in $a$.
    Indeed we can show that for all real numbers $x$ $\fmult{f}{g}(x) = ((x - a) \cdot \psi(x)) + \fmult{f}{g}(a)$.
        Let $x$ be a real number.

        (1) $f(x) = ((x - a) \cdot \varphi(x)) + f(a)$.

        (2) $g(x) = ((x - a) \cdot \gamma(x)) + g(a)$.

        $\fmult{f}{g}(x) .= f(x) \cdot g(x)$ (by def mult func)

        $.= (((x - a) \cdot \varphi(x)) + f(a)) \cdot g(x)$ (by 1)

        $.= (((x - a) \cdot \varphi(x)) \cdot g(x)) + (f(a) \cdot g(x))$ (by eq lemma 1) 

        $.= ((x - a) \cdot (\varphi(x) \cdot g(x))) + (f(a) \cdot (((x - a) \cdot \gamma(x)) + g(a)))$ (by mult ass, 2)

        $.= (((x - a) \cdot (\varphi(x) \cdot g(x))) + (f(a) \cdot ((x - a) \cdot \gamma(x)))) + (f(a) \cdot g(a))$ (by distributive law, add ass)

        $.= (((x - a) \cdot (\varphi(x) \cdot g(x))) + ((x - a) \cdot (f(a) \cdot \gamma(x)))) + (f(a) \cdot g(a))$ (by mult comm, mult ass)

        $.= ((x - a) \cdot ((\varphi(x) \cdot g(x)) + (f(a) \cdot \gamma(x)))) + (f(a) \cdot g(a))$ (by distributive law)

        $.= ((x - a) \cdot \psi(x)) + \fmult{f}{g}(a)$ (by 0, def mult func).
    End.

    $\derivin{\fmult{f}{g}}(a) = \psi(a) = \fmult{\varphi}{g}(a) + \fmult{\flin{0}{f(a)}}{\gamma}(a)
    = (\varphi(a) \cdot g(a)) + (f(a) \cdot \gamma(a))$ 

    $= (\derivin{f}(a) \cdot g(a)) + (f(a) \cdot \derivin{g}(a))$.
\end{proof}
[prover vampire]
\begin{corollary}[derivative multiplication rule 2]
    Let $f,g$ be differentiable real maps.
    $\fmult{f}{g}$ is differentiable and $\deriv{\fmult{f}{g}} = \fplus{\fmult{\deriv{f}}{g}}{\fmult{f}{\deriv{g}}}$.
\end{corollary}
\begin{proof}
    Let $x$ be a real number.

    $\deriv{\fmult{f}{g}}(x) = \derivin{\fmult{f}{g}}(x) = (\derivin{f}(x) \cdot g(x)) + (f(x) \cdot \derivin{g}(x))
    = (\deriv{f}(x) \cdot g(x)) + (f(x) \cdot \deriv{g}(x)) = \fplus{\fmult{\deriv{f}}{g}}{\fmult{f}{\deriv{g}}}(x)$.
\end{proof}
[prover eprover]
\end{forthel}


\subsection{Derivative Chain Rule}

\begin{forthel}

\begin{theorem}[derivative chain rule]
    Let $f$ be differentiable in $a$ and $g$ be differentiable in $f(a)$.
    $g \circ f$ is differentiable in $a$ and $\derivin{(g \circ f)}(a) = \derivin{g}(f(a)) \cdot \derivin{f}(a)$.
\end{theorem}
\begin{proof} 
    $g \circ f$ is a real map. 
    Take difference quotient map $\varphi$ of $f$ in $a$.
    Take difference quotient map $\gamma$ of $g$ in $f(a)$. 
    Take $\psi = \fmult{(\gamma \circ f)}{\varphi}$. 

    (1) Then $\psi$ is continuous in $a$ and for every $x \in \mathbb{R}$ $\psi(x) = \gamma(f(x)) \cdot \varphi(x)$.

    $\psi$ is the difference quotient map of $g \circ f$ in $a$.

    Indeed we can show that for all real numbers $x$ $(g \circ f)(x) = ((x - a) \cdot \psi(x)) + (g \circ f)(a)$.
        Let $x$ be a real number.

        (2)$f(x) - f(a) = (x - a) \cdot \varphi(x)$.
        (3) For all real numbers $y$ $g(y)= ((y - f(a)) \cdot \gamma(y)) + g(f(a))$.

        $g(f(x)) .= ((f(x) - f(a)) \cdot \gamma(f(x))) + g(f(a))$ (by 3) 

        $.= (((x - a) \cdot \varphi(x)) \cdot \gamma(f(x))) + g(f(a))$ (by 2) 

        $.= ((x - a) \cdot (\gamma(f(x)) \cdot \varphi(x))) + g(f(a))$ (by mult ass, mult comm) 

        $.= ((x - a) \cdot \psi(x)) + g(f(a))$ (by 1).
    End.

    $\derivin{(g \circ f)}(a) = \psi(a) = \gamma(f(a)) \cdot \varphi(a) = \derivin{g}(f(a)) \cdot \derivin{f}(a)$.
\end{proof}

\begin{corollary}[derivative chain rule 2]
    Let $f,g$ be differentiable real maps.
    $g \circ f$ is differentiable and $\deriv{(g \circ f)} = \fmult{(\deriv{g} \circ f)}{\deriv{f}}$.
\end{corollary}
\begin{proof}
    Let $x$ be a real number. (1) $\deriv{g}(f(x)) = (\deriv{g} \circ f)(x)$ (by composition).

        $\deriv{(g \circ f)}(x) .= \derivin{(g \circ f)}(x)$ (by deriv and derivin)

        $.= \derivin{g}(f(x)) \cdot \derivin{f}(x)$ (by derivative chain rule)

        $.= \deriv{g}(f(x)) \cdot \deriv{f}(x)$ (by deriv and derivin)

        $.= (\deriv{g} \circ f)(x) \cdot \deriv{f}(x)$ (by 1)

        $.= \fmult{(\deriv{g} \circ f)}{\deriv{f}}(x)$ (by def mult func).
\end{proof}

\end{forthel}


\subsection{Derivative With a Minus}

In contrast to the previous proofs of differentiability, in the proof about the 
differentiability of $\fminus{f}$ we use Frerix' rewriting algorithm. Sometimes this rewriting mechanism
solves equations that the normal mechanism was not able to do. On the other hand, even the smallest 
rewriting step must be precisely named in the rewriting algorithm for it to solve the equations, which 
is unhelpful with regard to the goal of making the formalization appear more natural. Perhaps the biggest 
advantage of Frerix' algorithm is its speed. It makes the verification process significantly faster than it 
would otherwise be. 

\begin{forthel}

\begin{definition}[def func minus]
    $\fminus{f}$ is a real map such that $\fminus{f}(x) = -f(x)$ for any real number $x$.
\end{definition}

\begin{lemma}[minus function lemma]
    $\fminus{f} = \flin{-1}{0} \circ f$.
\end{lemma}
\begin{proof}
    Let $x$ be a real number.

    $(\flin{-1}{0} \circ f)(x)
    .= -f(x)$ (by  composition, linear lem 2, eq lemma 12)
    $.= \fminus{f}(x)$.
\end{proof}

\begin{lemma}[minus derivative]
    Let $f$ be differentiable in $a$.
    Then $\fminus{f}$ is differentiable in $a$ and $\derivin{\fminus{f}}(a) = -\derivin{f}(a)$.
\end{lemma}
\begin{proof}
    $\derivin{\fminus{f}}(a)
    .= \derivin{(\flin{-1}{0} \circ f)}(a)$ (by minus function lemma)

    $.= \derivin{\flin{-1}{0}}(f(a)) \cdot \derivin{f}(a)$ (by derivative chain rule)

    $.= \flin{0}{-1}(f(a)) \cdot \derivin{f}(a)$ (by  deriv and derivin, derivative of linear function)

    $.= (-1) \cdot \derivin{f}(a)$ (by linear lem 1)
    
    $.= -\derivin{f}(a)$ (by eq lemma 12).
\end{proof}

\begin{corollary}[minus derivative 2]
    Let $f$ be differentiable.
    Then $\fminus{f}$ is differentiable and $\deriv{\fminus{f}} = \fminus{\deriv{f}}$.
\end{corollary} 

\end{forthel}

\end{document}

