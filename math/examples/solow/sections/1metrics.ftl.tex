\documentclass[../solow_main.tex]{subfiles}

\begin{document}

\chapter{Metric}\label{chapter:metric}

First we need to import the previous file, which we do by the following command.
\begin{forthel}
    [read \path{examples/solow/sections/0preparation.ftl.tex}]
\end{forthel}

The aim of the first chapter is to introduce absolute values, which are necessary for the following definitions of 
continuity and differentiability. We do so in the context of metrics as this is commonly done in standard textbooks
(see \cite{Fuente_2000}). 


\section{Definition of a Metric}


\begin{forthel}
    
Let $X,Y$ denote sets.

\begin{definition}
    Let $d$ be a map from $(X \times X)$ to $\mathbb{R}$.
    $d$ is positive definit on $X$ iff 
    for all $x,y \in X$ $d(x,y) = 0$ iff $x = y$.
\end{definition}
\begin{definition}
    Let $d$ be a map from $X \times X$ to $\mathbb{R}$.
    $d$ is reflexive on $X$ iff
    for all $x,y \in X$ $d(x,y) = d(y,x)$.
\end{definition}
\begin{definition}
    Let $d$ be a map from $X \times X$ to $\mathbb{R}$.
    $d$ holds triangle inequality on $X$ iff 
    For all $x,y,z \in X$ $d(x,z),d(x,y),d(y,z)$ are real numbers and $d(x,z) \leq d(x,y) + d(y,z)$.
\end{definition}

\begin{definition}
    A metric on $X$ is a map $d$ from $X \times X$ to $\mathbb{R}$ such that
    $d$ is positive definit on $X$ and $d$ is reflexive on $X$ and $d$ holds triangle inequality on $X$.
\end{definition}

[prover vampire][timelimit 20]
\begin{lemma}
    Let $x,y \in X$ and $d$ be a metric on $X$.
    Then $d(x,y)$ is a real number and $d(x,y) \geq 0$.
\end{lemma}
\begin{proof}
    $d$ holds triangle inequality on $X$. Thus $d(x,x) \leq d(x,y) + d(y,x)$.
    $0 = d(x,x) \leq d(x,y) + d(y,x) = d(x,y) + d(x,y) = 2 \cdot d(x,y)$.
    Thus $0 \leq 2 \cdot d(x,y)$. $d(x,y) \geq 0$. Indeed $d(x,y)$ is a real number.
\end{proof}


\begin{lemma}
    Let $Y$ be a subset of $X$.
    Then $Y \times Y$ is a subset of $X \times X$.
\end{lemma}

\begin{lemma}
    Let $d$ be a map from $X$ to $\mathbb{R}$ and $Y$ be a subset of $X$.
    Then $d \upharpoonright Y$ is a map from $Y$ to $\mathbb{R}$.
\end{lemma}

\begin{proposition}
    Let $d$ be a map from $X \times X$ to $\mathbb{R}$. Let $Y$ be a subset of $X$.
    If $d$ is a metric on $X$ then $d \upharpoonright (Y \times Y)$ is a metric on $Y$.
\end{proposition}
\begin{proof}
    Take $\delta = d \upharpoonright (Y \times Y)$.
    [timelimit 20] $\delta$ is a map from $Y \times Y$ to $\mathbb{R}$.
    Thus for all $x,y \in Y$ $(x,y) \in (Y \times Y)$ and $\delta(x,y)$ is a real number. [timelimit 3]

    Assume $d$ is a metric on $X$.
    $\delta$ is positive definit on $Y$.
    $\delta$ is reflexive on $Y$.
    $\delta$ holds triangle inequality on $Y$. Proof.
        Let $x,y,z \in Y$. Then $d(x,y),d(y,z)$ are real numbers and
        $\delta(x,z) = d(x,z) \leq d(x,y) + d(y,z) = \delta(x,y) + \delta(y,z)$.
        Therefore $\delta(x,z) \leq \delta(x,y) + \delta(y,z)$.
    End.
\end{proof}

\end{forthel}



\section{Absolute Value}


\begin{forthel}
Let $a,x,y$ denote real numbers.

\begin{definition}[def absolute value]
    $\abs{a}$ is a nonnegative real number such that (if $a \geq 0$ then $\abs{a} = a$) and if $a < 0$ then $\abs{a} = -a$.
\end{definition}

\begin{lemma}
    $\abs{a} = a$ or $\abs{a} = -a$.
\end{lemma}

\begin{lemma}[absolute value nonnegative]
    $\abs{a}$ is nonnegative.
\end{lemma}

\begin{lemma}[absolute value positive]
    Let $a$ be nonzero.
    $\abs{a}$ is positive.
\end{lemma}

\begin{lemma}[absolute lemma 1]
    $\abs{a} = \abs{-a}$.
\end{lemma}

[timelimit 30]
\begin{lemma}[absolute lemma 2]
    $\abs{x \cdot y} = \abs{x} \cdot \abs{y}$.
\end{lemma}
\begin{proof}
    $x \cdot y \geq 0$ or $x \cdot y < 0$.
    If $x \cdot y \geq 0$ then $\abs{x \cdot y} = x \cdot y = \abs{x} \cdot \abs{y}$. [timelimit 3]
    If $x \cdot y < 0$ then $\abs{x \cdot y} = - x \cdot y = \abs{x} \cdot \abs{y}$.
\end{proof}

\begin{lemma}[absolute lemma 3]
    $\abs{x - y} = \abs{y - x}$.
\end{lemma}
\begin{proof}
    $\abs{x - y} .= \abs{-(x - y)}$ (by absolute lemma 1)

    $.= \abs{-x - (-y)}$ (by eq lemma 15)

    $.= \abs{y - x}$ (by eq lemma 45, add comm).
\end{proof}

\begin{lemma}[absolute lemma 4]
    $\abs{x + y} \leq \abs{x} + \abs{y}$.
\end{lemma}
\begin{proof}
    If $x \geq 0$ and $y \geq 0$ then $\abs{x + y} = \abs{x} + \abs{y}$.
    If $x \leq 0$ and $y \leq 0$ then $\abs{x + y} = \abs{x} + \abs{y}$.
    Case $x > 0$ and $y < 0$. If $\abs{x} \geq \abs{y}$ then $\abs{x + y} = x + y < x = \abs{x} < \abs{x} + \abs{y}$.

    If $\abs{x} < \abs{y}$ then $\abs{x + y} = - (x + y) = -y - x < -y = \abs{y} < \abs{y} + \abs{x}$. End.

    Case $x < 0$ and $y > 0$. If $\abs{y} \geq \abs{x}$ then $\abs{x + y} = x + y < y = \abs{y} < \abs{x} + \abs{y}$.
    
    If $\abs{y} < \abs{x}$ then $\abs{x + y} = - (x + y) = -x - y < -x = \abs{x} < \abs{y} + \abs{x}$. End.
\end{proof}

\begin{lemma}
    $\abs{x} + \abs{y} = \abs{\abs{x} + \abs{y}}$.
\end{lemma}

\begin{lemma}[absolute lemma 5]
    Let $a$ be nonzero.
    $\abs{1/a} = 1/\abs{a}$.
\end{lemma}

\end{forthel}



\section{The Canonical Metric on the Real Numbers}


\begin{forthel}
    
\begin{definition}[Manhattan metric]
    $\Rmetric$ is a map from $\mathbb{R} \times \mathbb{R}$ to $\mathbb{R}$ such that
    for all $x,y \in \mathbb{R}$ $\Rmetric(x,y) = \abs{x - y}$.
\end{definition}

\begin{theorem}
    $\Rmetric$ is a metric on $\mathbb{R}$.
\end{theorem}
\begin{proof}
    $\mathbb{R}$ is a set and $\Rmetric$ is a map from $\mathbb{R} \times \mathbb{R}$ to $\mathbb{R}$.

    (1) $\Rmetric$ is positive definit on $\mathbb{R}$.
    Indeed we can show that for all real numbers $x,z$ $\Rmetric(x,z) = 0$ iff $x = z$.
    Let $x,z$ be real numbers.
    If $x = z$ then $\Rmetric(x,z) = 0$.
    Case $x \neq z$. Then $x - z \neq 0$ and $- (x - z) \neq 0$. Thus $\abs{x - z} \neq 0$. Thus $\Rmetric(x,z) \neq 0$. End.
    End.

    (2) $\Rmetric$ is reflexive on $\mathbb{R}$.
    Indeed $\Rmetric(x,z) = \abs{x - z} = \abs{z - x} = \Rmetric(z,x)$ for all real numbers $x,z$.

    (3) $\Rmetric$ holds triangle inequality on $\mathbb{R}$.
    Indeed we can show that for all real numbers $x,y,z$ $\Rmetric(x,z) \leq \Rmetric(x,y) + \Rmetric(y,z)$.
    Let $x,y,z$ be a real number.

    $\Rmetric(x,z) = \abs{x - z} = \abs{(x + (- y + y)) - z} = \abs{(x - y) + (y - z)} \leq \abs{\abs{x - y} + \abs{y - z}} =
    \abs{x - y} + \abs{y - z} = \Rmetric(x,y) + \Rmetric(y,z)$. End.
\end{proof}

\begin{lemma}[absolute lemma 6]
    Let $x,y$ be real numbers.
    $\abs{x} - \abs{y} \leq \abs{x - y}$.
\end{lemma}
\begin{proof}
    $\Rmetric$ holds triangle inequality on $\mathbb{R}$. Thus $\Rmetric(x,0) \leq \Rmetric(x,y) + \Rmetric(y,0)$. 
    $\abs{x} - \abs{y} = \Rmetric(x,0) - \abs{y} \leq (\Rmetric(x,y) + \Rmetric(y,0)) - \abs{y}
    = (\abs{x - y} + \abs{y}) - \abs{y} = \abs{x - y}.$
\end{proof}

\end{forthel}

\end{document}

