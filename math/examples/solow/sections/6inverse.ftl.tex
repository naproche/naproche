\documentclass[../solow_main.tex]{subfiles}

\begin{document}

\chapter{Inverse Function Theorem}


\begin{forthel}
    [read \path{examples/solow/sections/5extrema.ftl.tex}]
\end{forthel}

Now we are finally able to devote ourselves explicitly to the inverse function theorem. The inverse functon theorem says that under some conditions
$(f^{-1})'(y) = 1/f'(f^{-1}(y))$. One such condition, for which we want to prove the theorem, is that $f$ is continuously differentiable and $f'(x)$
is nonzero for $y = f(x)$. In lemma lemma mon 4 we show that in that case there exists an open neighbourhood $E$ of $x$ such that $f$ is strictly 
monotone on $E$. Then we know that $(f \upharpoonright E)^{-1}$ exists. In lemma lemma mon 3 we show that the inverse is continuous in $y$.
This we eventually use in our actually proof of the inverse function theorem to show that a function that we define there is continuous in $y$ and 
therefore a difference quotient map.

\begin{forthel}
    
Let $f$ denote a real map.
Let $a$ denote a real number.
Let $U$ denote an interval.
Let $E$ denote a subset of $\mathbb{R}$.

\begin{lemma}[lemma mon 1]
    Let $f$ be strictly monotone on $E$.
    $f \upharpoonright E$ is injective. 
\end{lemma}
\begin{proof}
    We can show that for all distinct elements $x,y$ in $E$ $f(x) \neq f(y)$.
    Let $x,y$ be distinct elements in $E$. $x,y \in E$.

    Case $f$ is strictly increasing on $E$. If $x < y$ then $f(x) < f(y)$. 
        If $x > y$ then $f(x) > f(y)$. $f(x) \neq f(y)$. End.

    Case $f$ is strictly decreasing on $E$. If $x < y$ then $f(x) > f(y)$.
        If $x > y$ then $f(x) < f(y)$. $f(x) \neq f(y)$. End.
    End.
\end{proof}

\end{forthel}

We want to show that if $E$ is an open neighbourhood of $a$ and $f$ continuous and strictly monotone on $E$, then $f[E]$ is an open neighbourhood of $f(a)$. Let us assume $f$ is strictly increasing as the case for $f$ strictly decreasing works analogously. We need to show that $f[E]$ is open. Therefore let $\xi$ be an element of $f[E]$. Take $x \in E$ $f(x) = \xi$ and, because $E$ is open, there exists $\delta > 0$ such that $\B_{\delta}(x) \subseteq E$. One corollary of the intermediate value theorem that we proved in the monotonicity subsection of the section about extrema, that for continuous strictly increasing functions 
$f[(x - \delta, x + \delta)] = (f(x - \delta),f(x + \delta)) \ni \xi$. Thus 
$$\exists d >0:\ \B_{d}(\xi) \subseteq (f(x - \delta),f(x + \delta)) \subseteq f[E].$$

\begin{forthel}

\begin{lemma}[lemma mon 2]
    Let $E$ be an open neighbourhood of $a$.
    Let $f$ be continuous and strictly monotone on $E$.
    Then $f[E]$ is an open neighbourhood of $f(a)$.
\end{lemma}
\begin{proof}
    $f(a)$ is an element of $f[E]$. $f[E]$ is a subset of $\mathbb{R}$.

    $f[E]$ is an open set.
    Indeed we can show that for every $\xi \in f[E]$ 
            there exists a positive real number $r$ such that $\B_{r}(\xi) \subseteq f[E]$.

        Let $\xi \in f[E]$. Take $x \in E$ such that $f(x) = \xi$.
        Take a positive real number $\delta$ such that $\B_{\delta}(x) \subseteq E$.
        Take a positive real number $d$ such that $d < \delta$.
        Take $y = x - d$, $z = x + d$. Then $y < x < z$.

        (1) Then $x,y,z \in \B_{\delta}(x)$. Proof.
            $x - \delta < x - d$ (by ineq lemma 22). Indeed $x - d \neq x - \delta$. [timelimit 30]
            $x - d, x + d \in (x - \delta, x + \delta)_{OI}$. Indeed $x+d$ is a real number and $x + d < x + \delta$.
        End.

        (2) $\B_{\delta}(x)$ is an interval (by ball and open interval).
        Indeed $(x - \delta, x + \delta)_{OI}$ is an interval (by interval lemma).
       
        (3) Case $f$ is strictly increasing on $\B_{\delta}(x)$. [prover vampire]
            $f[(y,z)_{OI}] = (f(y), f(z))_{OI}$ (by cor I intermediate value theorem, 1, 2). 
            Indeed $f$ is a continuous real map such that $f$ is strictly increasing on $\B_{\delta}(x)$.
            $\xi$ is an element of $(f(y), f(z))_{OI}$. Proof.
                $f(y) < f(x)$ (by 1,3). Indeed $y < x$. 
                $f(x) < f(z)$ (by 1,3). Indeed $x < z$. 
                Therefore $f(y) < \xi < f(z)$. 
            End. 
            Take a positive real number $r$ such that $\B_{r}(\xi) \subseteq (f(y), f(z))_{OI}$ (by open interval is open).
            Indeed $(f(y), f(z))_{OI}$ is an open set (by open interval is open).
            $\B_{r}(\xi)$ is a subset of $f[E]$. Proof. 
                $(y,z)_{OI} \subseteq E$. Indeed $(y,z)_{OI} \subseteq \B_{\delta}(x) \subseteq E$.
                Therefore $f[(y,z)_{OI}]$ is a subset of $f[E]$.
                Thus $\B_{r}(\xi) \subseteq f[E]$. Indeed
                $\B_{r}(\xi) \subseteq (f(y), f(z))_{OI} = f[(y,z)_{OI}] \subseteq f[E]$.
            End.
        End.

        Case $f$ is strictly decreasing on $\B_{\delta}(x)$. [timelimit 20] 
            $f[(y,z)_{OI}] = (f(z), f(y))_{OI}$ (by cor II intermediate value theorem, 1, 2).
            Indeed $f$ is a continuous real map such that $f$ is strictly decreasing on $\B_{\delta}(x)$.
            $\xi$ is an element of $(f(z), f(y))_{OI}$. Indeed $f(z) < \xi < f(y)$. Indeed $y < x < z$. [timelimit 3]
            Take a positive real number $r$ such that $\B_{r}(\xi) \subseteq (f(z), f(y))_{OI}$ (by open interval is open).
            $\B_{r}(\xi)$ is a subset of $f[E]$. Indeed $f[(y,z)_{OI}]$ is a subset of $f[E]$.
                Indeed $(y,z)_{OI} \subseteq \B_{\delta}(x) \subseteq E$.
        End. [prover eprover]

        $f$ is strictly increasing on $\B_{\delta}(x)$ or $f$ is strictly decreasing on $\B_{\delta}(x)$.
            Indeed $f$ is strictly monotone on $\B_{\delta}(x)$ (by subset monotone lemma).
    qed.
\end{proof}

\end{forthel}

In lemma mon 3 we want to show that $\inv{(f \upharpoonright E)}$ is locally continuous in $f(a)$. From lemma mon 1 we already know that $\inv{(f \upharpoonright E)}$ exists. Thus we can take a real map extension thereof and want to show that $\lim_{x \rightarrow f(a)} h(x) = h(f(a)) = a$.
We assume $f$ to be strictly increasing as the strictly decreasing case is again analogous. Since $E$ is open, we can take a positive real number $d$ such that $[a - d, a + d] \subseteq E$ and $d < \varepsilon$ for any positive $\varepsilon$. Because $f(a - d) < f(a) < f(a + d)$ we can take 
$\delta = \min\{f(a) - f(a - d), f(a + d) - f(a)\}$ (or rather $\delta < f(a) - f(a - d)$ and $\delta < f(a + d) - f(a)$, as we did not define any minimum function). Then $\delta$ is a domain restrictor for $\varepsilon$ and $h$ around $f(a)$ and $a$. Since $f$ is invertible,
\begin{align*}
    \text{if } & f[\B_{d}(a)] = f[(a - d, a + d)] = (f(a - d), f(a + d)) \supseteq \B_{\delta}(f(a)), \\
    \text{then } & \inv{(f \upharpoonright E)}[\B_{\delta}(f(a))] \subseteq \B_{d}(a) \subseteq \B_{\varepsilon}(a).
\end{align*}
Formally it takes some time to convince the ATP to verify this, but this is how the proof works.

\begin{forthel}
 
\begin{lemma}[lemma mon 3]
    Let $E$ be an open neighbourhood of $a$.
    Let $f$ be strictly monotone on $E$ and continuous.
    $\inv{(f \upharpoonright E)}$ is a real map on $f[E]$ and
    $\inv{(f \upharpoonright E)}$ is locally continuous in $f(a)$.
\end{lemma}
\begin{proof}
    Take $b = f(a)$.
    (0) $f \upharpoonright E$ is an injective map. $f[E]$ is a subset of $\mathbb{R}$.

    Take $g = \inv{(f \upharpoonright E)}$.

    (A) $g$ is a real map on $f[E]$. Indeed $\range(\inv{(f \upharpoonright E)}) \subseteq \mathbb{R}$ and $\dom(g) = f[E]$.

    (B) Let $h$ be a real map extension of $g$.
[prover vampire]
    $\limS{h}{b} = h(b)$.
    Indeed we can show that for every positive real number $\varepsilon$ there exists a domain restrictor
            for $\varepsilon$ and $h$ around $b$ and $h(b)$.
        Let $\varepsilon$ be a positive real number.

        Take a positive real number $\theta$ such that $\B_{\theta}(a) \subseteq E$.
        Take a positive real number $d$ such that $d < \varepsilon$ and $d < \theta$.

        (C) $[a - d, a + d]_{CI} \subseteq E$. Proof.
                $[a - d, a + d]_{CI} \subseteq (a - \theta, a + \theta)_{OI}$. Proof.
                    Let $x \in [a - d, a + d]_{CI}$. Then $x = a - d$ or $x = a + d$ or $x \in (a - d, a + d)_{OI}$.
                    $(a - d, a + d)_{OI} \subseteq (a - \theta, a + \theta)_{OI}$ (by open interval lemma). 
                        Indeed $a - \theta \leq a - d$ and $a + d \leq a + \theta$.
                    $a - d \in (a - \theta, a + \theta)_{OI}$ and $a + d \in (a - \theta, a + \theta)_{OI}$.
                        Indeed $a - \theta < a - d$ and $a + d < a + \theta$. Indeed $-\theta < -d$.
                End.
                $(a - \theta, a + \theta)_{OI}  = \B_{\theta}(a)$ (by ball and open interval).
                $\B_{\theta}(a) \subseteq E$.
            End.
        Take $\beta = f(a - d)$, $B = f(a + d)$.

        (1) Case $f$ is strictly increasing on $E$. 
            $b - \beta, B - b$ are positive real numbers. Proof. 
                $a-d, a, a+d \in E$ (by A). Indeed $a-d, a, a+d$ are elements in $[a-d, a+d]_{CI}$. $a-d < a < a + d$. 
                Therefore $f(a-d) < f(a) < f(a+d)$ (by 1). Thus $\beta < b < B$.
            End.
            Take a positive real number $\delta$ such that $\delta < b - \beta$ and $\delta < B - b$ (by ineq lemma 19).

                [timelimit 10]
            Let us show that $\delta$ is a domain restrictor for $\varepsilon$ and $h$ around $b$ and $h(b)$
                    (by domain restrictor lemma 2).
                Let $\upsilon \in \mathbb{R} \setminus \sset{b}{\mathbb{R}}$. Assume $\abs{\upsilon - b} < \delta$. 
                
                $f[(a - d, a + d)_{OI}] = (f(a - d), f(a + d))_{OI} = (\beta,B)_{OI}$ (by cor I intermediate value theorem). Proof. 
                    $f$ is a continuous real map such that $f$ is strictly increasing on $[a - d, a + d]_{CI}$.
                    $a - d, a + d \in [a - d, a + d]_{CI}$. $[a - d, a + d]_{CI}$ is an interval (by interval lemma).
                End.
                   
                $\upsilon \in (\beta,B)_{OI}$. Proof. 
                    $\upsilon \in \B_{\delta}(b)$ (by lemma 01).
                    $\B_{\delta}(b) = (b - \delta, b + \delta)_{OI}$ (by ball and open interval).
                    $(b - \delta, b + \delta)_{OI} \subseteq (\beta, B)_{OI}$ (by open interval lemma). 
                        Indeed $b - \delta \geq \beta$ and $b + \delta \leq B$.
                End. 
                Take $y \in (a - d, a + d)_{OI}$ such that $f(y) = \upsilon$.
                
                $\abs{y - a} < d$ (by lemma 01). Indeed $y \in \B_{d}(a)$ (by ball and open interval).

                $h(\upsilon) = y$. Proof.
                    $\abs{y - a} < d < \theta$. $y \in E$. $\upsilon \in f[E] = \dom(g)$.
                    $h(\upsilon) = g(\upsilon)$.
                    $\inv{(f \upharpoonright E)}(\upsilon) = y$ (by inverse lemma, 0).
                    Indeed $y \in \dom(f\upharpoonright E)$ and $(f \upharpoonright E)(y) = \upsilon$.
                End.

                $h(b) = a$. Indeed $b \in \dom(g)$ and $g(b) = a$ (by inverse lemma, 0).
                Indeed $a \in \dom(f \upharpoonright E)$ and $(f \upharpoonright E)(a) = b$.
                Thus $\abs{h(\upsilon) - h(b)} = \abs{y - a} < d < \varepsilon$. 
                Therefore $\abs{h(\upsilon) - h(b)} < \varepsilon$.
            End. 
        qed. [timelimit 10]


        (2) Case $f$ is strictly decreasing on $E$.
            $\beta - b, b - B$ are positive real numbers. Proof. 
                $a-d, a, a+d \in E$ (by A). Indeed $a-d, a, a+d$ are elements in $[a-d, a+d]_{CI}$. $a-d < a < a + d$. 
                Therefore $f(a+d) < f(a) < f(a-d)$ (by 2). Thus $B < b < \beta$.
            End.
            Take a positive real number $\delta$ such that $\delta < \beta - b$ and $\delta < b - B$ (by ineq lemma 19).

            Let us show that $\delta$ is a domain restrictor for $\varepsilon$ and $h$ around $b$ and $h(b)$
                    (by domain restrictor lemma 2). 
                Let $\upsilon \in \mathbb{R} \setminus \sset{b}{\mathbb{R}}$. Assume $\abs{\upsilon - b} < \delta$.

                $f[(a-d, a+d)_{OI}] = (f(a + d),f(a - d))_{OI} = (B,\beta)_{OI}$ (by cor II intermediate value theorem). Proof.
                    $f$ is a continuous real map such that $f$ is strictly decreasing on $[a - d, a + d]_{CI}$ (by C, 2).
                    $a - d, a + d$ are elements in $[a - d, a + d]_{CI}$. $[a - d, a + d]_{CI}$ is an interval (by interval lemma).
                End.
                
                $\upsilon \in (B,\beta)_{OI}$. Proof.
                    $\upsilon \in \B_{\delta}(b)$ (by lemma 01).
                    $\B_{\delta}(b) = (b - \delta, b + \delta)_{OI}$ (by ball and open interval).
                    $(b - \delta, b + \delta)_{OI} \subseteq (B, \beta)_{OI}$ (by open interval lemma).
                        Indeed $b - \delta \geq B$ and $b + \delta \leq \beta$.
                End.
                Take $y \in (a - d, a + d)_{OI}$ such that $f(y) = \upsilon$.
                
                $\abs{y - a} < d$ (by lemma 01). Indeed $y \in \B_{d}(a)$ (by ball and open interval).

                $h(\upsilon) = y$. Proof.
                    $\abs{y - a} < d < \theta$. $y \in E$. $\upsilon \in f[E] = \dom(g)$.
                    $h(\upsilon) = g(\upsilon)$. 
                    $\inv{(f \upharpoonright E)}(\upsilon) = y$ (by inverse lemma, 0).
                    Indeed $y \in \dom(f \upharpoonright E)$ and $(f \upharpoonright E)(y) = \upsilon$. 
                End.

                $b \in \dom(g)$.                    
                $h(b) = a$. Indeed $g(b) = a$ (by inverse lemma, 0).
                Indeed $a \in \dom(f \upharpoonright E)$ and $(f \upharpoonright E)(a) = b$. 
                $\abs{h(\upsilon) - h(b)} = \abs{y - a} < d < \varepsilon$. 
                Therefore $\abs{h(\upsilon) - h(b)} < \varepsilon$.
            End.
        qed.
    End. [timelimit 3]
\end{proof}

\end{forthel}

In the next lemma we want to show if $f$ is continuously differentiable and $f'(a) \neq 0$ then there also exists such an open neighborhood $E$ of $a$, where $f$ is strictly monotone. Let us just look at the case of $f'(a)$ being positive. Then we can Take a positive real number $\delta$ such that $\delta$ is a domain restrictor such that 
$$\forall x \in \B_{\delta}(a):\ \abs{f'(x) - f'(a)} < f'(a).$$
Then for all those $x$ $f'(x) > 0$, because if not, then $\abs{f'(x) - f'(a)} = f'(a) - f'(x) \geq f'(a)$. Therefore $f$ is strictly increasing on the neighborhood $\B_{\delta}(a)$ of $a$.

\begin{forthel}


Let $f$ is continuously differentiable stand for $f$ is differentiable and $\deriv{f}$ is continuous. [prover eprover]
 [timelimit 10]
\begin{lemma}[lemma mon 4]
    Let $f$ be continuously differentiable and $\derivin{f}(a) \neq 0$.
    Then there exists an open neighbourhood $E$ of $a$ such that $f$ is strictly monotone on $E$.
\end{lemma}
\begin{proof}
    Take $\varepsilon = \abs{\derivin{f}(a)}$. [prover vampire][timelimit 50]
    Take a positive real number $\delta$ such that $\delta$ is a domain restrictor for $\varepsilon$ and $\deriv{f}$ around $a$ and $\deriv{f}(a)$.
        Indeed $\deriv{f}$ is a real map and $\deriv{f}$ is continuous in $a$ and $\varepsilon$ is a positive real number. [prover eprover][timelimit 10]
    $\B_{\delta}(a)$ is an open neighbourhood of $a$ (by ball is open). Indeed $a \in \B_{\delta}(a)$. Take $B = \B_{\delta}(a)$.

    Let us show that $f$ is strictly monotone on $B$.

        Case $\derivin{f}(a) > 0$. Then $f$ is strictly increasing on $B$ (by derivative strictly increasing, ball and open interval, interval lemma).
            Indeed we can show that $\derivin{f}(x) > 0$ for any $x \in B$.

                Let $x \in B \setminus \sset{a}{\mathbb{R}}$. Assume $\derivin{f}(x) \leq 0$.
                Then $\abs{\derivin{f}(x) - \derivin{f}(a)} 
                = \abs{\derivin{f}(a) - \derivin{f}(x)}
                = \derivin{f}(a) - \derivin{f}(x)
                \geq \derivin{f}(a) = \varepsilon$. [prover vampire][timelimit 30]

                $\delta$ is not a domain restrictor for $\varepsilon$ and $\deriv{f}$ around $a$ and $\deriv{f}(a)$ (by domain restrictor lemma 2).
                    Indeed $x \in \mathbb{R} \setminus \sset{a}{\mathbb{R}}$ and 
                    $\abs{x - a} < \delta$ and $\abs{\deriv{f}(x) - \deriv{f}(a)} \geq \varepsilon$.
                Contradiction. [timelimit 3]
            End.
        End.

        Case $\derivin{f}(a) < 0$. Then $f$ is strictly decreasing on $B$ (by derivative strictly decreasing, ball and open interval, interval lemma).
            Indeed we can show that $\derivin{f}(x) < 0$ for any $x \in B$.

                Let $x \in B \setminus \sset{a}{\mathbb{R}}$. Assume $\derivin{f}(x) \geq 0$.
                Then $\abs{\derivin{f}(x) - \derivin{f}(a)}
                = \derivin{f}(x) - \derivin{f}(a)
                \geq -\derivin{f}(a) = \varepsilon$. [timelimit 30][prover vampire]

                $\delta$ is not a domain restrictor for $\varepsilon$ and $\deriv{f}$ around $a$ and $\deriv{f}(a)$ (by domain restrictor lemma 2).
                Indeed $x \in \mathbb{R} \setminus \sset{a}{\mathbb{R}}$ and $\abs{x - a} < \delta$ and $\abs{\deriv{f}(x) - \deriv{f}(a)} \geq \varepsilon$.
                Contradiction.
            End.
        End.
    qed. [prover eprover]
\end{proof}

\end{forthel}

So we want to show that $\inv{(f \upharpoonright E)}$ is locally differentiable in $f(a)$ for some neighborhood $E$ of $a$.
By lemma mon 4 we can find a neighborhood $E = \B_{d}(a)$ of $a$, where $f$ is strictly monotone on.
By lemma mon 1 we know then that $f \upharpoonright E$ is injective and therefore $\inv{(f \upharpoonright E)}$ exists.

Now we define $$ext(y) = \begin{cases}
    \inv{(f \upharpoonright E)}(y)    & : y \in f[E] \\
    a                               & : y \notin f[E]
\end{cases}$$
for all real numbers $y$, which makes $ext$ a real map extension of $\inv{(f \upharpoonright E)}$.

By lemma mon 2 we know that $f[E]$ is actually an open environment of $f(a)$. Thus the idea is to utilize the open set differentiation extension theorem from the section Locality, which says that if we have a differentiable real map extension of a partial function defined on an open set, the partial function is locally differentiable. So what is left to show is that $ext$ is differentiable, which we do by showing the existence of a difference quotient map. For that we define 
$$\psi(y) = \begin{cases}
    1/\varphi(\inv{(f \upharpoonright E)}(y)) & : y \in f[E] \\
    0                                       & : y \notin f[E]
\end{cases}$$
for any real number $y$, where $\varphi$ is the difference quotient map of $f$ in $a$; $\varphi$ exists, because $f$ is differentiable in $a$.
$\psi$ is a difference quotient map for $ext$ in $f(a)$ if it satisfies 
\begin{enumerate}
    \item[(1)] $ext(y) = (y - f(a)) \psi(y) + ext(f(a))$ for all $y$. 
    \item[(2)] $\psi$ is continuous in $f(a)$
\end{enumerate}
(1) we can show by case distinction with $y \in f[E]$ and $y \notin f[E]$ and then some simple calculation. For (2) we need lemma mon 3 though, which says that $\inv{(f \upharpoonright E)}$ is locally continuous in $f(a)$. Furthermore $\varphi$ is continuous in $a$ and therefore by the continuity of composition and the continuity of a division map is $\psi$ also continuous in $f(a)$.

Thus $ext$ is differentiable in $f(a)$ and since $\dom(\inv{(f \upharpoonright E)}) = f[E]$ is an open set, $\inv{(f \upharpoonright E)}$ is locally differentiable in $f(a)$ by open set differentiation extension and 
$$\lderivin{\inv{(f \upharpoonright E)}}(f(a)) = \psi(f(a)) = 1/\varphi(\inv{(f \upharpoonright E)}(f(a))) = 1/f'(a).$$

\begin{forthel}

\begin{theorem}[inverse function theorem]
    Let $f$ be continuously differentiable and $\derivin{f}(a) \neq 0$.
    Then there exists an open neighbourhood $E$ of $a$ such that $f$ is strictly monotone on $E$ and $\inv{(f \upharpoonright E)}$ is a real map on $f[E]$ and
    $\inv{(f \upharpoonright E)}$ is locally differentiable in $f(a)$ and $\lderivin{\inv{(f \upharpoonright E)}}(f(a)) = 1/\derivin{f}(a)$.
\end{theorem}
\begin{proof}
    Take an open neighbourhood $N$ of $a$ such that $f$ is strictly monotone on $N$ (by lemma mon 4).
    Take a positive real number $d$ such that $\B_{d}(a) \subseteq N$.
    Take a positive real number $\delta$ such that $\delta < d$.
    Take $E = \B_{\delta}(a)$.

    (0) $f$ is strictly monotone on $E$ (by subset monotone lemma). 
        Indeed $E$ is a subset of $N$. 
        Indeed $E = \B_{\delta}(a) \subseteq \B_{d}(a) \subseteq N$ (by ball subsets).

    $f \upharpoonright E$ is injective (by lemma mon 1, 0). Indeed $E$ is a subset of $\mathbb{R}$.
    Thus there exists $\inv{(f \upharpoonright E)}$.

    Take $g = f \upharpoonright E$. $g$ is a real map on $E$.
    Take $D = \range(g)$. $D$ is an open neighbourhood of $f(a)$ (by lemma mon 2, 0).
    Indeed $f$ is continuous and $D = f[E]$ and $E$ is an open neighbourhood of $a$.
    $\inv{g}$ is a real map on $D$.
    
    (3) Take a difference quotient map $\varphi$ of $f$ in $a$.
    For all $y \in D$ there exists an $x \in E$ such that $x = \inv{g}(y)$.

    Define $$ext(y) = \begin{cases}
        \inv{g}(y)    & : y \in D \\
        a             & : y \notin D
    \end{cases}$$
    for $y \in \mathbb{R}$. $ext$ is a real map extension of $\inv{g}$. 
    Indeed $ext(y)$ is a real number for any real number $y$. 
    Indeed $a$ is a real number and $\inv{g}(y)$ is a real number for every $y \in D$. 
 
    For all $y \in D$ $\varphi(\inv{g}(y)) \neq 0$. Proof. 
        $\range(\inv{g}) \subseteq E$. Indeed $E = \dom(g)$.
        We can show that for all $x \in E$ $\varphi(x) \neq 0$.
            Let $x \in E$.

            Case $x = a$. Trivial.

            Case $x \neq a$.
                $\varphi(x) = \Diffquot{f}{a}(x)$ (by prop difference quotient map, 3).
                    Indeed $f$ is differentiable in $a$ and $x \in \mathbb{R} \setminus \sset{a}{\mathbb{R}}$. 
                Therefore $x \neq a$ and $x - a$ is nonzero.
                $\Diffquot{f}{a}(x) = \frac{f(x) - f(a)}{x - a}$ (by def diffquot).
                (4) $f(x) - f(a)$ is nonzero. Indeed $x \neq a$ and $x,a \in E$ and $f(x) \neq f(a)$.
                $\frac{f(x) - f(a)}{x - a} \neq 0$ (by frac lemma 5,4). Indeed $x-a$ is nonzero.
            End.
        End.
    End.
  
    Define $$\psi(y) = \begin{cases}
        1/\varphi(\inv{g}(y))   & : y \in D \\
        0                       & : y \notin D
    \end{cases}$$
    for $y \in \mathbb{R}$. $\psi$ is a real map. Indeed $1/\varphi(\inv{g}(y))$ is a real number for any $y \in D$.

    Let us show that $\psi$ is a difference quotient map of $ext$ in $f(a)$.

        Let us show that for every real number $y$ $ext(y) = ((y - f(a)) \cdot \psi(y)) + ext(f(a))$.
            Let $y$ be a real number. 

            (A) $ext(f(a)) = a$.
            $((y - f(a)) \cdot \psi(y)) + ext(f(a)) .= ((y - f(a)) \cdot \psi(y)) + a$ (by A).

            Case $y = f(a)$. 
                $((f(a) - f(a)) \cdot \psi(y)) + a .= a$ (by additive inverse, eq lemma 9, add comm, existence zero)
                $.= ext(f(a))$ (by A).
            End. 

            Case $y \in D$ and $y \neq f(a)$.

                (B) $\psi(y) = \frac{\inv{g}(y) - a}{y - f(a)}$. Proof.
                    $\varphi(\inv{g}(y)) = \Diffquot{f}{a}(\inv{g}(y))$ (by prop difference quotient map,3).
                        Indeed $f$ is differentiable in $\inv{g}(y)$ and $\inv{g}(y) \in \mathbb{R} \setminus \sset{a}{\mathbb{R}}$.
                    $\inv{g}(y) \neq a$ and $\inv{g}(y) - a$ is nonzero.
                    $\Diffquot{f}{a}(\inv{g}(y)) .= \frac{f(\inv{g}(y)) - f(a)}{\inv{g}(y) - a}$ (by def diffquot).
                    $f(\inv{g}(y)) - f(a) \neq 0$.
                    $1/\frac{f(\inv{g}(y)) - f(a)}{\inv{g}(y) - a}.= \frac{\inv{g}(y) - a}{f(\inv{g}(y)) - f(a)}$ (by eq lemma 22).
                    $\psi(y) = 1 / \varphi(\inv{g}(y)) = \frac{\inv{g}(y) - a}{y - f(a)}$.
                End.

                $((y - f(a)) \cdot \psi(y)) + a .= ((y - f(a)) \cdot \frac{\inv{g}(y) - a}{y - f(a)}) + a$ (by B)
                $.= (\inv{g}(y) - a) + a$ (by fraction lemma 2)
                $.= \inv{g}(y)$ (by add ass, add comm, additive inverse, existence zero).

                $ext(y) = \inv{g}(y)$.
            End.

            Case $y \notin D$. 
                (C) $\psi(y) = 0$.
                $((y - f(a)) \cdot \psi(y)) + a .= a$ (by mult comm, C, eq lemma 9, add comm, existence zero).
                $ext(y) = a$.
            End.
        End.

        Let us show that $\psi$ is continuous in $f(a)$.

            (4) $E$ is an open neighbourhood of $a$. Proof.
                $a$ is an element of $E$.
                $E$ is an open set (by open interval is open).
                    Indeed $E = (a - \delta, a + \delta)_{OI}$ (by ball and open interval).
            End.

            $\inv{g}$ is locally continuous in $f(a)$ (by lemma mon 3, 4, 0). Indeed $f$ is continuous.
            $ext$ is continuous in $f(a)$. 
            $\varphi$ is continuous in $a$.
            [timelimit 50]
            Thus $\varphi \circ ext$ is continuous in $f(a)$ (by continuity of composition). Indeed $\inv{g}(f(a)) = a$.
            $\psi$ is continuous in $f(a)$ (by continuity division map). Proof.
                $\varphi \circ ext$ is a real map. $\psi$ is a division map of $\varphi \circ ext$ in $f(a)$ (by 4). 
            End. 
        End.
    End.
 
    Therefore $ext$ is differentiable in $f(a)$.
    $\dom(\inv{g}) = D$ and $D$ is an open neighbourhood of $f(a)$.
        $\inv{g}$ is a real map on $D$. Indeed $\range(g) = D$ and for every $y \in D$ $\inv{g}(y)$ is a real number.
    $\lderivin{\inv{g}}(f(a)) = \derivin{ext}(f(a))$ (by open set differentiation extension). Indeed
        $ext$ is a real map extension of $\inv{g}$ such that $ext$ is differentiable in $f(a)$.
[timelimit 3]
$\derivin{ext}(f(a)) = \psi(f(a)) = 1/\varphi(\inv{g}(f(a))) = 1/\varphi(a) = 1/\derivin{f}(a)$.
$\inv{(f \upharpoonright E)}$ is locally differentiable in $f(a)$. $\lderivin{\inv{(f \upharpoonright E)}}(f(a)) = 1/\derivin{f}(a)$.
 $f \upharpoonright E$ is injective. $\inv{(f \upharpoonright E)}$ is a real map on $f[E]$. Indeed $\range(f \upharpoonright E) = f[E]$.
\end{proof} [timelimit 3]

\begin{definition}
    A locally inverse function to $f$ in $a$ is a real map $g$ such that for some open neighbourhood $E$ of $a$ 
    $f \upharpoonright E$ is injective and $g \upharpoonright f[E] = \inv{(f \upharpoonright E)}$.
\end{definition}

\begin{corollary}[inverse function theorem II]
    Let $f$ be continuously differentiable and $\derivin{f}(a) \neq 0$.
    Let $g$ be a locally inverse function to $f$ in $a$.
    Then $g$ is differentiable in $f(a)$ and $\derivin{g}(f(a)) = 1/\derivin{f}(a)$.
\end{corollary} 
\begin{proof}
    Take an open neighbourhood $M$ of $a$ such that 
        $f \upharpoonright M$ is injective and $g \upharpoonright f[M] = \inv{(f \upharpoonright M)}$.
    [timelimit 100]
    Take an open neighbourhood $N$ of $a$ such that $f$ is strictly monotone on $N$
    and $\inv{(f \upharpoonright N)}$ is a real map on $f[N]$ and
    $\inv{(f \upharpoonright N)}$ is locally differentiable in $f(a)$ and 
    $\lderivin{\inv{(f \upharpoonright N)}}(f(a)) = 1/\derivin{f}(a)$ (by inverse function theorem). 
    Take $E = N \cap M$.
    
    $E$ is an open neighbourhood of $a$. 
    Indeed $a$ is an element of $E$ and $E$ is an open set (by intersection of open sets). 

    $f \upharpoonright E$ is injective. $\inv{(f \upharpoonright E)}$ is a real map on $f[E]$.
        Indeed $\dom(\inv{(f \upharpoonright E)}) = \range(f \upharpoonright E) = f[E]$ and 
        for every $y \in f[E]$ $\inv{(f \upharpoonright E)}(y)$ is a real number.
    Define $$ext(y) = \begin{cases}
        \inv{(f \upharpoonright N)}(y)  & : y \in f[N] \\
        0                               & : y \notin f[N]
    \end{cases}$$
    for $y \in \mathbb{R}$. $ext$ is a real map. Indeed $\range(ext)$ is a subset of $\mathbb{R}$.
    Indeed for any $y \in \dom(ext)$ $ext(y)$ is a real number.
    Indeed 0 is a real number and $\inv{(f \upharpoonright N)}(y)$ is a real number for any $y \in f[N]$.
    $\inv{(f \upharpoonright N)}$ is a real map on $f[N]$.

        $f[N]$ is a subset of $\mathbb{R}$. Take $\gamma = \inv{(f \upharpoonright N)}$.
        $\gamma$ is a real map on $f[N]$. $f(a)$ is an element of $f[N]$.
        $\gamma$ is locally differentiable in $f(a)$. [prover vampire]
    $\lderivin{\gamma}(f(a)) = \derivin{ext}(f(a))$ (by lemma differentiation extension).
        Indeed $ext$ is a real map extension of $\gamma$. [prover eprover]
    Thus $\lderivin{\inv{(f \upharpoonright N)}}(f(a)) = \derivin{ext}(f(a))$.

    Take $h = \inv{(f \upharpoonright E)}$.
    $h$ is locally differentiable in $f(a)$ and
    $\derivin{ext}(f(a)) = \lderivin{h}(f(a))$ (by open set differentiation extension). Proof.
        $h$ is a real map on $f[E]$.
        $f[E]$ is an open neighbourhood of $f(a)$ (by lemma mon 2, 1). Proof. 
            $f$ is strictly monotone on $E$ (by subset monotone lemma). Indeed $E$ is a subset of $N$.
            $f$ is continuous.
        End. 
        
        $ext$ is a real map extension of $h$ such that $ext$ is differentiable in $f(a)$.
        Indeed we can show that for every $y \in f[E]$ $\inv{(f \upharpoonright E)}(y) = ext(y)$.
            Let $y \in f[E]$. Then $y \in f[N]$. $h(y) = \inv{(f \upharpoonright N)}(y)$.
            $\inv{(f \upharpoonright N)}(y) = ext(y)$.
        End.
    End.

    $\lderivin{\inv{(f \upharpoonright E)}}(f(a)) = \derivin{g}(f(a))$ (by lemma differentiation extension). Proof.
        $f[E]$ is a subset of $\mathbb{R}$.
        $\inv{(f \upharpoonright E)}$ is a real map on $f[E]$. $f(a)$ is an element of $f[E]$.
        $\inv{(f \upharpoonright E)}$ is locally differentiable in $f(a)$. 
        $g$ is a real map extension of $\inv{(f \upharpoonright E)}$. Proof. 
            $f \upharpoonright M$ is injective. $\inv{(f \upharpoonright M)}$ is a real map on $f[M]$. $f[E] \subseteq f[M]$.
            For every $y \in f[E]$ $\inv{(f \upharpoonright E)}(y) = g(y)$. Proof.
                Assume the contrary. Take $y \in f[E]$ such that $\inv{(f \upharpoonright E)}(y) \neq g(y)$.
                Take $x = \inv{(f \upharpoonright E)}(y)$. Then $(f \upharpoonright E)(x) = y$. Thus $f(x) = y$. $x \in M$. Indeed $E \subseteq M$.
                Hence $(f \upharpoonright M)(x) = y$. Thus $\inv{(f \upharpoonright M)}(y) = x$. Indeed $y \in f[M]$.  
                $\inv{(f \upharpoonright E)}(y) = \inv{(f \upharpoonright M)}(y)$.
                $\inv{(f \upharpoonright M)}(y) \neq g(y)$. Contradiction.
            End.
        End.
    End.
\end{proof} 
[timelimit 3]
\end{forthel}

\end{document}
