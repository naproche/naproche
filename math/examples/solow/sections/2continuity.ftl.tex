

\documentclass[../solow_main.tex]{subfiles}
\begin{document}

\chapter{Continuity}\label{chapter:continuity}


\begin{forthel}
    [read \path{examples/solow/sections/1metrics.ftl.tex}][prover eprover]
\end{forthel}

In this chapter we introduce the concept of continuity and prove some common notions such as that 
linear functions are continuous and if $f$ and $g$ are continuous, then
so are $\fplus{f}{g}$, $\fmult{f}{g}$, and $(f \circ g)$.



\section{Definition of Continuity}

The typical definition of continuity is the famous $\varepsilon$-$\delta$ criterion.
$f: \mathbb{R} \rightarrow \mathbb{R}$ is continuous in $a$ iff
$$\forall \varepsilon > 0 \exists \delta > 0 \forall x \in \mathbb{R}:\ \abs{x - a} < \delta \implies \abs{f(x) - f(a)} < \varepsilon.$$
This definition is overwhelms, however, the type verification system. Instead we will use Skolem-functions 
to break this definitions apart.

Thus we introduce the term "domain restrictor" referring to the $\delta = \delta(\varepsilon)$ such that 
$$\forall x \in \mathbb{R} \setminus \sset{a}{\mathbb{R}}:\ \abs{x - a} < \delta \implies \abs{f(x) - f(a)} < \varepsilon.$$
We exclude $a$ from the set of real numbers, for which it should hold, because we want to reuse the concept of
domain restrictor in the following chapter for our definition of differentiation. 
There we will have that $$\abs{\frac{f(x) - f(a)}{x - a} - f'(a)} < \varepsilon,$$
where the option of $x$ being equal to $a$ would result in a division by zero error.

We define $f$ being continuous in $a$ as $\limS{f}{a} = f(a)$, where $\limS{f}{a}$ is a real number, for which 
we can find a domain restrictor for every positive real number $\varepsilon$.

Furthermore, we introduce a more general notion of domain restrictor, which would allow to define continuity more 
generally for any real-valued function and not just those with a domain of $\mathbb{R}$. We do show that our 
specific definition for $\mathbb{R}$ is equivalent to the more general one when it is applied to $\mathbb{R}$.
The choice to define continuity only on $\mathbb{R}$ is, however, a very deliberate one, as this is the case
that we need for almost all the following theorems that we are interested in. In order to make our definition more
focussed, we reduce the complexity for the theorem prover.

\begin{forthel}

Let $a,x,y$ denote real numbers. Let $\delta,\varepsilon$ denote positive real numbers.
Let $f$ denote a real maps.

\begin{definition}[def domain restrictor]
    A domain restrictor for $\varepsilon$ and $f$ around $a$ and $y$ is a positive real number $\delta$ such that
    for all $x \in \mathbb{R} \setminus \sset{a}{\mathbb{R}}$ 
    if $\Rmetric(x,a) < \delta$ then $\Rmetric(f(x),y) < \varepsilon$.
\end{definition}

\begin{lemma}[domain restrictor lemma]
    Let $\delta$ be a domain restrictor for $\varepsilon$ and $f$ around $a$ and $y$ and 
        $d$ be a positive real number such that $d < \delta$.
    Then $d$ is a domain restrictor for $\varepsilon$ and $f$ around $a$ and $y$.
\end{lemma}
\begin{proof}
    Let us show that for all $x \in \mathbb{R} \setminus \sset{a}{\mathbb{R}}$ 
            if $\Rmetric(x,a) < d$ then $\Rmetric(f(x),y) < \varepsilon$.
        Let $x \in \mathbb{R} \setminus \sset{a}{\mathbb{R}}$. Assume $\Rmetric(x,a) < d$. $d < \delta$.
        Thus $\Rmetric(x,a) < \delta$. Therefore $\Rmetric(f(x),y) < \varepsilon$.
    End.
\end{proof}
[timelimit 100]
\begin{lemma}[domain restrictor lemma 2]
    If there exists $x \in \mathbb{R} \setminus \sset{a}{\mathbb{R}}$ such that 
    $\abs{x - a} < \delta$ and $\abs{f(x) - y} \geq \varepsilon$ then 
    $\delta$ is not a domain restrictor for $\varepsilon$ and $f$ around $a$ and $y$.
\end{lemma}
[timelimit 10]
\begin{lemma}[domain restrictor lemma 3]
    Let $\varepsilon$ be a positive real number.
    Let $\delta$ be a domain restrictor for $\varepsilon$ and $f$ around $a$ and $f(a)$.
    If $\abs{x - a} < \delta$ then $\abs{f(x) - f(a)} < \varepsilon$.
\end{lemma}
\begin{proof}
    Case $x = a$. Trivial. 

    Case $x \neq a$. Then  $x \in \mathbb{R} \setminus \sset{a}{\mathbb{R}}$. 
        Assume $\abs{x - a} < \delta$. $(x,a),(f(x),f(a)) \in \dom(\Rmetric)$.
        Then $\Rmetric(x,a) < \delta$. $\abs{f(x) - f(a)} = \Rmetric(f(x),f(a)) < \varepsilon$.
    End.
\end{proof} 

\begin{definition}
    $\limS{f}{a}$ is a real number $y$ such that for every positive real number $\varepsilon$ there exists a domain restrictor
    for $\varepsilon$ and $f$ around $a$ and $y$.
\end{definition}

\begin{definition}
    $f$ is continuous in $a$ iff $\limS{f}{a} = f(a)$.
\end{definition}

\begin{definition}
    $f$ is continuous iff for every real number $a$ $f$ is continuous in $a$.
\end{definition}

\end{forthel}



\section{Continuity in the Real Numbers}

\subsection{Linear Function}

\begin{forthel}
Let $a,b$ denote real numbers.
\begin{definition}[linear function]
    $\flin{a}{b}$ is a real map such that $\flin{a}{b}(x) = (a \cdot x) + b$ for all real numbers $x$.
\end{definition}


\begin{lemma}[linear lem 1]
    $\flin{0}{a}(b) = a$.
\end{lemma}

\begin{lemma}[linear lem 2]
    $\flin{a}{0}(b) = a \cdot b$.
\end{lemma}

\begin{lemma}[linear lem 3]
    $\flin{a}{b}(0) = b$.
\end{lemma}

\end{forthel}

To prove that a function $f(x) = a \cdot x + b$ is continuous, we let $\varepsilon$ be a positive real number and 
take $\delta = \frac{\varepsilon}{\abs{a}}$. Then for all $x$ with a distance smaller $\delta$ to $a$ we know
$$\abs{f(x) - f(a)} = \abs{(a \cdot x + b) - (a \cdot a + b)} = \abs{a} \cdot \abs{x - a} < 
\abs{a} \cdot \frac{\varepsilon}{\abs{a}} = \varepsilon.$$

\begin{forthel}

\begin{theorem}[continuity linear function]
    $\flin{a}{b}$ is continuous.
\end{theorem}
\begin{proof}
    Take $f=\flin{a}{b}$. Let $\xi$ be a real number.
    
    Case $a = 0$. 
    Let $\varepsilon$ be a positive real number.
    For all $x \in \mathbb{R} \setminus \sset{\xi}{\mathbb{R}}$ $\Rmetric(f(x),f(\xi)) = \Rmetric(b,b) = 0 < \varepsilon$. 
    Therefore 1 is a domain restrictor for $\varepsilon$ and $f$ around $\xi$ and $f(\xi)$. End. 
    
    Case $a \neq 0$. 
    Let $\varepsilon$ be a positive real number. 
    Take $\delta = \abs{\frac{\varepsilon}{a}}$. Then $\delta$ is a positive real number.
    
    $\delta$ is a domain restrictor for $\varepsilon$ and $f$ around $\xi$ and $f(\xi)$. Proof.

    Let $x \in \mathbb{R} \setminus \sset{\xi}{\mathbb{R}}$. Assume $\abs{x - \xi} < \delta$.

    $\abs{f(x) - f(\xi)} = \abs{((a \cdot x) + b) - ((a \cdot \xi) + b)}$.

    $\abs{((a \cdot x) + b) - ((a \cdot \xi) + b)}
     .= \abs{((a \cdot x) + (b - b)) - (a \cdot \xi)}$

      (by eq lemma 15, add comm, add ass)

    $.= \abs{(a \cdot x) + (a \cdot -\xi)}$
    
    (by additive inverse, existence zero, mult comm, eq lemma 11)

    $.= \abs{a} \cdot \abs{x - \xi}$ (by distributive law, absolute lemma 2).

    Then $\abs{a} \cdot \abs{x - \xi} < \abs{a} \cdot \delta
    = \abs{a} \cdot \abs{\frac{\varepsilon}{a}} = \varepsilon$. End.
    QED.
\end{proof}

\end{forthel}



\subsection{Continuity Under Addition}

\begin{forthel}
Let $f,g$ denote real maps.

\begin{definition}[def add func]
    $\fplus{f}{g}$ is a real map such that $\fplus{f}{g}(x) = f(x) + g(x)$ for all real numbers $x$.
\end{definition}

\end{forthel}

For the proof of the continuity of $\fplus{f}{g}$ we take an $\delta$ so small that it serves 
as a domain restrictor for both $f$ and $g$ for $\varepsilon / 2$.
Then 
\begin{align*}
    \abs{\fplus{f}{g}(x) - \fplus{f}{g}(a)} & = \abs{(f(x) - f(a)) + (g(x) - g(a))} \\
    & \leq \abs{f(x) - f(a)} + \abs{g(x) - g(a)} < \frac{\varepsilon}{2} + \frac{\varepsilon}{2} = \varepsilon.
\end{align*}

\begin{forthel}
[timelimit 10]
\begin{theorem}[continuity of plus]
    Let $f,g$ be continuous in $a$.
    Then $\fplus{f}{g}$ is continuous in $a$.
\end{theorem}
\begin{proof}
    Let $\varepsilon$ be a positive real number. Take $\epsilon = \frac{\varepsilon}{2}$. Then $\epsilon$ is a positive real number.
  
    Take a positive real number $\tau$ such that $\tau$ is a domain restrictor for $\epsilon$ and $f$ around $a$ and $f(a)$.
    Take a positive real number $\theta$ such that $\theta$ is a domain restrictor for $\epsilon$ and $g$ around $a$ and $g(a)$.
    Take a positive real number $\delta$ such that $\delta < \tau$ and $\delta < \theta$ (by ineq lemma 19).

    We can show that $\delta$ is a domain restrictor for $\varepsilon$ and $\fplus{f}{g}$ around $a$ and $\fplus{f}{g}(a)$.

        Let $x \in \mathbb{R} \setminus \sset{a}{\mathbb{R}}$. Assume $\abs{x - a} < \delta$.

        Let us show that $\abs{\fplus{f}{g}(x) - \fplus{f}{g}(a)} \leq \abs{f(x) - f(a)} + \abs{g(x) - g(a)}$.

            $\abs{\fplus{f}{g}(x) - \fplus{f}{g}(a)} .= \abs{(f(x) + g(x)) - (f(a) + g(a))}$ (by def add func)

            $.= \abs{(f(x) - f(a)) + (g(x) - g(a))}$ 

                (by eq lemma 15, add ass, add comm).

            $\abs{(f(x) - f(a)) + (g(x) - g(a))} \leq \abs{f(x) - f(a)} + \abs{g(x) - g(a)}$ (by absolute lemma 4).
        End.

        $\abs{f(x) - f(a)} + \abs{g(x) - g(a)} < \varepsilon$.
        Proof. 
            $\abs{f(x) - f(a)}$, $\abs{g(x) - g(a)}$ and $\epsilon$ are real numbers.

            (A) $\abs{f(x) - f(a)} < \epsilon$ (by domain restrictor lemma 3). Indeed
            $\delta$ is a domain restrictor for $\epsilon$ and $f$ around $a$ and $f(a)$ (by domain restrictor lemma).

            (B) $\abs{g(x) - g(a)} < \epsilon$ (by domain restrictor lemma 3). Indeed
            $\delta$ is a domain restrictor for $\epsilon$ and $g$ around $a$ and $g(a)$ (by domain restrictor lemma). 

            Therefore $\abs{f(x) - f(a)} + \abs{g(x) - g(a)} < \epsilon + \epsilon$ (by ineq lemma 13, A, B).
            $\epsilon + \epsilon = 2 \cdot \epsilon = \varepsilon$. 
        End.

        Thus $\abs{\fplus{f}{g}(x) - \fplus{f}{g}(a)}  \leq \varepsilon$ (by transitivity).
    QED.
\end{proof}

\begin{corollary}
    Let $f$ be continuous. Let $g$ be continuous.
    Then $\fplus{f}{g}$ is continuous.
\end{corollary}
[timelimit 3]
\end{forthel}



\subsection{Continuity Under Multiplication}

\begin{forthel}

\begin{definition}[def mult func]
    $\fmult{f}{g}$ is a real map such that $\fmult{f}{g}(x) = f(x) \cdot g(x)$ for all real numbers $x$.
\end{definition}

\begin{lemma}[lemma 2]
    $\fmult{f}{g} = \fmult{g}{f}$.
\end{lemma}
\begin{proof}
    We can show that for every real number $x$ $\fmult{f}{g}(x) = \fmult{g}{f}(x)$.
    Let $x$ be a real number.

    $\fmult{f}{g}(x) .= f(x) \cdot g(x) 
     .= g(x) \cdot f(x)$ (by mult comm) 
    $.= \fmult{g}{f}(x)$.
    End.
\end{proof}

\end{forthel}

The proof of the continuity of $\fmult{f}{g}$ is certainly not winning any beauty contests, but it is working.
We Take a positive real number $\delta$ such that $\delta$ is a domain restrictor that is so small that $\abs{f(x) - f(a)} < \frac{\varepsilon}{2 \cdot (\abs{g(a)} + \varepsilon)} =: \nu$,
$\abs{g(x) - g(a)} < \frac{\varepsilon}{2 \cdot (\abs{f(a)} + 1)} =: \pi$, and $\abs{g(x) - g(a)} < \varepsilon$.

In (1) we first show that $\abs{\fmult{f}{g}(x) - \fmult{f}{g}(a)} \leq \abs{f(x) - f(a)} \cdot \abs{g(x)} + \abs{f(a)}\cdot \abs{g(x) - g(a)}$ 
by proving in particular (11).

In (2) we show that $\abs{f(x) - f(a)} \cdot \abs{g(x)} < \varepsilon/2$ using (21),(22), and (23).

In (3) we show that also $\abs{f(a)}\cdot \abs{g(x) - g(a)} < \varepsilon/2$ using (31) and (32).

Altogether we get then that $\abs{\fmult{f}{g}(x) - \fmult{f}{g}(a)} < \varepsilon$.


\begin{forthel}

\begin{theorem}[continuity under multiplication]
    Let $f,g$ be continuous in $a$.
    Then $\fmult{f}{g}$ is continuous in $a$.
\end{theorem}
\begin{proof}
        Let $\varepsilon$ be a positive real number.


        $2 \cdot (\abs{g(a)} + \varepsilon)$ is a positive real number.
        Proof. $\abs{g(a)} \geq 0$ and $\varepsilon > 0$. Thus $\abs{g(a)} + \varepsilon > 0$. [timelimit 20]
            Therefore $2 \cdot (\abs{g(a)} + \varepsilon) > 0$. [timelimit 3] End.
        
        (01) Take $\nu = \frac{\varepsilon}{2 \cdot (\abs{g(a)} + \varepsilon)}$. $\nu$ is a positive real number.

        $2 \cdot (\abs{f(a)} + 1)$ is a positive real number.
        Proof. $\abs{f(a)} \geq 0$. [timelimit 20] Therefore $\abs{f(a)} + 1 > 0$. Thus $2 \cdot (\abs{f(a)} + 1) > 0$. End.
        

        (02) Take $\pi = \frac{\varepsilon}{2 \cdot (\abs{f(a)} + 1)}$. $\pi$ is a positive real number.


        Take a positive real number $\tau$ such that $\tau$ is a domain restrictor for $\nu$ and $f$ around $a$ and $f(a)$.
        Take a positive real number $\theta$ such that $\theta$ is a domain restrictor for $\pi$ and $g$ around $a$ and $g(a)$.
        Take a positive real number $\vartheta$ such that $\vartheta$ is a domain restrictor for $\varepsilon$ and $g$ around $a$ and $g(a)$.
        Take a positive real number $\Theta$ such that $\Theta < \theta$ and $\Theta < \vartheta$.
        Take a positive real number $\delta$ such that $\delta < \tau$ and $\delta < \Theta$. [timelimit 3]

        Let us show that
        $\delta$ is a domain restrictor for for $\varepsilon$ and $\fmult{f}{g}$ around $a$ and $\fmult{f}{g}(a)$.
            Let $x \in \mathbb{R} \setminus \sset{a}{\mathbb{R}}$. Then $x$ is a real number. 
            
            Assume $\abs{x - a} < \delta$.


            (1) $\Rmetric(\fmult{f}{g}(x),\fmult{f}{g}(a)) \leq 
                        (\abs{f(x) - f(a)} \cdot \abs{g(x)}) + (\abs{f(a)} \cdot \abs{g(x) - g(a)})$. Proof.

                (11) $(f(x) \cdot g(x)) - (f(a) \cdot g(a)) = ((f(x) - f(a)) \cdot g(x)) + (f(a) \cdot (g(x) - g(a)))$. Indeed

                    $(f(x) \cdot g(x)) - (f(a) \cdot g(a)) $

                    $.= ((f(x) \cdot g(x)) - (f(a) \cdot g(a))) + ((f(a) \cdot g(x)) - (f(a)\cdot g(x)))$

                        (by existence zero, additive inverse)

                    $.= ((f(x) \cdot g(x)) - (f(a) \cdot g(x))) + ((f(a) \cdot g(x)) - (g(a) \cdot f(a)))$ 

                        (by add ass, add comm, mult comm)

                    $.= ((f(x) - f(a)) \cdot g(x)) + (f(a) \cdot (g(x) - g(a)))$ 

                        (by distributive law, eq lemma 1,mult comm, eq lemma 11).


                $\Rmetric(\fmult{f}{g}(x),\fmult{f}{g}(a)) 
                .= \abs{\fmult{f}{g}(x) - \fmult{f}{g}(a)}$ (by Manhattan metric)

                $.= \abs{(f(x) \cdot g(x)) - (f(a) \cdot g(a))}$ (by def mult func)

                $.= \abs{((f(x) - f(a)) \cdot g(x)) + (f(a) \cdot (g(x) - g(a)))}$ (by 11).

                $\abs{((f(x) - f(a)) \cdot g(x)) + (f(a) \cdot (g(x) - g(a)))}
                \leq \abs{(f(x) - f(a)) \cdot g(x)} + \abs{f(a) \cdot (g(x) - g(a))}$ (by absolute lemma 4).

                $\abs{(f(x) - f(a)) \cdot g(x)} + \abs{f(a) \cdot (g(x) - g(a))}$

                $.= (\abs{f(x) - f(a)} \cdot \abs{g(x)}) + (\abs{f(a)} \cdot \abs{g(x) - g(a)})$ (by absolute lemma 2).
            qed.
            
            (2) $\abs{f(x) - f(a)} \cdot \abs{g(x)} < \frac{\varepsilon}{2}$. Proof. [timelimit 20]         

                (21) $\abs{f(x) - f(a)} < \nu$ (by domain restrictor lemma 3). Indeed
                    $\delta$ is a domain restrictor for $\nu$ and $f$ around $a$ and $f(a)$ (by domain restrictor lemma). 
                
            
                (22) $\abs{g(x)} < \abs{g(a)} + \varepsilon$.

                Proof.
                    $\delta$ is a domain restrictor for $\varepsilon$ and $g$ around $a$ and $g(a)$ (by domain restrictor lemma). 
                    Indeed $\delta < \vartheta$.

                    $\abs{g(x)} - \abs{g(a)} \leq \abs{g(x) - g(a)} = \Rmetric(g(x),g(a)) < \varepsilon$. 

                    Therefore $\abs{g(x)} < \varepsilon + \abs{g(a)}$ (by ineq lemma 15). Indeed $\abs{g(x)} - \abs{g(a)} < \varepsilon$.
                End.[timelimit 3]

                $\abs{f(x) - f(a)} \cdot \abs{g(x)} < \nu \cdot (\abs{g(a)} + \varepsilon)$. Proof.
                    Take $r = \abs{f(x) - f(a)}, \rho = \abs{g(x)}, p = (\abs{g(a)} + \varepsilon)$.
                    [timelimit 10]
                    $r,\rho,p,\nu$ are nonnegative real numbers. 
                    $r \cdot \rho < \nu \cdot p$ (by ineq lemma 16).
                    Indeed $r < \nu$ and $\rho < p$ (by 21, 22). [timelimit 3]
                End.

                (23) $\nu \cdot (\abs{g(a)} + \varepsilon) = \frac{\varepsilon}{2}$.

                Proof. (231) Take $c = \abs{g(a)} + \varepsilon$.
                    $\nu \cdot (\abs{g(a)} + \varepsilon)
                    .= \frac{\varepsilon}{2 \cdot (\abs{g(a)} + \varepsilon)} \cdot (\abs{g(a)} + \varepsilon)$ (by 01)

                    $.= \frac{\varepsilon}{2 \cdot c} \cdot c$ (by 231) 

                    $.= (\frac{\varepsilon}{2} \cdot 1/c) \cdot c$ (by fraction lemma)

                    $.= \frac{\varepsilon}{2} \cdot (c \cdot 1/c)$ (by mult ass, mult comm)

                    $.= \frac{\varepsilon}{2}$ (by multiplicative inverse, mult comm, existence one).
                End.
            qed.

            (3) $\abs{f(a)} \cdot \abs{g(x) - g(a)} < \frac{\varepsilon}{2}$. Proof. [timelimit 20]

                (31) $\abs{g(x) - g(a)} < \pi$ (by domain restrictor lemma 3). Indeed
                    Thus $\delta$ is a domain restrictor for $\pi$ and $g$ around $a$ and $g(a)$ (by domain restrictor lemma). 
                    Indeed $\delta < \theta$.[timelimit 3]

                $\abs{f(a)} \cdot \abs{g(x) - g(a)} < (\abs{f(a)} + 1) \cdot \pi$. Proof.
                    Take $q = \abs{f(a)}$, $R = \abs{g(x) - g(a)}$, $P = \abs{f(a)} + 1$.
                    [timelimit 10]
                    $q,R,P,\pi$ are nonnegative real numbers.
                    $q \cdot R < P \cdot \pi$ (by ineq lemma 16).
                    Indeed $q < P$ and $R < \pi$. [timelimit 3]
                End.


                (32) $(\abs{f(a)} + 1) \cdot \pi = \frac{\varepsilon}{2}$.

                Proof. (321) Take $c= \abs{f(a)} + 1$.

                    $(\abs{f(a)} + 1) \cdot \pi
                    .= (\abs{f(a)} + 1) \cdot \frac{\varepsilon}{2 \cdot (\abs{f(a)} + 1)}$ (by 02)

                    $.= c \cdot \frac{\varepsilon}{2 \cdot c}$ (by 321)      

                    $.= c \cdot (\frac{\varepsilon}{2} \cdot 1/c)$ (by fraction lemma)

                    $.= (c \cdot 1/c) \cdot \frac{\varepsilon}{2}$ (by mult comm, mult ass)

                    $.= \frac{\varepsilon}{2}$ (by multiplicative inverse, mult comm, existence one). 
                End.

            qed. 
            
            (4) $\Rmetric(\fmult{f}{g}(x),\fmult{f}{g}(a)) < \varepsilon$. 
            Proof. 

                Take $Q = (\abs{f(x) - f(a)} \cdot \abs{g(x)})$, $S = (\abs{f(a)} \cdot \abs{g(x) - g(a)})$, $\epsilon = \frac{\varepsilon}{2}$.
                
                $Q + S < \epsilon + \epsilon$ (by ineq lemma 13).
                Indeed $Q < \epsilon$ and $S < \epsilon$ (by 2,3).
                $\epsilon + \epsilon = 2 \cdot \frac{\varepsilon}{2} = \varepsilon$.

                $\Rmetric(\fmult{f}{g}(x),\fmult{f}{g}(a)) \leq \varepsilon$ (by transitivity).
                
                Indeed $\Rmetric(\fmult{f}{g}(x),\fmult{f}{g}(a)) \leq Q + S < \varepsilon$ (by 1).
            End.
        End.
\end{proof}

\begin{corollary}
    Let $f$ be continuous. Let $g$ be continuous.
    $\fmult{f}{g}$ is continuous.
\end{corollary}

\end{forthel}



\subsection{Continuity Under composition}

The proof of continuity of $g \circ f$ is straightforward. Let $d$ be a domain restrictor for $f$ and $\varepsilon$.
Then Take a positive real number $\delta$ such that $\delta$ is a domain restrictor for $g$ and $d$. Then $\delta$ also serves as a domain restrictor for $g \circ f$ and $\varepsilon$.

\begin{forthel}

[timelimit 50]
\begin{theorem}[continuity of composition]
    Let $f$ be continuous in $a$ and $g$ be continuous in $f(a)$.
    Then $g \circ f$ is continuous in $a$.
\end{theorem}
\begin{proof}
    $g \circ f$ is a real map.
    Let $\varepsilon$ be a positive real number. 
    
    (1) Take a positive real number $d$ such that $d$ is a domain restrictor for $\varepsilon$ and $g$ around $f(a)$ and $g(f(a))$.

    (2) Take a positive real number $\delta$ such that $\delta$ is a domain restrictor for $d$ and $f$ around $a$ and $f(a)$.

    Let us show that $\delta$ is a domain restrictor for $\varepsilon$ and $(g \circ f)$ around $a$ and $g(f(a))$.
        Let $x \in \mathbb{R} \setminus \sset{a}{\mathbb{R}}$. Assume $\abs{x - a} < \delta$. Then $\abs{f(x) - f(a)} < d$.
        $\abs{g(f(x)) - g(f(a))} < \varepsilon$ (by domain restrictor lemma 3, 1). Indeed $f(x),f(a)$ are real numbers.
    End.
\end{proof}
[timelimit 3]

\begin{corollary}
    Let $f$ be continuous. Let $g$ be continuous.
    $g \circ f$ is continuous.
\end{corollary}


\end{forthel}


\end{document}

