\documentclass[../solow_main.tex]{subfiles}

\begin{document}

\chapter{Extrema and Mean Value Theorem}

In the last chapter before we are able to take on the inverse function theorem, we have to prove the mean value
theorem and some propositions about the relation of derivatives and the slope of the function, i.e. a function is strictly 
increasing if its derivative is larger than zero and strictly decreasing if smaller than zero. 
At the end of the chapter we also provide a proof that $f'(\xi) = 0$ and $f''(\xi) < 0$ is a sufficient condition for $\xi$
being a maximum. Using this condition for calculating some optimal state is absolutely ubiquitous in economics.

For the proof of the mean value theorem, we use Rolle's theorem, which says that when you got a differentiable function $f$ with
$f(a) = f(b)$ for some $a < b$ then there needs to be some value $\xi$ in between such $f'(\xi) = 0$.
For the proof of Rolle's theorem we need that such a function is bounded on $[a,b]$, for which we need the extreme value theorem,
and that the derivative of a maximum and minimum is zero, which we proof in section \ref{sec.local_extrema}. 


\begin{forthel}
    [read \path{examples/solow/sections/4locality.ftl.tex}][check off][prove off]
\end{forthel}

Before we can devote ourselves to the intermediate value theorem, we need to state the extreme value theorem. 

\section{Extreme Value Theorem}

\begin{forthel}
Let $a,b$ denote real numbers.
Let $f$ denote a real map.
Let $S$ denote a subset of $\mathbb{R}$.

Let $S$ is bounded stand for $S$ is bounded above and $S$ is bounded below.
Let $a$ bounds $S$ stand for $a$ is an upper bound of $S$ and $-a$ is a lower bound of $S$.

\end{forthel}

The following theorem is the boundedness theorem, which says that
the image of a continuous function on a closed interval is bounded.
This theorem we need in order to prove the extreme value theorem, namely when we have a continuous function $f$ on $[a,b]$
that we can find an $x \in [a,b]$ such that $f(x)$ is the supremum of the image of $f$ on $[a,b]$.

In the proof of the boundedness theorem we basically follow along the lines of mathcenter.oxford.emory.edu.
Here we define a function $h$ such that $$h(x) = \begin{cases}
        f[[a,\xi]_{CI}] & : \xi \leq b \\
        \emptyset & : \xi > b
    \end{cases}$$
for any real number $x$. Furthermore we define $B = \{x \in \mathbb{R} | a \leq x \leq b$ and $h(x)$ is bounded$\}$.
We then can take the supremum $s$ of $B$. In (1) we show that $s = b$ and in (2) that $b \in B$. Then we are done.
To prove (1) and (2) we choose a value $l$ very close to $s$ ($b$), 
for which we still, however, know that $f$ is bounded on $[a,l]$. Then we utilize that $f$ is continuous in $s$ and $b$ and 
therefore we can construct an upper bound for both.

\begin{forthel}

\begin{lemma}[bounded lemma 1]
    $a$ bounds $S$ iff $\abs{s} \leq a$ for every $s \in S$.
\end{lemma}

\begin{lemma}[bounded lemma 2]
    $S$ is bounded iff there exists a real number $K$ such that $K$ bounds $S$.
\end{lemma}
\begin{proof}
    Case $S$ is bounded. Take an upper bound $u$ of $S$. Take a lower bound $l$ of $S$.
        Take a real number $K$ such that $K \geq \abs{u}$ and $K \geq \abs{l}$. Then $K$ bounds $S$. 
    End.

    Case there exists a real number $K$ such that $K$ bounds $S$. Trivial.
\end{proof}

\begin{theorem}[boundedness theorem]
    Let $f$ be continuous.
    $f[[a,b]_{CI}] \subseteq \mathbb{R}$ and
    $f[[a,b]_{CI}]$ is bounded.
\end{theorem} 
\begin{proof}
    Case $a \geq b$. Trivial.
    Assume $a < b$.

    For any real number $\xi$ $[a,\xi]_{CI}$ is a subset of $\mathbb{R}$.
    Define $$h(\xi) = \begin{cases}
        f[[a,\xi]_{CI}] & : \xi \leq b \\ 
        \emptyset & : \xi > b
    \end{cases}$$
    for $\xi \in \mathbb{R}$.

    For every real number $x$ $h(x)$ is a subset of $\mathbb{R}$. Proof.
        Let $x$ be a real number.
        Case $x > b$. Trivial.
        Case $x \leq b$. $f[M] \subseteq \mathbb{R}$ for any $M\subseteq\mathbb{R}$.
            Therefore $f[[a,x]_{CI}]$ is a subset of $\mathbb{R}$.
        End.
    End.

    (A) For all real number $x,y$ if $x \leq y \leq b$ then $h(x) \subseteq h(y)$. 

    Proof. Let $x,y$ be real numbers. Assume $x \leq y \leq b$. Let $xi \in h(x)$. Then $xi \in f[[a,x]_{CI}]$.
        Take $xd \in [a,x]_{CI}$ such that $f(xd) = xi$. Then $xd \in [a,y]_{CI}$ and $y \leq b$.
        $f(xd) \in f[[a,y]_{CI}]$.
        Therefore $xi = f(xd) \in h(y)$.
    End.

    Define $B = \{x \in \mathbb{R} | a \leq x \leq b$ and $h(x)$ is bounded$\}$.

    Take supremum $s$ of $B$ (by completeness axiom).
    Indeed we can show that $B$ is a nonempty subset of $\mathbb{R}$ and bounded above.

        $h(a)$ is bounded. Indeed ($f(a) \geq y$ for all $y \in h(a)$) and ($f(a) \leq y$ for all $y \in h(a)$).
        Therefore $B$ is nonempty.

        $B$ is bounded above. Indeed for every $x \in B$ $x \leq b$.
    End.

    (1) $s = b$. Proof by contradiction.
        $b$ is an upper bound of $B$. Therefore $s \leq b$.
        Assume $s < b$.

        (10) Take a positive real number $\delta$ such that $\delta$ is a domain restrictor for 1 and $f$ around $s$ and $f(s)$.
        Indeed $1$ is a positive real number and $f$ is continuous in $s$.

        Take $l \in B$ such that $s-\delta < l$ (by lemma x01). 
            Indeed $B$ is a nonempty subset of $\mathbb{R}$ and $B$ is bounded above.
               
        Take real number $K$ such that $K$ bounds $h(l)$ (by bounded lemma 2). Indeed $h(l)$ is bounded. [prover vampire]
        Take a real number $u$ such that $s < u < s + \delta$ and $u \leq b$. 
        Proof. $b - s$ is a positive real number.
            Take positive real number $\upsilon$ such that $\upsilon < \delta$ and $\upsilon < b - s$ (by ineq lemma 19). 
            Take $u = s + \upsilon$. $s < u < s + \delta$. Indeed $s < s + \upsilon < s + \delta$. Indeed $0 < \upsilon < \delta$. $u < b$.
        End.

        $h(u) = f[[a,u]_{CI}]$.

        $h(u)$ is bounded (by bounded lemma 1, bounded lemma 2). Indeed for every $y \in h(u)$ $\abs{y} \leq K + 2$. 
                Indeed we can show that for every $x \in [a,u]_{CI}$ $\abs{f(x)} \leq K + 2$.
            Let $x \in [a,u]_{CI}$. Then $x$ is a real number such that $a \leq x \leq u$. [timelimit 20]

            Case $x \leq l$. $f(x) \in f[[a,l]_{CI}]= h(l)$. Therefore $\abs{f(x)} \leq K \leq K+2$. Indeed $K$ bounds $h(l)$.
            End.
 
            Case $x > l$.

                (1A) $\abs{f(x) - f(s)} < 1$ (by domain restrictor lemma 3, 10, 1 is positive). Proof.
                    $\abs{x - s} < \delta$. Indeed $s - \delta < x < s + \delta$. $f(x),f(s)$ are real numbers.
                End.

                (1B) $\abs{f(l) - f(s)} < 1$ (by domain restrictor lemma 3, 10, 1 is positive). Proof.
                    $\abs{l - s} < \delta$. $f(l),f(s)$ are real numbers.
                End. [timelimit 3][prover eprover]

                (1C) $\abs{f(x)} < 2 + \abs{f(l)}$ (by ineq lemma 15). Proof.

                    $\abs{f(x)} - \abs{f(l)} \leq \abs{f(x) - f(l)}$ (by absolute lemma 6). 

                    $\abs{f(x) - f(l)} .= \abs{(f(x) - f(l)) + (f(s) - f(s))}$ (by existence zero, additive inverse)

                    $.= \abs{(f(x) - f(s)) + (f(s) - f(l))}$ (by add ass, add comm).

                    $\abs{(f(x) - f(s)) + (f(s) - f(l))} \leq \abs{f(x) - f(s)} + \abs{f(s) - f(l)}$ (by absolute lemma 4).

                    $\abs{f(x) - f(s)} + \abs{f(s) - f(l)} < 1 + \abs{f(s) - f(l)}$ (by ineq lemma 14, 1A).

                    $1 + \abs{f(s) - f(l)} .= \abs{f(l) - f(s)} + 1$ (by add comm, absolute lemma 3).

                    $\abs{f(l) - f(s)} + 1 < 1 + 1 = 2$ (by ineq lemma 14, 1B). 

                    Thus $\abs{f(x)} - \abs{f(l)} < 2$.
                End. [timelimit 10]

                (1D) $\abs{f(l)} \leq K$. Proof. 
                     $l \in B$. $h(l)$ is bounded and $K$ bounds $h(l)$. $f(l) \in f[[a,l]_{CI}] = h(l)$. Indeed $l \in [a,l]_{CI}$.
                End. [timelimit 3]

                Therefore $\abs{f(x)} \leq 2 + \abs{f(l)} = \abs{f(l)} + 2 \leq K + 2$ (by 1C, add comm, ineq lemma 145, 1D).
                Then $\abs{f(x)} \leq K + 2$ (by transitivity).
            End. 
        qed. [timelimit 10]

        Then $u \in B$ and $s < u$. Contradiction. [timelimit 3]
    End.

    (2) $b \in B$. Proof.

        (20) Take a positive real number $\delta$ such that $\delta$ is a domain restrictor for 1 and $f$ around $b$ and $f(b)$. 
        Indeed 1 is a positive real number and $f$ is continuous in $b$.

        Take $l \in B$ such that $b - \delta < l$ (by lemma x01).
            Indeed $B$ is a nonempty subset of $\mathbb{R}$ and $B$ is bounded above and $b$ is the supremum of $B$.

        Take a real number $K$ such that $K$ bounds $h(l)$ (by bounded lemma 2). Indeed $h(l)$ is bounded.

        $h(b) = f[[a,b]_{CI}]$. [timelimit 10]

        (21) For every $y \in h(b)$ $\abs{y} \leq K + 2$.
            Indeed we can show that for every $x \in [a,b]_{CI}$ $\abs{f(x)} \leq K + 2$.
            Let $x \in [a,b]_{CI}$. Then $x$ is a real number such that $a \leq x \leq b$. 

            Case $x \leq l$. $f(x) \in f[[a,l]_{CI}] = h(l)$. Therefore $\abs{f(x)} \leq K \leq K+2$. Indeed $K$ bounds $h(l)$.
            End. 
            
            Case $x > l$.

                (2A) $\abs{f(x) - f(b)} < 1$ (by domain restrictor lemma 3, 20, 1 is positive). Proof.
                    $\abs{x - b} < \delta$. Indeed $b - \delta < x \leq b$.
                    $f(x),f(b)$ are real numbers.
                End.

                (2B) $\abs{f(l) - f(b)} < 1$ (by domain restrictor lemma 3, 20, 1 is positive). Proof.
                    $\abs{l - b} < \delta$. $f(l),f(b)$ are real numbers.
                End.

                (2C) $\abs{f(x)} < 2 + \abs{f(l)}$ (by ineq lemma 15). Proof. 

                    $\abs{f(x)} - \abs{f(l)} \leq \abs{f(x) - f(l)}$ (by absolute lemma 6). 

                    $\abs{f(x) - f(l)} .= \abs{(f(x) - f(l)) + (f(b) - f(b))}$ (by existence zero, additive inverse) 

                    $.= \abs{((f(x) - f(l)) - f(b)) + f(b)}$ (by add comm, add ass)

                    $.= \abs{(f(x) - f(b)) + (f(b) - f(l))}$ (by add comm, add ass). 

                    $\abs{(f(x) - f(b)) + (f(b) - f(l))} \leq \abs{f(x) - f(b)} + \abs{f(b) - f(l)}$ (by absolute lemma 4). 

                    $\abs{f(x) - f(b)} + \abs{f(b) - f(l)} < 1 + \abs{f(b) - f(l)}$ (by ineq lemma 14, 2A). 

                    $1 + \abs{f(b) - f(l)} .= \abs{f(l) - f(b)} + 1$ (by add comm, absolute lemma 3). 

                    $\abs{f(l) - f(b)} + 1 < 1 + 1 = 2$ (by ineq lemma 14, 2B).

                    Thus $\abs{f(x)} - \abs{f(l)} < 2$.
                End.

                (2D) $\abs{f(l)} \leq K$. Proof. 
                     $l \in B$ and $h(l)$ is bounded. $f(l) \in f[[a,l]_{CI}] = h(l)$. Indeed $l \in [a,l]_{CI}$.
                End. 
                
                Therefore $\abs{f(x)} \leq 2 + \abs{f(l)} = \abs{f(l)} + 2 \leq K + 2$ (by 2C, add comm, ineq lemma 145, 2D). 

                Therefore $\abs{f(x)} \leq K + 2$ (by transitivity).
            End.
        qed. [prover vampire][timelimit 50]

        $h(b)$ is bounded (by bounded lemma 1, bounded lemma 2, 21). 
    End. [prover eprover][timelimit 3]

    $h(b) = f[[a,b]_{CI}]$. Thus $f[[a,b]_{CI}]$ is bounded.
\end{proof}
\end{forthel}

From the boundedness theorem we know that $f[[a,b]]$ has a supremum that we call $s$. Now we assume, for purpose of contradiction,
that no such a $c$ as in the theorem exists and therefore is $f(x) < s$ for all $x \in [a,b]$.
Then we define $$g(x) = \begin{cases}
    \frac{1}{s - f(a)} & : x \leq a \\
    \frac{1}{s - f(x)} & : x \in (a,b)\\
    \frac{1}{s - f(b)} & : x \geq b
\end{cases}$$
for all real numbers $x$ and show that $g$ is continuous. The continuity seems obvious to us, but formally it takes up very significant 
time and most of the space of the proof to show the continuity in any point $x$ for all cases when $x < a$, $x = a$, $x \in (a,b)$, $x = b$,
and $x > b$. Now that $g$ is continuous, we know from the boundedness theorem that $g[[a,b]]$ is bounded. Therefore we can take 
a positive real number $K$ that bounds $g[[a,b]]$ and then we can show that $s - 1/K$ is an upper bound of $f[[a,b]]$, which serves us as 
a contradiction to $s$ being the lowest upper bound of $f[[a,b]]$, i.e. its supremum.

\begin{forthel}
[timelimit 100]
\begin{theorem}[extreme value theorem]
    Let $f$ be continuous and $a < b$.
    $f[[a,b]_{CI}]$ is a subset of $\mathbb{R}$ and 
    there exists a $c \in [a,b]_{CI}$ such that $f(c)$ is an upper bound of $f[[a,b]_{CI}]$.
\end{theorem}
\begin{proof}
    Take $S = f[[a,b]_{CI}]$. $S$ is a subset of $\mathbb{R}$. [prover vampire]

    Take a supremum $s$ of $S$ (by completeness axiom). Proof.
        $S$ is bounded above. Indeed $S$ is bounded (by boundedness theorem).
        $S$ is nonempty. Indeed $f(a)$ is an element of $S$. 
    End.

    Case there exists $c \in [a,b]_{CI}$ such that $f(c) = s$. Trivial.

    Case for every $c \in [a,b]_{CI}$ $f(c) \neq s$.
        Take $\varphi = \flin{-1}{s} \circ f$.

        (A) $\varphi(x) = s - f(x)$ for any $x \in [a,b]_{CI}$. Proof.
            Let $x \in [a,b]_{CI}$.

            $(\flin{-1}{s} \circ f)(x) .= ((-1) \cdot f(x)) + s$ (by composition, linear function)

            $.= s - f(x)$ (by eq lemma 12, add comm).
        End.

        (B) $\varphi$ is a continuous real map.

        (C) $\varphi(x)$ is a positive real number for any $x \in [a,b]_{CI}$. Proof.
            Let $x \in [a,b]_{CI}$. Then $s - f(x)$ is a positive real number. Indeed $f(x) < s$. 
            $\varphi(x)$ is a positive real number (by A).
        End.

        For any real number $x$ $x \leq a$ or $x \in (a,b)_{OI}$ or $x \geq b$.
        $\varphi(x)$ is nonzero for any $x \in [a,b]_{CI}$.

        Define $$g(x) = \begin{cases} 
            1/\varphi(a) & : x \leq a \\ 
            1/\varphi(x) & : x \in (a,b)_{OI} \\
            1/\varphi(b) & : x \geq b
        \end{cases}$$
        for $x \in \mathbb{R}$. [prover vampire] $g$ is a real map. 
        Indeed $g(x)$ is a real number for any real number $x$. Proof.
            Let $x$ be a real number.
            Case $x \leq a$ or $x \geq b$.
                $\varphi(a),\varphi(b)$ are positive real numbers (by C). Indeed $a,b \in [a,b]_{CI}$.
                $1/\varphi(a), 1/\varphi(b)$ are real numbers (by multiplicative inverse).
                Therefore $g(x)$ is a real number. 
            End.
            Case $x \in (a,b)_{OI}$. Trivial. 
        End. [timelimit 10][prover eprover]

        $g$ is continuous. Indeed we can show that for any real number $\xi$ $g$ is continuous in $\xi$.
            Let $\xi$ be a real number. 
            $\xi < a$ or $\xi > b$ or $\xi \in (a,b)_{OI}$ or $\xi = a$ or $\xi = b$.

            Case $\xi < a$. $a - \xi$ is a positive real number.
                We can show that for any positive real number $\varepsilon$
                            $a - \xi$ is a domain restrictor for $\varepsilon$ and $g$ around $\xi$ and $g(\xi)$.
                    Let $\varepsilon$ be a positive real number.

                    Let $x \in \mathbb{R} \setminus \sset{\xi}{\mathbb{R}}$. Assume $\abs{x - \xi} < a - \xi$.

                    Then $x < a$. Proof. 
                        If $x < \xi$ then $x < \xi < a$.
                        If $x \geq \xi$ then $\abs{x - \xi} = x - \xi < a - \xi$ and $x < a$.
                    End.

                    $\varphi(a)$ is nonzero.
                    Thus $\abs{g(x) - g(\xi)} = \abs{1/\varphi(a) - 1/\varphi(a)} = 0 < \varepsilon$.
                End.
            qed.

            Case $\xi > b$. $\xi - b$ is a positive real number.
                We can show that for any positive real number $\varepsilon$
                            $\xi - b$ is a domain restrictor for $\varepsilon$ and $g$ around $\xi$ and $g(\xi)$.
                    Let $\varepsilon$ be a positive real number.

                    Let $x \in \mathbb{R} \setminus \sset{\xi}{\mathbb{R}}$. Assume $\abs{x - \xi} < \xi - b$.

                    Then $x > b$. Proof.
                        If $x \geq \xi$ then $b < \xi \leq x$.
                        If $x < \xi$ then $\abs{x - \xi} = \abs{\xi - x} = \xi - x < \xi - b$ and $b < x$.
                    End.

                    $\varphi(b)$ is nonzero.
                    Thus $\abs{g(x) - g(\xi)} = \abs{1/\varphi(b) - 1/\varphi(b)} = 0 < \varepsilon$.
                End.
            qed.

            Case $\xi \in (a,b)_{OI}$.
                $g$ is a division map of $\varphi$ in $\xi$. Proof. 
                    Then $(a,b)_{OI}$ is an open neighbourhood of $\xi$.
                    Then ($\varphi(x)$ is nonzero and $g(x) = 1/\varphi(x)$) for all $x \in (a,b)_{OI}$ .
                End.

                $g$ is continuous in $\xi$ (by continuity division map). Indeed $\varphi$ is continuous in $\xi$.
            End.[timelimit 50]

            Case $\xi = a$.
                Take a division map $h$ of $\varphi$ in $a$ (by existence division map, B). 
                    Indeed $\varphi(a) \neq 0$ (by C).
                $h$ is continuous in $a$ (by continuity division map, B).

                Let $\varepsilon$ be a positive real number.
                Take an open neighbourhood $E$ of $a$ such that for every $x \in E$ $\varphi(x)$ is nonzero and $h(x) = 1/\varphi(x)$.
                Take a positive real number $\tau$ such that $\B_{\tau}(a) \subseteq E$. [prover vampire]
                Take a positive real number $\theta$ such that $\theta < \tau$ and $\theta < b - a$ (by ineq lemma 19). 
                    Indeed $\tau, b - a$ are positive real numbers.
                Take a positive real number $\vartheta$ such that $\vartheta$ is a domain restrictor for $\varepsilon$ and $h$ around $a$ and $h(a)$.
                Take a positive real number $\delta$ such that $\delta < \theta$ and $\delta < \vartheta$ (by ineq lemma 19).
                Then $\delta$ is a domain restrictor for $\varepsilon$ and $h$ around $a$ and $h(a)$ (by domain restrictor lemma). 
                For all real numbers $x$ if $\abs{x - a} < \delta$ then $\abs{x - a} < \theta < \tau$ and $\abs{x - a} < \theta < b - a$.
                For all real numbers $x$ if $\abs{x - a} < \delta$ then $\abs{x - a} < \tau$ and $x \in E$ and $\varphi(x)$ is nonzero and $h(x) = 1/\varphi(x)$.

                Let us show that $\delta$ is a domain restrictor for $\varepsilon$ and $g$ around $a$ and $g(a)$.
                    Let $x \in \mathbb{R} \setminus \sset{a}{\mathbb{R}}$. Assume $\abs{x - a} < \delta$.

                    Case $x \leq a$. $\abs{g(x) - g(a)} = \abs{1/\varphi(a) - 1/\varphi(a)} = 0 < \varepsilon$. End.

                    Case $x > a$. $x - a = \abs{x - a} < \delta < b - a$. $x - a < b - a$ and $x < b$. 
                        Then $x \in [a,b]_{CI}$. [prover eprover]
                        $g(x) = 1/\varphi(x) = h(x)$. $g(a) = 1/\varphi(a) = h(a)$. 
                        Therefore $\abs{g(x) - g(a)} = \abs{h(x) - h(a)} < \varepsilon$.
                    End.
                qed.
            End.

            Case $\xi = b$.
                Take a division map $h$ of $\varphi$ in $b$ (by existence division map, B).
                    Indeed $\varphi(b) \neq 0$ (by C).
                $h$ is continuous in $b$ (by continuity division map, B).

                Let $\varepsilon$ be a positive real number.
                Take an open neighbourhood $E$ of $b$ such that for every $x \in E$ $\varphi(x)$ is nonzero and $h(x) = 1/\varphi(x)$.
                Take a positive real number $\tau$ such that $\B_{\tau}(b) \subseteq E$. [prover vampire]
                Take a positive real number $\theta$ such that $\theta < \tau$ and $\theta < b - a$ (by ineq lemma 19).
                    Indeed $\tau, b - a$ are positive real numbers.
                Take a positive real number $\vartheta$ such that $\vartheta$ is a domain restrictor for $\varepsilon$ and $h$ around $b$ and $h(b)$.
                Take a positive real number $\delta$ such that $\delta < \theta$ and $\delta < \vartheta$ (by ineq lemma 19).
                Then $\delta$ is a domain restrictor for $\varepsilon$ and $h$ around $b$ and $h(b)$ (by domain restrictor lemma).
                For all real numbers $x$ if $\abs{x - b} < \delta$ then $\abs{x - b} < \theta < \tau$ and $\abs{x - b} < \theta < b - a$.
                For all real numbers $x$ if $\abs{x - b} < \delta$ then $\abs{x - b} < \tau$ and $x \in E$ and $h(x) = 1/\varphi(x)$. 

                Let us show that $\delta$ is a domain restrictor for $\varepsilon$ and $g$ around $b$ and $g(b)$.
                    Let $x \in \mathbb{R} \setminus \sset{b}{\mathbb{R}}$. Assume $\abs{x - b} < \delta$.

                    Case $x \geq b$. $\abs{g(x) - g(b)} = \abs{1/\varphi(b) - 1/\varphi(b)} = 0 < \varepsilon$. End. 

                    Case $x < b$. $x \in [a,b]_{CI}$. Indeed $b - x < \delta < b - a$ and $a < x$. Then $g(x) = 1/\varphi(x) = h(x)$. $g(b) = 1/\varphi(b) = h(b)$. 
                        Therefore $\abs{g(x) - g(b)} = \abs{h(x) - h(b)} < \varepsilon$ (by domain restrictor lemma 3).
                    End.
                qed.
            End.
        qed. [timelimit 50][prover eprover]

        $g[[a,b]_{CI}]$ is a subset of $\mathbb{R}$.
        Take a real number $\kappa$ such that $\kappa$ bounds $g[[a,b]_{CI}]$ (by bounded lemma 2). Indeed $g[[a,b]_{CI}]$ is bounded (by boundedness theorem).
        Take a positive real number $K$ such that $K \geq \kappa$. Then $K$ bounds $g[[a,b]_{CI}]$. 

        Let us show that $s - 1/K$ is an upper bound of $f[[a,b]_{CI}]$.
            Let $y \in f[[a,b]_{CI}]$. Take $x \in [a,b]_{CI}$ such that $f(x) = y$.
            Then $g(x) \leq K$. 
            Therefore $1/\varphi(x) \leq K$. 
            $1/(1/\varphi(x)) \geq 1/K$ (by ineq lemma 12). Indeed $0 < 1/\varphi(x) \leq K$.
            $\varphi(x) \geq 1/K$ (by eq lemma 8).
            $s - f(x) \geq 1/K$ (by A).
            $f(x) \leq s - 1/K$ (by ineq lemma 30). 
        End.

        $s - 1/K < s$. Contradiction.
    End.
\end{proof}


\begin{lemma}[extreme value theorem 2]
    Let $f$ be continuous and $a < b$.
    There exists an $i \in [a,b]_{CI}$ such that $f(i)$ is a lower bound of $f[[a,b]_{CI}]$.
\end{lemma}
\begin{proof}
    $\fminus{f}$ is continuous. $\fminus{f}[[a,b]_{CI}]$ is a subset of $\mathbb{R}$.
    Take $i \in [a,b]_{CI}$ such that $\fminus{f}(i)$ is an upper bound of $\fminus{f}[[a,b]_{CI}]$ (by extreme value theorem).

    Then $f(i)$ is a lower bound of $f[[a,b]_{CI}]$. Proof. [prover vampire]

        Let $y \in f[[a,b]_{CI}]$. Then $y$ is a real number and $-y \in \fminus{f}[[a,b]_{CI}]$. [prover eprover]
        Thus $-f(i) = \fminus{f}(i) \geq -y$ and $y \geq f(i)$.
    End.
\end{proof}
\end{forthel}


\section{Local Extrema}\label{sec.local_extrema}

\begin{forthel}
Let $f$ denote a real map.
Let $\xi$ denote a real number.

\begin{definition}
    A maximum domain restrictor for $f$ around $\xi$ is a positive real number $\delta$ 
    such that for all $x \in \B_{\delta}(\xi)$ $f(x) \leq f(\xi)$.
\end{definition}

\begin{definition}
    A local maximum of $f$ is a real number $\xi$ such that there exists a positive real number $\delta$ such that 
    $\delta$ is a maximum domain restrictor for $f$ around $\xi$.
\end{definition}

\end{forthel}

The proof that the derivative of a local maximum is zero, is a proof by contradiction. Assume $f'(\xi) = y$ for some $y \neq 0$.
If $y > 0$ then however small you choose $\delta$ you can always find an $x$ such that $\frac{f(x) - f(\xi)}{x - \xi} < 0$ and therefore
having at least a distance $y$ to $y$. In turn if $y < 0$ then again however small you choose $\delta$ you can find an $x$ such that 
$\frac{f(x) - f(\xi)}{x - \xi} > 0$ and therefore having at least a distance $y$ to $y$.
\begin{forthel}
[prover vampire][timelimit 3]
\begin{theorem}[derivative of maximum]
    Let $\xi$ be a local maximum of $f$. Let $f$ be differentiable in $\xi$.
    Then $\derivin{f}(\xi) = 0$.
\end{theorem}
\begin{proof}
    Take a maximum domain restrictor $\tau$ for $f$ around $\xi$.
    Take $y = \derivin{f}(\xi)$.

    We can show that $y = 0$.
        Assume $y \neq 0$. Take $\varepsilon = \abs{y}$.
        Take a positive real number $\theta$ such that $\theta$ is a domain restrictor for $\varepsilon$ and $\Diffquot{f}{\xi}$ around $\xi$ and $y$.
            Indeed $\limS{\Diffquot{f}{\xi}}{\xi} = y$.
        Take a positive real number $\delta$ such that $\delta < \tau$ and $\delta < \theta$ (by ineq lemma 19).

        Case $y > 0$. Take a real number $x$ such that $\xi < x < \xi + \delta$ (by ineq lemma 20). Indeed $\xi < \xi + \delta$.
            Then $\abs{x - \xi} = x - \xi < \delta < \theta$.

            Let us show that $\abs{\Diffquot{f}{\xi}(x) - y} \geq \varepsilon$.
                $f(x) - f(\xi) \leq 0$ (by ineq lemma 27). Proof.
                    $f(x),f(\xi)$ are real numbers.
                    $\xi$ is a local maximum of $f$. $x \in \B_{\tau}(\xi)$. $f(x) \leq f(\xi)$.
                End.
                $x - \xi > 0$. Therefore $\Diffquot{f}{\xi}(x) = \frac{f(x) - f(\xi)}{x - \xi} \leq 0$ (by frac lemma 3).
                $\Diffquot{f}{\xi}(x) - y < 0$. Indeed $-y < 0$ and $\Diffquot{f}{\xi}(x) \leq 0$. 
                Thus $\abs{\Diffquot{f}{\xi}(x) - y}
                = - (\Diffquot{f}{\xi}(x) - y)$.
                $- (\Diffquot{f}{\xi}(x) - y)
                .= y - \Diffquot{f}{\xi}(x)$ (by eq lemma 15, eq lemma 45, add comm).
                $y - \Diffquot{f}{\xi}(x) \geq y = \varepsilon$ (by ineq lemma 23). Indeed $-\Diffquot{f}{\xi}(x) \geq 0$. 
            End.

            [timelimit 10]
            Thus $\theta$ is not a domain restrictor for $\varepsilon$ and $\Diffquot{f}{\xi}$ around $\xi$ and $y$ (by domain restrictor lemma 2).
                Indeed $x \in \mathbb{R} \setminus \sset{\xi}{\mathbb{R}}$ and $\abs{x - \xi} < \theta$. [timelimit 3]
            Contradiction.
        qed.
       
        Case $y < 0$. Take a real number $x$ such that $\xi - \delta < x < \xi$ (by ineq lemma 20). 
                Indeed $\xi - \delta < \xi$ (by ineq lemma 24).
            [timelimit 30] Then $\abs{x - \xi} = \abs{\xi - x} = \xi - x < \delta < \theta$.

            Let us show that $\abs{\Diffquot{f}{\xi}(x) - y} \geq \varepsilon$.
                $f(x) - f(\xi) \leq 0$ (by ineq lemma 27). Proof. 
                    $f(x),f(\xi)$ are real numbers.
                    $\xi$ is a local maximum of $f$. $x \in \B_{\tau}(\xi)$. $f(x) \leq f(\xi)$. 
                End.
                $x - \xi < 0$. Therefore $\Diffquot{f}{\xi}(x) = \frac{f(x) - f(\xi)}{x - \xi} \geq 0$ (by frac lemma 4).
                $\Diffquot{f}{\xi}(x) - y \geq 0$ (by ineq lemma 21). Indeed $-y \geq 0$.
                $\abs{\Diffquot{f}{\xi}(x) - y} = \Diffquot{f}{\xi}(x) - y$.
                
                $-y + \Diffquot{f}{\xi}(x) \geq -y = \varepsilon$ (by ineq lemma 23). 
            End.

            Take $h = \Diffquot{f}{\xi}.$
            Thus $\theta$ is not a domain restrictor for $\varepsilon$ and $h$ around $\xi$ and $y$ (by domain restrictor lemma 2).
                Indeed $x \in \mathbb{R} \setminus \sset{\xi}{\mathbb{R}}$ and $\abs{x - \xi} < \theta$ and $h$ is a real map
                and $\abs{h(x) - y} \geq \varepsilon$. [timelimit 3]
            Contradiction.
        qed.
    QED.
\end{proof}

\begin{definition}
    A local minimum of $f$ is a real number $\xi$ such that $\xi$ is a local maximum of $\fminus{f}$.
\end{definition}

\begin{corollary}[derivative of minimum]
    Let $\xi$ be a local minimum of $f$. Let $f$ be differentiable in $\xi$.
    Then $\derivin{f}(\xi) = 0$.
\end{corollary}
\begin{proof}
    (A) $\xi$ is a local maximum of $\fminus{f}$.

    $-\derivin{f}(\xi) = \derivin{\fminus{f}}(\xi) = 0$ (by minus derivative, A, derivative of maximum).
\end{proof}

\end{forthel}


\section{Mean Value Theorem}

In the proof of Rolle's theorem we basically have three cases. If we have an inner local maximum, then the derivative of that local maximum is zero.
If we have a local minimum, then the derivative of that local minimum is zero. And if we have neither, the function must be constant on the interval
and therefore the derivative of any value therein must be zero.

\begin{forthel}
Let $f$ denote a real map.
Let $a,b$ denote real numbers.

\begin{theorem}[Rolles theorem]
    Let $f$ be differentiable. Let $a < b$ and $f(a) = f(b)$.
    Then there exists a $\xi \in (a,b)_{OI}$ such that $\derivin{f}(\xi) = 0$.
\end{theorem}
\begin{proof}
    Take $M = f[[a,b]_{CI}]$. $M$ is a subset of $\mathbb{R}$.
    Take $s \in [a,b]_{CI}$ such that $f(s)$ is an upper bound of $M$ (by extreme value theorem).
        Indeed $f$ is continuous and $a < b$.  
    
    Case $a \neq s \neq b$. 
        Then $s$ is a local maximum of $f$. Proof.
            Take a positive real number $\delta$ such that $\delta < s - a$ and $\delta < b - s$ (by ineq lemma 19). 
                Indeed $s - a, b - s$ are positive real numbers.
            Then $\delta$ is a maximum domain restrictor for $f$ around $s$. Proof.
                Let $x \in \B_{\delta}(s)$. 
                (A) Then $x \in [a,b]_{CI}$. Proof.
                    $x \in (s - \delta, s + \delta)_{OI}$ (by ball and open interval).
                    $a \leq x \leq b$ (by transitivity). Indeed $a < s - \delta < x < s + \delta < b$. 
                End.
                Therefore $f(s) \geq f(x)$. Indeed $f(x) \in f[[a,b]_{CI}]$ (by A).
            End.
        End.
    
        $\derivin{f}(s) = 0$ (by derivative of maximum). 
    End. 

    Case $s = a$ or $s = b$. 
        Take $i \in [a,b]_{CI}$ such that $f(i)$ is a lower bound of $M$ (by extreme value theorem 2).
            Indeed $f$ is continuous and $a < b$.

        Case $a \neq i \neq b$. 
            $i$ is a local minimum of $f$. Proof.
                Take a positive real number $\delta$ such that $\delta < i - a$ and $\delta < b - i$ (by ineq lemma 19).
                    Indeed $i - a, b - i$ are positive real numbers.
                Then $\delta$ is a maximum domain restrictor for $\fminus{f}$ around $i$. Proof.
                    Let $x \in \B_{\delta}(i)$. Then $x \in [a,b]_{CI}$. Proof.
                        $x \in (i - \delta, i + \delta)_{OI}$ (by ball and open interval).
                        $a \leq x \leq b$ (by transitivity). Indeed $a < i - \delta < x < i + \delta < b$.
                    End. [timelimit 10]
                    Therefore $\fminus{f}(i) \geq \fminus{f}(x)$. Indeed $f(i) \leq f(x)$ and $\fminus{f}(i) = -f(i) \geq -f(x) = \fminus{f}(x)$. 
                    [timelimit 3]
                End.
            End.

            Thus $\derivin{f}(i) = 0$ (by derivative of minimum). 
        End.
                            
        Case $i = a$ or $i = b$.  [timelimit 10]
            (A) Then for all $x \in (a,b)_{OI}$ $f(x) = f(a)$. Proof.
                $f(s)$ is an upper bound of $M$ and $f(s) = f(a)$.
                $f(i)$ is a lower bound of $M$ and $f(i) = f(a)$.
                For all $y \in M$ $y = f(a)$.
                Therefore for all $x \in (a,b)_{OI}$ $f(x) = f(a)$.
            End. 
            Take $e \in (a,b)_{OI}$.
            Take a positive real number $\delta$ such that $\delta < b - e$ and $\delta < e - a$ (by ineq lemma 19).
                Indeed $a < e < b$ and $b - e, e - a$ are positive real numbers.

            Let us show that for any positive real number $\varepsilon$ $\delta$ is a domain restrictor for $\varepsilon$
                    and $\Diffquot{f}{e}$ around $e$ and $0$.
                Let $\varepsilon$ be a positive real number. 
                We can show that $\delta$ is a domain restrictor for $\varepsilon$ and $\Diffquot{f}{e}$ around $e$ and $0$.
                    Let $x \in \mathbb{R} \setminus \sset{e}{\mathbb{R}}$.
                    Assume $\abs{x - e} < \delta$. $x \in (e - \delta, e + \delta)_{OI}$. 
                    $(e - \delta, e + \delta)_{OI} \subseteq (a,b)_{OI}$ (by open interval lemma). Indeed $a < e - \delta < e + \delta < b$.
                    $x - e$ is nonzero.
                    $\Diffquot{f}{e}(x) .= \frac{f(x) - f(e)}{x - e}$ (by def diffquot)
                    $.= 0$ (by A, additive inverse, eq lemma 9).
                    Thus $\abs{\Diffquot{f}{e}(x) - 0} < \varepsilon$.
                End.
            End.
        End.
    End.[timelimit 3]
\end{proof}

\end{forthel}

Following the standard proof of the mean value theorem based upon Rolle's theorem we define a function $h(x) = f(x) - \frac{f(b) - f(a)}{b - a}(x - a)$.
Then $h(a) = f(a) = h(b)$ and using Rolle's theorem we find an $\xi$ such that $h'(\xi) = 0$. Then we know that $f'(\xi) = \frac{f(b) - f(a)}{b - a}$.
\begin{forthel}
 [prover eprover]
\begin{corollary}[mean value theorem]
    Let $f$ be differentiable and $a < b$.
    Then there exists $\xi \in (a,b)_{OI}$ such that $\derivin{f}(\xi) = \Diffquot{f}{a}(b)$.
\end{corollary}
\begin{proof}
    (A) Take $h = \fplus{f}{\flin{-\Diffquot{f}{a}(b)}{\Diffquot{f}{a}(b) \cdot a}}$.

    $h$ is differentiable.
    
    $h(a) = h(b)$. Proof.

        $h(a) .= f(a) + \left(\left(-\Diffquot{f}{a}(b) \cdot a\right) + \left(\Diffquot{f}{a}(b) \cdot a\right)\right)$ (by A, def add func,linear function) 

        $.= f(a) + \left(\Diffquot{f}{a}(b) \cdot (a - a)\right)$ (by eq lemma 21, distributive law, add comm)

        $.= f(a)$ (by additive inverse, mult comm, eq lemma 9, existence zero).

        $h(b) .= f(b) + \left(\left(-\Diffquot{f}{a}(b) \cdot b\right) + \left(\Diffquot{f}{a}(b) \cdot a\right)\right)$ (by A, def add func,linear function)

        $.= f(b) + \left(\Diffquot{f}{a}(b) \cdot (a - b)\right)$ (by eq lemma 21, distributive law, add comm)

        $.= f(b) + \left((a - b) \cdot \frac{f(b) - f(a)}{b - a}\right)$ (by mult comm, def diffquot)

        $.= f(b) - \left((b - a) \cdot \frac{f(b) - f(a)}{b - a}\right)$ (by eq lemma 15, eq lemma 45, add comm, eq lemma 11)

        $.= (f(b) - f(b)) + f(a)$ (by fraction lemma 2, eq lemma 15, eq lemma 45, add ass)

        $.= f(a)$ (by additive inverse, add comm, existence zero).
    End. [timelimit 10]

    (B) Take $\xi\in (a,b)_{OI}$ such that $\derivin{h}(\xi) = 0$ (by Rolles theorem). 
        Indeed $h$ is a differentiable real map and $a < b$ and $h(a) = h(b)$.[timelimit 3]

    $\derivin{f}(\xi) = \Diffquot{f}{a}(b)$. Proof. 

        (C) $\deriv{h} .= \fplus{\deriv{f}}{\flin{0}{-\Diffquot{f}{a}(b)}}$ (by A, derivative addition rule 2, derivative of linear function).

        $\deriv{h}(\xi) .= \deriv{f}(\xi) + \flin{0}{-\Diffquot{f}{a}(b)}(\xi)$ (by C, def add func)
        $.= \deriv{f}(\xi) - \Diffquot{f}{a}(b)$ (by  linear lem 1).

        $\deriv{h}(\xi) = 0$ (by B).
        Thus $\deriv{f}(\xi) = \Diffquot{f}{a}(b)$.
    End.
\end{proof}

\begin{definition}
    Let $S$ be a subset of $\mathbb{R}$.
    $f$ is constant on $S$ iff for all $x,y \in S$ $f(x) = f(y)$.
\end{definition}
 
\begin{proposition}[derivative zero then constant]
    Let $I$ be an interval and $f$ be continuous.
    If $\derivin{f}(x) = 0$ for all $x \in I$ then $f$ is constant on $I$.
\end{proposition}
\begin{proof}
    Assume $\derivin{f}(x) = 0$ for all $x \in I$.
    Assume $f$ is not constant on $I$. Take $a,b \in I$ such that $f(a) \neq f(b)$.
    $a = b$ or $a < b$ or $b < a$.

    Case $a = b$. Trivial. [timelimit 10]
    
    Case $a < b$. 
    Then take $\xi \in (a,b)_{OI}$ such that $\derivin{f}(\xi) = \Diffquot{f}{a}(b)$ (by mean value theorem). [timelimit 3]
    $\derivin{f}(\xi) = \Diffquot{f}{a}(b) = \frac{f(b) - f(a)}{b - a}$ (by def diffquot).
        $\frac{f(b) - f(a)}{b - a} \neq 0$ (by frac lemma 5). 
        Indeed $f(b) - f(a), b - a$ are nonzero. $\xi \in I$.
    Contradiction. End.

    Case $b < a$. [timelimit 20]
    Then take $\xi \in (b,a)_{OI}$ such that $\derivin{f}(\xi) = \Diffquot{f}{b}(a)$ (by mean value theorem). [timelimit 3]
    $\derivin{f}(\xi) = \Diffquot{f}{b}(a) = \frac{f(a) - f(b)}{a - b}$ (by def diffquot).
        $\frac{f(a) - f(b)}{a - b} \neq 0$ (by frac lemma 5). 
        Indeed $f(a) - f(b), a - b$ are nonzero. $\xi \in I$.
    End.
\end{proof}

\end{forthel}


\section{Monotonicity}

\begin{forthel}
Let $f$ denote a real map.
Let $S$ denote a subset of $\mathbb{R}$.

\begin{definition}
    $f$ is strictly increasing on $S$ iff 
    for all $x,y \in S$ if $x < y$ then $f(x) < f(y)$.
\end{definition}

\begin{definition}
    $f$ is strictly decreasing on $S$ iff
    for all $x,y \in S$ if $x < y$ then $f(y) < f(x)$.
\end{definition}

\begin{definition}
    $f$ is strictly monotone on $S$ iff $f$ is strictly increasing on $S$ or $f$ is strictly decreasing on $S$.
\end{definition}
Let $f$ is strictly monotone stand for $f$ is strictly monotone on $\dom(f)$.

\begin{definition}
    $f$ is increasing on $S$ iff for all $x,y \in S$ if $x \leq y$ then $f(x) \leq f(y)$.
\end{definition}

\begin{definition}
    $f$ is decreasing on $S$ iff for all $x,y \in S$ if $x \leq y$ then $f(x) \geq f(y)$.
\end{definition}

\begin{lemma}[subset monotone lemma]
    Let $f$ be strictly monotone on $S$. Let $U$ be a subset of $S$.
    Then $f$ is strictly monotone on $U$.
\end{lemma}
\begin{proof}
    Case $f$ is strictly increasing on $S$. Trivial.
    Case $f$ is strictly decreasing on $S$. Trivial.
\end{proof}[prover vampire]

\begin{lemma}[lemma minus f increasing]
    $f$ is strictly decreasing on $S$ iff $\fminus{f}$ is strictly increasing on $S$.
\end{lemma}
\begin{proof}
    Case $f$ is strictly decreasing on $S$. Let $x,y \in S$. Assume $x < y$. Then $\fminus{f}(x) < \fminus{f}(y)$.
        Indeed $-\fminus{f}(x) = f(x) > f(y) = -\fminus{f}(y)$. Indeed $f(x) > f(y)$.
    End. 

    Case $\fminus{f}$ is strictly increasing on $S$. 
        Let $x,y \in S$. Assume $x < y$. Then $f(x) > f(y)$. Indeed $-f(x) = \fminus{f}(x) < \fminus{f}(y) = -f(y)$.
    End.
\end{proof}

Let $E$ denote an interval. [timelimit 10]
\begin{corollary}[cor I intermediate value theorem]
    Let $f$ be a continuous real map such that $f$ be strictly increasing on $E$. Let $x,z \in E$.
    Then $(f(x),f(z))_{OI} = f[(x,z)_{OI}]$.
\end{corollary}
\begin{proof}
    Let us show that $(f(x),f(z))_{OI} \subseteq f[(x,z)_{OI}]$.
        Case $x < z$ (by intermediate value theorem I). End.
        Case $x \geq z$. Then $f(x) \geq f(z)$ and $(f(x),f(z))_{OI} = \emptyset$. End.
    End.

    Let us show that $f[(x,z)_{OI}] \subseteq (f(x),f(z))_{OI}$.
        Let $\upsilon \in f[(x,z)_{OI}]$. Take $y \in (x,z)_{OI}$ such that $f(y) = \upsilon$.
        Then $f(x) < f(y) < f(z)$. Indeed $x,y,z \in E$ and $x < y < z$ and $f$ is strictly increasing on $E$. 
        Therefore $\upsilon \in (f(x),f(z))_{OI}$.
    End.
\end{proof}

\begin{corollary}[cor II intermediate value theorem]
    Let $f$ be a continuous real map such that $f$ is strictly decreasing on $E$. 
    Let $x,z \in E$. Then $(f(z),f(x))_{OI} = f[(x,z)_{OI}]$.
\end{corollary}
\begin{proof}
    Let us show that $(f(z),f(x))_{OI} \subseteq f[(x,z)_{OI}]$.
        Case $x \geq z$. $f(x) \leq f(z)$. End.  
        Case $x < z$ (by intermediate value theorem II). End.
    End.

    Let us show that $f[(x,z)_{OI}] \subseteq (f(z),f(x))_{OI}$.
        Let $\upsilon \in f[(x,z)_{OI}]$. Take $y \in (x,z)_{OI}$ such that $f(y) = \upsilon$.
        Then $f(x) > f(y) > f(z)$. Indeed $x,y,z \in E$ and $x < y < z$ and $f$ is strictly decreasing on $E$.
        Therefore $\upsilon \in (f(z),f(x))_{OI}$.
    End.
\end{proof} [timelimit 3]


\begin{lemma}[derivative strictly increasing]
    Let $f$ be differentiable. Let $\derivin{f}(x) > 0$ for every $x \in E$. 
    Then $f$ is strictly increasing on $E$.
\end{lemma}
\begin{proof}[by contradiction]
    Assume the contrary. Take $x,y \in E$ such that $x < y$ and $f(x) \geq f(y)$. Indeed $f$ is not strictly increasing on $E$.
    Take $\xi \in (x,y)_{OI}$ such that $\derivin{f}(\xi) = \Diffquot{f}{x}(y)$ (by mean value theorem).
    $\xi \in E$.

    $\derivin{f}(\xi) \leq 0$. Proof.
        $f(y) - f(x) \leq f(x) - f(x) = 0$. $y - x > 0$.

        Therefore $\derivin{f}(\xi) = \Diffquot{f}{x}(y) = \frac{f(y) - f(x)}{y - x} \leq 0$ (by frac lemma 3).
    End.

    Contradiction.
\end{proof}

\begin{corollary}[derivative strictly decreasing]
    Let $f$ be differentiable and $\derivin{f}(x) < 0$ for every $x \in E$. 
    Then $f$ is strictly decreasing on $E$.
\end{corollary}
\begin{proof}
    $E$ is a subset of $\mathbb{R}$.
    $\fminus{f}$ is strictly increasing on $E$ (by derivative strictly increasing).
    Indeed for every $x \in E$ $\derivin{\fminus{f}}(x)= -\derivin{f}(x) > 0$.

    Thus $f$ is strictly decreasing on $E$.
    Proof. Let $x,y \in E$. Assume $x < y$. 
        $\fminus{f}(x) < \fminus{f}(y)$ (by lemma minus f increasing).
        Thus $-f(x) < -f(y)$ and $f(y) < f(x)$.
    End. 
\end{proof}

\begin{lemma}[derivative increasing]
    Let $f$ be differentiable and $\derivin{f}(x) \geq 0$ for every $x \in E$.
    Then $f$ is increasing on $E$.
\end{lemma}
\begin{proof}
    Let $x,y \in E$. Assume that $x < y$ and $f(x) > f(y)$.
    Take $\xi \in (x,y)_{OI}$ such that $\derivin{f}(\xi) = \Diffquot{f}{x}(y)$ (by mean value theorem).
    $\xi \in E$.

    $\derivin{f}(\xi) < 0$. Proof. [prover eprover]
        $f(y) - f(x) < f(x) - f(x) = 0$. Indeed $f(y) < f(x)$. $y - x > 0$.

        Therefore $\derivin{f}(\xi) = \Diffquot{f}{x}(y) = \frac{f(y) - f(x)}{y - x} < 0$ (by frac lemma 3, frac lemma 5).
    End.

    Contradiction.
\end{proof}
 
\begin{corollary}[derivative decreasing]
    Let $f$ be differentiable and $\derivin{f}(x) \leq 0$ for every $x \in E$. 
    Then $f$ is decreasing on $E$.
\end{corollary}
\begin{proof}
    $\fminus{f}$ is increasing on $E$ (by derivative increasing).
    Indeed for every $x \in E$ $\derivin{\fminus{f}}(x) = -\derivin{f}(x) \geq 0$.

    Thus $f$ is decreasing on $E$.
    Proof. Let $x,y \in E$. Assume $x < y$.
        Then
        
        $-f(x) = \fminus{f}(x) \leq \fminus{f}(y) = -f(y)$.
        Thus $-f(x) \leq -f(y)$ and $f(y) \leq f(x)$.
    End. 
\end{proof}


\end{forthel}


\section{Sufficient Condition for a Local Maximum}

If we know that $f'(\xi) = 0$ and $f''(\xi) < 0$, we want to show in (A) that $f$ is increasing on some interval $[\xi - \delta, \xi]$
and in (B) that $f$ is decreasing on $[\xi, \xi + \delta]$. Knowing (A) and (B) we can show that $\xi$ is a local maximum.

\begin{forthel}
Let $f$ denote a real map.
Let $\xi$ denote a real number.
       [prover eprover]
\begin{theorem}[sufficient condition local maximum]
    Let $f$ be differentiable and $\derivin{f}(\xi) = 0$.
    Let $\deriv{f}$ be differentiable in $\xi$ and $\derivin{\deriv{f}}(\xi) < 0$.
    Then $\xi$ is a local maximum of $f$.
\end{theorem}
\begin{proof}
    Take $\upsilon = \derivin{\deriv{f}}(\xi)$.
    Take $\varepsilon = -\upsilon$. $\varepsilon$ is a positive real number. [prover vampire]
    Take a positive real number $d$ such that $d$ is a domain restrictor for $\varepsilon$ and $\Diffquot{\deriv{f}}{\xi}$ around $\xi$ and $\upsilon$.
        Indeed $\limS{\Diffquot{\deriv{f}}{\xi}}{\xi} = \upsilon$. [prover eprover]
    Take a positive real number $\delta$ such that $\delta < d$.

    Take $I = [\xi - \delta, \xi]_{CI}$.
    Take $J = [\xi, \xi + \delta]_{CI}$. 
    I is an interval and J is an interval (by interval lemma).

    (A) $f$ is increasing on $I$ (by derivative increasing).
    Indeed we can show that for every $x \in I$ $\derivin{f}(x) \geq 0$.

        Assume the contrary. Take $x \in I$ such that $\derivin{f}(x) < 0$.

        $\abs{\Diffquot{\deriv{f}}{\xi}(x) - \upsilon} \geq \varepsilon$. Proof.

            (1) $\deriv{f}(\xi) = 0$ and $x - \xi$ is nonzero.
            $\Diffquot{\deriv{f}}{\xi}(x) = \frac{\deriv{f}(x) - \deriv{f}(\xi)}{x - \xi}$ (by def diffquot). 
            
            $\frac{\deriv{f}(x) - \deriv{f}(\xi)}{x - \xi} .= \frac{\deriv{f}(x)}{x - \xi}$ (by 1, eq lemma 0, existence zero).

            $\frac{\deriv{f}(x)}{x - \xi} > 0$ (by frac lemma 4, frac lemma 5).
                Indeed $x - \xi < 0$ and $\deriv{f}(x) < 0$.

            (AA) Thus $\Diffquot{\deriv{f}}{\xi}(x) > 0$.

            Therefore
            $\abs{\Diffquot{\deriv{f}}{\xi}(x) - \upsilon} 
            = \Diffquot{\deriv{f}}{\xi}(x) - \upsilon$ (by ineq lemma 21).

            $-\upsilon + \Diffquot{\deriv{f}}{\xi}(x) \geq -\upsilon + 0 = \varepsilon$ (by ineq axiom addition, AA).
        End.

        $\abs{x - \xi} < d$. Proof.
            $\abs{x - \xi} = \abs{\xi - x} = \xi - x \leq \delta < d$. Indeed $\xi - x \leq \xi - (\xi - \delta) = \delta$.
        End.

        [timelimit 20]
        $d$ is not a domain restrictor for $\varepsilon$ and $\Diffquot{\deriv{f}}{\xi}$ around $\xi$ and $\upsilon$ (by domain restrictor lemma 2).
            Indeed $x\in \mathbb{R} \setminus \sset{\xi}{\mathbb{R}}$ and $\Diffquot{\deriv{f}}{\xi}$ is a real map. [timelimit 3]
        Contradiction.
    qed.
  
    (B) $f$ is decreasing on $J$ (by derivative decreasing).
    Indeed we can show that for every $x \in J$ $\derivin{f}(x) \leq 0$. 
        Assume the contrary. Take $x \in J$ such that $\derivin{f}(x) > 0$.

        $\abs{\Diffquot{\deriv{f}}{\xi}(x) - \upsilon} \geq \varepsilon$. Proof.
            $\Diffquot{\deriv{f}}{\xi}(x)
            = \frac{\deriv{f}(x) - \deriv{f}(\xi)}{x - \xi}
            = \frac{\deriv{f}(x)}{x - \xi}$.

            $\frac{\deriv{f}(x)}{x - \xi} > 0$ (by frac lemma 1, frac lemma 5).
                Indeed $x - \xi > 0$ and $\deriv{f}(x) > 0$.

            (BB) Thus $\Diffquot{\deriv{f}}{\xi}(x) > 0$.

            Therefore 
            $\abs{\Diffquot{\deriv{f}}{\xi}(x) - \upsilon}
            = \Diffquot{\deriv{f}}{\xi}(x) - \upsilon$ (by ineq lemma 21). 
            
            $-\upsilon + \Diffquot{\deriv{f}}{\xi}(x) \geq -\upsilon + 0 = \varepsilon$ (by ineq axiom addition, BB).
        End.

        $\abs{x - \xi} < d$. Proof.
            $\abs{x - \xi} = x - \xi \leq \delta < d$. Indeed $x \leq \xi + \delta$.
        End.
        
        [timelimit 20]
        $d$ is not a domain restrictor for $\varepsilon$ and $\Diffquot{\deriv{f}}{\xi}$ around $\xi$ and $\upsilon$ (by domain restrictor lemma 2).
            Indeed $x\in \mathbb{R} \setminus \sset{\xi}{\mathbb{R}}$ and $\Diffquot{\deriv{f}}{\xi}$ is a real map. [timelimit 3]
        Contradiction.
    qed. 

    $\delta$ is a maximum domain restrictor for $f$ around $\xi$. Proof.
        $f(x) \leq f(\xi)$ for all $x \in I$. $f(\xi) \geq f(x)$ for all $x \in J$.
        $\B_{\delta}(\xi) = (\xi - \delta, \xi + \delta)_{OI} \subseteq [\xi - \delta, \xi + \delta]_{CI} \subseteq I \cup J$.
    End.

    Therefore $\xi$ is a local maximum of $f$.
\end{proof}

\begin{corollary}[sufficient condition local minimum]
    Let $f$ be differentiable and $\derivin{f}(\xi) = 0$.
    Let $\deriv{f}$ be differentiable in $\xi$ and $\derivin{\deriv{f}}(\xi) > 0$.
    Then $\xi$ is a local minimum of $f$.
\end{corollary}
\begin{proof}
    $\fminus{f}$ is differentiable and $\derivin{\fminus{f}}(\xi) = 0$.
    $\deriv{\fminus{f}}$ is differentiable in $\xi$ and $\derivin{\deriv{\fminus{f}}}(\xi) < 0$.
    Then $\xi$ is a local maximum of $\fminus{f}$ (by sufficient condition local maximum).
    Therefore $\xi$ is a local minimum of $f$.
\end{proof}

Let $U$ denote an interval.
\begin{definition}
    $f$ is strictly convex on $U$ iff for every $x \in U$ $f$ is differentiable in $x$ and $\deriv{f}$ is differentiable in $x$ and 
    $\derivin{\deriv{f}}(x) < 0$.
\end{definition}
Let $f$ is strictly convex stand for $f$ is strictly convex on $\mathbb{R}$.

\begin{definition}
    A maximum of $f$ on $U$ is an element $\xi$ of $U$ such that
    for every $x \in U$ $f(\xi) \geq f(x)$.
\end{definition}
Let a maximum of $f$ stand for a maximum of $f$ on $\dom(f)$. [prover vampire]

\begin{corollary}[sufficient condition for maximum on interval]
    Let $f$ be strictly convex on $U$. Let $\xi$ be an element of $U$.
    If $\xi$ is a local maximum of $f$ then $\xi$ is a maximum of $f$ on $U$.
\end{corollary}
\begin{proof}
    Let $\xi$ be a local maximum of $f$. (A) Then $\derivin{f}(\xi) = 0$ (by derivative of maximum). 
    Define $I = \{x \in U | x \leq \xi\}$.
    Define $J = \{x \in U | x \geq \xi\}$. [timelimit 10]
    $I \cup J = U$. $I$ is an interval and $J$ is an interval.
    
    Let us show that $f(\xi) \geq f(x)$ for any $x \in I$.
        $\deriv{f}$ is decreasing on $I$ (by derivative decreasing). Indeed for every $x \in I$ $\derivin{\deriv{f}}(x) \leq 0$.
        For every $x \in I$ $\derivin{f}(x) = \deriv{f}(x) \geq \deriv{f}(\xi) = 0$. [timelimit 20]
        Therefore $f$ is increasing on $I$ (by derivative increasing). [timelimit 3]
    End.

    Let us show that $f(\xi) \geq f(x)$ for any $x \in J$.
        $\deriv{f}$ is decreasing on $J$ (by derivative decreasing). Indeed for every $x \in J$ $\derivin{\deriv{f}}(x) \leq 0$.
        For every $x \in J$ $\derivin{f}(x) = \deriv{f}(x) \leq \deriv{f}(\xi) = 0$. [timelimit 20]
        Therefore $f$ is decreasing on $J$ (by derivative decreasing). [timelimit 3]
    End.
\end{proof}

\begin{lemma}[maximum is local maximum]
    Let $E$ be an open neighbourhood of $\xi$ such that $E$ is an interval.
    Let $\xi$ is a maximum of $f$ on $E$. Then $\xi$ is a local maximum of $f$.
\end{lemma}
\begin{proof}
    Take a positive real number $\delta$ such that $\B_{\delta}(\xi) \subseteq E$.
    Then $\delta$ is a maximum domain restrictor for $f$ around $\xi$.
\end{proof}[prover eprover]
\end{forthel}


\end{document}
