\documentclass[../solow_main.tex]{subfiles}

\begin{document}

\chapter{Preparations}\label{chapter:preparation}

\begin{forthel}
    [read \path{examples/100_theorems.ftl.tex}]
\end{forthel}

In the first section we want to name all those small lemmata that we will need for the following formalization.
Especially for Frerix' rewriting algorithm any rewriting step needs to be explicitly mentioned. Therefore, many 
small lemmata are mentioned here, although automated theorem provers do not have any difficulty proving those theorems
even without any provided proof. In the subsection \ref{Prep.old} some lemmata, axioms and definitions are 
repeated as lemmata that are already given in \texttt{100\_theorems.ftl.tex}, just in order to give them names so that
they can be referenced later.

One term we already define here: as a "real map" we understand a function from and onto the real numbers.

\section{Old Lemmata}\label{Prep.old}

\begin{forthel}

\begin{lemma}[composition]
Let $F,G$ be maps such that $\range(G) \subseteq \dom(F)$.
$F \circ G$ is a map $H$ such that $\dom(H) = \dom(G)$ and
$H(x) = F(G(x))$ for all $x \in \dom(H)$.
\end{lemma}

\begin{lemma}[inverse]
Let $F$ be an injective map.
$\inv{F}$ is the map $G$ such that $\dom(G) = \range(F)$
and for all $v \in \dom(G)$ $G(v) \in \dom(F)$ and $F(G(v))=v$.
\end{lemma}


Let $x,y,z$ denote real numbers.

\begin{lemma}[add comm]
$x + y = y + x$.
\end{lemma}

\begin{lemma}[add ass]
$(x + y) + z = x + (y + z)$.
\end{lemma}

\begin{lemma}[existence zero]
$0$ is a real number such that
for every real number $x$ $x + 0 = x$.
\end{lemma}

\begin{lemma}[additive inverse]
$-x$ is a real number such that $x + (-x) = 0$.
\end{lemma}

\begin{lemma}[mult comm]
$x \cdot y = y \cdot x$.
\end{lemma}

\begin{lemma}[mult ass]
$((x \cdot y)) \cdot z = x \cdot (y \cdot z)$.
\end{lemma}

\begin{lemma}[existence one]
$1$ is a real number such that $1 \neq 0$ and
for every real number $x$ $1 \cdot x = x$.
\end{lemma}

\begin{lemma}[multiplicative inverse]
Assume $x \neq 0$. $1/x$ is a real number
such that $x \cdot (1/x) = 1$.
\end{lemma}

\begin{lemma}[distributive law]
$x \cdot (y + z) = (x \cdot y) + (x \cdot z)$.
\end{lemma}

\begin{proposition}[eq lemma 1]
$((y \cdot x)) + (z \cdot x) = (y + z) \cdot x$.
\end{proposition}

\begin{proposition}[eq lemma 2]
Let $x + y = x + z$. Then $y = z$.
\end{proposition}

\begin{proposition}[eq lemma 3]
If $x + y = x$ then $y = 0$.
\end{proposition}

\begin{proposition}[eq lemma 4]
If $x + y = 0$ then $y = -x$.
\end{proposition}

\begin{proposition}[eq lemma 45]
$-(-x) = x$.
\end{proposition}


\begin{proposition}[eq lemma 5]
Let $x$ be nonzero and $x \cdot y = x \cdot z$.
Then $y = z$.
\end{proposition}

\begin{proposition}[eq lemma 6]
If $x$ is nonzero and $x \cdot y = x$ then $y = 1$.
\end{proposition}

\begin{proposition}[eq lemma 7]
If $x$ is nonzero and $x \cdot y = 1$ then $y = 1/x$.
\end{proposition}

\begin{proposition}[eq lemma 8]
If $x$ is nonzero then $1/(1/x) = x$.
\end{proposition}

\begin{proposition}[eq lemma 9]
$0 \cdot x = 0$.
\end{proposition}

\begin{proposition}[eq lemma 10]
If $x$ is nonzero and $y \neq 0$ then $x \cdot y \neq 0$.
\end{proposition}

\begin{proposition}[eq lemma 11]
$(-x) \cdot y = -(x \cdot y)$.
\end{proposition}

\begin{proposition}[eq lemma 12]
$-x = -1 \cdot x$.
\end{proposition}

\begin{proposition}[eq lemma 13]
$(-x) \cdot (-y) = x \cdot y$.
\end{proposition}

\begin{lemma}[eq lemma 14]
$(1 - x) \cdot y = y - (x \cdot y)$.
\end{lemma}

\begin{lemma}[eq lemma 15]
$-(x + y) = -x - y$.
\end{lemma}

\begin{lemma}[eq lemma 16]
    Let $x$ be a real number.
    $x + x = 2 \cdot x$.
\end{lemma}
\begin{proof}
    $x + x = (1 \cdot x) + (1 \cdot x) = (1 + 1) \cdot x = 2 \cdot x$.
\end{proof}

\begin{lemma}[eq lemma 17]
Let $z \neq 0$. Then
$x = \frac{z \cdot x}{z}$.
\end{lemma}


\begin{lemma}[eq lemma 18]
Let $w$ be a nonzero real number. Then
$\frac{x - y}{w} + \frac{y - z}{w} = \frac{x - z}{w}$.
\end{lemma}


\begin{lemma}[reflexivity]
$x \leq x$.
\end{lemma}

\begin{lemma}[antisymmetry]
If $x \leq y$ and $x \geq y$ then $x=y$.
\end{lemma}

\begin{lemma}[transitivity]
If $x \leq y$ and $y \leq z$ then $x \leq z$.
\end{lemma}

\begin{lemma}[completeness]
$x \leq y$ or $x \geq y$.
\end{lemma}

\begin{lemma}[ineq axiom addition]
If $y \leq z$ then $x + y \leq x + z$ and $y + x \leq z + x$.
\end{lemma}

\begin{lemma}[ineq axiom multiplication]
If $x \geq 0$ and $y \leq z$ then $x \cdot y \leq x \cdot z$.
\end{lemma}


\begin{lemma}[ineq lemma 1]
$x \leq y$ iff $y \nless x$.
\end{lemma}

\begin{proposition}[ineq lemma 2]
$x > 0$ iff $-x < 0$.
\end{proposition}

\begin{proposition}[ineq lemma 3]
If $x \neq 0$ then $x \cdot x > 0$.
\end{proposition}

\begin{proposition}[ineq lemma 4]
$1 > 0$.
\end{proposition}

\begin{proposition}[ineq lemma 5]
$x \leq y$ iff $-x \geq -y$.
\end{proposition}

\begin{proposition}[ineq lemma 6]
If $x < 0$ and $y < z$ then
$x \cdot y > x \cdot z$.
\end{proposition}

\begin{proposition}[ineq lemma 7]
$-1 < 0$.
\end{proposition}

\begin{proposition}[ineq lemma 8]
$x - 1 < x$.
\end{proposition}

\begin{proposition}[ineq lemma 9]
$x < y < x + 1$ iff $0 < y - x < 1$.
\end{proposition}

\begin{proposition}[ineq lemma 10]
If $0 < y$ then $0 < 1/y$.
\end{proposition}

\begin{lemma}[ineq lemma 11]
If $\neg x \leq y$ then $x \geq y$.
\end{lemma}

\begin{proposition}[ineq lemma 12]
Assume $0 < x \leq y$.
Then $1/y \leq 1/x$.
\end{proposition}

Let $E$ denote a subset of $\mathbb{R}$.

\begin{lemma}[completeness axiom]
Assume that $E$ is nonempty and bounded above.
Then $E$ has a supremum.
\end{lemma}

\begin{lemma}[potence lemma 1]
$x^{1} = x$.
\end{lemma}

\begin{lemma}[potence lemma 2]
$x^{2} = x \cdot x$.
\end{lemma}

\begin{lemma}[potence lemma 3]
$x \cdot x^{i} = x^{i+1}$ for any natural number $i$.
\end{lemma}

\end{forthel}












\section{New Lemmata}

\begin{forthel}
Let a real map stand for a map from $\mathbb{R}$ to $\mathbb{R}$.

Let $w,x,y,z$ denote real numbers. [timelimit 10]
\begin{lemma}[fraction lemma]
    Assume $y \cdot z \neq 0$.
    Then $y \neq 0$ and $z \neq 0$ and $\frac{x}{y \cdot z} = \frac{x}{y} \cdot 1/z$.
\end{lemma}
\begin{proof}
    If $y = 0$ or $z = 0$ then $y \cdot z = 0$. Therefore $y,z$ are nonzero.
    $1 / (y \cdot z) = 1/y \cdot 1/z$.
    Indeed then $(y \cdot z) \cdot (1/y \cdot 1/z) = 1$. 
\end{proof}

\begin{lemma}[fraction lemma 2]
    Assume $y$ is nonzero.
    Then $y \cdot \frac{x}{y} = x$.
\end{lemma}

\begin{lemma}[frac lemma 1]
    Let $x \geq 0$ and $y > 0$.
    Then $\frac{x}{y} \geq 0$.
\end{lemma}

\begin{lemma}[frac lemma 2]
    Let $x \geq 0$ and $y < 0$.
    Then $\frac{x}{y} \leq 0$.
\end{lemma}
\begin{proof}
    $\frac{x}{-y} \geq 0$ (by frac lemma 1). Indeed $-y$ is positive.
    $-\frac{x}{y} = \frac{x}{-y}$.
\end{proof}

\begin{lemma}[frac lemma 3]
    Let $x \leq 0$ and $y > 0$.
    Then $\frac{x}{y} \leq 0$.
\end{lemma}
\begin{proof}
    $\frac{-x}{y} \geq 0$ (by frac lemma 1). Indeed $-x \geq 0$.
    $-\frac{x}{y} = \frac{-x}{y}$.
\end{proof}

\begin{lemma}[frac lemma 4]
    Let $x \leq 0$ and $y < 0$.
    Then $\frac{x}{y} \geq 0$.
\end{lemma}
\begin{proof}
    $ \frac{-x}{y} \leq 0$ (by frac lemma 2). Indeed $-x \geq 0$.
\end{proof}

\begin{lemma}[frac lemma 5]
    Let $x,y$ be nonzero.
    $\frac{x}{y} \neq 0$.
\end{lemma}


\begin{lemma}[eq lemma 0]
    $0 = -0$.
\end{lemma}

\begin{lemma}[eq lemma 20]
    Assume $z$ is nonzero.
    If $x = y \cdot z$ then $y = \frac{x}{z}$.
\end{lemma}

\begin{lemma}[eq lemma 21]
    $(-x) \cdot y = x \cdot (-y)$.
\end{lemma}

\begin{lemma}[eq lemma 26]
    Let $x,y$ be nonzero.
    $\frac{x}{y} \cdot \frac{y}{x} = 1$.
\end{lemma}

\begin{lemma}[eq lemma 27]
    $x + y = z$ iff $x = z - y$.
\end{lemma}

\begin{lemma}[eq lemma 28]
    Let $x$ be nonzero.
    $-1/x = 1/(-x)$.
\end{lemma}
\begin{proof}
    $(-x) \cdot (-1/x) = 1$.
\end{proof}

\begin{lemma}[eq lemma 29]
    Let $x,y$ be nonzero.
    $\frac{x}{x \cdot y} = 1/y$.
\end{lemma}

\begin{lemma}[eq lemma 30]
    Let $x,y$ be nonzero.
    $1/x \cdot 1/y = 1/(x \cdot y)$.
\end{lemma}
\begin{proof}
    $(x \cdot y) \cdot (1/x \cdot 1/y)
    .= ((x \cdot y) \cdot 1/x) \cdot 1/y$ (by mult ass, mult comm)

    $.= 1$ (by mult ass, mult comm, multiplicative inverse, existence one).
\end{proof}

\begin{lemma}[eq lemma 22]
    Let $x,y$ be nonzero.
    $1/\frac{x}{y} = \frac{y}{x}$.
\end{lemma}
\begin{proof}
    $1/\frac{x}{y} = \frac{y}{x}$ (by eq lemma 7, eq lemma 26). Indeed $\frac{x}{y}$ is a nonzero real number.
\end{proof}

\begin{lemma}[eq lemma 23]
    $\frac{x}{2} = x - \frac{x}{2}$.
\end{lemma}
\begin{proof}
    $\frac{x}{2} + \frac{x}{2} .= 2 \cdot \frac{x}{2}$ (by eq lemma 16)

    $.= x$ (by fraction lemma 2).
    $\frac{x}{2} = x - \frac{x}{2}$ (by eq lemma 27).
\end{proof}

\begin{lemma}[eq lemma 24]
    Let $x,y$ be nonzero.
    $1/x + 1/y = \frac{x + y}{x \cdot y}$.
\end{lemma}
\begin{proof}
    $1/x + 1/y .= \frac{y}{y \cdot x} + \frac{x}{x \cdot y}$ (by eq lemma 29)
    $.= \frac{x + y}{x \cdot y}$ (by mult comm, eq lemma 1, add comm).
\end{proof}

\begin{lemma}[eq lemma 25]
    Let $x,y$ be nonzero.
    $1/x - 1/y = \frac{y - x}{x \cdot y}$.
\end{lemma}
\begin{proof}
    $1/x - 1/y .= \frac{x - y}{x \cdot (-y)}$ (by eq lemma 28, eq lemma 24).
    $\frac{x - y}{x \cdot (-y)} .= \frac{-(y - x)}{x \cdot (-y)}$ (by eq lemma 15, eq lemma 45, add comm)
    $.= \frac{-(y - x)}{-(x \cdot y)}$ (by eq lemma 21, eq lemma 11)
    $.= \frac{y - x}{x \cdot y}$ (by eq lemma 28, eq lemma 13).
\end{proof}



\begin{lemma}
    Let $y$ be positive.
    $x - y < x + y$.
\end{lemma}

\begin{lemma}[ineq lemma 13]
    If $w < x$ and $y < z$ then $w + y < x + z$.
\end{lemma}

\begin{lemma}[ineq lemma 14]
    If $y < z$ then $y + x < z + x$. 
\end{lemma}

\begin{lemma}[ineq lemma 145]
    If $y \leq z$ then $y + x \leq z + x$.
\end{lemma}

\begin{lemma}[ineq lemma 15]
    $x - y < z$ iff $x < z + y$.
\end{lemma}
\begin{proof}
    Case $x - y < z$. $x = (x - y) + y < z + y$. End.

    Case $x < z + y$. $x - y < (z + y) - y = z$. End.
\end{proof}

\begin{lemma}[ineq lemma 16]
    Let $w,x,y,z$ be nonnegative real numbers such that $w < x$ and $y < z$.
    Then $w \cdot y < x \cdot z$.
\end{lemma}
\begin{proof}
    $w \cdot y \leq w \cdot z = z \cdot w \leq z \cdot x$ (by ineq axiom multiplication, mult comm).
    $w \cdot y \neq x \cdot z$.
\end{proof}

\begin{lemma}[ineq lemma 17]
    Let $x,y,z$ be nonnegative.
    If $y < z$ then $x \cdot y \leq x \cdot z$.
\end{lemma}

\begin{lemma}[ineq lemma 18]
    $x - y < z$ iff $x - z < y$.
\end{lemma}
\begin{proof}
    $x - y < z$ iff $x < z + y$ (by ineq lemma 15).
    
    $x < y + z$ iff $x - z < y$ (by ineq lemma 15).
\end{proof}

\begin{lemma}[ineq lemma x1]
    Let $y$ be positive.
    Then there exists a positive real number $x$ such that $x < y$.
\end{lemma}

\begin{lemma}[ineq lemma 19]
    Let $x,y$ be positive.
    Then there exists a positive real number $z$ such that $z < x$ and $z < y$.
\end{lemma}
\begin{proof}
    Case $x \leq y$. There exists a positive real number $z$ such that $z < x$. End.

    Case $x > y$. There exists a positive real number $z$ such that $z < y$. End.
\end{proof}

\begin{lemma}[ineq lemma 195]
    Let $x,y,z$ be positive.
    Then there exists a positive real number $w$ such that 
    $w < x$ and ($w < y$ and $w < z$).
\end{lemma}
\begin{proof}
    Take a positive real number $v$ such that $v < x$ and $v < y$ (by ineq lemma 19).
    Take a positive real number $w$ such that $w < v$ and $w < z$ (by ineq lemma 19).
    Then $w < x$ and $w < y$ and $w < z$.
\end{proof}

\begin{lemma}[ineq lemma 20]
    Assume $x < y$.
    There exists a real number $z$ such that $x < z < y$.
\end{lemma}
\begin{proof}
    $y - x$ is a positive real number.
    Take a positive real number $z$ such that $z < y - x$ (by ineq lemma x1).
    $x + z$ is a real number and $x < x + z < y$.
\end{proof}

\begin{lemma}[ineq lemma 21]
    Let $x,y$ be nonnegative.
    $x + y$ is a nonnegative real number.
\end{lemma}

\begin{lemma}[ineq lemma 215]
    Let $x,y$ be positive.
    $x \cdot y$ is a positive real number.
\end{lemma}
\begin{lemma}[ineq lemma 22]
    Assume $y \geq z$.
    $x - y \leq x - z$.
\end{lemma} 
\begin{proof}
    $-y \leq -z$. Therefore $x - y \leq x - z$.
\end{proof}

\begin{lemma}[ineq lemma 23]
    Let $y$ be nonnegative.
    $x + y \geq x$.
\end{lemma}

\begin{lemma}[ineq lemma 24]
    Assume $y$ is positive.
    $x - y < x$.
\end{lemma}

\begin{lemma}[ineq lemma 25]
    Let $z$ be positive.
    If $x < y$ then $x \cdot z < y \cdot z$.
\end{lemma}

\begin{lemma}[ineq lemma 26]
    If $x \geq 0$ and $y \leq z$ then $y \cdot x \leq z \cdot x$.
\end{lemma}

\begin{lemma}[ineq lemma 27]
    $x \leq y$ iff $x - y \leq 0$.
\end{lemma}

\begin{lemma}[ineq lemma 28]
    $x < y - z$ iff $x + z < y$.
\end{lemma}
\begin{proof}
    Case $x < y - z$. $x + z < (y - z) + z = y$. End. 
    
    Case $x + z < y$. $x = (x + z) - z. (x + z) - z < y - z$ (by ineq lemma 14). End.
\end{proof}

\begin{lemma}[ineq lemma 285]
   $x + z < y$ iff $z < y - x$.
\end{lemma}

\begin{lemma}[ineq lemma 29]
    If $x \leq y < z$ then $x < z$.
\end{lemma}

\begin{lemma}[ineq lemma 30]
    If $x - y \geq z$ then $x - z \geq y$.
\end{lemma}



\begin{lemma}[inverse lemma]
    Let $f$ be an injective map. For all $x \in \dom(f)$ and all $y \in \range(f)$
    $f(x) = y$ iff $\inv{f}(y) = x$.
\end{lemma}

\begin{lemma}[function lemma 2]
    Let $f,g$ be maps such that $\dom(f) = \dom(g)$.
    If for every $x \in \dom(g)$ $f(x) = g(x)$ then $f = g$.
\end{lemma}

\begin{lemma}
    Let $f$ be a real map and $x$ be a real number.
    Then $f(x)$ is a real number.
\end{lemma}
\begin{proof}
    $f(x) \in \mathbb{R}$.
\end{proof}

\begin{lemma}
    Let $f,g$ be real maps.
    $g \circ f$ is a real map.
\end{lemma}
\begin{proof}
    $\dom(g \circ f) = \dom(f) = \mathbb{R}$.
\end{proof}

\begin{lemma}[1 is positive]
    1 is a positive real number.
\end{lemma}

\begin{lemma}[lemma x01]
    Let $S$ be a nonempty subset of $\mathbb{R}$.
    Let $s$ be the supremum of $S$. Let $\varepsilon$ be a positive real number.
    Then there exists $x \in S$ such that $s-\varepsilon < x$.
\end{lemma}
\begin{proof}
    Take $K = s-\varepsilon$.
    Assume the contrary. Then $s-\varepsilon$ is an upper bound of $S$.
    Indeed we can show that for all $x \in S$ $x \leq K$.
        Assume the contrary. Take $x \in S$ such that $x > K$.
        Case $x > s$. $x \notin S$. End.
        Case $x \leq s$. Then there exists $\xi \in S$ such that $s-\varepsilon < \xi$. End.
    End.
\end{proof}

\begin{lemma}[lemma inverse]
    Let $f$ be an injective map.
    For all $v \in \dom(f)$ $f(v) \in \dom(\inv{f})$ and $\inv{f}(f(v)) = v$.
\end{lemma}
\begin{proof}
    Let $v \in \dom(f)$. Then $f(v) \in \dom(\inv{f})$. Indeed $\range(f) = \dom(\inv{f})$. Take $w = \inv{f}(f(v))$.
    $f(w) .= f(\inv{f}(f(v))) .= f(v)$ (by inverse). $w = v$. Indeed $f$ is injective.
\end{proof}
[timelimit 3]
\end{forthel}

\end{document}
