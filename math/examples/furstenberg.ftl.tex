\documentclass{article}
\usepackage{lang/naproche}
\usepackage{hyperref}
\hypersetup{
  colorlinks=true
}

\renewcommand{\mod}{\text{mod }}
\newcommand{\Int}{\mathbb{Z}}

\title{Furstenberg's Proof of the Infinitude of Primes}
\author{Andrei Paskevich (2007), Steffen Frerix (2018), Marcel Schütz (2021)}
\date{}


\begin{document}
  \maketitle

  \noindent
  Furstenberg's proof of the infinitude of primes is a topological proof of the fact that there are
  infinitely many primes. It was published 1955 while Furstenberg was still an undergraduate
  student \cite{Furstenberg1955}.

  \section{Integers}

  The central idea of Furstenberg's proof is to define a certain topology on $\Int$ from the
  properties of which we can deduce that the set of primes is infinite. So first we have to
  introduce the ring $\Int$ of integers. In fact, we do not need to give a full axiomatization of
  integer arithmetic; it suffices to assume that $\Int$ is an integral domain.

  Let us start by introducing the signature of the ring of integers.

  \begin{forthel}
    [read \path{examples/preliminaries.ftl.tex}]

    \begin{signature}\label{Integers}
      An integer is an object.
    \end{signature}

    Let $a,b,c,d,i,j,k,l,m,n$ stand for integers.

    \begin{signature}\label{IntZero}
      $0$ is an integer.
    \end{signature}

    \begin{signature}\label{IntOne}
      $1$ is an integer.
    \end{signature}

    \begin{signature}\label{IntNeg}
      $-a$ is an integer.
    \end{signature}

    \begin{signature}\label{IntPlus}
      $a + b$ is an integer.
    \end{signature}

    \begin{signature}\label{IntMult}
      $a \cdot b$ is an integer.
    \end{signature}

    Let $a - b$ stand for $a + (-b)$.
  \end{forthel}

  Moreover, we assume $(\Int, 0, +, -)$ to be an abelian group and $(\Int, 1, \cdot)$ to be a
  commutative monoid which satisfy the distribution laws.

  \begin{forthel}
    \begin{axiom}\label{AddAsso}
      $a + (b + c) = (a + b) + c$.
    \end{axiom}

    \begin{axiom}\label{AddComm}
      $a + b = b + a$.
    \end{axiom}

    \begin{axiom}\label{AddZero}
      $a + 0 = a = 0 + a$.
    \end{axiom}

    \begin{axiom}\label{AddNeg}
      $a - a = 0 = -a + a$.
    \end{axiom}

    \begin{axiom}\label{MulAsso}
      $a \cdot (b \cdot c) = (a \cdot b) \cdot c$.
    \end{axiom}

    \begin{axiom}\label{MulComm}
      $a \cdot b = b \cdot a$.
    \end{axiom}

    \begin{axiom}\label{MulOne}
      $a \cdot 1 = a = 1 \cdot a$.
    \end{axiom}

    \begin{axiom}\label{Distrib}
      $a \cdot (b + c) = (a \cdot b) + (a \cdot c)$ and$ (a + b) \cdot c = (a \cdot c) + (b \cdot c)$.
    \end{axiom}

    \begin{lemma}\label{MulZero}
      $a \cdot 0 = 0 = 0 \cdot a$.
    \end{lemma}
    \begin{proof}
      $a \cdot 0 = a \cdot (1 - 1) = a - a = 0$.
    \end{proof}

    \begin{lemma}\label{MulMinOne}
      $-1 \cdot a = -a = a \cdot -1$.
    \end{lemma}
    \begin{proof}
      $(-1 \cdot a) + a = 0$.
    \end{proof}
  \end{forthel}

  Furthermore, we assume that our ring is not trivial and that there are no non-trivial zero
  divisors.

  \begin{forthel}
    \begin{axiom}\label{NonTriv}
      $0 \neq 1$.
    \end{axiom}

    \begin{axiom}\label{ZeroDiv}
      $a \neq 0 \wedge b \neq 0 \implies a \cdot b \neq 0$.
    \end{axiom}

    \begin{lemma}
      $-(-a)$ is an integer.
    \end{lemma}
  \end{forthel}

  Let us continue with the notion of divisors and congruency.

  \begin{forthel}
    Let $a$ is nonzero stand for $a \neq 0$.
    Let $p,q$ stand for nonzero integers.

    \begin{definition}\label{Divisor}
      A divisor of $b$ is a nonzero integer $a$ such that for some $n$ $(a \cdot n = b)$.
    \end{definition}

    Let $a$ divides $b$ stand for $a$ is a divisor of $b$.
    Let $a \mid b$ stand for $a$ is a divisor of $b$.

    \begin{definition}\label{EquMod}
      $a = b ~(\mod q)$ iff $q \mid a-b$.
    \end{definition}

    \begin{lemma}\label{EquModRef}
      $a = a ~(\mod q)$.
    \end{lemma}

    \begin{lemma}\label{EquModSym}
      $a = b ~(\mod q) \implies b = a ~(\mod q)$.
    \end{lemma}
    \begin{proof}
      Assume that $a = b ~(\mod q)$.

      (1) Take $n$ such that $q \cdot n = a - b$.

      $q \cdot (-n) .= (-1) \cdot (q \cdot n)$ (by MulMinOne, MulAsso, MulComm) $.= (-1) \cdot (a - b)$ (by 1).
    \end{proof}

    \begin{lemma}\label{EquModTrn}
      $a = b ~(\mod q) \wedge b = c ~(\mod q) \implies a = c ~(\mod q)$.
    \end{lemma}
    \begin{proof}
      Assume that $a = b ~(\mod q) \wedge b = c ~(\mod q)$.
      Take $n$ such that $q \cdot n = a - b$.
      Take $m$ such that $q \cdot m = b - c$.
      We have $q \cdot (n + m) = a - c$.
    \end{proof}

    \begin{lemma}\label{EquModMul}
      $a = b ~(\mod p \cdot q) \implies a = b ~(\mod p) \wedge a = b ~(\mod q)$.
    \end{lemma}
    \begin{proof}
      Assume that $a = b ~(\mod p \cdot q)$.
      Take $m$ such that $(p \cdot q) \cdot m = a - b$.
      We have $p \cdot (q \cdot m) = a - b = q \cdot (p \cdot m)$.
    \end{proof}
  \end{forthel}

  Note that up to now every finite field could be a model of our theory. But since we want to prove
  that there are \textit{infinitely} many primes we must eventually add some axiom which eliminates
  such models.

  This is done by introducing the notion of prime integers. All we need to know about them for
  Furstenberg's proof is that every integer $n$ has a prime divisor iff $n \neq 1$ and $n \neq -1$.

  \begin{forthel}
    \begin{signature}\label{Prime}
      $a$ is prime is an atom.
    \end{signature}

    Let a prime stand for a prime nonzero integer.

    \begin{axiom}\label{PrimeDivisor}
      $n$ has a prime divisor iff $n \neq 1 \wedge n \neq -1$.
    \end{axiom}
  \end{forthel}

  Let us assume that some finite field is a model of our current theory. Recall that in any finite
  field every non-zero element is a divisor of $1$. Let us rephrase axiom \textit{PrimeDivisor}:

  \begin{center}
    $n$ has no prime divisor iff $n = 1 \vee n = -1$.
  \end{center}

  Then we immediately see that $1$ has no prime divisor. But since every non-zero element is a
  divisor of $1$, no non-zero element is prime. Hence $0$ has also no prime divisor (recall that any
  divisor must be non-zero). But then, again by the axiom, we get $0 = 1 \vee 0 = -1$, a
  contradiction (by \textit{NonTriv}).

  \section{Finite classes}

  Another important notion is that of finite subsets of $\Int$. We leave the characterization of
  what it means for a set to be finite for the next sections.

  \begin{forthel}
    Let $S,T$ stand for classes.

    \begin{signature}\label{FinSet}
      $S$ is finite is an atom.
    \end{signature}

    Let $S$ is infinite stand for $S$ is not finite.
  \end{forthel}


  \section{Sets of integers}

  Since Furstenberg's proof is a topological proof we have to define unions, intersections and also
  complements of sets.

  \begin{forthel}
    \begin{definition}
      $\Int$ is the class of integers.
    \end{definition}

    \begin{axiom}
      $\Int$ is a set.
    \end{axiom}

    Let $A,B,C,D$ stand for subclasses of $\Int$.

    \begin{definition}\label{IntegerSets}
      A family of integer sets is a class $S$ such that every element of $S$ is a subset of $\Int$.
    \end{definition}

    \begin{definition}\label{UnionSet}
      Let $S$ be a family of integer sets.
      $\bigcup S = \{ x \in \Int \mid x \text{ belongs to some element of } S \}$.
    \end{definition}

    \begin{lemma}
      Let $S$ be a family of integer sets.
      $\bigcup S$ is a subset of $\Int$.
    \end{lemma}

    \begin{definition}\label{Complement}
      $\overline{A} = \{ x \in \Int \mid x \text{ does not belong to } A \}$.
    \end{definition}

    \begin{lemma}
      $\overline{A}$ is a subset of $\Int$.
    \end{lemma}
   \end{forthel}


  \section{Introducing topology}

  In our next step towards a suitable topology on $\Int$ let us define arithmetic sequences, i.e.
  sets of the form $q \Int + a$.

  \begin{forthel}
    Let $a,b,c$ denote integers.
    Let $p,q$ stand for nonzero integers.
    Let $A,B,C$ stand for subsets of $\Int$.

    \begin{definition}\label{ArSeq}
      $q \Int + a = \{ b \in \Int \mid b = a ~(\mod q) \}$.
    \end{definition}

    \begin{lemma}
      $q \Int + a$ is a set.
    \end{lemma}
  \end{forthel}

  This allows us to define the so-called \textit{evenly spaced integer topology} where its open sets
  are defined as follows:

  \begin{forthel}
    \begin{definition}\label{Open}
      $A$ is open iff $A = \Int$ or for any $a \in A$ there exists $q$ such that
      $q \Int + a \subseteq A$.
    \end{definition}
  \end{forthel}

  Note that we have declared $q$ as a \textit{non-zero} integer. Otherwise, every set $A$ would be
  open since $0 \Int + a = \{ a \} \subseteq A$ for every $a \in A$.

  \begin{forthel}
    \begin{definition}\label{Closed}
      $A$ is closed iff $\overline{A}$ is open.
    \end{definition}

    \begin{definition}\label{OpenIntegerSets}
      An family of open sets is a family of integer sets $S$ such that every element of $S$ is open.
    \end{definition}
  \end{forthel}

  We can easily check that the open sets really form a topology on $\Int$.

  \begin{forthel}
    \begin{lemma}\label{UnionOpen}
      Let $S$ be an family of open sets.
      $\bigcup S$ is open.
    \end{lemma}
    \begin{proof}
      Let $x \in \bigcup S$.
      Take a set $M$ such that ($M$ is an element of $S$ and $x \in M$).
      Take $q$ such that $q \Int + x \subseteq M$.
      Then $q \Int + x \subseteq \bigcup S$.
    \end{proof}

    \begin{lemma}\label{InterOpen}
      Let $A,B$ be open subsets of $\Int$.
      Then $A \cap B$ is a subset of $\Int$ and $A \cap B$ is open.
    \end{lemma}
    \begin{proof}
      $A \cap B$ is a subset of $\Int$.
      Let $x \in A \cap B$.
      Then $x$ is an integer.
      Take $q$ such that $q \Int + x \subseteq A$.
      Take $p$ such that $p \Int + x \subseteq B$.

      Let us show that $p \cdot q$ is a nonzero integer and $(p \cdot q) \Int + x \subseteq A \cap B$.
        $p \cdot q$ is a nonzero integer.
        Let $a \in (p \cdot q) \Int + x$.

        $a \in p \Int + x$ and $a \in q \Int + x$. \newline
        Proof.
          $x$ is an integer and $a = x ~(\mod p \cdot q)$.
          $a = x ~(\mod p)$ and $a = x ~(\mod q)$ (by EquModMul).
        Qed.

        Therefore $a \in A$ and $a \in B$.
        Hence $a \in A \cap B$.
      End.
    \end{proof}

    \begin{lemma}\label{UnionClosed}
      Let $A,B$ be closed subsets of $\Int$.
      $A \cup B$ is closed.
    \end{lemma}
    \begin{proof}
      We have $\overline{A}, \overline{B} \subseteq \Int$.
      $\overline{A \cup B} = \overline{A} \cap \overline{B}$.
    \end{proof}
  \end{forthel}

  Now we state a consequence of finiteness:

  \begin{forthel}
    \begin{axiom}\label{UnionSClosed}
      Let $S$ be a finite family of integer sets such that all elements of $S$ are closed subsets of $\Int$.
      $\bigcup S$ is closed.
    \end{axiom}
  \end{forthel}

  This characterization allows us to prove that a family $S$ of closed sets is infinite by assuming
  $S$ to be finite and deriving a contradiction from this assumption together with the statement
  that $\bigcup S$ is closed. In Furstenberg's proof we will use this method to show that the family
  $\{ r \Int \mid r \text{ is prime} \}$ is infinite. To use the above argument we thus have to
  prove that any $r \Int$ -- or more general any $q \Int + a$ -- is closed.

  \begin{forthel}
    \begin{lemma}\label{ArSeqClosed}
      $q \Int + a$ is a closed subset of $\Int$.
    \end{lemma}
    \begin{proof}[by contradiction]
      $q \Int + a$ is a subset of $\Int$.
      Let $b \in \overline{q \Int + a}$.

      Let us show that $q \Int + b \subseteq \overline{q \Int + a}$.
        Let $c \in q \Int + b$.
        Assume not $c \in \overline{q \Int + a}$.
        Then $c = b ~(\mod q)$ and $a = c ~(\mod q)$.
        Hence $b = a ~(\mod q)$.
        Therefore $b \in q \Int + a$.
        Contradiction.
      End.
    \end{proof}
  \end{forthel}

  To prove that there are infinitely many primes we identify a prime number $r$ with the set
  $r \Int$ and show that the set $S = \{r \Int \mid r \textrm{ is a prime} \}$ is infinite. It is
  easy to see that $\bigcup S = \{ \text{integer } n \mid n \text{ has a prime divisor} \} =
  \Int \setminus \{ 1, -1 \}$. So if $S$ is finite then $\bigcup S$ is closed
  (by \textit{UnionClosed}) and hence $\{ 1, -1 \}$ is open. But then some arithmetic sequence
  $p \Int + 1$ (where $p$ is non-zero) is contained in $\{ 1, -1 \}$ which obviously cannot be.

  \begin{forthel}
    \begin{theorem}[Fuerstenberg]
      Let $S = \{ r \Int + 0 \mid r \text{ is a prime} \}$.
      $S$ is infinite.
    \end{theorem}
    \begin{proof}[by contradiction]
      $S$ is a family of integer sets.

      We have $\overline{\bigcup S} = \{ 1, -1 \}$. \newline
      Proof.
        Let us show that for any integer $n$ $n$ belongs to $\bigcup S$ iff $n$ has a prime divisor.
          Let $n$ be an integer.

          If $n$ has a prime divisor then $n$ belongs to $\bigcup S$. \newline
          Proof.
            Assume $n$ has a prime divisor.
            Take a prime divisor $p$ of $n$.
            $p \Int + 0 \in S$.
            $n \in p \Int + 0$.
          Qed.

          If $n$ belongs to $\bigcup S$ then $n$ has a prime divisor. \newline
          Proof.
            Assume $n$ belongs to $\bigcup S$.
            Take a prime $r$ such that $n \in r \Int + 0$.
            Then $r$ is a prime divisor of $n$.
          Qed.
        End.
      Qed.

      Assume that $S$ is finite.
      Then $\bigcup S$ is closed and $\overline{\bigcup S}$ is open.

      Take $p$ such that $p \Int + 1 \subseteq \overline{\bigcup S}$.

      $p \Int + 1$ has an element $x$ such that neither $x = 1$ nor $x = -1$. \newline
      Proof.
        $1 + p$ and $1 - p$ are integers.
        $1 + p$ and $1 - p$ belong to $p \Int + 1$.
        Indeed $1 + p = 1 ~(\mod p)$ and $1 - p = 1 ~(\mod p)$.
        $1 + p \neq  1 \wedge 1 - p \neq  1$.
        $1 + p \neq -1 \vee 1 - p \neq -1$.
      Qed.

      We have a contradiction.
    \end{proof}
  \end{forthel}

  Note that we cannot define $q \Int$ as $q \Int + 0$ in our formalization since then any term of
  the form $q \Int + a$ would be ambiguous: It could either be interpreted as $q \Int + a$ or as
  $(q \Int + 0) + a$. This is a result of some kind of overloading of the symbol $+$. We use $+$ on
  the one hand to denote integer addition and on the other hand it is part of the operator
  $\cdot ~ \Int + ~ \cdot$.

  \begin{thebibliography}{1}
    \bibitem{Furstenberg1955}
    Harry Furstenberg.
    ``On the Infinitude of Primes''.
    In: \textit{The American Mathematical Monthly} 62.1 (1955).
  \end{thebibliography}
\end{document}