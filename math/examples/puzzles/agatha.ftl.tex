\documentclass[english]{article}
\usepackage{../lang/puzzles}
\usepackage{naproche}
\usepackage{naproche-libraries}
\renewcommand\libarchive{Preliminaries}
\addbibresource{../lang/bibliography}

\title{Murder at Dreadsbury Mansion (Pelletier Problem 55)}
\author{\Naproche Formalization: Steffen Frerix and Peter Koepke}
\date{2018 and 2021}

\begin{document}

\maketitle

\section{Introduction}

Murder at Dreadsbury Mansion is a logic puzzle from Francis Jeffry Pelletier \cite{Pelletier1986}.
We quote the original text in the third section below where it can be compared directly with our formalization.

We imitate the language of the puzzle by adding nouns and other phrases to the input language ForTheL of the \Naproche system.
We then formulate the problem in a somewhat un-grammatical language which resembles the original text.
The conclusion of the case is formulated as a theorem and proved automatically by the background automated theorem prover (ATP) of \Naproche.
Modern ATPs have no difficulties with the Pelletier problem, so we don't
have to supply an explicit proof.

The source text of the formalization looks like a standard \LaTeX{} source
with signature, axiom and theorem environments:

\begin{verbatim}
...

\begin{signature}
  Aunt Agatha is a person.
\end{signature}

...

\begin{axiom}
  Some person that lives in Dreadsbury Mansion killed Aunt Agatha.
\end{axiom}

...

\begin{theorem}
  Agatha killed herself.
\end{theorem}
\end{verbatim}

The \LaTeX{} output of the formalization, on a light-grey background, was
achieved by redefining the printout of signature, axiom and theorem environments: signature and axiom have no visible effects, and the theorem environment prints out as ``Therefore: ...''.


\section{The Language of Dreadsbury}

\begin{forthel}
  [read \path{examples/lang/everyday-ontology.ftl.tex}]
\end{forthel}

\begin{forthel}
  \begin{signature}
    Aunt Agatha is a person.
  \end{signature}

  Let Agatha stand for Aunt Agatha.

  \begin{signature}
    The butler is a person.
  \end{signature}

  \begin{signature}
    Charles is a person.
  \end{signature}

  \begin{signature}
    Dreadsbury Mansion is a place.
  \end{signature}
\end{forthel}


\section{The Case}

We juxtapose the original text of the puzzle with our formalization:

\begin{quotation}
  \noindent Someone who lives in Dreadsbury Mansion killed Aunt Agatha.
  Agatha, the butler, and Charles live in Dreadsbury Mansion, and are the only
  people who live therein.
  A killer always hates his victim, and is never richer than his victim.
  Charles hates no one that Aunt Agatha hates.
  Agatha hates everyone except the butler.
  The butler hates everyone not richer than Aunt Agatha.
  The butler hates everyone Agatha hates.
  No one hates everyone.
  Agatha is not the butler.
  Therefore: Agatha killed herself.
\end{quotation}


\begin{forthel}
  \begin{axiom}
    Some person that lives in Dreadsbury Mansion killed Aunt Agatha.
  \end{axiom}
  \begin{axiom}
    Agatha, the butler and Charles live in Dreadsbury Mansion and for every person $X$ that lives in Dreadsbury Mansion $X$ is Aunt Agatha or $X$ is the butler or $X$ is Charles.
  \end{axiom}
  \begin{axiom}
    Every killer $X$ hates every victim of $X$ and is not richer than any victim of $X$.
  \end{axiom}
  \begin{axiom}
    Charles hates no person that is hated by Aunt Agatha.
  \end{axiom}
  \begin{axiom}
    Agatha hates every person that is not the butler.
  \end{axiom}
  \begin{axiom}
    The butler hates every person that is not richer than Aunt Agatha.
  \end{axiom}
  \begin{axiom}
    The butler hates every person that is hated by Agatha.
  \end{axiom}
  \begin{axiom}
    No person hates every person.
  \end{axiom}
  \begin{axiom}
    Agatha is not the butler.
  \end{axiom}
\end{forthel}


\section{The Solution}

\begin{forthel}
  \begin{theorem}
    Agatha killed herself.
  \end{theorem}
\end{forthel}

\printbibliography
\end{document}