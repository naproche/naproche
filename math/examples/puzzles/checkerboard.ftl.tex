\documentclass{article}
\usepackage{../latex/naproche}
\usepackage[english]{babel}

\title{The mutilated checkerboard}
\author{Naproche formalization and commentary:\\ Adrian De Lon, Peter Koepke, Anton Lorenzen}
\date{2021}

\newcommand{\Checkerboard}{\mathbf{C}}
\newcommand{\Corners}{\mathbf{C'}}
\newcommand{\Mutilated}{\mathbf{M}}
\newcommand{\Black}{\mathbf{B}}
\newcommand{\White}{\mathbf{W}}

\newcommand{\Swap}[1]{\operatorname{Swap}#1}
\newcommand{\Sw}[3]{\operatorname{Swap}_{#1}^{#2}(#3)}


\begin{document}

\maketitle





\section{Introduction}

Max Black proposed the mutilated checkerboard problem in his book
\textit{Critical Thinking} (1946).
%
It was later discussed by Martin Gardner in his
\textit{Scientific American} column, \textit{Mathematical Games}.
%
John McCarthy, one of the
founders of Artificial Intelligence described it as a
\textit{Tough Nut for Proof Procedures}
and discussed fully automatic or interactive proofs of the solution.

There have been several formalization of the Checkerboard problem before. A
survey article by Manfred Kerber and Martin Pollet called \emph{A Tough Nut
for Mathematical Knowledge Management} lists a couple of formalizations.
Our formalization is based on a set-theoretical formalization by John McCarthy.
Checking takes 2 or 3 minutes on a modest laptop.


    \begin{forthel}
        [read examples/preliminaries.ftl.tex]
        [synonym coordinate/-s]
        [synonym square/-s]
        [synonym subset/-s]

        Let $f : X\to Y$ stand for $f : X\rightarrow Y$.
    \end{forthel}


\section{Preliminaries about cardinality}


\begin{forthel}

\begin{definition}
    Let $B, C$ be sets.
    $B$ is equinumerous with $C$ iff there exists a map $F$ such that
    $F : B\leftrightarrow C$.
\end{definition}

\begin{lemma}
    Let $B, C$ be sets.
    Assume that $B$ is equinumerous with $C$.
    Then $C$ is equinumerous with $B$.
\end{lemma}

\begin{lemma}
    Let $A, B, C$ be sets.
    Assume $A$ is equinumerous with $B$.
    Assume $B$ is equinumerous with $C$.
    Then $A$ is equinumerous with $C$.
\end{lemma}
\begin{proof}
    Take a map $F$ such that $F : A\leftrightarrow B$.
    Take a map $G$ such that $G : B\rightarrow A$
    and (for all elements $x$ of $A$ we have $G(F(x)) = x$) and
    (for all elements $y$ of $B$ we have $F(G(y))=y$).
    Take a map $H$ such that $H : B\leftrightarrow C$.
    Take a map $I$ such that $I : C\rightarrow B$ and (for all elements $x$ of $B$ we have $I(H(x)) = x$) and
    (for all elements $y$ of $C$ we have $H(I(y))=y$).
    For every element $x$ of $A$ $H(F(x))$ is an object.
    Define $J(x) = H(F(x))$ for $x$ in $A$.
    For every element $y$ of $C$ $G(I(y))$ is an object.
[timelimit 30]
    $J : A\leftrightarrow C$. Indeed define $K(y) = G(I(y))$ for $y$ in $C$.
[/timelimit]
\end{proof}

\begin{definition}
    Let $X$ be a set.
    $X$ is Dedekind finite iff every proper subset of $X$
    is not equinumerous with $X$.
\end{definition}

\end{forthel}




\section{Setting up the checkerboard}

We introduce integer constants
to model checkerboards as a Cartesian product of the
integers $0, 1, 2, \ldots, 7$.

\begin{forthel}
    \begin{signature} An integer is an object. \end{signature}

    \begin{signature} $0$ is a integer. \end{signature}
    \begin{signature} $1$ is a integer. \end{signature}
    \begin{signature} $2$ is a integer. \end{signature}
    \begin{signature} $3$ is a integer. \end{signature}
    \begin{signature} $4$ is a integer. \end{signature}
    \begin{signature} $5$ is a integer. \end{signature}
    \begin{signature} $6$ is a integer. \end{signature}
    \begin{signature} $7$ is a integer. \end{signature}

    \begin{definition}
        A coordinate is an integer $x$
        such that $x = 0$ or $x = 1$ or $x = 2$ or $x = 3$ or $x = 4$
        or $x = 5$ or $x = 6$ or $x = 7$.
    \end{definition}

    Let $m, n, i, j, k, l$ denote coordinates.

    \begin{signature} A square is an object. \end{signature}
    \begin{axiom} $(m, n)$ is a square. \end{axiom}
    \begin{axiom}
        Let $x$ be a square.
        Then $x = (m,n)$ for some coordinate $m$ and some coordinate $n$.
    \end{axiom}
    Let $x, y, z$ denote squares.
\end{forthel}


The checkerboard is the Dedekind finite set containing all squares.

\begin{forthel}
    \begin{definition}
        The checkerboard is the class of all squares.
    \end{definition}

    \begin{axiom} The checkerboard is a set. \end{axiom}

    Let $\Checkerboard$ stand for the checkerboard.

    \begin{axiom}
        Every subset of the checkerboard is Dedekind finite.
    \end{axiom}
\end{forthel}


\section{The Mutilated Checkerboard}

Defining the mutilated checkerboard is straightforward:
we simply remove the two corners.

\begin{forthel}
    \begin{definition}
        $\Corners = \{ (0, 0), (7, 7) \}$.
    \end{definition}

    \begin{definition}
        $\Mutilated = \Checkerboard \setminus \Corners$.
    \end{definition}

    Let the mutilated checkerboard stand for $\Mutilated$.
\end{forthel}



\section{Dominoes}

To define dominoes,
we introduce concepts of adjacency by first declaring new relations and then axiomatizing them.
As usual, chaining of relation symbols indicates a conjunction.

\begin{forthel}
    Let $m, n, i, j, k, l$ denote coordinates.
    Let $x, y, z$ denote squares.

    \begin{signature}
        Let $m$ be an integer. A neighbour of $m$ is an integer.
    \end{signature}
    Let $m\sim n$ stand for $m$ is a neighbour of $n$.

    \begin{axiom}
        If $m\sim n$ then $n\sim m$.
    \end{axiom}
    \begin{axiom}
        $0\sim 1\sim 2\sim 3\sim 4\sim 5\sim 6\sim 7$.
    \end{axiom}

    \begin{definition}
        $x$ is adjacent to $y$ iff
        there exist integers $b, c, d, e$ such that
        $x = (b, c)$ and $y = (d, e)$ and
        (($b = d$  and $c\sim e$) or
        ($b\sim d$ and $c = e$)).
    \end{definition}

    \begin{definition}
        A domino is a set $D$ such that $D = \{x, y\}$ for some
        adjacent squares $x, y$.
    \end{definition}
\end{forthel}



\section{Domino Tilings}

\begin{forthel}
    \begin{definition}
        A domino tiling is a disjoint family $T$
        such that every element of $T$ is a domino.
    \end{definition}

    \begin{definition}
        Let $A$ be a subset of the checkerboard.
        A domino tiling of $A$ is a domino tiling $T$
        such that for every square $x$
        $x$ is an element of $A$ iff
        $x$ is an element of some element of $T$.
    \end{definition}
\end{forthel}

\noindent
We shall prove that the mutilated checkerboard has no domino tiling.


\section{Colours}

We shall solve the mutilated checkerboard problem by a cardinality argument.
Squares on an actual checkerboard are coloured black and white and we can count
colours on dominoes and on the mutilated checkerboard $\Mutilated$.

The introduction of colours can be viewed as a creative move typical of
mathematics: changing perspectives and introducing aspects that are not part
of the original problem. The mutilated checkerboard was first discussed under
a cognition-theoretic perspective: can one solve the problem \emph{without}
inventing new concepts and completely stay within the realm of squares,
subsets of the checkerboard and dominoes.

\begin{forthel}
    Let $x, y, z$ denote squares.

    \begin{signature} $x$ is black is a relation. \end{signature}
    Let $x$ is white stand for $x$ is not black.

    \begin{axiom} If $x$ is adjacent to $y$ then $x$ is black iff $y$ is white. \end{axiom}

    \begin{axiom} $(0,0)$ is black. \end{axiom}
    \begin{axiom} $(7,7)$ is black. \end{axiom}

    \begin{definition} $\Black$ is the class of black elements of $\Checkerboard$. \end{definition}
    \begin{definition} $\White$ is the class of white elements of $\Checkerboard$. \end{definition}

    \begin{lemma} $\Black$ is a set. \end{lemma}
    \begin{lemma} $\White$ is a set. \end{lemma}
\end{forthel}


\section{Counting Colours on Checkerboards}

The original checkerboard has an equal number of black and white squares.
Since our setup does not include numbers for counting, we rather work with
equinumerosity. The following argument formalizes that we can invert the
colours of a checkerboard by swapping the files $0$ and $1$, $2$ and $3$, etc..
We formalize swapping by a first-order function symbol $\Swap{}$.

\begin{forthel}
    Let $m, n, i, j, k, l$ denote coordinates.

    \begin{signature}
        Let $x$ be an element of $\Checkerboard$.
        $\Swap{x}$ is an element of $\Checkerboard$.
    \end{signature}

    \begin{axiom} $\Swap{(0, n)} = (1, n)$ and $\Swap{(1, n)} = (0, n)$. \end{axiom}

    \begin{axiom} $\Swap{(2, n)} = (3, n)$ and $\Swap{(3, n)} = (2, n)$. \end{axiom}

    \begin{axiom} $\Swap{(4, n)} = (5, n)$ and $\Swap{(5, n)} = (4, n)$. \end{axiom}

    \begin{axiom} $\Swap{(6, n)} = (7, n)$ and $\Swap{(7, n)} = (6, n)$. \end{axiom}

    \begin{lemma}
        Let $x$ be an element of $\Checkerboard$.
        $\Swap{x}$ is adjacent to $x$.
    \end{lemma}
    \begin{proof}
        Take integers $i$, $j$ such that $x = (i, j)$.
        Case $i = 0$. End.
        Case $i = 1$. End.
        Case $i = 2$. End.
        Case $i = 3$. End.
        Case $i = 4$. End.
        Case $i = 5$. End.
        Case $i = 6$. End.
        Case $i = 7$. End.
    \end{proof}
\end{forthel}

$\Swap{}$ is an involution.

\begin{forthel}
    \begin{lemma}
        Let $x$ be an element of $\Checkerboard$.
        $\Swap{(\Swap{x})} = x$.
    \end{lemma}
    \begin{proof}
        Take integers $i$, $j$ such that $x = (i, j)$.
        Case $i = 0$. End.
        Case $i = 1$. End.
        Case $i = 2$. End.
        Case $i = 3$. End.
        Case $i = 4$. End.
        Case $i = 5$. End.
        Case $i = 6$. End.
        Case $i = 7$. End.
    \end{proof}

    \begin{lemma}
        Let $x$ be an element of $\Checkerboard$.
        $x$ is black iff $\Swap{x}$ is white.
    \end{lemma}
\end{forthel}

Using $\Swap$ we can define a witness of $\Black \leftrightarrow \White$.

\begin{forthel}
    \begin{lemma}
        $\Black$ is equinumerous with $\White$.
    \end{lemma}
    \begin{proof}
        Define $F(x) = \Swap{x}$ for $x$ in $\Black$.
        Define $G(x) = \Swap{x}$ for $x$ in $\White$.
        $F(x)$ is white for all elements $x$ of $\dom(F)$.
        $G(y)$ is black for all elements $y$ of $\dom(G)$.
        Then $F : \Black \to \White$ and $G : \White \to \Black$.
        For all elements $x$ of $\Black$ we have $G(F(x)) = x$.
        For all elements $x$ of $\White$ we have $F(G(x)) = x$.
        $F : \Black \leftrightarrow \White$.
    \end{proof}
\end{forthel}

Given a domino tiling one can also swap the squares of each domino,
leading to similar properties.

\begin{forthel}
    \begin{signature}
        Let $A$ be a subset of the checkerboard.
        Let $T$ be a domino tiling of $A$.
        Let $x$ be an element of $A$.
        $\Sw{T}{A}{x}$ is a square $y$ such that there is an element $D$ of $T$
        such that $D = \{x,y\}$.
    \end{signature}

    \begin{lemma}
        Let $A$ be a subset of the checkerboard.
        Assume that $T$ is a domino tiling of $A$.
        Let $x$ be an element of $A$.
        Then $\Sw{T}{A}{x}$ is an element of $A$.
    \end{lemma}
    \begin{proof}
        Let $y = \Sw{T}{A}{x}$.
        Take an element $D$ of $T$ such that $D = \{x,y\}$.
        Then $y$ is an element of $A$.
    \end{proof}
\end{forthel}


Swapping dominoes is also an involution.

\begin{forthel}
    \begin{lemma}
        Let $A$ be a subset of the checkerboard.
        Assume that $T$ is a domino tiling of $A$.
        Let $x$ be an element of $A$.
        Then $\Sw{T}{A}{\Sw{T}{A}{x}} = x$.
    \end{lemma}
    \begin{proof}
        Let $y = \Sw{T}{A}{x}$.
        Take an element $Y$ of $T$ such that $Y = \{x, y\}$.
        Let $z = \Sw{T}{A}{y}$.
        Take an element $Z$ of $T$ such that $Z = \{y, z\}$.
        Then $x = z$.
    \end{proof}

    \begin{lemma}
        Let $A$ be a subset of the checkerboard.
        Assume that $T$ is a domino tiling of $A$.
        Let $x$ be a white element of $A$.
        Then $\Sw{T}{A}{x}$ is black.
    \end{lemma}
    \begin{proof}
        Let $y = \Sw{T}{A}{x}$.
        Take an element $Y$ of $T$ such that $Y = \{x,y\}$.
        [timelimit 10]
        Then $x$ is adjacent to $y$. 
        [/timelimit]
        Thus $y$ is black.
    \end{proof}

    \begin{lemma}
        Let $A$ be a subset of the checkerboard.
        Assume that $T$ is a domino tiling of $A$.
        Let $x$ be a black element of $A$.
        Then $\Sw{T}{A}{x}$ is white.
    \end{lemma}
    \begin{proof}
        Let $y = \Sw{T}{A}{x}$.
        Take an element $Y$ of $T$ such that $Y = \{x,y\}$.
        [timelimit 10]
        Then $x$ is adjacent to $y$. 
        [/timelimit]
        Thus $y$ is white.
    \end{proof}
\end{forthel}

\section{The Theorem}

\noindent We can easily show that a domino tiling involves as many black as white squares.

\begin{forthel}
    \begin{lemma}
        Let $A$ be a subset of the checkerboard.
        Let $T$ be a domino tiling of $A$.
        Then $A \cap \Black$ is
        equinumerous with $A \cap \White$.
    \end{lemma}
    \begin{proof}
        Define $F(x) = \Sw{T}{A}{x}$ for $x$ in $A \cap \Black$.
        Define $G(x) = \Sw{T}{A}{x}$ for $x$ in $A \cap \White$.
        $F: A \cap \Black \to A \cap \White$.
        Indeed  $F: A \cap \Black \to \White$.
        $G: A \cap \White \to A \cap \Black$.
        Indeed  $G: A \cap \White \to \Black$.
        For all elements $x$ of $A \cap \Black$ we have $G(F(x))=x$.
        For all elements $x$ of $A \cap \White$ we have $F(G(x))=x$.
        $F : A \cap \Black \leftrightarrow A \cap \White$.
    \end{proof}
\end{forthel}

\noindent In mutilating the checkerboard, one only removes black squares

\begin{forthel}
  \begin{lemma}
      $\Mutilated \cap \White = \White$.
  \end{lemma}

\begin{lemma}
    $\Mutilated\cap\Black$ is a proper subset of $\Black$.
\end{lemma}
\begin{proof}
    $(0,0)$ is an element of $\Black$.
    $(0,0)$ is not an element of $\Mutilated$.
    Thus $(0,0)$ is not an element of $\Mutilated\cap \Black$.
\end{proof}
\end{forthel}

\noindent Now the theorem follows by putting together the previous cardinality properties.
Note that the phrasing \textit{[...]\,has no domino tiling} in the theorem is automatically
derived from the definition of \textit{a domino tiling of\,[...]}.

\begin{forthel}
    \begin{theorem}
        The mutilated checkerboard has no domino tiling.
    \end{theorem}
    \begin{proof}[by contradiction]
        Assume $T$ is a domino tiling of $\Mutilated$.
        $\Mutilated \cap \Black$ is equinumerous with $\Mutilated \cap \White$.
        $\Mutilated \cap \Black$ is equinumerous with $\White$.
        $\Mutilated \cap \Black$ is equinumerous with $\Black$.
        Contradiction.
    \end{proof}
\end{forthel}

\end{document}
