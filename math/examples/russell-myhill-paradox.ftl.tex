\documentclass{article}
\usepackage{naproche}
\libinput[meta-inf]{example-preamble}
\libinput[libraries/foundations]{preamble}

\title{The Russell-Myhill Paradox}
\author{\Naproche formalization:\\[0.5em]Marcel Schütz}
\date{2024}

\begin{document}
  \pagenumbering{gobble}
  \maketitle

  \usemodule[meta-inf]{everyday-ontology.ftl}

  \noindent The \emph{Russell-Myhill Paradox} is a paradox discovered by 
  Bertrand Russell which states that two systems of propositions with the same 
  \emph{logical product} are not necessarily equal.
  It was discussed in Russell's 1903 \emph{Principles of Mathematics}
  \cite[Appendix B]{Russell1903} and rediscovered in 1958 by John Myhill
  \cite{Myhill1958}.

  \begin{forthel}
    \begin{signature*}
      Let $P$ be a system of propositions.
      The logical product of $P$ is a proposition $p$ such that $p$ is true iff every element of $P$ is true.
    \end{signature*}
    
    Let $\sqcap P$ denote the logical product of $P$.
    
    \begin{theorem*}[title=Russell-Myhill Paradox,id=russell_myhill_paradox]
      It is wrong that
      \[ \sqcap P = \sqcap Q \implies P = Q \]
      for all systems of propositions $P, Q$.
    \end{theorem*}
    \begin{proof}
      Assume the contrary.
      Define $Q = \{q \mid$ there exists a system of propositions $P$ such that $q = \sqcap P$ and $q \notin P\}$.
      Consider $q = \sqcap Q$.
      Then for any system of propositions $P$ such that $\sqcap q = P$ we have $P = Q$.
      Hence $q \in Q$ iff $q \notin Q$.
      Contradiction.
    \end{proof}
  \end{forthel}

  \printbibliography
  \printlicense[CcByNcSa]{2024}{Marcel Schütz}
\end{document}
