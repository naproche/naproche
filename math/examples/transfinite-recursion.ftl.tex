\documentclass{article}
\usepackage{naproche}
\libinput[meta-inf]{example-preamble}
\libinput[libraries/foundations]{preamble}
\libinput[libraries/set-theory]{preamble}

\title{Transfinite Recursion Theorem}
\author{\Naproche formalization:\\[0.5em]Marcel Schütz}
\date{2024}

\begin{document}
\begin{smodule}{transfinite-recursion.ftl}
  \maketitle

  \usemodule[libraries/set-theory]{transfinite-induction.ftl}
  \usemodule[libraries/set-theory]{recursive-maps.ftl}
  
  \noindent This is a formalization of the \emph{Transfinite Recursion Theorem}
  (cf. \cite{Koepke2018}).
  It states that for any map $G : A^{< \infty} \to A$, where $A^{< \infty}$
  denotes the class of all maps $\alpha \to A$ for some ordinal $\alpha$, there
  exists a unique map $F: \Ord \to A$ that is \emph{recursive regarding} $G$,
  i.e. \[F(\alpha) = G(F \restriction \alpha)\] for all ordinals $\alpha$.

  \inputref[libraries/set-theory]{coincidence-lemma.ftl.tex}
  
  \inputref[libraries/set-theory]{transfinite-recursion-1.ftl.tex}

  As a corollary of the transfinite recursion theorem we get that we can
  define maps recursively on the ordinals by case distinction:
  For given maps $G : \Ord \times A \to A$ and
  $H : \Ord \times A^{< \infty} \to A$ and an element $a \in A$ we can define
  a map $F : \Ord \to A$ by
  \begin{itemize}
    \item $F(0) = a$,
    \item $F(\succ(\alpha)) = G(\alpha, F(\alpha))$, and
    \item $F(\lambda) = H(\lambda, F \restriction \lambda)$
      for any limit ordinal $\lambda$.
  \end{itemize}

  \inputref[libraries/set-theory]{transfinite-recursion-2.ftl.tex}

  \printbibliography
  \printlicense[CcByNcSa]{2024}{Marcel Schütz}
\end{smodule}
\end{document}
