\documentclass{article}
\usepackage{naproche}
\libinput[latex]{example-preamble}
\libinput[libraries/foundations]{preamble}
\libinput[libraries/set-theory]{preamble}

\title{Transfinite Recursion Theorem}
\author{\Naproche formalization:\\[0.5em]Marcel Schütz}
\date{2024}

\begin{document}
\begin{smodule}{transfinite-recursion.ftl}
  \maketitle

  \usemodule[libraries/set-theory]{recursive-maps.ftl}
  
  \noindent This is a formalization of the \emph{Transfinite Recursion Theorem}
  (cf. \cite{Koepke2018}).
  It states that for any map $G \FUNfromto{\ORDfunspace{A}}{A}$, where
  $\ORDfunspace{A}$ denotes the class of all maps $\alpha \to A$ for some
  ordinal $\alpha$, there exists a unique map $F \FUNfromto{\Ord}{A}$ that is
  \emph{recursive regarding} $G$, i.e. \[F(\alpha) \eq G(F \FUNrest \alpha)\] for
  all ordinals $\alpha$.

  \inputref[libraries/set-theory]{coincidence-lemma.ftl.tex}
  
  \inputref[libraries/set-theory]{transfinite-recursion-1.ftl.tex}

  As a corollary of the transfinite recursion theorem we get that we can
  define maps recursively on the ordinals by case distinction:
  For given maps $G \FUNfromto{\Ord \SETprod A}{A}$ and
  $H \FUNfromto{\Ord \SETprod \ORDfunspace{A}}{A}$ and an element $a \in A$ we
  can define a map $F \FUNfromto{\Ord}{A}$ by
  \begin{itemize}
    \item $F(\ORDzero) = a$,
    \item $F(\ORDsucc(\alpha)) = G(\alpha, F(\alpha))$, and
    \item $F(\lambda) = H(\lambda, F \FUNrest \lambda)$
      for any limit ordinal $\lambda$.
  \end{itemize}

  \inputref[libraries/set-theory]{transfinite-recursion-2.ftl.tex}

  \printbibliography
  \printlicense[CcByNcSa]{2024}{Marcel Schütz}
\end{smodule}
\end{document}
