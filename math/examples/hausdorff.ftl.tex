\documentclass{article}
\usepackage{latex/naproche}
\usepackage[english]{babel}
\usepackage{amssymb}
\usepackage{hyperref}
\hypersetup{
  colorlinks=true
}

\newcommand{\surjects}{\twoheadrightarrow}

\title{Regularity of successor cardinals}
\author{Steffen Frerix (2018), Adrian De Lon (2021)}
\date{}

\begin{document}
  \maketitle


  \section{Preliminaries}

  \begin{forthel}
    [read \path{examples/preliminaries.ftl.tex}]

    Let $x, y, X, Y$ denote sets.
    Let $f$ denote a function.

    Let $f$ is surjective from $X$ onto $Y$ stand for $\dom(f) = X$ and $f[X] = Y$.

    Let $f : X \surjects Y$ stand for $f$ is surjective from $X$ onto $Y$.

    Let $X$ is nonempty stand for $X$ has an element.
  \end{forthel}


  \section{Ordinals}

  \begin{forthel}
    Let $x, y$ denote sets.

    \begin{signature}
      An ordinal is a set.
    \end{signature}

    Let $\alpha, \beta$ denote ordinals.

    \begin{axiom}
      Every element of $\alpha$ is an ordinal.
    \end{axiom}

    \begin{axiom}\label{Transitivity}
      Let $x \in y \in \alpha$.
      Then $x \in \alpha$.
    \end{axiom}

    \begin{signature}
      Let $\alpha, \beta$ be ordinals.
      $\alpha < \beta$ is a relation.
    \end{signature}

    \begin{axiom}
      $\alpha < \beta$ or $\beta < \alpha$ or $\beta = \alpha$.
    \end{axiom}

    \begin{axiom}
      If $\alpha < \beta$ then $\alpha$ is an element of $\beta$.
    \end{axiom}

    Let $a \leq b$ stand for $a = b$ or $a < b$.
  \end{forthel}


  \section{Cardinals}

  \begin{forthel}
    Let $X, Y$ denote sets.

    \begin{signature}
      A cardinal is an ordinal.
    \end{signature}

    Let $\mu, \nu$ denote cardinals.

    \begin{signature}
      $\mu$ is infinite is an atom.
    \end{signature}

    Let $\kappa$ denote an infinite cardinal.

    \begin{signature}\label{Cardinality}
      $|X|$ is a cardinal.
    \end{signature}

    \begin{axiom}[existence of surjection]\label{existence_of_surjection}
      Assume $X$ has an element.
      $|X| \leq |Y|$ iff there exists a function that is surjective from $Y$ onto $X$.
    \end{axiom}

    \begin{axiom}[Hessenberg]
      $|\kappa \times \kappa| = \kappa$.
    \end{axiom}

    \begin{axiom}
      $|\kappa| = \kappa$.
    \end{axiom}

    \begin{axiom}
      Let $Y$ be a subset of $X$.
      $|Y| \leq |X|$.
    \end{axiom}

    \begin{signature}
      Let $\kappa$ be an infinite cardinal.
      $\kappa^{+}$ is an infinite cardinal.
    \end{signature}

    \begin{axiom}
      $\kappa < \kappa^{+}$.
    \end{axiom}

    \begin{axiom}
      $|\alpha| \leq \kappa$ for every element $\alpha$ of $\kappa^{+}$.
    \end{axiom}

    \begin{axiom}
      For no cardinals $\mu, \nu$ we have $\mu < \nu$ and $\nu < \mu$.
    \end{axiom}

    \begin{axiom}
      There is no cardinal $\nu$ such that $\kappa < \nu < \kappa^{+}$.
    \end{axiom}

    \begin{axiom}
      The empty set is a cardinal $\eta$ such that $\eta$ is an element of
      every nonempty ordinal.
    \end{axiom}

    \begin{definition}
      The constant zero on $X$ is the function $f$ such that $\dom(f) = X$ and
      $f(x)$ is the empty set for every element $x$ of $X$.
    \end{definition}

    Let $0^{X}$ stand for the constant zero on $X$.
  \end{forthel}


  \section{Cofinality and regular cardinals}

  \begin{forthel}
    Let $\kappa$ denote an infinite cardinal.

    \begin{definition}[Cofinality]\label{Cofinality}
      Let $Y$ be a subset of $\kappa$.
      $Y$ is cofinal in $\kappa$ iff for every element $x$ of $\kappa$ there
      exists an element $y$ of $Y$ such that $x < y$.
    \end{definition}

    Let a cofinal subset of $\kappa$ stand for a subset of $\kappa$ that is
    cofinal in $\kappa$.

    \begin{definition}
      $\kappa$ is regular iff $|x| = \kappa$ for every cofinal subset $x$ of
      $\kappa$.
    \end{definition}
  \end{forthel}


  \section{Hausdorff's theorem}

  The following result appears in \cite[p.~443]{Hausdorff1908},
  where Hausdorff mentions that the proof is
  \textit{``ganz einfach''} (\textit{``very simple''}) and can be skipped.

  \begin{forthel}
    \begin{theorem}[Hausdorff]
      Let $\kappa$ be an infinite cardinal.
      Then $\kappa^{+}$ is regular.
    \end{theorem}
    \begin{proof}[by contradiction]
      Assume the contrary.
      Take a cofinal subset $x$ of $\kappa^{+}$ such that $|x| \neq \kappa^{+}$.
      Then $|x| \leq \kappa$.
      Take a function $f$ that is surjective from $\kappa$ onto $x$ (by
      \nameref{existence_of_surjection}).
      Indeed $x$ has an element and $|\kappa| = \kappa$.

      Define \[ g(z) =
        \begin{cases}
          \text{choose a function $h$ such that $h : \kappa \surjects z$ in $h$}
          & : \text{$z$ has an element}
          \\
          \text{$0^{\kappa}$}
          & : \text{$z$ has no element}
        \end{cases}
      \] for $z$ in $\kappa^{+}$.

      For all $\xi, \zeta \in \kappa$ $g(f(\xi))$ is a map such that $\zeta \in
      \dom(g(f(\xi)))$.
      Define $h(\xi,\zeta) = g(f(\xi))(\zeta)$ for $(\xi,\zeta)$ in $\kappa
      \times \kappa$.

      Let us show that $h$ is surjective from $\kappa \times \kappa$ onto
      $\kappa^{+}$.

        Every element of $\kappa^{+}$ is an element of $h[\kappa \times
        \kappa]$. \newline
        Proof.
          Let $n$ be an element of $\kappa^{+}$.
          Take an element $\xi$ of $\kappa$ such that $n < f(\xi)$.
          Take an element $\zeta$ of $\kappa$ such that $g(f(\xi))(\zeta) = n$.
          Indeed $g(f(\xi))$ is a function that is surjective from $\kappa$ onto
          $f(\xi)$.
          Then $n = h(\xi,\zeta)$.
          Therefore the thesis.
          Indeed $(\xi,\zeta)$ is an element of $\kappa \times \kappa$.
        End.

        Every element of $h[\kappa \times \kappa]$ is an element of
        $\kappa^{+}$. \newline
        Proof.
          Let $n$ be an element of $h[\kappa \times \kappa]$.
          We can take elements $a, b$ of $\kappa$ such that $n = h(a,b)$.
          Then $n = g(f(a))(b)$.
          $f(a)$ is an element of $\kappa^{+}$.
          Every element of $f(a)$ is an element of $\kappa^{+}$.

          Case $f(a)$ has an element.
            Then $g(f(a))$ is a function that is surjective from $\kappa$ onto
            $f(a)$.
            Hence $n \in f(a) \in \kappa^{+}$.
            Thus $n \in \kappa^{+}$.
          End.

          Case $f(a)$ has no element.
            Then $g(f(a)) = 0^{\kappa}$.
            Hence $n$ is the empty set.
            Thus $n \in \kappa^{+}$.
          End.
        End.
      End.

      Therefore $\kappa^{+} \leq |\kappa \times \kappa| = \kappa$.
      Contradiction.
    \end{proof}
  \end{forthel}

\begin{thebibliography}{9}
  \bibitem{Hausdorff1908}
  Felix Hausdorff.
  ``Grundzüge einer Theorie der geordneten Mengen''.
  In: \textit{Mathematische Annalen} 65 (1908),
  pp. 435--505.
\end{thebibliography}
\end{document}
