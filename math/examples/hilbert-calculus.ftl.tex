\documentclass{article}
\usepackage{naproche}
\libinput[meta-inf]{example-preamble}

\newcommand{\imp}{\supset}
\newcommand{\deducible}{\vdash}
\newcommand{\falsum}{\bot}
\newcommand{\negate}{\lnot}

\title{A Hilbert-style calculus embedded into Naproche}
\author{Adrian De Lon (2021)}
\date{}


\begin{document}
\pagenumbering{gobble}

\maketitle

\usemodule[libraries/meta-inf]{vocabulary.ftl}

The basic rules of the calculus are given as axioms.
Then further derived rules are proven as propositions. 

\section{The implicational propositional calculus}

\begin{forthel}

\begin{signature}
    A formula is an object.
\end{signature}
Let $P, Q, R, S$ denote formulas.

\begin{signature}
    $P \imp Q$ is a formula.
\end{signature}

\begin{signature}
    $\deducible P$ is a relation.
\end{signature}
Let $P$ is deducible stand for $\deducible P$.

\begin{axiom}[id=Detachment]
    Suppose $\deducible P$ and $\deducible P \imp Q$.
    Then $\deducible Q$.
\end{axiom}

\begin{axiom}[id=Implosion]
    $\deducible P \imp (Q \imp P)$.
\end{axiom}

\begin{axiom}[id=Chain]
    $\deducible (P \imp (Q \imp R)) \imp ((P \imp Q) \imp (P \imp R))$.
\end{axiom}

\begin{proposition}[id=Weakening]
    (0) Assume $\deducible Q$.
    Then $\deducible P\imp Q$.
\end{proposition}
\begin{proof}
    (1) $\deducible Q\imp (P\imp Q)$ (by Implosion).
    
    (2) $\deducible P\imp Q$ (by Detachment, 0, 1).
\end{proof}

\begin{proposition}[id=Weakening of implication]
    (0) Assume $\deducible Q\imp R$.
    Then $\deducible (P\imp Q)\imp (P \imp R)$.
\end{proposition}
\begin{proof}
    (1) $\deducible P\imp (Q\imp R)$ (by Weakening, 0).
    
    (2) $\deducible (P\imp (Q\imp R))\imp ((P\imp Q)\imp (P\imp R))$ (by Chain).
     
    (3) $\deducible (P\imp Q)\imp (P\imp R)$ (by Detachment, 1, 2).
\end{proof}

\begin{proposition}[id=Transitivity]
    %Suppose $P\imp Q$ and $Q\imp R$ are deducible.
    %Then $P\imp R$ is deducible.
    (0) Suppose $\deducible P\imp Q$ and $\deducible Q\imp R$.
    Then $\deducible P\imp R$.
\end{proposition}
\begin{proof}
    (1) $\deducible (P\imp Q)\imp (P\imp R)$ (by Weakening of implication, 0).
    
    (2) $\deducible P\imp R$  (by Detachment, 0, 1).
\end{proof}

\begin{proposition}[id=Identity]
    $\deducible P\imp P$.
\end{proposition}
\begin{proof}
    (1) $\deducible (P\imp ((P\imp P)\imp P))\imp ((P\imp (P\imp P))\imp (P\imp P))$ (by Chain).
    
    (2) $\deducible P\imp ((P\imp P)\imp P)$ (by Implosion).
    
    (3) $\deducible (P\imp (P\imp P))\imp (P\imp P)$ (by 1, 2, Detachment).
    
    (4) $\deducible P\imp (P\imp P)$ (by Implosion).
    
    (5) $\deducible P\imp P$ (by 3, 4, Detachment).
\end{proof}

\begin{proposition}
    (0) Suppose $\deducible P\imp (P\imp Q)$.
    Then $\deducible P\imp Q$.
\end{proposition}
\begin{proof}
    (1) $\deducible (P\imp (P\imp Q))\imp ((P\imp P)\imp (P\imp Q))$ (by Chain).
    
    (2) $\deducible (P\imp P)\imp (P\imp Q)$ (by 0, 1, Detachment).
    
    (3) $\deducible P\imp P$  (by Identity).
    
    (4) $\deducible P\imp Q$  (by 2, 3, Detachment).
\end{proof}
\end{forthel}

\section{Adding negation}

\begin{forthel}
Let $P, R$ denote formulas.

\begin{signature}[id=Falsum]
    $\falsum$ is a formula.
\end{signature}

Let $\negate$P stand for $P\imp\falsum$.

\begin{axiom}[id=Double negation]
    $\deducible\negate\negate P\imp P$.
\end{axiom}

\begin{proposition}[id=Explosion]
    $\deducible \falsum\imp P$.
\end{proposition}

\end{forthel}

\end{document}
