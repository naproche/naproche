\documentclass{article}
\usepackage{naproche}
\libinput[meta-inf]{example-preamble}
\libinput[libraries/foundations]{preamble}
\libinput[libraries/set-theory]{preamble}

\title{Hilbert's Paradox}
\author{\Naproche formalization:\\[0.5em]Marcel Schütz}
\date{2024}

\begin{document}
\begin{smodule}{hilbert-paradox.ftl}
  \pagenumbering{gobble}
  \maketitle

  \usemodule[libraries/set-theory]{zfc.ftl}
  \usemodule[libraries/foundations]{closure-under-arbitrary-unions.ftl}
  \usemodule[libraries/foundations]{closure-under-powersets.ftl}
  \usemodule[libraries/foundations]{universal-class.ftl}

  \noindent \emph{Hilbert's Paradox}, discovered around 1900 by David Hilbert, 
  demonstates that there cannot exists a set that is closed under powersets and 
  under unions of arbitrary subsets \cite{PeckhausKahl2002}.

  \inputref[libraries/set-theory]{hilbert-paradox.ftl.tex}

  \noindent Using Hilbert's Paradox it can further be shown that there exists 
  no universal set, i.e. that the class $\Set$ of all sets is a proper class:

  \inputref[libraries/set-theory]{the-sets-form-a-proper-class-1.ftl.tex}

  \printbibliography
  \printlicense[CcByNcSa]{2024}{Marcel Schütz}
\end{smodule}
\end{document}
