\documentclass{stex}
\libusepackage{naproche}
\libinput{preamble}
\begin{document}
\begin{smodule}{leq-is-antisymmetric.ftl}

\importmodule[libraries/natural-numbers]{definitions?leq.ftl}
\importmodule[libraries/natural-numbers]{theorems?addition-is-associative.ftl}
\importmodule[libraries/natural-numbers]{theorems?addition-is-left-cancellative.ftl}
\importmodule[libraries/natural-numbers]{theorems?zero-sum-implies-zero-summands.ftl}

\begin{proposition}[forthel,id=LeqIsAntisymmetricProp]
  Let $n, m$ be natural numbers.
  If $n \leq m$ and $m \leq n$ then $n \eq m$.
\end{proposition}
\begin{proof}[forthel]
  Assume $n \leq m$ and $m \leq n$.
  Take natural numbers $k, l$ such that $m \eq n \plus k$ and $n \eq m \plus l$.
  Then $m
    \eq n \plus k
    \eq (m \plus l) \plus k
    \eq m \plus (l \plus k)$.
  Hence $l \plus k \eq \zero$.
  Thus $l \eq \zero \eq k$.
  Therefore $m \eq n$.
\end{proof}

\end{smodule}
\end{document}