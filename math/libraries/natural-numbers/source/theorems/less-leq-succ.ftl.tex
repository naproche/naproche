\documentclass{stex}
\libusepackage{naproche}
\libinput{preamble}
\begin{document}
\begin{smodule}{less-leq-succ.ftl}

\importmodule[libraries/natural-numbers]{definitions?leq.ftl}
\importmodule[libraries/natural-numbers]{definitions?positive.ftl}
\importmodule[libraries/natural-numbers]{definitions?number-2.ftl}
\importmodule[libraries/natural-numbers]{theorems?positive-iff-nonzero.ftl}
\importmodule[libraries/natural-numbers]{theorems?addition-is-left-cancellative.ftl}
\importmodule[libraries/natural-numbers]{theorems?less-implies-neq.ftl}
\importmodule[libraries/natural-numbers]{theorems?zero-sum-implies-zero-summands.ftl}

\begin{proposition}[forthel,id=LessLeqSuccProp]
  Let $n, m$ be natural numbers.
  If $n \less m \leq n \plus \one$ then $m \eq n \plus \one$.
\end{proposition}
\begin{proof}[forthel]
  Assume $n \less m \leq n \plus \one$.
  Take a positive natural number $k$ such that $m \eq n \plus k$.
  Take a natural number $l$ such that $n \plus \one \eq m \plus l$.
  Then $n \plus \one
    \eq m \plus l
    \eq (n \plus k) \plus l
    \eq n \plus (k \plus l)$.
  Hence $k \plus l \eq \one$.

  We have $l \eq \zero$.
  \begin{proof}
    Assume the contrary.
    Then $k,l \gtr \zero$.

    \begin{case}{$k,l \eq \one$.}
      Then $k \plus l
        \eq \two
        \neq \one$.
      Contradiction.
    \end{case}

    \begin{case}{$k \eq \one and l \neq \one$.}
      Then $l \gtr \one$.
      Hence $k \plus l
        \gtr \one \plus l
        \gtr \one$.
      Contradiction.
    \end{case}

    \begin{case}{$k \neq \one and l \eq \one$.}
      Then $k \gtr \one$.
      Hence $k \plus l
        \gtr k \plus \one
        \gtr \one$.
      Contradiction.
    \end{case}

    \begin{case}{$k, l \neq \one$.}
      Take natural numbers $a, b$ such that $k \eq a \plus \one$ and $l \eq b \plus \one$.
      Indeed $k, l \neq \zero$.
      Hence $k \eq a \plus \one$ and $l \eq b \plus \one$.
      Thus $k, l \gtr \one$.
      Indeed $a, b$ are positive.
    \end{case}
  \end{proof}

  Then we have $n \plus \one
    \eq m \plus l
    \eq m \plus \zero
    \eq m$.
\end{proof}

\end{smodule}
\end{document}