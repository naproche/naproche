\documentclass{naproche-library}
\libinput{preamble}
\begin{document}
\IfInputref{}{\setsectionlevel{section}}
\begin{smodule}[title=Functions and Maps]{functions.ftl}
\importmodule[libraries/preliminaries]{sets.ftl}
\STEXexport{
  \let\oldcirc\circ
  \let\circ\relax
}
\symdef{im}[args=2]{#1^{\comp*}\dobrackets{#2}}
\symdef{preim}[args=2]{#1_{\comp*}\dobrackets{#2}}
\symdef{fun}[args=3]{#1\comp:#2\rightarrow#3}
\symdef{id}[args=1]{\maincomp{\textrm{id}}_{#1}}
\symdef{circ}[op=\oldcirc]{\,\oldcirc\,}
\symdef{restr}[args=2]{#1\comp{\big|}_{#2}}
\symdef{inv}[args=1]{#1^{\comp{-1}}}

\begin{forthel}
  Let $A,B,C$ denote classes.

  Let $F,G,H$ denote maps.
\end{forthel}

\begin{sfragment}{Map Extensionality}
  \begin{axiom*}[forthel,title=Map Extensionality Axiom,id=MapExtensionalityAx]
    If $\dom(F)\eq\dom(G)$ and $F(a)\eq G(a)$ for all $a\in\dom(F)$ then $F\eq G$.
  \end{axiom*}
\end{sfragment}

\begin{sfragment}{Image And Preimage}
  \begin{definition*}[forthel,id=ImageDef]
    Let $F$ be a map.
    $\emph{\im{F}{A}}=\{F(a)\mid a\in\dom(F)\text{ and }a\in A\}$.

    Let the \emph{image of $A$ under $F$} stand for $\im{F}{A}$.
  \end{definition*}

  \begin{definition*}[forthel,id=PreimageDef]
    $\emph{\preim{F}{B}}=\{a\in\dom(F)\mid F(a)\in B\}$.

    Let the \emph{preimage of $B$ under $F$} stand for $\preim{F}{B}$.
  \end{definition*}
\end{sfragment}

\begin{sfragment}{Maps Between Classes}
  \begin{definition*}[forthel,id=MapFromDef]
    A \emph{map of $A$} is a map $F$ such that $\dom(F)\eq A$.
  \end{definition*}

  \begin{definition*}[forthel,id=MapToDef]
    A \emph{map to $B$} is a map $F$ such that $F(a)\in B$ for all $a\in\dom(F)$.
  \end{definition*}

  \begin{definition*}[forthel,id=MapFromToDef]
    A \emph{map from $A$ to $B$} is a map $F$ such that $\dom(F)\eq A$ and $F(a)\in B$ for all $a\in\dom(F)$.
  \end{definition*}
\end{sfragment}

\begin{sfragment}{Identity Maps}
  \begin{definition*}[forthel,id=IdentityMapDef]
    Let $A$ be a class.
    $\emph{\id{A}}$ is the map of $A$ such that $\id{A}(a)eq a$ for all $a\in A$.

    Let the \emph{identity map on $A$} stand for $\id{A}$.
  \end{definition*}
\end{sfragment}

\begin{sfragment}{Composition}
  \begin{definition*}[forthel,id=CompositionDef]
    Let $F,G$ be maps such that $F(a)\in\dom(G)$ for all $a\in\dom(F)$.
    $\emph{G\circ F}$ is the map of $\dom(F)$ such that $G\circ F\eq G(F(a))$ for all $a\in\dom(F)$.

    Let the \emph{composition of $G$ and $F$} stand for $G\circ F$.
  \end{definition*}
\end{sfragment}

\begin{sfragment}{Restriction}
  \begin{definition*}[forthel,id=RestrictionDef]
    Let $F$ be a map and $A\subseteq\dom(F)$.
    $\emph{\restr{F}{A}}$ is the map of $A$ such that $\restr{F}{A}\eq F(a)$ for all $a\in A$.

    Let the \emph{restriction of $F$ to $A$} stand for $\restr{F}{A}$.
  \end{definition*}
\end{sfragment}

\begin{sfragment}{Invertible Maps}
  \begin{definition*}[forthel,id=InvertibleDef]
    An \emph{inverse of $F$} is a map $G$ such that $F(a)\eq b\iff G(b)\eq a$ for all $a\in\dom(F)$ and all $b\in\dom(G)$.

    Let $F$ is \emph{invertible} stand for $F$ has an inverse.
  \end{definition*}

  \begin{definition*}[forthel,id=InverseMapDef]
    Let $F$ be invertible.
    $\emph{\inv{F}}$ is the inverse of $F$.
  \end{definition*}
\end{sfragment}

\begin{sfragment}{Injective, Surjective and Bijective Maps}
  \begin{definition*}[forthel,id=InjectionDef]
    $F$ is \emph{injective} iff $F(a)\eq F(a')\implies a\eq a'$ for all $a,a'\in\dom(F)$.
  \end{definition*}

  \begin{definition*}[forthel,id=SurjectionDef]
    $F$ is \emph{surjective onto $B$} iff for all $b\in B$ there exists an $a\in\dom(F)$ such that $b\eq F(a)$.

    Let an \emph{surjective map from $A$ to $B$} stand for a map of $A$ that is surjective onto $B$.
  \end{definition*}

  \begin{definition*}[forthel,id=BijectionDef]
    A \emph{bijection between $A$ and $B$} is an injective surjective map from $A$ to $B$.
  \end{definition*}
\end{sfragment}

\begin{sfragment}{Equinumerosity}
  \begin{definition*}[forthel,id=EquinumerousDef]
    $A$ and $B$ are \emph{equinumerous} iff there exists a bijection between $A$ and $B$.
  \end{definition*}
\end{sfragment}

\begin{sfragment}{Finite and Infinite Sets}
  \begin{definition*}[forthel,id=InfiniteDef]
    $A$ is \emph{infinite} iff there exists an injective map from $A$ to some proper subclass of $A$.

    Let $A$ is \emph{finite} stand for $A$ is not infinite.
  \end{definition*}
\end{sfragment}
\end{smodule}
\end{document}
