\documentclass{naproche-library}
\libinput{preamble}
\begin{document}
\IfInputref{}{\setsectionlevel{section}}
\begin{smodule}[title=Natural Numbers]{natural-numbers.ftl}
\importmodule[meta-inf]{macros.ftl}
\importmodule[meta-inf]{vocabulary.ftl}
\symdef{Nat}{\mathbb N}
\symdef{plus}{\,+\,}
\symdef{zero}{0}
\symdef{one}{1}
\symdef{two}{2}
\symdef{three}{3}
\symdef{four}{4}
\symdef{five}{5}
\symdef{six}{6}
\symdef{seven}{7}
\symdef{eight}{8}
\symdef{nine}{9}

\begin{sfragment}{The Language of Natural Number Arithmetic}
  \begin{signature}[forthel,id=NatNumberSig]
    A \emph{natural number} is an object.

    Let $n,m,k$ denote natural numbers.
  \end{signature}

  \begin{definition}[forthel,id=NatDef]
    $\emph{\Nat}$ is the class of all natural numbers.
  \end{definition}

  \begin{signature}[forthel,id=PlusSig]
    $\emph{n\plus m}$ is a natural number.

    Let the \emph{sum of $n$ and $m$} stand for $n\plus m$.
  \end{signature}

  \STEXinvisible{\inlineforthel{Let $n+m$ stand for $n\plus m$.}}

  \begin{signature}[forthel,id=ZeroSig]
    $\emph{\zero}$ is a natural number.

    Let \emph{zero} stand for $\zero$.
    Let $n$ is \emph{nonzero} stand for $n\neq\zero$.
  \end{signature}

  \STEXinvisible{\inlineforthel{Let $0$ stand for $\zero$.}}

  \begin{signature}[forthel,id=OneSig]
    $\emph{\one}$ is a natural number.

    Let \emph{one} stand for $\one$.
    Let the \emph{direct successor of $n$} stand for $n\plus\one$.
  \end{signature}

  \STEXinvisible{\inlineforthel{Let $1$ stand for $\one$.}}

  \begin{definition}[forthel,id=TwoSig]
    $\emph{\two}=\one\plus\one$.

    Let \emph{two} stand for $\two$.
  \end{definition}

  \STEXinvisible{\inlineforthel{Let $2$ stand for $\two$.}}

  \begin{definition}[forthel,id=ThreeSig]
    $\emph{\three}=\two\plus\one$.

    Let \emph{three} stand for $\three$.
  \end{definition}

  \STEXinvisible{\inlineforthel{Let $3$ stand for $\three$.}}

  \begin{definition}[forthel,id=FourSig]
    $\emph{\four}=\three\plus\one$.

    Let \emph{four} stand for $\four$.
  \end{definition}

  \STEXinvisible{\inlineforthel{Let $4$ stand for $\four$.}}

  \begin{definition}[forthel,id=FiveSig]
    $\emph{\five}=\four\plus\one$.

    Let \emph{five} stand for $\five$.
  \end{definition}

  \STEXinvisible{\inlineforthel{Let $5$ stand for $\five$.}}

  \begin{definition}[forthel,id=SixSig]
    $\emph{\six}=\five\plus\one$.

    Let \emph{six} stand for $\six$.
  \end{definition}

  \STEXinvisible{\inlineforthel{Let $6$ stand for $\six$.}}

  \begin{definition}[forthel,id=SevenSig]
    $\emph{\seven}=\six\plus\one$.

    Let \emph{seven} stand for $\seven$.
  \end{definition}

  \STEXinvisible{\inlineforthel{Let $7$ stand for $\seven$.}}

  \begin{definition}[forthel,id=EightSig]
    $\emph{\eight}=\seven\plus\one$.

    Let \emph{eight} stand for $\eight$.
  \end{definition}

  \STEXinvisible{\inlineforthel{Let $8$ stand for $\eight$.}}

  \begin{definition}[forthel,id=NineSig]
    $\emph{\nine}=\eight\plus\one$.

    Let \emph{nine} stand for $\nine$.
  \end{definition}

  \STEXinvisible{\inlineforthel{Let $9$ stand for $\nine$.}}
\end{sfragment}

\begin{sfragment}{The Axioms of Natural Number Arithmetic}
  \begin{axiom}[forthel,title=Infinity Axiom,id=NatIsSetAx]
    $\Nat$ is a set.
  \end{axiom}

  \begin{axiom}[forthel,title=Peano Axiom I,id=PlusIsInjectiveAx]
    If $n\plus\one\eq m\plus\one$ then $n\eq m$.
  \end{axiom}
  \begin{axiom}[forthel,title=Peano Axiom II,id=ZeroIsNoSuccessorAx]
    There exists no natural number $n$ such that $n\plus\one\eq\zero$.
  \end{axiom}

  \begin{axiom}[forthel,title=Peano Axiom III,id=NatNumberIsZeroOrSuccessorAx]
    $n\eq\zero$ or $n\eq m\plus\one$ for some natural number $m$.
  \end{axiom}

  \begin{axiom}[forthel,title=Induction Axiom,id=InductionAx]
    $n\prec n\plus\one$.
  \end{axiom}

  \begin{axiom}[forthel,title=Addition Axiom I,id=AdditionWithZeroAx]
    $n\plus\zero\eq n$.
  \end{axiom}

  \begin{axiom}[forthel,title=Addition Axiom II,id=AdditionWithSuccessorAx]
    $n\plus(m\plus\one)\eq(n\plus m)\plus\one$.
  \end{axiom}
\end{sfragment}

\begin{sfragment}{Immediate Consequences of the Axioms}
  \begin{proposition}[forthel,id=NatNumberIsNotItsOwnSuccessorProp]
    For all $n\in\Nat$ we have $n\neq n\plus\one$.
  \end{proposition}
  \begin{proof}[forthel,method=induction on $n$]\end{proof}

  \begin{proposition}[forthel,id=ZeroIsLeftIdentityOfPlusProp]
    For all $n\in\Nat$ we have $\zero\plus n\eq n$.
  \end{proposition}
  \begin{proof}[forthel,method=induction on $n$]\end{proof}
\end{sfragment}

\begin{sfragment}{Computation Laws for Addition}
  \begin{proposition}[forthel,title=Associativity,id=PlusIsAssociativeProp]
    For all $k\in\Nat$ we have $n\plus(m\plus k)\eq(n\plus m)\plus k$.
  \end{proposition}
  \begin{proof}[forthel,method=induction on $k$]
    Let $k\in\Nat$.

    Case $k\eq\zero$. Obvious.

    Case $k\neq\zero$.
      Take $k'\in\Nat$ such that $k\eq k'\plus\one$.
      Then $k'\prec k$.
      Hence $n\plus(m\plus k')\eq(n\plus m)\plus k'$.
      Thus
      \begin{align*}
        n\plus(m\plus k)
          &\eq n\plus(m\plus(k'\plus\one))\\
          &\eq n\plus((m\plus k')\plus\one)\\
          &\eq (n\plus(m\plus k'))\plus\one\\
          &\eq ((n\plus m)\plus k')\plus\one\\
          &\eq (n\plus m)\plus(k'\plus\one)
      \end{align*}.
    End.
  \end{proof}

  \begin{proposition}[forthel,title=Commutativity,id=PlusIsCommutativeProp]
    For all $n,m\in\Nat$ we have $n\plus m\eq m\plus n$.
  \end{proposition}
  \begin{proof}[forthel,method=induction on $m$]
    We can show by induction on $n$ that for all $n\in\Nat$ we have $n\plus\zero\eq\zero\plus n$.
    Obvious.

    We can show by induction on $n$ that for all $n\in\Nat$ we have $n\plus\one\eq\one\plus n$.
    Obvious.

    Let $m\in\Nat$.

    Case $m\eq\zero$. Trivial.

    Case $m\neq\zero$.
      Let $n\in\Nat$.
      Take $m'\in\Nat$ such that $m\eq m'\plus\one$.
      Then $n\plus m'\eq m'\plus n$.
      Hence
      \begin{align*}
        n\plus m
          &\eq n\plus(m'\plus\one)\\
          &\eq (n\plus m')\plus\one\\
          &\eq (m'\plus n)\plus\one\\
          &\eq m'\plus(n\plus\one)\\
          &\eq m'\plus(\one\plus n)\\
          &\eq (m'\plus\one)\plus n
      \end{align*}.
    End.
  \end{proof}

  \begin{proposition}[forthel,title=Right cancellation,id=PlusIsRightCancellativeProp]
    For all $k\in\Nat$ if $n\plus k\eq m\plus k$ then $n\eq m$.
  \end{proposition}
  \begin{proof}[forthel,method=induction on $k$]
    Let $k\in\Nat$.

    Case $k\eq\zero$. Obvious.

    Case $k\neq\zero$. Obvious.
  \end{proof}

  \begin{corollary}[forthel,title=Left cancellation,id=PlusIsLeftCancellativeProp]
    If $k\plus n\eq k\plus m$ then $n\eq m$.
  \end{corollary}
  \begin{proof}[forthel]
    Assume $k\plus n\eq k\plus m$.
    We have $k\plus n\eq n\plus k$ and $k\plus m\eq m\plus k$.
    Hence $n\plus k\eq m\plus k$.
    Thus $n\eq m$.
  \end{proof}

  \begin{proposition}[forthel,id=ZeroSumImpliesZeroSummandsProp]
    If $n\plus m\eq\zero$ then $n\eq\zero$ and $m\eq\zero$.
  \end{proposition}
\end{sfragment}
\end{smodule}
\end{document}
