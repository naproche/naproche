\documentclass{stex}
\libusepackage{naproche}
\libinput{preamble}
\begin{document}
\begin{smodule}{symdiff-preserves-equality.ftl}

\importmodule[libraries/classes]{definitions?symdiff.ftl}
\importmodule[libraries/classes]{definitions?subclasses.ftl}

\begin{proposition}[forthel,id=SymdiffPreservesEqualityProp]
  Let $A, B, C$ be classes.
  Then $A \symdiff B \eq A \symdiff C \iff B \eq C$.
\end{proposition}
\begin{proof}[forthel]
  \begin{case}{$A \symdiff B \eq A \symdiff C$.}

    Let us show that $B \subclass C$.
      Let $b \elem B$.

      \begin{case}{$b \elem A$.}
        Then $b \nelem A \symdiff B$.
        Hence $b \nelem A \symdiff C$.
        Therefore $b \elem A \intersect C$.
        Indeed $A \symdiff C \eq (A \union C) \diff (A \intersect C)$.
        Hence $b \elem C$.
      \end{case}

      \begin{case}{$b \nelem A$.}
        Then $b \elem A \symdiff B$.
        Indeed $b \elem A \union B$ and $b \nelem A \intersect B$.
        Hence $b \elem A \symdiff C$.
        Thus $b \elem A \union C$ and $b \nelem A \intersect C$.
        Therefore $b \elem A$ or $b \elem C$.
        Then we have the thesis.
      \end{case}
    End.

    Let us show that $C \subclass B$.
      Let $c \elem C$.

      \begin{case}{$c \elem A$.}
        Then $c \nelem A \symdiff C$.
        Hence $c \nelem A \symdiff B$.
        Therefore $c \elem A \intersect B$.
        Indeed $c \nelem A \union B$ or $c \elem A \intersect B$.
        Hence $c \elem B$.
      \end{case}

      \begin{case}{$c \nelem A$.}
        Then $c \elem A \symdiff C$.
        Indeed $c \elem A \union C$ and $c \nelem A \intersect C$.
        Hence $c \elem A \symdiff B$.
        Thus $c \elem A \union B$ and $c \nelem A \intersect B$.
        Therefore $c \elem A$ or $c \elem B$.
        Then we have the thesis.
      \end{case}
    End.
  \end{case}
\end{proof}

\end{smodule}
\end{document}