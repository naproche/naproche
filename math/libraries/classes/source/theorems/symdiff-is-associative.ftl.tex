\documentclass{stex}
\libusepackage{naproche}
\libinput{preamble}
\begin{document}
\begin{smodule}[title=Symmetric Difference is Associative]{symdiff-is-associative.ftl}

\importmodule[libraries/classes]{definitions?symdiff.ftl}
\importmodule[libraries/classes]{definitions?subclasses.ftl}

\begin{proposition}[forthel,id=SymdiffIsAssociativeProp]
  Let $A, B, C$ be classes.
  Then $(A \symdiff B) \symdiff C \eq A \symdiff (B \symdiff C)$.
\end{proposition}
\begin{proof}[forthel]
  Take a class $X$ such that $X \eq (((A \diff B) \union (B \diff A)) \diff C) \union (C \diff ((A \diff B) \union (B \diff A)))$.

  Take a class $Y$ such that $Y \eq (A \diff ((B \diff C) \union (C \diff B))) \union (((B \diff C) \union (C \diff B)) \diff A)$.

  We have $A \symdiff B \eq (A \diff B) \union (B \diff A)$ and $B \symdiff C \eq (B \diff C) \union (C \diff B)$.
  Hence $(A \symdiff B) \symdiff C \eq X$ and $A \symdiff (B \symdiff C) \eq Y$.

  Let us show that (I) $X \subclass Y$.
    Let $x \elem X$.

    \begin{case}{$x \elem ((A \diff B) \union (B \diff A)) \diff C$.}
      Then $x \nelem C$.

      \begin{case}{$x \elem A \diff B$.}
        Then $x \nelem B \diff C$ and $x \nelem C \diff B$.
        $x \elem A$.
        Hence $x \elem A \diff ((B \diff C) \union (C \diff B))$.
        Thus $x \elem Y$.
      \end{case}

      \begin{case}{$x \elem B \diff A$.}
        Then $x \elem B \diff C$.
        Hence $x \elem (B \diff C) \union (C \diff B)$.
        $x \nelem A$.
        Thus $x \elem ((B \diff C) \union (C \diff B)) \diff A$.
        Therefore $x \elem Y$.
      \end{case}
    \end{case}

    \begin{case}{$x \elem C \diff ((A \diff B) \union (B \diff A))$.}
      Then $x \elem C$.
      $x \nelem A \diff B$ and $x \nelem B \diff A$.
      Hence not ($x \elem A \diff B$ or $x \elem B \diff A$).
      Thus not (($x \elem A$ and $x \nelem B$) or ($x \elem B$ and $x \nelem A$)).
      Therefore ($x \nelem A$ or $x \elem B$) and ($x \nelem B$ or $x \elem A$).

      \begin{case}{$x \elem A$.}
        Then $x \elem B$.
        Hence $x \nelem (B \diff C) \union (C \diff B)$.
        Thus $x \elem A \diff ((B \diff C) \union (C \diff B))$.
        Therefore $x \elem Y$.
      \end{case}

      \begin{case}{$x \nelem A$.}
        Then $x \nelem B$.
        Hence $x \elem C \diff B$.
        Thus $x \elem (B \diff C) \union (C \diff B)$.
        Therefore $x \elem ((B \diff C) \union (C \diff B)) \diff A$.
        Then we have $x \elem Y$.
      \end{case}
    \end{case}
  End.

  Let us show that (II) $Y \subclass X$.
    Let $y \elem Y$.

    \begin{case}{$y \elem A \diff ((B \diff C) \union (C \diff B))$.}
      Then $y \elem A$.
      $y \nelem B \diff C$ and $y \nelem C \diff B$.
      Hence not ($y \elem B \diff C$ or $y \elem C \diff B$).
      Thus not (($y \elem B$ and $y \nelem C$) or ($y \elem C$ and $y \nelem B$)).
      Therefore ($y \nelem B$ or $y \elem C$) and ($y \nelem C$ or $y \elem B$).

      \begin{case}{$y \elem B$.}
        Then $y \elem C$.
        $y \nelem A \diff B$ and $y \nelem B \diff A$.
        Hence $y \nelem (A \diff B) \union (B \diff A)$.
        Thus $y \elem C \diff ((A \diff B) \union (B \diff A))$.
        Therefore $y \elem X$.
      \end{case}

      \begin{case}{$y \nelem B$.}
        Then $y \nelem C$.
        $y \elem A \diff B$.
        Hence $y \elem (A \diff B) \union (B \diff A)$.
        Thus $y \elem ((A \diff B) \union (B \diff A)) \diff C$.
        Therefore $y \elem X$.
      \end{case}
    \end{case}

    \begin{case}{$y \elem ((B \diff C) \union (C \diff B)) \diff A$.}
      Then $y \nelem A$.

      \begin{case}{$y \elem B \diff C$.}
        Then $y \elem B \diff A$.
        Hence $y \elem (A \diff B) \union (B \diff A)$.
        Thus $y \elem ((A \diff B) \union (B \diff A)) \diff C$.
        Therefore $y \elem X$.
      \end{case}

      \begin{case}{$y \elem C \diff B$.}
        Then $y \elem C$.
        $y \nelem A \diff B$ and $y \nelem B \diff A$.
        Hence $y \nelem (A \diff B) \union (B \diff A)$.
        Thus $y \elem C \diff ((A \diff B) \union (B \diff A))$.
        Therefore $y \elem X$.
      \end{case}
    \end{case}
  End.
\end{proof}

\end{smodule}
\end{document}