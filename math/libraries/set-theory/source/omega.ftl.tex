\documentclass{naproche-library}
\libinput{preamble}
\begin{document}
\IfInputref{}{\setsectionlevel{section}}
\begin{smodule}[title=Natural Numbers]{omega.ftl}
\importmodule[libraries/set-theory]{zero.ftl}
\importmodule[libraries/set-theory]{successor-ordinals.ftl}

\begin{definition}[forthel,id=SET_THEORY_03_4310076227584000]
  \[ \omega = \class{n \in \Ord | \classtext{$n \in X$ for every $X \subseteq \Ord$ such that $0 \in X$ and for all $x \in X$ we have $\succ(x) \in X$}}. \]

Let a natural number stand for an element of $\omega$.
\end{definition}

\begin{proposition}[forthel,id=SET_THEORY_03_3576717620805632]
  $0 \in \omega$.
\end{proposition}

\begin{proposition}[forthel,id=SET_THEORY_03_8807317141192704]
  Let $n \in \omega$.
  Then $\succ(n) \in \omega$.
\end{proposition}

\begin{proposition}[forthel,id=SET_THEORY_03_344585425387520]
  Let $\Phi \subseteq \omega$.
  Assume that $0 \in \Phi$ and for every $x \in \Phi$ we have
  $\succ(x) \in \Phi$.
  Then $\Phi = \omega$.
\end{proposition}
\begin{proof}[forthel]
  Suppose $\Phi \neq \omega$.
  Consider an element $n$ of $\omega$ that is not contained in $\Phi$.
  Take $\Phi' = \Phi \setminus \set{n}$.

  (1) $0 \in \Phi'$.
  Indeed $0 \in \Phi$ and $0 \neq n$.

  (2) For each $x \in \Phi'$ we have $\succ(x) \in \Phi'$. \newline
  Proof.
    Let $x \in \Phi'$.
    Then $\succ(x) \in \Phi$.

    Let us show that $\succ(x) \neq n$.
      Assume $\succ(x) = n$.
      Then $x \notin \Phi$.
      Indeed $n \notin \Phi$ and if $x \in \Phi$ then
      $n = \succ(x) \in \Phi$.
      Contradiction.
    End.

    Thus $\succ(x) \in \Phi'$.
  Qed.

  Therefore every element of $\omega$ lies in $\Phi'$.
  Indeed $\Phi' \subseteq \Ord$.
  Consequently $n \in \Phi'$.
  Contradiction.
\end{proof}

\begin{corollary}[forthel,id=SET_THEORY_03_4847727433220096]
  $\omega$ is a set.
\end{corollary}
\begin{proof}[forthel]
  Define $f(n) = \succ(n)$ for $n \in \omega$.
  Take a subset $X$ of $\omega$ that is inductive regarding $0$ and $f$.
  Indeed $f$ is a map from $\omega$ to $\omega$.
  Then we have $0 \in X$ and for each $n \in X$ we have $\succ(n) \in X$.
  Thus $X = \omega$.
  Therefore $\omega$ is a set.
\end{proof}

\begin{proposition}[forthel,id=SET_THEORY_03_5885789275684864]
  Let $n \in \omega$.
  Then $n = 0$ or $n = \succ(m)$ for some $m \in \omega$.
\end{proposition}
\begin{proof}[forthel]
  Assume the contrary.
  Consider a $k \in \omega$ such that neither $k = 0$ nor $k = \succ(m)$ for
  some $m \in \omega$.
  Take a class $\omega'$ such that $\omega' = \omega \setminus \set{k}$. %!
  Then $\omega'$ is a set.

  (1) $0 \in \omega'$.
  Indeed $k \neq 0$.

  (2) For all $m \in \omega'$ we have $\succ(m) \in \omega'$. \newline
  Proof.
    Let $m \in \omega'$.
    Then $\succ(m) \neq k$.
    Hence $\succ(m) \in \omega'$.
  Qed.

  Thus every element of $\omega$ is contained in $\omega'$.
  Therefore $k \in \omega'$.
  Contradiction.
\end{proof}

\begin{proposition}[forthel,id=SET_THEORY_03_5057540872208384]
  Every element of $\omega$ is an ordinal.
\end{proposition}
\end{smodule}
\end{document}
