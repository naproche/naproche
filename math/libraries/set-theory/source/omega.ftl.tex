\documentclass{naproche-library}
\libinput{preamble}
\begin{document}
\IfInputref{}{\setsectionlevel{section}}
\begin{smodule}[title=The Ordinal $\omega$]{omega.ftl}
\importmodule[libraries/set-theory]{zero.ftl}
\importmodule[libraries/set-theory]{successor-ordinals.ftl}
\symdef{ORDomega}{\omega}

\begin{definition}[forthel,id=SET_THEORY_03_4310076227584000]
  \[ \emph{\ORDomega} = \class{n \in \Ord | \classtext{$n \in X$ for every $X \SETinclude \Ord$ such that $\ORDzero \in X$ and for all $x \in X$ we have $\ORDsucc(x) \in X$}}. \]

Let a \emph{natural number} stand for an element of $\ORDomega$.
\end{definition}

\begin{proposition}[forthel,id=SET_THEORY_03_3576717620805632]
  $\ORDzero \in \ORDomega$.
\end{proposition}

\begin{proposition}[forthel,id=SET_THEORY_03_8807317141192704]
  Let $n \in \ORDomega$.
  Then $\ORDsucc(n) \in \ORDomega$.
\end{proposition}

\begin{proposition}[forthel,id=SET_THEORY_03_344585425387520]
  Let $\Phi \SETinclude \ORDomega$.
  Assume that $\ORDzero \in \Phi$ and for every $x \in \Phi$ we have
  $\ORDsucc(x) \in \Phi$.
  Then $\Phi \eq \ORDomega$.
\end{proposition}
\begin{proof}[forthel]
  Suppose $\Phi \neq \ORDomega$.
  Consider an element $n$ of $\ORDomega$ that is not contained in $\Phi$.
  Take $\Phi' \eq \Phi \SETdiff \SETsingleton{n}$.

  (1) $\ORDzero \in \Phi'$.
  Indeed $\ORDzero \in \Phi$ and $\ORDzero \neq n$.

  (2) For each $x \in \Phi'$ we have $\ORDsucc(x) \in \Phi'$.
  \begin{proof}
    Let $x \in \Phi'$.
    Then $\ORDsucc(x) \in \Phi$.

    Let us show that $\ORDsucc(x) \neq n$.
      Assume $\ORDsucc(x) \eq n$.
      Then $x \notin \Phi$.
      Indeed $n \notin \Phi$ and if $x \in \Phi$ then
      $n \eq \ORDsucc(x) \in \Phi$.
      Contradiction.
    End.

    Thus $\ORDsucc(x) \in \Phi'$.
  \end{proof}

  Therefore every element of $\ORDomega$ lies in $\Phi'$.
  Indeed $\Phi' \SETinclude \Ord$.
  Consequently $n \in \Phi'$.
  Contradiction.
\end{proof}

\begin{corollary}[forthel,id=SET_THEORY_03_4847727433220096]
  $\ORDomega$ is a set.
\end{corollary}
\begin{proof}[forthel]
  Define $f(n) = \ORDsucc(n)$ for $n \in \ORDomega$.
  Take a subset $X$ of $\ORDomega$ that is inductive regarding $\ORDzero$ and $f$.
  Indeed $f$ is a map from $\ORDomega$ to $\ORDomega$.
  Then we have $\ORDzero \in X$ and for each $n \in X$ we have $\ORDsucc(n) \in X$.
  Thus $X \eq \ORDomega$.
  Therefore $\ORDomega$ is a set.
\end{proof}

\begin{proposition}[forthel,id=SET_THEORY_03_5885789275684864]
  Let $n \in \ORDomega$.
  Then $n \eq \ORDzero$ or $n \eq \ORDsucc(m)$ for some $m \in \ORDomega$.
\end{proposition}
\begin{proof}[forthel]
  Assume the contrary.
  Consider a $k \in \ORDomega$ such that neither $k \eq \ORDzero$ nor $k \eq \ORDsucc(m)$ for
  some $m \in \ORDomega$.
  Take a class $\ORDomega'$ such that $\ORDomega' \eq \ORDomega \SETdiff \SETsingleton{k}$. %!
  Then $\ORDomega'$ is a set.

  (1) $\ORDzero \in \ORDomega'$.
  Indeed $k \neq \ORDzero$.

  (2) For all $m \in \ORDomega'$ we have $\ORDsucc(m) \in \ORDomega'$.
  \begin{proof}
    Let $m \in \ORDomega'$.
    Then $\ORDsucc(m) \neq k$.
    Hence $\ORDsucc(m) \in \ORDomega'$.
  \end{proof}

  Thus every element of $\ORDomega$ is contained in $\ORDomega'$.
  Therefore $k \in \ORDomega'$.
  Contradiction.
\end{proof}

\begin{proposition}[forthel,id=SET_THEORY_03_5057540872208384]
  Every element of $\ORDomega$ is an ordinal.
\end{proposition}
\end{smodule}
\end{document}
