\documentclass{naproche-library}
\libinput{preamble}
\begin{document}
\IfInputref{}{\setsectionlevel{section}}
\begin{smodule}[title=Transfinite Recursion II]{transfinite-recursion-2.ftl}
\importmodule[libraries/set-theory]{transfinite-recursion-1.ftl}

\begin{lemma}[forthel,id=transfinite_recursion_029348573987483]
  Let $A$ be a class and $\alpha$ be an ordinal and $F \FUNfromto{\Ord}{A}$.
  Then $(\alpha, F(\alpha)) \SETin \Ord \SETprod A$.
\end{lemma}

\begin{lemma}[forthel,id=transfinite_recursion_570293857449281]
  Let $A$ be a class and $\lambda$ be a limit ordinal and $F \FUNfromto{\Ord}{A}$.
  Then $(\lambda, F \FUNrest \lambda) \SETin \Ord \SETprod \ORDfunspace{A}$.
\end{lemma}

\begin{corollary}[forthel,id=transfinite_recursion_298647309184593]
  Let $A$ be a class.
  Let $a \SETin A$ and $G \FUNfromto{\Ord \SETprod A}{A}$ and $H \FUNfromto{\Ord \SETprod \ORDfunspace{A}}{A}$.
  Then there exists a map $F$ from $\Ord$ to $A$ such that
  \[ F(\ORDzero) \eq a \]
  and for all ordinals $\alpha$ we have
  \[ F(\ORDsucc(\alpha)) \eq G(\alpha, F(\alpha)) \]
  and for all limit ordinals $\lambda$ we have
  \[ F(\lambda) \eq H(\lambda, F \FUNrest \lambda). \]
\end{corollary}
\begin{proof}[forthel]
  $(\ORDpred(\FUNdom(f)), f(\ORDpred(\FUNdom(f)))) \SETin \Ord \SETprod A$ for all $f \SETin \ORDfunspace{A}$ such that $\FUNdom(f)$ is a successor ordinal.

  Define  \[ J(f) =
    \begin{cases}
      a
      & : \FUNdom(f) \eq \ORDzero
      \\
      G(\ORDpred(\FUNdom(f)), f(\ORDpred(\FUNdom(f))))
      & : \text{$\FUNdom(f)$ is a successor ordinal}
      \\
      H(\FUNdom(f), f)
      & : \text{$\FUNdom(f)$ is a limit ordinal}
    \end{cases} \]
  for $f \in \ORDfunspace{A}$.

  Then $J$ is a map from $\ORDfunspace{A}$ to $A$.
  Indeed we can show that for any $f \SETin \ORDfunspace{A}$ we have $J(f) \SETin A$.
    Let $f \SETin \ORDfunspace{A}$.
    Take $\alpha \SETin \Ord$ such that $f \FUNfromto{\alpha}{A}$.
    If $\alpha \eq \ORDzero$ then $J(f) \eq a \SETin A$.
    If $\alpha$ is a successor ordinal then $J(f) \eq
    G(\ORDpred(\alpha), f(\ORDpred(\alpha))) \SETin A$.
    [prover vampire]
    If $\alpha$ is a limit ordinal then $J(f) \eq H(\alpha, f) \SETin A$.
  End.

  Hence we can take a map $F$ from $\Ord$ to $A$ that is recursive regarding $J$.
  Then $F \FUNrest \alpha \SETin \ORDfunspace{A}$ for any ordinal $\alpha$.

  (1) $F(\ORDzero) \eq a$.
  \begin{proof}
    $F(\ORDzero)
      \eq J(F \FUNrest \ORDzero)
      \eq a$.
  \end{proof}

  (2) $F(\ORDsucc(\alpha)) \eq G(\alpha, F(\alpha))$ for all ordinals $\alpha$.
  \begin{proof}
    Let $\alpha$ be an ordinal.
    Then $F(\ORDsucc(\alpha))
      \eq J(F \FUNrest \ORDsucc(\alpha))
      \eq G(\ORDpred(\ORDsucc(\alpha)), (F \FUNrest \ORDsucc(\alpha))(\ORDpred(\ORDsucc(\alpha))))
      \eq G(\alpha, (F \FUNrest \ORDsucc(\alpha))(\alpha))
      \eq G(\alpha, F(\alpha))$.
  \end{proof}

  (3) $F(\lambda) \eq H(\lambda, F \FUNrest \lambda)$ for all limit ordinals $\lambda$.
  \begin{proof}
    Let $\lambda$ be a limit ordinal.
    Then $F(\lambda)
      \eq J(F \FUNrest \lambda)
      \eq H(\lambda, F \FUNrest \lambda)$.
  \end{proof}

  Hence the thesis (by 1, 2, 3).
\end{proof}
\end{smodule}
\end{document}
