\documentclass{naproche-library}
\libinput{preamble}
\begin{document}
\IfInputref{}{\setsectionlevel{section}}
\begin{smodule}[title=Transfinite Induction]{transfinite-induction.ftl}
\importmodule[libraries/set-theory]{ordering.ftl}

\begin{theorem}[forthel,id=SET_THEORY_02_8493935460614144]
  Let $\Phi$ be a class.
  Assume that for all ordinals $\alpha$ if $\Phi$ contains all $\beta \in \alpha$ then $\Phi$ contains $\alpha$.
  Then $\Phi$ contains every ordinal.
\end{theorem}
\begin{proof}[forthel]
  Define $B = \{ x \mid x$ is a set and if $x \in \Ord$ then $x \in \Phi \}$.

  Let us show that for all sets $x$ if $B$ contains every element of $x$ that is a set then $B$ contains $x$.
    Let $x$ be a set.
    Assume that every element of $x$ that is a set is contained in $B$.

    Case $x \notin \Ord$. Trivial.

    Case $x \in \Ord$.
      Then $\Phi$ contains all ordinals less than $x$.
      Hence $\Phi$ contains $x$.
      Thus $x \in B$.
    End.
  End.

  [prover vampire]
  Hence $B$ contains every set (by \printref{FOUNDATIONS_11_2812087589928960}).
  [prover eprover]
  Thus $\Phi$ contains every ordinal.
\end{proof}

\begin{theorem}[forthel,id=SET_THEORY_02_7892040431960064]
  Let $\Phi$ be a class.
  
  (Initial case) Assume that $\Phi$ contains $\ORDzero$.
  
  (Successor step) Assume that for all ordinals $\alpha$ if $\alpha \in \Phi$ then $\ORDsucc(\alpha) \in \Phi$.
  
  (Limit step) Assume that for all limit ordinals $\lambda$ if every $\alpha \in \lambda$ is contained in $\Phi$ then $\lambda \in \Phi$.
  
  Then $\Phi$ contains every ordinal.
\end{theorem}
\begin{proof}[forthel]
  Let us show that for all ordinals $\alpha$ if $\Phi$ contains all ordinals
  less than $\alpha$ then $\Phi$ contains $\alpha$.
    Let $\alpha$ be an ordinal.
    Then $\alpha \eq \ORDzero$ or $\alpha$ is a successor ordinal or $\alpha$ is a
    limit ordinal.
    Assume that $\Phi$ contains all $\beta \in \alpha$.

    Case $\alpha \eq \ORDzero$. Trivial.

    Case $\alpha$ is a successor ordinal.
      Take an ordinal $\beta$ such that $\alpha \eq \ORDsucc(\beta)$.
      Then $\beta \in \Phi$.
      Hence $\alpha \in \Phi$ (by successor step).
    End.

    Case $\alpha$ is a limit ordinal.
      Then $\beta \in \Phi$ for all ordinals $\beta$ less than $\alpha$.
      Hence $\alpha \in \Phi$ (by limit step).
    End.
  End.

  [prover vampire]
  Thus $\Phi$ contains every ordinal (by \printref{SET_THEORY_02_8493935460614144}).
  [prover eprover]
\end{proof}
\end{smodule}
\end{document}
