\documentclass{naproche-library}
\libinput{preamble}
\begin{document}
\IfInputref{}{\setsectionlevel{section}}
\begin{smodule}[title=Natural Numbers are Cardinal Numbers]{natural-numbers-are-cardinals.ftl}
\importmodule[libraries/set-theory]{omega-is-limit-ordinal.ftl}
\importmodule[libraries/set-theory]{cardinals.ftl}

\begin{theorem}[forthel,id=SET_THEORY_07_2948332552978432]
  Let $n \SETin \ORDomega$.
  Then $\SETcard{n} \eq n$.
\end{theorem}
\begin{proof}[forthel]
  Define $\Phi = \{ n' \in \ORDomega \mid \SETcard{n'} \eq n' \}$.

  (1) $\ORDzero \SETin \Phi$.
  Indeed $\SETcard{\ORDzero} \eq \SETcard{\SETempty} \eq \ORDzero$.

  (2) For all $n' \SETin \Phi$ we have $\ORDsucc(n') \SETin \Phi$.
  \begin{proof}
    Let $n' \SETin \Phi$.
    Then $\SETcard{n'} \eq n'$.
    We have $\SETcard{\ORDsucc(n')} \ORDleq \ORDsucc(n')$.

    Let us show that $\ORDsucc(n') \ORDleq \SETcard{\ORDsucc(n')}$.
      Assume the contrary.
      Then $\SETcard{\ORDsucc(n')} \ORDless \ORDsucc(n')$.
      Take a bijection $f$ between $\SETcard{\ORDsucc(n')}$ and $\ORDsucc(n')$.
      $\SETcard{\ORDsucc(n')}$ is nonzero.
      Hence we can take a $m \SETin \ORDomega$ such that $\SETcard{\ORDsucc(n')} \eq \ORDsucc(m)$.
      Indeed $\SETcard{\ORDsucc(n')} \SETin \ORDsucc(\ORDsucc(n')) \SETin \ORDomega$.
      Then $\FUNinv{f}(n') \ORDleq m$.

      We can show that $\FUNinv{f}(n') \ORDless m$.
        Assume the contrary.
        Then $\FUNinv{f}(n') \eq m$.
        $f \FUNrest m$ is a bijection between $m$ and $\FUNim{f}{m}$ (by \printref{FOUNDATIONS_08_647446231252992}).
        Indeed $f$ is an injective map from $\SETcard{\ORDsucc(n')}$ to $\ORDsucc(n')$ and
        $m \SETinclude \SETcard{\ORDsucc(n')}$.
        We have $\FUNim{f}{m} \SETinclude n'$ and $n' \SETinclude \FUNim{f}{m}$.
        Hence $\FUNim{f}{m} \eq n'$.
        Thus $f \FUNrest m$ is a bijection between $m$ and $n'$.
        Therefore $n'
          \eq \SETcard{n'}
          \ORDleq m
          \ORDless \SETcard{\ORDsucc(n')}
          \ORDless \ORDsucc(n')$.
        Consequently $m \eq n'$.
        Then we have $\ORDsucc(n') \eq \SETcard{\ORDsucc(n')} \ORDless \ORDsucc(n')$.
        Contradiction.
      End.

      Define \[ g(i) =
        \begin{cases}
          f(i)  & : i \neq \FUNinv{f}(n')
          \\
          f(m)  & : i \eq \FUNinv{f}(n')
        \end{cases} \]
      for $i \in m$.

      $g$ is a map from $m$ to $n'$.
      Indeed we can show that $g(i) \SETin n'$ for each $i \SETin m$.
      \begin{proof}
        Let $i \SETin m$.
        We have $g(i) \SETin \ORDsucc(n')$.
        Indeed $f(i), f(m) \SETin \ORDsucc(n')$ and ($g(i) \eq f(i)$ or $g(i) \eq f(m)$).
        Hence if $g(i) \neq n'$ then $g(i) \SETin n'$.

        \begin{case}{$i \neq \FUNinv{f}(n')$.}
          Then $g(i) \eq f(i) \SETin \ORDsucc(n')$.
          If $g(i) \eq n'$ then $f(i) \eq n' \eq f(\FUNinv{f}(n'))$.
          Hence if $g(i) \eq n'$ then $i \eq \FUNinv{f}(n')$.
          Thus $g(i) \neq n'$.
          Therefore $g(i) \SETin n'$.
        \end{case}

        \begin{case}{$i \eq \FUNinv{f}(n')$.}
          Then $g(i)
            \eq f(m)
            \neq f(\FUNinv{f}(n'))
            \eq n'$.
          Hence $g(i) \SETin n'$.
        \end{case}
      \end{proof}

      $g$ is surjective onto $n'$.
      Indeed we can show that for all $k \SETin n'$ there exists a $l \SETin m$ such that $k \eq g(l)$.
      \begin{proof}
        Let $k \SETin n'$.
        Then $\FUNinv{f}(k) \neq \FUNinv{f}(n')$.

        \begin{case}{$\FUNinv{f}(k) \eq m$.}
          Then $k
            \eq f(\FUNinv{f}(k))
            \eq f(m)
            \eq g(\FUNinv{f}(n'))$.
        \end{case}

        \begin{case}{$\FUNinv{f}(k) \neq m$.}
          Then $\FUNinv{f}(k) \SETin m$.
          Indeed $\FUNinv{f}(k) \SETin \SETcard{\ORDsucc(n')} \eq \ORDsucc(m) \eq m \SETunion \SETsingleton{m}$.
          Hence $k
            \eq f(\FUNinv{f}(k))
            \eq g(\FUNinv{f}(k))$.
        \end{case}
      \end{proof}

      $g$ is injective.
      Indeed we can show that for all $i, j \SETin m$ if $i \neq j$ then $g(i) \neq g(j)$.
      \begin{proof}
        Let $i, j \SETin m$.
        Assume $i \neq j$.

        \begin{case}{$i, j \neq \FUNinv{f}(n')$.}
          Then $g(i)
            \eq f(i)
            \neq f(j)
            \eq g(j)$.
        \end{case}

        \begin{case}{$i \eq \FUNinv{f}(n')$.}
          Then $j \neq \FUNinv{f}(n')$.
          Hence $g(i)
            \eq g(\FUNinv{f}(n'))
            \eq f(m)
            \neq f(j)
            \eq g(j)$.
          Indeed $m \neq j$.
        \end{case}

        \begin{case}{$j \eq \FUNinv{f}(n')$.}
          Then $i \neq \FUNinv{f}(n')$.
          Hence $g(i)
            \eq f(i)
            \neq f(m)
            \eq g(\FUNinv{f}(n'))
            \eq g(j)$.
          Indeed $i \neq m$.
        \end{case}
      \end{proof}
    End.
  \end{proof}

  Thus $\ORDomega \SETinclude \Phi$ (by \printref{SET_THEORY_03_344585425387520}).
  Consequently $n \SETin \Phi$.
  Therefore $\SETcard{n} \eq n$.
\end{proof}

\begin{corollary}[forthel,id=SET_THEORY_07_7061392098066432]
  Every natural number is a cardinal.
\end{corollary}
\end{smodule}
\end{document}
