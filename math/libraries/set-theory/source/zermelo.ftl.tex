\documentclass{naproche-library}
\libinput{preamble}
\begin{document}
\IfInputref{}{\setsectionlevel{section}}
\begin{smodule}[title=Zermelo's Well-Ordering Theorem]{zermelo.ftl}
\importmodule[libraries/foundations]{equinumerosity.ftl}
\importmodule[libraries/foundations]{choice-maps.ftl}
\importmodule[libraries/foundations]{proper-classes.ftl}
\importmodule[libraries/set-theory]{transfinite-recursion-1.ftl}

\begin{theorem}[forthel,title=Zermelo's Well-Ordering Theorem,id=zermelo]
  Every set is equinumerous to some ordinal.
\end{theorem}
\begin{proof}[forthel]
  Let $x$ be a set.
  Consider a choice map $g$ for $\SETpow(x) \SETdiff \SETsingleton{\SETempty}$.
  Take $A \eq x \SETunion \SETsingleton{x}$.
  Every $F \in \ORDfunspace{A}$ is a map from some ordinal to $A$.
  For any $F \in \ORDfunspace{A}$ if $x \SETdiff \FUNrange(F) \neq \SETempty$ then $x \SETdiff \FUNrange(F) \in \dom(g)$.
  Indeed $x \SETdiff \FUNrange(F)$ is a subset of $x$ for any $F \in \ORDfunspace{A}$.
  Define \[ G(F) =
    \begin{cases}
      g(x \SETdiff \FUNrange(F))
      & : x \SETdiff \FUNrange(F) \neq \SETempty
      \\
      x
      & : x \SETdiff \FUNrange(F) \eq \SETempty
    \end{cases} \]
  for $F \in \ORDfunspace{A}$.
  We can show that for any $F \in \ORDfunspace{A}$ if $x \SETdiff \FUNrange(F) \neq \SETempty$ then $G(F) \in x \SETdiff \FUNrange(F)$.
    Let $F \in \ORDfunspace{A}$.
    Assume $x \SETdiff \FUNrange(F) \neq \SETempty$.  
    Then $G(F) \in x \SETdiff \FUNrange(F)$.
  End.
  $G$ is a map from $\ORDfunspace{A}$ to $A$.
  Indeed we can show that for any $F \in \ORDfunspace{A}$ we have $G(F) \in A$.
    Let $F \in \ORDfunspace{A}$.
    If $x \SETdiff \FUNrange(F) \neq \SETempty$ then $G(F) \in x \SETdiff \FUNrange(F)$.
    If $x \SETdiff \FUNrange(F) \eq \SETempty$ then $G(F) \eq x$.
    Hence $G(F) \in A$.
  End.
  Hence we can take a map $F$ from $\Ord$ to $A$ that is recursive regarding $G$.
  For any ordinal $\alpha$ we have $F \FUNrest \alpha \in \ORDfunspace{A}$.

  For any $\alpha \in \Ord$ we have
  \[ x \SETdiff F[\alpha] \neq \SETempty \implies F(\alpha) \in x \SETdiff F[\alpha] \]
  and
  \[ x \SETdiff F[\alpha] \eq \SETempty \implies F(\alpha) \eq x. \]
  Proof.
    Let $\alpha \in \Ord$.
    We have $\FUNim{F}{\alpha} \eq \{ F(\beta) \mid \beta \in \alpha \}$.
    Hence $\FUNim{F}{\alpha} \eq \{ G(F \FUNrest \beta) \mid \beta \in \alpha \}$.
    We have $\FUNrange(F \FUNrest \alpha) \eq \{ F(\beta) \mid \beta \in \alpha \}$.
    Thus $\FUNrange(F \FUNrest \alpha) \eq \FUNim{F}{\alpha}$.

    Case $x \SETdiff \FUNim{F}{\alpha} \neq \SETempty$.
      Then $x \SETdiff \FUNrange(F \FUNrest \alpha) \neq \SETempty$.
      Hence $F(\alpha)
        \eq G(F \FUNrest \alpha)
        \in x \SETdiff \FUNrange(F \FUNrest \alpha)
        \eq x \SETdiff F[\alpha]$.
    End.

    Case $x \SETdiff \FUNim{F}{\alpha} \neq \SETempty$.
      Then $x \SETdiff \FUNrange(F \FUNrest \alpha) \eq \SETempty$.
      Hence $F(\alpha)
        \eq G(F \FUNrest \alpha)
        \eq x$.
    End.
  Qed.

  (1) For any ordinals $\alpha, \beta$ such that $\alpha \ORDless \beta$ and $F(\beta) \neq x$ we have $F(\alpha), F(\beta) \in x$ and $F(\alpha) \neq F(\beta)$. \newline
  Proof.
    Let $\alpha, \beta \in \Ord$.
    Assume $\alpha \ORDless \beta$ and $F(\beta) \neq x$.
    Then $x \SETdiff \FUNim{F}{\beta} \neq \SETempty$.
    (a) Hence $F(\beta) \in x \SETdiff \FUNim{F}{\beta}$.
    We have $\FUNim{F}{\alpha} \SETinclude \FUNim{F}{\beta}$.
    Thus $x \SETdiff \FUNim{F}{\alpha} \neq \SETempty$.
    (b) Therefore $F(\alpha) \in x \SETdiff \FUNim{F}{\alpha}$.
    Consequently $F(\alpha), F(\beta) \in x$ (by a, b).
    We have $F(\alpha) \in \FUNim{F}{\beta}$ and $F(\beta) \notin \FUNim{F}{\beta}$.
    Thus $F(\alpha) \neq F(\beta)$.
  Qed.

  (2) There exists an ordinal $\alpha$ such that $F(\alpha) \eq x$. \newline
  Proof.
    Assume the contrary.
    Then $F$ is a map from $\Ord$ to $x$.

    Let us show that $F$ is injective.
      Let $\alpha, \beta \in \Ord$.
      Assume $\alpha \neq \beta$.
      Then $\alpha \ORDless \beta$ or $\beta \ORDless \alpha$.
      Hence $F(\alpha) \neq F(\beta)$ (by 1).
      Indeed $F(\alpha), F(\beta) \neq x$.
    End.

    Thus $F$ is an injective map from some proper class to some set.
    Contradiction.
  Qed.

  Define $\Phi = \{ \alpha \in \Ord \mid F(\alpha) \eq x \}$.
  $\Phi$ is nonempty.
  Let us show that we can take an $\alpha \in \Phi$ such that $\alpha \ORDleq \beta$ for all $\beta \in \Phi$.
    Take an $\alpha \in \Phi$ such that for no $\beta \in \Phi$ we have $\beta \ORDless \alpha$ (by \printref{FOUNDATIONS_10_9532145874510315}).
    Then $\alpha \ORDleq \beta$ for all $\beta \in \Phi$.
  End.
  Take $f \eq F \FUNrest \alpha$.
  Then $f$ is a map from $\alpha$ to $x$.
  Indeed for no $\beta \in \alpha$ we have $F(\beta) \eq x$.

  (3) $f$ is surjective onto $x$. \newline
  Proof.
    $x \SETdiff \FUNim{F}{\alpha} \eq \SETempty$.
    Hence $\FUNrange(f)
      \eq \FUNim{f}{\alpha}
      \eq \FUNim{F}{\alpha}
      \eq x$.
  Qed.

  (4) $f$ is injective. \newline
  Proof.
    Let $\beta, \gamma \in \alpha$.
    Assume $\beta \neq \gamma$.
    We have $f(\beta), f(\gamma) \neq x$.
    Hence $f(\beta) \neq f(\gamma)$ (by 1).
    Indeed $\beta \ORDless \gamma$ or $\gamma \ORDless \beta$.
  Qed.

  Therefore $f$ is a bijection between $\alpha$ and $x$.
  Consequently $x$ and $\alpha$ are equinumerous.
\end{proof}
\end{smodule}
\end{document}
