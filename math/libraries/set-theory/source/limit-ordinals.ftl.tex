\documentclass{naproche-library}
\libinput{preamble}
\begin{document}
\IfInputref{}{\setsectionlevel{section}}
\begin{smodule}[title=Limit Ordinals]{limit-ordinals.ftl}
\importmodule[libraries/set-theory]{zero.ftl}
\importmodule[libraries/set-theory]{successor-ordinals.ftl}

\begin{convention}[forthel]
  Let $\lim x$ stand for $\bigcup x$.
\end{convention}

\begin{proposition}[forthel,id=SET_THEORY_02_7202164443185152]
  Let $x$ be a subset of $\Ord$.
  Then $\lim x$ is an ordinal.
\end{proposition}
\begin{proof}[forthel]
  (1) $\lim x$ is transitive. \newline
  Proof.
    Let $y \in \lim x$ and $z \in y$.
    Take $w \in x$ such that $y \in w$.
    Hence $w$ is transitive.
    Thus $z \in w$.
    Therefore $z \in \lim x$.
  Qed.

  (2) Every element of $\lim x$ is transitive. \newline
  Proof.
    Let $y \in \lim x$.
    Let $z \in y$ and $v \in z$.
    Take $w \in x$ such that $y \in w$.
    Hence $w$ is an ordinal.
    Thus $y$ is an ordinal.
    Therefore $y$ is transitive.
    Consequently $v \in y$.
  Qed.
\end{proof}

\begin{definition}[forthel,id=SET_THEORY_02_7678388934279168]
  A \emph{limit ordinal} is an ordinal $\lambda$ such that neither $\lambda$ is a successor ordinal nor $\lambda = 0$.
\end{definition}
\end{smodule}
\end{document}
