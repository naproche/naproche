\documentclass{naproche-library}
\libinput{preamble}
\begin{document}
\IfInputref{}{\setsectionlevel{section}}
\begin{smodule}[title=The Knaster-Tarski Fixed Point Theorem]{knaster-tarski.ftl}
\importmodule[libraries/foundations]{equinumerosity.ftl}
\importmodule[libraries/foundations]{maps-and-systems-of-sets.ftl}
\importmodule[libraries/set-theory]{zfc.ftl}

\begin{theorem}[forthel,title=Knaster-Tarski,id=knaster_tarski]
  Let $x$ be a set.
  Let $h$ be a map from $\SETpow(x)$ to $\SETpow(x)$ that preserves subsets.
  Then $h$ has a fixed point.
\end{theorem}
\begin{proof}[forthel]
  (1) Define $A = \{y \in \SETpow(x) \mid y \SETinclude h(y)\}$.
  Then $A$ is a subset of $\SETpow(x)$.
  We have $\SETunionover A \in \SETpow(x)$.

  Let us show that (2) $\SETunionover A \SETinclude h(\SETunionover A)$.
    Let $u \in \SETunionover A$.
    Take $y \in A$ such that $u \in y$.
    Then $u \in h(y)$.
    We have $y \SETinclude \SETunionover A$.
    Hence $h(y) \SETinclude h(\SETunionover A)$.
    Indeed $h$ is a subset preserving map between systems of sets and $y, \SETunionover A \in \dom(h)$.
    Thus $h(y) \SETinclude h(\SETunionover A)$.
    Therefore $u \in h(\SETunionover A)$.
  End.

  Then $h(\SETunionover A) \in A$ (by 1).
  Indeed $h(\SETunionover A) \SETinclude x$.
  (3) Hence $h(\SETunionover A) \SETinclude \SETunionover A$.
  Indeed every element of $h(\SETunionover A)$ is an element of some element of
  $A$.

  Thus $h(\SETunionover A) \eq \SETunionover A$ (by 2, 3).
\end{proof}
\end{smodule}
\end{document}
