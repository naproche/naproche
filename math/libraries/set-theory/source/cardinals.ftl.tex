\documentclass{naproche-library}
\libinput{preamble}
\begin{document}
\IfInputref{}{\setsectionlevel{section}}
\begin{smodule}[title=Cardinal Numbers]{cardinals.ftl}
\importmodule[libraries/set-theory]{axioms?choice.ftl}
\importmodule[libraries/set-theory]{ordering.ftl}
\importmodule[libraries/foundations]{equinumerosity.ftl}
\symdef{SETcard}[args=1]{\comp|#1\comp|}

\begin{definition}[forthel,id=SET_THEORY_06_8286266038681600]
  Let $x$ be a set.
  $\emph{\SETcard{x}}$ is the ordinal $\kappa$ such that $\kappa$ is equinumerous to $x$ and every ordinal that is equinumerous to $x$ is greater than or equal to $\kappa$.

  Let the \emph{cardinality of $x$} stand for $\SETcard{x}$.
\end{definition}

\inlineforthel{[synonym number/-s][synonym cardinal/-s]}

\begin{definition}[forthel,id=SET_THEORY_06_6818986081648640]
  A \emph{cardinal number} is an ordinal $\kappa$ such that $\kappa \eq \SETcard{x}$ for some
  set $x$.

  Let a \emph{cardinal} stand for a cardinal number.
\end{definition}

\begin{proposition}[forthel,id=SET_THEORY_06_2820082336006144]
  Let $\kappa$ be a cardinal.
  Then $\SETcard{\kappa} \eq \kappa$.
\end{proposition}
\begin{proof}[forthel]
  $\kappa$ is an ordinal that is equinumerous to $\kappa$.
  Hence $\SETcard{\kappa} \ORDleq \kappa$.
  Consider a set $x$ such that $\kappa \eq \SETcard{x}$.
  Then $\SETcard{\kappa}$ is an ordinal that is equinumerous to $x$.
  Hence $\kappa \ORDleq \SETcard{\kappa}$.
  Thus $\SETcard{\kappa} \eq \kappa$.
\end{proof}

\begin{proposition}[forthel,id=SET_THEORY_06_6920913721229312]
  Let $x, y$ be sets.
  Then $x$ and $y$ are equinumerous iff $\SETcard{x} \eq \SETcard{y}$.
\end{proposition}
\begin{proof}[forthel]
  \begin{case}{$x$ and $y$ are equinumerous.}
    Take a bijection $f$ between $x$ and $y$.
    Consider a bijection $g$ between $y$ and $\SETcard{y}$.
    Then $g \FUNcomp f$ is a bijection between $x$ and $\SETcard{y}$ (by \printref{FOUNDATIONS_08_6435206693126144}).
    Hence $x$ and $\SETcard{y}$ are equinumerous.
    Thus $\SETcard{y} \ORDgeq \SETcard{x}$.

    $\FUNinv{f}$ is a bijection between $y$ and $x$.
    Consider a bijection $h$ between $x$ and $\SETcard{x}$.
    Then $h \FUNcomp \FUNinv{f}$ is a bijection between $y$ and $\SETcard{x}$ (by \printref{FOUNDATIONS_08_6435206693126144}).
    Hence $y$ and $\SETcard{x}$ are equinumerous.
    Thus $\SETcard{x} \ORDgeq \SETcard{y}$.

    Therefore $\SETcard{x} \eq \SETcard{y}$.
  \end{case}

  \begin{case}{$\SETcard{x} \eq \SETcard{y}$.}
    Consider a bijection $f$ between $x$ and $\SETcard{x}$ and a bijection $g$
    between $\SETcard{y}$ and $y$.
    Then $g \FUNcomp f$ is a bijection between $x$ and $y$.
    Hence $x$ and $y$ are equinumerous.
  \end{case}
\end{proof}
\end{smodule}
\end{document}
