\documentclass{naproche-library}
\libinput{set-theory-preamble}
\begin{document}
\IfInputref{}{\setsectionlevel{section}}
\begin{smodule}[title=$\omega$ is a Cardinal Number]{omega-is-cardinal.ftl}
\importmodule[libraries/set-theory]{natural-numbers-are-cardinals.ftl}

\begin{proposition}[forthel,id=SET_THEORY_07_4952029518626816]
  $|\omega| = \omega$.
\end{proposition}
\begin{proof}[forthel]
  We have $|\omega| \leq \omega$.

  Let us show that $|\omega|$ is not less than $\omega$.
    Assume the contrary.
    Then $|\omega| \in \omega$.
    Take $n = |\omega|$ and a bijection $f$ between $n$ and $\omega$.

    Define \[ g(k) =
      \begin{cases}
        \succ(f(k)) & : k < n
        \\
        0           & : k = n
      \end{cases} \]
    for $k \in \succ(n)$.
    Then $g$ is a map from $\succ(n)$ to $\omega$.
    Indeed we can show that $g(k) \in \omega$ for all $k \in \succ(n)$.
      Let $k \in \succ(n)$.
      Case $k < n$. Obvious.
      Case $k = n$. Obvious.
    End.

    $g$ is injective.
    Indeed we can show that for all $k, k' \in \succ(n)$ if $k \neq k'$
    then $g(k) \neq g(k')$. \newline
    Proof.
      Let $k, k' \in \succ(n)$.
      Assume $k \neq k'$.

      Case $k, k' < n$.
        Then $f(k) \neq f(k')$.
        Hence $\succ(f(k)) \neq \succ(f(k'))$.
        Thus $g(k) \neq g(k')$.
      End.

      Case $k < n$ and $k' = n$.
        We have $\succ(f(k)) \neq 0$.
        Hence $g(k) \neq g(k')$.
      End.

      Case $k = n$ and $k' < n$.
        We have $\succ(f(k')) \neq 0$.
        Hence $g(k) \neq g(k')$.
      End.
    Qed.

    $g$ is surjective onto $\omega$.
    Indeed we can show that for any $m \in \omega$ there exists a $k \in \succ(n)$ such that $m = g(k)$. \newline
    Proof.
      Let $m \in \omega$.
      Then $f^{-1}(m) \in n$.

      Case $m = 0$.
        Then $m = g(n)$.
      End.

      Case $m \neq 0$.
        Take $m' \in \omega$ such that $m = \succ(m')$.
        Then $m
          = \succ(m')
          = \succ(f(f^{-1}(m')))
          = g(f^{-1}(m'))$.
        Indeed $f(f^{-1}(m')) = m'$ and $f^{-1}(m') < n$.
      End.
    End.

    Hence $g$ is a bijection between $\succ(n)$ and $\omega$.
    Then we have $n
      = |n|
      = |\succ(n)|
      = \succ(n)$.
    Contradiction.
  End.
\end{proof}

\begin{corollary}[forthel,id=SET_THEORY_07_2717623053713408]
  $\omega$ is a cardinal.
\end{corollary}
\end{smodule}
\end{document}
