\documentclass{naproche-library}
\libinput{preamble}
\begin{document}
\IfInputref{}{\setsectionlevel{section}}
\begin{smodule}[title=Successor Cardinals]{successor-cardinals.ftl}
\importmodule[libraries/set-theory]{cardinals.ftl}
\importmodule[libraries/set-theory]{cantor.ftl}

\begin{lemma}[forthel,id=SET_THEORY_06_489264832709174]
  For every ordinal $\alpha$ there exists a cardinal greater than $\alpha$.
\end{lemma}
\begin{proof}[forthel]
  Let $\alpha$ be an ordinal.
  Take $\kappa = |\pow(\alpha)|$.
  Then $\kappa > |\alpha|$.

  Let us show that $\kappa > \alpha$.
    Assume the contrary.
    Then $|\pow(\alpha)|
      = \kappa
      \leq \alpha$.
    Hence $\kappa
      = |\pow(\alpha)|
      = ||\pow(\alpha)||
      \leq |\alpha|$.
    Contradiction.
  End.
\end{proof}

\begin{definition}[forthel,id=SET_THEORY_06_9568425123021254]
  Let $\kappa$ be a cardinal.
  $\kappa^+$ is the cardinal such that $\kappa < \kappa^+$ and there is no cardinal $\nu$ such that $\kappa < \nu < \kappa^+$.
\end{definition}

\begin{definition}[forthel,id=SET_THEORY_06_6818986081648640]
  A successor cardinal is a cardinal number $\kappa$ such that $\kappa = \nu^+$ for some cardinal number $\nu$.
\end{definition}

\begin{proposition}[forthel,id=SET_THEORY_06_5231202126545218]
  Let $\kappa$ be a cardinal.
  Then $|\alpha| \leq \kappa$ for every $\alpha \in \kappa^+$.
\end{proposition}
\end{smodule}
\end{document}
