\documentclass{naproche-library}
\libinput{preamble}
\begin{document}
\IfInputref{}{\setsectionlevel{section}}
\begin{smodule}[title=Successor Ordinals]{successor-ordinals.ftl}
\importmodule[libraries/set-theory]{ordinals.ftl}
\symdef{ORDsucc}{\textrm{succ}}
\symdef{ORDpred}{\textrm{pred}}

\begin{definition}[forthel,id=SET_THEORY_02_8166925802668032]
  Let $\alpha$ be an ordinal.
  $\emph{\ORDsucc(\alpha)} = \alpha \SETunion \SETsingleton{\alpha}$.
\end{definition}

\begin{proposition}[forthel,id=SET_THEORY_02_1624410224066560]
  Let $\alpha$ be an ordinal.
  Then $\ORDsucc(\alpha)$ is an ordinal.
\end{proposition}
\begin{proof}[forthel]
  (1) $\ORDsucc(\alpha)$ is transitive. \newline
  Proof.
    Let $x \in \ORDsucc(\alpha)$ and $y \in x$.
    Then $x \in \alpha$ or $x \eq \alpha$.
    Hence $y \in \alpha$.
    Thus $y \in \ORDsucc(\alpha)$.
  Qed.

  (2) Every element of $\ORDsucc(\alpha)$ is transitive. \newline
  Proof.
    Let $x \in \ORDsucc(\alpha)$.
    Then $x \in \alpha$ or $x \eq \alpha$.
    Hence $x$ is transitive.
    Indeed $\alpha$ is transitive and every element of $\alpha$ is transitive.
  Qed.
\end{proof}

\begin{definition}[forthel,id=SET_THEORY_02_7129712109289472]
  A \emph{successor ordinal} is an ordinal $\alpha$ such that $\alpha \eq \ORDsucc(\beta)$ for some ordinal $\beta$.
\end{definition}

\begin{proposition}[forthel,id=SET_THEORY_02_8651096763400192]
  Let $\alpha, \beta$ be ordinals.
  If $\ORDsucc(\alpha) \eq \ORDsucc(\beta)$ then $\alpha \eq \beta$.
\end{proposition}
\begin{proof}[forthel]
  Assume $\ORDsucc(\alpha) \eq \ORDsucc(\beta)$.

  (1) $\alpha \SETinclude \beta$. \newline
  Proof.
    Let $\gamma \in \alpha$.
    Then $\gamma \in \alpha \SETunion \SETsingleton{\alpha}
      \eq \ORDsucc(\alpha)
      \eq \ORDsucc(\beta)
      \eq \beta \SETunion \SETsingleton{\beta}$.
    Hence $\gamma \in \beta$ or $\gamma \eq \beta$.
    Assume $\gamma \eq \beta$.
    Then $\beta \in \alpha$.
    Hence $\beta
      \eq (\beta \SETunion \SETsingleton{\beta}) \SETdiff \SETsingleton{\gamma}
      \eq (\alpha \SETunion \SETsingleton{\alpha}) \SETdiff \SETsingleton{\gamma}
      \eq (\alpha \SETdiff \SETsingleton{\gamma}) \SETunion \SETsingleton{\alpha}$.
    Therefore $\alpha \in \beta$.
    Consequently $\alpha \in \beta \in \alpha$.
    Contradiction.
  Qed.

  (2) $\beta \SETinclude \alpha$. \newline
  Proof.
    Let $\gamma \in \beta$.
    Then $\gamma \in \beta \SETunion \SETsingleton{\beta}
      \eq \ORDsucc(\beta)
      \eq \ORDsucc(\alpha)
      \eq \alpha \SETunion \SETsingleton{\alpha}$.
    Hence $\gamma \in \alpha$ or $\gamma \eq \alpha$.
    Assume $\gamma \eq \alpha$.
    Then $\alpha \in \beta$.
    Hence $\alpha
      \eq (\alpha \SETunion \SETsingleton{\alpha}) \SETdiff \SETsingleton{\gamma}
      \eq (\beta \SETunion \SETsingleton{\beta}) \SETdiff \SETsingleton{\gamma}
      \eq (\beta \SETdiff \SETsingleton{\gamma}) \SETunion \SETsingleton{\beta}$.
    Therefore $\beta \in \alpha$.
    Consequently $\beta \in \alpha \in \beta$.
    Contradiction.
  Qed.
\end{proof}

\begin{definition}[forthel,id=SET_THEORY_02_735071524880384]
  Let $\alpha$ be a successor ordinal.
  $\emph{\ORDpred(\alpha)}$ is the ordinal $\beta$ such that $\alpha \eq \ORDsucc(\beta)$.
\end{definition}
\end{smodule}
\end{document}
