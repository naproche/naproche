\documentclass{stex}
\libusepackage{naproche}
\libinput{preamble}
\begin{document}
\begin{smodule}{inverse-of-restr.ftl}

\importmodule[libraries/classes]{definitions?subclasses.ftl}
\importmodule[libraries/maps]{definitions?surjections-between-classes.ftl}
\importmodule[libraries/maps]{definitions?inverse-map.ftl}
\importmodule[libraries/maps]{definitions?restr.ftl}
\importmodule[libraries/maps]{definitions?im.ftl}
\importmodule[libraries/maps]{definitions?values.ftl}
\importmodule[libraries/maps]{definitions?comp.ftl}
\importmodule[libraries/maps]{definitions?id.ftl}

\begin{proposition}[forthel,id=InvertibilityOfRestrProp]
  Let $A, B$ be classes and $f \fromonto{A}{B}$ and $X \subclass A$.
  Assume that $f$ is invertible.
  Then $f \rest X$ is invertible and $\inv{f\rest X} \eq \inv{f} \rest (\im{f}{X})$.
\end{proposition}
\begin{proof}[forthel]
  $f \rest X$ is a surjective map from $X$ onto $\im{f}{X}$.
  Take $g = \inv{f} \rest (\im{f}{X})$.
  Then $g$ is a map of $\im{f}{X}$.

  Let us show that $X \subclass \range(g)$.
    Let $a \elem X$.
    Then $f(a) \elem \im{f}{X}$.
    Hence $g(f(a)) \eq \inv{f}(f(a)) \eq a$.
    Thus $a$ is a value of $g$.
  End.

  Let us show that $\range(g) \subclass X$.
    Let $a \elem \range(g)$.
    Take $b \elem \im{f}{X}$ such that $a \eq g(b)$.
    Take $c \elem X$ such that $b \eq f(c)$.
    Then $a
      \eq (\inv{f} \rest (\im{f}{X}))(b)
      \eq \inv{f}(b)
      \eq \inv{f}(f(c))
      \eq c$.
    Hence $a \elem X$.
  End.

  Hence $\range(g) \eq X$.
  Thus $g$ is a surjective map onto $X$.

  Let us show that $g((f \rest X)(a)) \eq a$ for all $a \elem X$.
    Let $a \elem X$.
    Then $g((f \rest X)(a))
      \eq g(f(a))
      \eq (\inv{f} \rest (\im{f}{X}))(f(a))
      \eq \inv{f}(f(a))
      \eq a$.
  End.

  Let us show that $((f \rest X)(g(b))) \eq b$ for all $b \elem \im{f}{X}$.
    Let $b \elem \im{f}{X}$.
    Take $a \elem X$ such that $b \eq f(a)$.
    We have $g(b)
      \eq g(f(a))
      \eq (\inv{f} \rest (\im{f}{X}))(f(a))
      \eq \inv{f}(f(a))
      \eq a$.
    Hence $(f \rest X)(g(b))
      \eq (f \rest X)(a)
      \eq f(a)
      \eq b$.
  End.

  Thus $g \compo (f \rest X) \eq \id{X}$ and $(f \rest X) \compo g \eq \id{\im{f}{X}}$.
  Therefore $g$ is the inverse of $f \rest X$.
\end{proof}

\end{smodule}
\end{document}