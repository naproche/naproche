\documentclass{naproche-library}
\libinput{preamble}
\begin{document}
\begin{smodule}[title=The Inverse is selfinverse]{inverse-is-selfinverse.ftl}

\importmodule[libraries/maps]{definitions?surjections-between-classes.ftl}
\importmodule[libraries/maps]{definitions?inverse-map.ftl}
\importmodule[libraries/maps]{definitions?comp.ftl}
\importmodule[libraries/maps]{definitions?id.ftl}

\begin{proposition}[forthel,id=InvIsSelfinverse]
  Let $A, B$ be classes and $f \fromonto{A}{B}$.
  Assume that $f$ is invertible.
  Then $\inv{f}$ is an invertible surjective map from $B$ onto $A$ such that $\inv{(\inv{f})} \eq f$.
\end{proposition}
\begin{proof}[forthel]
  $\inv{f}$ is a map from $B$ to $A$.
  Indeed $\range(f) \eq B$ and $\dom(f) \eq A$.
  $\inv{f}$ is surjective onto $A$.
  Indeed for any $a \elem A$ we have $\inv{f}(f(a)) \eq a$.
  $\inv{f}$ is the inverse of $f$.
  Thus $f \compo \inv{f} \eq \id{B}$ and $\inv{f} \compo f \eq \id{A}$.
  Therefore $f$ is the inverse of $\inv{f}$.
\end{proof}

\end{smodule}
\end{document}