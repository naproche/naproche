\documentclass{naproche-library}
\libinput{preamble}
\begin{document}
\IfInputref{}{\setsectionlevel{section}}
\begin{smodule}[title=Closure Under Finite Intersections]{closure-under-finite-intersections.ftl}
\importmodule[libraries/foundations]{systems-of-sets.ftl}
\importmodule[libraries/foundations]{finite-and-infinite-classes.ftl}

\begin{definition}[forthel,id=FOUNDATIONS_14_4297814324543488]
  Let $X$ be a system of sets.
  $X$ is \emph{closed under finite intersections} iff $\SETintersectover U \in X$ for every nonempty finite subclass $U$ of $X$.
\end{definition}

\begin{proposition}[forthel,id=FOUNDATIONS_17_1405012582334464]
  Let $X$ be a system of sets.
  $X$ is closed under finite intersections iff $U \SETintersect V \in X$ for every $U, V \in X$.
\end{proposition}
\begin{proof}[forthel]
  \begin{case}{$X$ is closed under finite intersections.}
    Let $U, V \in X$.
    Then $\SETpair{U}{V}$ is a nonempty finite subclass of $X$.
    Hence $U \SETintersect V \eq \SETintersectover \SETpair{U}{V} \in X$.
  \end{case}

  \begin{case}{$U \SETintersect V \in X$ for every $U, V \in X$.}
    Define $\Phi = \{ n \in \Nat \mid \SETintersectover U \in X$ for every nonempty subclass $U$ of $X$ that has $n$ elements $\}$.

    (1) $\Phi$ contains $\NATzero$.

    (2) For every $n \in \Phi$ we have $n \NATplus \NATone \in \Phi$.
    \begin{proof}
      Let $n \in \Phi$.
      Then $\SETintersectover U \in X$ for every nonempty subclass $U$ of $X$ that has $n$ elements.

      Let us show that $\SETintersectover U \in X$ for every nonempty subclass $U$ of $X$ that has $n \NATplus \NATone$ elements.

        \begin{case}{$n \eq \NATzero$.} \end{case}

        \begin{case}{$n \neq \NATzero$.}
          Let $U$ be a nonempty subclass of $X$ such that $U$ has $n \NATplus \NATone$ elements.
          Take a bijection $f$ between $\NATsegment{\NATone}{n \NATplus \NATone}$ and $U$.
          We have $\NATsegment{\NATone}{n \NATplus \NATone} \eq \NATsegment{\NATone}{n} \SETunion \SETsingleton{n \NATplus \NATone}$.
          Take $V \eq \FUNim{f}{\NATsegment{\NATone}{n}}$.
          We have $\NATsegment{\NATone}{n} \SETinclude \NATsegment{\NATone}{n \NATplus \NATone}$.

          Let us show that $V \SETinclude U$.
            Let $x \in V$.
            Take $k \in \NATsegment{\NATone}{n}$ such that $x \eq f(k)$.
            Indeed we can show that there exists a $k \in \NATsegment{\NATone}{n}$ such that $x \eq f(k)$.
              Assume the contrary.
              Then $x \neq f(k)$ for all $k \in \NATsegment{\NATone}{n}$.
              Hence $x \notin \FUNim{f}{\NATsegment{\NATone}{n}} \eq V$.
              Contradiction.
            End.
            Hence $x \in U$.
            Indeed $x \in \FUNim{f}{\NATsegment{\NATone}{n \NATplus \NATone}}$.
          End.

          $V$ is nonempty.
          Indeed $f(\NATone) \in \FUNim{f}{\NATsegment{\NATone}{n}}$.
          Indeed $\NATone \in \NATsegment{\NATone}{n}$.
          Hence $f \FUNrest \NATsegment{\NATone}{n}$ is a bijection between $\NATsegment{\NATone}{n}$ and $V$ (by \cref{FOUNDATIONS_08_647446231252992}).
          Thus $V$ has $n$ elements.
          Consequently $\SETintersectover V \in X$.

          Let us show that $U \eq V \SETunion \SETsingleton{f(n \NATplus \NATone)}$.

            (1) $\FUNim{f}{A \SETunion B} \eq \FUNim{f}{A} \SETunion \FUNim{f}{B}$ for all $A, B \SETinclude \dom(f)$.

            (2) $\FUNim{f}{\SETsingleton{a}} \eq \SETsingleton{f(a)}$ for all $a \in \dom(f)$.

            Hence $U
              \eq f[\dom(f)]
              \eq f[\NATsegment{\NATone}{n \NATplus \NATone}]
              \eq f[\NATsegment{\NATone}{n} \SETunion \SETsingleton{n \NATplus \NATone}]
              \eq f[\NATsegment{\NATone}{n}] \SETunion f[\SETsingleton{n \NATplus \NATone}]
              \eq f[\NATsegment{\NATone}{n}] \SETunion \SETsingleton{f(n \NATplus \NATone)}
              \eq V \SETunion \SETsingleton{f(n \NATplus \NATone)}$.
            Indeed $n \NATplus \NATone \in \dom(f)$ and $\NATsegment{\NATone}{n}, \SETsingleton{\NATone \NATplus n} \SETinclude \dom(f)$.
          End.

          Let us show that $\SETintersectover (A \SETunion B) \eq (\SETintersectover A) \SETintersect (\SETintersectover B)$ for any nonempty systems of sets $A, B$.
            Let $A, B$ be nonempty systems of sets.
            $\SETintersectover (A \SETunion B) \SETinclude (\SETintersectover A) \SETintersect (\SETintersectover B)$.
            $((\SETintersectover A) \SETintersect (\SETintersectover B)) \SETinclude \SETintersectover (A \SETunion B)$. %!
          End.

          Let us show that $f(n \NATplus \NATone)$ and $f(k)$ are sets for every $k \in \NATsegment{\NATone}{n}$.
            Let $k \in \NATsegment{\NATone}{n}$.
            Then $f(k), f(n \NATplus \NATone) \in U$.
            Hence $f(k)$ and $f(n \NATplus \NATone)$ are sets.
          End.

          Hence $V$ and $\SETsingleton{f(n \NATplus \NATone)}$ are nonempty systems of sets.
          Thus $\SETintersectover U
            \eq \SETintersectover (V \SETunion \SETsingleton{f(n \NATplus \NATone)})
            \eq (\SETintersectover V) \SETintersect (\SETintersectover \SETsingleton{f(n \NATplus \NATone)})
            \eq (\SETintersectover V) \SETintersect f(n \NATplus \NATone)
            \in X$.
        \end{case}
      End.
    \end{proof}

    Therefore $\Phi$ contains every natural number (by \printref{ARITHMETIC_01_4764664342773760}).
    Thus $\SETintersectover U \in X$ for every nonempty finite subclass $U$ of $X$.
    Consequently $X$ is closed under finite intersections.
  \end{case}
\end{proof}
\end{smodule}
\end{document}
