\documentclass{naproche-library}
\libinput{preamble}
\begin{document}
\IfInputref{}{\setsectionlevel{section}}
\begin{smodule}[title=Classes]{classes.ftl}
\importmodule[meta-inf]{vocabulary.ftl}
\importmodule[meta-inf]{macros.ftl}
\symdef{SETinclude}{\;\subseteq\;}
\notation{SETinclude}{\;\subset\;}
\symdef{SETstrictinclude}{\;\subsetneq\;}
\symdef{SETniclude}{\;\supseteq\;}
\notation{SETniclude}{\;\supset\;}
\symdef{SETstrictniclude}{\;\supsetneq\;}
\symdef{SETempty}{\emptyset}
\symdef{SETpair}[args=2]{\comp{\{}#1\comp,#2\comp{\}}}
\symdef{SETsingleton}[args=1]{\comp{\{}#1\comp{\}}}
\symdef{SETunion}{\,\cup\,}
\symdef{SETintersect}{\,\cap\,}
\symdef{SETdiff}{\,\setminus\,}

\begin{sfragment}{Sub- and Superclasses}
  \begin{definition}[forthel,id=FOUNDATIONS_01_3275578358628352]
    Let $A$ be a class.
    A \emph{subclass of $A$} is a class $B$ such that every element of $B$ is an
    element of $A$.

    Let $\emph{B \SETinclude A}$ stand for $B$ is a subclass of $A$.

    Let a \emph{superclass of $B$} stand for a class $A$ such that $B \SETinclude A$.
    Let $\emph{B \SETniclude A}$ stand for $B$ is a superclass of $A$.

    Let a \emph{proper subclass of $A$} stand for a subclass $B$ of $A$ such that $B \neq A$.
    Let $\emph{B \SETstrictinclude A}$ stand for $B$ is a proper subclass of $A$.

    Let a \emph{proper superclass of $B$} stand for a superclass $A$ of $B$ such that $A \neq B$.
    Let $\emph{B \SETstrictniclude A}$ stand for $B$ is a proper superclass of $A$.

    Let \emph{$A$ includes $B$} stand for $B \SETinclude A$.
    Let $B$ is \emph{included in $A$} stand for $B \SETinclude A$.
  \end{definition}

  \begin{proposition}[forthel,id=FOUNDATIONS_01_5994555614691328]
    Let $A$ be a class.
    Then $A \SETinclude A$.
  \end{proposition}
  \begin{proof}[forthel]
    Every element of $A$ is contained in $A$.
    Therefore $A \SETinclude A$.
  \end{proof}

  \begin{proposition}[forthel,id=FOUNDATIONS_01_3939677545431040]
    Let $A, B, C$ be classes.
    If $A \SETinclude B$ and $B \SETinclude C$ then $A \SETinclude C$.
  \end{proposition}
  \begin{proof}[forthel]
    Assume $A \SETinclude B$ and $B \SETinclude C$.
    Then every element of $A$ is contained in $B$ and every element of $B$ is contained in $C$.
    Hence every element of $A$ is contained in $C$.
    Thus $A \SETinclude C$.
  \end{proof}

  \begin{proposition}[forthel,id=FOUNDATIONS_01_7159957847801856]
    Let $A, B$ be classes.
    If $A \SETinclude B$ and $B \SETinclude A$ then $A \eq B$.
  \end{proposition}
  \begin{proof}[forthel]
    Assume $A \SETinclude B$ and $B \SETinclude A$.
    Then every element of $A$ is contained in $B$ and every element of $B$ is contained in $A$.
    Hence $A \eq B$.
  \end{proof}
\end{sfragment}

\begin{sfragment}{The Empty Class}
  \begin{definition}[forthel,id=FOUNDATIONS_01_6252477624090624]
    Let $A$ be a class.
    $A$ is \emph{empty} iff $A$ has no elements.

    Let $A$ is \emph{nonempty} stand for $A$ is not empty.
  \end{definition}

  \begin{definition}[forthel,id=FOUNDATIONS_01_7939928493129728]
    $\emph{\SETempty} = \{ x \mid x \neq x \}$.
  \end{definition}

  \begin{proposition}[forthel,id=FOUNDATIONS_01_2263153161273344]
    Let $A$ be a class.
    $A$ is empty iff $A \eq \SETempty$.
  \end{proposition}
  \begin{proof}[forthel]
    We can show that $\SETempty$ is empty.
    Indeed any element $x$ of $\SETempty$ is nonequal to $x$.
    Hence if $A \eq \SETempty$ then $A$ is empty.
    If $A$ is empty then $A$ and $\SETempty$ have no elements.
    Hence if $A$ is empty then $A \SETinclude \SETempty$ and $\SETempty \SETinclude A$.
    Thus if $A$ is empty then $A \eq \SETempty$.
  \end{proof}

  \begin{corollary}[forthel,id=FOUNDATIONS_01_1495141426659328]
    $\SETempty$ is empty.
  \end{corollary}

  \begin{corollary}[forthel,id=FOUNDATIONS_01_6931785090859008]
    Let $A$ be a class.
    Then $\SETempty \SETinclude A$.
  \end{corollary}
  \begin{proof}[forthel]
    $\SETempty$ has no elements.
    Hence every element of $\SETempty$ is contained in $A$.
  \end{proof}
\end{sfragment}

\begin{sfragment}{Unordered Pairs}
  \begin{definition}[forthel,id=FOUNDATIONS_01_3471035364016128]
    Let $a, b$ be objects.
    The \emph{unordered pair of $a$ and $b$} is $\{ x \mid x \eq a$ or $x \eq b \}$.

    Let $\emph{\SETpair{a}{b}}$ stand for the unordered pair of $a$ and $b$.
  \end{definition}

  \begin{definition}[forthel,id=FOUNDATIONS_01_605432672419840]
    An \emph{unordered pair} is a class $A$ such that $A \eq \SETpair{a}{b}$ for some distinct objects $a, b$.
  \end{definition}

  \begin{definition}[forthel,id=FOUNDATIONS_01_1160414603771904]
    Let $a$ be an object.
    The \emph{singleton class of $a$} is $\{ x \mid x \eq a \}$.

    Let $\emph{\SETsingleton{a}}$ stand for the singleton class of $a$.
  \end{definition}

  \begin{definition}[forthel,id=FOUNDATIONS_01_6786618161627136]
    A \emph{singleton class} is a class $A$ such that $A \eq \SETsingleton{a}$ for some object $a$.
  \end{definition}

  \begin{proposition}[forthel,id=FOUNDATIONS_01_6125259604361216]
    Let $a, a', b, b'$ be objects.
    Assume $\SETpair{a}{b} \eq \SETpair{a'}{b'}$.
    Then ($a \eq a'$ and $b \eq b'$) or ($a \eq b'$ and $b \eq a'$).
  \end{proposition}
  \begin{proof}[forthel]
    We have $a \eq a'$ or $a \eq b'$.
    If $a \eq a'$ then $b \eq b'$.
    If $a \eq b'$ then $b \eq a'$.
    Hence ($a \eq a'$ and $b \eq b'$) or ($a \eq b'$ and $b \eq a'$).
  \end{proof}

  \begin{corollary}[forthel,id=FOUNDATIONS_01_6954678910713856]
    Let $a, a'$ be objects.
    If $\SETsingleton{a} \eq \SETsingleton{a'}$ then $a \eq a'$.
  \end{corollary}

  \begin{definition}[forthel,id=FOUNDATIONS_01_29867480293464351]
    Let $A$ be a class.
    A \emph{unique element of $A$} is an element $a$ of $A$ such that for each $x \in A$ we have $x \eq a$.
  \end{definition}

  \begin{proposition}[forthel,id=FOUNDATIONS_01_0166348647163481]
    Let $A$ be a class.
    Then $A$ has a unique element iff $A \eq \SETsingleton{a}$ for some object $a$.
  \end{proposition}
\end{sfragment}

\begin{sfragment}{Unions, Intersections, Complements}
  \begin{definition}[forthel,id=FOUNDATIONS_01_2159753924968448]
    Let $A, B$ be classes.
    The \emph{union of $A$ and $B$} is $\{ x \mid x \in A$ or $x \in B \}$.

    Let $\emph{A \SETunion B}$ stand for the union of $A$ and $B$.
  \end{definition}

  \begin{definition}[forthel,id=FOUNDATIONS_01_5744033011859456]
    Let $A, B$ be classes.
    The \emph{intersection of $A$ and $B$} is $\{ x \mid x \in A$ and $x \in B \}$.

    Let $\emph{A \SETintersect B}$ stand for the intersection of $A$ and $B$.
  \end{definition}

  \begin{definition}[forthel,id=FOUNDATIONS_01_7620345041256448]
    Let $A, B$ be classes.
    The \emph{complement of $B$ in $A$} is $\{ x \mid x \in A$ and $x \notin B \}$.

    Let $\emph{A \SETdiff B}$ stand for the complement of $B$ in $A$.
  \end{definition}
\end{sfragment}

\begin{sfragment}{Disjoint Classes}
  \begin{definition}[forthel,id=FOUNDATIONS_01_4981913324355584]
    Let $A, B$ be classes.
    $A$ and $B$ are \emph{disjoint} iff $A$ and $B$ have no common elements.
  \end{definition}

  \begin{proposition}[forthel,id=FOUNDATIONS_01_1211191546347520]
    Let $A, B$ be classes.
    Then $A$ and $B$ are disjoint iff $A \SETintersect B$ is empty.
  \end{proposition}
\end{sfragment}
\end{smodule}
\end{document}
