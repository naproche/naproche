\documentclass{naproche-library}
\libinput{preamble}
\begin{document}
\IfInputref{}{\setsectionlevel{section}}
\begin{smodule}[title=Systems of Sets]{systems-of-sets.ftl}
\importmodule[libraries/foundations]{sub-and-supersets.ftl}

\begin{definition}[forthel,id=FOUNDATIONS_10_5805323570905088]
  A system of sets is a class $X$ such that every element of $X$ is a set.
\end{definition}

\begin{definition}[forthel,id=FOUNDATIONS_10_1631952387964928]
  A system of nonempty sets is a class $X$ such that every element of $X$ is a nonempty set.
\end{definition}

\begin{definition}[forthel,id=FOUNDATIONS_10_943381479948288]
  Let $A$ be a class.
  A system of subsets of $A$ is a class $X$ such that every element of $X$ is a subset of $A$.
\end{definition}

\begin{proposition}[forthel,id=FOUNDATIONS_10_8268633648136192]
  Let $A$ be a class.
  Then $\emptyset$ is a system of subsets of $A$.
\end{proposition}

\begin{proposition}[forthel,id=FOUNDATIONS_10_7546016869908480]
  Let $A$ be a class.
  Then $\pow(A)$ is a system of subsets of $A$.
\end{proposition}

\begin{proposition}
  Let $X, Y$ be systems of sets.
  Then $X \cup Y$ is a system of sets.
\end{proposition}

\begin{proposition}
  Let $X, Y$ be systems of sets.
  Then $X \cap Y$ is a system of sets.
\end{proposition}

\begin{proposition}
  Let $X, Y$ be systems of sets.
  Then $X \setminus Y$ is a system of sets.
\end{proposition}

\begin{sfragment}{Unions Over Systems of Sets}
  \begin{definition}[forthel,id=FOUNDATIONS_10_541772562300928]
    Let $X$ be a system of sets.
    The union over $X$ is $\{ a \mid a \in x$ for some $x \in X \}$.

    Let $\bigcup X$ stand for the union over $X$.
  \end{definition}

  \begin{proposition}[forthel,id=FOUNDATIONS_10_4872701241982976]
    $\bigcup \emptyset = \emptyset$.
  \end{proposition}
  \begin{proof}[forthel]
    $\bigcup \emptyset = \{ a \mid a \in x$ for some $x \in \emptyset \}$.
    $\emptyset$ has no elements.
    Hence there is no object $a$ such that $a \in x$ for some $x \in \emptyset$.
    Thus $\bigcup \emptyset = \emptyset$.
  \end{proof}

  \begin{proposition}[forthel,id=FOUNDATIONS_10_2559541585641472]
    Let $x, y$ be sets.
    Then $\bigcup \set{x, y} = x \cup y$.
  \end{proposition}
  \begin{proof}[forthel]
    Let us show that $\bigcup \set{x, y} \subseteq x \cup y$.
      Let $a \in \bigcup \set{x, y}$.
      Then $a$ is contained in some element of $\set{x, y}$.
      Hence $a \in x$ or $a \in y$.
      Thus $a \in x \cup y$.
    End.

    Let us show that $x \cup y \subseteq \bigcup \set{x, y}$.
      Let $a \in x \cup y$.
      Then $a \in x$ or $a \in y$.
      Hence $a$ is contained in some element of $\set{x, y}$.
      Therefore $a \in \bigcup \set{x, y}$.
    End.
  \end{proof}

  \begin{corollary}[forthel,id=FOUNDATIONS_10_2157223832715264]
    Let $x$ be a set.
    Then $\bigcup \set{x} = x$.
  \end{corollary}
\end{sfragment}

\begin{sfragment}{Intersections Over Systems of Sets}
  \begin{definition}[forthel,id=FOUNDATIONS_10_2659345095458816]
    Let $X$ be a system of sets.
    The intersection over $X$ is $\{ a \mid a \in x$ for all $x \in X \}$.

    Let $\bigcap X$ stand for the intersection over $X$.
  \end{definition}

  \begin{proposition}[forthel,id=FOUNDATIONS_10_2809770322952192]
    $\bigcap \emptyset$ is the class of all objects.
  \end{proposition}
  \begin{proof}[forthel]
    Define $V = \{ x \mid x$ is an object $\}$.
    We have $\bigcap \emptyset \subseteq V$.
    Indeed every element of $\bigcap \emptyset$ is an object.

    Let us show that $V \subseteq \bigcap \emptyset$.
      Let $a \in V$.
      Then $a$ is an object.
      For every $x \in \emptyset$ we have $a \in x$.
      Indeed $\emptyset$ has no elements.
      Thus $a \in \bigcap \emptyset$.
    End.
  \end{proof}

  \begin{proposition}[forthel,id=FOUNDATIONS_10_7851827447988224]
    Let $x, y$ be sets.
    Then $\bigcap \set{x, y} = x \cap y$.
  \end{proposition}
  \begin{proof}[forthel]
    Let us show that $\bigcap \set{x, y} \subseteq x \cap y$.
      Let $a \in \bigcap \set{x, y}$.
      Then $a$ is contained in every element of $\set{x, y}$.
      Hence $a \in x$ and $a \in y$.
      Thus $a \in x \cap y$.
    End.

    Let us show that $x \cap y \subseteq \bigcap \set{x, y}$.
      Let $a \in x \cap y$.
      Then $a \in x$ and $a \in y$.
      Hence $a$ is contained in every element of $\set{x, y}$.
      Therefore $a \in \bigcap \set{x, y}$.
    End.
  \end{proof}

  \begin{corollary}[forthel,id=FOUNDATIONS_10_7239895674257408]
    Let $x$ be a set.
    Then $\bigcap \set{x} = x$.
  \end{corollary}
\end{sfragment}
\end{smodule}
\end{document}
