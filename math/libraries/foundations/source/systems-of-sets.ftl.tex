\documentclass{naproche-library}
\libinput{preamble}
\begin{document}
\IfInputref{}{\setsectionlevel{section}}
\begin{smodule}[title=Systems of Sets]{systems-of-sets.ftl}
\importmodule[libraries/foundations]{sub-and-supersets.ftl}
\symdef{SETunionover}{\bigcup\,}
\symdef{SETintersectover}{\bigcap\,}

\inlineforthel{[synonym system/-s]}

\begin{definition}[forthel,id=FOUNDATIONS_10_5805323570905088]
  A \emph{system of sets} is a class $X$ such that every element of $X$ is a set.
\end{definition}

\begin{definition}[forthel,id=FOUNDATIONS_10_1631952387964928]
  A \emph{system of nonempty sets} is a class $X$ such that every element of $X$ is a nonempty set.
\end{definition}

\begin{definition}[forthel,id=FOUNDATIONS_10_943381479948288]
  Let $A$ be a class.
  A \emph{system of subsets of $A$} is a class $X$ such that every element of $X$ is a subset of $A$.
\end{definition}

\begin{proposition}[forthel,id=FOUNDATIONS_10_8268633648136192]
  Let $A$ be a class.
  Then $\SETempty$ is a system of subsets of $A$.
\end{proposition}

\begin{proposition}[forthel,id=FOUNDATIONS_10_7546016869908480]
  Let $A$ be a class.
  Then $\SETpow(A)$ is a system of subsets of $A$.
\end{proposition}

\begin{proposition}[forthel,id=FOUNDATIONS_10_2794857209384572]
  Let $X, Y$ be systems of sets.
  Then $X \SETunion Y$ is a system of sets.
\end{proposition}

\begin{proposition}[forthel,id=FOUNDATIONS_10_1983754920483934]
  Let $X, Y$ be systems of sets.
  Then $X \SETintersect Y$ is a system of sets.
\end{proposition}

\begin{proposition}[forthel,id=FOUNDATIONS_10_0928495478392003]
  Let $X, Y$ be systems of sets.
  Then $X \SETdiff Y$ is a system of sets.
\end{proposition}

\begin{sfragment}{Unions Over Systems of Sets}
  \begin{definition}[forthel,id=FOUNDATIONS_10_541772562300928]
    Let $X$ be a system of sets.
    $\emph{\SETunionover X} = \{ a \mid a \SETin x\text{ for some }x \SETin X \}$.

    Let \emph{union over $X$} stand for $\SETunionover X$.
  \end{definition}

  \begin{proposition}[forthel,id=FOUNDATIONS_10_4872701241982976]
    $\SETunionover \SETempty \eq \SETempty$.
  \end{proposition}
  \begin{proof}[forthel]
    $\SETunionover \SETempty = \{ a \mid a \SETin x\text{ for some }x \SETin \SETempty \}$.
    $\SETempty$ has no elements.
    Hence there is no object $a$ such that $a \SETin x$ for some $x \SETin \SETempty$.
    Thus $\SETunionover \SETempty \eq \SETempty$.
  \end{proof}

  \begin{proposition}[forthel,id=FOUNDATIONS_10_2559541585641472]
    Let $x, y$ be sets.
    Then $\SETunionover \SETpair{x}{y} \eq x \SETunion y$.
  \end{proposition}
  \begin{proof}[forthel]
    Let us show that $\SETunionover \SETpair{x}{y} \SETinclude x \SETunion y$.
      Let $a \SETin \SETunionover \SETpair{x}{y}$.
      Then $a$ is contained in some element of $\SETpair{x}{y}$.
      Hence $a \SETin x$ or $a \SETin y$.
      Thus $a \SETin x \SETunion y$.
    End.

    Let us show that $x \SETunion y \SETinclude \SETunionover \SETpair{x}{y}$.
      Let $a \SETin x \SETunion y$.
      Then $a \SETin x$ or $a \SETin y$.
      Hence $a$ is contained in some element of $\SETpair{x}{y}$.
      Therefore $a \SETin \SETunionover \SETpair{x}{y}$.
    End.
  \end{proof}

  \begin{corollary}[forthel,id=FOUNDATIONS_10_2157223832715264]
    Let $x$ be a set.
    Then $\SETunionover \SETsingleton{x} \eq x$.
  \end{corollary}
\end{sfragment}

\begin{sfragment}{Intersections Over Systems of Sets}
  \begin{definition}[forthel,id=FOUNDATIONS_10_2659345095458816]
    Let $X$ be a system of sets.
    $\emph{\SETintersectover X} = \{ a \mid a \SETin x\text{ for all }x \SETin X \}$.

    Let the \emph{intersection over $X$} stand for $\SETintersectover X$.
  \end{definition}

  \begin{proposition}[forthel,id=FOUNDATIONS_10_2809770322952192]
    $\SETintersectover \SETempty$ is the class of all objects.
  \end{proposition}
  \begin{proof}[forthel]
    Define $V = \{ x \mid x\text{ is an object}\}$.
    We have $\SETintersectover \SETempty \SETinclude V$.
    Indeed every element of $\SETintersectover \SETempty$ is an object.

    Let us show that $V \SETinclude \SETintersectover \SETempty$.
      Let $a \SETin V$.
      Then $a$ is an object.
      For every $x \SETin \SETempty$ we have $a \SETin x$.
      Indeed $\SETempty$ has no elements.
      Thus $a \SETin \SETintersectover \SETempty$.
    End.
  \end{proof}

  \begin{proposition}[forthel,id=FOUNDATIONS_10_7851827447988224]
    Let $x, y$ be sets.
    Then $\SETintersectover \SETpair{x}{y} \eq x \SETintersect y$.
  \end{proposition}
  \begin{proof}[forthel]
    Let us show that $\SETintersectover \SETpair{x}{y} \SETinclude x \SETintersect y$.
      Let $a \SETin \SETintersectover \SETpair{x}{y}$.
      Then $a$ is contained in every element of $\SETpair{x}{y}$.
      Hence $a \SETin x$ and $a \SETin y$.
      Thus $a \SETin x \SETintersect y$.
    End.

    Let us show that $x \SETintersect y \SETinclude \SETintersectover \SETpair{x}{y}$.
      Let $a \SETin x \SETintersect y$.
      Then $a \SETin x$ and $a \SETin y$.
      Hence $a$ is contained in every element of $\SETpair{x}{y}$.
      Therefore $a \SETin \SETintersectover \SETpair{x}{y}$.
    End.
  \end{proof}

  \begin{corollary}[forthel,id=FOUNDATIONS_10_7239895674257408]
    Let $x$ be a set.
    Then $\SETintersectover \SETsingleton{x} \eq x$.
  \end{corollary}
\end{sfragment}
\end{smodule}
\end{document}
