\documentclass{naproche-library}
\libinput{preamble}
\begin{document}
\IfInputref{}{\setsectionlevel{section}}
\begin{smodule}[title=Computation Laws For Classes]{computation-laws-for-classes.ftl}
\importmodule[libraries/foundations]{classes.ftl}

\begin{sfragment}{Commutativity of Union and Intersection}
  \begin{proposition}[forthel,id=FOUNDATIONS_02_8446177632583680]
    Let $A, B$ be classes.
    Then $A \cup B = B \cup A$.
  \end{proposition}
  \begin{proof}[forthel]
    Let us show that $A \cup B \subseteq B \cup A$.
      Let $x \in A \cup B$.
      Then $x \in A$ or $x \in B$.
      Hence $x \in B$ or $x \in A$.
      Thus $x \in B \cup A$.
    End.

    Let us show that $B \cup A \subseteq A \cup B$.
      Let $x \in B \cup A$.
      Then $x \in B$ or $x \in A$.
      Hence $x \in A$ or $x \in B$.
      Thus $x \in A \cup B$.
    End.
  \end{proof}

  \begin{proposition}[forthel,id=FOUNDATIONS_02_7565102251245568]
    Let $A, B$ be classes.
    Then $A \cap B = B \cap A$.
  \end{proposition}
  \begin{proof}[forthel]
    Let us show that $A \cap B \subseteq B \cap A$.
      Let $x \in A \cap B$.
      Then $x \in A$ and $x \in B$.
      Hence $x \in B$ and $x \in A$.
      Thus $x \in B \cap A$.
    End.

    Let us show that $B \cap A \subseteq A \cap B$.
      Let $x \in B \cap A$.
      Then $x \in B$ and $x \in A$.
      Hence $x \in A$ and $x \in B$.
      Thus $x \in A \cap B$.
    End.
  \end{proof}
\end{sfragment}

\begin{sfragment}{Associativity of Union and Intersection}
  \begin{proposition}[forthel,id=FOUNDATIONS_02_3854032263184384]
    Let $A, B, C$ be classes.
    Then $(A \cup B) \cup C = A \cup (B \cup C)$.
  \end{proposition}
  \begin{proof}[forthel]
    Let us show that $((A \cup B) \cup C) \subseteq A \cup (B \cup C)$. %!
      Let $x \in (A \cup B) \cup C$.
      Then $x \in A \cup B$ or $x \in C$.
      Hence $x \in A$ or $x \in B$ or $x \in C$.
      Thus $x \in A$ or $x \in (B \cup C)$.
      Therefore $x \in A \cup (B \cup C)$.
    End.

    Let us show that $A \cup (B \cup C) \subseteq (A \cup B) \cup C$.
      Let $x \in A \cup (B \cup C)$.
      Then $x \in A$ or $x \in B \cup C$.
      Hence $x \in A$ or $x \in B$ or $x \in C$.
      Thus $x \in A \cup B$ or $x \in C$.
      Therefore $x \in (A \cup B) \cup C$.
    End.
  \end{proof}

  \begin{proposition}[forthel,id=FOUNDATIONS_02_906751977193472]
    Let $A, B, C$ be classes.
    Then $(A \cap B) \cap C = A \cap (B \cap C)$.
  \end{proposition}
  \begin{proof}[forthel]
    Let us show that $((A \cap B) \cap C) \subseteq A \cap (B \cap C)$. %!
      Let $x \in (A \cap B) \cap C$.
      Then $x \in A \cap B$ and $x \in C$.
      Hence $x \in A$ and $x \in B$ and $x \in C$.
      Thus $x \in A$ and $x \in (B \cap C)$.
      Therefore$ x \in A \cap (B \cap C)$.
    End.

    Let us show that $A \cap (B \cap C) \subseteq (A \cap B) \cap C$.
      Let $x \in A \cap (B \cap C)$.
      Then $x \in A$ and $x \in B \cap C$.
      Hence $x \in A$ and $x \in B$ and $x \in C$.
      Thus $x \in A \cap B$ and $x \in C$.
      Therefore $x \in (A \cap B) \cap C$.
    End.
  \end{proof}
\end{sfragment}

\begin{sfragment}{Distributivity of Union and Intersection}
  \begin{proposition}[forthel,id=FOUNDATIONS_02_371139087958016]
    Let $A, B, C$ be classes.
    Then $A \cap (B \cup C) = (A \cap B) \cup (A \cap C)$.
  \end{proposition}
  \begin{proof}[forthel]
    Let us show that $A \cap (B \cup C) \subseteq (A \cap B) \cup (A \cap C)$.
      Let $x \in A \cap (B \cup C)$.
      Then $x \in A$ and $x \in B \cup C$.
      Hence $x \in A$ and ($x \in B$ or $x \in C$).
      Thus ($x \in A$ and $x \in B$) or ($x \in A$ and $x \in C$).
      Therefore $x \in A \cap B$ or $x \in A \cap C$.
      Hence $x \in (A \cap B) \cup (A \cap C)$.
    End.

    Let us show that $((A \cap B) \cup (A \cap C)) \subseteq A \cap (B \cup C)$. %!
      Let $x \in (A \cap B) \cup (A \cap C)$.
      Then $x \in A \cap B$ or $x \in A \cap C$.
      Hence ($x \in A$ and $x \in B$) or ($x \in A$ and $x \in C$).
      Thus $x \in A$ and ($x \in B$ or $x \in C$).
      Therefore $x \in A$ and $x \in B \cup C$.
      Hence$ x \in A \cap (B \cup C)$.
    End.
  \end{proof}

  \begin{proposition}[forthel,id=FOUNDATIONS_02_5937390721957888]
    Let $A, B, C$ be classes.
    Then $A \cup (B \cap C) = (A \cup B) \cap (A \cup C)$.
  \end{proposition}
  \begin{proof}[forthel]
    Let us show that $A \cup (B \cap C) \subseteq (A \cup B) \cap (A \cup C)$.
      Let $x \in A \cup (B \cap C)$.
      Then $x \in A$ or $x \in B \cap C$.
      Hence $x \in A$ or ($x \in B$ and $x \in C$).
      Thus ($x \in A$ or $x \in B$) and ($x \in$ A or $x \in C$).
      Therefore $x \in A \cup B$ and $x \in A \cup C$.
      Hence $x \in (A \cup B) \cap (A \cup C)$.
    End.

    Let us show that $((A \cup B) \cap (A \cup C)) \subseteq A \cup (B \cap C)$. %!
      Let $x \in (A \cup B) \cap (A \cup C)$.
      Then $x \in A \cup B$ and $x \in A \cup C$.
      Hence ($x \in A$ or $x \in B$) and ($x \in A$ or $x \in C$).
      Thus $x \in A$ or ($x \in B$ and $x \in C$).
      Therefore $x \in A$ or $x \in B \cap C$.
      Hence $x \in A \cup (B \cap C)$.
    End.
  \end{proof}
\end{sfragment}

\begin{sfragment}{Idempocy Laws for Union and Intersection}
  \begin{proposition}[forthel,id=FOUNDATIONS_02_2096996737351680]
    Let $A$ be a class.
    Then $A \cup A = A$.
  \end{proposition}
  \begin{proof}[forthel]
    $A \cup A = \{x \mid x \in A$ or $x \in A \}$.
    Hence $A \cup A = \{ x \mid x \in A \}$.
    Thus $A \cup A = A$.
  \end{proof}

  \begin{proposition}[forthel,id=FOUNDATIONS_02_4053144145231872]
    Let $A$ be a class.
    Then $A \cap A = A$.
  \end{proposition}
  \begin{proof}[forthel]
    $A \cap A = \{ x \mid x \in A$ and $x \in A \}$.
    Hence $A \cap A = \{ x \mid x \in A \}$.
    Thus $A \cap A = A$.
  \end{proof}
\end{sfragment}

\begin{sfragment}{Distributivity of Complement}
  \begin{proposition}[forthel,id=FOUNDATIONS_02_5296031436636160]
    Let $A, B, C$ be classes.
    Then $A \setminus (B \cap C) = (A \setminus B) \cup (A \setminus C)$.
  \end{proposition}
  \begin{proof}[forthel]
    Let us show that $A \setminus (B \cap C) \subseteq (A \setminus B) \cup (A \setminus C)$.
      Let $x \in A \setminus (B \cap C)$.
      Then $x \in A$ and $x \notin B \cap C$.
      Hence it is wrong that ($x \in B$ and $x \in C$).
      Thus $x \notin B$ or $x \notin C$.
      Therefore $x \in A$ and ($x \notin B$ or $x \notin C$).
      Then ($x \in A$ and $x \notin B$) or ($x \in A$ and $x \notin C$).
      Hence $x \in A \setminus B$ or $x \in A \setminus C$.
      Thus $x \in (A \setminus B) \cup (A \setminus C)$.
    End.

    Let us show that $((A \setminus B) \cup (A \setminus C)) \subseteq A \setminus (B \cap C)$. %!
      Let $x \in (A \setminus B) \cup (A \setminus C)$.
      Then $x \in A \setminus B$ or $x \in A \setminus C$.
      Hence ($x \in A$ and $x \notin B$) or ($x \in A$ and $x \notin C$).
      Thus $x \in A$ and ($x \notin B$ or $x \notin C$).
      Therefore $x \in A$ and not ($x \in B$ and $x \in C$).
      Then $x \in A$ and not $x \in B \cap C$.
      Hence $x \in A \setminus (B \cap C)$.
    End.
  \end{proof}

  \begin{proposition}[forthel,id=FOUNDATIONS_02_2909554153095168]
    Let $A, B, C$ be classes.
    Then $A \setminus (B \cup C) = (A \setminus B) \cap (A \setminus C)$.
  \end{proposition}
  \begin{proof}[forthel]
    Let us show that $A \setminus (B \cup C) \subseteq (A \setminus B) \cap (A \setminus C)$.
      Let $x \in A \setminus (B \cup C)$.
      Then $x \in A$ and $x \notin B \cup C$.
      Hence it is wrong that ($x \in B$ or $x \in C$).
      Thus $x \notin B$ and $x \notin C$.
      Therefore $x \in A$ and ($x \notin B$ and $x \notin C$).
      Then ($x \in A$ and $x \notin B$) and ($x \in A$ and $x \notin C$).
      Hence $x \in A \setminus B$ and $x \in A \setminus C$.
      Thus $x \in (A \setminus B) \cap (A \setminus C)$.
    End.

    Let us show that $((A \setminus B) \cap (A \setminus C)) \subseteq A \setminus (B \cup C)$. %!
      Let $x \in (A \setminus B) \cap (A \setminus C)$.
      Then $x \in A \setminus B$ and $x \in A \setminus C$.
      Hence ($x \in A$ and $x \notin B$) and ($x \in A$ and $x \notin C$).
      Thus $x \in A$ and ($x \notin B$ and $x \notin C$).
      Therefore $x \in A$ and not ($x \in B$ or $x \in C$).
      Then $x \in A$ and not $x \in B \cup C$.
      Hence $x \in A \setminus (B \cup C)$.
    End.
  \end{proof}
\end{sfragment}

\begin{sfragment}{Subclass Laws}
  \begin{proposition}[forthel,id=FOUNDATIONS_02_3793981508943872]
    Let $A, B$ be classes.
    Then $A \subseteq A \cup B$.
  \end{proposition}
  \begin{proof}[forthel]
    Let $x \in A$.
    Then $x \in A$ or $x \in B$.
    Hence $x \in A \cup B$.
  \end{proof}

  \begin{proposition}[forthel,id=FOUNDATIONS_02_1591517646946304]
    Let $A, B$ be classes.
    Then $A \cap B \subseteq A$.
  \end{proposition}
  \begin{proof}[forthel]
    Let $x \in A \cap B$.
    Then $x \in A$ and $x \in B$.
    Hence $x \in A$.
  \end{proof}

  \begin{proposition}[forthel,id=FOUNDATIONS_02_6657236858306560]
    Let $A, B$ be classes.
    Then $A \subseteq B$ iff $A \cup B = B$.
  \end{proposition}
  \begin{proof}[forthel]
    Case $A \subseteq B$.

      Let us show that $A \cup B \subseteq B$.
        Let $x \in A \cup B$.
        Then $x \in A$ or $x \in B$.
        If $x \in A$ then $x \in B$.
        Hence $x \in B$.
      End.

      Let us show that $B \subseteq A \cup B$.
        Let $x \in B$.
        Then $x \in A$ or $x \in B$.
        Hence $x \in A \cup B$.
      End.
    End.

    Case $A \cup B = B$.
      Let $x \in A$.
      Then $x \in A$ or $x \in B$.
      Hence $x \in A \cup B = B$.
    End.
  \end{proof}

  \begin{proposition}[forthel,id=FOUNDATIONS_02_2356449346846720]
    Let $A, B$ be classes.
    Then $A \subseteq B$ iff $A \cap B = A$.
  \end{proposition}
  \begin{proof}[forthel]
    Case $A \subseteq B$.

      Let us show that $A \cap B \subseteq A$.
        Let $x \in A \cap B$.
        Then $x \in A$ and $x \in B$.
        Hence $x \in A$.
      End.

      Let us show that $A \subseteq A \cap B$.
        Let $x \in A$.
        Then $x \in B$.
        Hence $x \in A$ and $x \in B$.
        Thus $x \in A \cap B$.
      End.
    End.

    Case $A \cap B = A$.
      Let $x \in A$.
      Then $x \in A \cap B$.
      Hence $x \in A$ and $x \in B$.
      Thus $x \in B$.
    End.
  \end{proof}
\end{sfragment}

\begin{sfragment}{Complement Laws}
  \begin{proposition}[forthel,id=FOUNDATIONS_02_7433299337150464]
    Let $A$ be a class.
    Then $A \setminus A = \emptyset$.
  \end{proposition}
  \begin{proof}[forthel]
    $A \setminus A$ has no elements.
    Indeed $A \setminus A = \{ x \mid x \in A$ and $x \notin A \}$.
    Hence the thesis.
  \end{proof}

  \begin{proposition}[forthel,id=FOUNDATIONS_02_3783696985358336]
    Let $A$ be a class.
    Then $A \setminus \emptyset = A$.
  \end{proposition}
  \begin{proof}[forthel]
    $A \setminus \emptyset = \{ x \mid x \in A$ and $x \notin \emptyset \}$.
    No element is an element of $\emptyset$.
    Hence $A \setminus \emptyset = \{ x \mid x \in A \}$.
    Then we have the thesis.
  \end{proof}

  \begin{proposition}[forthel,id=FOUNDATIONS_02_7083929257377792]
    Let $A, B$ be classes.
    Then $A \setminus (A \setminus B) = A \cap B$.
  \end{proposition}
  \begin{proof}[forthel]
    Let us show that $A \setminus (A \setminus B) \subseteq A \cap B$.
      Let $x \in A \setminus (A \setminus B)$.
      Then $x \in A$ and $x \notin A \setminus B$.
      Hence $x \notin A$ or $x \in B$.
      Thus $x \in B$.
      Therefore $x \in A \cap B$.
    End.

    Let us show that $A \cap B \subseteq A \setminus (A \setminus B)$.
      Let $x \in A \cap B$.
      Then $x \in A$ and $x \in B$.
      Hence $x \notin A$ or $x \in B$.
      Thus $x \notin A \setminus B$.
      Therefore $x \in A \setminus (A \setminus B)$.
    End.
  \end{proof}

  \begin{proposition}[forthel,id=FOUNDATIONS_02_4938646769631232]
    Let $A, B$ be classes.
    Then $B \subseteq A$ iff $A \setminus (A \setminus B) = B$.
  \end{proposition}
  \begin{proof}[forthel]
    Case $B \subseteq A$. Obvious.

    Case $A \setminus (A \setminus B) = B$.
      Then every element of $B$ is an element of $A \setminus (A \setminus B)$.
      Thus every element of $B$ is an element of $A$.
      Then we have the thesis.
    End.
  \end{proof}

  \begin{proposition}[forthel,id=FOUNDATIONS_02_5811954316738560]
    Let $A, B, C$ be classes.
    Then $A \cap (B \setminus C) = (A \cap B) \setminus (A \cap C)$.
  \end{proposition}
  \begin{proof}[forthel]
    Let us show that $A \cap (B \setminus C) \subseteq (A \cap B) \setminus (A \cap C)$.
      Let $x \in A \cap (B \setminus C)$.
      Then $x \in A$ and $x \in B \setminus C$.
      Hence $x \in A$ and $x \in B$.
      Thus $x \in A \cap B$ and $x \notin C$.
      Therefore $x \notin A \cap C$.
      Then we have $x \in (A \cap B) \setminus (A \cap C)$.
    End.

    Let us show that $((A \cap B) \setminus (A \cap C)) \subseteq A \cap (B \setminus C)$. %!
      Let $x \in (A \cap B) \setminus (A \cap C)$.
      Then $x \in A$ and $x \in B$.
      $x \notin A \cap C$.
      Hence $x \notin C$.
      Thus $x \in B \setminus C$.
      Therefore $x \in A \cap (B \setminus C)$.
    End.
  \end{proof}
\end{sfragment}
\end{smodule}
\end{document}
