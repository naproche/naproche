\documentclass{naproche-library}
\libinput{preamble}
\begin{document}
\IfInputref{}{\setsectionlevel{section}}
\begin{smodule}[title=Ordered Pairs and Cartesian Products]{pairs-and-products.ftl}
\importmodule[libraries/foundations]{classes.ftl}
\symdef{SETprod}{\,\times\,}

\begin{sfragment}{Pairs}
  \begin{axiom}[forthel,id=FOUNDATIONS_04_8464577431863296]
    Let $a, a', b, b'$ be objects.
    If $(a, b) \eq (a', b')$ then $a \eq a'$ and $b \eq b'$.
  \end{axiom}

  \begin{definition}[forthel,id=FOUNDATIONS_04_4782386822774784]
    A \emph{pair} is an object $p$ such that $p \eq (a, b)$ for some objects $a, b$.

    Let an \emph{ordered pair} stand for a pair.
  \end{definition}
\end{sfragment}

\begin{sfragment}{Cartesian Products}
  \begin{definition}[forthel,id=FOUNDATIONS_04_2877806274936832]
    Let $A, B$ be classes.
    The \emph{Cartesian product of $A$ and $B$} is $\{ (a, b) \mid a \in A$ and $b \in B \}$.

    Let the \emph{direct product of $A$ and $B$} stand for  the Cartesian product of $A$ and $B$.
    Let $\emph{A \SETprod B}$ stand for the Cartesian product of $A$ and $B$.
  \end{definition}

  \begin{proposition}[forthel,id=FOUNDATIONS_04_1581118511906816]
    Let $A, B$ be classes and $a, b$ be objects.
    Then $(a, b) \in A \SETprod B$ iff $a \in A$ and $b \in B$.
  \end{proposition}
  \begin{proof}[forthel]
    \begin{case}{$(a, b) \in A \SETprod B$.}
      We can take $a' \in A$ and $b' \in B$ such that $(a, b) \eq (a', b')$.
      Then $a \eq a'$ and $b \eq b'$.
      Hence $a \in A$ and $b \in B$.
    \end{case}

    \begin{case}{$a \in A$ and $b \in B$.}
      $a$ and $a$ are objects.
      Hence $(a, b)$ is an object.
      Therefore $(a, b) \in A \SETprod B$.
    \end{case}
  \end{proof}

  \begin{proposition}[forthel,id=FOUNDATIONS_04_2198552029691904]
    Let $A, B$ be classes.
    Then $A \SETprod B$ is empty iff $A$ is empty or $B$ is empty.
  \end{proposition}
  \begin{proof}[forthel]
    \begin{case}{$A \SETprod B$ is empty.}
      Assume that $A$ and $B$ are nonempty.
      Then we can take an element $a$ of $A$ and an element $b$ of $B$.
      Then $(a, b) \in A \SETprod B$.
      Contradiction.
    \end{case}

    \begin{case}{$A$ is empty or $B$ is empty.}
      Assume that $A \SETprod B$ is nonempty.
      Then we can take an element $c$ of $A \SETprod B$.
      Then $c \eq (a, b)$ for some $a \in A$ and some $b \in B$.
      Hence $A$ and $B$ are nonempty.
      Contradiction.
    \end{case}
  \end{proof}

  \begin{proposition}[forthel,id=FOUNDATIONS_04_7971087096741888]
    Let $a, b$ be objects.
    Then $\SETsingleton{a} \SETprod \SETsingleton{b} \eq \SETsingleton{(a, b)}$.
  \end{proposition}
  \begin{proof}[forthel]
    Let us show that $\SETsingleton{a} \SETprod \SETsingleton{b} \SETinclude \SETsingleton{(a, b)}$.
      Let $c \in \SETsingleton{a} \SETprod \SETsingleton{b}$.
      Take $a' \in \SETsingleton{a}$ and $b' \in \SETsingleton{b}$ such that $c \eq (a', b')$.
      We have $a' \eq a$ and $b' \eq b$.
      Hence $c \eq (a, b)$.
      Thus $c \in \SETsingleton{(a, b)}$.
    End.

    Let us show that $\SETsingleton{(a, b)} \SETinclude \SETsingleton{a} \SETprod \SETsingleton{b}$.
      Let $c \in \SETsingleton{(a, b)}$.
      Then $c \eq (a, b)$.
      We have $a \in \SETsingleton{a}$ and $b \in \SETsingleton{b}$.
      Hence $c \in \SETsingleton{a} \SETprod \SETsingleton{b}$.
    End.
  \end{proof}

  \begin{proposition}[forthel,id=FOUNDATIONS_04_7456594440749056]
    Let $A, A', B, B'$ be nonempty classes.
    If $A \SETprod B \eq A' \SETprod B'$ then $A \eq A'$ and $B \eq B'$.
  \end{proposition}
  \begin{proof}[forthel]
    Assume $A \SETprod B \eq A' \SETprod B'$.

    (1) $A \SETinclude A'$ and $B \SETinclude B'$.
    \begin{proof}
      Let $a \in A$ and $b \in B$.
      Then $(a,b) \in A \SETprod B$.
      Hence $(a,b) \in A' \SETprod B'$.
      Thus $a \in A'$ and $b \in B'$.
    \end{proof}

    (2) $A' \SETinclude A$ and $B' \SETinclude B$.
    \begin{proof}
      Let $a \in A'$ and $b \in B'$.
      Then $(a,b) \in A' \SETprod B'$.
      Hence $(a,b) \in A \SETprod B$.
      Thus $a \in A$ and $b \in B$.
    \end{proof}
  \end{proof}
\end{sfragment}
\end{smodule}
\end{document}
