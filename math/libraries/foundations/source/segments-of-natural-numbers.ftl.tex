\documentclass{naproche-library}
\libinput{preamble}
\begin{document}
\IfInputref{}{\setsectionlevel{section}}
\begin{smodule}[title=Segments of the Natural Numbers]{segments-of-natural-numbers.ftl}
\importmodule[libraries/arithmetics]{ordering.ftl}
\symdef{NATsegment}[args=2]{\comp{\{}#1\comp{,\dots,}#2\comp{\}}}

\begin{definition}[forthel,id=FOUNDATIONS_14_3625613501923328]
  Let $n, m$ be natural numbers.
  $\emph{\NATsegment{n}{m}} = \{ k \in \Nat \mid n \NATleq k \NATleq m \}$.
\end{definition}

\begin{proposition}[forthel,id=FOUNDATIONS_14_145331933151232]
  Let $n, m$ be natural numbers.
  If $\NATsegment{\NATone}{n} \eq \NATsegment{\NATone}{m}$ then $n \eq m$.
\end{proposition}
\begin{proof}[forthel]
  Assume $\NATsegment{\NATone}{n} \eq \NATsegment{\NATone}{m}$.

  \begin{case}{$n \eq \NATzero$.}
    Then $\NATsegment{\NATone}{n} \eq \SETempty$.
    Thus $\NATsegment{\NATone}{m} \eq \SETempty$.
    Hence there exists no $k \SETin \Nat$ such that $\NATone \NATleq k \NATleq m$.
    Therefore $m \eq \NATzero$.
    Consequently $n \eq m$.
  \end{case}

  \begin{case}{$m \eq \NATzero$.}
    Then $\NATsegment{\NATone}{m} \eq \SETempty$.
    Thus $\NATsegment{\NATone}{n} \eq \SETempty$.
    Hence there exists no $k \SETin \Nat$ such that $\NATone \NATleq k \NATleq n$.
    Therefore $n \eq \NATzero$.
    Consequently $n \eq m$.
  \end{case}

  \begin{case}{$n, m \NATgeq \NATone$.}
    For all $k \SETin \Nat$ we have $\NATone \NATleq k \NATleq n$ iff $\NATone \NATleq k \NATleq m$.
    Hence for all $k \SETin \Nat$ we have $k \NATleq n$ iff $k \NATleq m$.

    Let us show by contradiction that $n \eq m$.
      Suppose $n \neq m$.
      Then $n \NATgtr m$ or $m \NATgtr n$.

      \begin{case}{$n \NATgtr m$.}
        Take $k = m \NATplus \NATone$.
        Then $k \NATleq n$ and $k \NATnleq m$.
        Hence it is wrong that $k \NATleq n$ iff $k \NATleq m$.
        Contradiction.
      \end{case}

      \begin{case}{$m \NATgtr n$.}
        Take $k = n \NATplus \NATone$.
        Then $k \NATleq m$ and $k \NATnleq m$.
        Hence it is wrong that $k \NATleq n$ iff $k \NATleq m$.
        Contradiction.
      \end{case}
    End.
  \end{case}
\end{proof}

\begin{proposition}[forthel,id=FOUNDATIONS_14_658708738605056]
  Let $n$ be a natural number.
  Then $\NATsegment{\NATone}{n \NATplus \NATone} \eq \NATsegment{\NATone}{n} \SETunion \SETsingleton{n \NATplus \NATone}$.
\end{proposition}
\begin{proof}[forthel]
  We have
  $\NATsegment{\NATone}{n \NATplus \NATone} \SETinclude \NATsegment{\NATone}{n} \SETunion \SETsingleton{n \NATplus \NATone}$ and $\NATsegment{\NATone}{n} \SETunion \SETsingleton{n \NATplus \NATone} \SETinclude \NATsegment{\NATone}{n \NATplus \NATone}$.
\end{proof}
\end{smodule}
\end{document}
