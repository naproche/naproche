\documentclass{naproche-library}
\libinput{preamble}
\begin{document}
\IfInputref{}{\setsectionlevel{section}}
\begin{smodule}[title=Exponentiation]{exponentiation.ftl}
\importmodule[libraries/arithmetics]{multiplication.ftl}

\begin{sfragment}{Definition}
  \begin{signature}[forthel,id=ARITHMETIC_09_3663815629602816]
    Let $n, m$ be natural numbers.
    $n^{m}$ is a natural number.
  \end{signature}

  \begin{axiom}[forthel,id=ARITHMETIC_09_5368818025103360]
    Let $n$ be a natural number.
    Then $n^{0} = 1$.
  \end{axiom}

  \begin{axiom}[forthel,id=ARITHMETIC_09_4140498660884480]
    Let $n, m$ be natural numbers.
    Then $n^{m + 1} = n^{m} \cdot n$.
  \end{axiom}
\end{sfragment}

\begin{sfragment}{Computation Laws}
  \begin{sfragment}{Exponentiation with $0$, $1$ and $2$}
    \begin{proposition}[forthel,id=ARITHMETIC_09_4673644676513792]
      Let $n$ be a natural number.
      Assume $n \neq 0$.
      Then $0^{n} = 0$.
    \end{proposition}
    \begin{proof}[forthel]
      Take a natural number $m$ such that $n = m + 1$.
      Then $0^{n}
        = 0^{m + 1}
        = 0^{m} \cdot 0
        = 0$.
      Indeed $0^{m + 1} = 0^{m} \cdot 0$.
    \end{proof}

    \begin{proposition}[forthel,id=ARITHMETIC_09_7376849881530368]
      Let $n$ be a natural number.
      Then $1^{n} = 1$.
    \end{proposition}
    \begin{proof}[forthel]
      Define $\Phi = \{ n' \in \Nat \mid 1^{n'} = 1 \}$.

      (1) $\Phi$ contains $0$.

      (2) For all $n' \in \Phi$ we have $n' + 1 \in \Phi$. \newline
      Proof.
        Let $n' \in \Phi$.
        Then $1^{n' + 1}
          = 1^{n'} \cdot 1
          = 1 \cdot 1
          = 1$.
      Qed.

      Hence every natural number is contained in $\Phi$ (by \printref{ARITHMETIC_01_4764664342773760}).
      Thus $1^{n} = 1$.
    \end{proof}

    \begin{proposition}[forthel,id=ARITHMETIC_09_4975279749464064]
      Let $n$ be a natural number.
      Then $n^{1} = n$.
    \end{proposition}
    \begin{proof}[forthel]
      We have $n^{1}
        = n^{0 + 1}
        = n^{0} \cdot n
        = 1 \cdot n
        = n$.
    \end{proof}

    \begin{proposition}[forthel,id=ARITHMETIC_09_8513812055457792]
      Let $n$ be a natural number.
      Then $n^{2} = n \cdot n$.
    \end{proposition}
    \begin{proof}[forthel]
      We have $n^{2}
        = n^{1 + 1}
        = n^{1} \cdot n
        = n \cdot n$.
    \end{proof}
  \end{sfragment}

  \begin{sfragment}{Sums as Exponents}
    \begin{proposition}[forthel,id=ARITHMETIC_09_8152207530655744]
      Let $n, m, k$ be natural numbers.
      Then $k^{n + m} = k^{n} \cdot k^{m}$.
    \end{proposition}
    \begin{proof}[forthel]
      Define $\Phi = \{ m' \in \Nat \mid k^{n + m'} = k^{n} \cdot k^{m'} \}$.

      (1) $\Phi$ contains $0$. \newline
      Indeed $k^{n + 0}
        = k^{n}
        = k^{n} \cdot 1
        = k^{n} \cdot k^{0}$.

      (2) For all $m' \in \Phi$ we have $m' + 1 \in \Phi$. \newline
      Proof.
        Let $m' \in \Phi$.
        Then
        \[  k^{n + (m' + 1)}                  \]
        \[    = k^{(n + m') + 1}              \]
        \[    = k^{n + m'} \cdot k            \]
        \[    = (k^{n} \cdot k^{m'}) \cdot k  \]
        \[    = k^{n} \cdot (k^{m'} \cdot k)  \]
        \[    = k^{n} \cdot k^{m' + 1}.       \]
      Qed.

      Hence every natural number is contained in $\Phi$ (by \printref{ARITHMETIC_01_4764664342773760}).
      Thus $k^{n + m} = k^{n} \cdot k^{m}$.
    \end{proof}
  \end{sfragment}

  \begin{sfragment}{Products as Exponents}
    \begin{proposition}[forthel,id=ARITHMETIC_09_7827956571308032]
      Let $n, m, k$ be natural numbers.
      Then $n^{m \cdot k} = (n^{m})^{k}$.
    \end{proposition}
    \begin{proof}[forthel]
      Define $\Phi = \{ k' \in \Nat \mid n^{m \cdot k'} = (n^{m})^{k'} \}$.

      (1) $\Phi$ contains $0$.
      Indeed $(n^{m})^{0}
        = 1
        = n^{0}
        = n^{m \cdot 0}$.

      (2) For all $k' \in \Phi$ we have $k' + 1 \in \Phi$. \newline
      Proof.
        Let $k' \in \Phi$.
        Then
        \[  (n^{m})^{k' + 1}                \]
        \[    = (n^{m})^{k'} \cdot n^{m}    \]
        \[    = n^{m \cdot k'} \cdot n^{m}  \]
        \[    = n^{(m \cdot k') + m}        \]
        \[    = n^{m \cdot (k' + 1)}.       \]
      Qed.

      Therefore every natural number is contained in $\Phi$ (by \printref{ARITHMETIC_01_4764664342773760}).
      Consequently $n^{m \cdot k} = (n^{m})^{k}$.
    \end{proof}
  \end{sfragment}

  \begin{sfragment}{Products as Base}
    \begin{proposition}[forthel,id=ARITHMETIC_09_2563032276271104]
      Let $n, m, k$ be natural numbers.
      Then $(n \cdot m)^{k} = n^{k} \cdot m^{k}$.
    \end{proposition}
    \begin{proof}[forthel]
      Define $\Phi = \{ k' \in \Nat \mid (n \cdot m)^{k'} = n^{k'} \cdot m^{k'} \}$.

      (1) $\Phi$ contains $0$.
      Indeed $((n \cdot m)^{0})
        = 1
        = 1 \cdot 1
        = n^{0} \cdot m^{0}$. %!

      (2) For all $k' \in \Phi$ we have $k' + 1 \in \Phi$. \newline
      Proof.
        Let $k' \in \Phi$.

        Let us show that $(n^{k'} \cdot m^{k'}) \cdot (n \cdot m) = (n^{k'} \cdot n) \cdot (m^{k'} \cdot m)$.
          \[  (n^{k'} \cdot m^{k'}) \cdot (n \cdot m)       \]
          \[    = ((n^{k'} \cdot m^{k'}) \cdot n) \cdot m   \]
          \[    = (n^{k'} \cdot (m^{k'} \cdot n)) \cdot m   \]
          \[    = (n^{k'} \cdot (n \cdot m^{k'})) \cdot m   \]
          \[    = ((n^{k'} \cdot n) \cdot m^{k'}) \cdot m   \]
          \[    = (n^{k'} \cdot n) \cdot (m^{k'} \cdot m).  \]
        Qed.

        Hence
        \[  (n \cdot m)^{k' + 1}                          \]
        \[    = (n \cdot m)^{k'} \cdot (n \cdot m)        \]
        \[    = (n^{k'} \cdot m^{k'}) \cdot (n \cdot m)   \]
        \[    = (n^{k'} \cdot n) \cdot (m^{k'} \cdot m)   \]
        \[    = n^{k' + 1} \cdot m^{k' + 1}.              \]
      Qed.

      Therefore every natural number is contained in $\Phi$ (by \printref{ARITHMETIC_01_4764664342773760}).
      Consequently $(n \cdot m)^{k} = n^{k} \cdot m^{k}$.
    \end{proof}
  \end{sfragment}

  \begin{sfragment}{Zeroes of Exponentiation}
    \begin{proposition}[forthel,id=ARITHMETIC_09_3860221447372800]
      Let $n, m$ be natural numbers.
      Then $n^{m} = 0$ iff $n = 0$ and $m \neq 0$.
    \end{proposition}
    \begin{proof}[forthel]
      Case $n^{m} = 0$.
        Define $\Phi = \{ m' \in \Nat \mid$ if $n^{m'} = 0$ then $n = 0$ and $m' \neq 0 \}$.

        (1) $\Phi$ contains $0$.
        Indeed if $n^{0} = 0$ then we have a contradiction.

        (2) For all $m' \in \Phi$ we have $m' + 1 \in \Phi$. \newline
        Proof.
          Let $m' \in \Phi$.

          Let us show that if $n^{m' + 1} = 0$ then $n = 0$ and $m' + 1 \neq 0$.
            Assume $n^{m' + 1} = 0$.
            Then $0 = n^{m' + 1} = n^{m'} \cdot n$.
            Hence $n^{m'} = 0$ or $n = 0$.
            We have $m' + 1 \neq 0$ and if $n^{m'} = 0$ then $n = 0$.
            Hence $n = 0$ and $m' + 1 \neq 0$.
          End.
        Qed.

        Thus every natural number is contained in $\Phi$ (by \printref{ARITHMETIC_01_4764664342773760}).
        Consequently $m \in \Phi$.
        Therefore $n = 0$ and $m \neq 0$.
      End.

      Case $n = 0$ and $m \neq 0$.
        Take a natural number $k$ such that $m = k + 1$.
        Then $n^{m}
          = n^{k + 1}
          = n^{k} \cdot n
          = 0^{k} \cdot 0
          = 0$.
      End.
    \end{proof}
  \end{sfragment}
\end{sfragment}
\end{smodule}
\end{document}
