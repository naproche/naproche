\documentclass{naproche-library}
\libinput{preamble}
\begin{document}
\IfInputref{}{\setsectionlevel{section}}
\begin{smodule}[title=Prime Numbers]{primes.ftl}
\importmodule[libraries/arithmetics]{divisibility.ftl}
\importmodule[libraries/arithmetics]{modular-arithmetics.ftl}
\symdef{Prime}{\mathbb P}

\begin{definition}[forthel,id=ARITHMETIC_10_5450464558579712]
  Let $n$ be a natural number.
  $n$ is prime iff $n > 1$ and $n$ has no nontrivial divisors.

  Let $n$ is compound stand for $n$ is not prime.
  Let a prime number stand for a natural number that is prime.
\end{definition}

\begin{definition}[forthel,id=ARITHMETIC_10_3834705971511296]
  $\Prime$ is the class of all prime numbers.
\end{definition}

\begin{proposition}[forthel,id=ARITHMETIC_10_7801379464675328]
  Let $n$ be a natural number such that $n > 1$.
  Then $n$ is prime iff every divisor of $n$ is a trivial divisor of $n$.
\end{proposition}

\begin{proposition}[forthel,id=ARITHMETIC_10_3606185106210816]
  Let $n$ be a natural number such that $n > 1$.
  Then $n$ has a prime divisor.
\end{proposition}
\begin{proof}[forthel]
  Define $\Phi = \{ n' \in \Nat \mid$ if $n' > 1$ then $n'$ has a prime divisor $\}$.

  Let us show that for every $n' \in \Nat$ if $\Phi$ contains all
  predecessors of $n'$ then $\Phi$ contains $n'$.
    Let $n' \in \Nat$.
    Assume that $\Phi$ contains all predecessors of $n'$.
    We have $n' = 0$ or $n' = 1$ or $n'$ is prime or $n'$ is composite.

    Case $n' = 0$ or $n' = 1$. Trivial.

    Case $n'$ is prime. Obvious.

    Case $n'$ is composite.
      Take a nontrivial divisor $m$ of $n'$.
      Then $1 < m < n'$.
      $m$ is contained in $\Phi$.
      Hence we can take a prime divisor $p$ of $m$.
      Then we have $p \mid m \mid n'$.
      Thus $p \mid n'$.
      Therefore $p$ is a prime divisor of $n'$.
    End.
  End.

  Thus every natural number belongs to $\Phi$ (by \printref{ARITHMETIC_04_3609801697263616}).
\end{proof}

\begin{definition}[forthel,id=ARITHMETIC_10_463197419077632]
  Let $n, m$ be natural numbers.
  $n$ and $m$ are coprime iff for all nonzero natural numbers $k$ such that $k \mid n$ and $k \mid m$ we have $k = 1$.

  Let $n$ and $m$ are relatively prime stand for $n$ and $m$ are coprime.
  Let $n$ and $m$ are mutually prime stand for $n$ and $m$ are coprime.
  Let $n$ is prime to $m$ stand for $n$ and $m$ are coprime.
\end{definition}

\begin{proposition}[forthel,id=ARITHMETIC_10_5776394594287616]
  Let $n, m$ be natural numbers.
  $n$ and $m$ are coprime iff $n$ and $m$ have no common prime divisor.
\end{proposition}
\begin{proof}[forthel]
  Case $n$ and $m$ are coprime.
    Let $p$ be a prime number such that $p \mid n$ and $p \mid m$.
    Then $p$ is nonzero and $p \neq 1$.
    Contradiction.
  End.

  Case $n$ and $m$ have no common prime divisor.
    Assume that $n$ and $m$ are not coprime.
    Let $k$ be a nonzero natural number such that $k \mid n$ and $k \mid m$.
    Assume that $k \neq 1$.
    Consider a prime divisor $p$ of $k$.
    Then $p \mid k \mid n,m$.
    Hence $p \mid n$ and $p \mid m$.
    Contradiction.
  End.
\end{proof}

\begin{proposition}[forthel,id=ARITHMETIC_10_7212152851005440]
  Let $n, m$ be natural numbers and $p$ be a prime number.
  If $p$ does not divide $n$ then $p$ and $n$ are coprime.
\end{proposition}
\begin{proof}[forthel]
  Assume $p \nmid n$.
  Suppose that $p$ and $n$ are not coprime.
  Take a nonzero natural number $k$ such that $k \mid p$ and $k \mid n$.
  Then $k = p$.
  Hence $p \mid n$.
  Contradiction.
\end{proof}

\begin{proposition}[forthel,id=ARITHMETIC_10_8313676557713408]
  Let $n, m$ be natural numbers and $p$ be a prime number.
  If $p \mid n \cdot m$ then $p \mid n$ or $p \mid m$.
\end{proposition}
\begin{proof}[forthel]
  Assume $p \mid n \cdot m$.

  Case $p \mid n$. Trivial.

  Case $p \nmid n$.
    Define $\Phi = \{ k \in \Nat \mid k \neq 0$ and $p \mid k \cdot m \}$.
    Then $p \in \Phi$ and $n \in \Phi$.
    Hence $\Phi$ contains some natural number.
    Thus we can take $a \in \Phi$ such that $a \leq k$ for all $k \in \Phi$.

    Let us show that $a$ divides all elements of $\Phi$.
      Let $k \in \Phi$.
      Take natural numbers $q, r$ such that $k = (a \cdot q) + r$ and $r < a$ (by \printref{ARITHMETIC_08_7743986617810944}).
      Indeed $a$ is nonzero.
      Then $k \cdot m
        = ((q \cdot a) + r) \cdot m
        = ((q \cdot a) \cdot m) + (r \cdot m)$.
      We have $p \mid k \cdot m$.
      Hence $p \mid ((q \cdot a) \cdot m) + (r \cdot m)$.

      We can show that $p \mid r \cdot m$.
        We have $p \mid a \cdot m$.
        Hence $p \mid (q \cdot a) \cdot m$.
        Indeed $((q \cdot a) \cdot m) = q \cdot (a \cdot m)$. %!
        Take $A = (q \cdot a) \cdot m$ and $B = r \cdot m$. %!
        Then $p \mid A + B$ and $p \mid A$.
        Thus $p \mid B$ (by \printref{ARITHMETIC_07_1076947887063040}).
        Indeed $p, A$ and $B$ are natural numbers.
        Consequently $p \mid r \cdot m$.
      End.

      Therefore $r = 0$.
      Indeed if $r \neq 0$ then $r$ is an element of $\Phi$ that is less than $a$.
      Hence $k = q \cdot a$.
      Thus $a$ divides $k$.
    End.

    Then we have $a \mid p$ and $a \mid n$.
    Hence $a = p$ or $a = 1$.
    Thus $a = 1$.
    Indeed if $a = p$ then $p \mid n$.
    Then $1 \in \Phi$.
    Therefore $p \mid 1 \cdot m = m$.
  End.
\end{proof}
\end{smodule}
\end{document}
