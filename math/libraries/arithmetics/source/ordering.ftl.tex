\documentclass{naproche-library}
\libinput{preamble}
\begin{document}
\IfInputref{}{\setsectionlevel{section}}
\begin{smodule}[title=The Standard Ordering of the Natural Numbers]{ordering.ftl}
\importmodule[libraries/arithmetics]{natural-numbers.ftl}
\symdef{NATless}{\;<\;}
\symdef{NATnless}{\;\not<\;}
\symdef{NATgtr}{\;>\;}
\symdef{NATngtr}{\;\not>\;}
\symdef{NATleq}{\;\leq\;}
\symdef{NATnleq}{\;\not\leq\;}
\symdef{NATgeq}{\;\geq\;}
\symdef{NATngeq}{\;\not\geq\;}
\symdef{Natlessthan}[args=1]{\comp{\mathbb{N}}_{\comp<#1}}
\symdef{Natgtrthan}[args=1]{\comp[\mathbb{N}]_{\comp>#1}}
\symdef{Natleqthan}[args=1]{\comp{\mathbb{N}}_{\comp\leq#1}}
\symdef{Natgeqthan}[args=1]{\comp{\mathbb{N}}_{\comp\geq#1}}

\begin{sfragment}{Definitions and Immediate Consequences}
  \begin{definition}[forthel,id=ARITHMETIC_04_1926295512416256]
    Let $n, m$ be natural numbers.
    $\emph{n \NATless m}$ iff there exists a nonzero natural number $k$ such that $m \eq n \NATplus k$.

    Let $n$ is \emph{less than $m$} stand for $n \NATless m$.
    Let $\emph{n \NATgtr m}$ stand for $m \NATless n$.
    Let $n$ is \emph{greater than $m$} stand for $n \NATgtr m$.
    Let $\emph{n \NATnless m}$ stand for $n$ is not less than $m$.
    Let $\emph{n \NATngtr m}$ stand for $n$ is not greater than $m$.
  \end{definition}

  \begin{definition}[forthel,id=ARITHMETIC_04_3668680374222848]
    Let $n$ be a natural number.
    $\emph{\Natlessthan{n}} = \{ k \SETin \Nat \mid k \NATless n \}$.
  \end{definition}

  \begin{definition}[forthel,id=ARITHMETIC_04_3670333934534656]
    Let $n$ be a natural number.
    $\emph{\Natgtrthan{n}} = \{ k \SETin \Nat \mid k \NATgtr n \}$.
  \end{definition}

  \begin{definition}[forthel,id=ARITHMETIC_04_7916616566177792]
    Let $n$ be a natural number.
    $n$ is \emph{positive} iff $n \NATgtr \NATzero$.
\end{definition}

  \begin{definition}[forthel,id=ARITHMETIC_04_4593841531256832]
    Let $n, m$ be natural numbers.
    $\emph{n \NATleq m}$ iff there exists a natural number $k$ such that $m \eq n \NATplus k$.

    Let $n$ is \emph{less than or equal to $m$} stand for $n \NATleq m$.
    Let $\emph{n \NATgeq m}$ stand for $m \NATleq n$.
    Let $n$ is \emph{greater than or equal to $m$} stand for $n \NATgeq m$.
    Let $\emph{n \NATnleq m}$ stand for $n$ is not less than or equal to $m$.
    Let $\emph{n \NATngeq m}$ stand for $n$ is not greater than or equal to $m$.
  \end{definition}

  \begin{definition}[forthel,id=ARITHMETIC_04_72501526790144]
    Let $n$ be a natural number.
    $\emph{\Natleqthan{n}} = \{ k \SETin \Nat \mid k \NATleq n \}$.
  \end{definition}

  \begin{definition}[forthel,id=ARITHMETIC_04_1706933421604864]
    Let $n$ be a natural number.
    $\emph{\Natgeqthan{n}} = \{ k \SETin \Nat \mid k \NATgeq n \}$.
  \end{definition}

  \begin{proposition}[forthel,id=ARITHMETIC_04_5385415374667776]
    Let $n, m$ be natural numbers.
    $n \NATleq m$ iff $n \NATless m$ or $n \eq m$.
  \end{proposition}
  \begin{proof}[forthel]
    \begin{case}{$n \NATleq m$.}
      Take a natural number $k$ such that $m \eq n \NATplus k$.
      If $k \eq \NATzero$ then $n \eq m$. If $k \neq \NATzero$ then $n \NATless m$.
    \end{case}

    \begin{case}{$n \NATless m$ or $n \eq m$.}
      If $n \NATless m$ then there is a positive natural number $k$ such that $m \eq n \NATplus k$.
      If $n \eq m$ then $m \eq n \NATplus \NATzero$.
      Thus if $n \NATless m$ then there is a natural number $k$ such that $m \eq n \NATplus k$.
    \end{case}
  \end{proof}

  \inlineforthel{[synonym predecessor/-s]}

  \begin{definition}[forthel,id=ARITHMETIC_04_6232154608500736]
    Let $n$ be a natural number.
    A \emph{predecessor of $n$} is a natural number that is less than $n$.
  \end{definition}

  \inlineforthel{[synonym successor/-s]}

  \begin{definition}[forthel,id=ARITHMETIC_04_8147686326796288]
    Let $n$ be a natural number.
    A \emph{successor of $n$} is a natural number that is greater than $n$.
  \end{definition}

  \begin{proposition}[forthel,id=ARITHMETIC_04_4826285599621120]
    Let $n$ be a natural number.
    Then $n$ is positive iff $n$ is nonzero.
  \end{proposition}
  \begin{proof}[forthel]
    \begin{case}{$n$ is positive.}
      Take a positive natural number $k$ such that $n \eq \NATzero \NATplus k \eq k$.
      Then we have $n \neq \NATzero$.
    \end{case}

    \begin{case}{$n$ is nonzero.}
      Take a natural number $k$ such that $n \eq k \NATplus \NATone$.
      Then $n \eq \NATzero \NATplus (k \NATplus \NATone)$.
      $k \NATplus \NATone$ is positive.
      Hence $\NATzero \NATless n$.
    \end{case}
  \end{proof}
\end{sfragment}

\begin{sfragment}{Basic Properties}
  \begin{proposition}[forthel,id=ARITHMETIC_04_1037693395927040]
    Let $n$ be a natural number.
    Then \[ n \NATnless n. \]
  \end{proposition}
  \begin{proof}[forthel]
    Assume the contrary.
    Then we can take a positive natural number $k$ such that $n \eq n \NATplus k$.
    Then we have $\NATzero \eq k$.
    Contradiction.
  \end{proof}

  \begin{proposition}[forthel,id=ARITHMETIC_04_8266284905005056]
    Let $n, m$ be natural numbers.
    Then \[ n \NATless m \implies n \neq m. \]
  \end{proposition}
  \begin{proof}[forthel]
    Assume $n \NATless m$.
    Take a positive natural number $k$ such that $m \eq n \NATplus k$.
    If $n \eq m$ then $k \eq \NATzero$.
    Hence $n \neq m$.
  \end{proof}

  \begin{proposition}[forthel,id=ARITHMETIC_04_4190604718243840]
    Let $n, m$ be natural numbers.
    If $n \NATleq m$ and $m \NATleq n$ then $n \eq m$.
  \end{proposition}
  \begin{proof}[forthel]
    Assume $n \NATleq m$ and $m \NATleq n$.
    Take natural numbers $k, l$ such that $m \eq n \NATplus k$ and $n \eq m \NATplus l$.
    Then $m
      \eq n \NATplus k
      \eq (m \NATplus l) \NATplus k
      \eq m \NATplus (l \NATplus k)$.
    Hence $l \NATplus k \eq \NATzero$.
    Thus $l \eq \NATzero \eq k$.
    Indeed if $l \neq \NATzero$ or $k \neq \NATzero$ then $l \NATplus k$ is the direct successor of
    some natural number.
    Therefore $m \eq n$.
  \end{proof}

  \begin{proposition}[forthel,id=ARITHMETIC_04_6413905244979200]
    Let $n, m, k$ be natural numbers.
    If $n \NATless m \NATless k$ then $n \NATless k$.
  \end{proposition}
  \begin{proof}[forthel]
    Assume $n \NATless m \NATless k$.
    Take a positive natural number $a$ such that $m \eq n \NATplus a$.
    Take a positive natural number $b$ such that $k \eq m \NATplus b$.
    Then $k
      \eq m \NATplus b
      \eq (n \NATplus a) \NATplus b
      \eq n \NATplus (a \NATplus b)$.
    $a \NATplus b$ is positive.
    Hence $n \NATless k$.
  \end{proof}

  \begin{proposition}[forthel,id=ARITHMETIC_04_5480385953660928]
    Let $n, m, k$ be natural numbers.
    If $n \NATleq m \NATleq k$ then $n \NATleq k$.
  \end{proposition}
  \begin{proof}[forthel]
    Assume $n \NATleq m \NATleq k$.
    \begin{case}{$n \eq m \eq k$.} \end{case}
    \begin{case}{$n \eq m \NATless k$.} \end{case}
    \begin{case}{$n \NATless m \eq k$.} \end{case}
    \begin{case}{$n \NATless m \NATless k$.} \end{case}
  \end{proof}

  \begin{proposition}[forthel,id=ARITHMETIC_04_5098403656630272]
    Let $n, m, k$ be natural numbers.
    If $n \NATleq m \NATless k$ then $n \NATless k$.
  \end{proposition}
  \begin{proof}[forthel]
    Assume $n \NATleq m \NATless k$.
    If $n \eq m$ then $n \NATless k$.
    If $n \NATless m$ then $n \NATless k$.
  \end{proof}

  \begin{proposition}[forthel,id=ARITHMETIC_04_4809599527944192]
    Let $n, m, k$ be natural numbers.
    If $n \NATless m \NATleq k$ then $n \NATless k$.
  \end{proposition}
  \begin{proof}[forthel]
    Assume $n \NATless m \NATleq k$.
    If $m \eq k$ then $n \NATless k$.
    If $m \NATless k$ then $n \NATless k$.
  \end{proof}

  \begin{proposition}[forthel,id=ARITHMETIC_04_8584998051381248]
    Let $n, m$ be natural numbers.
    If $n \NATless m$ then $n \NATplus \NATone \NATleq m$.
  \end{proposition}
  \begin{proof}[forthel]
    Assume $n \NATless m$.
    Take a positive natural number $k$ such that $m \eq n \NATplus k$.

    \begin{case}{$k \eq \NATone$.}
      Then $m \eq n \NATplus \NATone$.
      Hence $n \NATplus \NATone \NATleq m$.
    \end{case}

    \begin{case}{$k \neq \NATone$.}
      Then we can take a natural number $l$ such that $k \eq l \NATplus \NATone$.
      Then $m
        \eq n \NATplus (l \NATplus \NATone)
        \eq (n \NATplus l) \NATplus \NATone
        \eq (n \NATplus \NATone) \NATplus l$.
      $l$ is positive.
      Thus $n \NATplus \NATone \NATless m$.
    \end{case}
  \end{proof}

  \begin{proposition}[forthel,id=ARITHMETIC_04_8201937860165632]
    Let $n, m$ be natural numbers.
    Then $n \NATless m$ or $n \eq m$ or $n \NATgtr m$.
  \end{proposition}
  \begin{proof}[forthel]
    Define $\Phi = \{ m' \SETin \Nat \mid n \NATless m'\text{ or }n \eq m'\text{ or }n \NATgtr m' \}$.

    (1) $\Phi$ contains $\NATzero$.

    (2) For all $m' \SETin \Phi$ we have $m' \NATplus \NATone \SETin \Phi$.
    \begin{proof}
      Let $m' \SETin \Phi$.

      \begin{case}{$n \NATless m'$.} \end{case}

      \begin{case}{$n \eq m'$.} \end{case}

      \begin{case}{$n \NATgtr m'$.}
        Take a positive natural number $k$ such that $n \eq m' \NATplus k$.

        \begin{case}{$k \eq \NATone$.} \end{case}

        \begin{case}{$k \neq \NATone$.}
          Take a natural number $l$ such that $n \eq (m' \NATplus \NATone) \NATplus l$.
          Hence $n \NATgtr m' \NATplus \NATone$.
          Indeed $l$ is positive.
        \end{case}
      \end{case}
    \end{proof}

    Thus every natural number is contained in $\Phi$ (by \printref{ARITHMETIC_01_4764664342773760}).
    Therefore $n \NATless m$ or $n \eq m$ or $n \NATgtr m$.
  \end{proof}

  \begin{proposition}[forthel,id=ARITHMETIC_04_6991525988794368]
    Let $n, m$ be natural numbers.
    Then $n \NATnless m$ iff $n \NATgeq m$.
  \end{proposition}
  \begin{proof}[forthel]
    \begin{case}{$n \NATnless m$.}
      Then $n \eq m$ or $n \NATgtr m$.
      Hence $n \NATgeq m$.
    \end{case}

    \begin{case}{$n \NATgeq m$.}
      Assume $n \NATless m$.
      Then $n \NATleq m$.
      Hence $n \eq m$.
      Contradiction.
    \end{case}
  \end{proof}
\end{sfragment}

\begin{sfragment}{Ordering and Successors}
  \begin{proposition}[forthel,id=ARITHMETIC_04_7006203091615744]
    Let $n, m$ be natural numbers.
    If $n \NATless m \NATleq n \NATplus \NATone$ then $m \eq n \NATplus \NATone$.
  \end{proposition}
  \begin{proof}[forthel]
    Assume $n \NATless m \NATleq n \NATplus \NATone$.
    Take a positive natural number $k$ such that $m \eq n \NATplus k$.
    Take a natural number $l$ such that $n \NATplus \NATone \eq m \NATplus l$.
    Then $n \NATplus \NATone
      \eq m \NATplus l
      \eq (n \NATplus k) \NATplus l
      \eq n \NATplus (k \NATplus l)$.
    Hence $k \NATplus l \eq \NATone$.

    We have $l \eq \NATzero$.
    \begin{proof}
      Assume the contrary.
      Then $k,l \NATgtr \NATzero$.

      \begin{case}{$k,l \eq \NATone$.}
        Then $k \NATplus l
          \eq \NATtwo
          \neq \NATone$.
        Contradiction.
      \end{case}

      \begin{case}{$k \eq \NATone and l \neq \NATone$.}
        Then $l \NATgtr \NATone$.
        Hence $k \NATplus l
          \NATgtr \NATone \NATplus l
          \NATgtr \NATone$.
        Contradiction.
      \end{case}

      \begin{case}{$k \neq \NATone and l \eq \NATone$.}
        Then $k \NATgtr \NATone$.
        Hence $k \NATplus l
          \NATgtr k \NATplus \NATone
          \NATgtr \NATone$.
        Contradiction.
      \end{case}

      \begin{case}{$k, l \neq \NATone$.}
        Take natural numbers $a, b$ such that $k \eq a \NATplus \NATone$ and $l \eq b \NATplus \NATone$.
        Indeed $k, l \neq \NATzero$.
        Hence $k \eq a \NATplus \NATone$ and $l \eq b \NATplus \NATone$.
        Thus $k, l \NATgtr \NATone$. Indeed $a, b$ are positive.
      \end{case}
    \end{proof}

    Then we have $n \NATplus \NATone
      \eq m \NATplus l
      \eq m \NATplus \NATzero
      \eq m$.
  \end{proof}

  \begin{proposition}[forthel,id=ARITHMETIC_04_8792330561650688]
    Let $n, m$ be natural numbers.
    If $n \NATleq m \NATless n \NATplus \NATone$ then $n \eq m$.
  \end{proposition}
  \begin{proof}[forthel]
    Assume $n \NATleq m \NATless n \NATplus \NATone$.

    \begin{case}{$n \eq m$.} \end{case}

    \begin{case}{$n \NATless m$.}
      Then $n \NATless m \NATleq n \NATplus \NATone$.
      Hence $n \eq m$.
    \end{case}
  \end{proof}

  \begin{corollary}[forthel,id=ARITHMETIC_04_1802826644717568]
    Let $n$ be a natural number.
    There is no natural number $m$ such that $n \NATless m \NATless n \NATplus \NATone$.
  \end{corollary}
  \begin{proof}[forthel]
    Assume the contrary.
    Take a natural number $m$ such that $n \NATless m \NATless n \NATplus \NATone$.
    Then $n \NATless m \NATleq n \NATplus \NATone$ and $n \NATleq m \NATless n \NATplus \NATone$.
    Hence $m \eq n \NATplus \NATone$ and $m \eq n$.
    Hence $n \eq n \NATplus \NATone$.
    Contradiction.
  \end{proof}

  \begin{proposition}[forthel,id=ARITHMETIC_04_990407185924096]
    Let $n$ be a natural number.
    Then $n \NATplus \NATone \NATgeq \NATone$.
  \end{proposition}
  \begin{proof}[forthel]
    \begin{case}{$n \eq \NATzero$.} \end{case}

    \begin{case}{$n \neq \NATzero$.}
      Then $n \NATgtr \NATzero$.
      Hence $n \NATplus \NATone \NATgtr \NATzero \NATplus \NATone \eq \NATone$.
    \end{case}
  \end{proof}
\end{sfragment}

\begin{sfragment}{Ordering and Addition}
  \begin{proposition}[forthel,id=ARITHMETIC_04_7354062662008832]
    Let $n, m, k$ be natural numbers.
    Then $n \NATless m$ iff $n \NATplus k \NATless m \NATplus k$.
  \end{proposition}
  \begin{proof}[forthel]
    \begin{case}{$n \NATless m$.}
      Take a positive natural number $l$ such that $m \eq n \NATplus l$.
      Then $m \NATplus k
        \eq (n \NATplus l) \NATplus k
        \eq (n \NATplus k) \NATplus l$.
      Hence $n \NATplus k \NATless m \NATplus k$.
    \end{case}

    \begin{case}{$n \NATplus k \NATless m \NATplus k$.}
      Take a positive natural number $l$ such that $m \NATplus k \eq (n \NATplus k) \NATplus l$.
      $(n \NATplus k) \NATplus l
        \eq n \NATplus (k \NATplus l)
        \eq n \NATplus (l \NATplus k)
        \eq (n \NATplus l) \NATplus k$.
      Hence $m \NATplus k \eq (n \NATplus l) \NATplus k$.
      Thus $m \eq n \NATplus l$ (by \printref{ARITHMETIC_03_3137702874578944}).
      Therefore $n \NATless m$.
    \end{case}
  \end{proof}

  \begin{corollary}[forthel,id=ARITHMETIC_04_1901366129721344]
    Let $n, m, k$ be natural numbers.
    Then $n \NATless m$ iff $k \NATplus n \NATless k \NATplus m$.
  \end{corollary}
  \begin{proof}[forthel]
    We have $k \NATplus n \eq n \NATplus k$ and $k \NATplus m \eq m \NATplus k$.
    Hence $k \NATplus n \NATless k \NATplus m$ iff $n \NATplus k \NATless m \NATplus k$.
  \end{proof}

  \begin{corollary}[forthel,id=ARITHMETIC_04_4203390999461888]
    Let $n, m, k$ be natural numbers.
    Then $n \NATleq m$ iff $k \NATplus n \NATleq k \NATplus m$.
  \end{corollary}

  \begin{corollary}[forthel,id=ARITHMETIC_04_5512590832697344]
    Let $n, m, k$ be natural numbers.
    Then $n \NATleq m$ iff $n \NATplus k \NATleq m \NATplus k$.
  \end{corollary}
\end{sfragment}

\begin{sfragment}{Induction Revisited}
  \begin{proposition}[forthel,id=ARITHMETIC_04_272317502455808]
    Let $A$ be a nonempty subclass of $\Nat$.
    Then there exists a $m \SETin A$ such that $m \NATleq n$ for all $n \SETin A$.
  \end{proposition}
  \begin{proof}[forthel]
    Assume the contrary.

    Let us show that for each $n \SETin A$ there exists a $m \SETin A$ such that $m \NATless n$.
      Let $n \SETin A$.
      Assume that there exists no $m \SETin A$ such that $m \NATless n$.
      Then $n \NATleq m$ for all $m \SETin A$.
      Contradiction.
    End.

    (a) Define $\Phi = \{ n \SETin \Nat \mid n\text{ is less than any element of }A \}$.

    (1) $\Phi$ contains $\NATzero$.
    \begin{proof}
      $\NATzero \SETnotin A$.
      Hence $\NATzero$ is less than every element of $A$.
      Thus $\NATzero \SETin \Phi$.
    \end{proof}

    (2) For all $n \SETin \Phi$ we have $n \NATplus \NATone \SETin \Phi$.
    \begin{proof}
      Let $n \SETin \Phi$.
      Then $n$ is less than any element of $A$.
      Assume that $\Phi$ does not contain $n \NATplus \NATone$.
      Then we can take an $m \SETin A$ such that $n \NATplus \NATone \NATnless m$ (by a).
      Then $n \NATless m \NATleq n \NATplus \NATone$.
      Hence $m \eq n \NATplus \NATone$.
      Contradiction.
    \end{proof}

    Then $\Phi$ contains every natural number (by \printref{ARITHMETIC_01_4764664342773760}).
    Therefore every natural number is less than any element of $A$.
    Consequently $A$ has no elements.
    Contradiction.
  \end{proof}

  \begin{theorem}[forthel,id=ARITHMETIC_04_3609801697263616]
    Let $A$ be a class.
    Assume for all $n \SETin \Nat$ if $A$ contains all predecessors of $n$ then $A$ contains $n$.
    Then $A$ contains every natural number.
  \end{theorem}
  \begin{proof}[forthel]
    Assume the contrary.
    Take a natural number $n$ that is not contained in $A$.
    Then $n$ is contained in $\Nat \SETdiff A$.
    Hence we can take a $m \SETin \Nat \SETdiff A$ such that $m \NATleq k$ for all $k \SETin \Nat \SETdiff A$.
    Then $\Nat \SETdiff A$ does not contain any predecessor of $m$.
    Therefore $A$ contains all predecessors of $m$.
    Consequently $A$ contains $m$.
    Contradiction.
  \end{proof}

  \begin{theorem}[forthel,id=ARITHMETIC_04_4976599269113856]
    Let $A$ be a class.
    Let $k$ be a natural number such that $k \SETin A$.
    Assume that for all $n \SETin \Natgeqthan{k}$ if $n \SETin A$ then $n \NATplus \NATone \SETin A$.
    Then for all $n \SETin \Natgeqthan{k}$ we have $n \SETin A$.
  \end{theorem}
  \begin{proof}[forthel]
    Define $\Phi = \{n \SETin \Nat \mid\text{ if }n \NATgeq k\text{ then }n \SETin A \}$.

    (1) $\Phi$ contains $\NATzero$.
    Indeed if $\NATzero \NATgeq k$ then $\NATzero \eq k \SETin A$.

    (2) For all $n \SETin \Phi$ we have $n \NATplus \NATone \SETin \Phi$.
    \begin{proof}
      Let $n \SETin \Phi$.

      Let us show that if $n \NATplus \NATone \NATgeq k$ then $n \NATplus \NATone \SETin A$.
        Assume $n \NATplus \NATone \NATgeq k$.

        \begin{case}{$n \NATless k$.}
          Then $n \NATplus \NATone \eq k$.
          Hence $n \NATplus \NATone \SETin A$.
        \end{case}

        \begin{case}{$n \NATgeq k$.}
          Then $n \SETin A$.
          Hence $n \NATplus \NATone \SETin A$.
        \end{case}
      End.

      Therefore $n \NATplus \NATone \SETin \Phi$.
    \end{proof}

    Thus $\Phi$ contains every natural number (by \printref{ARITHMETIC_01_4764664342773760}).
    Consequently for all $n \SETin \Natgeqthan{k}$ we have $n \SETin A$.
  \end{proof}
\end{sfragment}
\end{smodule}
\end{document}
