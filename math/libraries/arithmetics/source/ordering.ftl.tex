\documentclass{naproche-library}
\libinput{preamble}
\begin{document}
\IfInputref{}{\setsectionlevel{section}}
\begin{smodule}[title=The Standard Ordering of the Natural Numbers]{ordering.ftl}
\importmodule[libraries/arithmetics]{natural-numbers.ftl}

\begin{sfragment}{Definitions and Immediate Consequences}
  \begin{definition}[forthel,id=ARITHMETIC_04_1926295512416256]
    Let $n, m$ be natural numbers.
    $\emph{n < m}$ iff there exists a nonzero natural number $k$ such that $m = n + k$.

    Let $n$ is \emph{less than $m$} stand for $n < m$.
    Let $\emph{n > m}$ stand for $m < n$.
    Let $n$ is \emph{greater than $m$} stand for $n > m$.
    Let $\emph{n \nless m}$ stand for $n$ is not less than $m$.
    Let $\emph{n \ngtr m}$ stand for $n$ is not greater than $m$.
  \end{definition}

  \begin{definition}[forthel,id=ARITHMETIC_04_3668680374222848]
    Let $n$ be a natural number.
    $\emph{\Nat_{< n}} = \{ k \in \Nat \mid k < n \}$.
  \end{definition}

  \begin{definition}[forthel,id=ARITHMETIC_04_3670333934534656]
    Let $n$ be a natural number.
    $\emph{\Nat_{> n}} = \{ k \in \Nat \mid k > n \}$.
  \end{definition}

  \begin{definition}[forthel,id=ARITHMETIC_04_7916616566177792]
    Let $n$ be a natural number.
    $n$ is \emph{positive} iff $n > 0$.
\end{definition}

  \begin{definition}[forthel,id=ARITHMETIC_04_4593841531256832]
    Let $n, m$ be natural numbers.
    $\emph{n \leq m}$ iff there exists a natural number $k$ such that $m = n + k$.

    Let $n$ is \emph{less than or equal to $m$} stand for $n \leq m$.
    Let $\emph{n \geq m}$ stand for $m \leq n$.
    Let $n$ is \emph{greater than or equal to $m$} stand for $n \geq m$.
    Let $\emph{n \nleq m}$ stand for $n$ is not less than or equal to $m$.
    Let $\emph{n \ngeq m}$ stand for $n$ is not greater than or equal to $m$.
  \end{definition}

  \begin{definition}[forthel,id=ARITHMETIC_04_72501526790144]
    Let $n$ be a natural number.
    $\emph{\Nat_{\leq n}} = \{ k \in \Nat \mid k \leq n \}$.
  \end{definition}

  \begin{definition}[forthel,id=ARITHMETIC_04_1706933421604864]
    Let $n$ be a natural number.
    $\emph{\Nat_{\geq n}} = \{ k \in \Nat \mid k \geq n \}$.
  \end{definition}

  \begin{proposition}[forthel,id=ARITHMETIC_04_5385415374667776]
    Let $n, m$ be natural numbers.
    $n \leq m$ iff $n < m$ or $n = m$.
  \end{proposition}
  \begin{proof}[forthel]
    Case $n \leq m$.
      Take a natural number $k$ such that $m = n + k$.
      If $k = 0$ then $n = m$. If $k \neq 0$ then $n < m$.
    End.

    Case $n < m$ or $n = m$.
      If $n < m$ then there is a positive natural number $k$ such that $m = n + k$.
      If $n = m$ then $m = n + 0$.
      Thus if $n < m$ then there is a natural number $k$ such that $m = n + k$.
    End.
  \end{proof}

  \begin{definition}[forthel,id=ARITHMETIC_04_6232154608500736]
    Let $n$ be a natural number.
    A \emph{predecessor of $n$} is a natural number that is less than $n$.
  \end{definition}

  \begin{definition}[forthel,id=ARITHMETIC_04_8147686326796288]
    Let $n$ be a natural number.
    A \emph{successor of $n$} is a natural number that is greater than $n$.
  \end{definition}

  \begin{proposition}[forthel,id=ARITHMETIC_04_4826285599621120]
    Let $n$ be a natural number.
    Then $n$ is positive iff $n$ is nonzero.
  \end{proposition}
  \begin{proof}[forthel]
    Case $n$ is positive.
      Take a positive natural number $k$ such that $n = 0 + k = k$.
      Then we have $n \neq 0$.
    End.

    Case $n$ is nonzero.
      Take a natural number $k$ such that $n = k + 1$.
      Then $n = 0 + (k + 1)$.
      $k + 1$ is positive.
      Hence $0 < n$.
    End.
  \end{proof}
\end{sfragment}

\begin{sfragment}{Basic Properties}
  \begin{proposition}[forthel,id=ARITHMETIC_04_1037693395927040]
    Let $n$ be a natural number.
    Then \[ n \nless n. \]
  \end{proposition}
  \begin{proof}[forthel]
    Assume the contrary.
    Then we can take a positive natural number $k$ such that $n = n + k$.
    Then we have $0 = k$.
    Contradiction.
  \end{proof}

  \begin{proposition}[forthel,id=ARITHMETIC_04_8266284905005056]
    Let $n, m$ be natural numbers.
    Then \[ n < m \implies n \neq m. \]
  \end{proposition}
  \begin{proof}[forthel]
    Assume $n < m$.
    Take a positive natural number $k$ such that $m = n + k$.
    If $n = m$ then $k = 0$.
    Hence $n \neq m$.
  \end{proof}

  \begin{proposition}[forthel,id=ARITHMETIC_04_4190604718243840]
    Let $n, m$ be natural numbers.
    If $n \leq m$ and $m \leq n$ then $n = m$.
  \end{proposition}
  \begin{proof}[forthel]
    Assume $n \leq m$ and $m \leq n$.
    Take natural numbers $k, l$ such that $m = n + k$ and $n = m + l$.
    Then $m
      = n + k
      = (m + l) + k
      = m + (l + k)$.
    Hence $l + k = 0$.
    Thus $l = 0 = k$.
    Indeed if $l \neq 0$ or $k \neq 0$ then $l + k$ is the direct successor of
    some natural number.
    Therefore $m = n$.
  \end{proof}

  \begin{proposition}[forthel,id=ARITHMETIC_04_6413905244979200]
    Let $n, m, k$ be natural numbers.
    If $n < m < k$ then $n < k$.
  \end{proposition}
  \begin{proof}[forthel]
    Assume $n < m < k$.
    Take a positive natural number $a$ such that $m = n + a$.
    Take a positive natural number $b$ such that $k = m + b$.
    Then $k
      = m + b
      = (n + a) + b
      = n + (a + b)$.
    $a + b$ is positive.
    Hence $n < k$.
  \end{proof}

  \begin{proposition}[forthel,id=ARITHMETIC_04_5480385953660928]
    Let $n, m, k$ be natural numbers.
    If $n \leq m \leq k$ then $n \leq k$.
  \end{proposition}
  \begin{proof}[forthel]
    Assume $n \leq m \leq k$.
    Case $n = m = k$. Obvious.
    Case $n = m < k$. Obvious.
    Case $n < m = k$. Obvious.
    Case $n < m < k$. Obvious.
  \end{proof}

  \begin{proposition}[forthel,id=ARITHMETIC_04_5098403656630272]
    Let $n, m, k$ be natural numbers.
    If $n \leq m < k$ then $n < k$.
  \end{proposition}
  \begin{proof}[forthel]
    Assume $n \leq m < k$.
    If $n = m$ then $n < k$.
    If $n < m$ then $n < k$.
  \end{proof}

  \begin{proposition}[forthel,id=ARITHMETIC_04_4809599527944192]
    Let $n, m, k$ be natural numbers.
    If $n < m \leq k$ then $n < k$.
  \end{proposition}
  \begin{proof}[forthel]
    Assume $n < m \leq k$.
    If $m = k$ then $n < k$.
    If $m < k$ then $n < k$.
  \end{proof}

  \begin{proposition}[forthel,id=ARITHMETIC_04_8584998051381248]
    Let $n, m$ be natural numbers.
    If $n < m$ then $n + 1 \leq m$.
  \end{proposition}
  \begin{proof}[forthel]
    Assume $n < m$.
    Take a positive natural number $k$ such that $m = n + k$.

    Case $k = 1$.
      Then $m = n + 1$.
      Hence $n + 1 \leq m$.
    End.

    Case $k \neq 1$.
      Then we can take a natural number $l$ such that $k = l + 1$.
      Then $m
        = n + (l + 1)
        = (n + l) + 1
        = (n + 1) + l$.
      $l$ is positive.
      Thus $n + 1 < m$.
    End.
  \end{proof}

  \begin{proposition}[forthel,id=ARITHMETIC_04_8201937860165632]
    Let $n, m$ be natural numbers.
    Then $n < m$ or $n = m$ or $n > m$.
  \end{proposition}
  \begin{proof}[forthel]
    Define $\Phi = \{ m' \in \Nat \mid n < m'$ or $n = m'$ or $n > m' \}$.

    (1) $\Phi$ contains $0$.

    (2) For all $m' \in \Phi$ we have $m' + 1 \in \Phi$. \newline
    Proof.
      Let $m' \in \Phi$.

      Case $n < m'$. Obvious.

      Case $n = m'$. Obvious.

      Case $n > m'$.
        Take a positive natural number $k$ such that $n = m' + k$.

        Case $k = 1$. Obvious.

        Case$k \neq 1$.
          Take a natural number $l$ such that $n = (m' + 1) + l$.
          Hence $n > m' + 1$.
          Indeed $l$ is positive.
        End.
      Qed.
    Qed.

    Thus every natural number is contained in $\Phi$ (by \printref{ARITHMETIC_01_4764664342773760}).
    Therefore $n < m$ or $n = m$ or $n > m$.
  \end{proof}

  \begin{proposition}[forthel,id=ARITHMETIC_04_6991525988794368]
    Let $n, m$ be natural numbers.
    Then $n \nless m$ iff $n \geq m$.
  \end{proposition}
  \begin{proof}[forthel]
    Case $n \nless m$.
      Then $n = m$ or $n > m$.
      Hence $n \geq m$.
    End.

    Case $n \geq m$.
      Assume $n < m$.
      Then $n \leq m$.
      Hence $n = m$.
      Contradiction.
    End.
  \end{proof}
\end{sfragment}

\begin{sfragment}{Ordering and Successors}
  \begin{proposition}[forthel,id=ARITHMETIC_04_7006203091615744]
    Let $n, m$ be natural numbers.
    If $n < m \leq n + 1$ then $m = n + 1$.
  \end{proposition}
  \begin{proof}[forthel]
    Assume $n < m \leq n + 1$.
    Take a positive natural number $k$ such that $m = n + k$.
    Take a natural number $l$ such that $n + 1 = m + l$.
    Then $n + 1
      = m + l
      = (n + k) + l
      = n + (k + l)$.
    Hence $k + l = 1$.

    We have $l = 0$. \newline
    Proof.
      Assume the contrary.
      Then $k,l > 0$.

      Case $k,l = 1$.
        Then $k + l
          = 2
          \neq 1$.
        Contradiction.
      End.

      Case $k = 1 and l \neq 1$.
        Then $l > 1$.
        Hence $k + l
          > 1 + l
          > 1$.
        Contradiction.
      End.

      Case $k \neq 1 and l = 1$.
        Then $k > 1$.
        Hence $k + l
          > k + 1
          > 1$.
        Contradiction.
      End.

      Case $k, l \neq 1$.
        Take natural numbers $a, b$ such that $k = a + 1$ and $l = b + 1$.
        Indeed $k, l \neq 0$.
        Hence $k = a + 1$ and $l = b + 1$.
        Thus $k, l > 1$. Indeed $a, b$ are positive.
      End.
    Qed.

    Then we have $n + 1
      = m + l
      = m + 0
      = m$.
  \end{proof}

  \begin{proposition}[forthel,id=ARITHMETIC_04_8792330561650688]
    Let $n, m$ be natural numbers.
    If $n \leq m < n + 1$ then $n = m$.
  \end{proposition}
  \begin{proof}[forthel]
    Assume $n \leq m < n + 1$.

    Case $n = m$. Obvious.

    Case $n < m$.
      Then $n < m \leq n + 1$.
      Hence $n = m$.
    End.
  \end{proof}

  \begin{corollary}[forthel,id=ARITHMETIC_04_1802826644717568]
    Let $n$ be a natural number.
    There is no natural number $m$ such that $n < m < n + 1$.
  \end{corollary}
  \begin{proof}[forthel]
    Assume the contrary.
    Take a natural number $m$ such that $n < m < n + 1$.
    Then $n < m \leq n + 1$ and $n \leq m < n + 1$.
    Hence $m = n + 1$ and $m = n$.
    Hence $n = n + 1$.
    Contradiction.
  \end{proof}

  \begin{proposition}[forthel,id=ARITHMETIC_04_990407185924096]
    Let $n$ be a natural number.
    Then $n + 1 \geq 1$.
  \end{proposition}
  \begin{proof}[forthel]
    Case $n = 0$. Obvious.

    Case $n \neq 0$.
      Then $n > 0$.
      Hence $n + 1 > 0 + 1 = 1$.
    End.
  \end{proof}
\end{sfragment}

\begin{sfragment}{Ordering and Addition}
  \begin{proposition}[forthel,id=ARITHMETIC_04_7354062662008832]
    Let $n, m, k$ be natural numbers.
    Then $n < m$ iff $n + k < m + k$.
  \end{proposition}
  \begin{proof}[forthel]
    Case $n < m$.
      Take a positive natural number $l$ such that $m = n + l$.
      Then $m + k
        = (n + l) + k
        = (n + k) + l$.
      Hence $n + k < m + k$.
    End.

    Case $n + k < m + k$.
      Take a positive natural number $l$ such that $m + k = (n + k) + l$.
      $(n + k) + l
        = n + (k + l)
        = n + (l + k)
        = (n + l) + k$.
      Hence $m + k = (n + l) + k$.
      Thus $m = n + l$ (by \printref{ARITHMETIC_03_3137702874578944}).
      Therefore $n < m$.
    End.
  \end{proof}

  \begin{corollary}[forthel,id=ARITHMETIC_04_1901366129721344]
    Let $n, m, k$ be natural numbers.
    Then $n < m$ iff $k + n < k + m$.
  \end{corollary}
  \begin{proof}[forthel]
    We have $k + n = n + k$ and $k + m = m + k$.
    Hence $k + n < k + m$ iff $n + k < m + k$.
  \end{proof}

  \begin{corollary}[forthel,id=ARITHMETIC_04_4203390999461888]
    Let $n, m, k$ be natural numbers.
    Then $n \leq m$ iff $k + n \leq k + m$.
  \end{corollary}

  \begin{corollary}[forthel,id=ARITHMETIC_04_5512590832697344]
    Let $n, m, k$ be natural numbers.
    Then $n \leq m$ iff $n + k \leq m + k$.
  \end{corollary}
\end{sfragment}

\begin{sfragment}{Induction Revisited}
  \begin{proposition}[forthel,id=ARITHMETIC_04_272317502455808]
    Let $A$ be a nonempty subclass of $\Nat$.
    Then there exists a $m \in A$ such that $m \leq n$ for all $n \in A$.
  \end{proposition}
  \begin{proof}[forthel]
    Assume the contrary.

    Let us show that for each $n \in A$ there exists a $m \in A$ such that $m < n$.
      Let $n \in A$.
      Assume that there exists no $m \in A$ such that $m < n$.
      Then $n \leq m$ for all $m \in A$.
      Contradiction.
    End.

    (a) Define $\Phi = \{ n \in \Nat \mid n$ is less than any element of $A \}$.

    (1) $\Phi$ contains $0$. \newline
    Proof.
      $0 \notin A$.
      Hence $0$ is less than every element of $A$.
      Thus $0 \in \Phi$.
    Qed.

    (2) For all $n \in \Phi$ we have $n + 1 \in \Phi$. \newline
    Proof.
      Let $n \in \Phi$.
      Then $n$ is less than any element of $A$.
      Assume that $\Phi$ does not contain $n + 1$.
      Then we can take an $m \in A$ such that $n + 1 \nless m$ (by a).
      Then $n < m \leq n + 1$.
      Hence $m = n + 1$.
      Contradiction.
    Qed.

    Then $\Phi$ contains every natural number (by \printref{ARITHMETIC_01_4764664342773760}).
    Therefore every natural number is less than any element of $A$.
    Consequently $A$ has no elements.
    Contradiction.
  \end{proof}

  \begin{theorem}[forthel,id=ARITHMETIC_04_3609801697263616]
    Let $A$ be a class.
    Assume for all $n \in \Nat$ if $A$ contains all predecessors of $n$ then $A$ contains $n$.
    Then $A$ contains every natural number.
  \end{theorem}
  \begin{proof}[forthel]
    Assume the contrary.
    Take a natural number $n$ that is not contained in $A$.
    Then $n$ is contained in $\Nat \setminus A$.
    Hence we can take a $m \in \Nat \setminus A$ such that $m \leq k$ for all $k \in \Nat \setminus A$.
    Then $\Nat \setminus A$ does not contain any predecessor of $m$.
    Therefore $A$ contains all predecessors of $m$.
    Consequently $A$ contains $m$.
    Contradiction.
  \end{proof}

  \begin{theorem}[forthel,id=ARITHMETIC_04_4976599269113856]
    Let $A$ be a class.
    Let $k$ be a natural number such that $k \in A$.
    Assume that for all $n \in \Nat_{\geq k}$ if $n \in A$ then $n + 1 \in A$.
    Then for all $n \in \Nat_{\geq k}$ we have $n \in A$.
  \end{theorem}
  \begin{proof}[forthel]
    Define $\Phi = \{n \in \Nat \mid$ if $n \geq k$ then $n \in A \}$.

    (1) $\Phi$ contains $0$.
    Indeed if $0 \geq k$ then $0 = k \in A$.

    (2) For all $n \in \Phi$ we have $n + 1 \in \Phi$. \newline
    Proof.
      Let $n \in \Phi$.

      Let us show that if $n + 1 \geq k$ then $n + 1 \in A$.
        Assume $n + 1 \geq k$.

        Case $n < k$.
          Then $n + 1 = k$.
          Hence $n + 1 \in A$.
        End.

        Case $n \geq k$.
          Then $n \in A$.
          Hence $n + 1 \in A$.
        End.
      End.

      Therefore $n + 1 \in \Phi$.
    Qed.

    Thus $\Phi$ contains every natural number (by \printref{ARITHMETIC_01_4764664342773760}).
    Consequently for all $n \in \Nat_{\geq k}$ we have $n \in A$.
  \end{proof}
\end{sfragment}
\end{smodule}
\end{document}
