\documentclass{naproche-library}
\libinput{preamble}
\begin{document}
\IfInputref{}{\setsectionlevel{section}}
\begin{smodule}[title=Multiplication and Ordering]{multiplication-and-ordering.ftl}
\importmodule[libraries/arithmetics]{multiplication.ftl}
\importmodule[libraries/arithmetics]{ordering.ftl}

\begin{proposition}[forthel,id=ARITHMETIC_06_8817333933965312]
  Let $n, m, k$ be natural numbers.
  Assume $k \neq \NATzero$.
  Then $n \NATless m$ iff $n \NATmul k \NATless m \NATmul k$.
\end{proposition}
\begin{proof}[forthel]
  \begin{case}{$n \NATmul k \NATless m \NATmul k$.}
    Define $\Phi = \{ n' \in \Nat \mid$ if $n' \NATmul k \NATless m \NATmul k$ then $n' \NATless m \}$.

    (1) $\Phi$ contains $\NATzero$.

    (2) For all $n' \in \Phi$ we have $n' \NATplus \NATone \in \Phi$.
    \begin{proof}
      Let $n' \in \Phi$.

      Let us show that if $(n' \NATplus \NATone) \NATmul k \NATless m \NATmul k$ then $n' \NATplus \NATone \NATless m$.
        Assume $(n' \NATplus \NATone) \NATmul k \NATless m \NATmul k$.
        Then $(n' \NATmul k) \NATplus k \NATless m \NATmul k$.
        Hence $n' \NATmul k \NATless m \NATmul k$.
        Thus $n' \NATless m$.
        Then $n' \NATplus \NATone \NATleq m$.
        If $n' \NATplus \NATone \eq m$ then $(n' \NATplus \NATone) \NATmul k \eq m \NATmul k$.
        Hence $n' \NATplus \NATone \NATless m$.
      End.
    \end{proof}

    Therefore every natural number is contained in $\Phi$ (by \printref{ARITHMETIC_01_4764664342773760}).
    Consequently $n \NATless m$.
  \end{case}

  \begin{case}{$n \NATless m$.}
    Take a positive natural number $l$ such that $m \eq n \NATplus l$.
    Then $m \NATmul k \eq (n \NATplus l) \NATmul k \eq (n \NATmul k) \NATplus (l \NATmul k)$.
    $l \NATmul k$ is positive.
    Hence $n \NATmul k \NATless m \NATmul k$.
  \end{case}
\end{proof}

\begin{corollary}[forthel,id=ARITHMETIC_06_5048640368279552]
  Let $n, m, k$ be natural numbers.
  Assume $k \neq \NATzero$.
  Then $n \NATless m$ iff $k \NATmul n \NATless k \NATmul m$.
\end{corollary}
\begin{proof}[forthel]
  The thesis (by \printref{ARITHMETIC_06_8817333933965312}, \printref{ARITHMETIC_06_1764759896588288}).
\end{proof}

\begin{proposition}[forthel,id=ARITHMETIC_06_1826268599287808]
  Let $n, m, k$ be natural numbers.
  If $n, m \NATgtr k$ then $n \NATmul m \NATgtr k$.
\end{proposition}
\begin{proof}[forthel]
  Define $\Phi = \{ n' \in \Nat \mid$ if $n', m \NATgtr k$ then $n' \NATmul m \NATgtr k \}$.

  (1) $\Phi$ contains $\NATzero$.

  (2) For all $n' \in \Phi$ we have $n' \NATplus \NATone \in \Phi$.
  \begin{proof}
    Let $n' \in \Phi$.

    Let us show that if $n' \NATplus \NATone, m \NATgtr k$ then $(n' \NATplus \NATone) \NATmul m \NATgtr k$.
      Assume $n' \NATplus \NATone, m \NATgtr k$.
      Then $(n' \NATplus \NATone) \NATmul m \eq (n' \NATmul m) \NATplus m$.
      If $n' \eq \NATzero$ then
      $(n' \NATmul m) \NATplus m
        \eq \NATzero \NATplus m
        \eq m
        \NATgtr k$.
      If $n' \neq \NATzero$ then
      $(n' \NATmul m) \NATplus m
        \NATgtr m
        \NATgtr k$.
      Indeed if $n' \neq \NATzero$ then $n' \NATmul m \NATgtr \NATzero$.
      Indeed $m \NATgtr \NATzero$.
      Hence $(n' \NATplus \NATone) \NATmul m \NATgtr k$.
    End.
  \end{proof}

  Thus every natural number is contained in $\Phi$ (by \printref{ARITHMETIC_01_4764664342773760}).
  Therefore if $n, m \NATgtr k$ then $n \NATmul m \NATgtr k$.
\end{proof}

\begin{corollary}[forthel,id=ARITHMETIC_06_1751605544222720]
  Let $n, m, k$ be natural numbers.
  If $n \NATleq m$ then $k \NATmul n \NATleq k \NATmul m$.
\end{corollary}

\begin{corollary}[forthel,id=ARITHMETIC_06_3965209318260736]
  Let $n, m, k$ be natural numbers.
  Assume $k \neq \NATzero$.
  If $k \NATmul n \NATleq k \NATmul m$ then $n \NATleq m$.
\end{corollary}
\begin{proof}[forthel]
  If $k \NATmul n \eq k \NATmul m$ then $n \eq m$ (by \printref{ARITHMETIC_06_8575191374364672}).
  If $k \NATmul n \NATless k \NATmul m$ then $n \NATless m$ (by \printref{ARITHMETIC_06_5048640368279552}).
\end{proof}

\begin{corollary}[forthel,id=ARITHMETIC_06_8946886668976128]
  Let $n, m, k$ be natural numbers.
  If $n \NATleq m$ then $n \NATmul k \NATleq m \NATmul k$.
\end{corollary}

\begin{corollary}[forthel,id=ARITHMETIC_06_4374428949413888]
  Let $n, m, k$ be natural numbers.
  Assume $k \neq \NATzero$.
  If $n \NATmul k \NATleq m \NATmul k$ then $n \NATleq m$.
\end{corollary}

\begin{proposition}[forthel,id=ARITHMETIC_06_8813409145454592]
  Let $n, m, k$ be natural numbers.
  Assume $m \NATgtr \NATzero$ and $k \NATgtr \NATone$.
  Then $k \NATmul m \NATgtr m$.
\end{proposition}
\begin{proof}[forthel]
  Take a natural number $l$ such that $k \eq l \NATplus \NATtwo$.
  Then
  \[  k \NATmul m                       \]
  \[    \eq (l \NATplus \NATtwo) \NATmul m             \]
  \[    \eq (l \NATmul m) \NATplus (\NATtwo \NATmul m)   \]
  \[    \eq (l \NATmul m) \NATplus (m \NATplus m)       \]
  \[    \eq ((l \NATmul m) \NATplus m) \NATplus m       \]
  \[    \eq ((l \NATplus \NATone) \NATmul m) \NATplus m       \]
  \[    \NATgeq \NATone \NATplus m                    \]
  \[    \NATgtr m.                          \]
  Indeed $((l \NATplus \NATone) \NATmul m) \NATplus m \NATgeq \NATone \NATplus m$.
\end{proof}
\end{smodule}
\end{document}
