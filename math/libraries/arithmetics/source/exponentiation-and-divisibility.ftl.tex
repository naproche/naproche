\documentclass{naproche-library}
\libinput{preamble}
\begin{document}
\IfInputref{}{\setsectionlevel{section}}
\begin{smodule}[title=Exponentiation and Divisibility]{exponentiation-and-divisibility.ftl}
\importmodule[libraries/arithmetics]{exponentiation-and-ordering.ftl}
\importmodule[libraries/arithmetics]{divisibility.ftl}

\begin{proposition}[forthel,id=ARITHMETIC_13_8426075493236736]
  Let $n, m, k$ be natural numbers such that $n, m \neq 0$ and $k > 1$.
  Then $k^{n} \mid k^{m}$ iff $n \leq m$.
\end{proposition}
\begin{proof}[forthel]
  Case $k^{n} \mid k^{m}$.
    Assume $n > m$.
    Take a nonzero natural number $l$ such that $n = m + l$.
    Then $k^{n}
      = k^{m + l}
      = k^{m} \cdot k^{l}$.
    Hence $k^{m} \mid k^{n}$.
    Thus $k^{m} = k^{n}$.
    Therefore $m = n$.
    Contradiction.
  End.

  Case $n \leq m$.
    Take a natural number $l$ such that $m = n + l$.
    Then $k^{m}
      = k^{n + l}
      = k^{n} \cdot k^{l}$.
    Hence $k^{n} \mid k^{m}$.
  End.
\end{proof}

\begin{proposition}[forthel,id=ARITHMETIC_13_797196163219456]
  Let $n$ be a composite natural number.
  Then $n$ has a nontrivial divisor $m$ such that $m^{2} \leq n$.
\end{proposition}
\begin{proof}[forthel]
  Define $A = \{ m \in \Nat \mid m$ is a nontrivial divisor of $n \}$.
  $A$ is nonempty.
  Hence we can take a $m \in A$ such that $m \leq l$ for all $l \in A$.
  Consider a natural number $k$ such that $m \cdot k = n$.
  Then $m \leq k$.
  Hence $m^{2} = m \cdot m \leq m \cdot k = n$.
  Therefore $m^{2} \leq n$.
\end{proof}
\end{smodule}
\end{document}