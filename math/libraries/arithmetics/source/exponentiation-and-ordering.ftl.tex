\documentclass{naproche-library}
\libinput{preamble}
\begin{document}
\IfInputref{}{\setsectionlevel{section}}
\begin{smodule}[title=Exponentiation and Ordering]{exponentiation-and-ordering.ftl}
\importmodule[libraries/arithmetics]{multiplication-and-ordering.ftl}
\importmodule[libraries/arithmetics]{exponentiation.ftl}

\begin{proposition}[forthel,id=ARITHMETIC_09_3373702288769024]
  Let $n, m, k$ be natural numbers.
  Assume $k \neq \NATzero$.
  Then $n \NATless m$ iff $\NATexp{n}{k} \NATless \NATexp{m}{k}$.
\end{proposition}
\begin{proof}[forthel]
  \begin{case}{$n \NATless m$.}
    Define $\Phi = \{ k' \in \Nat \mid\text{ if }k' \NATgtr \NATone\text{ then }\NATexp{n}{k'} \NATless \NATexp{m}{k'} \}$.

    (1) $\Phi$ contains $\NATzero$.

    (2) $\Phi$ contains $\NATone$.

    (3) $\Phi$ contains $\NATtwo$.
    \begin{proof}
      \begin{case}{$n \eq \NATzero$ or $m \eq \NATzero$.} \end{case}

      \begin{case}{$n, m \neq \NATzero$.}
        Then $n \NATmul n
          \NATless n \NATmul m
          \NATless m \NATmul m$.
        Hence $\NATexp{n}{\NATtwo}
          \eq n \NATmul n
          \NATless n \NATmul m
          \NATless m \NATmul m
          \eq \NATexp{m}{\NATtwo}$.
      \end{case}
    \end{proof}

    (4) For all $k' \in \Phi$ we have $k' \NATplus \NATone \in \Phi$.
    \begin{proof}
      Let $k' \in \Phi$.

      Let us show that if $k' \NATplus \NATone \NATgtr \NATone$ then $\NATexp{n}{k' \NATplus \NATone} \NATless \NATexp{m}{k' \NATplus \NATone}$.
        Assume $k' \NATplus \NATone \NATgtr \NATone$.
        Then $\NATexp{n}{k'} \NATless \NATexp{m}{k'}$.
        Indeed $k' \neq \NATzero$ and $if k' \eq \NATone$ then $\NATexp{n}{k'} \NATless \NATexp{m}{k'}$.

        \begin{case}{$k' \NATleq \NATone$.}
          Then $k' \eq \NATzero$ or $k' \eq \NATone$.
          Hence $k' \NATplus \NATone \eq \NATone$ or $k' \NATplus \NATone \eq \NATtwo$.
          Thus $k' \NATplus \NATone \in \Phi$.
          Therefore $\NATexp{n}{k' \NATplus \NATone} \NATless \NATexp{m}{k' \NATplus \NATone}$.
        \end{case}

        \begin{case}{$k' \NATgtr \NATone$.}
          \begin{case}{$n \eq \NATzero$.}
            Then $m \neq \NATzero$.
            Hence $\NATexp{n}{k' \NATplus \NATone}
              \eq \NATzero
              \NATless \NATexp{m}{k'} \NATmul m
              \eq \NATexp{m}{k' \NATplus \NATone}$.
            Thus $\NATexp{n}{k' \NATplus \NATone} \NATless \NATexp{m}{k' \NATplus \NATone}$.
          \end{case}

          \begin{case}{$n \neq \NATzero$.}
            Then $\NATexp{n}{k'} \NATmul n
              \NATless \NATexp{m}{k'} \NATmul n
              \NATless \NATexp{m}{k'} \NATmul m$.
            Indeed $\NATexp{n}{k'} \NATless \NATexp{m}{k'} \neq \NATzero$.
            Take $A \eq \NATexp{n}{k' \NATplus \NATone}$ and $B \eq \NATexp{m}{k' \NATplus \NATone}$. %!
            Then $A
              \eq \NATexp{n}{k' \NATplus \NATone}
              \eq \NATexp{n}{k'} \NATmul n
              \NATless \NATexp{m}{k'} \NATmul n
              \NATless \NATexp{m}{k'} \NATmul m
              \eq \NATexp{m}{k' \NATplus \NATone}
              \eq B$.
            Take $X \eq \NATexp{m}{k'} \NATmul n$ and $Y \eq \NATexp{m}{k'} \NATmul m$.
            Then $A \NATless X \NATless Y \eq B$.
            Hence $A \NATless B$.
            Thus $\NATexp{n}{k' \NATplus \NATone} \NATless \NATexp{m}{k' \NATplus \NATone}$.
          \end{case}
        \end{case}

        Hence $\NATexp{n}{k' \NATplus \NATone} \NATless \NATexp{m}{k' \NATplus \NATone}$.
        Indeed $k' \NATleq \NATone$ or $k' \NATgtr \NATone$.
      End.

      Thus $k' \NATplus \NATone \in \Phi$.
    \end{proof}

    Therefore every natural number is contained in $\Phi$ (by \printref{ARITHMETIC_01_4764664342773760}).
    Consequently $\NATexp{n}{k} \NATless \NATexp{m}{k}$.
  \end{case}

  \begin{case}{$\NATexp{n}{k} \NATless \NATexp{m}{k}$.}
    Define $\Psi = \{ k' \in \Nat \mid\text{ if }n \NATgeq m\text{ then }\NATexp{n}{k'} \NATgeq \NATexp{m}{k'} \}$.

    (1) $\Psi$ contains $\NATzero$.

    (2) For all $k' \in \Psi$ we have $k' \NATplus \NATone \in \Psi$.
    \begin{proof}
      Let $k' \in \Psi$.

      Let us show that if $n \NATgeq m$ then $\NATexp{n}{k' \NATplus \NATone} \NATgeq \NATexp{m}{k' \NATplus \NATone}$.
        Assume $n \NATgeq m$.
        Then $\NATexp{n}{k'} \NATgeq \NATexp{m}{k'}$.
        Hence $\NATexp{n}{k'} \NATmul n \NATgeq \NATexp{m}{k'} \NATmul n \NATgeq \NATexp{m}{k'} \NATmul m$.
        Take $A \eq \NATexp{n}{k' \NATplus \NATone}$ and $B \eq \NATexp{m}{k' \NATplus \NATone}$. %!
        Thus $A
          \eq \NATexp{n}{k' \NATplus \NATone}
          \eq \NATexp{n}{k'} \NATmul n
          \NATgeq \NATexp{m}{k'} \NATmul n
          \NATgeq \NATexp{m}{k'} \NATmul m
          \eq \NATexp{m}{k' \NATplus \NATone}
          \eq B$.
        Therefore $\NATexp{n}{k' \NATplus \NATone} \eq A \NATgeq B \eq \NATexp{m}{k' \NATplus \NATone}$.
      End.

      Hence $k' \NATplus \NATone \in \Psi$.
    \end{proof}

    Thus every natural number is contained in $\Psi$ (by \printref{ARITHMETIC_01_4764664342773760}).
    Therefore if $n \NATgeq m$ then $\NATexp{n}{k} \NATgeq \NATexp{m}{k}$.
    Consequently $n \NATless m$.
  \end{case}
\end{proof}

\begin{corollary}[forthel,id=ARITHMETIC_09_2797602550579200]
  Let $n, m, k$ be natural numbers.
  Assume $k \neq \NATzero$.
  Then \[ \NATexp{n}{k} \eq \NATexp{m}{k} \implies n \eq m. \]
\end{corollary}
\begin{proof}[forthel]
  Assume $\NATexp{n}{k} \eq \NATexp{m}{k}$.
  Suppose $n \neq m$.
  Then $n \NATless m$ or $m \NATless n$.
  If $n \NATless m$ then $\NATexp{n}{k} \NATless \NATexp{m}{k}$.
  If $m \NATless n$ then $\NATexp{m}{k} \NATless \NATexp{n}{k}$.
  Thus $\NATexp{n}{k} \neq \NATexp{m}{k}$.
  Contradiction.
\end{proof}

\begin{corollary}[forthel,id=ARITHMETIC_09_6875081963732992]
  Let $n, m, k$ be natural numbers.
  Assume $k \neq \NATzero$.
  Then \[ \NATexp{n}{k} \NATleq \NATexp{m}{k} \iff n \NATleq m. \]
\end{corollary}
\begin{proof}[forthel]
  If $\NATexp{n}{k} \NATless \NATexp{m}{k}$ then $n \NATless m$.
  If $\NATexp{n}{k} \eq \NATexp{m}{k}$ then $n \eq m$.

  If $n \NATless m$ then $\NATexp{n}{k} \NATless \NATexp{m}{k}$.
  If $n \eq m$ then $\NATexp{n}{k} \eq \NATexp{m}{k}$.
\end{proof}

\begin{proposition}[forthel,id=ARITHMETIC_09_3349764703780864]
  Let $n, m, k$ be natural numbers.
  Assume $k \NATgtr \NATone$.
  Then \[ n \NATless m \iff \NATexp{k}{n} \NATless \NATexp{k}{m}. \]
\end{proposition}
\begin{proof}[forthel]
  \begin{case}{$n \NATless m$.}
    Define $\Phi = \{ m' \in \Nat \mid\text{ if }n \NATless m'\text{ then }\NATexp{k}{n} \NATless \NATexp{k}{m'} \}$.

    (1) $\Phi$ contains $\NATzero$.

    (2) For all $m' \in \Phi$ we have $m' \NATplus \NATone \in \Phi$.
    \begin{proof}
      Let $m' \in \Phi$.

      Let us show that if $n \NATless m' \NATplus \NATone$ then $\NATexp{k}{n} \NATless \NATexp{k}{m' \NATplus \NATone}$.
        Assume $n \NATless m' \NATplus \NATone$.
        Then $n \NATleq m'$.
        We have $\NATexp{k}{m'} \NATmul \NATone \NATless \NATexp{k}{m'} \NATmul k$.
        Indeed $\NATexp{k}{m'} \neq \NATzero$.

        \begin{case}{$n \eq m'$.}
          Take $A \eq \NATexp{k}{n}$ and $B \eq \NATexp{k}{m' \NATplus \NATone}$. %!
          Then $A
            \eq \NATexp{k}{n}
            \eq \NATexp{k}{m'}
            \NATless \NATexp{k}{m'} \NATmul k
            \eq \NATexp{k}{m' \NATplus \NATone}
            \eq B$.
          Hence $\NATexp{k}{n} \eq A \NATless B \eq \NATexp{k}{m' \NATplus \NATone}$.
        \end{case}

        \begin{case}{$n \NATless m'$.}
          Take $A \eq \NATexp{k}{n}$ and $B \eq \NATexp{k}{m' \NATplus \NATone}$. %!
          Then $A
            \eq \NATexp{k}{n}
            \NATless \NATexp{k}{m'}
            \NATless \NATexp{k}{m'} \NATmul k
            \eq \NATexp{k}{m' \NATplus \NATone}
            \eq B$.
          Hence $\NATexp{k}{n} \eq A \NATless B \eq \NATexp{k}{m' \NATplus \NATone}$.
        \end{case}
      End.
    \end{proof}

    Hence every natural number is contained in $\Phi$ (by \printref{ARITHMETIC_01_4764664342773760}).
    Thus $\NATexp{k}{n} \NATless \NATexp{k}{m}$.
  \end{case}

  \begin{case}{$\NATexp{k}{n} \NATless \NATexp{k}{m}$.}
    Define $\Psi = \{ n' \in \Nat \mid\text{ if }n' \NATgeq m\text{ then }
    \NATexp{k}{n'} \NATgeq \NATexp{k}{m} \}$.

    (1) $\NATzero$ is contained in $\Psi$.

    (2) For all $n' \in \Psi$ we have $n' \NATplus \NATone \in \Psi$.
    \begin{proof}
      Let $n' \in \Psi$.

      Let us show that if $n' \NATplus \NATone \NATgeq m$ then $\NATexp{k}{n' \NATplus \NATone} \NATgeq \NATexp{k}{m}$.
        Assume $n' \NATplus \NATone \NATgeq m$.

        \begin{case}{$n' \NATplus \NATone \eq m$.} \end{case}

        \begin{case}{$n' \NATplus \NATone \NATgtr m$.}
          Then $n' \NATgeq m$.
          Hence $\NATexp{k}{n'} \NATgeq \NATexp{k}{m}$.
          We have $\NATexp{k}{n'} \NATmul \NATone \NATleq \NATexp{k}{n'} \NATmul k$.
          Indeed $\NATone \NATleq k$ and $\NATexp{k}{n'} \neq \NATzero$.
          Take $A \eq \NATexp{k}{m}$ and $B \eq \NATexp{k}{n' \NATplus \NATone}$. %!
          Then $A
            \eq \NATexp{k}{m}
            \NATleq \NATexp{k}{n'}
            \eq \NATexp{k}{n'} \NATmul \NATone
            \NATleq \NATexp{k}{n'} \NATmul k
            \eq \NATexp{k}{n' \NATplus \NATone}
            \eq B$.
          Hence $\NATexp{k}{m} \eq A \NATleq B \eq \NATexp{k}{n' \NATplus \NATone}$.
        \end{case}
      End.
    \end{proof}

    Thus every natural number is contained in $\Psi$ (by \printref{ARITHMETIC_01_4764664342773760}).
    Therefore if $n \NATgeq m$ then $\NATexp{k}{n} \NATgeq \NATexp{k}{m}$.
    Consequently $n \NATless m$.
  \end{case}
\end{proof}

\begin{corollary}[forthel,id=ARITHMETIC_09_6780506905509888]
  Let $n, m, k$ be natural numbers.
  Assume $k \NATgtr \NATone$.
  If $\NATexp{k}{n} \eq \NATexp{k}{m}$ then $n \eq m$.
\end{corollary}
\begin{proof}[forthel]
  Assume $\NATexp{k}{n} \eq \NATexp{k}{m}$.
  Suppose $n \neq m$.
  Then $n \NATless m$ or $m \NATless n$.
  If $n \NATless m$ then $\NATexp{k}{n} \NATless \NATexp{k}{m}$.
  If $m \NATless n$ then $\NATexp{k}{m} \NATless \NATexp{k}{n}$.
  Thus $\NATexp{k}{n} \neq \NATexp{k}{m}$.
  Contradiction.
\end{proof}

\begin{corollary}[forthel,id=ARITHMETIC_09_2876620253691904]
  Let $n, m, k$ be natural numbers.
  Assume $k \NATgtr \NATone$.
  Then $n \NATleq m$ iff $\NATexp{k}{n} \NATleq \NATexp{k}{m}$.
\end{corollary}
\begin{proof}[forthel]
  If $n \NATleq m$ then $\NATexp{k}{n} \NATleq \NATexp{k}{m}$.

  If $\NATexp{k}{n} \eq \NATexp{k}{m}$ then $n \eq m$.
  If $\NATexp{k}{n} \NATless \NATexp{k}{m}$ then $n \NATless m$.
\end{proof}

\begin{proposition}[forthel,id=ARITHMETIC_09_6984104377581568]
  Let $n$ be a natural number.
  Then $\NATexp{n \NATplus \NATone}{\NATtwo} \eq (\NATexp{n}{\NATtwo} \NATplus (\NATtwo \NATmul n)) \NATplus \NATone$.
\end{proposition}
\begin{proof}[forthel]
  We have
  \[  \NATexp{n \NATplus \NATone}{\NATtwo}                       \]
  \[    \eq (n \NATplus \NATone) \NATmul (n \NATplus \NATone)         \]
  \[    \eq ((n \NATplus \NATone) \NATmul n) \NATplus (n \NATplus \NATone)   \]
  \[    \eq ((n \NATmul n) \NATplus n) \NATplus (n \NATplus \NATone)   \]
  \[    \eq (\NATexp{n}{\NATtwo} \NATplus n) \NATplus (n \NATplus \NATone)         \]
  \[    \eq ((\NATexp{n}{\NATtwo} \NATplus n) \NATplus n) \NATplus \NATone         \]
  \[    \eq (\NATexp{n}{\NATtwo} \NATplus (n \NATplus n)) \NATplus \NATone         \]
  \[    \eq (\NATexp{n}{\NATtwo} \NATplus (\NATtwo \NATmul n)) \NATplus \NATone.    \]
\end{proof}

\begin{proposition}[forthel,id=ARITHMETIC_09_134060414337024]
  Let $n$ be a natural number.
  Assume $n \NATgeq \NATthree$.
  Then $\NATexp{n}{\NATtwo} \NATgtr (\NATtwo \NATmul n) \NATplus \NATone$.
\end{proposition}
\begin{proof}[forthel]
  (a) Define $\Phi = \{ n' \in \Natgeqthan{\NATthree} \mid \NATexp{n'}{\NATtwo} \NATgtr (\NATtwo \NATmul n') \NATplus \NATone \}$.

  (1) $\Phi$ contains $\NATthree$.
  Indeed $\NATexp{\NATthree}{\NATtwo} \NATgtr (\NATtwo \NATmul \NATthree) \NATplus \NATone$.

  (2) For all $n' \in \Phi$ we have $n' \NATplus \NATone \in \Phi$.
  \begin{proof}
    Let $n' \in \Phi$.
    Then $n' \NATgeq \NATthree$.

    Take $V \eq (\NATexp{n'}{\NATtwo} \NATplus (\NATtwo \NATmul n')) \NATplus \NATone$ and $W \eq (((\NATtwo \NATmul n') \NATplus \NATone) \NATplus (\NATtwo \NATmul n')) \NATplus \NATone$.
    Then $V \NATgtr W$.
    \begin{proof}
      We have $\NATexp{n'}{\NATtwo} \NATgtr (\NATtwo \NATmul n') \NATplus \NATone$.
      Hence $\NATexp{n'}{\NATtwo} \NATplus (\NATtwo \NATmul n') \NATgtr ((\NATtwo \NATmul n') \NATplus \NATone) \NATplus (\NATtwo \NATmul n')$ (by \printref{ARITHMETIC_04_7354062662008832}).
    \end{proof}

    Take $X \eq ((\NATtwo \NATmul n') \NATplus (\NATtwo \NATmul n')) \NATplus \NATone$.
    Then $W \NATgtr X$.
    \begin{proof}
      We have $((\NATtwo \NATmul n') \NATplus \NATone) \NATplus (\NATtwo \NATmul n') \NATgtr (\NATtwo \NATmul n') \NATplus (\NATtwo \NATmul n')$.
      Indeed $(\NATtwo \NATmul n') \NATplus \NATone \NATgtr \NATtwo \NATmul n'$.
    \end{proof}

    Take $Y \eq (\NATtwo \NATmul (n' \NATplus n')) \NATplus \NATone$ and $Z \eq (\NATtwo \NATmul (n' \NATplus \NATone)) \NATplus \NATone$.
    Then $Y \NATgtr Z$.
    \begin{proof}
      We have $n' \NATplus n' \NATgtr n' \NATplus \NATone$ and $\NATtwo \neq \NATzero$.
      Thus $\NATtwo \NATmul (n' \NATplus n') \NATgtr \NATtwo \NATmul (n' \NATplus \NATone)$ (by \printref{ARITHMETIC_06_5048640368279552}).
      Indeed $n' \NATplus n'$ and $n' \NATplus \NATone$ are natural numbers.
    \end{proof}

    Then $\NATexp{n' \NATplus \NATone}{\NATtwo} \eq V \NATgtr W \NATgtr X \eq Y \NATgtr Z \eq (\NATtwo \NATmul (n' \NATplus \NATone)) \NATplus \NATone$.
    Hence $\NATexp{n' \NATplus \NATone}{\NATtwo} \NATgtr (\NATtwo \NATmul (n' \NATplus \NATone)) \NATplus \NATone$.
  \end{proof}

  Therefore $\Phi$ contains every element of $\Natgeqthan{\NATthree}$ (by \printref{ARITHMETIC_04_4976599269113856}, 1, 2).
  Consequently $\NATexp{n}{\NATtwo} \NATgtr (\NATtwo \NATmul n) \NATplus \NATone$ (by a).
\end{proof}

\begin{proposition}[forthel,id=ARITHMETIC_09_1097950367318016]
  Let $n$ be a natural number.
  Assume $n \NATgeq \NATfive$.
  Then $\NATexp{\NATtwo}{n} \NATgtr \NATexp{n}{\NATtwo}$.
\end{proposition}
\begin{proof}[forthel]
  (a) Define $\Phi = \{ n' \in \Natgeqthan{\NATfive} \mid \NATexp{\NATtwo}{n'} \NATgtr \NATexp{n'}{\NATtwo} \}$.

  (1) $\Phi$ contains $\NATfive$.
  Indeed we can show that $\NATexp{\NATtwo}{\NATfive} \NATgtr \NATexp{\NATfive}{\NATtwo}$.
    We have $\NATexp{\NATtwo}{\NATfive}
      \eq \NATtwo \NATmul (\NATtwo \NATmul (\NATtwo \NATmul (\NATtwo \NATmul \NATtwo)))
      \eq \NATeight \NATmul \NATfour
      \eq (\NATfive \NATplus \NATthree) \NATmul \NATfour
      \eq (\NATfive \NATmul \NATfour) \NATplus (\NATthree \NATmul \NATfour)
      \eq (\NATfive \NATmul \NATfour) \NATplus ((\NATfive \NATmul \NATtwo) \NATplus \NATtwo)
      \eq (\NATfive \NATmul (\NATfour \NATplus \NATtwo)) \NATplus \NATtwo
      \eq (\NATfive \NATmul \NATsix) \NATplus \NATtwo
      \NATgtr \NATfive \NATmul \NATsix
      \NATgtr \NATfive \NATmul \NATfive
      \eq \NATexp{\NATfive}{\NATtwo}$.
    Indeed $((\NATfive \NATmul \NATfour) \NATplus ((\NATfive \NATmul \NATtwo) \NATplus \NATtwo)) \eq (\NATfive \NATmul (\NATfour \NATplus \NATtwo)) \NATplus \NATtwo$. %!
  End.

  (2) For all $n' \in \Phi$ we have $n' \NATplus \NATone \in \Phi$.
  \begin{proof}
    Let $n' \in \Phi$.
    Then $n' \NATgeq \NATfive$ and $\NATexp{\NATtwo}{n'} \NATgtr \NATexp{n'}{\NATtwo}$.

    Take $V \eq \NATexp{\NATtwo}{n'} \NATmul \NATtwo$ and $W \eq \NATexp{n'}{\NATtwo} \NATmul \NATtwo$.
    Then $V \NATgtr W$.
    Indeed $\NATtwo \neq \NATzero$.

    Take $X \eq \NATexp{n'}{\NATtwo} \NATplus \NATexp{n'}{\NATtwo}$.
    Then $W \eq X$.

    Take $Y \eq \NATexp{n'}{\NATtwo} \NATplus ((\NATtwo \NATmul n') \NATplus \NATone)$.
    Then $X \NATgtr Y$.
    Indeed $\NATexp{n'}{\NATtwo} \NATgtr (\NATtwo \NATmul n') \NATplus \NATone$.

    Take $Z \eq \NATexp{n' \NATplus \NATone}{\NATtwo}$.
    Then $Y \eq Z$.

    Then $\NATexp{\NATtwo}{n' \NATplus \NATone} \eq V \NATgtr W \eq X \NATgtr Y \eq Z \eq \NATexp{n' \NATplus \NATone}{\NATtwo}$.
    Hence $\NATexp{\NATtwo}{n' \NATplus \NATone} \NATgtr \NATexp{n' \NATplus \NATone}{\NATtwo}$.
  \end{proof}

  Therefore $\Phi$ contains every element of $\Natgeqthan{\NATfive}$ (by \printref{ARITHMETIC_04_4976599269113856}).
  Consequently $\NATexp{\NATtwo}{n} \NATgtr \NATexp{n}{\NATtwo}$ (by a).
\end{proof}
\end{smodule}
\end{document}
