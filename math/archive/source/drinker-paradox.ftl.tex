\documentclass{article}
\usepackage{naproche}
\libinput{preamble}

\title{The Drinker Paradox in \Naproche}
\author{Marcel Schütz}
\date{2024--2025}

\begin{document}
\begin{smodule}{drinker-paradox.ftl}
\maketitle

\importmodule[libraries/everyday-ontology]{classes?person.ftl}
\importmodule[libraries/everyday-ontology]{classes?place.ftl}
\importmodule[libraries/everyday-ontology]{properties?inside.ftl}
\importmodule[libraries/everyday-ontology]{properties?drinking.ftl}

\noindent The \emph{Drinker Paradox} is a principle of classical predicate 
logic popularised by the logician Raymond Smullyan in his 1978 book
\textit{What Is the Name of this Book?} \cite{Smullyan1978} which can be 
stated as:

\begin{quotation}
  \noindent There is someone in the pub such that, if he is drinking then 
  everyone in the pub is drinking.
\end{quotation}

\begin{forthel}

  \begin{signature*}
    The pub is a place.
  \end{signature*}
  
  \begin{theorem*}[title=Drinker Paradox,id=drinker_paradox]
    Assume that there is a person inside the pub.
    Then there is a person $P$ inside the pub such that if $P$ is drinking then every person inside the pub is drinking.
  \end{theorem*}
  \begin{proof}[method=case analysis]
    \begin{case}{every person inside the pub is drinking.}
      Choose a person $P$ inside the pub.
      Then $P$ is drinking and every person inside the pub is drinking.
      Hence if $P$ is drinking then every person inside the pub is drinking.
    \end{case}
  
    \begin{case}{there is a person inside the pub that is not drinking.}
      Consider a person $P$ inside the pub that is not drinking.
      Then if $P$ is drinking then every person inside the pub is drinking.
    \end{case}
  \end{proof}
\end{forthel}

\printbibliography
\printlicense[CcByNcSa]{2024--2025}{Marcel Schütz}
\end{smodule}
\end{document}
