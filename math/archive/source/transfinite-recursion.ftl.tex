\documentclass{article}
\usepackage{naproche}
\libinput{preamble}

\title{Transfinite Recursion Theorem}
\author{\Naproche formalization:\\[0.5em]Marcel Schütz}
\date{2024}

\begin{document}
\begin{smodule}{transfinite-recursion.ftl}
\maketitle

\importmodule[libraries/set-theory]{transfinite-induction.ftl}
\importmodule[libraries/set-theory]{recursive-maps.ftl}

\noindent This is a formalization of the \emph{Transfinite Recursion Theorem}
(cf. \cite{Koepke2018}).
It states that for any map $G \FUNfromto{\ORDfunspace{A}}{A}$, where
$\ORDfunspace{A}$ denotes the class of all maps $\alpha \to A$ for some
ordinal $\alpha$, there exists a unique map $F \FUNfromto{\Ord}{A}$ that is
\emph{recursive regarding} $G$, i.e. \[F(\alpha) \eq G(F \FUNrest \alpha)\] for
all ordinals $\alpha$.

\begin{lemma}[forthel,title=Coincidence Lemma,id=transfinite_recursion_coincidence]
  Let $A$ be a class and $G$ be a map from $\ORDfunspace{A}$ to $A$.
  Let $H, H'$ be maps to $A$ that are recursive regarding $G$.
  Then \[ H(\alpha) \eq H'(\alpha) \] for all $\alpha \SETin \FUNdom(H) \SETintersect \FUNdom(H')$.
\end{lemma}
\begin{proof}[forthel]
  Define $\Phi = \{ \alpha \in \Ord \mid\text{ if }
  \alpha \SETin \FUNdom(H) \SETintersect \FUNdom(H')\text{ then }H(\alpha) \eq H'(\alpha) \}$.

  For all ordinals $\alpha$ if every ordinal less than $\alpha$ lies in $\Phi$ then $\alpha \SETin \Phi$.
  \begin{proof}
    Let $\alpha \SETin \Ord$.
    Assume that every $y \SETin \alpha$ lies in $\Phi$.

    Let us show that if $\alpha \SETin \FUNdom(H) \SETintersect \FUNdom(H')$ then
    $H(\alpha) \eq H'(\alpha)$.
      Suppose $\alpha \SETin \FUNdom(H) \SETintersect \FUNdom(H')$.
      Then $\alpha \SETinclude \FUNdom(H), \FUNdom(H')$.
      Indeed $\FUNdom(H)$ and $\FUNdom(H')$ are transitive classes.
      Hence for all $y \SETin \alpha$ we have $y \SETin \FUNdom(H) \SETintersect \FUNdom(H')$.
      Thus $H(y) \eq H'(y)$ for all $y \SETin \alpha$.
      Therefore $H \FUNrest \alpha \eq H' \FUNrest \alpha$.
      $H$ and $H'$ are recursive regarding $G$.
      We have $H \FUNrest \alpha, H' \FUNrest \alpha \SETin \ORDfunspace{A}$.
      Hence $H(\alpha)
        \eq G(H \FUNrest \alpha)
        \eq G(H' \FUNrest \alpha)
        \eq H'(\alpha)$.
    End.

    Thus $\alpha \SETin \Phi$.
  \end{proof}

  [prover vampire]
  Then $\Phi$ contains every ordinal (by \printref{SET_THEORY_02_8493935460614144}).
  Therefore we have $H(\alpha) \eq H'(\alpha)$ for all $\alpha \SETin \FUNdom(H) \SETintersect \FUNdom(H')$.
\end{proof}

\begin{theorem}[forthel,title=Transfinite Recursion Theorem: Existence,id=transfinite_recursion_existence]
  Let $A$ be a class and $G$ be a map from $\ORDfunspace{A}$ to $A$.
  Then there exists a map $F$ from $\Ord$ to $A$ that is recursive regarding $G$.
\end{theorem}
\begin{proof}[forthel]
  Every ordinal is contained in the domain of some map $H$ to $A$ such that $H$ is recursive regarding $G$.
  \begin{proof}
    Define \[ \Phi = \class{\alpha \in \Ord | \classtext{$\alpha$ is contained in the domain of some map to $A$ that is recursive regarding $G$}}. \]

    Let us show that for every ordinal $\alpha$ if every ordinal less than $\alpha$ lies in $\Phi$ then $\alpha \SETin \Phi$.
      Let $\alpha$ be an ordinal.
      Assume that every ordinal less than $\alpha$ lies in $\Phi$.
      Then for all $y \SETin \alpha$ there exists a map $h$ to $A$ such that $h$ is recursive regarding $G$ and $y \SETin \FUNdom(h)$.
      Define $H'(y) =$ ``choose a map $h$ to $A$ such that $h$ is recursive regarding $G$ and $y \SETin \FUNdom(h)$ in $h(y)$'' for $y \in \alpha$.
      Then $H'$ is a map from $\alpha$ to $A$.
      We have $H' \eq H' \FUNrest \alpha$.
      Define \[ H(\beta) =
        \begin{cases}
          H'(\beta)                 & : \beta \ORDless \alpha \\
          G(H' \FUNrest \beta)  & : \beta \eq \alpha
        \end{cases} \]
      for $\beta \in \ORDsucc(\alpha)$.
      
      Let us show that $H \FUNrest \beta \SETin \ORDfunspace{A}$ for all $\beta \SETin \FUNdom(H)$.
        Let $\beta \SETin \FUNdom(H)$.
        $\FUNdom(H \FUNrest \beta) \eq \beta$ and $(H \FUNrest \beta)(b) \eq H(b)$ for all $b \SETin \beta$.
        $H(b) \SETin A$ for all $b \SETin \beta$.
        Hence $H \FUNrest \beta$ is a map from $\beta$ to $A$.
      End.

      (a) $\FUNdom(H)$ is a transitive subclass of $\Ord$.

      (b) For all $\beta \SETin \FUNdom(H)$ we have $H(\beta) \eq G(H \FUNrest \beta)$.
      \begin{proof}
        Let $\beta \SETin \FUNdom(H)$.
        Then $\beta \ORDless \alpha$ or $\beta \eq \alpha$.

        \begin{case}{$\beta \ORDless \alpha$.}
          Choose a map $h$ to $A$ such that $h$ is recursive regarding $G$ and $\beta \SETin \FUNdom(h)$ and $H'(\beta) \eq h(\beta)$.

          Let us show that for all $y \SETin \beta$ we have $h(y) \eq H(y)$.
            Let $y \SETin \beta$.
            Then $H(y) \eq H'(y)$.
            Choose a map $h'$ to $A$ such that $h'$ is recursive regarding $G$ and $y \SETin \FUNdom(h')$ and $H'(y) \eq h'(y)$.
            [prover vampire]
            Then $h'(y) \eq h(y)$ (by \nameref{transfinite_recursion_coincidence}).
            Indeed $y \SETin \FUNdom(h) \SETintersect \FUNdom(h')$.
          End.

          Hence $h \FUNrest \beta \eq H \FUNrest \beta$.
          Thus $H(\beta)
            \eq H'(\beta)
            \eq h(\beta)
            \eq G(h \FUNrest \beta)
            \eq G(H \FUNrest \beta)$.
        \end{case}

        \begin{case}{$\beta \eq \alpha$.}
          We have $H \FUNrest \alpha \eq H' \FUNrest \alpha$.
        \end{case}
      \end{proof}

      Hence $H$ is a map to $A$ such that $H$ is recursive regarding $G$ and $\alpha \SETin \FUNdom(H)$.
      Thus $\alpha \SETin \Phi$.
    End.

    [prover vampire]
    Therefore $\Phi$ contains every ordinal (by \printref{SET_THEORY_02_8493935460614144}).
    Consequently every ordinal is contained in the domain of some map $H$ to $A$ such that $H$ is recursive regarding $G$.
  \end{proof}

  Define $F(\alpha) =$ ``choose a map $H$ to $A$ such that $H$ is recursive regarding $G$ and $\alpha \SETin \FUNdom(H)$ in $H(\alpha)$'' for $\alpha \in \Ord$.
  Then $F$ is a map from $\Ord$ to $A$.

  $F$ is recursive regarding $G$.
  \begin{proof}
    (a) $\FUNdom(F)$ is a transitive subclass of $\Ord$.

    (b) For all $\alpha \SETin \Ord$ we have $F(\alpha) \eq G(F \FUNrest \alpha)$.
    \begin{proof}
      Let $\alpha \SETin \Ord$.
      Choose a map $H$ to $A$ such that $H$ is recursive regarding $G$ and $\alpha \SETin \FUNdom(H)$ and $F(\alpha) \eq H(\alpha)$.

      Let us show that $F(\beta) \eq H(\beta)$ for all $\beta \SETin \alpha$.
        Let $\beta \SETin \alpha$.
        Choose a map $H'$ to $A$ such that $H'$ is recursive regarding $G$ and $\beta \SETin \FUNdom(H')$ and $F(\beta) \eq H'(\beta)$.
        [prover vampire]
        Then $H(\beta) \eq H'(\beta)$ (by \nameref{transfinite_recursion_coincidence}).
        Indeed $\beta \SETin \FUNdom(H) \SETintersect \FUNdom(H')$.
        Therefore $F(\beta) \eq H'(\beta)$.
      End.

      Hence $H \FUNrest \alpha \eq F \FUNrest \alpha$.
      Thus $F(\alpha)
        \eq H(\alpha)
        \eq G(H \FUNrest \alpha)
        \eq G(F \FUNrest \alpha)$.
    \end{proof}
  \end{proof}
\end{proof}

\begin{theorem}[forthel,title=Transfinite Recursion Theorem: Uniqueness,id=transfinite_recursion_uniqueness]
  Let $A$ be a class and $G$ be a map from $\ORDfunspace{A}$ to $A$.
  Let $F, F'$ be maps from $\Ord$ to $A$ that are recursive regarding $G$.
  Then $F \eq F'$.
\end{theorem}
\begin{proof}[forthel]
  [prover vampire]
  $F$ and $F'$ are recursive regarding $G$.
  Then $F(\alpha) \eq F'(\alpha)$ for all $\alpha \SETin \FUNdom(F) \SETintersect \FUNdom(F')$ (by \nameref{transfinite_recursion_coincidence}).
  We have $\FUNdom(F) \eq \Ord \eq \FUNdom(F')$.
  Hence $F(\alpha) \eq F'(\alpha)$ for all $\alpha \SETin \Ord$.
  Thus $F \eq F'$.
\end{proof}

As a corollary of the transfinite recursion theorem we get that we can
define maps recursively on the ordinals by case distinction:
For given maps $G \FUNfromto{\Ord \SETprod A}{A}$ and
$H \FUNfromto{\Ord \SETprod \ORDfunspace{A}}{A}$ and an element $a \in A$ we
can define a map $F \FUNfromto{\Ord}{A}$ by
\begin{itemize}
  \item $F(\ORDzero) = a$,
  \item $F(\ORDsucc(\alpha)) = G(\alpha, F(\alpha))$, and
  \item $F(\lambda) = H(\lambda, F \FUNrest \lambda)$
    for any limit ordinal $\lambda$.
\end{itemize}

\begin{lemma}[forthel,id=transfinite_recursion_029348573987483]
  Let $A$ be a class and $\alpha$ be an ordinal and $F \FUNfromto{\Ord}{A}$.
  Then $(\alpha, F(\alpha)) \SETin \Ord \SETprod A$.
\end{lemma}

\begin{lemma}[forthel,id=transfinite_recursion_570293857449281]
  Let $A$ be a class and $\lambda$ be a limit ordinal and $F \FUNfromto{\Ord}{A}$.
  Then $(\lambda, F \FUNrest \lambda) \SETin \Ord \SETprod \ORDfunspace{A}$.
\end{lemma}

\begin{corollary}[forthel,id=transfinite_recursion_298647309184593]
  Let $A$ be a class.
  Let $a \SETin A$ and $G \FUNfromto{\Ord \SETprod A}{A}$ and $H \FUNfromto{\Ord \SETprod \ORDfunspace{A}}{A}$.
  Then there exists a map $F$ from $\Ord$ to $A$ such that
  \[ F(\ORDzero) \eq a \]
  and for all ordinals $\alpha$ we have
  \[ F(\ORDsucc(\alpha)) \eq G(\alpha, F(\alpha)) \]
  and for all limit ordinals $\lambda$ we have
  \[ F(\lambda) \eq H(\lambda, F \FUNrest \lambda). \]
\end{corollary}
\begin{proof}[forthel]
  $(\ORDpred(\FUNdom(f)), f(\ORDpred(\FUNdom(f)))) \SETin \Ord \SETprod A$ for all $f \SETin \ORDfunspace{A}$ such that $\FUNdom(f)$ is a successor ordinal.

  Define  \[ J(f) =
    \begin{cases}
      a
      & : \FUNdom(f) \eq \ORDzero
      \\
      G(\ORDpred(\FUNdom(f)), f(\ORDpred(\FUNdom(f))))
      & : \text{$\FUNdom(f)$ is a successor ordinal}
      \\
      H(\FUNdom(f), f)
      & : \text{$\FUNdom(f)$ is a limit ordinal}
    \end{cases} \]
  for $f \in \ORDfunspace{A}$.

  Then $J$ is a map from $\ORDfunspace{A}$ to $A$.
  Indeed we can show that for any $f \SETin \ORDfunspace{A}$ we have $J(f) \SETin A$.
    Let $f \SETin \ORDfunspace{A}$.
    Take $\alpha \SETin \Ord$ such that $f \FUNfromto{\alpha}{A}$.
    If $\alpha \eq \ORDzero$ then $J(f) \eq a \SETin A$.
    If $\alpha$ is a successor ordinal then $J(f) \eq
    G(\ORDpred(\alpha), f(\ORDpred(\alpha))) \SETin A$.
    [prover vampire]
    If $\alpha$ is a limit ordinal then $J(f) \eq H(\alpha, f) \SETin A$.
  End.

  Hence we can take a map $F$ from $\Ord$ to $A$ that is recursive regarding $J$.
  Then $F \FUNrest \alpha \SETin \ORDfunspace{A}$ for any ordinal $\alpha$.

  (1) $F(\ORDzero) \eq a$.
  \begin{proof}
    $F(\ORDzero)
      \eq J(F \FUNrest \ORDzero)
      \eq a$.
  \end{proof}

  (2) $F(\ORDsucc(\alpha)) \eq G(\alpha, F(\alpha))$ for all ordinals $\alpha$.
  \begin{proof}
    Let $\alpha$ be an ordinal.
    Then $F(\ORDsucc(\alpha))
      \eq J(F \FUNrest \ORDsucc(\alpha))
      \eq G(\ORDpred(\ORDsucc(\alpha)), (F \FUNrest \ORDsucc(\alpha))(\ORDpred(\ORDsucc(\alpha))))
      \eq G(\alpha, (F \FUNrest \ORDsucc(\alpha))(\alpha))
      \eq G(\alpha, F(\alpha))$.
  \end{proof}

  (3) $F(\lambda) \eq H(\lambda, F \FUNrest \lambda)$ for all limit ordinals $\lambda$.
  \begin{proof}
    Let $\lambda$ be a limit ordinal.
    Then $F(\lambda)
      \eq J(F \FUNrest \lambda)
      \eq H(\lambda, F \FUNrest \lambda)$.
  \end{proof}

  Hence the thesis (by 1, 2, 3).
\end{proof}

\printbibliography
\printlicense[CcByNcSa]{2024}{Marcel Schütz}
\end{smodule}
\end{document}
