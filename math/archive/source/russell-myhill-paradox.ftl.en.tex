\documentclass{article}
\usepackage{naproche}
\libinput{preamble}

\title{The Russell-Myhill Paradox}
\author{Marcel Schütz}
\date{2024--2025}

\begin{document}
\begin{smodule}{russell-myhill-paradox.ftl}
\maketitle

\importmodule[libraries/preliminaries]{preliminaries.ftl}

\symdef{logprod}{\mathop{\comp\sqcap}}

\noindent The \emph{Russell-Myhill Paradox} is a paradox discovered by 
Bertrand Russell which states that two systems of propositions with the same 
\emph{logical product} are not necessarily equal.
It was discussed in Russell's 1903 \emph{Principles of Mathematics}
\cite[Appendix B]{Russell1903} and rediscovered in 1958 by John Myhill
\cite{Myhill1958}.

\begin{forthel}
  \begin{signature*}
    A proposition is an object.
  \end{signature*}

  \begin{signature*}
    Let $p$ be a proposition.
    $p$ is true is an atom.
  \end{signature*}

  \begin{definition*}
    A system of propositions is a class $P$ such that every element of $P$ is a proposition.
  \end{definition*}

  \begin{signature*}
    Let $P$ be a system of propositions.
    The \emph{logical product of $P$} is a proposition $p$ such that $p$ is true iff every element of $P$ is true.
  \end{signature*}
  
  Let $\emph{\logprod P}$ denote the logical product of $P$.
  
  \begin{theorem*}[title=Russell-Myhill Paradox,id=russell_myhill_paradox]
    It is wrong that
    \[ \logprod P \eq \logprod Q \implies P \eq Q \]
    for all systems of propositions $P, Q$.
  \end{theorem*}
  \begin{proof}
    Assume the contrary.
    Define $Q = \{q \mid$ there exists a system of propositions $P$ such that $q \eq \logprod P$ and $q \nelem P\}$.
    Consider $q = \logprod Q$.
    Then for any system of propositions $P$ such that $\logprod P \eq q$ we have $P \eq Q$.
    Hence $q \elem Q$ iff $q \nelem Q$.
    Contradiction.
  \end{proof}
\end{forthel}

\printbibliography
\printlicense[CcByNcSa]{Marcel Schütz (2024--2025)}
\end{smodule}
\end{document}
