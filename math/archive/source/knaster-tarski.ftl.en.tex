\documentclass{stex}
\libusepackage{naproche}
\libinput{preamble}

\title{The Knaster-Tarski Fixed Point Theorem}
\author{Alexander Holz (2018), Marcel Schütz (2018--2025)}
\date{}

\begin{document}
\begin{smodule}{knaster-tarski.ftl}
\maketitle

\importmodule[libraries/classes]{definitions?powerclass.ftl}
\importmodule[libraries/sets]{axioms?powerset.ftl}
\importmodule[libraries/maps]{definitions?fixed-points.ftl}
\importmodule[libraries/maps]{definitions?maps-from-classes-to-classes.ftl}
\importmodule[libraries/systems-of-sets]{definitions?subset-preserving-maps.ftl}
\importmodule[libraries/systems-of-sets]{definitions?union-over-systems-of-sets.ftl}
\importmodule[libraries/systems-of-sets]{axioms?big-union.ftl}

\noindent This is a formalization of the \textit{Knaster-Tarski Fixed Point
Theorem}, i.e. of the assertion that every subset-preserving map has a fixed
point \cite{Schroeder2012}.

\begin{theorem}[forthel,title=Knaster-Tarski,id=knaster_tarski]
  Let $x$ be a set.
  Let $h$ be a map from $\pow(x)$ to $\pow(x)$ that preserves subsets.
  Then $h$ has a fixed point.
\end{theorem}
\begin{proof}[forthel]
  (1) Define $A = \{y \in \pow(x) \mid y \subclass h(y)\}$.
  Then $A$ is a subset of $\pow(x)$.
  We have $\unionover A \elem \pow(x)$.

  Let us show that (2) $\unionover A \subclass h(\unionover A)$.
  \begin{proof}
    Let $u \elem \unionover A$.
    Take $y \elem A$ such that $u \elem y$.
    Then $u \elem h(y)$.
    We have $y \subclass \unionover A$.
    Hence $h(y) \subclass h(\unionover A)$.
    Indeed $h$ is a subset preserving map between systems of sets and $y, \unionover A \elem \dom(h)$.
    Thus $h(y) \subclass h(\unionover A)$.
    Therefore $u \elem h(\unionover A)$.
  \end{proof}

  Then $h(\unionover A) \elem A$ (by 1).
  Indeed $h(\unionover A) \subclass x$.
  (3) Hence $h(\unionover A) \subclass \unionover A$.
  Indeed every element of $h(\unionover A)$ is an element of some element of $A$.

  Thus $h(\unionover A) \eq \unionover A$ (by 2, 3).
\end{proof}

\printbibliography
\printlicense[CcByNcSa]{Alexander Holz (2018), Marcel Schütz (2018--2025)}
\end{smodule}
\end{document}
