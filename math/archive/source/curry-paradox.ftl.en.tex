\documentclass{article}
\usepackage{naproche}
\libinput{preamble}

\title{Curry's Paradox}
\author{Marcel Schütz}
\date{2024--2025}

\begin{document}
\begin{smodule}{curry-paradox.ftl}
\maketitle

\importmodule[libraries/preliminaries]{preliminaries.ftl}

\symdef{varzero}{\comp{x_0}}
\symdef{abs}[args=2]{\comp\lambda#1\comp.\,#2}
\symdef{app}[args=2]{\dobrackets{#1}\cobrackets{#2}}
\symdef{fix}{\texttt{fix}}
\symdef{impl}{\mathop{\comp\rightarrow}}

\noindent \emph{Curry's Paradox} is a paradox described by Haskell Curry in 
1942 \cite{Curry1942}.
It allows the derivation of an arbitrary statement from a self-referential 
expression that presupposes its own validity.

\section*{Untyped $\lambda$-Calculus}

\noindent Our formalization of Curry's Paradox in \Naproche is based on an 
untyped $\lambda$-calculus:

\begin{forthel}
  \begin{signature*}
    An \emph{expression} is a notion.

    Let $E, E'$ denote expressions.
  \end{signature*}

  \begin{signature*}
    A \emph{variable} is an expression.
  \end{signature*}

  \begin{signature*}[for=varzero]
    $\emph{\varzero}$ is a variable.
  \end{signature*}

  \begin{signature*}[title=Abstraction,id=abstraction,for=abs]
    Let $x$ be a variable.
    $\emph{\abs{x}{E}}$ is an expression.
  \end{signature*}

  \begin{signature*}[title=Application,id=application,for=app]
    $\emph{\app{E}{E'}}$ is an expression.
  \end{signature*}

  \begin{signature*}[title=Fixed-point combinator,id=fixed_point_combinator,for=fix]
    $\emph{\fix}$ is an expression such that
    \[\app{\fix}{E} = \app{E}{\app{\fix}{E}}\]
    for every expression $E$.
  \end{signature*}
\end{forthel}


\section*{Propositional Logic}

\noindent We extend this $\lambda$-calculus by a fragment of propositional
logic:

\begin{forthel}  
  Let $E, E'$ denote expressions.
  Let $x$ denote a variable.

  \begin{signature*}[title=Implication,id=implication,for=impl]
    $\emph{E \impl E'}$ is an expression.
  \end{signature*}

  \begin{signature*}[title=Truth,id=truth]
    $E$ is \emph{true} is a relation.
  \end{signature*}

  \begin{axiom*}[title=$\beta$-reduction,id=beta_reduction]
    $\app{\abs{x}{x \impl E'}}{E} = E \impl E'$.
  \end{axiom*}

  \begin{axiom*}[title=Reflexivity,id=reflexivity]
    $E \impl E$ is true.
  \end{axiom*}

  \begin{axiom*}[title=Modus Ponens,id=modus_ponens]
    If $E$ is true and $E \impl E'$ is true then $E'$ is true.
  \end{axiom*}

  \begin{axiom*}[title=Strengthening,id=strengthening]
    If $E \impl (E \impl E')$ is true then $E \impl E'$ is true.
  \end{axiom*}
\end{forthel}


\section*{Curry's paradox}

\noindent Using the fixpoint combinator from above we can formulate a 
self-referential expression $X$ of the form ``If $X$ is true then $E$ is
true'' for any arbitrary expression $E$ by defining
$X = \app{\fix}{\abs{\varzero}{\varzero \impl E}}$.
From the existence of such an expression $X$ together with the above axioms 
we can then derive that any expression $E$ is true.

\begin{forthel}
  \begin{theorem*}[title=Curry's Paradox,id=curry_paradox]
    Every expression is true.
  \end{theorem*}
  \begin{proof}
    Let $E$ be an expression.
    Take $N = \abs{\varzero}{\varzero \impl E}$ and $X = \app{\fix}{N}.$

    (1) Then $X = \app{N}{X} = X \impl E$ (by \nameref{beta_reduction}).

    Hence $X \impl (X \impl E)$ is true (by 1, \nameref{reflexivity}).

    (2) Thus $X \impl E$ is true (by \nameref{strengthening}).

    (3) Therefore $X$ is true (by 1, 2).

    Consequently $E$ is true (by \nameref{modus_ponens}, 2, 3).
  \end{proof}
\end{forthel}

\printbibliography
\printlicense[CcByNcSa]{Marcel Schütz (2024--2025)}
\end{smodule}
\end{document}
