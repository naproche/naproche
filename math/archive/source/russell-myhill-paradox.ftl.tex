\documentclass{article}
\usepackage{naproche}
\libinput{preamble}

\title{The Russell-Myhill Paradox in \Naproche}
\author{Marcel Schütz}
\date{2024--2025}

\begin{document}
\begin{smodule}{russell-myhill-paradox.ftl}
\maketitle

\noindent The \emph{Russell-Myhill Paradox} is a paradox discovered by 
Bertrand Russell which states that two systems of propositions with the same 
\emph{logical product} are not necessarily equal.
It was discussed in Russell's 1903 \emph{Principles of Mathematics}
\cite[Appendix B]{Russell1903} and rediscovered in 1958 by John Myhill
\cite{Myhill1958}.

\begin{forthel}
  \begin{signature*}
    A proposition is an object.
  \end{signature*}

  \begin{signature*}
    Let $p$ be a proposition.
    $p$ is true is an atom.
  \end{signature*}

  [synonym system/-s]

  \begin{definition*}
    A system of propositions is a class $P$ such that every element of $P$ is a proposition.
  \end{definition*}

  \begin{signature*}
    Let $P$ be a system of propositions.
    The logical product of $P$ is a proposition $p$ such that $p$ is true iff every element of $P$ is true.
  \end{signature*}
  
  Let $\sqcap P$ denote the logical product of $P$.
  
  \begin{theorem*}[title=Russell-Myhill Paradox,id=russell_myhill_paradox]
    It is wrong that
    \[ \sqcap P = \sqcap Q \implies P = Q \]
    for all systems of propositions $P, Q$.
  \end{theorem*}
  \begin{proof}
    Assume the contrary.
    Define $Q = \{q \mid$ there exists a system of propositions $P$ such that $q = \sqcap P$ and $q \notin P\}$.
    Consider $q = \sqcap Q$.
    Then for any system of propositions $P$ such that $\sqcap P = q$ we have $P = Q$.
    Hence $q \in Q$ iff $q \notin Q$.
    Contradiction.
  \end{proof}
\end{forthel}

\printbibliography
\printlicense[CcByNcSa]{2024--2025}{Marcel Schütz}
\end{smodule}
\end{document}
