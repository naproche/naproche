\documentclass{article}
\usepackage{naproche}
\usepackage{graphicx}
\usepackage{upgreek}
\usepackage{titlesec}
\libinput{preamble}

\title{Peano's \textit{Principles of Arithmetics}}
\author{Marcel Schütz}
\date{2025}

\begin{document}
\begin{smodule}{peano.ftl}

\titleformat{\section}[block]{\Large}{\S\thesection.}{1em}{\centering}[\vspace{1em}---]
\titleformat{\subsection}[block]{\vspace{2em}\large\itshape}{}{0em}{\centering}

\importmodule[libraries/classes]{definitions?subclasses.ftl}

\symdef{Nat}{\mathbb N}
\symdef{one}{\comp1}
\symdef{two}{\comp2}
\symdef{three}{\comp3}
\symdef{four}{\comp4}
\symdef{succ}[args=1]{\comp{\textsf{succ}}\dobrackets{#1}}
\symdef{plus}{\mathrel{\comp+}}
\symdef{triplus}[args=3]{#1\mathrel{\comp+}#2\mathrel{\comp+}#3}

% (An approximation to) Peano's original notation:
\notation*{True}[peano]{\mathrm V}
\notation*{False}[peano]{\Uplambda}
\notation*{And}[peano]{\mathrel{\comp{.\raisebox{1ex}..}}}
\notation*{Implies}[peano]{\mathrel{\comp{.\reflectbox c.}}}
\notation*{Iff}[peano]{\mathrel{\comp{.{=}.}}}
\notation*{Not}[peano]{\mathrel{\comp{-{=}}}}
\notation*{elem}[peano]{\mathrel{\comp\epsilon}}
\notation*{subclass}[peano]{\mathrel{\comp{\reflectbox c}}}
\notation*{Nat}[peano]{\comp{\mathrm N}}
\notation*{succ}[peano]{#1\mathbin{\comp+}\comp1}

\maketitle

\noindent This is a translation of \S 1 of Giuseppe Peano's
\emph{Arithmetices Principia} (eng: \emph{Principles of Arithmetics})
\cite{Peano1889, Peano1889_translation} to \ForTheL.
It introduces Peano's language and famous axioms of arithmetics as well as a
number of theorems about addition of natural numbers.
                                      
\vspace{1em}\hrule

\section{Numbers and Addition.}

\subsection{Explanations.}

\begin{forthel}
  Let $a,b,c,d$ denote mathematical objects.

  \begin{signature*}
    A \emph{number} is a mathematical object.
  \end{signature*}

  \begin{definition*}
    $\emph{\Nat}$ is the class of all numbers.
  \end{definition*}

  \begin{signature*}
    $\emph{\one}$ is a mathematical object.
  \end{signature*}

  \begin{signature*}
    $\emph{\succ{a}}$ is a mathematical object.
  \end{signature*}
\end{forthel}


\subsection{Axioms.}

\begin{forthel}
  \begin{axiom*}[title=1,id=P1]
    $\one \elem \Nat$.
  \end{axiom*}

  \begin{axiom*}[title=2,id=P2]
    $a \elem \Nat \Implies a \eq a$.
  \end{axiom*}
  
  \begin{axiom*}[title=3,id=P3]
    $a,b,c \elem \Nat \Implies (a \eq b \Iff b \eq a)$.
  \end{axiom*}
  
  \begin{axiom*}[title=4,id=P4]
    $a,b \elem \Nat \Implies ((a \eq b \And b \eq c) \Implies a \eq c)$.
  \end{axiom*}
  
  \begin{axiom*}[title=5,id=P5]
    $(a \eq b \And b \elem \Nat) \Implies a \elem \Nat$.
  \end{axiom*}
  
  \begin{axiom*}[title=6,id=P6]
    $a \elem \Nat \Implies \succ{a} \elem \Nat$.
  \end{axiom*}
  
  \begin{axiom*}[title=7,id=P7]
    $a,b \elem \Nat \Implies (a \eq b \Iff \succ{a} \eq \succ{b})$.
  \end{axiom*}
  
  \begin{axiom*}[title=8,id=P8]
    $a \elem \Nat \Implies \succ{a} \neq \one$.
  \end{axiom*}
  
  \begin{axiom*}[title=9,id=P9]
    Let $k$ be a class.
    Then $(\one \elem k \And \forall x \elem \Nat: x \elem k \Implies \succ{x} \elem k) \Implies \Nat \subclass k$.
  \end{axiom*}
\end{forthel}


\subsection{Definitions.}

\begin{forthel}
  \begin{definition*}[title=10.i,id=P10_1]
    $\emph{\two} = \succ{\one}$.
  \end{definition*}

  \begin{definition*}[title=10.ii,id=P10_2]
    $\emph{\three} = \succ{\two}$.
  \end{definition*}
  
  \begin{definition*}[title=10.iii,id=P10_3]
    $\emph{\four} = \succ{\three}$.
  \end{definition*}
\end{forthel}


\subsection{Theorems.}

\begin{forthel}
  \begin{theorem*}[title=11,id=P11]
    $\two \elem \Nat$.
  \end{theorem*}
  \begin{proof}
    (1) $\one \elem \Nat$ (by \ref{P1}).

    (2) $\one \elem \Nat \Implies \succ{\one} \elem \Nat$ (by \ref{P6}).

    (3) $\succ{\one} \elem \Nat$ (by 1, 2).

    (4) $\two \eq \succ{\one}$ (by \ref{P10_1}).

    Therefore $\two \elem \Nat$.
  \end{proof}

  \begin{theorem*}[title=12,id=P12]
    $\three,\four \elem \Nat$.
  \end{theorem*}

  \begin{theorem*}[title=13,id=P13]
    $(a,b,c,d \elem \Nat \And a \eq b \And b \eq c \And c \eq d) \Implies a \eq d$.
  \end{theorem*}

  \begin{theorem*}[title=14,id=P14]
    $(a,b,c \elem \Nat \And a \eq b \And b \eq c \And a \neq c) \Iff \False$.
  \end{theorem*}

  \begin{theorem*}[title=15,id=P15]
    $(a,b,c \elem \Nat a \eq b \And b \neq c) \Implies a \neq c$.
  \end{theorem*}

  \begin{theorem*}[title=16,id=P16]
    $(a,b \elem \Nat \And \succ{a} \eq \succ{b}) \Implies a \eq b$.
  \end{theorem*}

  \begin{theorem*}[title=17,id=P17]
    $a,b \elem \Nat \Implies (a \neq b \Iff \succ{a} \neq \succ{b})$.
  \end{theorem*}
\end{forthel}


\subsection{Definition.}

\begin{forthel}
  \begin{signature*}
    $\emph{a \plus b}$ is a mathematical object.
  \end{signature*}

  \begin{axiom*}
    $a \elem \Nat \Implies a \plus \one \eq \succ{a}$.
  \end{axiom*}

  \begin{axiom*}[title=18,id=P18]
    $a,b \elem \Nat \Implies a \plus (\succ{b}) \eq \succ{a \plus b}$.
  \end{axiom*}
\end{forthel}


\subsection{Theorems.}

\begin{forthel}
  \begin{theorem*}[title=19,id=P19]
    $a,b \elem \Nat \Implies a \plus b \elem \Nat$.
  \end{theorem*}
  \begin{proof}
    Define $T = \compclass{b}{a \plus b \elem \Nat}$.
    
    We have $a \elem \Nat$
    Hence $\succ{a} \elem \Nat$ (by \ref{P6}).
    
    (1) Thus $\one \elem T$.

    (2) For all $b \elem \Nat$ if $b \elem T$ then $\succ{b} \elem T$.
    \begin{proof}
      Let $b \elem \Nat$.
      Assume $b \elem T$.
      Then $a \plus b \elem \Nat$.
      Hence $\succ{a \plus b} \elem \Nat$ (by \ref{P6}).
      Thus $a \plus \succ{b} \elem \Nat$ (by \ref{P18}).
      Consequently $\succ{b} \elem T$.
    \end{proof}

    (3) Therefore $\Nat \subclass T$ (by 1, 2, \ref{P9}).

    We have $b \elem \Nat$.
    Hence $b \elem T$.
    Thus $a \plus b \elem \Nat$.
  \end{proof}

  \begin{definition*}[title=20,id=P20]
    $\emph{\triplus{a}{b}{c}} = (a \plus b) \plus c$.
  \end{definition*}

  \begin{theorem*}[title=21,id=P21]
    $a,b,c \elem \Nat \Implies \triplus{a}{b}{c} \elem \Nat$.
  \end{theorem*}

  \begin{theorem*}[title=22,id=P22]
    $a,b,c \elem \Nat \Implies (a \eq b \Iff a \plus c \eq b \plus c)$.
  \end{theorem*}

  \begin{theorem*}[title=23,id=P23]
    $a,b,c \elem \Nat \Implies (a \plus (b \plus c) \eq \triplus{a}{b}{c})$.
  \end{theorem*}

  \begin{theorem*}[title=24,id=P24]
    $a \elem \Nat \Implies \one \plus a \eq a \plus \one$.
  \end{theorem*}

  \begin{theorem*}[title=24',id=P24prime]
    $a,b \elem \Nat \Implies \triplus{\one}{a}{b} \eq \triplus{a}{\one}{b}$.
  \end{theorem*}

  \begin{theorem*}[title=25,id=P25]
    $a,b \elem \Nat a \plus b \eq b \plus a$.
  \end{theorem*}

  \begin{theorem*}[title=26,id=P26]
    $a,b,c \elem \Nat \Implies (a \eq b \Iff c \plus a \eq c \plus b)$.
  \end{theorem*}

  \begin{theorem*}[title=27,id=P27]
    $a,b,c \elem \Nat \Implies \triplus{a}{b}{c} \eq \triplus{a}{c}{b}$.
  \end{theorem*}

  \begin{theorem*}[title=28,id=P28]
    $(a,b,c,d \elem \Nat \And a \eq b \And c \eq d) \Implies a \plus c \eq b \plus d$.
  \end{theorem*}
\end{forthel}

\vspace{1em}\hrule

\titleformat{\section}[hang]{\Large\bfseries}{}{0em}{}

\printbibliography
\printlicense[CcByNcSa]{Marcel Schütz (2025)}

% https://github.com/mdnahas/Peano_Book/blob/master/Peano.pdf
\end{smodule}
\end{document}
