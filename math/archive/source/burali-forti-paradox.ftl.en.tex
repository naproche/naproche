\documentclass{article}
\usepackage{naproche}
\libinput{preamble}

\title{Burali-Forti's Paradox}
\author{Marcel Schütz}
\date{2024--2025}

\begin{document}
\begin{smodule}{burali-forti-paradox.ftl}
\maketitle

\importmodule[libraries/set-theory]{ordinals.ftl}

\noindent \emph{Burali-Forti's Paradox}, named after Cesare Burali-Forti, 
demonstates that the collection $\Ord$ of all ordinal numbers is a
proper class \cite{BuraliForti1897}.

\begin{theorem}[forthel,title=Burali-Forti's Paradox,id=burali_forti_paradox]
  $\Ord$ is not a set.
\end{theorem}
\begin{proof}[forthel]
  Assume the contrary.
  $\Ord$ is transitive and every element of $\Ord$ is transitive.
  Hence $\Ord$ is an ordinal.
  Thus $\Ord \SETin \Ord$.
  Contradiction.
\end{proof}

\printbibliography
\printlicense[CcByNcSa]{Marcel Schütz (2024--2025)}
\end{smodule}
\end{document}
