\documentclass{stex}
\usepackage{graphicx}
\usepackage{upgreek}
\usepackage{titlesec}
\libusepackage{naproche}
\libusepackage{copyright}
\libinput{preamble}

\title{Peano's \textit{Principles of Arithmetics}}
\author{Marcel Schütz (2025)}
\date{}

\begin{document}
\begin{smodule}{peano.ftl}

\titleformat{\section}[block]{\Large}{\S\thesection.}{1em}{\centering}[\vspace{1em}---]
\titleformat{\subsection}[block]{\vspace{2em}\large\itshape}{}{0em}{\centering}

\importmodule[libraries/foundations]{classes.ftl}

\symdef{Nat}{\mathbb N}
\symdef{one}{\comp1}
\symdef{two}{\comp2}
\symdef{three}{\comp3}
\symdef{four}{\comp4}
\symdef{succ}[args=1]{\comp{\textsf{succ}}\dobrackets{#1}}
\symdef{plus}{\,\comp+\,}
\symdef{triplus}[args=3]{#1\,\comp+\,#2\,\comp+\,#3}
\symdecl*{number}

% (An approximation to) Peano's original notation:
\notation*{True}[peano]{\mathrm V}
\notation*{False}[peano]{\Uplambda}
\notation*{And}[peano]{\;\comp{.\raisebox{1ex}..}\;}
\notation*{Implies}[peano]{\;\comp{.\reflectbox c.}\;}
\notation*{Iff}[peano]{\;\comp{.{=}.}\;}
\notation*{Not}[peano]{\;\comp{-{=}}\;}
\notation*{Elem}[peano]{\;\comp\epsilon\;}
\notation*{SETinclude}[peano]{\;\comp{\reflectbox c}\;}
\notation*{Nat}[peano]{\comp{\mathrm N}}
\notation*{succ}[peano]{#1\,\comp+\,\comp1}

\maketitle

\noindent This is a translation of \S 1 of Giuseppe Peano's
\emph{Arithmetices Principia} (eng: \emph{Principles of Arithmetics})
\cite{Peano1889, Verheyen2022} to \ForTheL.
It introduces Peano's language and famous axioms of arithmetics as well as a
number of theorems about addition of natural numbers.
                                      
\vspace{1em}\hrule

\section{Numbers and Addition.}

\subsection{Explanations.}

\begin{forthel}
  Let $a,b,c,d$ denote mathematical objects.

  \begin{signature*}[for=number]
    A \emph{number} is a mathematical object.
  \end{signature*}

  \begin{definition*}[for=Nat]
    $\emph{\Nat}$ is the class of all numbers.
  \end{definition*}

  \begin{signature*}[for=one]
    $\emph{\one}$ is a mathematical object.
  \end{signature*}

  \begin{signature*}[for=succ]
    $\emph{\succ{a}}$ is a mathematical object.
  \end{signature*}
\end{forthel}


\subsection{Axioms.}

\symdecl*{P1}
\symdecl*{P2}
\symdecl*{P3}
\symdecl*{P4}
\symdecl*{P5}
\symdecl*{P6}
\symdecl*{P7}
\symdecl*{P8}
\symdecl*{P9}

\begin{forthel}
  \begin{axiom*}[title=1,name=P1]
    $\one \Elem \Nat$.
  \end{axiom*}

  \begin{axiom*}[title=2,name=P2]
    $a \Elem \Nat \Implies a \Eq a$.
  \end{axiom*}
  
  \begin{axiom*}[title=3,name=P3]
    $a,b,c \Elem \Nat \Implies (a \Eq b \Iff b \Eq a)$.
  \end{axiom*}
  
  \begin{axiom*}[title=4,name=P4]
    $a,b \Elem \Nat \Implies ((a \Eq b \And b \Eq c) \Implies a \Eq c)$.
  \end{axiom*}
  
  \begin{axiom*}[title=5,name=P5]
    $((a \Eq b \And b \Elem \Nat) \Implies a \Elem \Nat)$.
    % Without the outer parentheses Naproche complains about the "\Implies"
  \end{axiom*}
  
  \begin{axiom*}[title=6,name=P6]
    $a \Elem \Nat \Implies \succ{a} \Elem \Nat$.
  \end{axiom*}
  
  \begin{axiom*}[title=7,name=P7]
    $a,b \Elem \Nat \Implies (a \Eq b \Iff \succ{a} \Eq \succ{b})$.
  \end{axiom*}
  
  \begin{axiom*}[title=8,name=P8]
    $a \Elem \Nat \Implies \succ{a} \NotEq \one$.
  \end{axiom*}
  
  \begin{axiom*}[title=9,name=P9]
    Let $k$ be a class.
    Then $(\one \Elem k \And \Forall x \Elem \Nat: x \Elem k \Implies \succ{x} \Elem k) \Implies\linebreak \Nat \SETinclude k$.
  \end{axiom*}
\end{forthel}


\subsection{Definitions.}

\begin{forthel}
  \begin{definition*}[title=10.i,for=two]
    $\emph{\two} = \succ{\one}$.
  \end{definition*}

  \begin{definition*}[title=10.ii,for=three]
    $\emph{\three} = \succ{\two}$.
  \end{definition*}
  
  \begin{definition*}[title=10.iii,for=four]
    $\emph{\four} = \succ{\three}$.
  \end{definition*}
\end{forthel}


\subsection{Theorems.}

\symdecl*{P11}
\symdecl*{P12}
\symdecl*{P13}
\symdecl*{P14}
\symdecl*{P15}
\symdecl*{P16}
\symdecl*{P17}

\begin{forthel}
  \begin{theorem*}[title=11,name=P11]
    $\two \Elem \Nat$.
  \end{theorem*}
  \begin{proof}
    (1) $\one \Elem \Nat$ (by \sn{P1}).

    (2) $\one \Elem \Nat \Implies \succ{\one} \Elem \Nat$ (by \sn{P6}).

    (3) $\succ{\one} \Elem \Nat$ (by 1, 2).

    (4) $\two \Eq \succ{\one}$.

    Therefore $\two \Elem \Nat$.
  \end{proof}

  \begin{theorem*}[title=12,name=P12]
    $\three,\four \Elem \Nat$.
  \end{theorem*}

  \begin{theorem*}[title=13,name=P13]
    $(a,b,c,d \Elem \Nat \And a \Eq b \And b \Eq c \And c \Eq d) \Implies a \Eq d$.
  \end{theorem*}

  \begin{theorem*}[title=14,name=P14]
    $(a,b,c \Elem \Nat \And a \Eq b \And b \Eq c \And a \NotEq c) \Iff \False$.
  \end{theorem*}

  \begin{theorem*}[title=15,name=P15]
    $(a,b,c \Elem \Nat \And a \Eq b \And b \NotEq c) \Implies a \NotEq c$.
  \end{theorem*}

  \begin{theorem*}[title=16,name=P16]
    $(a,b \Elem \Nat \And \succ{a} \Eq \succ{b}) \Implies a \Eq b$.
  \end{theorem*}

  \begin{theorem*}[title=17,name=P17]
    $a,b \Elem \Nat \Implies (a \NotEq b \Iff \succ{a} \NotEq \succ{b})$.
  \end{theorem*}
\end{forthel}


\subsection{Definition.}

\symdecl*{P18}

\begin{forthel}
  \begin{signature*}[for=plus]
    $\emph{a \plus b}$ is a mathematical object.
  \end{signature*}

  \begin{axiom*}
    $a \Elem \Nat \Implies a \plus \one \Eq \succ{a}$.
  \end{axiom*}

  \begin{axiom*}[title=18,name=P18]
    $a,b \Elem \Nat \Implies a \plus (\succ{b}) \Eq \succ{a \plus b}$.
  \end{axiom*}
\end{forthel}


\subsection{Theorems.}

\symdecl*{P19}
\symdecl*{P20}
\symdecl*{P21}
\symdecl*{P22}
\symdecl*{P23}
\symdecl*{P24}
\symdecl*{P24prime}
\symdecl*{P25}
\symdecl*{P26}
\symdecl*{P27}
\symdecl*{P28}

\begin{forthel}
  \begin{theorem*}[title=19,name=P19]
    $a,b \Elem \Nat \Implies a \plus b \Elem \Nat$.
  \end{theorem*}
  \begin{proof}
    Let $a,b \Elem \Nat$.
    Define $T = \CClass{b'}{a \plus b' \Elem \Nat}$.
    
    We have $a \Elem \Nat$.
    Hence $\succ{a} \Elem \Nat$ (by \sn{P6}).
    
    (1) Thus $\one \Elem T$.

    (2) For all $b' \Elem \Nat$ if $b' \Elem T$ then $\succ{b'} \Elem T$.
    \begin{proof}
      Let $b' \Elem \Nat$.
      Assume $b' \Elem T$.
      Then $a \plus b' \Elem \Nat$.
      Hence $\succ{a \plus b'} \Elem \Nat$ (by \sn{P6}).
      Thus $a \plus \succ{b'} \Elem \Nat$ (by \sn{P18}).
      Consequently $\succ{b'} \Elem T$.
    \end{proof}

    (3) Therefore $\Nat \SETinclude T$ (by 1, 2, \sn{P9}).

    We have $b \Elem \Nat$.
    Hence $b \Elem T$.
    Thus $a \plus b \Elem \Nat$.
  \end{proof}

  \begin{definition*}[title=20,for=triplus]
    $\emph{\triplus{a}{b}{c}} = (a \plus b) \plus c$.
  \end{definition*}

  \begin{theorem*}[title=21,name=P21]
    $a,b,c \Elem \Nat \Implies \triplus{a}{b}{c} \Elem \Nat$.
  \end{theorem*}

  \begin{theorem*}[title=22,name=P22]
    $a,b,c \Elem \Nat \Implies (a \Eq b \Iff a \plus c \Eq b \plus c)$.
  \end{theorem*}

  \begin{theorem*}[title=23,name=P23]
    $a,b,c \Elem \Nat \Implies (a \plus (b \plus c) \Eq \triplus{a}{b}{c})$.
  \end{theorem*}

  \begin{theorem*}[title=24,name=P24]
    $a \Elem \Nat \Implies \one \plus a \Eq a \plus \one$.
  \end{theorem*}

  \begin{theorem*}[title=24',name=P24prime]
    $a,b \Elem \Nat \Implies \triplus{\one}{a}{b} \Eq \triplus{a}{\one}{b}$.
  \end{theorem*}

  \begin{theorem*}[title=25,name=P25]
    $a,b \Elem \Nat \And a \plus b \Eq b \plus a$.
  \end{theorem*}

  \begin{theorem*}[title=26,name=P26]
    $a,b,c \Elem \Nat \Implies (a \Eq b \Iff c \plus a \Eq c \plus b)$.
  \end{theorem*}

  \begin{theorem*}[title=27,name=P27]
    $a,b,c \Elem \Nat \Implies \triplus{a}{b}{c} \Eq \triplus{a}{c}{b}$.
  \end{theorem*}

  \begin{theorem*}[title=28,name=P28]
    $(a,b,c,d \Elem \Nat \And a \Eq b \And c \Eq d) \Implies a \plus c \Eq b \plus d$.
  \end{theorem*}
\end{forthel}

\vspace{1em}\hrule

\titleformat{\section}[hang]{\Large\bfseries}{}{0em}{}

\printbibliography
\printlicense[CcByNcSa]{Marcel Schütz (2025)}

\end{smodule}
\end{document}
