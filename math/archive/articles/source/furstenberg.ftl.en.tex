\documentclass{stex}
\libusepackage{naproche}
\libusepackage{copyright}
\libinput{preamble}

\newcommand{\N}{\mathrm{N}}
\newcommand{\Int}{\mathbb{Z}}
\newcommand{\Ps}{\mathrm{P}}

\title{Furstenberg's Proof of the Infinitude of Primes in \Naproche}
\author{Andrei Paskevich, Marcel Schütz}
\date{2025}

\begin{document}
\begin{smodule}{furstenberg.ftl}
\maketitle

\importmodule[libraries/arithmetics]{primes.ftl}
\importmodule[libraries/arithmetics]{nat-is-a-set.ftl}
\importmodule[libraries/set-theory]{zfc.ftl}
\importmodule[libraries/foundations]{closure-under-finite-unions.ftl}

\symdecl*{Furstenberg}

\noindent This is a formalization of Furstenberg's topological proof of the
infinitude of primes \cite[p. 353]{Furstenberg1955}.

The central idea of Furstenberg's proof is to define a certain topology on
$\Nat$ from the properties of which we can deduce that the set of
primes is infinite.\footnote{Actually, Furstenberg's proof makes use of a
topology on $\Int$. But this topology can as well be restricted to
$\Nat$ without substantially changing the proof.}

\begin{forthel}
  Let $n, m, k$ denote natural numbers.
  Let $p, q$ denote nonzero natural numbers.

  \begin{definition}
    Let $A$ be a subset of $\Nat$.
    $A^\complement = \Nat \SETdiff A$.
  \end{definition}

  Let the complement of $A$ stand for $A^\complement$.

  \begin{lemma}
    The complement of any subset of $\Nat$ is a subset of $\Nat$.
  \end{lemma}
\end{forthel}

Towards a suitable topology on $\Nat$ let us define \textit{arithmetic
sequences} $\N_{n, q}$ on $\Nat$.

\begin{forthel}
  \begin{definition}
    $\N_{n, q} = \SClass{m}{\Nat}{\NATcong{m}{n}{q}}$.
  \end{definition}
\end{forthel}

This allows us to define the \textit{evenly spaced natural number
topology} on $\Nat$, whose open sets are defined as follows.

\begin{forthel}
  \begin{definition}
    Let $U$ be a subset of $\Nat$.
    $U$ is open iff for any $n \in U$ there exists a $q$ such that
    $\N_{n, q} \SETinclude U$.
  \end{definition}

  \begin{definition}
    A system of open sets is a system of sets $S$ such that every element of
    $S$ is an open subset of $\Nat$.
  \end{definition}

  \begin{lemma}
    Every system of open sets is a set.
  \end{lemma}
  \begin{proof}
    Let $S$ be a system of open sets.
    Then $S \SETinclude \SETpow(\Nat)$.
    Hence $S$ is a set.
  \end{proof}
\end{forthel}

We can show that the open sets indeed form a topology on $\Nat$.

\begin{forthel}
  \begin{lemma}
    $\Nat$ and $\SETempty$ are open.
  \end{lemma}

  \begin{lemma}
    Let $U,V$ be open subsets of $\Nat$.
    Then $U \SETintersect V$ is open.
  \end{lemma}
  \begin{proof}
    Let $n \in U \SETintersect V$.
    Take a $q$ such that $\N_{n, q} \SETinclude U$.
    Take a $p$ such that $\N_{n, p} \SETinclude V$.
    Then $p \NATmul q \NotEq \NATzero$.

    Let us show that $\N_{n, p \NATmul q} \SETinclude U \SETintersect V$.
      Let $m \in \N_{n, p \NATmul q}$.
      We have $\NATcong{m}{n}{p \NATmul q}$.
      Hence $\NATcong{m}{n}{p}$ and $\NATcong{m}{n}{q}$.
      Thus $m \in \N_{n, p}$ and $m \in \N_{n, q}$.
      Therefore $m \in U$ and $m \in V$.
      Consequently $m \in U \SETintersect V$.
    End.
  \end{proof}

  \begin{lemma}
    Let $S$ be a system of open sets.
    Then $\SETunionover S$ is open.
  \end{lemma}
  \begin{proof}
    Let $n \in \SETunionover S$.
    Take a set $M$ such that $n \in M \in S$.
    Consider a $q$ such that $\N_{n, q} \SETinclude M$.
    Then $\N_{n, q} \SETinclude \SETunionover S$.
  \end{proof}
\end{forthel}

Now that we have a topology of open sets on $\Nat$, we can continue
with a characterization of closed sets whose key property is that they are
closed under finite unions.

\begin{forthel}
  \begin{definition}
    Let $A$ be a subset of $\Nat$.
    $A$ is closed iff $A^\complement$ is open.
  \end{definition}

  \begin{definition}
    A system of closed sets is a system of sets $S$ such that every element of
    $S$ is a closed subset of $\Nat$.
  \end{definition}

  \begin{lemma}
    Every system of closed sets is a set.
  \end{lemma}
  \begin{proof}
    Let $S$ be a system of closed sets.
    Then $S \SETinclude \SETpow(\Nat)$.
    $\SETpow(\Nat)$ is a set.
    Hence $S$ is a set.
  \end{proof}

  \begin{lemma}
    Let $S$ be a finite system of closed sets.
    Then $\SETunionover S$ is closed.
  \end{lemma}
  \begin{proof}
    Define $C = \CClass{X}{X \text{ is a closed subset of } \Nat}$.

    Let us show that $A \SETunion B \in C$ for any $A, B \in C$.
      Let $A, B \in C$.
      Then $A, B$ are closed subsets of $\Nat$.
      We have $((A \SETunion B)^\complement) = A^\complement \SETintersect B^\complement$. %!
      $A^\complement$ and $B^\complement$ are open.
      Hence $A^\complement \SETintersect B^\complement$ is open.
      Thus $A \SETunion B$ is a closed subset of $\Nat$.
    End.

    Therefore $C$ is closed under finite unions.
    Consequently $\SETunionover S \in C$.
    Indeed $S$ is a subset of $C$.
  \end{proof}
\end{forthel}

An important step towards Furstenberg's proof is to show that arithmetic
sequences are closed.

\begin{forthel}
  \begin{lemma}
    $\N_{n, q}$ is closed.
  \end{lemma}
  \begin{proof}
    Let $m \in (\N_{n, q})^\complement$.

    Let us show that $\N_{m, q} \SETinclude (\N_{n, q})^\complement$.
      Let $k \in \N_{m, q}$.
      Assume $k \notin (\N_{n, q})^\complement$.
      Then $\NATcong{k}{m}{q}$ and $\NATcong{n}{k}{q}$.
      Hence $\NATcong{m}{n}{q}$.
      Therefore $m \in \N_{n, q}$.
      Contradiction.
    End.
  \end{proof}
\end{forthel}

Identifying each prime number $p$ with the arithmetic sequence $\N_{0, p}$
yields a bijection between the set $\Prime$ of all prime numbers and the set
$\Ps$ of all such sequences $\N_{0, p}$.
Thus to show that there are infinitely many primes it suffices to show that
$\Ps$ is infinite.

\begin{forthel}
  \begin{definition}
    $\Ps = \CClass{\N_{\NATzero, p}}{p \in \Prime}$.
  \end{definition}

  \begin{lemma}
    $\Ps$ is a system of closed sets.
  \end{lemma}
  \begin{proof}
    $\N_{\NATzero, p}$ is a closed subset of $\Nat$ for every $p \in \Prime$.
  \end{proof}

  \begin{lemma}
    $\Ps$ is a set that is equinumerous to $\Prime$.
  \end{lemma}
  \begin{proof}
    (1) $\Ps$ is a set.
    Indeed $\Ps \SETinclude \SETpow(\Nat)$.

    (2) $\Ps$ is equinumerous to $\Prime$.
    \begin{proof}
      Define $f(p) = \N_{\NATzero,p}$ for $p \in \Prime$.

      Let us show that $f$ is injective.
        Let $p, q \in \Prime$.
        Assume $f(p) = f(q)$.
        Then $\N_{\NATzero, p} = \N_{\NATzero, q}$.
        We have $\N_{\NATzero, p} = \SClass{m}{\Nat}{\NATcong{m}{\NATzero}{p}}$ and
        $\N_{\NATzero, q} = \SClass{m}{\Nat}{\NATcong{m}{\NATzero}{q}}$.

        We can show that for all $m \in \Nat$ we have $p \NATmid m$ iff $q \NATmid m$.
          Let $m \in \Nat$.
          Then $\NATcong{m}{\NATzero}{p}$ iff $\NATcong{m}{\NATzero}{q}$.
          Thus $m \NATmod p = \NATzero \NATmod p$ iff $m \NATmod q = \NATzero \NATmod q$.
          We have $\NATzero \NATmod p = \NATzero = \NATzero \NATmod q$.
          Hence $m \NATmod p = \NATzero$ iff $m \NATmod q = \NATzero$.
          Therefore $p \NATmid m$ iff $q \NATmid m$.
        End.

        Consequently $p = q$.
      End.

      $f$ is surjective onto $\Ps$.
      Thus $f$ is a bijection between $\Prime$ and $\Ps$.
    \end{proof}
  \end{proof}

  \begin{theorem}[title=Furstenberg,name=Furstenberg]
    $\Prime$ is infinite.
  \end{theorem}
  \begin{proof}
    $\SETunionover \Ps$ is a subset of $\Nat$.

    Let us show that for any $n \in \Nat$ we have $n \in \SETunionover \Ps$ iff $n$
    has a prime divisor.
      Let $n \in \Nat$.

      If $n$ has a prime divisor then $n$ belongs to $\SETunionover \Ps$.
      \begin{proof}
        Assume $n$ has a prime divisor.
        Take a prime divisor $p$ of $n$.
        We have $\N_{\NATzero, p} \in \Ps$.
        Hence $n \in \N_{\NATzero, p}$.
      \end{proof}

      If $n$ belongs to $\SETunionover \Ps$ then $n$ has a prime divisor.
      \begin{proof}
        Assume that $n$ belongs to $\SETunionover \Ps$.
        Take a prime number $r$ such that $n \in \N_{\NATzero, r}$.
        Hence $\NATcong{n}{\NATzero}{r}$.
        Thus $n \NATmod r = \NATzero \NATmod r = \NATzero$.
        Therefore $r$ is a prime divisor of $n$.
      \end{proof}
    End.

    Hence For all $n \in \Nat$ we have $n \in (\SETunionover \Ps)^\complement$ iff
    $n$ has no prime divisor.
    $\NATone$ has no prime divisor and any natural number having no prime
    divisor is equal to $\NATone$.
    Therefore $(\SETunionover \Ps)^\complement = \SETsingleton{\NATone}$.
    Indeed $((\SETunionover \Ps)^\complement) \SETinclude \SETsingleton{\NATone}$ and $\SETsingleton{\NATone}
    \SETinclude (\SETunionover \Ps)^\complement$. %!

    $\Ps$ is infinite.
    \begin{proof}[method=contradiction]
      Assume that $\Ps$ is finite.
      Then $\SETunionover \Ps$ is closed and $(\SETunionover \Ps)^\complement$ is open.
      Take a $p$ such that $\N_{\NATone, p} \SETinclude (\SETunionover \Ps)^\complement$.
      $\NATone \NATplus p$ is an element of $\N_{\NATone, p}$.
      Indeed $\NATcong{\NATone \NATplus p}{\NATone}{p}$
      (by \sn{congruency of sum}).
      $\NATone \NATplus p$ is not equal to $\NATone$.
      Hence $\NATone \NATplus p \notin (\SETunionover \Ps)^\complement$.
      Contradiction.
    \end{proof}
  \end{proof}
\end{forthel}

\printbibliography
\printcopyright{Andrei Paskevich (2008), Marcel Schütz (2021--2025)}
\end{smodule}
\end{document}
