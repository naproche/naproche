\documentclass{stex}
\libusepackage{naproche}
\libusepackage{copyright}
\libinput{preamble}

\title{The Barber Paradox in \Naproche}
\author{Marcel Schütz}
\date{2025}

\begin{document}
\begin{smodule}{barber-paradox.ftl}
\maketitle

\importmodule[libraries/everyday-ontology]{classes?person.ftl}
\importmodule[libraries/everyday-ontology]{actions?shave.ftl}

\noindent The \emph{Barber Paradox} is a puzzle illustrating Russell's
Paradox, formulated by Bertrand Russell in 1919 \cite[p. 355]{Russell1919}:

\begin{quotation}
  \noindent You can define the barber as ``one who shaves all those, and 
  those only, who do not shave themselves''.
  The question is, does the barber shave himself?
\end{quotation}

\noindent In \Naproche we can formalize this paradox as follows.

\begin{forthel}
  \begin{signature*}
    The barber is a person such that for any person $P$ the barber shaves $P$ iff $P$ does not shave himself.
  \end{signature*}
  
  \begin{theorem*}[title=Barber Paradox,id=barber_paradox]
    The barber shaves himself iff the barber does not shave himself.
  \end{theorem*}
\end{forthel}

\printbibliography
\printlicense[CcByNcSa]{Marcel Schütz (2024--2025)}
\end{smodule}
\end{document}
