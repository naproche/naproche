\documentclass{stex}
\libusepackage{naproche}
\libusepackage{copyright}
\libinput{preamble}

\title{Regularity of Successor Cardinals in \Naproche}
\author{Steffen Frerix, Adrian De Lon, Marcel Schütz}
\date{2025}

\begin{document}
\begin{smodule}{hausdorff.ftl}
\maketitle

\importmodule[libraries/foundations]{constant-maps.ftl}
\importmodule[libraries/set-theory]{cardinals-and-maps.ftl}
\importmodule[libraries/set-theory]{successor-cardinals.ftl}
\importmodule[libraries/set-theory]{regular-cardinals.ftl}

\symdecl*{Hessenberg}
\symdecl*{Hausdorff}

\noindent This is a formalization of a theorem of Felix Hausdorff stating that
successor cardinals are always regular.

In lack of a formalization in \Naproche of a proof of \emph{Hessenberg's
Theorem} \cite{Hessenberg1906}, i.e. the fact that
$\SETcard{X \times X} \Eq \SETcard{X}$ for every
infinite set $X$, which is required by the subsequent
proof of Hausdorff's Theorem, we postulate it as an axiom:

\begin{axiom*}[forthel,title=Hessenberg's Theorem,name=Hessenberg]
  $\SETcard{X \SETprod X} \Eq \SETcard{X}$ for every infinite set $X$.
\end{axiom*}

The following result appears in \cite[p.~443]{Hausdorff1908},
where Hausdorff mentions that the proof of the regularity of successor
cardinals is \textit{``ganz einfach''} (\textit{``very simple''})
and can be skipped.

\begin{forthel}
  \begin{theorem*}[title=Hausdorff,name=Hausdorff]
    Let $\kappa$ be a cardinal.
    Then $\CARDsucc{\kappa}$ is regular.
  \end{theorem*}
  \begin{proof}[method=contradiction]
    \begin{case}{$\kappa$ is finite.} \end{case}

    \begin{case}{$\kappa$ is infinite.}
      Assume the contrary.
      Take a cofinal subset $x$ of $\CARDsucc{\kappa}$ such that $\SETcard{x} \NotEq \CARDsucc{\kappa}$.
      Then $\SETcard{x} \ORDleq \kappa$.
      Take a surjective map $f$ from $\kappa$ onto $x$ (by \sn{existence condition for surjections}).
      Indeed $x$ and $\kappa$ are nonempty and $\SETcard{\kappa} \Eq \kappa$.
      Then $f(\xi) \Elem \CARDsucc{\kappa}$ for all $\xi \Elem \kappa$.
  
      Let us show that for all $z \Elem \CARDsucc{\kappa}$ if $z$ is nonempty then there exists a surjective map from $\kappa$ onto $z$.
        Let $z \Elem \CARDsucc{\kappa}$.
        Assume that $z$ is nonempty.
        $\kappa$ is nonempty.
        Hence the thesis (by \sn{existence condition for surjections}).
        Indeed $\SETcard{\kappa} \ORDgeq \SETcard{z}$.
      End.
  
      Define \[ g(z) =
        \begin{cases}
          \text{choose $h \FUNfromonto{\kappa}{z}$ in $h$}
          & : \text{$z$ has an element}
          \\
          \text{$\FUNconst{\kappa}{\ORDzero}$}
          & : \text{$z$ has no element}
        \end{cases}
      \] for $z \Elem \CARDsucc{\kappa}$.
  
      Let us show that for all $\xi, \zeta \Elem \kappa$ $g(f(\xi))$ is a map such that $\zeta \Elem\linebreak \Dom(g(f(\xi)))$.
        Let $\xi, \zeta \Elem \kappa$.
        If $f(\xi)$ has an element then $g(f(\xi))$ is a surjective map from $\kappa$ onto $f(\xi)$.
        If $f(\xi)$ has no element then $g(f(\xi)) \Eq\linebreak \FUNconst{\kappa}{\ORDzero}$.
        Hence $\Dom(g(f(\xi))) \Eq \kappa$.
        Therefore $\zeta \Elem \Dom(g(f(\xi)))$.
      End.
  
      [prover vampire]
      For all objects $\xi, \zeta$ we have $\xi, \zeta \Elem \kappa$ iff $(\xi, \zeta) \Elem \kappa \SETprod \kappa$.
      Define $h(\xi,\zeta) \DefEq g(f(\xi))(\zeta)$ for $(\xi,\zeta) \Elem \kappa \SETprod \kappa$.
      [prover eprover]

      Let us show that $h$ is surjective onto $\CARDsucc{\kappa}$.
  
        Every element of $\CARDsucc{\kappa}$ is an element of $\FUNim{h}{\kappa \SETprod \kappa}$.
        \begin{proof}
          Let $n$ be an element of $\CARDsucc{\kappa}$.
          Take an element $\xi$ of $\kappa$ such that $n \ORDless f(\xi)$.
          Take an element $\zeta$ of $\kappa$ such that $g(f(\xi))(\zeta) \Eq n$.
          Indeed $g(f(\xi))$ is a surjective map from $\kappa$ onto $f(\xi)$.
          Then $n \Eq h(\xi,\zeta)$.
        \end{proof}
  
        Every element of $\FUNim{h}{\kappa \SETprod \kappa}$ is an element of $\CARDsucc{\kappa}$.
        \begin{proof}
          Let $n$ be an element of $\FUNim{h}{\kappa \SETprod \kappa}$.
          We can take elements $a, b$ of $\kappa$ such that $n \Eq h(a,b)$.
          Indeed $\FUNim{h}{\kappa \SETprod \kappa} \Eq \CClass{h(a,b)}{a,b \Elem\linebreak \kappa}$.
          Indeed $\kappa \SETprod \kappa \Eq \CClass{(a,b)}{a,b \Elem \kappa}$.
          Then $n \Eq g(f(a))(b)$.
          Indeed $(a,b) \Elem \kappa \SETprod \kappa$ and $h(a,b) \Eq g(f(a))(b)$.
          $f(a)$ is an element of $\CARDsucc{\kappa}$.
          Every element of $f(a)$ is an element of $\CARDsucc{\kappa}$.
  
          \begin{case}{$f(a)$ has an element.}
            Then $g(f(a))$ is a surjective map from $\kappa$ onto $f(a)$.
            Hence $n \Elem f(a) \Elem \CARDsucc{\kappa}$.
            Thus $n \Elem \CARDsucc{\kappa}$.
          \end{case}
  
          \begin{case}{$f(a)$ has no element.}
            Then $g(f(a)) \Eq \FUNconst{\kappa}{\ORDzero}$.
            Hence $n$ is the empty set.
            Thus $n \Elem \CARDsucc{\kappa}$.
          \end{case}
        \end{proof}
  
        Hence $\FUNrange(h) \Eq \FUNim{h}{\kappa \SETprod \kappa} \Eq \CARDsucc{\kappa}$.
      End.
  
      Therefore $\SETcard{\CARDsucc{\kappa}} \ORDleq \SETcard{\kappa \SETprod \kappa}$ (by \sn{existence condition for surjections}).
      Indeed $\kappa \SETprod \kappa$ and $\CARDsucc{\kappa}$ are nonempty sets and $h$ is a surjective map from $\kappa \SETprod \kappa$ to $\CARDsucc{\kappa}$.
      Consequently $\CARDsucc{\kappa} \ORDleq \kappa$.
      Contradiction.
    \end{case}
  \end{proof}
\end{forthel}

\printbibliography
\printcopyright{Steffen Frerix (2018), Adrian De Lon (2021), Marcel Schütz (2024--2025)}
\end{smodule}
\end{document}
