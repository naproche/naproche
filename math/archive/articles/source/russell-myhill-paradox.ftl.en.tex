\documentclass{stex}
\libusepackage{naproche}
\libusepackage{copyright}
\libinput{preamble}

\title{The Russell-Myhill Paradox in \Naproche}
\author{Marcel Schütz}
\date{2025}

\begin{document}
\begin{smodule}{russell-myhill-paradox.ftl}

\usemodule[libraries/meta]{preliminaries.ftl}

\symdef{LogProd}[name=logical product]{\mathop{\sqcap}}
\symdecl*{proposition}
\symdecl*{true}
\symdecl*{system of propositions}
\symdecl*{Russell-Myhill Paradox}

\maketitle

\noindent The \emph{Russell-Myhill Paradox} is a paradox discovered by 
Bertrand Russell which states that two systems of propositions with the same 
\emph{logical product} are not necessarily equal.
It was discussed in Russell's 1903 \emph{Principles of Mathematics}
\cite[Appendix B]{Russell1903} and rediscovered in 1958 by John Myhill
\cite{Myhill1958}.

\begin{forthel}
  \begin{signature*}[for=proposition]
    A \emph{proposition} is an object.
  \end{signature*}

  \begin{signature*}[for=true]
    Let $p$ be a proposition.
    $p$ is \emph{true} is an atom.
  \end{signature*}

  \begin{definition*}[for=system of propositions]
    A \emph{system of propositions} is a class $P$ such that every element of $P$ is a proposition.
  \end{definition*}

  \begin{signature*}[for=logical product]
    Let $P$ be a system of propositions.
    The \emph{logical product of $P$} is a proposition $p$ such that $p$ is true iff every element of $P$ is true.
    Let $\emph{\LogProd P}$ denote the logical product of $P$.
  \end{signature*}
  
  \begin{theorem*}[title=Russell-Myhill Paradox,name=Russell-Myhill Paradox]
    It is wrong that
    \[ \LogProd P \Eq \LogProd Q \Implies P \Eq Q \]
    for all systems of propositions $P, Q$.
  \end{theorem*}
  \begin{proof}
    Assume the contrary.
    Define $Q = \CClass{q}{\text{there exists a system of propositions } P \text{ such that } q \Eq \LogProd P \text{ and } q \NotElem P}$.
    Consider $q = \LogProd Q$.
    Then for any system of propositions $P$ such that $\LogProd P \Eq q$ we have $P \Eq Q$.
    Hence $q \Elem Q$ iff $q \NotElem Q$.
    Contradiction.
  \end{proof}
\end{forthel}

\printbibliography
\printlicense[CcByNcSa]{Marcel Schütz (2024--2025)}
\end{smodule}
\end{document}
