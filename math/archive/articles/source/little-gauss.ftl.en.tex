\documentclass{stex}
\libusepackage{naproche}
\libusepackage{copyright}
\libinput{preamble}

\title{``Little Gauß''' Theorem in \Naproche}
\author{Christian Schöner, Marcel Schütz}
\date{2025}

\begin{document}
\begin{smodule}{little-gauss.ftl}
\maketitle

\importmodule[libraries/arithmetics]{multiplication.ftl}

\symdef{gauss}{\mathcal{G}}
\symdecl*{Little Gauss}

\noindent This is a formalization of ``Little Gauß''' Theorem, i.e. of
the assertion that
\[\sum_{k = \NATzero}^n k \Eq \frac{k(k \NATplus \NATone)}\NATtwo\]
for all $n \Elem \Nat$.
In \Naproche we can define the sum $\sum_{n = \NATzero}^k n$ by the function
$\gauss : \Nat \to \Nat$ with $\gauss(\NATzero) = \NATzero$ and
$\gauss(n \NATplus \NATone) = \gauss(n) \NATplus (n \NATplus \NATone)$:

\begin{forthel}
  Let $n$ denote a natural number.

  \begin{signature*}[for=gauss]
    $\gauss(n)$ is a natural number.
  \end{signature*}

  \begin{axiom*}
    $\gauss(\NATzero) \Eq \NATzero$.
  \end{axiom*}

  \begin{axiom*}
    $\gauss(n \NATplus \NATone) \Eq \gauss(n) \NATplus (n \NATplus \NATone)$.
  \end{axiom*}
\end{forthel}

\noindent Then the theorem can be stated as follows.

\begin{forthel}
  \begin{theorem*}[title=Little Gauß,name=Little Gauss]
    For all natural numbers $n$ we have
    \[\NATtwo \NATmul \gauss(n) \Eq n \NATmul (n \NATplus \NATone).\]
  \end{theorem*}
  \begin{proof}
    Define $\Phi = \SClass{n}{\Nat}{\NATtwo \NATmul \gauss(n) \Eq n \NATmul (n \NATplus \NATone)}$.
    
    (1) $\NATzero \Elem \Phi$.

    (2) For all $n \Elem \Phi$ we have $n \NATplus \NATone \Elem \Phi$.
    \begin{proof}
      Let $n \Elem \Phi$.
      Then $\NATtwo \NATmul \gauss(n) \Eq n \NATmul (n \NATplus \NATone)$.
      Hence
      \begin{align*}
              \NATtwo \NATmul \gauss(n \NATplus \NATone)
        & \Eq \NATtwo \NATmul (\gauss(n) \NATplus (n \NATplus \NATone)) \\
        & \Eq (\NATtwo \NATmul \gauss(n)) \NATplus (\NATtwo \NATmul (n \NATplus \NATone)) \\
        & \Eq (n \NATmul (n \NATplus \NATone)) \NATplus (\NATtwo \NATmul (n \NATplus \NATone)) \\
        & \Eq (n \NATplus \NATtwo) \NATmul (n \NATplus \NATone) \\
        & \Eq (n \NATplus \NATone) \NATmul (n \NATplus \NATtwo) \\
        & \Eq (n \NATplus \NATone) \NATmul ((n \NATplus \NATone) \NATplus \NATone)
      \end{align*}.
    \end{proof}
  \end{proof}
\end{forthel}

\printbibliography
\printlicense[CcByNcSa]{Christian Schöner (2024), Marcel Schütz (2024--2025)}
\end{smodule}
\end{document}
