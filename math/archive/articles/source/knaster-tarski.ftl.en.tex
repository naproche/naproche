\documentclass{stex}
\libusepackage{naproche}
\libusepackage{copyright}
\libinput{preamble}

\title{The Knaster-Tarski Fixed Point Theorem in \Naproche}
\author{Alexander Holz, Marcel Schütz}
\date{2025}

\begin{document}
\begin{smodule}{knaster-tarski.ftl}
\maketitle

\importmodule[libraries/foundations]{maps-and-systems-of-sets.ftl}
\importmodule[libraries/set-theory]{zfc.ftl}

\noindent This is a formalization of the \textit{Knaster-Tarski Fixed Point
Theorem}, i.e. of the assertion that every subset-preserving map has a fixed
point \cite{Schroeder2012}.

\begin{theorem}[forthel,title=Knaster-Tarski,id=knaster_tarski]
  Let $x$ be a set.
  Let $h$ be a map from $\SETpow(x)$ to $\SETpow(x)$ that preserves subsets.
  Then $h$ has a fixed point.
\end{theorem}
\begin{proof}[forthel]
  (1) Define $A = \SClass{y}{\SETpow(x)}{y \SETinclude h(y)}$.
  Then $A$ is a subset of $\SETpow(x)$.
  We have $\SETunionover A \Elem \SETpow(x)$.

  Let us show that (2) $\SETunionover A \SETinclude h(\SETunionover A)$.
    Let $u \Elem \SETunionover A$.
    Take $y \Elem A$ such that $u \Elem y$.
    Then $u \Elem h(y)$.
    We have $y \SETinclude \SETunionover A$.
    Hence $h(y) \SETinclude h(\SETunionover A)$.
    Indeed $h$ is a subset preserving map between systems of sets and $y, \SETunionover A \Elem \Dom(h)$.
    Thus $h(y) \SETinclude h(\SETunionover A)$.
    Therefore $u \Elem h(\SETunionover A)$.
  End.

  Then $h(\SETunionover A) \Elem A$ (by 1).
  Indeed $h(\SETunionover A) \SETinclude x$.
  (3) Hence $h(\SETunionover A) \SETinclude \SETunionover A$.
  Indeed every element of $h(\SETunionover A)$ is an element of some element of
  $A$.

  Thus $h(\SETunionover A) \Eq \SETunionover A$ (by 2, 3).
\end{proof}


\printbibliography
\printlicense[CcByNcSa]{Alexander Holz (2018), Marcel Schütz (2018--2025)}
\end{smodule}
\end{document}
