\documentclass{stex}
\libusepackage{naproche}
\libusepackage{copyright}
\libinput{preamble}

\title{``Socrates is Mortal'' in \Naproche}
\author{Marcel Schütz}
\date{2025}

\begin{document}
\begin{smodule}{socrates.ftl}
\maketitle

\importmodule[libraries/everyday-ontology]{individuals?socrates.ftl}
\importmodule[libraries/everyday-ontology]{axioms?all-humans-are-mortal.ftl}

\noindent Probably one of the most famous logical arguments is the following
syllogism, mentioned the first time by John Stuart Mill in 1843
\cite{Wheeler2023} in his \textit{System of Logic}
\cite[Book II, Chapter 3]{Mill1843}:
\begin{center}
  \begin{tabular}{c}
    All humans are mortal. \\
    Socrates is a human. \\
    \hline
    Therefore Socrates is mortal.
  \end{tabular}
\end{center}
We can formalize this syllogism in \Naproche without changing its wording:

\begin{forthel}
  \begin{theorem}[title=Mill's Syllogism,id=mill_syllogism]
    Socrates is mortal.
  \end{theorem}
  \begin{proof}
    All humans are mortal.
    Socrates is a human.
    Therefore Socrates is mortal.
  \end{proof}
\end{forthel}

A similar syllogism was given by Sextus Empiricus in his \textit{Outlines of
Pyrrhonism} \cite{SextusEmpiricus1933}:
\begin{center}
  \begin{tabular}{c}
    Socrates is human. \\
    Every human being is an animal. \\
    \hline
    Therefore Socrates is an animal.
  \end{tabular}
\end{center}
Again, this syllogism can be adopted word-by-word to \Naproche:

\begin{forthel}
  \begin{theorem}[title=Sextus Empiricus' Syllogism,id=sextus_empiricus_syllogism]
    Socrates is an animal.
  \end{theorem}
  \begin{proof}
    Socrates is human.
    Every human being is an animal.
    Therefore Socrates is an animal.
  \end{proof}
\end{forthel}

\printbibliography
\printlicense[CcByNcSa]{Marcel Schütz (2025)}

\end{smodule}
\end{document}
