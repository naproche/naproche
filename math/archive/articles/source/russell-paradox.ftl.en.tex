\documentclass{stex}
\libusepackage{naproche}
\libusepackage{copyright}
\libinput{preamble}

\title{Russell's Paradox in \Naproche}
\author{Marcel Schütz}
\date{2025}

\begin{document}
\begin{smodule}{russell-paradox.ftl}

\usemodule[libraries/meta]{preliminaries.ftl}

\symdecl*{Russell Paradox}

\maketitle

\noindent \emph{Russell's Paradox} is a set-theoretical paradox discovered by
Bertrand Russell around 1902 \cite[chapter XV]{Frege1980} which shows that 
there exist statements $\varphi$ whose extension
$\CClass{x}{\varphi(x)}$ cannot constitute a set -- or in other words:
Not every class is a set.

\begin{forthel}
  \begin{theorem}[title=Russell's Paradox,name=Russell Paradox]
    There exists a class that is not a set.
  \end{theorem}
  \begin{proof}[forthel]
    Assume the contrary.
    Define $R = \CClass{x}{x \text{ is a set such that } x \notin x}$.
    Then $R$ is a set.
    Hence $R \in R$ iff $R \notin R$.
    Contradiction.
  \end{proof}
\end{forthel}

\printbibliography
\printlicense[CcByNcSa]{Marcel Schütz (2024--2025)}
\end{smodule}
\end{document}
