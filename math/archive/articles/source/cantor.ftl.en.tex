\documentclass{stex}
\libusepackage{naproche}
\libusepackage{copyright}
\libinput{preamble}

\title{Cantor's Theorem in \Naproche}
\author{Marcel Schütz}
\date{2025}

\begin{document}
\begin{smodule}{cantor.ftl}
\maketitle

\importmodule[libraries/set-theory]{cardinals-and-maps.ftl}
\importmodule[libraries/foundations]{cantor.ftl}

\symdecl*{Cantor II}

\noindent This is a formalization of \emph{Cantor's Theorem}
\cite{Cantor1891}, i.e. of the assertion that any set $x$ is strictly smaller
than its powerset $\SETpow(x)$:

\begin{theorem}[forthel,title=Cantor's Theorem,name=Cantor II]
  Let $x$ be a set.
  Then $\SETcard{x} \ORDless \SETcard{\SETpow(x)}$.
\end{theorem}
\begin{proof}[forthel]
  \begin{case}{$x$ is empty.} \end{case}

  \begin{case}{$x$ is nonempty.}
    Assume the contrary.
    Then $\SETcard{x} \ORDgeq \SETcard{\SETpow(x)}$.
    Hence there exists a surjective map from $x$ onto $\SETpow(x)$ (by \sn{existence condition for surjections}).
    Indeed $\SETpow(x)$ is a nonempty set.
    Contradiction (by \sr{Cantor I}{Cantor (elementary version)}).
  \end{case}
\end{proof}

\printbibliography
\printlicense[CcByNcSa]{Marcel Schütz (2024--2025)}
\end{smodule}
\end{document}
