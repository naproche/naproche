\documentclass{stex}
\libusepackage{naproche}
\libusepackage{copyright}
\libinput{preamble}

\title{Hilbert's Paradox in \Naproche}
\author{Marcel Schütz}
\date{2025}

\begin{document}
\begin{smodule}{hilbert-paradox.ftl}
\maketitle

\importmodule[libraries/set-theory]{zfc.ftl}
\importmodule[libraries/foundations]{closure-under-arbitrary-unions.ftl}
\importmodule[libraries/foundations]{closure-under-powersets.ftl}
\importmodule[libraries/foundations]{universal-class.ftl}

\noindent \emph{Hilbert's Paradox}, discovered around 1900 by David Hilbert, 
demonstates that there cannot exists a set that is closed under powersets and 
under unions of arbitrary subsets \cite{PeckhausKahl2002}.

\begin{theorem}[forthel,title=Hilbert's Paradox,id=hilbert_paradox]
  There exists no system of sets that is closed under powersets and closed under unions.
\end{theorem}
\begin{proof}[forthel]
  Assume the contrary.
  Consider a system of sets $S$ that is closed under powersets and closed under unions.
  We have $S \SETinclude S$.
  Hence $\SETunionover S \Elem S$.
  Thus $\SETpow(\SETunionover S) \Elem S$.
  Contradiction.
\end{proof}

\noindent Using Hilbert's Paradox it can further be shown that there exists 
no universal set, i.e. that the class $\Set$ of all sets is a proper class:

\begin{proposition}[forthel,id=hilbert_paradox_corollary]
  $\Set$ is not a set.
\end{proposition}
\begin{proof}[forthel]
  Assume the contrary.
  Then $\Set$ is closed under powersets and closed under unions.
  Contradiction (by \nameref{hilbert_paradox}).
\end{proof}

\printbibliography
\printlicense[CcByNcSa]{Marcel Schütz (2024--2025)}
\end{smodule}
\end{document}
