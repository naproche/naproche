\documentclass{stex}
\libusepackage{naproche}
\libusepackage{copyright}
\libinput{preamble}
\libinput[libraries/meta]{preamble}

\title{The Ontology of \Naproche}
\author{Marcel Schütz}
\date{2025}

\begin{document}

\usemodule[libraries/meta]{preliminaries.ftl}

\maketitle

\begin{abstract}
  This document provides an overview of the object-level ontology of \Naproche.
  Moreover, it introduces common (symbolic and verbal) macros for the notions
  provided by this ontology which extend the core language of \ForTheL.
\end{abstract}

\ifstexhtml\else\tableofcontents\fi


\section{Introduction}\label{sec:introduction}

This document describes the notions and axioms hard-coded in \Naproche.
Furthermore, it presents some additional symbolic and verbal macros that extend
the core language of \ForTheL.
These macros are provided by the library \path{libraries/meta} which contains a
module \path{preliminaries.ftl} that exports all these macros.
Every library module (excluding those in the library \path{meta}) is expected to
import the module \path{preliminaries.ftl}.
So by importing any library module to your formalization, you automatically have
access to all macros described in this document.


\section{Propositions}\label{sec:propositions}

\ForTheL is essentially a surface language for (unsorted) first-order logic.
We therefore provide some symbolic notations for logical connectives and
quantifiers that are supported by \ForTheL:

\inputref[libraries/meta]{propositions/truth.ftl.en}
\inputref[libraries/meta]{propositions/falsity.ftl.en}
\inputref[libraries/meta]{propositions/negation.ftl.en}
\inputref[libraries/meta]{propositions/conjunction.ftl.en}
\inputref[libraries/meta]{propositions/disjunction.ftl.en}
\inputref[libraries/meta]{propositions/implication.ftl.en}
\inputref[libraries/meta]{propositions/equivalence.ftl.en}
\inputref[libraries/meta]{propositions/universal-quantification.ftl.en}
\inputref[libraries/meta]{propositions/existential-quantification.ftl.en}


\section{Notions}\label{sec:notions}

\subsection{Objects}\label{sec:objects}

\emph{Objects} are mathematical entities that are ``small'' enough to be
contained in classes.

\inputref[libraries/meta]{notions/objects.ftl.en}


\subsection{Ordered Pairs}\label{sec:ordered-pairs}

\inputref[libraries/meta]{notions/pairs.ftl.en}


\subsection{Classes and Sets}\label{sec:classes-and-sets}

\inputref[libraries/meta]{notions/classes.ftl.en}

Sets are considered to be classes that are ``small'' enough to be contained in
other classes:

\inputref[libraries/meta]{notions/sets.ftl.en}
\inputref[libraries/meta]{notions/elements.ftl.en}


\subsection{Comprehension Terms}\label{sec:comprehension-terms}

In \ForTheL we have three types of comprehension terms:
\begin{itemize}
  \item ``Enumeration classes'', i.e. terms of the form
  $\EClass{a_1,\dots,a_n}$, that represent the class that contains precisely the
  objects $a_1,\dots,a_n$.
  Typical examples are $\EClass{4,7,11}$ (i.e. the class consisting of the
  numbers $4$, $7$ and $11$) or $\EClass{x,y+1,z-x}$ (i.e. the class consisting
  of the elements $x$, $y+1$ and $z-x$ for some fixed objects $x$, $y$ and $z$).
  \item ``Comprehension classes'', i.e. terms of the form
  $\CClass{\tau}{\varphi}$, that represent the class of all objects of the form
  $\tau$ such that the components of $\tau$ satisfy the condition $\varphi$.
  Typical examples are $\CClass{p}{p \text{ is an odd prime number}}$ (i.e the
  class of all odd prime numbers) or $\CClass{(p,n)}{n \in \mathbb{N}
  \text{ and } p \text{ is a prime divisor of } n}$ (i.e. the class of all pairs
  $(p,n)$ of natural numbers such that $p$ is a prime divisor of $n$).
  \item ``Separation classes'', i.e. terms of the form
  $\SClass{\tau}{A}{\varphi}$, that represent the subclass of a class $A$ of all
  objects of the form $\tau$ such that the components of $\tau$ satisfy the
  condition $\varphi$.
  Typical examples are $\SClass{p}{\mathbb{P}}{p \text{ is odd}}$ (i.e the
  class of all odd prime numbers) or $\SClass{(p,n)}{\mathbb{N}\times\mathbb{N}}
  {p \text{ is a prime divisor of } n}$ (i.e. the class of all pairs
  $(p,n)$ of natural numbers such that $p$ is a prime divisor of $n$).
\end{itemize}

\inputref[libraries/meta]{notions/enumeration-class.ftl.en}
\inputref[libraries/meta]{notions/comprehension-class.ftl.en}
\inputref[libraries/meta]{notions/separation-class.ftl.en}


\subsection{Maps and Functions}\label{sec:maps-and-functions}

\inputref[libraries/meta]{notions/maps.ftl.en}

Similar to the notion of sets which are considered to be ``small'' classes, we
consider \emph{functions} to be ``small'' maps.

\inputref[libraries/meta]{notions/functions.ftl.en}
\inputref[libraries/meta]{notions/domain.ftl.en}
\inputref[libraries/meta]{notions/application.ftl.en}


\section{Relations}\label{sec:relations}

\inputref[libraries/meta]{relations/equality.ftl.en}

\Naproche has a hard-coded induction proof tactic (cf. section
\ref{sec:misc-axioms}) that relies on an ``induction ordering'' $\ILess$.
An important special case of this tactic is the following induction principle:
If, for all $x$, we can show that a predicate $\varphi[y]$ holds for every
$y \ILess x$, then $\varphi[x]$ holds for every $x$.
Typical instances of this principle are the (strong) induction principle from
natural number arithmetics (with $\ILess$ being the $<$-relation) or the
$\in$-induction principle from set theory (with $\ILess$ being the
$\in$-relation).

\inputref[libraries/meta]{relations/induction-order.ftl.en}


\section{Axioms}\label{sec:axioms}

\subsection{Extensionality Axioms}\label{sec:extensionality-axioms}

To provide equality conditions for ordered pairs, classes and maps, the
following extensionality axioms are hard-coded in \Naproche.

\inputref[libraries/meta]{axioms/pair-extensionality.ftl.en}
\inputref[libraries/meta]{axioms/class-extensionality.ftl.en}
\inputref[libraries/meta]{axioms/map-extensionality.ftl.en}


\subsection{Comprehension Term Axioms}\label{sec:comprehension-term-axioms}

The following axioms specify the elements of comprehension terms (cf. section
\ref{sec:comprehension-terms}).

\inputref[libraries/meta]{axioms/separation.ftl.en}
\inputref[libraries/meta]{axioms/enumeration-class.ftl.en}
\inputref[libraries/meta]{axioms/comprehension-class.ftl.en}


\subsection{Miscellaneous Axioms}\label{sec:misc-axioms}

Subclasses of sets that are formed via separation class terms (cf.
\ref{sec:comprehension-terms}) are again sets:

\inputref[libraries/meta]{axioms/separation-class.ftl.en}

As noted in section \ref{sec:maps-and-functions}, functions are considered to be
``small'' maps. Whether a map is small or not is determined by the ``size'' of
its domain:

\inputref[libraries/meta]{axioms/function-condition.ftl.en}

To facilitate proofs by induction, \Naproche provides a hard-coded induction
proof tactic which corresponds to the following induction principle.

\inputref[libraries/meta]{axioms/induction.ftl.en}


\printlicense[CcByNcSa]{Marcel Schütz (2025)}

\end{document}
