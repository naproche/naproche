\documentclass{stex}
\libusepackage{naproche}
\libinput{preamble}
\begin{document}
\begin{smodule}{omega-is-limit-ordinal.ftl}
\importmodule[libraries/set-theory]{omega.ftl}
\importmodule[libraries/set-theory]{limit-ordinals.ftl}

\begin{proposition}[forthel,id=SET_THEORY_03_764451995254784]
  $\ORDomega$ is a limit ordinal.
\end{proposition}
\begin{proof}[forthel]
  $\ORDomega$ is transitive.
  \begin{proof}
    Define $\Phi = \{ n \in \ORDomega \mid\text{ for all }m \SETin n\text{ we have }m \SETin \ORDomega \}$.

    (1) $\ORDzero \SETin \Phi$.

    (2) For all $n \SETin \Phi$ we have $\ORDsucc(n) \SETin \Phi$.
    \begin{proof}
      Let $n \SETin \Phi$.
      Then every element of $n$ is contained in $\ORDomega$.
      Hence every element of $\ORDsucc(n)$ is contained in $\ORDomega$.
      Thus $\ORDsucc(n) \SETin \Phi$.
    \end{proof}

    Therefore $\ORDomega \SETinclude \Phi$.
    Consequently $\ORDomega$ is transitive.
  \end{proof}

  Every element of $\ORDomega$ is an ordinal.
  Hence every element of $\ORDomega$ is transitive.
  Thus $\ORDomega$ is an ordinal.

  $\ORDomega$ is a limit ordinal.
  \begin{proof}
    Assume the contrary.
    We have $\ORDomega \neq \ORDzero$.
    Hence $\ORDomega$ is a successor ordinal.
    Take an ordinal $\alpha$ such that $\ORDsucc(\alpha) \eq \ORDomega$.
    Then $\alpha \SETin \ORDomega$.
    Thus $\ORDomega \eq \ORDsucc(\alpha) \SETin \ORDomega$.
    Contradiction.
  \end{proof}
\end{proof}
\end{smodule}
\end{document}
