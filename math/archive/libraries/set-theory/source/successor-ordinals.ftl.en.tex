\documentclass{stex}
\libusepackage{naproche}
\libinput{preamble}
\begin{document}
\begin{smodule}{successor-ordinals.ftl}
\importmodule[libraries/set-theory]{ordinals.ftl}
\symdef{ORDsucc}{\textrm{succ}}
\symdef{ORDpred}{\textrm{pred}}
\symdecl*{successor ordinal}

\begin{definition}[forthel,for=ORDsucc]
  Let $\alpha$ be an ordinal.
  $\emph{\ORDsucc(\alpha)} \DefEq \alpha \SETunion \SETsingleton{\alpha}$.
\end{definition}

\begin{proposition}[forthel]
  Let $\alpha$ be an ordinal.
  Then $\ORDsucc(\alpha)$ is an ordinal.
\end{proposition}
\begin{proof}[forthel]
  (1) $\ORDsucc(\alpha)$ is transitive.
  \begin{proof}
    Let $x \Elem \ORDsucc(\alpha)$ and $y \Elem x$.
    Then $x \Elem \alpha$ or $x \Eq \alpha$.
    Hence $y \Elem \alpha$.
    Thus $y \Elem \ORDsucc(\alpha)$.
  \end{proof}

  (2) Every element of $\ORDsucc(\alpha)$ is transitive.
  \begin{proof}
    Let $x \Elem \ORDsucc(\alpha)$.
    Then $x \Elem \alpha$ or $x \Eq \alpha$.
    Hence $x$ is transitive.
    Indeed $\alpha$ is transitive and every element of $\alpha$ is transitive.
  \end{proof}
\end{proof}

\begin{definition}[forthel,for=successor ordinal]
  A \emph{successor ordinal} is an ordinal $\alpha$ such that $\alpha \Eq \ORDsucc(\beta)$ for some ordinal $\beta$.
\end{definition}

\begin{proposition}[forthel]
  Let $\alpha, \beta$ be ordinals.
  If $\ORDsucc(\alpha) \Eq \ORDsucc(\beta)$ then $\alpha \Eq \beta$.
\end{proposition}
\begin{proof}[forthel]
  Assume $\ORDsucc(\alpha) \Eq \ORDsucc(\beta)$.

  (1) $\alpha \SETinclude \beta$.
  \begin{proof}
    Let $\gamma \Elem \alpha$.
    Then $\gamma \Elem \alpha \SETunion \SETsingleton{\alpha}
      \Eq \ORDsucc(\alpha)
      \Eq \ORDsucc(\beta)
      \Eq \beta \SETunion \SETsingleton{\beta}$.
    Hence $\gamma \Elem \beta$ or $\gamma \Eq \beta$.
    Assume $\gamma \Eq \beta$.
    Then $\beta \Elem \alpha$.
    Hence $\beta
      \Eq (\beta \SETunion \SETsingleton{\beta}) \SETdiff \SETsingleton{\gamma}
      \Eq (\alpha \SETunion \SETsingleton{\alpha}) \SETdiff \SETsingleton{\gamma}
      \Eq (\alpha \SETdiff \SETsingleton{\gamma}) \SETunion \SETsingleton{\alpha}$.
    Therefore $\alpha \Elem \beta$.
    Consequently $\alpha \Elem \beta \Elem \alpha$.
    Contradiction.
  \end{proof}

  (2) $\beta \SETinclude \alpha$.
  \begin{proof}
    Let $\gamma \Elem \beta$.
    Then $\gamma \Elem \beta \SETunion \SETsingleton{\beta}
      \Eq \ORDsucc(\beta)
      \Eq \ORDsucc(\alpha)
      \Eq \alpha \SETunion \SETsingleton{\alpha}$.
    Hence $\gamma \Elem \alpha$ or $\gamma \Eq \alpha$.
    Assume $\gamma \Eq \alpha$.
    Then $\alpha \Elem \beta$.
    Hence $\alpha
      \Eq (\alpha \SETunion \SETsingleton{\alpha}) \SETdiff \SETsingleton{\gamma}
      \Eq (\beta \SETunion \SETsingleton{\beta}) \SETdiff \SETsingleton{\gamma}
      \Eq (\beta \SETdiff \SETsingleton{\gamma}) \SETunion \SETsingleton{\beta}$.
    Therefore $\beta \Elem \alpha$.
    Consequently $\beta \Elem \alpha \Elem \beta$.
    Contradiction.
  \end{proof}
\end{proof}

\begin{definition}[forthel,for=ORDpred]
  Let $\alpha$ be a successor ordinal.
  $\emph{\ORDpred(\alpha)}$ is the ordinal $\beta$ such that $\alpha \Eq \ORDsucc(\beta)$.
\end{definition}
\end{smodule}
\end{document}
