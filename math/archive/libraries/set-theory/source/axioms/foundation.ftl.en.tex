\documentclass{stex}
\libusepackage{naproche}
\libinput{preamble}
\begin{document}
\begin{smodule}{foundation.ftl}
\importmodule[libraries/foundations]{systems-of-sets.ftl}

\symdecl*{Foundation Axiom}
\symdecl*{wellfoundedness of membership relation}
\symdecl*{epsilon induction}

\begin{axiom}[forthel,title=Foundation,name=Foundation Axiom]
  Let $X$ be a nonempty system of sets.
  Then $X$ has an element $x$ such that $X$ and $x$ are disjoint.
\end{axiom}

\begin{corollary}[forthel,name=wellfoundedness of membership relation]
  Let $X$ be a nonempty system of sets.
  Then $X$ has an element $x$ such that for no $y \Elem X$ we have $y \Elem x$.
\end{corollary}

\begin{proposition}[forthel,name=epsilon induction]
  Let $\Phi$ be a class.
  Assume that for all sets $x$ if $\Phi$ contains every element of $x$ that is a set then $\Phi$ contains $x$.
  Then $\Phi$ contains every set.
\end{proposition}
\begin{proof}[forthel]
  Assume the contrary.
  Define $M \DefEq \CClass{x}{x\text{ is a set such that }x \NotElem \Phi}$.
  Then $M$ is nonempty.
  Hence we can take a $x \Elem M$ such that for no $y \Elem M$ we have $y \Elem x$.
  Then $x$ is a set such that every element of $x$ that is a set is contained in $\Phi$.
  Thus $\Phi$ contains $x$.
  Contradiction.
\end{proof}
\end{smodule}
\end{document}
