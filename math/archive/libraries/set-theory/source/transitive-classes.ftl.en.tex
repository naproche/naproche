\documentclass{stex}
\libusepackage{naproche}
\libinput{preamble}
\begin{document}
\begin{smodule}{transitive-classes.ftl}
\importmodule[libraries/set-theory]{zfc.ftl}
\symdecl*{transitive}
\symdecl*{system of transitive sets}

\begin{definition}[forthel,for=transitive]
  Let $A$ be a class.
  $A$ is \emph{transitive} iff every element of $A$ is a subset of $A$.
\end{definition}

\begin{proposition}[forthel]
  Let $X$ be a system of sets.
  Then $X$ is transitive iff for every $x \Elem X$ and every $y \Elem x$ we have $y \Elem X$.
\end{proposition}

\begin{definition}[forthel,for=system of transitive sets]
  A \emph{system of transitive sets} is a system of sets $X$ such that every member of $X$ is a transitive set.
\end{definition}

\begin{proposition}[forthel]
  Every transitive class is a system of sets.
\end{proposition}

\begin{proposition}[forthel]
  Let $X$ be a system of sets.
  Then $X$ is transitive iff $\SETunionover X \SETinclude X$.
\end{proposition}
\begin{proof}[forthel]
  \begin{case}{$X$ is transitive.}
    Let $x \Elem \SETunionover X$.
    Take a member $y$ of $X$ such that $x \Elem y$.
    Then $y$ is a subset of $X$.
    Hence $x$ is an element of $X$.
  \end{case}

  \begin{case}{$\SETunionover X \SETinclude X$.}
    Let $x \Elem X$.

    Let us show that $x \SETinclude X$.
      Let $y \Elem x$.
      Then $y \Elem \SETunionover X$.
      Hence $y \Elem X$.
    End.
  \end{case}
\end{proof}

\begin{proposition}[forthel]
  Let $A$ be a transitive class.
  Then $\SETunionover A$ is transitive.
\end{proposition}
\begin{proof}[forthel]
  Let $x \Elem \SETunionover A$.

  Let us show that $x \SETinclude \SETunionover A$.
    Let $y \Elem x$.
    Take a member $z$ of $A$ such that $x \Elem z$.
    Then $z \SETinclude A$.
    Hence $x \Elem A$.
    Thus $y$ is an element of some member of $A$.
    Therefore $y \Elem \SETunionover A$.
  End.
\end{proof}

\begin{proposition}[forthel]
  Let $X$ be a system of transitive sets.
  Then $\SETunionover X$ is transitive.
\end{proposition}
\begin{proof}[forthel]
  Let $x \Elem \SETunionover X$ and $y \Elem x$.
  Take $z \Elem X$ such that $x \Elem z$.
  Then $z$ is transitive.
  Hence $x \SETinclude z$.
  Thus $y \Elem z$.
  Therefore $y \Elem \SETunionover X$.
\end{proof}

\begin{proposition}[forthel]
  Let $X$ be a system of transitive sets.
  Then $X \SETunion \SETunionover X$ is transitive.
\end{proposition}
\begin{proof}[forthel]
  Let $x \Elem X \SETunion \SETunionover X$.

  Let us show that $x \SETinclude X \SETunion \SETunionover X$.
    Let $u \Elem x$.
    We have $x \Elem X$ or $x \Elem \SETunionover X$.
    If $x \Elem X$then $u \Elem \SETunionover X$.
    If $x \Elem \SETunionover X$ then $u \Elem \SETunionover X$.
    Indeed $\SETunionover X$ is transitive.
    Hence $u \Elem \SETunionover X$.
    Thus $u \Elem X \SETunion \SETunionover X$.
  End.
\end{proof}

\begin{proposition}[forthel]
  Let $X$ be a system of sets.
  Then $X$ is transitive iff $X \SETinclude \SETpow(X)$.
\end{proposition}
\begin{proof}[forthel]
  \begin{case}{$X$ is transitive.}
    Let $x \Elem X$.
    Then $x \SETinclude X$.
    Hence $x \Elem \SETpow(X)$.
  \end{case}

  \begin{case}{$X \SETinclude \SETpow(X)$.}
    Let $x \Elem X$.
    Then $x \Elem \SETpow(X)$.
    Hence $x \SETinclude X$.
  \end{case}
\end{proof}

\begin{proposition}[forthel]
  Let $A$ be a transitive class.
  Then $\SETpow(A)$ is transitive.
\end{proposition}
\begin{proof}[forthel]
  Let $x \Elem \SETpow(A)$.
  Then $x \SETinclude A$.

  Let us show that $x \SETinclude \SETpow(A)$.
    Let $y \Elem x$.
    Then $y \Elem A$.
    Hence $y \SETinclude A$.
    Thus $y \Elem \SETpow(A)$.
  End.
\end{proof}
\end{smodule}
\end{document}
