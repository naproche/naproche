\documentclass{stex}
\libusepackage{naproche}
\libinput{preamble}
\begin{document}
\begin{smodule}{infinite-cardinals.ftl}
\importmodule[libraries/foundations]{proper-classes.ftl}
\importmodule[libraries/set-theory]{finite-and-infinite-sets.ftl}
\importmodule[articles]{cantor.ftl}
\symdef{Card}{\mathbb C}

\begin{definition}[forthel]
  $\emph{\Card}$ is the collection of all infinite cardinals.
\end{definition}

\begin{proposition}[forthel]
  Let $\alpha$ be an infinite ordinal.
  Then $\SETcard{\ORDsucc(\alpha)} \Eq \SETcard{\alpha}$.
\end{proposition}
\begin{proof}[forthel]
  For any $\beta \Elem \ORDsucc(\alpha)$ we have
  $\beta \ORDless \ORDomega$ or $\ORDomega \ORDleq \beta \ORDless \alpha$ or $\beta \Eq \alpha$.
  Define \[ f(\beta) =
    \begin{cases}
      \ORDsucc(\beta)  & : \beta \ORDless \ORDomega
      \\
      \beta         & : \ORDomega \ORDleq \beta \ORDless \alpha
      \\
      \ORDzero             & : \beta \Eq \alpha
    \end{cases} \]
  for $\beta \in \ORDsucc(\alpha)$.

  Then $f$ is a map from $\ORDsucc(\alpha)$ to $\alpha$.
  Indeed we can show that $f(\beta) \Elem \alpha$ for all
  $\beta \Elem \ORDsucc(\alpha)$.
  \begin{proof}
    Let $\beta \Elem \ORDsucc(\alpha)$.

    \begin{case}{$\beta \ORDless \ORDomega$.}
      Then $f(\beta)
        \Eq \ORDsucc(\beta)
        \ORDless \ORDomega
        \ORDleq \alpha$.
    \end{case}

    \begin{case}{$\ORDomega \ORDleq \beta \ORDless \alpha$.}
      Then $f(\beta)
        \Eq \beta
        \ORDless \alpha$.
    \end{case}

    \begin{case}{$\beta \Eq \alpha$.}
      Then $f(\beta)
        \Eq \ORDzero
        \ORDless \alpha$.
    \end{case}
  \end{proof}

  $f$ is surjective onto $\alpha$.
  Indeed we can show that for any $\beta \Elem \alpha$ there exists a
  $\gamma \Elem \ORDsucc(\alpha)$ such that $\beta \Eq f(\gamma)$.
  \begin{proof}
    Let $\beta \Elem \alpha$.
    Then $\beta \Eq \ORDzero$ or $\ORDzero \ORDless \beta \ORDless \ORDomega$ or $\beta \ORDgeq \ORDomega$.

    \begin{case}{$\beta \Eq \ORDzero$.}
      Then $\beta \Eq f(\alpha)$.
    \end{case}

    \begin{case}{$\ORDzero \ORDless \beta \ORDless \ORDomega$.}
      Take an ordinal $\beta'$ such that $\beta \Eq \ORDsucc(\beta')$.
      Then $\beta' \ORDless \ORDomega$.
      Hence $\beta \Eq f(\beta')$.
    \end{case}

    \begin{case}{$\beta \ORDgeq \ORDomega$.}
      Then $\beta \Eq f(\beta)$.
    \end{case}
  \end{proof}

  $f$ is injective.
  Indeed we can show that for all $\beta, \gamma \Elem \ORDsucc(\alpha)$ if
  $\beta \neq \gamma$ then $f(\beta) \neq f(\gamma)$.
  \begin{proof}
    Let $\beta, \gamma \Elem \ORDsucc(\alpha)$.
    Assume $\beta \neq \gamma$.

    \begin{case}{$\beta \ORDless \ORDomega$.}
      If $\gamma \Eq \alpha$ then
      $f(\beta)
        \Eq \ORDsucc(\beta)
        \neq \ORDzero
        \Eq f(\gamma)$.
      If $\ORDomega \ORDleq \gamma \ORDless \alpha$ then
      $f(\beta)
        \Eq \ORDsucc(\beta)
        \ORDless \ORDomega
        \ORDleq \gamma
        \Eq f(\gamma)$.
    \end{case}

    \begin{case}{$\ORDomega \ORDleq \beta \ORDless \alpha$.}
      If $\gamma \Eq \alpha$ then
      $f(\beta)
        \Eq \beta
        \ORDgeq \ORDomega
        \ORDgtr \ORDzero
        \Eq f(\gamma)$.
      If $\gamma \ORDless \ORDomega$ then
      $f(\beta)
        \Eq \beta
        \ORDgeq \ORDomega
        \ORDgtr \ORDsucc(\gamma)
        \Eq f(\gamma)$.
    \end{case}

    \begin{case}{$\beta \Eq \alpha$.}
      If $\gamma \ORDless \ORDomega$ then
      $f(\beta)
        \Eq \ORDzero
        \neq \ORDsucc(\gamma)
        \Eq f(\gamma)$.
      If $\ORDomega \ORDleq \gamma \ORDless \alpha$ then
      $f(\beta)
        \Eq \ORDzero
        \ORDless \ORDomega
        \ORDleq \gamma
        \Eq f(\gamma)$.
    \end{case}
  \end{proof}

  Hence $f$ is a bijection between $\ORDsucc(\alpha)$ and $\alpha$.
  Therefore $\ORDsucc(\alpha)$ and $\alpha$ are equinumerous.
  Consequently $\SETcard{\ORDsucc(\alpha)} \Eq \SETcard{\alpha}$.
\end{proof}

\begin{proposition}[forthel]
  Every infinite cardinal is a limit ordinal.
\end{proposition}
\begin{proof}[forthel]
  Let $\kappa$ be an infinite cardinal.
  Suppose that $\kappa$ is not a limit ordinal.
  $\kappa \neq \ORDzero$.
  Hence $\kappa$ is a successor ordinal.
  Indeed $\kappa$ is an ordinal.
  Thus we can take an ordinal $\alpha$ such that $\kappa \Eq \ORDsucc(\alpha)$.
  We have $\alpha \ORDgtr \kappa \ORDgeq \ORDomega$.
  Hence $\SETcard{\ORDsucc(\alpha)} \Eq \SETcard{\alpha}$.
  Thus $\alpha \ORDless \SETcard{\kappa}$ and $\kappa$ is equinumerous to $\kappa$.
  Contradiction.
\end{proof}
\end{smodule}
\end{document}
