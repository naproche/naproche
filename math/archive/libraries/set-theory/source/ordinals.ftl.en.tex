\documentclass{stex}
\libusepackage{naproche}
\libinput{preamble}
\begin{document}
\begin{smodule}{ordinals.ftl}
\importmodule[libraries/set-theory]{transitive-classes.ftl}
\symdef{Ord}{\mathbb O}
\symdecl*{ordinal number}
\symdecl*{ordinal}

\begin{definition}[forthel,for={ordinal number,ordinal}]
  An \emph{ordinal number} is a transitive set $\alpha$ such that every element of $\alpha$ is a transitive set.
  Let an \emph{ordinal} stand for an ordinal number.
\end{definition}

\begin{definition}[forthel,for=Ord]
  $\emph{\Ord}$ is the class of all ordinals.
\end{definition}

\begin{proposition}[forthel]
  Let $\alpha$ be an ordinal.
  Then every element of $\alpha$ is an ordinal.
\end{proposition}
\begin{proof}[forthel]
  Let $x$ be an element of $\alpha$.
  Then $x$ is transitive.

  Let us show that every element of $x$ is a subset of $x$.
    Let $y$ be an element of $x$.
    Then $y$ is a subset of $x$.
    Let $z$ be an element of $y$.
    Every element of $y$ is an element of $x$.
    Hence $z$ is an element of $x$.
  End.

  Thus every element of $x$ is transitive.
  Therefore $x$ is an ordinal.
\end{proof}
\end{smodule}
\end{document}
