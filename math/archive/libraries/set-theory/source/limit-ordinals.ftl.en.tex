\documentclass{stex}
\libusepackage{naproche}
\libinput{preamble}
\begin{document}
\begin{smodule}{limit-ordinals.ftl}
\importmodule[libraries/set-theory]{zero.ftl}
\importmodule[libraries/set-theory]{successor-ordinals.ftl}
\symdef{ORDlim}{\mathop{\textsf{lim}}}

\begin{convention}[forthel]
  Let $\ORDlim x$ stand for $\SETunionover x$.
\end{convention}

\begin{proposition}[forthel,id=SET_THEORY_02_7202164443185152]
  Let $x$ be a subset of $\Ord$.
  Then $\ORDlim x$ is an ordinal.
\end{proposition}
\begin{proof}[forthel]
  (1) $\ORDlim x$ is transitive.
  \begin{proof}
    Let $y \SETin \ORDlim x$ and $z \SETin y$.
    Take $w \SETin x$ such that $y \SETin w$.
    Hence $w$ is transitive.
    Thus $z \SETin w$.
    Therefore $z \SETin \ORDlim x$.
  \end{proof}

  (2) Every element of $\ORDlim x$ is transitive.
  \begin{proof}
    Let $y \SETin \ORDlim x$.
    Let $z \SETin y$ and $v \SETin z$.
    Take $w \SETin x$ such that $y \SETin w$.
    Hence $w$ is an ordinal.
    Thus $y$ is an ordinal.
    Therefore $y$ is transitive.
    Consequently $v \SETin y$.
  \end{proof}
\end{proof}

\begin{definition}[forthel,id=SET_THEORY_02_7678388934279168]
  A \emph{limit ordinal} is an ordinal $\lambda$ such that neither $\lambda$ is a successor ordinal nor $\lambda \eq \ORDzero$.
\end{definition}
\end{smodule}
\end{document}
