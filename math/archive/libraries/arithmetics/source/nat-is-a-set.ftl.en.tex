\documentclass{stex}
\libusepackage{naproche}
\libinput{preamble}
\begin{document}
\begin{smodule}{nat-is-a-set.ftl}
\importmodule[libraries/arithmetics]{natural-numbers.ftl}
\importmodule[libraries/set-theory]{axioms?infinity.ftl}

\begin{proposition}[forthel,id=ARITHMETIC_07_4685510236547454]
  $\Nat$ is a set.
\end{proposition}
\begin{proof}[forthel]
  Define $f(n) = n \NATplus \NATone$ for $n \in \Nat$.
  Then $f$ is a map from $\Nat$ to $\Nat$.
  Hence we can take a subset $X$ of $\Nat$ that is inductive regarding
  $\NATzero$ and $f$.
  Then $\NATzero \Elem X$ and for all $n \Elem X$ we have $n \NATplus \NATone \Elem X$.
  Hence $X$ contains every natural number.
  Thus we have $\Nat \SETinclude X$ and $X \SETinclude \Nat$.
  Therefore $\Nat \Eq X$.
  Consequently $\Nat$ is a set.
\end{proof}
\end{smodule}
\end{document}
