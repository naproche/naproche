\documentclass{stex}
\libusepackage{naproche}
\libinput{preamble}
\begin{document}
\begin{smodule}{natural-numbers.ftl}
\importmodule[libraries/foundations]{classes.ftl}
\symdef{Nat}{\mathbb N}
\symdef{NATplus}{\,+\,}
\symdef{NATzero}{0}
\symdef{NATone}{1}
\symdef{NATtwo}{2}
\symdef{NATthree}{3}
\symdef{NATfour}{4}
\symdef{NATfive}{5}
\symdef{NATsix}{6}
\symdef{NATseven}{7}
\symdef{NATeight}{8}
\symdef{NATnine}{9}

\begin{sfragment}{The Language of Natural Number Arithmetic}
  \begin{signature}[forthel,id=ARITHMETIC_01_3074681254969344]
    A \emph{natural number} is an object.
  \end{signature}

  \begin{definition}[forthel,id=ARITHMETIC_01_7367148418629632]
    $\emph{\Nat}$ is the class of natural numbers.
  \end{definition}

  \begin{signature}[forthel,id=ARITHMETIC_01_1567933815848960]
    Let $n, m$ be natural numbers.
    $\emph{n \NATplus m}$ is a natural number.

    Let the \emph{sum of $n$ and $m$} stand for $n \NATplus m$.
  \end{signature}

  \begin{signature}[forthel,id=ARITHMETIC_01_7633304715001856]
    $\emph{\NATzero}$ is a natural number.

    Let \emph{zero} stand for $\NATzero$.
    Let $n$ is \emph{nonzero} stand for $n \NotEq \NATzero$.
  \end{signature}

  \begin{signature}[forthel,id=ARITHMETIC_01_1200254566985232]
    $\emph{\NATone}$ is a natural number.

    Let \emph{one} stand for $\NATone$.
    Let the \emph{direct successor of $n$} stand for $n \NATplus \NATone$.
  \end{signature}

  \begin{definition}[forthel,id=ARITHMETIC_01_4584236572999680]
    $\emph{\NATtwo} = \NATone \NATplus \NATone$.

    Let \emph{two} stand for $\NATtwo$.
  \end{definition}

  \begin{definition}[forthel,id=ARITHMETIC_01_3836725109456896]
    $\emph{\NATthree} = \NATtwo \NATplus \NATone$.

    Let \emph{three} stand for $\NATthree$.
  \end{definition}

  \begin{definition}[forthel,id=ARITHMETIC_01_1709884968009728]
    $\emph{\NATfour} = \NATthree \NATplus \NATone$.

    Let \emph{four} stand for $\NATfour$.
  \end{definition}

  \begin{definition}[forthel,id=ARITHMETIC_01_6734726333202432]
    $\emph{\NATfive} = \NATfour \NATplus \NATone$.

    Let \emph{five} stand for $\NATfive$.
  \end{definition}

  \begin{definition}[forthel,id=ARITHMETIC_01_949139189792768]
    $\emph{\NATsix} = \NATfive \NATplus \NATone$.

    Let \emph{six} stand for $\NATsix$.
  \end{definition}

  \begin{definition}[forthel,id=ARITHMETIC_01_7245471749767168]
    $\emph{\NATseven} = \NATsix \NATplus \NATone$.

    Let \emph{seven} stand for $\NATseven$.
  \end{definition}

  \begin{definition}[forthel,id=ARITHMETIC_01_5658172888973312]
    $\emph{\NATeight} = \NATseven \NATplus \NATone$.

    Let \emph{eight} stand for $\NATeight$.
  \end{definition}

  \begin{definition}[forthel,id=ARITHMETIC_01_7371844250238976]
    $\emph{\NATnine} = \NATeight \NATplus \NATone$.

    Let \emph{nine} stand for $\NATnine$.
  \end{definition}
\end{sfragment}

\begin{sfragment}{The Axioms of Natural Number Arithmetic}
  \begin{axiom}[forthel,id=ARITHMETIC_01_3604163883696128]
    Let $n, m$ be natural numbers.
    If $n \NATplus \NATone \Eq m \NATplus \NATone$ then $n \Eq m$.
  \end{axiom}

  \begin{axiom}[forthel,id=ARITHMETIC_01_4454289938317312]
    There exists no natural number $n$ such that $n \NATplus \NATone \Eq \NATzero$.
  \end{axiom}

  \begin{axiom}[forthel,title=Induction,id=ARITHMETIC_01_4764664342773760]
    Let $\Phi$ be a class.
    Assume $\NATzero \Elem \Phi$ and for all natural numbers $n$ if $n \Elem \Phi$ then
    $n \NATplus \NATone \Elem \Phi$.
    Then $\Phi$ contains every natural number.
  \end{axiom}

  \begin{axiom}[forthel,id=ARITHMETIC_03_5962332515874540]
    Then $\NATone \Eq \NATzero \NATplus \NATone$.
  \end{axiom}

  \begin{axiom}[forthel,id=ARITHMETIC_03_4827955356237824]
    Let $n$ be a natural number.
    Then $n \NATplus \NATzero \Eq n$.
  \end{axiom}

  \begin{axiom}[forthel,id=ARITHMETIC_03_4427935386238924]
    Let $n, m$ be natural numbers.
    Then $n \NATplus (m \NATplus \NATone) \Eq (n \NATplus m) \NATplus \NATone$.
  \end{axiom}
\end{sfragment}

\begin{sfragment}{Immediate Consequences of the Axioms}
  \begin{proposition}[forthel,id=ARITHMETIC_01_4966080109871104]
    Let $n$ be a natural number.
    Then $n \Eq \NATzero$ or $n \Eq m \NATplus \NATone$ for some natural number $m$.
  \end{proposition}
  \begin{proof}[forthel]
    Define $\Phi = \SClass{n'}{\Nat}{n' \Eq \NATzero\text{ or }n' \Eq m' \NATplus \NATone\text{ for some natural number }m'}$.
    $\NATzero \Elem \Phi$ and for all $n' \Elem  \Phi$ we have $n' \NATplus \NATone \Elem \Phi$.
    Hence every natural number is contained in $\Phi$.
    Thus $n \Eq \NATzero$ or $n \Eq m \NATplus \NATone$ for some natural number $m$.
  \end{proof}

  \begin{proposition}[forthel,id=ARITHMETIC_01_5996049267163136]
    Let $n$ be a natural number.
    Then $n \NotEq n \NATplus \NATone$.
  \end{proposition}
  \begin{proof}[forthel]
    Define $\Phi = \SClass{n'}{\Nat}{n' \NotEq n' \NATplus \NATone}$.

    (1) $\NATzero$ belongs to $\Phi$.

    (2) For all $n' \Elem \Phi$ we have $n' \NATplus \NATone \Elem \Phi$.
    \begin{proof}
      Let $n' \Elem \Phi$.
      Then $n' \NotEq n' \NATplus \NATone$.
      If $n' \NATplus \NATone \Eq (n' \NATplus \NATone) \NATplus \NATone$ then $n' \Eq n' \NATplus \NATone$.
      Thus it is wrong that $n' \NATplus \NATone \Eq (n' \NATplus \NATone) \NATplus \NATone$.
      Hence $n' \NATplus \NATone \Elem \Phi$.
    \end{proof}

    Therefore every natural number is an element of $\Phi$.
    Consequently $n \NotEq n \NATplus \NATone$.
  \end{proof}
\end{sfragment}

\begin{sfragment}{Computation Laws for Addition}
  \begin{sfragment}{Associativity}
    \begin{proposition}[forthel,id=ARITHMETIC_03_3235893452210176]
      Let $n, m, k$ be natural numbers.
      Then $n \NATplus (m \NATplus k) \Eq (n \NATplus m) \NATplus k$.
    \end{proposition}
    \begin{proof}[forthel]
      Define $\Phi = \SClass{k'}{\Nat}{n \NATplus (m \NATplus k') \Eq (n \NATplus m) \NATplus k'}$.

      (1) $\NATzero$ is contained in $\Phi$.
      Indeed $n \NATplus (m \NATplus \NATzero) \Eq n \NATplus m \Eq (n \NATplus m) \NATplus \NATzero$.

      (2) For all $k' \Elem \Phi$ we have $k' \NATplus \NATone \Elem \Phi$.
      \begin{proof}
        Let $k' \Elem \Phi$.
        Then $n \NATplus (m \NATplus k') \Eq (n \NATplus m) \NATplus k'$.
        Hence
        \[  n \NATplus (m \NATplus (k' \NATplus \NATone))        \]
        \[    \Eq n \NATplus ((m \NATplus k') \NATplus \NATone)    \]
        \[    \Eq (n \NATplus (m \NATplus k')) \NATplus \NATone    \]
        \[    \Eq ((n \NATplus m) \NATplus k') \NATplus \NATone    \]
        \[    \Eq (n \NATplus m) \NATplus (k' \NATplus \NATone).   \]
        Thus $k' \NATplus \NATone \Elem \Phi$.
      \end{proof}

      Thus every natural number is an element of $\Phi$.
      Therefore $n \NATplus (m \NATplus k) \Eq (n \NATplus m) \NATplus k$.
    \end{proof}
  \end{sfragment}

  \begin{sfragment}{Commutativity}
    \begin{proposition}[forthel,id=ARITHMETIC_03_4029553232052224]
      Let $n, m$ be natural numbers.
      Then $n \NATplus m \Eq m \NATplus n$.
    \end{proposition}
    \begin{proof}[forthel]
      Define $\Phi = \SClass{m'}{\Nat}{n \NATplus m' \Eq m' \NATplus n}$.

      (1) $\NATzero$ is an element of $\Phi$.
      \begin{proof}
        Define $\Psi = \SClass{n'}{\Nat}{n' \NATplus \NATzero \Eq \NATzero \NATplus n'}$.

        (1a) $\NATzero$ belongs to $\Psi$.

        (1b) For all $n' \Elem \Psi$ we have $n' \NATplus \NATone \Elem \Psi$.
        \begin{proof}
          Let $n' \Elem \Psi$.
          Then $n' \NATplus \NATzero \Eq \NATzero \NATplus n'$.
          Hence
          \[  (n' \NATplus \NATone) \NATplus \NATzero        \]
          \[    \Eq n' \NATplus \NATone          \]
          \[    \Eq (n' \NATplus \NATzero) \NATplus \NATone    \]
          \[    \Eq (\NATzero \NATplus n') \NATplus \NATone    \]
          \[    \Eq \NATzero \NATplus (n' \NATplus \NATone).   \]
        \end{proof}

        Hence every natural number belongs to $\Psi$.
        Thus $n \NATplus \NATzero \Eq \NATzero \NATplus n$.
        Therefore $\NATzero$ is an element of $\Phi$.
      \end{proof}

      Let us show that (2) $n \NATplus \NATone \Eq \NATone \NATplus n$.
      \begin{proof}
        Define $\Theta = \SClass{n'}{\Nat}{n' \NATplus \NATone \Eq \NATone \NATplus n'}$.

        (2a) $\NATzero$ is an element of $\Theta$.

        (2b) For all $n' \Elem \Theta$ we have $n' \NATplus \NATone \Elem \Theta$.
        \begin{proof}
          Let $n' \Elem \Theta$.
          Then $n' \NATplus \NATone \Eq \NATone \NATplus n'$.
          Hence
          \[  (n' \NATplus \NATone) \NATplus \NATone        \]
          \[    \Eq (\NATone \NATplus n') \NATplus \NATone    \]
          \[    \Eq \NATone \NATplus (n' \NATplus \NATone).   \]
          Thus $n' \NATplus \NATone \Elem \Theta$.
        \end{proof}

        Thus every natural number belongs to $\Theta$.
        Therefore $n \NATplus \NATone \Eq \NATone \NATplus n$.
      \end{proof}

      (3) For all $m' \Elem \Phi$ we have $m' \NATplus \NATone \Elem \Phi$.
      \begin{proof}
        Let $m' \Elem \Phi$.
        Then $n \NATplus m' \Eq m' \NATplus n$.
        Hence
        \[  n \NATplus (m'  \NATplus \NATone)       \]
        \[    \Eq (n \NATplus m') \NATplus \NATone    \]
        \[    \Eq (m' \NATplus n) \NATplus \NATone    \]
        \[    \Eq m' \NATplus (n \NATplus \NATone)    \]
        \[    \Eq m' \NATplus (\NATone \NATplus n)    \]
        \[    \Eq (m' \NATplus \NATone) \NATplus n.   \]
        Thus $m' \NATplus \NATone \Elem \Phi$.
      \end{proof}

      Thus every natural number is an element of $\Phi$.
      Therefore $n \NATplus m \Eq m \NATplus n$.
    \end{proof}
  \end{sfragment}

  \begin{sfragment}{Cancellation}
    \begin{proposition}[forthel,id=ARITHMETIC_03_3137702874578944]
      Let $n, m, k$ be natural numbers.
      If $n \NATplus k \Eq m \NATplus k$ then $n \Eq m$.
    \end{proposition}
    \begin{proof}[forthel]
      Define $\Phi = \SClass{k'}{\Nat}{\text{ if }n \NATplus k' \Eq m \NATplus k'\text{ then }n \Eq m}$.

      (1) $\NATzero$ is an element of $\Phi$.

      (2) For all $k' \Elem \Phi$ we have $k' \NATplus \NATone \Elem \Phi$.
      \begin{proof}
        Let $k' \Elem \Phi$.
        Suppose $n \NATplus (k' \NATplus \NATone) \Eq m \NATplus (k' \NATplus \NATone)$.
        Then $(n \NATplus k') \NATplus \NATone \Eq (m \NATplus k') \NATplus \NATone$.
        Hence $n \NATplus k' \Eq m \NATplus k'$.
        Thus $n \Eq m$.
      \end{proof}

      Therefore every natural number is an element of $\Phi$.
      Consequently if $n \NATplus k \Eq m \NATplus k$ then $n \Eq m$.
    \end{proof}

    \begin{corollary}[forthel,id=ARITHMETIC_03_8445946379632640]
      Let $n, m, k$ be natural numbers.
      If $k \NATplus n \Eq k \NATplus m$ then $n \Eq m$.
    \end{corollary}
    \begin{proof}[forthel]
      Assume $k \NATplus n \Eq k \NATplus m$.
      We have $k \NATplus n \Eq n \NATplus k$ and $k \NATplus m \Eq m \NATplus k$.
      Hence $n \NATplus k \Eq m \NATplus k$.
      Thus $n \Eq m$.
    \end{proof}
  \end{sfragment}

  \begin{sfragment}{Zero Sums}
    \begin{proposition}[forthel,id=ARITHMETIC_03_3520602170195968]
      Let $n, m$ be natural numbers.
      If $n \NATplus m \Eq \NATzero$ then $n \Eq \NATzero$ and $m \Eq \NATzero$.
    \end{proposition}
    \begin{proof}[forthel]
      Assume $n \NATplus m \Eq \NATzero$.
      Suppose $n \NotEq \NATzero$ or $m \NotEq \NATzero$.
      Then we can take a $k \Elem \Nat$ such that $n \Eq k \NATplus \NATone$ or $m \Eq k \NATplus \NATone$.
      Hence there exists a natural number $l$ such that
      $n \NATplus m
        \Eq l \NATplus (k \NATplus \NATone)
        \Eq (l \NATplus k) \NATplus \NATone
        \NotEq \NATzero$.
      Contradiction.
    \end{proof}
  \end{sfragment}
\end{sfragment}
\end{smodule}
\end{document}
