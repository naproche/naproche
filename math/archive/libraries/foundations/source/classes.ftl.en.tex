\documentclass{stex}
\libusepackage{naproche}
\libinput{preamble}
\begin{document}
\begin{smodule}{classes.ftl}
\importmodule[libraries/meta]{preliminaries.ftl}

\symdef{SETinclude}{\;\subseteq\;}
\notation{SETinclude}{\;\subset\;}
\symdef{SETstrictinclude}{\;\subsetneq\;}
\symdef{SETniclude}{\;\supseteq\;}
\notation{SETniclude}{\;\supset\;}
\symdef{SETstrictniclude}{\;\supsetneq\;}
\symdef{SETempty}{\emptyset}
\symdef{SETpair}[args=2]{\comp{\{}#1\comp,#2\comp{\}}}
\symdef{SETsingleton}[args=1]{\comp{\{}#1\comp{\}}}
\symdef{SETunion}{\,\cup\,}
\symdef{SETintersect}{\,\cap\,}
\symdef{SETdiff}{\,\setminus\,}
\symdecl*{subclass}
\symdecl*{superclass}
\symdecl*{proper subclass}
\symdecl*{proper superclass}
\symdecl*{include}
\symdecl*{empty}
\symdecl*{nonempty}
\symdecl*{unordered pair}
\symdecl*{singleton class}
\symdecl*{unique element}
\symdecl*{union}
\symdecl*{intersection}
\symdecl*{complement}
\symdecl*{disjoint}

\begin{sfragment}{Sub- and Superclasses}
  \begin{definition}[forthel,for={subclass,SETinclude,superclass,SETniclude,proper subclass,SETstrictinclude,proper superclass,SETstrictniclude,include}]
    Let $A$ be a class.
    A \emph{subclass of $A$} is a class $B$ such that every element of $B$ is an
    element of $A$.
    Let $\emph{B \SETinclude A}$ stand for $B$ is a subclass of $A$.
    Let a \emph{superclass of $B$} stand for a class $A$ such that $B \SETinclude A$.
    Let $\emph{B \SETniclude A}$ stand for $B$ is a superclass of $A$.
    Let a \emph{proper subclass of $A$} stand for a subclass $B$ of $A$ such that $B \NotEq A$.
    Let $\emph{B \SETstrictinclude A}$ stand for $B$ is a proper subclass of $A$.
    Let a \emph{proper superclass of $B$} stand for a superclass $A$ of $B$ such that $A \NotEq B$.
    Let $\emph{B \SETstrictniclude A}$ stand for $B$ is a proper superclass of $A$.
    Let \emph{$A$ includes $B$} stand for $B \SETinclude A$.
    Let $B$ is \emph{included in $A$} stand for $B \SETinclude A$.
  \end{definition}

  \begin{proposition}[forthel]
    Let $A$ be a class.
    Then $A \SETinclude A$.
  \end{proposition}
  \begin{proof}[forthel]
    Every element of $A$ is contained in $A$.
    Therefore $A \SETinclude A$.
  \end{proof}

  \begin{proposition}[forthel]
    Let $A, B, C$ be classes.
    If $A \SETinclude B$ and $B \SETinclude C$ then $A \SETinclude C$.
  \end{proposition}
  \begin{proof}[forthel]
    Assume $A \SETinclude B$ and $B \SETinclude C$.
    Then every element of $A$ is contained in $B$ and every element of $B$ is contained in $C$.
    Hence every element of $A$ is contained in $C$.
    Thus $A \SETinclude C$.
  \end{proof}

  \begin{proposition}[forthel]
    Let $A, B$ be classes.
    If $A \SETinclude B$ and $B \SETinclude A$ then $A \Eq B$.
  \end{proposition}
  \begin{proof}[forthel]
    Assume $A \SETinclude B$ and $B \SETinclude A$.
    Then every element of $A$ is contained in $B$ and every element of $B$ is contained in $A$.
    Hence $A \Eq B$.
  \end{proof}
\end{sfragment}

\begin{sfragment}{The Empty Class}
  \begin{definition}[forthel,for={empty,nonempty}]
    Let $A$ be a class.
    $A$ is \emph{empty} iff $A$ has no elements.
    Let $A$ is \emph{nonempty} stand for $A$ is not empty.
  \end{definition}

  \begin{definition}[forthel,for=SETempty]
    $\emph{\SETempty} \DefEq \CClass{x}{x \NotEq x}$.
  \end{definition}

  \begin{proposition}[forthel]
    Let $A$ be a class.
    $A$ is empty iff $A \Eq \SETempty$.
  \end{proposition}
  \begin{proof}[forthel]
    We can show that $\SETempty$ is empty.
    Indeed any element $x$ of $\SETempty$ is nonequal to $x$.
    Hence if $A \Eq \SETempty$ then $A$ is empty.
    If $A$ is empty then $A$ and $\SETempty$ have no elements.
    Hence if $A$ is empty then $A \SETinclude \SETempty$ and $\SETempty \SETinclude A$.
    Thus if $A$ is empty then $A \Eq \SETempty$.
  \end{proof}

  \begin{corollary}[forthel]
    $\SETempty$ is empty.
  \end{corollary}

  \begin{corollary}[forthel]
    Let $A$ be a class.
    Then $\SETempty \SETinclude A$.
  \end{corollary}
  \begin{proof}[forthel]
    $\SETempty$ has no elements.
    Hence every element of $\SETempty$ is contained in $A$.
  \end{proof}
\end{sfragment}

\begin{sfragment}{Unordered Pairs}
  \begin{definition}[forthel,for={SETpair,unordered pair}]
    Let $a, b$ be objects.
    $\emph{\SETpair{a}{b}} \DefEq \CClass{x}{x \Eq a\text{ or }x \Eq b}$.
    Let the \emph{unordered pair of $a$ and $b$} stand for $\SETpair{a}{b}$.
  \end{definition}

  \begin{definition}[forthel,for=unordered pair]
    An \emph{unordered pair} is a class $A$ such that $A \Eq \SETpair{a}{b}$ for some distinct objects $a, b$.
  \end{definition}

  \begin{definition}[forthel,for={SETsingleton,singleton class}]
    Let $a$ be an object.
    $\emph{\SETsingleton{a}} \DefEq \CClass{x}{x \Eq a}$.
    Let the \emph{singleton class of $a$} stand for $\SETsingleton{a}$.
  \end{definition}

  \begin{definition}[forthel,for=singleton class]
    A \emph{singleton class} is a class $A$ such that $A \Eq \SETsingleton{a}$ for some object $a$.
  \end{definition}

  \begin{proposition}[forthel]
    Let $a, a', b, b'$ be objects.
    Assume $\SETpair{a}{b} \Eq \SETpair{a'}{b'}$.
    Then ($a \Eq a'$ and $b \Eq b'$) or ($a \Eq b'$ and $b \Eq a'$).
  \end{proposition}
  \begin{proof}[forthel]
    We have $a \Eq a'$ or $a \Eq b'$.
    If $a \Eq a'$ then $b \Eq b'$.
    If $a \Eq b'$ then $b \Eq a'$.
    Hence ($a \Eq a'$ and $b \Eq b'$) or ($a \Eq b'$ and $b \Eq a'$).
  \end{proof}

  \begin{corollary}[forthel]
    Let $a, a'$ be objects.
    If $\SETsingleton{a} \Eq \SETsingleton{a'}$ then $a \Eq a'$.
  \end{corollary}

  \begin{definition}[forthel,for=unique element]
    Let $A$ be a class.
    A \emph{unique element of $A$} is an element $a$ of $A$ such that for each $x \Elem A$ we have $x \Eq a$.
  \end{definition}

  \begin{proposition}[forthel]
    Let $A$ be a class.
    Then $A$ has a unique element iff $A \Eq \SETsingleton{a}$ for some object $a$.
  \end{proposition}
\end{sfragment}

\begin{sfragment}{Unions, Intersections, Complements}
  \begin{definition}[forthel,for={SETunion,union}]
    Let $A, B$ be classes.
    $\emph{A \SETunion B} \DefEq \CClass{x}{x \Elem A\text{ or }x \Elem B}$.
    Let the \emph{union of $A$ and $B$} stand for $A \SETunion B$.
  \end{definition}

  \begin{definition}[forthel,for={SETintersect,intersection}]
    Let $A, B$ be classes.
    $\emph{A \SETintersect B} \DefEq \CClass{x}{x \Elem A\text{ and }x \Elem B}$.
    Let the \emph{intersection of $A$ and $B$} stand for $A \SETintersect B$.
  \end{definition}

  \begin{definition}[forthel,for={SETdiff,complement}]
    Let $A, B$ be classes.
    $\emph{A \SETdiff B} \DefEq \CClass{x}{x \Elem A\text{ and }x \NotElem B}$.
    Let the \emph{complement of $B$ in $A$} stand for $A \SETdiff B$.
  \end{definition}
\end{sfragment}

\begin{sfragment}{Disjoint Classes}
  \begin{definition}[forthel,for=disjoint]
    Let $A, B$ be classes.
    $A$ and $B$ are \emph{disjoint} iff $A$ and $B$ have no common elements.
  \end{definition}

  \begin{proposition}[forthel]
    Let $A, B$ be classes.
    Then $A$ and $B$ are disjoint iff $A \SETintersect B$ is empty.
  \end{proposition}
\end{sfragment}
\end{smodule}
\end{document}
