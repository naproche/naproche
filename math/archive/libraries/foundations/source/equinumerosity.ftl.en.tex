\documentclass{stex}
\libusepackage{naproche}
\libinput{preamble}
\begin{document}
\begin{smodule}{equinumerosity.ftl}
\importmodule[libraries/foundations]{invertible-maps.ftl}

\begin{definition}[forthel]
  Let $A, B$ be classes.
  $A$ is \emph{equinumerous to $B$} iff there exists a bijection between $A$ and $B$.
\end{definition}

\begin{proposition}[forthel]
  Let $A$ be a class.
  Then $A$ is equinumerous to $A$.
\end{proposition}
\begin{proof}[forthel]
  $\FUNid{A}$ is a bijection between $A$ and $A$.
\end{proof}

\begin{proposition}[forthel]
  Let $A, B$ be classes.
  If $A$ and $B$ are equinumerous then $B$ and $A$ are equinumerous.
\end{proposition}
\begin{proof}[forthel]
  Assume that $A$ and $B$ are equinumerous.
  Take a bijection $f$ between $A$ and $B$.
  Then $\FUNinv{f}$ is a bijection between $B$ and $A$.
  Hence $B$ and $A$ are equinumerous.
\end{proof}

\begin{proposition}[forthel]
  Let $A, B, C$ be classes.
  If $A$ and $B$ are equinumerous and $B$ and $C$ are equinumerous then $A$ and $C$ are equinumerous.
\end{proposition}
\begin{proof}[forthel]
  Assume that $A$ and $B$ are equinumerous and $B$ and $C$ are equinumerous.
  Take a bijection $f$ between $A$ and $B$ and a bijection $g$ between $B$ and $C$.
  Then $g \FUNcomp f$ is a bijection between $A$ and $C$.
  Hence $A$ and $C$ are equinumerous.
\end{proof}
\end{smodule}
\end{document}
