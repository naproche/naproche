\documentclass{article}
\usepackage{naproche}
\libinput[meta-inf]{example-preamble}
\libinput[libraries]{foundations-preamble}

\title{Transfinite Recursion Theorem}
\author{\Naproche formalization:\\[0.5em]Marcel Schütz}
\date{2024}

\begin{document}
  \maketitle

  \usemodule[libraries]{set-theory?transfinite-induction.ftl}
  \usemodule[libraries]{set-theory?recursive-maps.ftl}
  
  \noindent This is a formalization of the \emph{Transfinite Recursion Theorem}
  (cf. \cite{Koepke2018}).
  It states that for any map $G : A^{< \infty} \to A$, where $A^{< \infty}$
  denotes the class of all maps $\alpha \to A$ for some ordinal $\alpha$, there
  exists a unique map $F: \Ord \to A$ that is \emph{recursive regarding} $G$,
  i.e. \[F(\alpha) = G(F \restriction \alpha)\] for all ordinals $\alpha$.

  \begin{forthel}
    \begin{lemma*}[title=Coincidence Lemma,id=coincidence]
      Let $A$ be a class and $G$ be a map from $A^{< \infty}$ to $A$.
      Let $H, H'$ be maps to $A$ that are recursive regarding $G$.
      Then \[ H(\alpha) = H'(\alpha) \] for all $\alpha \in \dom(H) \cap \dom(H')$.
    \end{lemma*}
    \begin{proof}
      Define $\Phi = \{ \alpha \in \Ord \mid$ if
      $\alpha \in \dom(H) \cap \dom(H')$ then $H(\alpha) = H'(\alpha) \}$.

      For all ordinals $\alpha$ if every ordinal less than $\alpha$ lies in $\Phi$ then $\alpha \in \Phi$. \\
      Proof.
        Let $\alpha \in \Ord$.
        Assume that every $y \in \alpha$ lies in $\Phi$.

        Let us show that if $\alpha \in \dom(H) \cap \dom(H')$ then
        $H(\alpha) = H'(\alpha)$.
          Suppose $\alpha \in \dom(H) \cap \dom(H')$.
          Then $\alpha \subseteq \dom(H), \dom(H')$.
          Indeed $\dom(H)$ and $\dom(H')$ are transitive classes.
          Hence for all $y \in \alpha$ we have $y \in \dom(H) \cap \dom(H')$.
          Thus $H(y) = H'(y)$ for all $y \in \alpha$.
          Therefore $H \restriction \alpha = H' \restriction \alpha$.
          $H$ and $H'$ are recursive regarding $G$.
          We have $H \restriction \alpha, H' \restriction \alpha \in A^{< \infty}$.
          Hence $H(\alpha)
            = G(H \restriction \alpha)
            = G(H' \restriction \alpha)
            = H'(\alpha)$.
        End.

        Thus $\alpha \in \Phi$.
      Qed.

      [prover vampire]
      Then $\Phi$ contains every ordinal (by \printref{SET_THEORY_02_8493935460614144}).
      Therefore we have $H(\alpha) = H'(\alpha)$ for all $\alpha \in \dom(H) \cap \dom(H')$.
    \end{proof}
  \end{forthel}

  \begin{forthel}
    \begin{theorem*}[title=Transfinite Recursion Theorem: Existence,id=recursion_existence]
      Let $A$ be a class and $G$ be a map from $A^{< \infty}$ to $A$.
      Then there exists a map $F$ from $\Ord$ to $A$ that is recursive regarding $G$.
    \end{theorem*}
    \begin{proof}
      Every ordinal is contained in the domain of some map $H$ to $A$ such that $H$ is recursive regarding $G$. \\
      Proof.
        Define \[ \Phi = \class{\alpha \in \Ord | \classtext{$\alpha$ is contained in the domain of some map to $A$ that is recursive regarding $G$}}. \]

        Let us show that for every ordinal $\alpha$ if every ordinal less than $\alpha$ lies in $\Phi$ then $\alpha \in \Phi$.
          Let $\alpha$ be an ordinal.
          Assume that every ordinal less than $\alpha$ lies in $\Phi$.
          Then for all $y \in \alpha$ there exists a map $h$ to $A$ such that $h$ is recursive regarding $G$ and $y \in \dom(h)$.
          Define $H'(y) =$ ``choose a map $h$ to $A$ such that $h$ is recursive regarding $G$ and $y \in \dom(h)$ in $h(y)$'' for $y \in \alpha$.
          Then $H'$ is a map from $\alpha$ to $A$.
          We have $H' = H' \restriction \alpha$.
          Define \[ H(\beta) =
            \begin{cases}
              H'(\beta)                 & : \beta < \alpha \\
              G(H' \restriction \beta)  & : \beta = \alpha
            \end{cases} \]
          for $\beta \in \succ(\alpha)$.
          
          Let us show that $H \restriction \beta \in A^{< \infty}$ for all $\beta \in \dom(H)$.
            Let $\beta \in \dom(H)$.
            $\dom(H \restriction \beta) = \beta$ and $(H \restriction \beta)(b) = H(b)$ for all $b \in \beta$.
            $H(b) \in A$ for all $b \in \beta$.
            Hence $H \restriction \beta$ is a map from $\beta$ to $A$.
          End.

          (a) $\dom(H)$ is a transitive subclass of $\Ord$.

          (b) For all $\beta \in \dom(H)$ we have $H(\beta) = G(H \restriction \beta)$. \\
          Proof.
            Let $\beta \in \dom(H)$.
            Then $\beta < \alpha$ or $\beta = \alpha$.

            Case $\beta < \alpha$.
              Choose a map $h$ to $A$ such that $h$ is recursive regarding $G$ and $\beta \in \dom(h)$ and $H'(\beta) = h(\beta)$.

              Let us show that for all $y \in \beta$ we have $h(y) = H(y)$.
                Let $y \in \beta$.
                Then $H(y) = H'(y)$.
                Choose a map $h'$ to $A$ such that $h'$ is recursive regarding $G$ and $y \in \dom(h')$ and $H'(y) = h'(y)$.
                [prover vampire]
                Then $h'(y) = h(y)$ (by \nameref{coincidence}).
                Indeed $y \in \dom(h) \cap \dom(h')$.
              End.

              Hence $h \restriction \beta = H \restriction \beta$.
              Thus $H(\beta)
                = H'(\beta)
                = h(\beta)
                = G(h \restriction \beta)
                = G(H \restriction \beta)$.
            End.

            Case $\beta = \alpha$.
              We have $H \restriction \alpha = H' \restriction \alpha$.
            End.
          Qed.

          Hence $H$ is a map to $A$ such that $H$ is recursive regarding $G$ and $\alpha \in \dom(H)$.
          Thus $\alpha \in \Phi$.
        End.

        [prover vampire]
        Therefore $\Phi$ contains every ordinal (by \printref{SET_THEORY_02_8493935460614144}).
        Consequently every ordinal is contained in the domain of some map $H$ to $A$ such that $H$ is recursive regarding $G$.
      Qed.

      Define $F(\alpha) =$ ``choose a map $H$ to $A$ such that $H$ is recursive regarding $G$ and $\alpha \in \dom(H)$ in $H(\alpha)$'' for $\alpha \in \Ord$.
      Then $F$ is a map from $\Ord$ to $A$.

      $F$ is recursive regarding $G$. \\
      Proof.
        (a) $\dom(F)$ is a transitive subclass of $\Ord$.

        (b) For all $\alpha \in \Ord$ we have $F(\alpha) = G(F \restriction \alpha)$. \\
        Proof.
          Let $\alpha \in \Ord$.
          Choose a map $H$ to $A$ such that $H$ is recursive regarding $G$ and $\alpha \in \dom(H)$ and $F(\alpha) = H(\alpha)$.

          Let us show that $F(\beta) = H(\beta)$ for all $\beta \in \alpha$.
            Let $\beta \in \alpha$.
            Choose a map $H'$ to $A$ such that $H'$ is recursive regarding $G$ and $\beta \in \dom(H')$ and $F(\beta) = H'(\beta)$.
            [prover vampire]
            Then $H(\beta) = H'(\beta)$ (by \nameref{coincidence}).
            Indeed $\beta \in \dom(H) \cap \dom(H')$.
            Therefore $F(\beta) = H'(\beta)$.
          End.

          Hence $H \restriction \alpha = F \restriction \alpha$.
          Thus $F(\alpha)
            = H(\alpha)
            = G(H \restriction \alpha)
            = G(F \restriction \alpha)$.
        Qed.
      Qed.
    \end{proof}
  \end{forthel}

  \begin{forthel}
    \begin{theorem*}[title=Transfinite Recursion Theorem: Uniqueness,id=recursion_uniqueness]
      Let $A$ be a class and $G$ be a map from $A^{< \infty}$ to $A$.
      Let $F, F'$ be maps from $\Ord$ to $A$ that are recursive regarding $G$.
      Then $F = F'$.
    \end{theorem*}
    \begin{proof}[]
      [prover vampire]
      $F$ and $F'$ are recursive regarding $G$.
      Then $F(\alpha) = F'(\alpha)$ for all $\alpha \in \dom(F) \cap \dom(F')$ (by \nameref{coincidence}).
      We have $\dom(F) = \Ord = \dom(F')$.
      Hence $F(\alpha) = F'(\alpha)$ for all $\alpha \in \Ord$.
      Thus $F = F'$.
    \end{proof}
  \end{forthel}

  As a corollary of the transfinite recursion theorem we get that we can
  define maps recursively on the ordinals by case distinction:
  For given maps $G : \Ord \times A \to A$ and
  $H : \Ord \times A^{< \infty} \to A$ and an element $a \in A$ we can define
  a map $F : \Ord \to A$ by
  \begin{itemize}
    \item $F(0) = a$,
    \item $F(\succ(\alpha)) = G(\alpha, F(\alpha))$, and
    \item $F(\lambda) = H(\lambda, F \restriction \lambda)$
      for any limit ordinal $\lambda$.
  \end{itemize}

  To establish the well-formedness of the conclusion of that corollary in
  \linebreak
  \Naproche, we need two additional lemmas:

  \begin{forthel}
    \begin{lemma*}
      Let $A$ be a class and $\alpha$ be an ordinal and $F : \Ord \to A$.
      Then $(\alpha, F(\alpha)) \in \Ord \times A$.
    \end{lemma*}

    \begin{lemma*}
      Let $A$ be a class and $\lambda$ be a limit ordinal and $F : \Ord \to A$.
      Then $(\lambda, F \restriction \lambda) \in \Ord \times A^{< \infty}$.
    \end{lemma*}
  \end{forthel}

  \begin{forthel}
    \begin{corollary*}[id=recursion]
      Let $A$ be a class.
      Let $a \in A$ and $G : \Ord \times A \to A$ and $H : \Ord \times A^{< \infty} \to A$.
      Then there exists a map $F$ from $\Ord$ to $A$ such that
      \[ F(0) = a \]
      and for all ordinals $\alpha$ we have
      \[ F(\succ(\alpha)) = G(\alpha, F(\alpha)) \]
      and for all limit ordinals $\lambda$ we have
      \[ F(\lambda) = H(\lambda, F \restriction \lambda). \]
    \end{corollary*}
    \begin{proof}
      $(\pred(\dom(f)), f(\pred(\dom(f)))) \in \Ord \times A$ for all $f \in A^{< \infty}$ such that $\dom(f)$ is a successor ordinal.

      Define  \[ J(f) =
        \begin{cases}
          a
          & : \dom(f) = 0
          \\
          G(\pred(\dom(f)), f(\pred(\dom(f))))
          & : \text{$\dom(f)$ is a successor ordinal}
          \\
          H(\dom(f), f)
          & : \text{$\dom(f)$ is a limit ordinal}
        \end{cases} \]
      for $f \in A^{< \infty}$.

      Then $J$ is a map from $A^{< \infty}$ to $A$.
      Indeed we can show that for any $f \in A^{< \infty}$ we have $J(f) \in A$.
        Let $f \in A^{< \infty}$.
        Take $\alpha \in \Ord$ such that $f : \alpha \to A$.
        If $\alpha = 0$ then $J(f) = a \in A$.
        If $\alpha$ is a successor ordinal then $J(f) =
        G(\pred(\alpha), f(\pred(\alpha))) \in A$.
        [prover vampire]
        If $\alpha$ is a limit ordinal then $J(f) = H(\alpha, f) \in A$.
      End.

      Hence we can take a map $F$ from $\Ord$ to $A$ that is recursive regarding $J$.
      Then $F \restriction \alpha \in A^{< \infty}$ for any ordinal $\alpha$.

      (1) $F(0) = a$. \\
      Proof.
        $F(0)
          = J(F \restriction 0)
          = a$.
      Qed.

      (2) $F(\succ(\alpha)) = G(\alpha, F(\alpha))$ for all ordinals $\alpha$. \\
      Proof.
        Let $\alpha$ be an ordinal.
        Then $F(\succ(\alpha))
          = J(F \restriction \succ(\alpha))
          = G(\pred(\succ(\alpha)), (F \restriction \succ(\alpha))(\pred(\succ(\alpha))))
          = G(\alpha, (F \restriction \succ(\alpha))(\alpha))
          = G(\alpha, F(\alpha))$.
      Qed.

      (3) $F(\lambda) = H(\lambda, F \restriction \lambda)$ for all limit ordinals $\lambda$. \\
      Proof.
        Let $\lambda$ be a limit ordinal.
        Then $F(\lambda)
          = J(F \restriction \lambda)
          = H(\lambda, F \restriction \lambda)$.
      Qed.

      Hence the thesis (by 1, 2, 3).
    \end{proof}
  \end{forthel}

  \printbibliography

  \section*{Copyright}
  \doclicenseThis
\end{document}
