\documentclass[11pt]{article}
\usepackage{amssymb}
\usepackage{url}
\usepackage{../lib/tex/naproche}

\author{Peter Koepke, University of Bonn}

\title{Ontological Axioms for \Naproche}

\begin{document}

\newcommand{\Naproche}{$\mathbb{N}$aproche}
\maketitle

\section{Basic Ideas}

The natural proof assistant \Naproche{} is intended to 
approximate and support ordinary mathematical practices. 
\Naproche{} employs the controlled natural language ForTheL
as its input language. Some ForTheL notions are already
built into \Naproche{}, as well as some basic properties of
these notions.
There are
mathematical objects, and sets and classes of mathematical
objects. Sets are classes which are objects themselves and
can thus be used in further mathematical constructions. Functions
and maps map objects to objects, where functions are those
maps which are objects.

Modelling mathematical notions by objects corresponds
to the intuition that numbers, points, etc. should
not have internal set-theoretical
structurings, contrary to purely set-theoretical
foundations of mathematics. This is also advantageous
for automated proving since it prevents proof searches to
dig into non-informative internal structurings.

As in common mathematics, sets and functions are required to
satisfy separation and replacement properties known from the
set theories of Kelley-Morse or Zermelo-Fraenkel.

The ontology presented here is more hierarchical than
ad-hoc first-order axiomatizations in some previous
\Naproche{} formalizations. Sometimes
this results in more complex and longer ontological checking tasks.
Controlling the complexitiy of automated proofs will be a major
issue for the further development.

\section{Importing Some Mathematical Language}
We import singular/plural forms of words that will be used in
our formalizations (\path{examples/vocabulary.ftl}). 
In the long run this should be replaced by
employing a proper English vocabulary. We also 
import some alternative formulations for 
useful mathematical phrases (\path{examples/macros.ftl}).
\begin{forthel}
[read vocabulary.ftl]

[read macros.ftl]
\end{forthel}

\section{Sets and Classes}
The notions of \textit{classes} and \textit{sets} are already
provided by \Naproche{}. Classes are usually defined by
abstraction terms $\{\dots\mid\dots\}$. Since such terms need to be
processed by the parser we cannot introduce them simply by first-order
definitions but have to implement them in the software.
That sets are classes which are objects, or extensionality can be proved
from inbuilt assumptions, but it is important to reprove or restate such
facts so that they can directly be accessed by the ATP.

\begin{forthel}

\begin{lemma} Every set is a class. \end{lemma}

\begin{lemma} Every class that is an object is a set. \end{lemma}

\begin{axiom} Every element of every class is an object.
\end{axiom}

Let $S,T$ denote classes.

\begin{definition} The empty set is the set that has
no elements.
\end{definition}

\begin{definition}
A subclass of $S$ is a class $T$ such that every $x \in T$
belongs to $S$.
\end{definition}

Let $T \subseteq S$ stand for $T$ is a subclass of $S$.

\begin{lemma} [Class Extensionality] If $S$ is a subclass of $T$ and
$T$ is a subclass of $S$ then $S = T$.
\end{lemma}

\begin{definition}
A subset of $S$ is a set $X$ such that $X \subseteq S$.
\end{definition}

\begin{axiom} [Separation Axiom] Assume that $X$ is a set.
Let $T$ be a subclass of $X$. Then $T$ is a set.
\end{axiom}

\end{forthel}
\section{Pairs and Products}
Since we prefer objects over sets if possible, we do not work
with Kuratowski-style set-theoretical ordered pairs, but
axiomatize them as objects.

\begin{forthel}

\begin{axiom}
$(a,b)$ is an object for all objects $a,b$.
\end{axiom}

\begin{axiom}
For any objects $a,b,c,d$ if $(a,b) = (c,d)$ then $a = c$
and $b = d$.
\end{axiom}

\begin{definition}
$S \times T = \{(x,y) | x \in S$ and $y \in T\}$.
\end{definition}

\begin{axiom} Let $X,Y$ be sets. Then
$X \times Y$ is a set.
\end{axiom}

\begin{lemma}
Let $x,y$ be objects.
If $(x,y)$ is an element of $S \times T$ then $x$ is an
element of $S$ and $y$ is an element of $T$.
\end{lemma}

\end{forthel}
\section{Functions and Maps}
The treatment of functions and maps is similar to that
of sets and classes.
\begin{forthel}

\begin{lemma} Every function is a map.
\end{lemma}

\begin{lemma} Every map that is an object is a function.
\end{lemma}

Let $f,g$ stand for maps.

\begin{lemma} The domain of every map is a class.
\end{lemma}

\begin{lemma} Let $x \in Dom(f)$. Then $f(x)$ is an object.
\end{lemma}

\begin{lemma} [Functional Extensionality] Assume $Dom(f)=Dom(g)$ and for every $x \in Dom(f)$
$f(x)=g(x)$. Then $f=g$.
\end{lemma}

\begin{definition}
Assume $S$ is a subclass of the domain of $f$.
$f[S] = \{ f(x) | x \in S \}$.
\end{definition}

\begin{definition}
$f : S \rightarrow T$ iff $Dom (f) = S$ and $f[S] \subseteq T$.
\end{definition}

\begin{axiom} [Replacement Axiom]
Assume $X$ is a subset of the domain of $f$.
Then $f[X]$ is a set.
\end{axiom}

\end{forthel}

\end{document}
