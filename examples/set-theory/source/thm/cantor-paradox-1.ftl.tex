\documentclass{article}
\usepackage[lang=english,stex]{naproche}
\libinput{preamble}
\libinput[naproche/examples/foundations]{preamble}
\begin{document}
\begin{smodule}{cantor-paradox-1.ftl}
  \importmodule[naproche/examples/set-theory]{thm?burali-forti-paradox.ftl}

  \begin{forthel}
    %[prove off][check off]
    [readtex \path{set-theory/sections/06_cardinals.ftl.tex}]
    %[prove on][check on]
  \end{forthel}

  \begin{ftheorem*}[label=cantor_paradox_1,title=Cantor's First Paradox]
    There exists no set that contains all cardinals.
  \end{ftheorem*}
  \begin{fproof}
    Assume the contrary.
    Consider a set $X$ that contains all cardinals.
    Define $C = \fclass{\kappa \in X}{\text{$\kappa$ is a cardinal}}$.
    Define $O = \fclass{\alpha}{\text{$\alpha$ is an ordinal}}$.

    Let us show that $O \subseteq \bigcup C$.
      Let $\alpha \in O$.
      Choose a cardinal $\kappa$ greater than $\alpha$.
      Then $\alpha \in \kappa \in C$.
      Thus $\alpha \in \bigcup C$.
    End.

    $\bigcup C$ is a set.
    Hence $O$ is a set.
    Contradiction (by \printref{burali_forti_paradox}).
    Indeed $O$ contains every ordinal.
  \end{fproof}
\end{smodule}
\end{document}
