\documentclass[../set-theory.tex]{subfiles}

\begin{document}
  \chapter{Finite and infinite sets}\label{chapter:finite-and-infinite-sets}

  \filename{set-theory/sections/07_cardinals.ftl.tex}

  \begin{forthel}
    %[prove off][check off]

    [readtex \path{set-theory/sections/03_finite-ordinals.ftl.tex}]

    [readtex \path{set-theory/sections/06_cardinals.ftl.tex}]

    %[prove on][check on]
  \end{forthel}


  \begin{forthel}
    \begin{proposition}\printlabel{SET_THEORY_07_6139396896063488}
      $\card{\emptyset} = 0$.
    \end{proposition}
  \end{forthel}

  \begin{forthel}
    \begin{proposition}\printlabel{SET_THEORY_07_836893598023680}
      Let $a$ be an object.
      Then $\card{\set{a}} = 1$.
    \end{proposition}
    \begin{proof}
      Define $f(x) = 0$ for $x \in \set{a}$.
      Then $f$ is a map from $\set{a}$ to $1$.
      $f$ is injective and surjective onto $1$.
      Hence $f$ is a bijection between $\set{a}$ and $1$.
      Consequently $\set{a}$ and $1$ are equinumerous.
      Thus $\card{\set{a}} = \card{1} = 1$.
    \end{proof}
  \end{forthel}

  \begin{forthel}
    \begin{proposition}\printlabel{SET_THEORY_07_5465279026954240}
      Let $a, b$ be distinct objects.
      Then $\card{\set{a,b}} = 2$.
    \end{proposition}
    \begin{proof}
      Define \[ f(x) =
        \begin{cases}
          0 & x = a
          \\
          1 & x = b
        \end{cases} \]
      for $x \in \set{a,b}$.
      Then $f$ is a map from $\set{a,b}$ to $2$.
      $f$ is injective and surjective onto $2$.
      Hence $f$ is a bijection between $\set{a,b}$ and $2$.
      Consequently $\set{a,b}$ and $2$ are equinumerous.
      Thus $\card{\set{a,b}} = \card{2} = 2$.
    \end{proof}
  \end{forthel}

  \begin{forthel}
    \begin{theorem}\printlabel{SET_THEORY_07_2948332552978432}
      Let $n \in \omega$.
      Then $\card{n} = n$.
    \end{theorem}
    \begin{proof}
      Define $\Phi = \{ n' \in \omega \mid \card{n'} = n' \}$.

      (1) $0 \in \Phi$.
      Indeed $\card{0} = \card{\emptyset} = 0$.

      (2) For all $n' \in \Phi$ we have $\succ{n'} \in \Phi$. \\
      Proof.
        Let $n' \in \Phi$.
        Then $\card{n'} = n'$.
        We have $\card{\succ{n'}} \leq \succ{n'}$.

        Let us show that $\succ{n'} \leq \card{\succ{n'}}$.
          Assume the contrary.
          Then $\card{\succ{n'}} \less \succ{n'}$.
          Take a bijection $f$ between $\card{\succ{n'}}$ and $\succ{n'}$.
          $\card{\succ{n'}}$ is nonzero.
          Hence we can take a $m \in \omega$ such that $\card{\succ{n'}} =
          \succ{m}$.
          Then $\inverse{f}(n') \leq m$.

          We can show that $\inverse{f}(n') \less m$.
            Assume the contrary.
            Then $\inverse{f}(n') = m$.
            $f \restriction m$ is a bijection between $m$ and $\image{f}(m)$ (by
            \cref{FOUNDATIONS_08_647446231252992}).
            Indeed $f$ is an injective map from $\card{\succ{n'}}$ to $\succ{n'}$ and
            $m \subseteq \card{\succ{n'}}$.
            We have $\image{f}(m) \subseteq n'$ and $n' \subseteq \image{f}(m)$.
            Hence $\image{f}(m) = n'$.
            Thus $f \restriction m$ is a bijection between $m$ and $n'$.
            Therefore $n'
              = \card{n'}
              \leq m
              \less \card{\succ{n'}}
              \less \succ{n'}$.
            Consequently $m = n'$.
            Then we have $\succ{n'} = \card{\succ{n'}} \less \succ{n'}$.
            Contradiction.
          End.

          Define \[ g(i) =
            \begin{cases}
              f(i)  & : i \neq \inverse{f}(n')
              \\
              f(m)  & : i = \inverse{f}(n')
            \end{cases} \]
          for $i \in m$.

          $g$ is a map from $m$ to $n'$.
          Indeed we can show that $g(i) \in n'$ for each $i \in m$. \\
          Proof.
            Let $i \in m$.

            Case $i \neq \inverse{f}(n')$.
              Then $g(i) = f(i) \in \succ{n'}$.
              If $g(i) = n'$ then $f(i) = n' = f(\inverse{f}(n'))$.
              Hence if $g(i) = n'$ then $i = \inverse{f}(n')$.
              Thus $g(i) \neq n'$.
              Therefore $g(i) \in n'$.
            End.

            Case $i = \inverse{f}(n')$.
              Then $g(i)
                = f(m)
                \neq f(\inverse{f}(n'))
                = n'$.
              Hence $g(i) \in n'$.
            End.
          Qed.

          $g$ is surjective onto $n'$.
          Indeed we can show that for all $k \in n'$ there exists a $l \in m$
          such that $k = g(l)$. \\
          Proof.
            Let $k \in n'$.
            Then $\inverse{f}(k) \neq \inverse{f}(n')$.

            Case $\inverse{f}(k) = m$.
              Then $k
                = f(\inverse{f}(k))
                = f(m)
                = g(\inverse{f}(n'))$.
            End.

            Case $\inverse{f}(k) \neq m$.
              Then $\inverse{f}(k) \in m$.
              Indeed $\inverse{f}(k) \in \card{\succ{n'}} = \succ{m} = m \cup \set{m}$.
              Hence $k
                = f(\inverse{f}(k))
                = g(\inverse{f}(k))$.
            End.
          Qed.

          $g$ is injective.
          Indeed we can show that for all $i, j \in m$ if $i \neq j$ then
          $g(i) \neq g(j)$. \\
          Proof.
            Let $i, j \in m$.
            Assume $i \neq j$.

            Case $i, j \neq \inverse{f}(n')$.
              Then $g(i)
                = f(i)
                \neq f(j)
                = g(j)$.
            End.

            Case $i = \inverse{f}(n')$.
              Then $j \neq \inverse{f}(n')$.
              Hence $g(i)
                = g(\inverse{f}(n'))
                = f(m)
                \neq f(j)
                = g(j)$.
              Indeed $m \neq j$.
            End.

            Case $j = \inverse{f}(n')$.
              Then $i \neq \inverse{f}(n')$.
              Hence $g(i)
                = f(i)
                \neq f(m)
                = g(\inverse{f}(n'))
                = g(j)$.
              Indeed $i \neq m$.
            End.
          Qed.
        End.
      End.

      Thus $\omega \subseteq \Phi$ (by \cref{SET_THEORY_03_344585425387520}).
      Consequently $n \in \Phi$.
      Therefore $\card{n} = n$.
    \end{proof}
  \end{forthel}

  \begin{forthel}
    \begin{corollary}\printlabel{SET_THEORY_07_7061392098066432}
      Every element of $\omega$ is a cardinal.
    \end{corollary}
  \end{forthel}

  \begin{forthel}
    \begin{proposition}\printlabel{SET_THEORY_07_4952029518626816}
      $\card{\omega} = \omega$.
    \end{proposition}
    \begin{proof}
      We have $\card{\omega} \leq \omega$.

      Let us show that $\card{\omega}$ is not less than $\omega$.
        Assume the contrary.
        Then $\card{\omega} \in \omega$.
        Take $n = \card{\omega}$ and a bijection $f$ between $n$ and $\omega$.

        Define \[ g(k) =
          \begin{cases}
            \succ{f(k)} & : k \less n
            \\
            0           & : k = n
          \end{cases} \]
        for $k \in \succ{n}$.
        Then $g$ is a map from $\succ{n}$ to $\omega$.

        $g$ is injective.
        Indeed we can show that for all $k, k' \in \succ{n}$ if $k \neq k'$
        then $g(k) \neq g(k')$. \\
        Proof.
          Let $k, k' \in \succ{n}$.
          Assume $k \neq k'$.

          Case $k, k' \less n$.
            Then $f(k) \neq f(k')$.
            Hence $\succ{f(k)} \neq \succ{f(k')}$.
            Thus $g(k) \neq g(k')$.
          End.

          Case $k \less n$ and $k' = n$.
            We have $\succ{f(k)} \neq 0$.
            Hence $g(k) \neq g(k')$.
          End.

          Case $k = n$ and $k' \less n$.
            We have $\succ{f(k')} \neq 0$.
            Hence $g(k) \neq g(k')$.
          End.
        Qed.

        $g$ is surjective onto $\omega$.
        Indeed we can show that for any $m \in \omega$ there exists a
        $k \in \succ{n}$ such that $m = g(k)$. \\
        Proof.
          Let $m \in \omega$.
          Then $\inverse{f}(m) \in n$.

          Case $m = 0$.
            Then $m = g(n)$.
          End.

          Case $m \neq 0$.
            Take $m' \in \omega$ such that $m = \succ{m'}$.
            Then $m
              = \succ{m'}
              = \succ{f(\inverse{f}(m'))}
              = g(\inverse{f}(m'))$.
            Indeed $\inverse{f}(m') \less n$.
          End.
        End.

        Hence $g$ is a bijection between $\succ{n}$ and $\omega$.
        Then we have $\card{n} = \card{\succ{n}}$.
        Thus $n = \succ{n}$.
      End.
    \end{proof}
  \end{forthel}

  \begin{forthel}
    \begin{corollary}\printlabel{SET_THEORY_07_2717623053713408}
      $\omega$ is a cardinal.
    \end{corollary}
  \end{forthel}

  \begin{forthel}
    \begin{definition}\printlabel{SET_THEORY_07_5346658235711488}
      Let $x$ be a set.
      $x$ is finite iff $\card{x} \less \omega$.
    \end{definition}
  \end{forthel}

  \begin{forthel}
    \begin{definition}\printlabel{SET_THEORY_07_8295412068777984}
      Let $x$ be a set.
      $x$ is infinite iff $x$ is not finite.
    \end{definition}
  \end{forthel}

  \begin{forthel}
    \begin{definition}\printlabel{SET_THEORY_07_8808604616359936}
      Let $x$ be a set.
      $x$ is countable iff $\card{x} \leq \omega$.
    \end{definition}
  \end{forthel}

  \begin{forthel}
    \begin{definition}\printlabel{SET_THEORY_07_2935263915409408}
      Let $x$ be a set.
      $x$ is uncountable iff $x$ is not countable.
    \end{definition}
  \end{forthel}

  \begin{forthel}
    \begin{definition}\printlabel{SET_THEORY_07_5679866426949632}
      Let $x$ be a set.
      $x$ is countably infinite iff $\card{x} = \omega$.
    \end{definition}
  \end{forthel}

  \begin{forthel}
    \begin{proposition}\printlabel{SET_THEORY_07_3806229474312192}
      Let $x$ be a set.
      Then $x$ is finite iff $\card{x} = n$ for some $n \in \omega$.
    \end{proposition}
  \end{forthel}

  \begin{forthel}
    \begin{proposition}\printlabel{SET_THEORY_07_3174577070931968}
      Let $x$ be a set.
      Then $x$ is infinite iff $\card{x} \geq \omega$.
    \end{proposition}
    \begin{proof}
      $\card{x} \geq \omega$ iff $\card{x} \nless \omega$.
    \end{proof}
  \end{forthel}

  \begin{forthel}
    \begin{proposition}\printlabel{SET_THEORY_07_4281623468048384}
      Let $x$ be a set.
      Then $x$ is uncountable iff $\card{x} \gtr \omega$.
    \end{proposition}
  \end{forthel}

  \begin{forthel}
    \begin{definition}\printlabel{SET_THEORY_07_4231078585827328}
      $\Card$ is the collection of all infinite cardinals.
    \end{definition}
  \end{forthel}

  \begin{forthel}
    \begin{proposition}\printlabel{SET_THEORY_07_4285360123150336}
      $\Card$ is a proper class.
    \end{proposition}
    \begin{proof}
      Suppose that $\Card$ is a set.
      Then $\bigcup \Card$ is a set.

      Let us show that $\bigcup \Card$ contains every ordinal.
        Let $\alpha$ be an ordinal.
        Choose an infinite ordinal $\beta$ such that $\beta \geq \alpha$.
        Choose a cardinal $\kappa$ greater than $\beta$.
        Then $\alpha \in \kappa \in \Card$.
        Hence $\alpha \in \bigcup \Card$.
      End.

      Therefore $\Ord \subseteq \bigcup \Card$.
      Thus $\Ord$ is a set.
      Contradiction.
    \end{proof}
  \end{forthel}

  \begin{forthel}
    \begin{proposition}\printlabel{SET_THEORY_07_8189062544359424}
      Let $\alpha$ be an infinite ordinal.
      Then $\card{\succ{\alpha}} = \card{\alpha}$.
    \end{proposition}
    \begin{proof}
      For any $\beta \in \succ{\alpha}$ we have
      $\beta \less \omega$ or $\omega \leq \beta \less \alpha$ or $\beta = \alpha$.
      Define \[ f(\beta) =
        \begin{cases}
          \succ{\beta}  & : \beta \less \omega
          \\
          \beta         & : \omega \leq \beta \less \alpha
          \\
          0             & : \beta = \alpha
        \end{cases} \]
      for $\beta \in \succ{\alpha}$.

      Then $f$ is a map from $\succ{\alpha}$ to $\alpha$.
      Indeed we can show that $f(\beta) \in \alpha$ for all
      $\beta \in \succ{\alpha}$. \\
      Proof.
        Let $\beta \in \succ{\alpha}$.

        Case $\beta \less \omega$.
          Then $f(\beta)
            = \succ{\beta}
            \less \omega
            \leq \alpha$.
        End.

        Case $\omega \leq \beta \less \alpha$.
          Then $f(\beta)
            = \beta
            \less \alpha$.
        End.

        Case $\beta = \alpha$.
          Then $f(\beta)
            = 0
            \less \alpha$.
        End.
      Qed.

      $f$ is surjective onto $\alpha$.
      Indeed we can show that for any $\beta \in \alpha$ there exists a
      $\gamma \in \succ{\alpha}$ such that $\beta = f(\gamma)$. \\
      Proof.
        Let $\beta \in \alpha$.
        Then $\beta = 0$ or $0 \less \beta \less \omega$ or $\beta \geq \omega$.

        Case $\beta = 0$.
          Then $\beta = f(\alpha)$.
        End.

        Case $0 \less \beta \less \omega$.
          Take an ordinal $\beta'$ such that $\beta = \succ{\beta'}$.
          Then $\beta' \less \omega$.
          Hence $\beta = f(\beta')$.
        End.

        Case $\beta \geq \omega$.
          Then $\beta = f(\beta)$.
        End.
      Qed.

      $f$ is injective.
      Indeed we can show that for all $\beta, \gamma \in \succ{\alpha}$ if
      $\beta \neq \gamma$ then $f(\beta) \neq f(\gamma)$. \\
      Proof.
        Let $\beta, \gamma \in \succ{\alpha}$.
        Assume $\beta \neq \gamma$.

        Case $\beta \less \omega$.
          If $\gamma = \alpha$ then
          $f(\beta)
            = \succ{\beta}
            \neq 0
            = f(\gamma)$.
          If $\omega \leq \gamma \less \alpha$ then
          $f(\beta)
            = \succ{\beta}
            \less \omega
            \leq \gamma
            = f(\gamma)$.
        End.

        Case $\omega \leq \beta \less \alpha$.
          If $\gamma = \alpha$ then
          $f(\beta)
            = \beta
            \geq \omega
            \gtr 0
            = f(\gamma)$.
          If $\gamma \less \omega$ then
          $f(\beta)
            = \beta
            \geq \omega
            \gtr \succ{\gamma}
            = f(\gamma)$.
        End.

        Case $\beta = \alpha$.
          If $\gamma \less \omega$ then
          $f(\beta)
            = 0
            \neq \succ{\gamma}
            = f(\gamma)$.
          If $\omega \leq \gamma \less \alpha$ then
          $f(\beta)
            = 0
            \less \omega
            \leq \gamma
            = f(\gamma)$.
        End.
      Qed.

      Hence $f$ is a bijection between $\succ{\alpha}$ and $\alpha$.
      Therefore $\succ{\alpha}$ and $\alpha$ are equinumerous.
      Consequently $\card{\succ{\alpha}} = \card{\alpha}$.
    \end{proof}
  \end{forthel}

  \begin{forthel}
    \begin{proposition}\printlabel{SET_THEORY_07_8700732632989696}
      Every infinite cardinal is a limit ordinal.
    \end{proposition}
    \begin{proof}
      Let $\kappa$ be an infinite cardinal.
      Suppose that $\kappa$ is not a limit ordinal.
      $\kappa \neq 0$.
      Hence $\kappa$ is a successor ordinal.
      Thus we can take an ordinal $\alpha$ such that $\kappa = \succ{\alpha}$.
      We have $\alpha \gtr \kappa \geq \omega$.
      Hence $\card{\succ{\alpha}} = \card{\alpha}$.
      Thus $\alpha \less \card{\kappa}$ and $\kappa$ is equinumerous to $\kappa$.
      Contradiction.
    \end{proof}
  \end{forthel}
\end{document}
