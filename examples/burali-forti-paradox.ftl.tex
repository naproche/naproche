\documentclass{article}
\usepackage{naproche}
\libinput[meta-inf]{example-preamble}
\libinput[libraries]{foundations-preamble}

\title{Burali-Forti's Paradox}
\author{\Naproche formalization:\\[0.5em]Marcel Schütz}
\date{2024}

\begin{document}
  \pagenumbering{gobble}
  \maketitle

  \usemodule[libraries]{set-theory?ordinals.ftl}
  \usemodule[libraries]{foundations?proper-classes.ftl}
  
  \noindent \emph{Burali-Forti's Paradox}, named after Cesare Burali-Forti, 
  demonstates that the collection $\Ord$ of all ordinal numbers is a
  proper class \cite{BuraliForti1897}.

  \begin{forthel}
    \begin{theorem*}[title=Burali-Forti's Paradox,id=burali_forti_paradox]
      $\Ord$ is a proper class.
    \end{theorem*}
    \begin{proof}
      Assume the contrary.
      $\Ord$ is transitive and every element of $\Ord$ is transitive.
      Hence $\Ord$ is an ordinal.
      Thus $\Ord \in \Ord$.
      Contradiction.
    \end{proof}
  \end{forthel}

  \printbibliography

  \section*{Copyright}
  \doclicenseThis
\end{document}
