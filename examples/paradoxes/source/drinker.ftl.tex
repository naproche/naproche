\documentclass{stex}
\libinput{preamble}

\title{The Drinker Paradox in \Naproche}
\author{Marcel Schütz}
\date{2023}

\begin{document}
\maketitle

\usemodule[naproche/examples/preliminaries]{everyday-ontology.ftl}

\noindent The \emph{Drinker Paradox} is a principle of classical predicate logic popularised by the logician Raymond Smullyan in his 1978 book \textit{What Is the Name of this Book?} \cite{Smullyan1978} which can be stated as:

\begin{quotation}
  \noindent There is someone in the pub such that, if he is drinking then everyone in the pub is drinking.
\end{quotation}

\begin{fsignature*}
  The pub is an object.
\end{fsignature*}

\begin{ftheorem*}[label=drinker_paradox,title=Drinker Paradox]
  Assume that there is a person inside the pub.
  Then there is a person $P$ inside the pub such that if $P$ is drinking then every person inside the pub is drinking.
\end{ftheorem*}
\begin{fproof}[method=case analysis]
  Case every person inside the pub is drinking.
    Choose a person $P$ inside the pub.
    Then $P$ is drinking and every person inside the pub is drinking.
    Hence if $P$ is drinking then every person inside the pub is drinking.
  End.

  Case there is a person inside the pub that is not drinking.
    Consider a person $P$ inside the pub that is not drinking.
    Then if $P$ is drinking then every person inside the pub is drinking.
  End.
\end{fproof}

\printbibliography
\end{document}
