\documentclass{stex}
\libinput{preamble}
\begin{document}
\title{The Barber Paradox in \Naproche}
\author{Marcel Schütz}
\date{2023}
\maketitle
\begin{smodule}{barber.ftl}
  \importmodule[naproche/examples/preliminaries]{everyday-ontology.ftl}

  \noindent The \emph{Barber Paradox} is a puzzle illustrating Russell's Paradox, formulated by Bertrand Russell in 1919 \cite[p. 355]{Russell1919}:

  \begin{quotation}
    \noindent You can define the barber as “one who shaves all those, and those only, who do not shave themselves”.
    The question is, does the barber shave himself?
  \end{quotation}

  \noindent In \Naproche we can formalize this paradox as follows.

  \begin{forthel}
    \begin{signature*}
      The barber is a man such that for any man $M$ the barber shaves $M$ iff $M$ does not shave himself.
    \end{signature*}

    \begin{theorem*}[Barber Paradox]\label{barber_paradox}
      The barber shaves himself iff the barber does not shave himself.
    \end{theorem*}
  \end{forthel}
\end{smodule}
\printbibliography
\end{document}
