\documentclass{stex}
\libinput{preamble}

\newcommand{\var}[1]{\mathrm{v}_{#1}}
\newcommand{\abs}[2]{\lambda\var{#1}.\ #2}
\newcommand{\app}[2]{(#1)(#2)}
\newcommand{\fix}{\mathtt{fix}}

\title{Curry's Paradox in \Naproche}
\author{Marcel Schütz}
\date{2023}

\begin{document}
\maketitle

\noindent \emph{Curry's Paradox} is a paradox described by Haskell Curry in 1942 \cite{Curry1942}.
It allows the derivation of an arbitrary statement from a self-referential expression that presupposes its own validity.

\section*{Untyped $\lambda$-calculus}

\begin{forthel}
  %[prove off][check off]
  [readtex \path{arithmetic/sections/01_natural-numbers.ftl.tex}]
  %[prove on][check on]
\end{forthel}

\noindent Our formalization of Curry's Paradox in \Naproche is based on an untyped $\lambda$-calculus:

\begin{fsignature*}
  A expression is a notion.
\end{fsignature*}

Let $A, B$ denote expressions.

\begin{fsignature*}
  A variable is an expression.
\end{fsignature*}

Let $n$ denote a natural number.

\begin{fsignature*}
  $\var{n}$ is a variable.
\end{fsignature*}

\begin{fsignature*}[label=abstraction,title=Abstraction]
  $\abs{n}{A}$ is an expression.
\end{fsignature*}

\begin{fsignature*}[label=application,title=Application]
  $\app{A}{B}$ is an expression.
\end{fsignature*}

\begin{fsignature*}[label=fixed_point_combinator,title=Fixed-point combinator]
  $\fix$ is an expression such that
  \[\app{\fix}{A} = \app{A}{\app{\fix}{A}}\]
  for every expression $A$.
\end{fsignature*}


\section*{Propositional logic}

\noindent We extend this $\lambda$-calculus by a fragment of propositional logic:

\begin{fsignature*}[label=implication,title=Implication]
  $A \rightarrow B$ is an expression.
\end{fsignature*}

\begin{fsignature*}[label=truth]
  $A$ is true is a relation.
\end{fsignature*}

\begin{faxiom}[label=beta_reduction]
  $\app{\abs{n}{\var{n} \rightarrow B}}{A} = A \rightarrow B$.
\end{faxiom}

\begin{faxiom}[label=reflexivity]
  $A \rightarrow A$ is true.
\end{faxiom}

\begin{faxiom}[label=modus_ponens]
  If $A$ is true and $A \rightarrow B$ is true then $B$ is true.
\end{faxiom}

\begin{faxiom}[label=strengthening]
  If $A \rightarrow (A \rightarrow B)$ is true then $A \rightarrow B$ is true.
\end{faxiom}


\section*{Curry's paradox}

\noindent Using the fixpoint combinator from above we can formulate a self-referential expression $X$ of the form “If $X$ is true then $A$ is true” for any arbitrary expression $A$ by defining $X = \app{\fix}{\abs{0}{\var{0} \rightarrow A}}$.
From the existence of such an expression $X$ together with the above axioms we can then derive that any expression $A$ is true.

\begin{ftheorem*}[label=curry_paradox,title=Curry's paradox]
  Every expression is true.
\end{ftheorem*}
\begin{fproof}
  Let $A$ be an expression.
  Take $N = \abs{0}{\var{0} \rightarrow A}$ and $X = \app{\fix}{N}.$

  (1) Then $X = \app{N}{X} = X \rightarrow A$ (by \printref{beta_reduction}).

  Hence $X \rightarrow (X \rightarrow A)$ is true (by 1, \printref{reflexivity}).

  (2) Thus $X \rightarrow A$ is true (by \printref{strengthening}).

  (3) Therefore $X$ is true (by 1, 2).

  Consequently $A$ is true (by \printref{modus_ponens}, 2, 3).
\end{fproof}

\printbibliography
\end{document}
