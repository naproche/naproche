\documentclass{stex}
\libinput{preamble}
\libinput[naproche/examples/foundations]{preamble}
\libinput[naproche/examples/set-theory]{preamble}

\usepackage{xr}
\externaldocument{../../set-theory/set-theory}

\title{Cantor's Second Paradox in \Naproche}
\author{Marcel Schütz}
\date{2023}

\begin{document}
\maketitle

\noindent \emph{Cantor's Second Paradox} denotes the observation that the collection of all set cannot be a set itself.
It was shown by Georg Cantor in 1899 via his famous theorem stating that the cardinality of any set is strictly smaller than the cardinality of its powerset \cite[chapter 163]{Cantor1991}.

\inputref[naproche/examples/set-theory]{thm/cantor-paradox-2.ftl.tex}

\printbibliography
\end{document}
