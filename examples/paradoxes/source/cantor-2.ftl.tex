\documentclass{stex}
\libinput{preamble}
\usepackage{xr}
\externaldocument{../../set-theory/set-theory}
\begin{document}
\title{Cantor's Second Paradox in \Naproche}
\author{Marcel Schütz}
\date{2023}
\maketitle
\begin{smodule}{cantor-2}
\usemodule{mod?georg-cantor}
\usemodule{mod?cantor-paradox-2}
\begin{sparagraph}
  \emph{\sn{Cantor's Second Paradox}} denotes the observation that the collection of all set cannot be a set itself.
  It was shown by \sr{Cantor}{Georg Cantor} in 1899 via his famous theorem stating that the cardinality of any set is strictly smaller than the cardinality of its powerset \cite[chapter 163]{Cantor1991}.
\end{sparagraph}

\begin{forthel}
  %[prove off][check off]
  [readtex \path{set-theory/sections/06_cardinals.ftl.tex}]
  %[prove on][check on]

  \begin{theorem*}[Cantor's Second Paradox]\label{cantor_paradox_2}
    There exists no set that contains all sets.
  \end{theorem*}
  \begin{proof}
    Assume the contrary.
    Consider a set $V$ that contains all sets.
    Then $\pow(V)$ is a set.
    Hence $\pow(V)$ is a subset of $V$.
    Thus $|\pow(V)| \leq |V|$.
    Contradiction.
    Indeed $|x| < |\pow(x)|$ for any set $x$ (by \nameref{SET_THEORY_06_914271456198656}).
  \end{proof}
\end{forthel}
\end{smodule}
\printbibliography
\end{document}
