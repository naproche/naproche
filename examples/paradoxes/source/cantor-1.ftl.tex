\documentclass{stex}
\libinput{preamble}
\usepackage{xr}
\externaldocument{burali-forti.ftl}
\begin{document}
\title{Cantor's First Paradox in \Naproche}
\author{Marcel Schütz}
\date{2023}
\maketitle
\begin{smodule}{cantor-1.ftl}
  \importmodule[naproche/examples/paradoxes]{burali-forti.ftl}

  \noindent \emph{Cantor's First Paradox}, discovered by Georg Cantor in 1897, states that there cannot exists a set that contains all cardinal numbers \cite[chapter 156]{Cantor1991}.

  \begin{forthel}
    %[prove off][check off]
    [readtex \path{set-theory/sections/06_cardinals.ftl.tex}]
    %[prove on][check on]
  \end{forthel}

  \begin{forthel}
    \begin{theorem*}[Cantor's First Paradox]\label{cantor_paradox_1}
      There exists no set that contains all cardinals.
    \end{theorem*}
    \begin{proof}
      Assume the contrary.
      Consider a set $X$ that contains all cardinals.
      Define $C = \class{\kappa \in X}{\text{$\kappa$ is a cardinal}}$.
      Define $O = \class{\alpha}{\text{$\alpha$ is an ordinal}}$.

      Let us show that $O \subseteq \bigcup C$.
        Let $\alpha \in O$.
        Choose a cardinal $\kappa$ greater than $\alpha$.
        Then $\alpha \in \kappa \in C$.
        Thus $\alpha \in \bigcup C$.
      End.

      $\bigcup C$ is a set.
      Hence $O$ is a set.
      Contradiction (by \nameref{burali_forti_paradox}).
      Indeed $O$ contains every ordinal.
    \end{proof}
  \end{forthel}
\end{smodule}
\printbibliography
\end{document}
