\documentclass{stex}
\libinput{preamble}
\begin{document}
\begin{smodule}{hilbert-no-universal-set.ftl}
  \importmodule[naproche/examples/paradoxes]{mod?hilbert-paradox.ftl}
  \importmodule[naproche/examples/foundations]{def/classes?universal-class.ftl}

  \begin{fproposition*}[label=hilbert_paradox_corollary]
    There exists no universal set.
  \end{fproposition*}
  \begin{fproof}[method=contradiction]
    Assume the contrary.
    Consider a universal set $X$.
    Define $V = \fclass{x \in X}{\text{$x$ is a set}}$.
    Then $V$ is closed under powersets and closed under unions.
    Hence $V$ is not a set (by \printref{hilbert_paradox}).
  \end{fproof}
\end{smodule}
\end{document}
