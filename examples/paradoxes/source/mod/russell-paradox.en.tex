\documentclass{stex}
\libinput{preamble}
\begin{document}
\begin{smodule}{russell-paradox}
\usemodule{mod?bertrand-russell}
\symdecl*{Russell's Paradox}
\begin{sparagraph}[style=symdoc]
  \vardef{phivar}[args=1]{\maincomp{\varphi}\dobrackets{#1}}
  \vardef{xvar}{x}
  \definame{Russell's Paradox} is a set-theoretical paradox discovered by \sr{Russell}{Bertrand Russell} around 1902 which shows that there exist statements $\phivar!$ whose extension $\{\xvar\mid\phivar{\xvar}\}$ cannot constitute a set.
\end{sparagraph}
\end{smodule}
\end{document}