\documentclass{stex}
\libinput{preamble}
\begin{document}
\title{The Russell-Myhill Paradox in \Naproche}
\author{Marcel Schütz}
\date{2023}
\maketitle
\begin{smodule}{russell-myhill.ftl}
  \importmodule[naproche/examples/preliminaries]{everyday-ontology.ftl}

  \noindent The \emph{Russell-Myhill Paradox} is a paradox discovered by Bertrand Russell which states that two systems of propositions with the same so-called \emph{logical product} are not necessarily equal.
  It was discussed in Russell's 1903 \emph{Principles of Mathematics} \cite[Appendix B]{Russell1903} and rediscovered in 1958 by John Myhill \cite{Myhill1958}.

\begin{forthel}
    \begin{signature*}
      Let $P$ be a system of propositions.
      The logical product of $P$ is a proposition $p$ such that $p$ is true iff every element of $P$ is true.
    \end{signature*}

    Let $\sqcap P$ denote the logical product of $P$.

    \begin{theorem*}[Russell-Myhill Paradox]\label{russell_myhill_paradox}
      It is wrong that
      \[ \sqcap P = \sqcap Q \implies P = Q \]
      for all systems of propositions $P, Q$.
    \end{theorem*}
    \begin{proof}
      Assume the contrary.
      Define \[Q = \class{q}{\classtext{there exists a system of propositions $P$ such that $q = \sqcap P$ and $q \notin P$}}.\]
      Consider $q = \sqcap Q$.
      Then for any system of propositions $P$ such that $\sqcap q = P$ we have $P = Q$.
      Hence $q \in Q$ iff $q \notin Q$.
      Contradiction.
    \end{proof}
  \end{forthel}
\end{smodule}
\printbibliography
\end{document}
