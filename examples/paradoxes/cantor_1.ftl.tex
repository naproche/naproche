\documentclass[english]{article}
\usepackage[type={CC},modifier={zero},version={1.0},imagemodifier=-80x15]{doclicense}
\usepackage{biblatex}
\usepackage{xr}
\usepackage[settheory]{../../lib/tex/naproche}
\addbibresource{REFERENCES.bib}
\externaldocument{burali-forti.ftl}
\begin{document}
\title{Cantor's First Paradox in \Naproche}
\author{Marcel Schütz}
\date{2023}
\pagenumbering{gobble}
\maketitle

\noindent \href{https://en.wikipedia.org/wiki/Cantor%27s_paradox}{\emph{Cantor's First Paradox}}, discovered by \href{https://en.wikipedia.org/wiki/Georg_Cantor}{Georg Cantor} in 1897, states that there cannot exists a \href{https://en.wikipedia.org/wiki/Set_(mathematics)}{set} that contains all \href{https://en.wikipedia.org/wiki/Cardinal_number}{cardinal numbers} \cite[chapter 156]{Cantor1991}.

\begin{forthel}
  %[prove off][check off]
  [readtex \path{set-theory/sections/06_cardinals.ftl.tex}]

  [readtex \path{paradoxes/burali-forti.ftl.tex}]
  %[prove on][check on]

  \begin{theorem*}[Cantor's First Paradox]\label{cantor_paradox_1}
    There exists no set that contains all cardinals.
  \end{theorem*}
  \begin{proof}
    Assume the contrary.
    Consider a set $X$ that contains all cardinals.
    Define $C = \{ \kappa \in X \mid \kappa$ is a cardinal$\}$.
    Define $O = \{ \alpha \mid \alpha$ is an ordinal$\}$.

    Let us show that $O \subseteq \bigcup C$.
      Let $\alpha \in O$.
      Choose a cardinal $\kappa$ greater than $\alpha$.
      Then $\alpha \in \kappa \in C$.
      Thus $\alpha \in \bigcup C$.
    End.

    $\bigcup C$ is a set.
    Hence $O$ is a set.
    Contradiction (by \nameref{burali_forti_paradox}).
    Indeed $O$ contains every ordinal.
  \end{proof}
\end{forthel}

\printbibliography
\vfill
\doclicenseThis
\end{document}
