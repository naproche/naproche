\documentclass[english]{article}
\usepackage[type={CC},modifier={zero},version={1.0},imagemodifier=-80x15]{doclicense}
\usepackage{biblatex}
\usepackage{../../lib/tex/naproche}
\addbibresource{REFERENCES.bib}
\begin{document}
\title{The Barber Paradox in \Naproche}
\author{Marcel Schütz}
\date{2023}
\pagenumbering{gobble}
\maketitle

\noindent The \href{https://en.wikipedia.org/wiki/Barber_paradox}{\emph{Barber Paradox}} is a puzzle illustrating \href{https://en.wikipedia.org/wiki/Russell%27s_paradox}{Russell's Paradox}, formulated by \href{https://en.wikipedia.org/wiki/Bertrand_Russell}{Bertrand Russell} in 1919 \cite[p. 355]{Russell1919}:

\begin{quotation}
  \noindent You can define the barber as “one who shaves all those, and those only, who do not shave themselves”.
  The question is, does the barber shave himself?
\end{quotation}

\noindent In \Naproche we can formalize this paradox as follows.

\begin{forthel}
  [readtex \path{paradoxes/ONTOLOGY.ftl.tex}]

  \begin{signature*}
    The barber is a man such that for any man $M$ the barber shaves $M$ iff $M$ does not shave himself.
  \end{signature*}

  \begin{theorem*}[Barber Paradox]\label{barber_paradox}
    The barber shaves himself iff the barber does not shave himself.
  \end{theorem*}
\end{forthel}

\printbibliography
\vfill
\doclicenseThis
\end{document}
