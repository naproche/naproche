\documentclass[english]{article}
\usepackage[type={CC},modifier={zero},version={1.0},imagemodifier=-80x15]{doclicense}
\usepackage{biblatex}
\usepackage{../../lib/tex/naproche}
\addbibresource{REFERENCES.bib}
\begin{document}
\title{The Russell-Myhill Paradox in \Naproche}
\author{Marcel Schütz}
\date{2023}
\pagenumbering{gobble}
\maketitle

\noindent The \href{https://en.wikipedia.org/wiki/John_Myhill#Contributions}{\emph{Russell-Myhill Paradox}} is a paradox discovered by \href{https://en.wikipedia.org/wiki/Bertrand_Russell}{Bertrand Russell} which states that two systems of \href{https://en.wikipedia.org/wiki/Proposition}{propositions} with the same so-called \emph{logical product} are not necessarily equal.
It was discussed in \href{https://en.wikipedia.org/wiki/Bertrand_Russell}{Russell's} 1903 \href{https://en.wikipedia.org/wiki/The_Principles_of_Mathematics}{\emph{Principles of Mathematics}} \cite[Appendix B]{Russell1903} and rediscovered in 1958 by \href{https://en.wikipedia.org/wiki/John_Myhill}{John Myhill} \cite{Myhill1958}.

\begin{forthel}
  [readtex \path{paradoxes/ONTOLOGY.ftl.tex}]

  \begin{signature*}
    Let $P$ be a system of propositions.
    The logical product of $P$ is a proposition $p$ such that $p$ is true iff every element of $P$ is true.
  \end{signature*}

  Let $\sqcap P$ denote the logical product of $P$.

  \begin{theorem*}[Russell-Myhill Paradox]\label{russell_myhill_paradox}
    It is wrong that
    \[ \sqcap P = \sqcap Q \implies P = Q \]
    for all systems of propositions $P, Q$.
  \end{theorem*}
  \begin{proof}
    Assume the contrary.
    Define $Q = \{ q \mid$ there exists a system of propositions $P$ such that $q = \sqcap P$ and $q \notin P \}$.
    Consider $q = \sqcap Q$.
    Then for any system of propositions $P$ such that $\sqcap q = P$ we have $P = Q$.
    Hence $q \in Q$ iff $q \notin Q$.
    Contradiction.
  \end{proof}
\end{forthel}

\printbibliography
\vfill
\doclicenseThis
\end{document}
