\documentclass[english]{article}
\usepackage[type={CC},modifier={zero},version={1.0},imagemodifier=-80x15]{doclicense}
\usepackage{biblatex}
\usepackage{xr}
\usepackage[foundations]{../../lib/tex/naproche}
\addbibresource{REFERENCES.bib}
\externaldocument{../set-theory/set-theory}
\begin{document}
\title{Cantor's Second Paradox in \Naproche}
\author{Marcel Schütz}
\date{2023}
\pagenumbering{gobble}
\maketitle

\noindent \emph{Cantor's Second Paradox} denotes the observation that the collection of all \href{https://en.wikipedia.org/wiki/Set_(mathematics)}{sets} cannot be a \href{https://en.wikipedia.org/wiki/Set_(mathematics)}{set} itself.
It was shown by \href{https://en.wikipedia.org/wiki/Georg_Cantor}{Georg Cantor} in 1899 via his famous theorem stating that the \href{https://en.wikipedia.org/wiki/Cardinality}{cardinality} of any \href{https://en.wikipedia.org/wiki/Set_(mathematics)}{set} is strictly smaller than the \href{https://en.wikipedia.org/wiki/Cardinality}{cardinality} of its \href{https://en.wikipedia.org/wiki/Power_set}{powerset} \cite[chapter 163]{Cantor1991}.

\begin{forthel}
  %[prove off][check off]
  [readtex \path{set-theory/sections/06_cardinals.ftl.tex}]
  %[prove on][check on]

  \begin{theorem*}[Cantor's Second Paradox]\label{cantor_paradox_2}
    There exists no set that contains all sets.
  \end{theorem*}
  \begin{proof}
    Assume the contrary.
    Consider a set $V$ that contains all sets.
    Then $\pow(V)$ is a set.
    Hence $\pow(V)$ is a subset of $V$.
    Thus $|\pow(V)| \leq |V|$.
    Contradiction.
    Indeed $|x| < |\pow(x)|$ for any set $x$ (by \nameref{SET_THEORY_06_914271456198656}).
  \end{proof}
\end{forthel}

\printbibliography
\vfill
\doclicenseThis
\end{document}
