\documentclass[english]{article}
\usepackage[type={CC},modifier={zero},version={1.0},imagemodifier=-80x15]{doclicense}
\usepackage{biblatex}
\usepackage[settheory]{../../lib/tex/naproche}
\addbibresource{REFERENCES.bib}
\begin{document}
\title{Burali-Forti's Paradox in \Naproche}
\author{Marcel Schütz}
\date{2023}
\pagenumbering{gobble}
\maketitle

\noindent \href{https://en.wikipedia.org/wiki/Burali-Forti_paradox}{\emph{Burali-Forti's paradox}}, named after \href{https://en.wikipedia.org/wiki/Cesare_Burali-Forti}{Cesare Burali-Forti}, demonstates that there cannot exists a \href{https://en.wikipedia.org/wiki/Set_(mathematics)}{set} that contains all \href{https://en.wikipedia.org/wiki/Ordinal_number}{ordinal numbers} \cite{BuraliForti1897}.

\begin{forthel}
  %[prove off][check off]
  [readtex \path{set-theory/sections/02_ordinals.ftl.tex}]
  %[prove on][check on]

  \begin{theorem*}[Burali-Forti's Paradox]\label{burali_forti_paradox}
    There exists no set that contains all ordinals.
  \end{theorem*}
  \begin{proof}[by contradiction]
    Assume the contrary.
    Consider a set $O$ that contains all ordinals.
    Then $O$ is transitive and every element of $O$ is transitive.
    Hence $O$ is an ordinal.
    Thus $O \in O$.
    Contradiction.
  \end{proof}
\end{forthel}

\printbibliography
\vfill
\doclicenseThis
\end{document}
