\documentclass[english]{article}
\usepackage[type={CC},modifier={zero},version={1.0},imagemodifier=-80x15]{doclicense}
\usepackage{biblatex}
\usepackage{../../lib/tex/naproche}
\addbibresource{REFERENCES.bib}
\begin{document}
\title{Russell's Paradox in \Naproche}
\author{Marcel Schütz}
\date{2023}
\pagenumbering{gobble}
\maketitle

\noindent \href{https://en.wikipedia.org/wiki/Russell%27s_paradox}{\emph{Russell's Paradox}} is a set-theoretical paradox discovered by \href{https://en.wikipedia.org/wiki/Bertrand_Russell}{Bertrand Russell} around 1902 \cite[chapter XV]{Frege1980} which shows that there exist statements $\varphi$ whose \href{https://en.wikipedia.org/wiki/Extension_(semantics)}{extension} $\{ x \mid \varphi(x) \}$ does not constitute a \href{https://en.wikipedia.org/wiki/Set_(mathematics)}{set} – or in other words: Not every \href{https://en.wikipedia.org/wiki/Class_(set_theory)}{class} is a \href{https://en.wikipedia.org/wiki/Set_(mathematics)}{set}.

\begin{forthel}
  [readtex \path{paradoxes/ONTOLOGY.ftl.tex}]

  \begin{theorem*}[Russell's Paradox]\label{russell_paradox}
    It is wrong that every class is a set.
  \end{theorem*}
  \begin{proof}
    Assume the contrary.
    Define $R = \{ x \mid x$ is a set such that $x \notin x \}$.
    Then $R$ is a set.
    Hence $R \in R$ iff $R \notin R$.
    Contradiction.
  \end{proof}
\end{forthel}

\printbibliography
\vfill
\doclicenseThis
\end{document}
