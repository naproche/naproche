\documentclass{naproche-library}
\libinput{preamble}
\begin{document}
\begin{smodule}{natural-numbers.ftl}
  \importmodule[meta-inf]{macros.ftl}
  \importmodule[meta-inf]{vocabulary.ftl}

  \symdef{Nat}{\mathbb N}
  \symdef{plus}{\,+\,}

  \section{Natural Numbers}

  \subsection{The Language of Natural Number Arithmetic}

  \begin{signature}[forthel,id=NatNumberSig]
    A \emph{natural number} is an object.

    Let $n,m,k$ denote natural numbers.
  \end{signature}

  \begin{definition}[forthel,id=NatDef]
    $\Nat$ is the class of natural numbers.
  \end{definition}

  \begin{signature}[forthel,id=PlusSig]
    $n\plus m$ is a natural number.

    Let the \emph{sum of $n$ and $m$} stand for $n\plus m$.
  \end{signature}

  \STEXinvisible{
    \begin{forthel}
      Let $n+m$ stand for $n\plus m$.
    \end{forthel}
  }

  \begin{signature}[forthel,id=ZeroSig]
    $0$ is a natural number.

    Let \emph{zero} stand for $0$.
    Let $n$ is \emph{nonzero} stand for $n\neq0$.
  \end{signature}

  \begin{signature}[forthel,id=OneSig]
    $1$ is a natural number.

    Let \emph{one} stand for $1$.
    Let the \emph{direct successor of $n$} stand for $n\plus1$.
  \end{signature}

  \begin{definition}[forthel,id=TwoSig]
    $2=1\plus1$.

    Let \emph{two} stand for $2$.
  \end{definition}

  \begin{definition}[forthel,id=ThreeSig]
    $3=2\plus1$.

    Let \emph{three} stand for $3$.
  \end{definition}

  \begin{definition}[forthel,id=FourSig]
    $4=3\plus1$.

    Let \emph{four} stand for $4$.
  \end{definition}

  \begin{definition}[forthel,id=FiveSig]
    $5=4\plus1$.

    Let \emph{five} stand for $5$.
  \end{definition}

  \begin{definition}[forthel,id=SixSig]
    $6=5\plus1$.

    Let \emph{six} stand for $6$.
  \end{definition}

  \begin{definition}[forthel,id=SevenSig]
    $7=6\plus1$.

    Let \emph{seven} stand for $7$.
  \end{definition}

  \begin{definition}[forthel,id=EightSig]
    $8=7\plus1$.

    Let \emph{eight} stand for $8$.
  \end{definition}

  \begin{definition}[forthel,id=NineSig]
    $9=8\plus1$.

    Let \emph{nine} stand for $9$.
  \end{definition}


  \subsection{The Axioms of Natural Number Arithmetic}

  \begin{axiom}[forthel,title=Infinity Axiom,id=NatIsSetAx]
    $\Nat$ is a set.
  \end{axiom}

  \begin{axiom}[forthel,title=Peano Axiom I,id=PlusIsInjectiveAx]
    If $n\plus1\eq m\plus1$ then $n\eq m$.
  \end{axiom}

  \begin{axiom}[forthel,title=Peano Axiom II,id=ZeroIsNoSuccessorAx]
    There exists no natural number $n$ such that $n\plus1\eq0$.
  \end{axiom}

  \begin{axiom}[forthel,title=Peano Axiom III,id=NatNumberIsZeroOrSuccessorAx]
    $n\eq0$ or $n\eq m\plus1$ for some natural number $m$.
  \end{axiom}

  \begin{axiom}[forthel,title=Induction Axiom,id=InductionAx]
    $n\prec n\plus1$.
  \end{axiom}

  \begin{axiom}[forthel,title=Addition Axiom I,id=AdditionWithZeroAx]
    $n\plus0\eq n$.
  \end{axiom}

  \begin{axiom}[forthel,title=Addition Axiom II,id=AdditionWithSuccessorAx]
    $n\plus(m\plus1)\eq(n\plus m)\plus1$.
  \end{axiom}


  \subsection{Immediate Consequences of the Axioms}

  \begin{proposition}[forthel,id=NatNumberIsNotItsOwnSuccessorProp]
    For all $n\in\Nat$ we have $n\neq n\plus1$.
  \end{proposition}
  \begin{proof}[forthel,method=induction on $n$]\end{proof}

  \begin{proposition}[forthel,id=ZeroIsLeftIdentityOfPlusProp]
    For all $n\in\Nat$ we have $0\plus n\eq n$.
  \end{proposition}
  \begin{proof}[forthel,method=induction on $n$]\end{proof}


  \subsection{Computation Laws for Addition}

  \begin{proposition}[forthel,title=Associativity,id=PlusIsAssociativeProp]
    For all $k\in\Nat$ we have $n\plus(m\plus k)\eq(n\plus m)\plus k$.
  \end{proposition}
  \begin{proof}[forthel,method=induction on $k$]
    Let $k\in\Nat$.

    Case $k\eq0$. Obvious.

    Case $k\neq0$.
      Take $k'\in\Nat$ such that $k\eq k'\plus1$.
      Then $k'\prec k$.
      Hence $n\plus(m\plus k')\eq(n\plus m)\plus k'$.
      Thus
      \begin{align*}
        n\plus(m\plus k)
          &\eq n\plus(m\plus(k'\plus1))\\
          &\eq n\plus((m\plus k')\plus1)\\
          &\eq (n\plus(m\plus k'))\plus1\\
          &\eq ((n\plus m)\plus k')\plus1\\
          &\eq (n\plus m)\plus(k'\plus1)
      \end{align*}.
    End.
  \end{proof}

  \begin{proposition}[forthel,title=Commutativity,id=PlusIsCommutativeProp]
    For all $n,m\in\Nat$ we have $n\plus m\eq m\plus n$.
  \end{proposition}
  \begin{proof}[forthel,method=induction on $m$]
    We can show by induction on $n$ that for all $n\in\Nat$ we have $n\plus0\eq0\plus n$.
    Obvious.

    We can show by induction on $n$ that for all $n\in\Nat$ we have $n\plus1\eq1\plus n$.
    Obvious.

    Let $m\in\Nat$.

    Case $m\eq0$. Trivial.

    Case $m\neq0$.
      Let $n\in\Nat$.
      Take $m'\in\Nat$ such that $m\eq m'\plus1$.
      Then $n\plus m'\eq m'\plus n$.
      Hence
      \begin{align*}
        n\plus m
          &\eq n\plus(m'\plus1)\\
          &\eq (n\plus m')\plus1\\
          &\eq (m'\plus n)\plus1\\
          &\eq m'\plus(n\plus1)\\
          &\eq m'\plus(1\plus n)\\
          &\eq (m'\plus1)\plus n
      \end{align*}.
    End.
  \end{proof}

  \begin{proposition}[forthel,title=Right cancellation,id=PlusIsRightCancellativeProp]
    For all $k\in\Nat$ if $n\plus k\eq m\plus k$ then $n\eq m$.
  \end{proposition}
  \begin{proof}[forthel,method=induction on $k$]
    Let $k\in\Nat$.

    Case $k\eq0$. Obvious.

    Case $k\neq0$. Obvious.
  \end{proof}

  \begin{corollary}[forthel,title=Left cancellation,id=PlusIsLeftCancellativeProp]
    If $k\plus n\eq k\plus m$ then $n\eq m$.
  \end{corollary}
  \begin{proof}[forthel]
    Assume $k\plus n\eq k\plus m$.
    We have $k\plus n\eq n\plus k$ and $k\plus m\eq m\plus k$.
    Hence $n\plus k\eq m\plus k$.
    Thus $n\eq m$.
  \end{proof}

  \begin{proposition}[forthel,id=ZeroSumImpliesZeroSummandsProp]
    If $n\plus m\eq0$ then $n\eq0$ and $m\eq0$.
  \end{proposition}
\end{smodule}
\end{document}
