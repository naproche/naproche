\documentclass{naproche-library}
\libinput{preamble}
\begin{document}
\begin{smodule}{poset.ftl}

\importmodule[preliminaries]{sets.ftl}

\STEXexport{
  \let\oldleq\leq
  \let\leq\relax
  \let\oldgeq\geq
  \let\geq\relax
}

\begin{mathstructure}{poset}
  \symdef{univ}{X}
  \symdef{leq}[op=\oldleq]{\;\oldleq\;}
  \symdef{geq}[op=\oldgeq]{\;\oldgeq\;}
  \symdef{less}[op=<]{\;<\;}
  \symdef{more}[op=>]{\;>\;}
\end{mathstructure}

\vardef{Pvar}[return={\poset[comp=##1]}]{\mathcal P}

\section{Posets}

\subsection{Definition}
  
\begin{signature*}[forthel,id=PosetSig]
  A \emph{partially ordered set $\Pvar{}[univ,leq]$} is a structure with the following properties:

  \begin{itemize}
    \item \label{posetUnivSig}
      $\Pvar{univ}$ is a set.
    \item \label{posetLeqSig}
      Let $x, y \in \Pvar{univ}$. $x {\Pvar{leq}} y$ is a relation.
    \item \label{posetReflAx}
      For all $x \in \Pvar{univ}$ it holds that $x {\Pvar{leq}} x$.
    \item \label{posetTransAx}
      For all $x, y, z \in \Pvar{univ}$ it holds that $x {\Pvar{leq}} y {\Pvar{leq}} z$ implies $x {\Pvar{leq}} z$.
    \item \label{posetAntisymAx}
      For all $x, y \in \Pvar{univ}$ it holds that $x {\Pvar{leq}} y$ and $y {\Pvar{leq}} x$ implies $x\eq y$.
  \end{itemize}

  Let a \emph{poset} stand for a partially ordered set.
  Let $x {\Pvar{geq}} y$ stand for $y {\Pvar{leq}} x$.
  Let $x {\Pvar{less}} y$ stand for $x {\Pvar{leq}} y$ and $x\neq y$.
  Let $x {\Pvar{more}} y$ stand for $x {\Pvar{geq}} y$ and $x\neq y$.
\end{signature*}

\begin{forthel}
  Let $\Pvar!=\Pvar{}[univ,leq]$ denote a poset.
\end{forthel}

\subsection{Upper and Lower Bounds}

\subsubsection{Upper Bounds}

\begin{definition*}[forthel,id=UpperBoundOfPairDef]
  Let $x, y \in \Pvar{univ}$.
  An \emph{upper bound of $x$ and $y$ in $\Pvar!$} is an element $u$ of $\Pvar{univ}$ such that $u {\Pvar{geq}} x$ and $u {\Pvar{geq}} y$.
\end{definition*}
    
\begin{definition*}[forthel,id=UpperBoundOfSetDef]
  Let $A \subseteq \Pvar{univ}$.
  An \emph{upper bound of $A$ in $\Pvar!$} is an element $u$ of $\Pvar{univ}$ such that $u {\Pvar{geq}} x$ for all $x \in A$.
\end{definition*}

\subsubsection{Lower Bounds}
  
\begin{definition*}[forthel,id=LowerBoundOfPairDef]
  Let $x, y \in \Pvar{univ}$.
  A \emph{lower bound of $x$ and $y$ in $\Pvar!$} is an element $l$ of $\Pvar{univ}$ such that $l {\Pvar{leq}} x$ and $l {\Pvar{leq}} y$.
\end{definition*}
    
\begin{definition*}[forthel,id=LowerBoundOfSetDef]
  Let $A \subseteq \Pvar{univ}$.
  A \emph{lower bound of $A$ in $\Pvar!$} is an element $l$ of $\Pvar{univ}$ such that $l {\Pvar{leq}} x$ for all $x \in A$.
\end{definition*}

\subsection{Joins and Meets}

\subsubsection{Joins}

\begin{definition*}[forthel,id=JoinOfPairDef]
  Let $x, y \in \Pvar{univ}$.
  A \emph{join of $x$ and $y$ in $\Pvar!$} is an upper bound $j$ of $x$ and $y$ in $\Pvar!$ such that for any upper bound $u$ of $x$ and $y$ in $\Pvar!$ we have $j {\Pvar{leq}} u$.

  Let a \emph{least upper bound of $x$ and $y$ in $\Pvar!$} stand for a join of $x$ and $y$ in $\Pvar!$.
  Let a \emph{supremum of $x$ and $y$ in $\Pvar!$} stand for a join of $x$ and $y$ in $\Pvar!$.
\end{definition*}

\begin{proposition*}[forthel,id=JoinOfPairIsUniqueThm]
  Let $x,y\in\Pvar{univ}$ and $j, j'$ be joins of $x$ and $y$ in $\Pvar!$.
  Then $j\eq j'$.
\end{proposition*}
    
\begin{definition*}[forthel,id=JoinOfSetDef]
  Let $A \subseteq \Pvar{univ}$.
  A \emph{join of $A$ in $\Pvar!$} is an upper bound $j$ of $A$ in $\Pvar!$ such that for any upper bound $u$ of $A$ in $\Pvar!$ we have $j {\Pvar{leq}} u$.

  Let a \emph{least upper bound of $A$ in $\Pvar!$} stand for a join of $A$ in $\Pvar!$.
  Let a \emph{supremum of $A$ in $\Pvar!$} stand for a join of $A$ in $\Pvar!$.
\end{definition*}

\begin{proposition*}[forthel,id=JoinOfSetIsUniqueThm]
  Let $A \subseteq \Pvar{univ}$ and $j, j'$ be joins of $A$ in $\Pvar!$.
  Then $j\eq j'$.
\end{proposition*}

\subsubsection{Meets}

\begin{definition*}[forthel,id=MeetOfPairDef]
  Let $x, y \in \Pvar{univ}$.
  A \emph{meet of $x$ and $y$ in $\Pvar!$} is a lower bound $m$ of $x$ and $y$ in $\Pvar!$ such that for any lower bound $l$ of $x$ and $y$ in $\Pvar!$ we have $m {\Pvar{geq}} l$.

  Let a \emph{greatest lower bound of $x$ and $y$ in $\Pvar!$} stand for a meet of $x$ and $y$ in $\Pvar!$.
  Let a \emph{infimum of $x$ and $y$ in $\Pvar!$} stand for a meet of $x$ and $y$ in $\Pvar!$.
\end{definition*}

\begin{proposition*}[forthel,id=MeetOfPairIsUniqueThm]
  Let $x, y \in \Pvar{univ}$ and $m, m'$ be meets of $x$ and $y$ in $\Pvar!$.
  Then $m\eq m'$.
\end{proposition*}
    
\begin{definition*}[forthel,id=MeetOfSetDef]
  Let $A \subseteq \Pvar{univ}$.
  A \emph{meet of $A$ in $\Pvar!$} is a lower bound $m$ of $A$ in $\Pvar!$ such that for any lower bound $l$ of $A$ in $\Pvar!$ we have $m {\Pvar{geq}} l$.

  Let a \emph{least upper bound of $A$ in $\Pvar!$} stand for a meet of $A$ in $\Pvar!$.
  Let a \emph{infimum of $A$ in $\Pvar!$} stand for a meet of $A$ in $\Pvar!$.
\end{definition*}

\begin{proposition*}[forthel,id=MeetOfSetIsUniqueThm]
  Let $A \subseteq \Pvar{univ}$ and $m, m'$ be meets of $A$ in $\Pvar!$.
  Then $m\eq m'$.
\end{proposition*}

\subsection{Greatest and Least Elements}

\subsubsection{Greatest Elements}

\begin{definition*}[forthel,id=GreatestElementDef]
  A \emph{greatest element of $\Pvar!$} is an element $g$ of $\Pvar{univ}$ such that $g {\Pvar{geq}} x$ for all $x \in \Pvar{univ}$.
\end{definition*}

\begin{proposition*}[forthel,id=GreatestElementIsUniqueThm]
  Let $g, g'$ be greatest elements of $\Pvar!$.
  Then $g\eq g'$.
\end{proposition*}

\subsubsection{Least Elements}

\begin{definition*}[forthel,id=LeastElementDef]
  A \emph{least element of $\Pvar!$} is an element $l$ of $\Pvar{univ}$ such that $l {\Pvar{leq}} x$ for all $x \in \Pvar{univ}$.
\end{definition*}

\begin{proposition*}[forthel,id=LeastElementIsUniqueThm]
  Let $l, l'$ be greatest elements of $\Pvar!$.
  Then $l\eq l'$.
\end{proposition*}
\end{smodule}
\end{document}
