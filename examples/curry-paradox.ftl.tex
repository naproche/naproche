\documentclass{article}
\usepackage[utf8]{inputenc}
\usepackage[english]{babel}
\usepackage{meta-inf/lib/naproche}
\usepackage{amssymb}
\usepackage{pgffor}

\newcommand{\id}{\textrm{id}}
\newcommand{\range}{\textrm{range}}
\newcommand{\const}[2]{\textrm{const}_{#1}^{#2}}
\newcommand{\symdiff}{\mathop{\triangle}}
\newcommand{\pow}{\mathcal{P}}
\newcommand{\onto}{\twoheadrightarrow}
\newcommand{\into}{\hookrightarrow}

% E.g. `\set{1, 2, 3, 4}` prints the numbers 1, 2, 3 and 4 as a comma
% separated list which is enclosed within curly braces.
\def\set#1{\ensuremath{
  \{\foreach[count=\i]\x in {#1}{
    \ifnum\i>1,\,\fi
    \x
  }\}
}}

\renewcommand{\succ}{\textrm{succ}}
\newcommand{\pred}{\textrm{pred}}
\newcommand{\V}{\textbf{V}}
\newcommand{\Ord}{\textbf{Ord}}
\newcommand{\Card}{\textbf{Card}}


\usepackage[backend=bibtex]{biblatex}
\usepackage{csquotes}
\addbibresource{meta-inf/lib/bibliography}

\usepackage[
  type=CC,
  modifier=by-nc-sa,
  version=4.0,
]{doclicense}

\newcommand{\var}[1]{\textrm{v}_{#1}}
\newcommand{\abs}[2]{\lambda\var{#1}.\ #2}
\newcommand{\app}[2]{(#1)(#2)}
\newcommand{\fix}{\texttt{fix}}

\title{Curry's Paradox}
\author{\Naproche formalization:\\[0.5em]Marcel Schütz}
\date{2024}

\begin{document}
  \maketitle

  \noindent \emph{Curry's Paradox} is a paradox described by Haskell Curry in 
  1942 \cite{Curry1942}.
  It allows the derivation of an arbitrary statement from a self-referential 
  expression that presupposes its own validity.

  \section*{Untyped $\lambda$-Calculus}

  \begin{imports}
    \begin{forthel}
      [prove off][check off]
      [readtex \path{libraries/source/arithmetics/natural-numbers.ftl.tex}]
      [prove on][check on]
    \end{forthel}
  \end{imports}

  \noindent Our formalization of Curry's Paradox in \Naproche is based on an 
  untyped $\lambda$-calculus:

  \begin{forthel}
    \begin{signature*}
      An expression is a notion.
    \end{signature*}
  
    Let $E, E'$ denote expressions.
  
    \begin{signature*}
      A variable is an expression.
    \end{signature*}
  
    Let $n$ denote a natural number.
  
    \begin{signature*}
      $\var{n}$ is a variable.
    \end{signature*}
  
    \begin{signature*}[Abstraction]\label{abstraction}
      $\abs{n}{E}$ is an expression.
    \end{signature*}
  
    \begin{signature*}[Application]\label{application}
      $\app{E}{E'}$ is an expression.
    \end{signature*}
  
    \begin{signature*}[Fixed-point combinator]\label{fixed_point_combinator}
      $\fix$ is an expression such that
      \[\app{\fix}{E} = \app{E}{\app{\fix}{E}}\]
      for every expression $E$.
    \end{signature*}
  \end{forthel}


  \section*{Propositional Logic}

  \noindent We extend this $\lambda$-calculus by a fragment of propositional
  logic:

  \begin{forthel}  
    Let $E, E'$ denote expressions.
    Let $n$ stand for a natural number.

    \begin{signature*}[Implication]\label{implication}
      $E \rightarrow E'$ is an expression.
    \end{signature*}
  
    \begin{signature*}[Truth]\label{truth}
      $E$ is true is a relation.
    \end{signature*}
  
    \begin{axiom*}[$\beta$-reduction]\label{beta_reduction}
      $\app{\abs{n}{\var{n} \rightarrow E'}}{E} = E \rightarrow E'$.
    \end{axiom*}
  
    \begin{axiom*}[Reflexivity]\label{reflexivity}
      $E \rightarrow E$ is true.
    \end{axiom*}
  
    \begin{axiom*}[Modus Ponens]\label{modus_ponens}
      If $E$ is true and $E \rightarrow E'$ is true then $E'$ is true.
    \end{axiom*}
  
    \begin{axiom*}[Strengthening]\label{strengthening}
      If $E \rightarrow (E \rightarrow E')$ is true then $E \rightarrow E'$ is true.
    \end{axiom*}
  \end{forthel}


  \section*{Curry's paradox}

  \noindent Using the fixpoint combinator from above we can formulate a 
  self-referential expression $X$ of the form ``If $X$ is true then $E$ is
  true'' for any arbitrary expression $E$ by defining
  $X = \app{\fix}{\abs{0}{\var{0} \rightarrow E}}$.
  From the existence of such an expression $X$ together with the above axioms 
  we can then derive that any expression $E$ is true.

  \begin{forthel}
    \begin{theorem*}[Curry's paradox]\label{curry_paradox}
      Every expression is true.
    \end{theorem*}
    \begin{proof}
      Let $E$ be an expression.
      Take $N = \abs{0}{\var{0} \rightarrow E}$ and $X = \app{\fix}{N}.$
  
      (1) Then $X = \app{N}{X} = X \rightarrow E$ (by \nameref{beta_reduction}).
  
      Hence $X \rightarrow (X \rightarrow E)$ is true (by 1, \nameref{reflexivity}).
  
      (2) Thus $X \rightarrow E$ is true (by \nameref{strengthening}).
  
      (3) Therefore $X$ is true (by 1, 2).
  
      Consequently $E$ is true (by \nameref{modus_ponens}, 2, 3).
    \end{proof}
  \end{forthel}

  \printbibliography

  \section*{Copyright}
  \doclicenseThis
\end{document}
