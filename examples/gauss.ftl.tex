\documentclass{article}
\usepackage[english]{babel}
\usepackage{../lib/tex/naproche}
\usepackage{amssymb}
\usepackage{pgffor}

\newcommand{\id}{\textrm{id}}
\newcommand{\range}{\textrm{range}}
\newcommand{\const}[2]{\textrm{const}_{#1}^{#2}}
\newcommand{\symdiff}{\mathop{\triangle}}
\newcommand{\pow}{\mathcal{P}}
\newcommand{\onto}{\twoheadrightarrow}
\newcommand{\into}{\hookrightarrow}

% E.g. `\set{1, 2, 3, 4}` prints the numbers 1, 2, 3 and 4 as a comma
% separated list which is enclosed within curly braces.
\def\set#1{\ensuremath{
  \{\foreach[count=\i]\x in {#1}{
    \ifnum\i>1,\,\fi
    \x
  }\}
}}

\useackage{amssymb}

\newcommand{\Nat}{\mathbb{N}}
\newcommand{\Prime}{\mathbb{P}}
\renewcommand{\succ}{\textrm{succ}}
\newcommand{\pred}{\textrm{pred}}
\newcommand{\add}{\textrm{add}}
\newcommand{\mul}{\textrm{mul}}
\renewcommand{\exp}{\textrm{exp}}
\newcommand{\fac}{\textrm{fac}}
\renewcommand{\div}{\mathrel{\textrm{div}}}
\renewcommand{\mod}{\mathrel{\textrm{mod}}}


\usepackage[
  type=CC,
  modifier=by-nc-sa,
  version=4.0,
]{doclicense}

\newcommand{\gauss}{\mathcal{G}}

\title{Little Gauß' Theorem}
\author{\Naproche formalization: \vspace{0.5em} \\
Christian Schöner, Marcel Schütz}
\date{2024}

\begin{document}
  \pagenumbering{gobble}
  \maketitle

  \begin{imports}
    \begin{forthel}
      %[prove off][check off]
      [readtex \path{libraries/source/arithmetics/multiplication.ftl.tex}]
      %[prove on][check on]
    \end{forthel}
  \end{imports}

  \noindent This is a formalization of ``Little Gauß' Theorem'', i.e. of
  the assertion that
  \[\sum_{k=0}^n k = \frac{k(k + 1)}2\]
  for all $n \in \Nat$.
  In \Naproche we can define the sum $\sum_{n=0}^k n$ by the function
  $\gauss : \Nat \to \Nat$ with $\gauss(0) = 0$ and
  $\gauss(n + 1) = \gauss(n) + (n + 1)$:

  \begin{forthel}
    \begin{signature*}
      Let $n$ be a natural number.
      $\gauss(n)$ is a natural number.
    \end{signature*}

    \begin{axiom*}
      $\gauss(0) = 0$.
    \end{axiom*}

    \begin{axiom*}
      Let $n$ be a natural number.
      Then $\gauss(n + 1) = \gauss(n) + (n + 1)$.
    \end{axiom*}
  \end{forthel}

  \noindent Then the theorem can be stated as follows.

  \begin{forthel}
    \begin{theorem*}[Little Gauß]\label{little_gauss}
      For all natural numbers $n$ we have
      \[2 \cdot \gauss(n) = n \cdot (n + 1).\]
    \end{theorem*}
    \begin{proof}
      Define $\Phi = \{n \in \Nat \mid 2 \cdot \gauss(n) = n \cdot (n + 1)\}$.

      (1) $0 \in \Phi$.

      (2) For all $n \in \Phi$ we have $n + 1 \in \Phi$.\\
      Proof.
        Let $n \in \Phi$.
        Then $2 \cdot \gauss(n) = n \cdot (n + 1)$.
        Hence $2 \cdot \gauss(n + 1)
          = 2 \cdot (\gauss(n) + (n + 1))
          = (2 \cdot \gauss(n)) + (2 \cdot (n + 1))
          = (n \cdot (n + 1)) + (2 \cdot (n + 1))
          = ((n + 1) \cdot n) + ((n + 1) \cdot 2)
          = (n + 1) \cdot (n + 2)
          = (n + 1) \cdot ((n + 1) + 1)$.
        Therefore $n + 1 \in \Phi$.
      Qed.

      Thus $\Phi$ contains every natural number.
      Consequently $2 \cdot \gauss(n) = n \cdot (n + 1)$ for every $n \in \Nat$.
    \end{proof}
  \end{forthel}

  \section*{Copyright}
  \doclicenseThis
\end{document}
