\documentclass{article}
\usepackage[stex]{naproche}
\libinput{preamble}
\begin{document}
\begin{smodule}[creators={Marcel Schütz}]{symmetric-difference-is-right-injective.ftl}
  \importmodule[naproche/examples/foundations]{def/classes?symmetric-difference.ftl}
  \importmodule[naproche/examples/foundations]{def/classes?subclass.ftl}

  \begin{fproposition*}[label=6530966763864064]
    Let $A, B, C$ be classes.
    If $\symdiff{A}{B} = \symdiff{A}{C}$ then $B = C$.
  \end{fproposition*}
  \begin{fproof}
    Assume $\symdiff{A}{B} = \symdiff{A}{C}$.

    Let us show that $B \subclass[eq] C$.
      Let $b \in B$.

      Case $b \in A$.
        Then $b \notin \symdiff{A}{B}$.
        Hence $b \notin \symdiff{A}{C}$.
        Therefore $b \in \intersection{A,C}$.
        Indeed $\symdiff{A}{C} = \complement{\union{A,C}}{\intersection{A,C}}$.
        Hence $b \in C$.
      End.

      Case $b \notin A$.
        Then $b \in \symdiff{A}{B}$.
        Indeed $b \in \union{A,B}$ and $b \notin \intersection{A,C}$.
        Hence $b \in \symdiff{A}{C}$.
        Thus $b \in \union{A,C}$ and $b \notin \intersection{A,C}$.
        Therefore $b \in A$ or $b \in C$.
        Then we have the thesis.
      End.
    End.

    Let us show that $C \subclass[eq] B$.
      Let $c \in C$.

      Case $c \in A$.
        Then $c \notin \symdiff{A}{C}$.
        Hence $c \notin \symdiff{A}{B}$.
        Therefore $c \in \intersection{A,B}$.
        Indeed $c \notin \union{A,B}$ or $c \in \intersection{A,B}$.
        Hence $c \in B$.
      End.

      Case $c \notin A$.
        Then $c \in \symdiff{A}{C}$.
        Indeed $c \in \union{A,C}$ and $c \notin \intersection{A,C}$.
        Hence $c \in \symdiff{A}{B}$.
        Thus $c \in \union{A,B}$ and $c \notin \intersection{A,B}$.
        Therefore $c \in A$ or $c \in B$.
        Then we have the thesis.
      End.
    End.
  \end{fproof}
\end{smodule}
\end{document}
