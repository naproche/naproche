\documentclass{article}
\usepackage[stex]{naproche}
\libinput{preamble}
\begin{document}
\begin{smodule}[creators={Marcel Schütz}]{complement-of-intersection-is-union-of-complements.ftl}
  \importmodule[naproche/examples/foundations]{def/classes?intersection.ftl}
  \importmodule[naproche/examples/foundations]{def/classes?union.ftl}
  \importmodule[naproche/examples/foundations]{def/classes?complement.ftl}
  \importmodule[naproche/examples/foundations]{thm?subclass-antisym.ftl}

  \begin{fproposition*}[label=2864251928576000]
    Let $A, B, C$ be classes.
    Then $\complement{A}{\intersection{B,C}} = \union{\complement{A}{B},\complement{A}{C}}$.
  \end{fproposition*}
  \begin{fproof}
    Let us show that $\complement{A}{\intersection{B,C}} \subclass[eq]
    \union{\complement{A}{B},\complement{A}{C}}$.
      Let $x \in \complement{A}{\intersection{B,C}}$.
      Then $x \in A$ and $x \notin \intersection{B,C}$.
      Hence it is wrong that ($x \in B$ and $x \in C$).
      Thus $x \notin B$ or $x \notin C$.
      Therefore $x \in A$ and ($x \notin B$ or $x \notin C$).
      Then ($x \in A$ and $x \notin B$) or ($x \in A$ and $x \notin C$).
      Hence $x \in \complement{A}{B}$ or $x \in \complement{A}{C}$.
      Thus $x \in \union{\complement{A}{B},\complement{A}{C}}$.
    End.

    Let us show that $\union{\complement{A}{B},\complement{A}{C}} \subclass[eq]
    \complement{A}{\intersection{B}{C}}$.
      Let $x \in \union{\complement{A}{B},\complement{A}{C}}$.
      Then $x \in \complement{A}{B}$ or $x \in \complement{A}{C}$.
      Hence ($x \in A$ and $x \notin B$) or ($x \in A$ and $x \notin C$).
      Thus $x \in A$ and ($x \notin B$ or $x \notin C$).
      Therefore $x \in A$ and not ($x \in B$ and $x \in C$).
      Then $x \in A$ and not $x \in \intersection{B,C}$.
      Hence $x \in \complement{A}{\intersection{B,C}}$.
    End.
  \end{fproof}
\end{smodule}
\end{document}
