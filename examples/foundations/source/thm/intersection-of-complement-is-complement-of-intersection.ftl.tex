\documentclass{article}
\usepackage[stex]{naproche}
\libinput{preamble}
\begin{document}
\begin{smodule}[creators={Marcel Schütz}]{intersection-of-complement-is-complement-of-intersection.ftl}
  \importmodule[naproche/examples/foundations]{def/classes?complement.ftl}
  \importmodule[naproche/examples/foundations]{def/classes?intersection.ftl}
  \importmodule[naproche/examples/foundations]{thm?subclass-antisym.ftl}

  \begin{fproposition*}[label=5672367154003968]
    Let $A, B, C$ be classes.
    Then $A \intersection (B \complement C) = (A \intersection B) \complement (A \intersection C)$.
  \end{fproposition*}
  \begin{proof}
    Let us show that $A \intersection (B \complement C) \subclass[eq]
    (A \intersection B) \complement (A \intersection C)$.
      Let $x \in A \intersection (B \complement C)$.
      Then $x \in A$ and $x \in B \complement C$.
      Hence $x \in A$ and $x \in B$.
      Thus $x \in A \intersection B$ and $x \notin C$.
      Therefore $x \notin A \intersection C$.
      Then we have $x \in (A \intersection B) \complement (A \intersection C)$.
    End.

    Let us show that $((A \intersection B) \complement (A \intersection C)) \subclass[eq]
    A \intersection (B \complement C)$. %!
      Let $x \in (A \intersection B) \complement (A \intersection C)$.
      Then $x \in A$ and $x \in B$.
      $x \notin A \intersection C$.
      Hence $x \notin C$.
      Thus $x \in B \complement C$.
      Therefore $x \in A \intersection (B \complement C)$.
    End.
  \end{proof}
\end{smodule}
\end{document}
