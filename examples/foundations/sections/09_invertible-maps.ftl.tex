\documentclass[../../set-theory/set-theory.tex]{subfiles}

\begin{document}
  \chapter{Invertible maps and involutions}\label{chapter:invertible-maps}

  \filename{foundations/sections/09_invertible-maps.ftl.tex}

  \begin{forthel}
    %[prove off][check off]

    [readtex \path{foundations/sections/08_injections-surjections-bijections.ftl.tex}]

    %[prove on][check on]
  \end{forthel}


  \section{Invertible maps}

  \begin{forthel}
    \begin{definition}\printlabel{FOUNDATIONS_09_7776974319648768}
      Let $f$ be a map.
      An inverse of $f$ is a map $g$ from $\range{f}$ to $\dom(f)$ such that
      \[ f(a) = b \iff g(b) = a \]
      for all $a \in \dom(f)$ and all $b \in \dom(g)$.
    \end{definition}
  \end{forthel}

  \begin{forthel}
    \begin{definition}\printlabel{FOUNDATIONS_09_3430350086733824}
      Let $f$ be a map.
      $f$ is invertible iff $f$ has an inverse.
    \end{definition}
  \end{forthel}

  \begin{forthel}
    \begin{lemma}\printlabel{FOUNDATIONS_09_5108611793551360}
      Let $f$ be a map and $g, g'$ be inverses of $f$.
      Then $g = g'$.
    \end{lemma}
    \begin{proof}
      We have $\dom(g) = \range{f} = \dom(g')$.

      Let us show that $g(b) = g'(b)$ for all $b \in \range{f}$.
        Let $b \in \range{f}$.
        Take $a = g'(b)$.
        Then $g(b) = a$ iff $f(a) = b$.
        We have $f(a) = b$ iff $g'(b) = a$.
        Thus $g(b) = g'(b)$.
      End.
    \end{proof}
  \end{forthel}

  \begin{forthel}
    \begin{definition}\printlabel{FOUNDATIONS_09_6458627204317184}
      Let $f$ be an invertible map.
      $\inverse{f}$ is the inverse of $f$.
    \end{definition}

    Let $f$ is involutory stand for $f$ is the inverse of $f$.
    Let $f$ is selfinverse stand for $f$ is the inverse of $f$.
  \end{forthel}


  \section{Some basic facts about invertible maps}

  \begin{forthel}
    \begin{proposition}\printlabel{FOUNDATIONS_09_7840743571849216}
      Let $A, B$ be classes and $f : A \onto B$ and $g : B \onto A$.
      Then $g$ is the inverse of $f$ iff $g \circ f = \id{A}$ and $f \circ g =
      \id{B}$.
    \end{proposition}
    \begin{proof}
      Case $g$ is the inverse of $f$.
        We have
        $\dom(g \circ f)
          = \dom(f)
          = A
          = \dom(\id{A})$.
        For all $a \in A$ we have
        $(g \circ f)(a)
          = g(f(a))
          = a$.
        Hence $g \circ f = \id{A}$.

        We have
        $\dom(f \circ g)
          = \dom(g)
          = B
          = \dom(\id{B})$.
        For all $b \in B$ we have
        $(f \circ g)(b)
          = f(g(b))
          = b$.
        Hence $f \circ g = \id{B}$.
      End.

      Case $g \circ f = \id{A}$ and $f \circ g = \id{B}$.
        Then $\dom(g)
          = B
          = \range{f}$
        and $\range{g}
          = A
          = \dom(f)$.
        Let $a \in \dom(f)$ and $b \in \dom(g)$.
        If $f(a) = b$ then
        $g(b)
          = g(f(a))
          = (g \circ f)(a)
          = \id{A}(a)
          = a$.
        If $g(b) = a$ then
        $f(a)
          = f(g(b))
          = (f \circ g)(b)
          = \id{B}(b)
          = b$.
        Hence $f(a) = b$ iff $g(b) = a$.
      End.
    \end{proof}
  \end{forthel}

  \begin{forthel}
    \begin{proposition}\printlabel{FOUNDATIONS_09_8414736098000896}
      Let $A, B$ be classes and $f : A \onto B$.
      Assume that $f$ is invertible.
      Then $\inverse{f}$ is an invertible surjective map from $B$ onto $A$ such that
      \[ \inverse{\inverse{f}} = f. \]
    \end{proposition}
    \begin{proof}
      $\inverse{f}$ is a map from $B$ to $A$.
      Indeed $\range{f} = B$ and $\dom(f) = A$.
      $\inverse{f}$ is surjective onto $A$.
      Indeed for any $a \in A$ we have $\inverse{f}(f(a)) = a$.
      $\inverse{f}$ is the inverse of $f$.
      Thus $f \circ \inverse{f} = \id{B}$ and $\inverse{f} \circ f = \id{A}$.
      Therefore $f$ is the inverse of $\inverse{f}$.
    \end{proof}
  \end{forthel}

  \begin{forthel}
    \begin{proposition}\printlabel{FOUNDATIONS_09_4577560740495360}
      Let $A, B$ be classes and $f : A \onto B$.
      Assume that $f$ is invertible.
      Then \[ f \circ \inverse{f} = \id{B} \] and \[ \inverse{f} \circ f = \id{A}. \]
    \end{proposition}
    \begin{proof}
      $\inverse{f}$ is a surjective map from $B$ onto $A$ .
      $\inverse{f}$ is the inverse of $f$.
    \end{proof}
  \end{forthel}

  \begin{forthel}
    \begin{proposition}\printlabel{FOUNDATIONS_09_4606651604664320}
      Let $A, B$ be classes and $f : A \onto B$ and $a \in A$.
      Assume that $f$ is invertible.
      Then \[ \inverse{f}(f(a)) = a. \]
    \end{proposition}
    \begin{proof}
      We have $\inverse{f}(f(a)) = (\inverse{f} \circ f)(a) = \id{A}(a) = a$.
    \end{proof}

    \begin{proposition}
      Let $A, B$ be classes and $f : A \onto B$ and $b \in B$.
      Assume that $f$ is invertible.
      Then \[ f(\inverse{f}(b)) = b. \]
    \end{proposition}
    \begin{proof}
      We have
      $f(\inverse{f}(b))
        = (f \circ \inverse{f})(b)
        = \id{B}(b)
        = b$.
    \end{proof}
  \end{forthel}

  \begin{forthel}
    \begin{proposition}\printlabel{FOUNDATIONS_09_7619151963095040}
      Let $A, B, C$ be classes and $f : A \onto B$ and $g : B \onto C$.
      Assume that $f$ and $g$ are invertible.
      Then $g \circ f$ is invertible and
      \[ \inverse{g \circ f} = \inverse{f} \circ \inverse{g}. \]
    \end{proposition}
    \begin{proof}
      $\inverse{f}$ is a surjective map from $B$ onto $A$.
      $\inverse{g}$ is a surjective map from $C$ onto $B$.
      Take $h = \inverse{f} \circ \inverse{g}$.
      Then $h$ is a surjective map from $C$ onto $A$
      (by \cref{FOUNDATIONS_08_8542698338254848}).
      $g \circ f$ is a map from $A$ to $C$.

      Let us show that $((g \circ f) \circ h) = \id{C}$.
        We have $f \circ (\inverse{f} \circ \inverse{g}) = (f \circ \inverse{f}) \circ \inverse{g}$.
        Indeed $f \circ (\inverse{f} \circ \inverse{g})$ and
        $(f \circ \inverse{f}) \circ \inverse{g}$ are maps of $C$.
        $f \circ h$ is a map from $C$ to $B$.
        Hence
        \[  (g \circ f) \circ h                           \]
        \[    = g \circ (f \circ h)                       \]
        \[    = g \circ (f \circ (\inverse{f} \circ \inverse{g}))   \]
        \[    = g \circ ((f \circ \inverse{f}) \circ \inverse{g})   \]
        \[    = g \circ (\id{B} \circ \inverse{g})            \]
        \[    = g \circ \inverse{g}                            \]
        \[    = \id{C}.                                  \]
      End.

      Let us show that $h \circ (g \circ f) = \id{A}$.
        We have $(\inverse{f} \circ \inverse{g}) \circ g = \inverse{f} \circ (\inverse{g} \circ g)$.
        $g \circ f$ is a map from $A$ to $C$.
        Hence
        \[  h \circ (g \circ f)                           \]
        \[    = (h \circ g) \circ f                       \]
        \[    = ((\inverse{f} \circ \inverse{g}) \circ g) \circ f   \]
        \[    = (\inverse{f} \circ (\inverse{g} \circ g)) \circ f   \]
        \[    = (\inverse{f} \circ \id{B}) \circ f            \]
        \[    = \inverse{f} \circ f                            \]
        \[    = \id{A}.                                  \]
      End.

      Thus $h$ is the inverse of $g \circ f$.
      Indeed $g \circ f$ is a surjective map from $A$ onto $C$ and $h$ is a
      surjective map from $C$ onto $A$.
    \end{proof}
  \end{forthel}

  \begin{forthel}
    \begin{proposition}\printlabel{FOUNDATIONS_09_6374884963778560}
      Let $A, B$ be classes and $f : A \onto B$ and $X \subseteq A$.
      Assume that $f$ is invertible.
      Then $f \restriction X$ is invertible and
      \[ \inverse{f\restriction X} = \inverse{f} \restriction (\image{f}(X)). \]
    \end{proposition}
    \begin{proof}
      $f \restriction X$ is a surjective map from $X$ onto $\image{f}(X)$.
      Take $g = \inverse{f} \restriction (\image{f}(X))$.
      Then $g$ is a map of $\image{f}(X)$.

      Let us show that $X \subseteq \range{g}$.
        Let $a \in X$.
        Then $f(a) \in \image{f}(X)$.
        Hence $g(f(a)) = \inverse{f}(f(a)) = a$.
        Thus $a$ is a value of $g$.
      End.

      Let us show that $\range{g} \subseteq X$.
        Let $a \in \range{g}$.
        Take $b \in \image{f}(X)$ such that $a = g(b)$.
        Take $c \in X$ such that $b = f(c)$.
        Then $a
          = (\inverse{f} \restriction (\image{f}(X)))(b)
          = \inverse{f}(b)
          = \inverse{f}(f(c))
          = c$.
        Hence $a \in X$.
      End.

      Hence $\range{g} = X$.
      Thus $g$ is a surjective map onto $X$.

      Let us show that $g((f \restriction X)(a)) = a$ for all $a \in X$.
        Let $a \in X$.
        Then $g((f \restriction X)(a))
          = g(f(a))
          = (\inverse{f} \restriction (\image{f}(X)))(f(a))
          = \inverse{f}(f(a))
          = a$.
      End.

      Let us show that $((f \restriction X)(g(b))) = b$ for all $b \in \image{f}(X)$.
        Let $b \in \image{f}(X)$.
        Take $a \in X$ such that $b = f(a)$.
        We have $g(b)
          = g(f(a))
          = (\inverse{f} \restriction (\image{f}(X)))(f(a))
          = \inverse{f}(f(a))
          = a$.
        Hence $(f \restriction X)(g(b))
          = (f \restriction X)(a)
          = f(a)
          = b$.
      End.

      Thus $g \circ (f \restriction X) = \id{X}$ and
      $(f \restriction X) \circ g = \id{\image{f}(X)}$.
      Therefore $g$ is the inverse of $f \restriction X$.
    \end{proof}
  \end{forthel}

  \begin{forthel}
    \begin{proposition}\printlabel{FOUNDATIONS_09_7726021377785856}
      Let $A, B$ be classes and $f : A \onto B$ and $Y \subseteq B$.
      Assume that $f$ is invertible.
      Then \[ \preimage{f}(Y) = \image{\inverse{f}}(Y). \]
    \end{proposition}
    \begin{proof}
      We have $\image{\inverse{f}}(Y) = \{ \inverse{f}(b) \mid b \in Y \}$ and
      $\preimage{f}(Y) = \{ a \in A \mid f(a) \in Y \}$.

      Let us show that $\preimage{f}(Y) \subseteq \image{\inverse{f}}(Y)$.
        Let $a \in \preimage{f}(Y)$.
        Take $b \in Y$ such that $b = f(a)$.
        Then $\inverse{f}(b) = \inverse{f}(f(a)) = a$.
        Hence $a \in \image{\inverse{f}}(Y)$.
      End.

      Let us show that $\image{\inverse{f}}(Y) \subseteq \preimage{f}(Y)$.
        Let $a \in \image{\inverse{f}}(Y)$.
        Take $b \in Y$ such that $a = \inverse{f}(b)$.
        Then $f(a) = f(\inverse{f}(b)) = b$.
        Hence $a \in \preimage{f}(Y)$.
      End.
    \end{proof}
  \end{forthel}

  \begin{forthel}
    \begin{corollary}\printlabel{FOUNDATIONS_09_8607784268464128}
      Let $A, B$ be classes and $f : A \onto B$ and $b \in B$.
      Assume that $f$ is invertible.
      Then \[ \preimage{f}(\set{b}) = \set{\inverse{f}(b)}. \]
    \end{corollary}
    \begin{proof}
      $\preimage{f}(\set{b}) = \image{\inverse{f}}(\set{b})$.
      We have $\image{\inverse{f}}(\set{b}) = \{ \inverse{f}(c) \mid c \in \set{b} \}$.
      Hence $\image{\inverse{f}}(\set{b}) = \set{\inverse{f}(b)}$.
    \end{proof}
  \end{forthel}

  \begin{forthel}
    \begin{proposition}\printlabel{FOUNDATIONS_09_6777575974109184}
      Let $A, B$ be classes and $f : A \onto B$.
      Then $f$ is invertible iff $f$ is injective.
    \end{proposition}
    \begin{proof}
      Case $f$ is invertible.
        Let $a, b \in A$.
        Assume $f(a) = f(b)$.
        Then $a = \inverse{f}(f(a)) = \inverse{f}(f(b)) = b$.
      End.

      Case $f$ is injective.
        Define $g(b) =$ ``choose $a \in A$ such that $f(a) = b$ in $a$'' for
        $b \in B$.
        Then $g$ is a map from $B$ to $A$.
        For all $a \in A$ we have $a = g(f(a))$.
        Hence $g$ is a surjective map from $B$ onto $A$.
        For all $a \in A$ we have $g(f(a)) = a$.
        For all $b \in B$ we have $f(g(b)) = b$.
        Hence $g$ is the inverse of $f$.
      End.
    \end{proof}
  \end{forthel}

  \begin{forthel}
    \begin{corollary}\printlabel{FOUNDATIONS_09_5708971514003456}
      Let $A, B$ be classes and $f : A \onto B$.
      Assume that $f$ is invertible.
      Then $\inverse{f}$ is a bijection between $B$ and $A$.
    \end{corollary}
    \begin{proof}
      $\inverse{f}$ is a surjective map from $B$ onto $A$.
      $\inverse{f}$ is invertible.
      Hence $\inverse{f}$ is injective.
      Therefore $\inverse{f}$ is a bijection between $B$ and $A$.
    \end{proof}
  \end{forthel}


  \section{Involutions}

  \begin{forthel}
    \begin{definition}\printlabel{FOUNDATIONS_09_7282039688527872}
      Let $A$ be a class.
      An involution on $A$ is a selfinverse map $f$ on $A$.
    \end{definition}
  \end{forthel}

  \begin{forthel}
    \begin{proposition}\printlabel{FOUNDATIONS_09_7944474185433088}
      Let $A$ be a class.
      $\id{A}$ is an involution on $A$.
    \end{proposition}
    \begin{proof}
      We have $\id{A} \circ \id{A} = \id{A}$.
      Hence $\id{A}$ is selfinverse.
    \end{proof}
  \end{forthel}

  \begin{forthel}
    \begin{proposition}\printlabel{FOUNDATIONS_09_6897019612299264}
      Let $A$ be a class and $f, g$ be involutions on $A$.
      Then $g \circ f$ is an involution on $A$ iff $g \circ f = f \circ g$.
    \end{proposition}
    \begin{proof}
      Case $g \circ f$ is an involution on $A$.
        Then $\inverse{g \circ f} = \inverse{f} \circ \inverse{g} = f \circ g$.
      End.

      Case $g \circ f = f \circ g$.
        $f \circ f$, $f \circ g$ and $f \circ g$ are maps on $A$.
        Hence
        \[  (g \circ f) \circ (g \circ f)       \]
        \[    = (g \circ f) \circ (f \circ g)   \]
        \[    = ((g \circ f) \circ f) \circ g   \]
        \[    = (g \circ (f \circ f)) \circ g   \]
        \[    = (g \circ \id{A}) \circ g       \]
        \[    = g \circ g                       \]
        \[    = \id{A}.                        \]
        Thus $g \circ f$ is selfinverse.
      End.
    \end{proof}
  \end{forthel}

  \begin{forthel}
    \begin{corollary}\printlabel{FOUNDATIONS_09_5958206868160512}
      Let $A$ be a class and $f$ be an involutions on $A$.
      Then $f \circ f$ is an involution on $A$.
    \end{corollary}
  \end{forthel}

  \begin{forthel}
    \begin{proposition}\printlabel{FOUNDATIONS_09_2314262743613440}
      Let $A$ be a class and $f$ be an involution on $A$.
      Then $f$ is a permutation of $A$.
    \end{proposition}
    \begin{proof}
      $f$ is an invertible map of $A$ that surjects onto $A$.
      Hence $f$ is a bijection between $A$ and $A$.
      Thus $f$ is a permutation of $A$.
    \end{proof}
  \end{forthel}
\end{document}
