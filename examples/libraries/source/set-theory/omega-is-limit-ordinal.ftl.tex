\documentclass{naproche-library}
\libinput{set-theory-preamble}
\begin{document}
\begin{smodule}{omega-is-limit-ordinal.ftl}
  \importmodule[libraries]{set-theory?omega.ftl}
  \importmodule[libraries]{set-theory?limit-ordinals.ftl}

  \section*{$\omega$ is a Limit Ordinal}

  \begin{proposition}[forthel,id=SET_THEORY_03_764451995254784,printid]
    $\omega$ is a limit ordinal.
  \end{proposition}
  \begin{proof}[forthel]
    $\omega$ is transitive. \\
    Proof.
      Define $\Phi = \{ n \in \omega \mid$ for all $m \in n$ we have $m \in \omega \}$.

      (1) $0 \in \Phi$.

      (2) For all $n \in \Phi$ we have $\succ(n) \in \Phi$. \\
      Proof.
        Let $n \in \Phi$.
        Then every element of $n$ is contained in $\omega$.
        Hence every element of $\succ(n)$ is contained in $\omega$.
        Thus $\succ(n) \in \Phi$.
      Qed.

      Therefore $\omega \subseteq \Phi$.
      Consequently $\omega$ is transitive.
    Qed.

    Every element of $\omega$ is an ordinal.
    Hence every element of $\omega$ is transitive.
    Thus $\omega$ is an ordinal.

    $\omega$ is a limit ordinal. \\
    Proof.
      Assume the contrary.
      We have $\omega \neq 0$.
      Hence $\omega$ is a successor ordinal.
      Take an ordinal $\alpha$ such that $\succ(\alpha) = \omega$.
      Then $\alpha \in \omega$.
      Thus $\omega = \succ(\alpha) \in \omega$.
      Contradiction.
    Qed.
  \end{proof}
\end{smodule}
\end{document}
