\documentclass[10pt]{article}
\usepackage[english]{babel}
\usepackage{../../../../lib/tex/naproche}
% Top-level sections and proofs

\newtheoremstyle{foundations}% name of the style to be used
  {\dimexpr\ftlparskip+4pt\relax}% measure of space to leave above the theorem.
  % (We must add \ftlparskip since otherwise the name of the label would not fit
  % into the border drawn around the environment.)
  {4pt}% measure of space to leave below the theorem.
  {\normalfont}% name of font to use in the body of the theorem
  {0pt}% measure of space to indent
  {\bfseries}% name of head font
  {}% punctuation between head and body
  { }% space after theorem head; " " = normal inter-word space
  {\thmname{#1}\thmnumber{ #2}\thmnote{ (#3)}.}

\let\definition\relax
\let\enddefinition\relax
\let\signature\relax
\let\endsignature\relax
\let\axiom\relax
\let\endaxiom\relax
\let\theorem\relax
\let\endtheorem\relax
\let\proposition\relax
\let\endproposition\relax
\let\lemma\relax
\let\endlemma\relax
\let\corollary\relax
\let\endcorollary\relax

\theoremstyle{foundations}

\newmdtheoremenv[
  backgroundcolor=lightgray!10,
  linecolor=black,
  innertopmargin=1em
]{definition}{Definition}[section]
\newmdtheoremenv[
  backgroundcolor=lightgray!10,
  linecolor=black,
  innertopmargin=1em
]{signature}[definition]{Signature}
\newmdtheoremenv[
  backgroundcolor=lightgray!10,
  linecolor=black,
  innertopmargin=1em
]{axiom}[definition]{Axiom}
\newmdtheoremenv[
  backgroundcolor=lightgray!10,
  linecolor=black,
  innertopmargin=1em
]{theorem}[definition]{Theorem}
\newmdtheoremenv[
  backgroundcolor=lightgray!10,
  linecolor=black,
  innertopmargin=1em
]{proposition}[definition]{Proposition}
\newmdtheoremenv[
  backgroundcolor=lightgray!10,
  linecolor=black,
  innertopmargin=1em
]{lemma}[definition]{Lemma}
\newmdtheoremenv[
  backgroundcolor=lightgray!10,
  linecolor=black,
  innertopmargin=1em
]{corollary}[definition]{Corollary}


% Hide imports

\usepackage{scontents}
\newenvsc{imports}[store-env=imports,print-env=false]


% Labels of top-level sections

\renewcommand{\showlabelfont}{\scriptsize\ttfamily\color{gray}}


% Non-symbolic "\implies" and "\iff"

\renewcommand{\implies}{\quad \text{implies} \quad}
\renewcommand{\iff}{\quad \text{iff} \quad}

\usepackage{amssymb}
\usepackage{pgffor}

\newcommand{\id}{\textrm{id}}
\newcommand{\range}{\textrm{range}}
\newcommand{\const}[2]{\textrm{const}_{#1}^{#2}}
\newcommand{\symdiff}{\mathop{\triangle}}
\newcommand{\pow}{\mathcal{P}}
\newcommand{\onto}{\twoheadrightarrow}
\newcommand{\into}{\hookrightarrow}

% E.g. `\set{1, 2, 3, 4}` prints the numbers 1, 2, 3 and 4 as a comma
% separated list which is enclosed within curly braces.
\def\set#1{\ensuremath{
  \{\foreach[count=\i]\x in {#1}{
    \ifnum\i>1,\,\fi
    \x
  }\}
}}

\renewcommand{\succ}{\textrm{succ}}
\newcommand{\pred}{\textrm{pred}}
\newcommand{\V}{\textbf{V}}
\newcommand{\Ord}{\textbf{Ord}}
\newcommand{\Card}{\textbf{Card}}

\begin{document}
  \begin{imports}
    \begin{forthel}
      %[prove off][check off]
      [readtex \path{libraries/source/set-theory/ordinals.ftl.tex}]
      %[prove on][check on]
    \end{forthel}
  \end{imports}


  \section*{Successors}

  \begin{forthel}
    \begin{definition}\printlabel{SET_THEORY_02_8166925802668032}
      Let $\alpha$ be an ordinal.
      $\succ(\alpha) = \alpha \cup \set{\alpha}$.
    \end{definition}
  \end{forthel}

  \begin{forthel}
    \begin{proposition}\printlabel{SET_THEORY_02_1624410224066560}
      Let $\alpha$ be an ordinal.
      Then $\succ(\alpha)$ is an ordinal.
    \end{proposition}
    \begin{proof}
      (1) $\succ(\alpha)$ is transitive. \\
      Proof.
        Let $x \in \succ(\alpha)$ and $y \in x$.
        Then $x \in \alpha$ or $x = \alpha$.
        Hence $y \in \alpha$.
        Thus $y \in \succ(\alpha)$.
      Qed.

      (2) Every element of $\succ(\alpha)$ is transitive. \\
      Proof.
        Let $x \in \succ(\alpha)$.
        Then $x \in \alpha$ or $x = \alpha$.
        Hence $x$ is transitive.
        Indeed $\alpha$ is transitive and every element of $\alpha$ is transitive.
      Qed.
    \end{proof}
  \end{forthel}

  \begin{forthel}
    \begin{definition}\printlabel{SET_THEORY_02_7129712109289472}
      A successor ordinal is an ordinal $\alpha$ such that $\alpha = \succ(\beta)$ for some ordinal $\beta$.
    \end{definition}
  \end{forthel}

  \begin{forthel}
    \begin{proposition}\printlabel{SET_THEORY_02_8651096763400192}
      Let $\alpha, \beta$ be ordinals.
      If $\succ(\alpha) = \succ(\beta)$ then $\alpha = \beta$.
    \end{proposition}
    \begin{proof}
      Assume $\succ(\alpha) = \succ(\beta)$.

      (1) $\alpha \subseteq \beta$. \\
      Proof.
        Let $\gamma \in \alpha$.
        Then $\gamma \in \alpha \cup \set{\alpha}
          = \succ(\alpha)
          = \succ(\beta)
          = \beta \cup \set{\beta}$.
        Hence $\gamma \in \beta$ or $\gamma = \beta$.
        Assume $\gamma = \beta$.
        Then $\beta \in \alpha$.
        Hence $\beta
          = (\beta \cup \set{\beta}) \setminus \set{\gamma}
          = (\alpha \cup \set{\alpha}) \setminus \set{\gamma}
          = (\alpha \setminus \set{\gamma}) \cup \set{\alpha}$.
        Therefore $\alpha \in \beta$.
        Consequently $\alpha \in \beta \in \alpha$.
        Contradiction.
      Qed.

      (2) $\beta \subseteq \alpha$. \\
      Proof.
        Let $\gamma \in \beta$.
        Then $\gamma \in \beta \cup \set{\beta}
          = \succ(\beta)
          = \succ(\alpha)
          = \alpha \cup \set{\alpha}$.
        Hence $\gamma \in \alpha$ or $\gamma = \alpha$.
        Assume $\gamma = \alpha$.
        Then $\alpha \in \beta$.
        Hence $\alpha
          = (\alpha \cup \set{\alpha}) \setminus \set{\gamma}
          = (\beta \cup \set{\beta}) \setminus \set{\gamma}
          = (\beta \setminus \set{\gamma}) \cup \set{\beta}$.
        Therefore $\beta \in \alpha$.
        Consequently $\beta \in \alpha \in \beta$.
        Contradiction.
      Qed.
    \end{proof}
  \end{forthel}

  \begin{forthel}
    \begin{definition}\printlabel{SET_THEORY_02_735071524880384}
      Let $\alpha$ be a successor ordinal.
      $\pred(\alpha)$ is the ordinal $\beta$ such that $\alpha = \succ(\beta)$.
    \end{definition}
  \end{forthel}
\end{document}
