\documentclass{naproche-library}
\libinput{set-theory-preamble}
\begin{document}
\begin{smodule}{transfinite-recursion-1.ftl}
  \importmodule[libraries]{set-theory?transfinite-recursion-1.ftl}

  \begin{lemma*}[forthel,id=transfinite-recursion_029348573987483]
    Let $A$ be a class and $\alpha$ be an ordinal and $F : \Ord \to A$.
    Then $(\alpha, F(\alpha)) \in \Ord \times A$.
  \end{lemma*}

  \begin{lemma*}[forthel,id=transfinite-recursion_570293857449281]
    Let $A$ be a class and $\lambda$ be a limit ordinal and $F : \Ord \to A$.
    Then $(\lambda, F \restriction \lambda) \in \Ord \times A^{< \infty}$.
  \end{lemma*}
  
  \begin{corollary*}[forthel,id=transfinite-recursion_298647309184593]
    Let $A$ be a class.
    Let $a \in A$ and $G : \Ord \times A \to A$ and $H : \Ord \times A^{< \infty} \to A$.
    Then there exists a map $F$ from $\Ord$ to $A$ such that
    \[ F(0) = a \]
    and for all ordinals $\alpha$ we have
    \[ F(\succ(\alpha)) = G(\alpha, F(\alpha)) \]
    and for all limit ordinals $\lambda$ we have
    \[ F(\lambda) = H(\lambda, F \restriction \lambda). \]
  \end{corollary*}
  \begin{proof}[forthel]
    $(\pred(\dom(f)), f(\pred(\dom(f)))) \in \Ord \times A$ for all $f \in A^{< \infty}$ such that $\dom(f)$ is a successor ordinal.

    Define  \[ J(f) =
      \begin{cases}
        a
        & : \dom(f) = 0
        \\
        G(\pred(\dom(f)), f(\pred(\dom(f))))
        & : \text{$\dom(f)$ is a successor ordinal}
        \\
        H(\dom(f), f)
        & : \text{$\dom(f)$ is a limit ordinal}
      \end{cases} \]
    for $f \in A^{< \infty}$.

    Then $J$ is a map from $A^{< \infty}$ to $A$.
    Indeed we can show that for any $f \in A^{< \infty}$ we have $J(f) \in A$.
      Let $f \in A^{< \infty}$.
      Take $\alpha \in \Ord$ such that $f : \alpha \to A$.
      If $\alpha = 0$ then $J(f) = a \in A$.
      If $\alpha$ is a successor ordinal then $J(f) =
      G(\pred(\alpha), f(\pred(\alpha))) \in A$.
      [prover vampire]
      If $\alpha$ is a limit ordinal then $J(f) = H(\alpha, f) \in A$.
    End.

    Hence we can take a map $F$ from $\Ord$ to $A$ that is recursive regarding $J$.
    Then $F \restriction \alpha \in A^{< \infty}$ for any ordinal $\alpha$.

    (1) $F(0) = a$. \\
    Proof.
      $F(0)
        = J(F \restriction 0)
        = a$.
    Qed.

    (2) $F(\succ(\alpha)) = G(\alpha, F(\alpha))$ for all ordinals $\alpha$. \\
    Proof.
      Let $\alpha$ be an ordinal.
      Then $F(\succ(\alpha))
        = J(F \restriction \succ(\alpha))
        = G(\pred(\succ(\alpha)), (F \restriction \succ(\alpha))(\pred(\succ(\alpha))))
        = G(\alpha, (F \restriction \succ(\alpha))(\alpha))
        = G(\alpha, F(\alpha))$.
    Qed.

    (3) $F(\lambda) = H(\lambda, F \restriction \lambda)$ for all limit ordinals $\lambda$. \\
    Proof.
      Let $\lambda$ be a limit ordinal.
      Then $F(\lambda)
        = J(F \restriction \lambda)
        = H(\lambda, F \restriction \lambda)$.
    Qed.

    Hence the thesis (by 1, 2, 3).
  \end{proof}
\end{smodule}
\end{document}
