\documentclass[10pt]{article}
\usepackage[english]{babel}
\usepackage{../../../meta-inf/lib/naproche}
% Top-level sections and proofs

\newtheoremstyle{foundations}% name of the style to be used
  {\dimexpr\ftlparskip+4pt\relax}% measure of space to leave above the theorem.
  % (We must add \ftlparskip since otherwise the name of the label would not fit
  % into the border drawn around the environment.)
  {4pt}% measure of space to leave below the theorem.
  {\normalfont}% name of font to use in the body of the theorem
  {0pt}% measure of space to indent
  {\bfseries}% name of head font
  {}% punctuation between head and body
  { }% space after theorem head; " " = normal inter-word space
  {\thmname{#1}\thmnumber{ #2}\thmnote{ (#3)}.}

\let\definition\relax
\let\enddefinition\relax
\let\signature\relax
\let\endsignature\relax
\let\axiom\relax
\let\endaxiom\relax
\let\theorem\relax
\let\endtheorem\relax
\let\proposition\relax
\let\endproposition\relax
\let\lemma\relax
\let\endlemma\relax
\let\corollary\relax
\let\endcorollary\relax

\theoremstyle{foundations}

\newmdtheoremenv[
  backgroundcolor=lightgray!10,
  linecolor=black,
  innertopmargin=1em
]{definition}{Definition}[section]
\newmdtheoremenv[
  backgroundcolor=lightgray!10,
  linecolor=black,
  innertopmargin=1em
]{signature}[definition]{Signature}
\newmdtheoremenv[
  backgroundcolor=lightgray!10,
  linecolor=black,
  innertopmargin=1em
]{axiom}[definition]{Axiom}
\newmdtheoremenv[
  backgroundcolor=lightgray!10,
  linecolor=black,
  innertopmargin=1em
]{theorem}[definition]{Theorem}
\newmdtheoremenv[
  backgroundcolor=lightgray!10,
  linecolor=black,
  innertopmargin=1em
]{proposition}[definition]{Proposition}
\newmdtheoremenv[
  backgroundcolor=lightgray!10,
  linecolor=black,
  innertopmargin=1em
]{lemma}[definition]{Lemma}
\newmdtheoremenv[
  backgroundcolor=lightgray!10,
  linecolor=black,
  innertopmargin=1em
]{corollary}[definition]{Corollary}


% Hide imports

\usepackage{scontents}
\newenvsc{imports}[store-env=imports,print-env=false]


% Labels of top-level sections

\renewcommand{\showlabelfont}{\scriptsize\ttfamily\color{gray}}


% Non-symbolic "\implies" and "\iff"

\renewcommand{\implies}{\quad \text{implies} \quad}
\renewcommand{\iff}{\quad \text{iff} \quad}

\usepackage{amssymb}
\usepackage{pgffor}

\newcommand{\id}{\textrm{id}}
\newcommand{\range}{\textrm{range}}
\newcommand{\const}[2]{\textrm{const}_{#1}^{#2}}
\newcommand{\symdiff}{\mathop{\triangle}}
\newcommand{\pow}{\mathcal{P}}
\newcommand{\onto}{\twoheadrightarrow}
\newcommand{\into}{\hookrightarrow}

% E.g. `\set{1, 2, 3, 4}` prints the numbers 1, 2, 3 and 4 as a comma
% separated list which is enclosed within curly braces.
\def\set#1{\ensuremath{
  \{\foreach[count=\i]\x in {#1}{
    \ifnum\i>1,\,\fi
    \x
  }\}
}}

\renewcommand{\succ}{\textrm{succ}}
\newcommand{\pred}{\textrm{pred}}
\newcommand{\V}{\textbf{V}}
\newcommand{\Ord}{\textbf{Ord}}
\newcommand{\Card}{\textbf{Card}}

\begin{document}
  \begin{imports}
    \begin{forthel}
      %[prove off][check off]
      [readtex \path{libraries/source/set-theory/omega-is-limit-ordinal.ftl.tex}]
      [readtex \path{libraries/source/set-theory/cardinals.ftl.tex}]
      %[prove on][check on]
    \end{forthel}
  \end{imports}


  \section*{Natural Numbers are Cardinal Numbers}

  \begin{forthel}
    \begin{theorem}\printlabel{SET_THEORY_07_2948332552978432}
      Let $n \in \omega$.
      Then $|n| = n$.
    \end{theorem}
    \begin{proof}
      Define $\Phi = \{ n' \in \omega \mid |n'| = n' \}$.

      (1) $0 \in \Phi$.
      Indeed $|0| = |\emptyset| = 0$.

      (2) For all $n' \in \Phi$ we have $\succ(n') \in \Phi$. \\
      Proof.
        Let $n' \in \Phi$.
        Then $|n'| = n'$.
        We have $|\succ(n')| \leq \succ(n')$.

        Let us show that $\succ(n') \leq |\succ(n')|$.
          Assume the contrary.
          Then $|\succ(n')| < \succ(n')$.
          Take a bijection $f$ between $|\succ(n')|$ and $\succ(n')$.
          $|\succ(n')|$ is nonzero.
          Hence we can take a $m \in \omega$ such that $|\succ(n')| = \succ(m)$.
          Indeed $|\succ(n')| \in \succ(\succ(n')) \in \omega$.
          Then $f^{-1}(n') \leq m$.

          We can show that $f^{-1}(n') < m$.
            Assume the contrary.
            Then $f^{-1}(n') = m$.
            $f \restriction m$ is a bijection between $m$ and $f[m]$ (by \printref{FOUNDATIONS_08_647446231252992}).
            Indeed $f$ is an injective map from $|\succ(n')|$ to $\succ(n')$ and
            $m \subseteq |\succ(n')|$.
            We have $f[m] \subseteq n'$ and $n' \subseteq f[m]$.
            Hence $f[m] = n'$.
            Thus $f \restriction m$ is a bijection between $m$ and $n'$.
            Therefore $n'
              = |n'|
              \leq m
              < |\succ(n')|
              < \succ(n')$.
            Consequently $m = n'$.
            Then we have $\succ(n') = |\succ(n')| < \succ(n')$.
            Contradiction.
          End.

          Define \[ g(i) =
            \begin{cases}
              f(i)  & : i \neq f^{-1}(n')
              \\
              f(m)  & : i = f^{-1}(n')
            \end{cases} \]
          for $i \in m$.

          $g$ is a map from $m$ to $n'$.
          Indeed we can show that $g(i) \in n'$ for each $i \in m$. \\
          Proof.
            Let $i \in m$.
            We have $g(i) \in \succ(n')$.
            Indeed $f(i), f(m) \in \succ(n')$ and ($g(i) = f(i)$ or $g(i) = f(m)$).
            Hence if $g(i) \neq n'$ then $g(i) \in n'$.

            Case $i \neq f^{-1}(n')$.
              Then $g(i) = f(i) \in \succ(n')$.
              If $g(i) = n'$ then $f(i) = n' = f(f^{-1}(n'))$.
              Hence if $g(i) = n'$ then $i = f^{-1}(n')$.
              Thus $g(i) \neq n'$.
              Therefore $g(i) \in n'$.
            End.

            Case $i = f^{-1}(n')$.
              Then $g(i)
                = f(m)
                \neq f(f^{-1}(n'))
                = n'$.
              Hence $g(i) \in n'$.
            End.
          Qed.

          $g$ is surjective onto $n'$.
          Indeed we can show that for all $k \in n'$ there exists a $l \in m$ such that $k = g(l)$. \\
          Proof.
            Let $k \in n'$.
            Then $f^{-1}(k) \neq f^{-1}(n')$.

            Case $f^{-1}(k) = m$.
              Then $k
                = f(f^{-1}(k))
                = f(m)
                = g(f^{-1}(n'))$.
            End.

            Case $f^{-1}(k) \neq m$.
              Then $f^{-1}(k) \in m$.
              Indeed $f^{-1}(k) \in |\succ(n')| = \succ(m) = m \cup \set{m}$.
              Hence $k
                = f(f^{-1}(k))
                = g(f^{-1}(k))$.
            End.
          Qed.

          $g$ is injective.
          Indeed we can show that for all $i, j \in m$ if $i \neq j$ then $g(i) \neq g(j)$. \\
          Proof.
            Let $i, j \in m$.
            Assume $i \neq j$.

            Case $i, j \neq f^{-1}(n')$.
              Then $g(i)
                = f(i)
                \neq f(j)
                = g(j)$.
            End.

            Case $i = f^{-1}(n')$.
              Then $j \neq f^{-1}(n')$.
              Hence $g(i)
                = g(f^{-1}(n'))
                = f(m)
                \neq f(j)
                = g(j)$.
              Indeed $m \neq j$.
            End.

            Case $j = f^{-1}(n')$.
              Then $i \neq f^{-1}(n')$.
              Hence $g(i)
                = f(i)
                \neq f(m)
                = g(f^{-1}(n'))
                = g(j)$.
              Indeed $i \neq m$.
            End.
          Qed.
        End.
      End.

      Thus $\omega \subseteq \Phi$ (by \printref{SET_THEORY_03_344585425387520}).
      Consequently $n \in \Phi$.
      Therefore $|n| = n$.
    \end{proof}
  \end{forthel}

  \begin{forthel}
    \begin{corollary}\printlabel{SET_THEORY_07_7061392098066432}
      Every natural number is a cardinal.
    \end{corollary}
  \end{forthel}
\end{document}
