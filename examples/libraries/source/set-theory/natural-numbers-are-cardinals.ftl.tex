\documentclass{naproche-library}
\libinput{set-theory-preamble}
\begin{document}
\begin{smodule}{natural-numbers-are-cardinals.ftl}
  \importmodule[libraries]{set-theory?omega-is-limit-ordinal.ftl}
  \importmodule[libraries]{set-theory?cardinals.ftl}

  \section*{Natural Numbers are Cardinal Numbers}

  \begin{theorem}[forthel,id=SET_THEORY_07_2948332552978432,printid]
    Let $n \in \omega$.
    Then $|n| = n$.
  \end{theorem}
  \begin{proof}[forthel]
    Define $\Phi = \{ n' \in \omega \mid |n'| = n' \}$.

    (1) $0 \in \Phi$.
    Indeed $|0| = |\emptyset| = 0$.

    (2) For all $n' \in \Phi$ we have $\succ(n') \in \Phi$. \\
    Proof.
      Let $n' \in \Phi$.
      Then $|n'| = n'$.
      We have $|\succ(n')| \leq \succ(n')$.

      Let us show that $\succ(n') \leq |\succ(n')|$.
        Assume the contrary.
        Then $|\succ(n')| < \succ(n')$.
        Take a bijection $f$ between $|\succ(n')|$ and $\succ(n')$.
        $|\succ(n')|$ is nonzero.
        Hence we can take a $m \in \omega$ such that $|\succ(n')| = \succ(m)$.
        Indeed $|\succ(n')| \in \succ(\succ(n')) \in \omega$.
        Then $f^{-1}(n') \leq m$.

        We can show that $f^{-1}(n') < m$.
          Assume the contrary.
          Then $f^{-1}(n') = m$.
          $f \restriction m$ is a bijection between $m$ and $f[m]$ (by \printref{FOUNDATIONS_08_647446231252992}).
          Indeed $f$ is an injective map from $|\succ(n')|$ to $\succ(n')$ and
          $m \subseteq |\succ(n')|$.
          We have $f[m] \subseteq n'$ and $n' \subseteq f[m]$.
          Hence $f[m] = n'$.
          Thus $f \restriction m$ is a bijection between $m$ and $n'$.
          Therefore $n'
            = |n'|
            \leq m
            < |\succ(n')|
            < \succ(n')$.
          Consequently $m = n'$.
          Then we have $\succ(n') = |\succ(n')| < \succ(n')$.
          Contradiction.
        End.

        Define \[ g(i) =
          \begin{cases}
            f(i)  & : i \neq f^{-1}(n')
            \\
            f(m)  & : i = f^{-1}(n')
          \end{cases} \]
        for $i \in m$.

        $g$ is a map from $m$ to $n'$.
        Indeed we can show that $g(i) \in n'$ for each $i \in m$. \\
        Proof.
          Let $i \in m$.
          We have $g(i) \in \succ(n')$.
          Indeed $f(i), f(m) \in \succ(n')$ and ($g(i) = f(i)$ or $g(i) = f(m)$).
          Hence if $g(i) \neq n'$ then $g(i) \in n'$.

          Case $i \neq f^{-1}(n')$.
            Then $g(i) = f(i) \in \succ(n')$.
            If $g(i) = n'$ then $f(i) = n' = f(f^{-1}(n'))$.
            Hence if $g(i) = n'$ then $i = f^{-1}(n')$.
            Thus $g(i) \neq n'$.
            Therefore $g(i) \in n'$.
          End.

          Case $i = f^{-1}(n')$.
            Then $g(i)
              = f(m)
              \neq f(f^{-1}(n'))
              = n'$.
            Hence $g(i) \in n'$.
          End.
        Qed.

        $g$ is surjective onto $n'$.
        Indeed we can show that for all $k \in n'$ there exists a $l \in m$ such that $k = g(l)$. \\
        Proof.
          Let $k \in n'$.
          Then $f^{-1}(k) \neq f^{-1}(n')$.

          Case $f^{-1}(k) = m$.
            Then $k
              = f(f^{-1}(k))
              = f(m)
              = g(f^{-1}(n'))$.
          End.

          Case $f^{-1}(k) \neq m$.
            Then $f^{-1}(k) \in m$.
            Indeed $f^{-1}(k) \in |\succ(n')| = \succ(m) = m \cup \set{m}$.
            Hence $k
              = f(f^{-1}(k))
              = g(f^{-1}(k))$.
          End.
        Qed.

        $g$ is injective.
        Indeed we can show that for all $i, j \in m$ if $i \neq j$ then $g(i) \neq g(j)$. \\
        Proof.
          Let $i, j \in m$.
          Assume $i \neq j$.

          Case $i, j \neq f^{-1}(n')$.
            Then $g(i)
              = f(i)
              \neq f(j)
              = g(j)$.
          End.

          Case $i = f^{-1}(n')$.
            Then $j \neq f^{-1}(n')$.
            Hence $g(i)
              = g(f^{-1}(n'))
              = f(m)
              \neq f(j)
              = g(j)$.
            Indeed $m \neq j$.
          End.

          Case $j = f^{-1}(n')$.
            Then $i \neq f^{-1}(n')$.
            Hence $g(i)
              = f(i)
              \neq f(m)
              = g(f^{-1}(n'))
              = g(j)$.
            Indeed $i \neq m$.
          End.
        Qed.
      End.
    End.

    Thus $\omega \subseteq \Phi$ (by \printref{SET_THEORY_03_344585425387520}).
    Consequently $n \in \Phi$.
    Therefore $|n| = n$.
  \end{proof}

  \begin{corollary}[forthel,id=SET_THEORY_07_7061392098066432,printid]
    Every natural number is a cardinal.
  \end{corollary}
\end{smodule}
\end{document}
