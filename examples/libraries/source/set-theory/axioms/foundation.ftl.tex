\documentclass{naproche-library}
\libinput{set-theory-preamble}
\begin{document}
\begin{smodule}{foundation.ftl}
  \importmodule[libraries]{foundations?systems-of-sets.ftl}

  \section{Foundation}

  \begin{axiom}[forthel,title=Foundation,id=FOUNDATIONS_10_1320008569323520,printid]
    Let $X$ be a nonempty system of sets.
    Then $X$ has an element $x$ such that $X$ and $x$ are disjoint.
  \end{axiom}
  
  \begin{corollary}[forthel,id=FOUNDATIONS_10_9532145874510315,printid]
    Let $X$ be a nonempty system of sets.
    Then $X$ has an element $x$ such that for no $y \in X$ we have $y \in x$.
  \end{corollary}
  
  \begin{proposition}[forthel,id=FOUNDATIONS_11_2812087589928960,printid]
    Let $\Phi$ be a class.
    Assume that for all sets $x$ if $\Phi$ contains every element of $x$ that is a set then $\Phi$ contains $x$.
    Then $\Phi$ contains every set.
  \end{proposition}
  \begin{proof}[forthel]
    Assume the contrary.
    Define $M = \{ x \mid x$ is a set such that $x \notin \Phi \}$.
    Then $M$ is nonempty.
    Hence we can take a $x \in M$ such that for no $y \in M$ we have $y \in x$.
    Then $x$ is a set such that every element of $x$ that is a set is contained in $\Phi$.
    Thus $\Phi$ contains $x$.
    Contradiction.
  \end{proof}
\end{smodule}
\end{document}
