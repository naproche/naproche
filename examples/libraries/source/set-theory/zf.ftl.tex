\documentclass{naproche-library}
\libinput{set-theory-preamble}
\begin{document}
\begin{smodule}{zf.ftl}
  \importmodule[libraries]{set-theory/axioms?set-existence.ftl}
  \importmodule[libraries]{set-theory/axioms?separation.ftl}
  \importmodule[libraries]{set-theory/axioms?pairing.ftl}
  \importmodule[libraries]{set-theory/axioms?union.ftl}
  \importmodule[libraries]{set-theory/axioms?powerset.ftl}
  \importmodule[libraries]{set-theory/axioms?infinity.ftl}
  \importmodule[libraries]{set-theory/axioms?replacement.ftl}
  \importmodule[libraries]{set-theory/axioms?foundation.ftl}
  \importmodule[libraries]{set-theory/axioms?function.ftl}
  \importmodule[libraries]{foundations?pairs-and-products.ftl}
  \importmodule[libraries]{foundations?invertible-maps.ftl}

  \section*{ZFC}

  \begin{proposition}[forthel,id=FOUNDATIONS_10_5891530432708608]
    $\emptyset$ is a set.
  \end{proposition}
  \begin{proof}[forthel]
    Take a set $x$ (by \printref{FOUNDATIONS_10_2362039748001792}).
    Define $A = \{ y \in x \mid y \neq y \}$.
    Then $A$ is a set (by \printref{FOUNDATIONS_10_2263707272871936}).
    We have $A = \emptyset$.
    Hence $\emptyset$ is a set.
  \end{proof}

  \begin{proposition}[forthel,id=FOUNDATIONS_10_4052198354845696]
    Let $x, y$ be sets.
    Then $x \cup y$ is a set.
  \end{proposition}
  \begin{proof}[forthel]
    Take $X = \set{x,y}$.
    Then $X$ is a set.
    Hence $\bigcup X$ is a set (by \printref{FOUNDATIONS_10_5536459412996096}).
    Indeed $X$ is a system of sets.
    We have $x \cup y = \bigcup X$.
    Thus $x \cup y$ is a set.
  \end{proof}

  \begin{proposition}[forthel,id=FOUNDATIONS_10_4475839687163904]
    Let $x, y$ be sets.
    Then $x \cap y$ is a set.
  \end{proposition}
  \begin{proof}[forthel]
    We have $x \cap y \subseteq x$.
    Hence $x \cap y$ is a set (by \printref{FOUNDATIONS_10_2263707272871936}).
  \end{proof}

  \begin{proposition}[forthel,id=FOUNDATIONS_10_7795203882614784]
    Let $x, y$ be sets.
    Then $x \setminus y$ is a set.
  \end{proposition}
  \begin{proof}[forthel]
    We have $x \setminus y \subseteq x$.
    Hence $x \setminus y$ is a set (by \printref{FOUNDATIONS_10_2263707272871936}).
  \end{proof}

  \begin{proposition}[forthel,id=FOUNDATIONS_10_4458706448154624]
    Let $x, y$ be sets.
    Then $x \times y$ is a set.
  \end{proposition}
  \begin{proof}[forthel]
    $\set{a}$ and $\set{a,b}$ are sets for each $a \in x$ and each $b \in y$.
    Define $P = \{ \set{\set{a}, \set{a,b}} \mid a \in x$ and $b \in y \}$.

    (1) $P$ is a set. \\
    Proof.
      Let us show that $P \subseteq \pow(\pow(x \cup y))$.
        Let $p \in P$.
        Consider $a \in x$ and $b \in y$ such that $p = \set{\set{a},\set{a,b}}$.
        Then $a, b \in x \cup y$.
        Hence $\set{a}, \set{a,b} \in \pow(x \cup y)$.
        Thus $\set{\set{a}, \set{a,b}} \in \pow(\pow(x \cup y))$.
      End.

      $x \cup y$ is a set.
      Consequently $\pow(\pow(x \cup y))$ is a set (by \printref{FOUNDATIONS_10_5862230203564032}).
      Therefore $P$ is a set (by \printref{FOUNDATIONS_10_2263707272871936}).
    Qed.

    Define $l(p) =$ ``choose $a \in x$, choose $b \in y$ such that $p = \set{\set{a}, \set{a,b}}$ in $a$'' for $p \in P$.
    Define $r(p) =$ ``choose $a \in x$, choose $b \in y$ such that $p = \set{\set{a}, \set{a,b}}$ in $b$'' for $p \in P$.

    Define $f(p) = (l(p), r(p))$ for $p \in P$.

    Let us show that for any objects $u, u', v, v'$ if
    $\set{\set{u}, \set{u,v}} = \set{\set{u'}, \set{u',v'}}$ then $u = u'$ and $v = v'$.
      Let $u, u', v, v'$ be objects.
      Assume $\set{\set{u}, \set{u,v}} = \set{\set{u'}, \set{u',v'}}$.
      Then ($\set{u} = \set{u'}$ or $\set{u} = \set{u',v'}$) and ($\set{u,v} = \set{u'}$ or $\set{u,v} = \set{u',v'}$).
      Thus ($\set{u} = \set{u'}$ and ($\set{u,v} = \set{u'}$ or $\set{u,v} = \set{u',v'}$)) or ($\set{u} = \set{u',v'}$ and ($\set{u,v} = \set{u'}$ or $\set{u,v} = \set{u',v'}$)).

      Case $\set{u} = \set{u'}$ and ($\set{u,v} = \set{u'}$ or $\set{u,v} = \set{u',v'}$).
        We have $\set{u} = \set{u'}$.
        Hence $u = u'$.

        Case $\set{u,v} = \set{u'}$.
          Then $u = u'= v$.
          Hence $\set{\set{u}, \set{u,u}} = \set{\set{u}, \set{u,v'}}$ (by 1).
          Thus $\set{\set{u}} = \set{\set{u}, \set{u,v'}}$.
          Therefore $\set{u} = \set{u,v'}$.
          Consequently $v' = u = v$.
        End.

        Case $\set{u,v} = \set{u',v'}$.
          Then $\set{u,v} = \set{u,v'}$.
          Hence $v = v'$.
        End.
      End.

      Case $\set{u} = \set{u',v'}$ and ($\set{u,v} = \set{u'}$ or $\set{u,v} = \set{u',v'}$).
        We have $\set{u} = \set{u',v'}$.
        Hence $u = u'$.

        Case $\set{u,v} = \set{u'}$.
          Then $u = v = u'$.
          Hence $v = v'$.
        End.

        Case $\set{u,v} = \set{u',v'}$.
          Then $\set{u,v} = \set{u,v'}$.
          Hence $v = v'$.
        End.
      End.
    End.

    $\set{\set{a}, \set{a,b}} \in \dom(f)$ for any $a \in x$ and any $b \in y$. \\ % Needed for ontological checking
    Proof.
      Let $a \in x$ and $b \in y$.
      Then $\set{\set{a}, \set{a,b}} \in P$.
    Qed.

    Let us show that for any $a \in x$ and any $b \in y$ we have $f(\set{\set{a}, \set{a,b}}) = (a,b)$.
      Let $a \in x$ and $b \in y$.
      Take $p = \set{\set{a}, \set{a,b}}$.
      Then $p$ is a set.
      Then we can choose $a' \in x$ and $b' \in y$ such that $p = \set{\set{a'}, \set{a',b'}}$ and $l(p) = a'$.
      Then $a = a'$ and $b = b'$.
      Hence $l(p) = a$.
      Choose $a'' \in x$ and $b'' \in y$ such that $p = \set{\set{a''}, \set{a'',b''}}$ and $r(p) = b''$.
      Then $a = a''$ and $b = b''$.
      Thus $r(p) = b$.
      Therefore $f(p) = (a,b)$.
    End.

    (2) $x \times y = f[P]$. \\
    Proof.
      For all $p \in P$ we have $l(p) \in x$ and $r(p) \in y$.
      Hence $f(p) \in x \times y$ for all $p \in P$.
      Therefore $f[P] \subseteq x \times y$.

      Let us show that $x \times y \subseteq f[P]$.
        Let $z \in x \times y$.
        Take $a \in x$ and $b \in y$ such that $z = (a,b)$.
        Then $(a,b) = f(\set{\set{a}, \set{a,b}})$.
        Hence there exists a $p \in P$ such that $(a,b) = f(p)$.
        Thus $(a,b) \in f[P]$.
      End.

      Consequently $x \times y = f[P]$.
    Qed.

    Thus $x \times y$ is the image of some set under some map.
    Therefore $x \times y$ is a set (by \printref{FOUNDATIONS_10_8142956584239104}).
  \end{proof}

  \begin{proposition}[forthel,id=FOUNDATIONS_10_5486815207227392]
    Let $X$ be a nonempty system of sets.
    Then $\bigcap X$ is a set.
  \end{proposition}
  \begin{proof}[forthel]
    Take an element $x$ of $X$.
    Then $\bigcap X \subseteq x$.
    Hence $\bigcap X$ is a set (by \printref{FOUNDATIONS_10_2263707272871936}).
  \end{proof}

  \begin{proposition}[forthel,id=FOUNDATIONS_10_7598384349184000]
    Let $f$ be a map such that $\dom(f)$ is a set.
    Then $\range(f)$ is a set.
  \end{proposition}
  \begin{proof}[forthel]
    $\range(f) = f_{*}(\dom(f))$ and $f_{*}(\dom(f))$ is a set.
    Hence $\range(f)$ is a set (by \printref{FOUNDATIONS_10_8142956584239104}).
  \end{proof}

  \begin{proposition}[forthel,id=FOUNDATIONS_10_8631339572002816]
    Let $A$ be a class and $x$ be a set.
    Assume that there exists an injective map from $A$ to $x$.
    Then $A$ is a set.
  \end{proposition}
  \begin{proof}[forthel]
    Consider an injective map $f$ from $A$ to $x$.
    Then $f^{-1}$ is a bijection between $\range(f)$ and $A$.
    $\range(f)$ is a set and $A$ is the image of $\range(f)$ under $f^{-1}$.
    Indeed $\range(f) \subseteq x$.
    Thus $A$ is a set (by \printref{FOUNDATIONS_10_8142956584239104}).
  \end{proof}

  \begin{proposition}[forthel,id=FOUNDATIONS_10_8812282138066944]
    There exist no sets $x, y$ such that $x \in y$ and $y \in x$.
  \end{proposition}
  \begin{proof}[forthel]
    Assume the contrary.
    Take sets $x,y$ such that $x \in y$ and $y \in x$.
    Consider an element $z$ of $\set{x,y}$ such that $\set{x,y}$ and $z$ are disjoint (by \printref{FOUNDATIONS_10_1320008569323520}).
    Indeed $\set{x,y}$ is a nonempty system of sets.
    Then we have $z = x$ or $z = y$.

    Case $z = x$.
      Then $x$ and $\set{x,y}$ are disjoint.
      Hence $y \notin x$.
      Contradiction.
    End.

    Case $z = y$.
      Then $y$ and $\set{x,y}$ are disjoint.
      Hence $x \notin y$.
      Contradiction.
    End.
  \end{proof}

  \begin{corollary}[forthel,id=FOUNDATIONS_10_3086917813927936]
    Let $x$ be a set.
    Then $x \notin x$.
  \end{corollary}

  \begin{proposition}[forthel,id=FOUNDATIONS_10_4589652321021547]
    Let $f, g$ be functions.
    Assume that $\dom(f) = \dom(g)$ and $f(a) = g(a)$ for all $a \in \dom(f)$.
    Then $f = g$.
  \end{proposition}

  \begin{proposition}[forthel,id=FOUNDATIONS_10_4105036244189184]
    Let $x, y$ be sets.
    Then $[x \to y]$ is a set.
  \end{proposition}
  \begin{proof}[forthel]
    Define $R = \{ F \in \pow(x \times y) \mid$ (for all $a \in x$ there exists a $b \in y$ such that $(a,b) \in F$) and for all $a \in x$ and all $b, b' \in y$ such that $(a,b), (a,b') \in F$ we have $b = b' \}$.

    [prover vampire]
    Every element of $R$ is a set. % Needed for ontological checking
    Define $h(F) = \fun a \in x.$ ``choose $b \in y$ such that $(a,b) \in F$ in $b$'' for $F \in R$.
    [prover eprover]

    Let us show that $[x \to y] \subseteq \range(h)$.
      Let $f \in [x \to y]$.
      Define $F = \{ (a,f(a)) \mid a \in x \}$.

      Then $F \in R$. \\
      Proof.
        Define $g(a) = (a,f(a))$ for $a \in x$.
        Then $F = \range(g)$.
        Hence $F$ is a set.
        Thus $F \in \pow(x \times y)$.
        Indeed $F \subseteq x \times y$.

        (1) For all $a \in x$ there exists a $b \in y$ such that $(a,b) \in F$.

        (2) For all $a \in x$ and all $b, b' \in y$ such that $(a,b),
        (a,b') \in F$ we have $b = b'$.

        [prover vampire]
        Hence the thesis.
        [prover eprover]
      End.

      $\dom(f) = \dom(h(F))$ and for each $a \in \dom(f)$ we have $h(F)(a) = f(a)$.
      Hence $f = h(F)$.
      Thus $f \in \range(h)$.
    End.

    Therefore $[x \to y]$ is a set.
    Indeed $R$ is a set.
  \end{proof}
\end{smodule}
\end{document}
