\documentclass[10pt]{article}
\usepackage[english]{babel}
\usepackage{../../../../lib/tex/naproche}
% Top-level sections and proofs

\newtheoremstyle{foundations}% name of the style to be used
  {\dimexpr\ftlparskip+4pt\relax}% measure of space to leave above the theorem.
  % (We must add \ftlparskip since otherwise the name of the label would not fit
  % into the border drawn around the environment.)
  {4pt}% measure of space to leave below the theorem.
  {\normalfont}% name of font to use in the body of the theorem
  {0pt}% measure of space to indent
  {\bfseries}% name of head font
  {}% punctuation between head and body
  { }% space after theorem head; " " = normal inter-word space
  {\thmname{#1}\thmnumber{ #2}\thmnote{ (#3)}.}

\let\definition\relax
\let\enddefinition\relax
\let\signature\relax
\let\endsignature\relax
\let\axiom\relax
\let\endaxiom\relax
\let\theorem\relax
\let\endtheorem\relax
\let\proposition\relax
\let\endproposition\relax
\let\lemma\relax
\let\endlemma\relax
\let\corollary\relax
\let\endcorollary\relax

\theoremstyle{foundations}

\newmdtheoremenv[
  backgroundcolor=lightgray!10,
  linecolor=black,
  innertopmargin=1em
]{definition}{Definition}[section]
\newmdtheoremenv[
  backgroundcolor=lightgray!10,
  linecolor=black,
  innertopmargin=1em
]{signature}[definition]{Signature}
\newmdtheoremenv[
  backgroundcolor=lightgray!10,
  linecolor=black,
  innertopmargin=1em
]{axiom}[definition]{Axiom}
\newmdtheoremenv[
  backgroundcolor=lightgray!10,
  linecolor=black,
  innertopmargin=1em
]{theorem}[definition]{Theorem}
\newmdtheoremenv[
  backgroundcolor=lightgray!10,
  linecolor=black,
  innertopmargin=1em
]{proposition}[definition]{Proposition}
\newmdtheoremenv[
  backgroundcolor=lightgray!10,
  linecolor=black,
  innertopmargin=1em
]{lemma}[definition]{Lemma}
\newmdtheoremenv[
  backgroundcolor=lightgray!10,
  linecolor=black,
  innertopmargin=1em
]{corollary}[definition]{Corollary}


% Hide imports

\usepackage{scontents}
\newenvsc{imports}[store-env=imports,print-env=false]


% Labels of top-level sections

\renewcommand{\showlabelfont}{\scriptsize\ttfamily\color{gray}}


% Non-symbolic "\implies" and "\iff"

\renewcommand{\implies}{\quad \text{implies} \quad}
\renewcommand{\iff}{\quad \text{iff} \quad}

\usepackage{amssymb}
\usepackage{pgffor}

\newcommand{\id}{\textrm{id}}
\newcommand{\range}{\textrm{range}}
\newcommand{\const}[2]{\textrm{const}_{#1}^{#2}}
\newcommand{\symdiff}{\mathop{\triangle}}
\newcommand{\pow}{\mathcal{P}}
\newcommand{\onto}{\twoheadrightarrow}
\newcommand{\into}{\hookrightarrow}

% E.g. `\set{1, 2, 3, 4}` prints the numbers 1, 2, 3 and 4 as a comma
% separated list which is enclosed within curly braces.
\def\set#1{\ensuremath{
  \{\foreach[count=\i]\x in {#1}{
    \ifnum\i>1,\,\fi
    \x
  }\}
}}

\renewcommand{\succ}{\textrm{succ}}
\newcommand{\pred}{\textrm{pred}}
\newcommand{\V}{\textbf{V}}
\newcommand{\Ord}{\textbf{Ord}}
\newcommand{\Card}{\textbf{Card}}

\begin{document}
  \begin{imports}
    \begin{forthel}
      %[prove off][check off]
      [readtex \path{libraries/source/set-theory/natural-numbers-are-cardinals.ftl.tex}]
      %[prove on][check on]
    \end{forthel}
  \end{imports}


  \section*{$\omega$ is a Cardinal Number}

  \begin{forthel}
    \begin{proposition}\printlabel{SET_THEORY_07_4952029518626816}
      $|\omega| = \omega$.
    \end{proposition}
    \begin{proof}
      We have $|\omega| \leq \omega$.

      Let us show that $|\omega|$ is not less than $\omega$.
        Assume the contrary.
        Then $|\omega| \in \omega$.
        Take $n = |\omega|$ and a bijection $f$ between $n$ and $\omega$.

        Define \[ g(k) =
          \begin{cases}
            \succ(f(k)) & : k < n
            \\
            0           & : k = n
          \end{cases} \]
        for $k \in \succ(n)$.
        Then $g$ is a map from $\succ(n)$ to $\omega$.
        Indeed we can show that $g(k) \in \omega$ for all $k \in \succ(n)$.
          Let $k \in \succ(n)$.
          Case $k < n$. Obvious.
          Case $k = n$. Obvious.
        End.

        $g$ is injective.
        Indeed we can show that for all $k, k' \in \succ(n)$ if $k \neq k'$
        then $g(k) \neq g(k')$. \\
        Proof.
          Let $k, k' \in \succ(n)$.
          Assume $k \neq k'$.

          Case $k, k' < n$.
            Then $f(k) \neq f(k')$.
            Hence $\succ(f(k)) \neq \succ(f(k'))$.
            Thus $g(k) \neq g(k')$.
          End.

          Case $k < n$ and $k' = n$.
            We have $\succ(f(k)) \neq 0$.
            Hence $g(k) \neq g(k')$.
          End.

          Case $k = n$ and $k' < n$.
            We have $\succ(f(k')) \neq 0$.
            Hence $g(k) \neq g(k')$.
          End.
        Qed.

        $g$ is surjective onto $\omega$.
        Indeed we can show that for any $m \in \omega$ there exists a $k \in \succ(n)$ such that $m = g(k)$. \\
        Proof.
          Let $m \in \omega$.
          Then $f^{-1}(m) \in n$.

          Case $m = 0$.
            Then $m = g(n)$.
          End.

          Case $m \neq 0$.
            Take $m' \in \omega$ such that $m = \succ(m')$.
            Then $m
              = \succ(m')
              = \succ(f(f^{-1}(m')))
              = g(f^{-1}(m'))$.
            Indeed $f(f^{-1}(m')) = m'$ and $f^{-1}(m') < n$.
          End.
        End.

        Hence $g$ is a bijection between $\succ(n)$ and $\omega$.
        Then we have $n
          = |n|
          = |\succ(n)|
          = \succ(n)$.
        Contradiction.
      End.
    \end{proof}
  \end{forthel}

  \begin{forthel}
    \begin{corollary}\printlabel{SET_THEORY_07_2717623053713408}
      $\omega$ is a cardinal.
    \end{corollary}
  \end{forthel}
\end{document}
