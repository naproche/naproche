\documentclass[10pt]{article}
\usepackage[english]{babel}
\usepackage{../../../meta-inf/lib/naproche}
% Top-level sections and proofs

\newtheoremstyle{foundations}% name of the style to be used
  {\dimexpr\ftlparskip+4pt\relax}% measure of space to leave above the theorem.
  % (We must add \ftlparskip since otherwise the name of the label would not fit
  % into the border drawn around the environment.)
  {4pt}% measure of space to leave below the theorem.
  {\normalfont}% name of font to use in the body of the theorem
  {0pt}% measure of space to indent
  {\bfseries}% name of head font
  {}% punctuation between head and body
  { }% space after theorem head; " " = normal inter-word space
  {\thmname{#1}\thmnumber{ #2}\thmnote{ (#3)}.}

\let\definition\relax
\let\enddefinition\relax
\let\signature\relax
\let\endsignature\relax
\let\axiom\relax
\let\endaxiom\relax
\let\theorem\relax
\let\endtheorem\relax
\let\proposition\relax
\let\endproposition\relax
\let\lemma\relax
\let\endlemma\relax
\let\corollary\relax
\let\endcorollary\relax

\theoremstyle{foundations}

\newmdtheoremenv[
  backgroundcolor=lightgray!10,
  linecolor=black,
  innertopmargin=1em
]{definition}{Definition}[section]
\newmdtheoremenv[
  backgroundcolor=lightgray!10,
  linecolor=black,
  innertopmargin=1em
]{signature}[definition]{Signature}
\newmdtheoremenv[
  backgroundcolor=lightgray!10,
  linecolor=black,
  innertopmargin=1em
]{axiom}[definition]{Axiom}
\newmdtheoremenv[
  backgroundcolor=lightgray!10,
  linecolor=black,
  innertopmargin=1em
]{theorem}[definition]{Theorem}
\newmdtheoremenv[
  backgroundcolor=lightgray!10,
  linecolor=black,
  innertopmargin=1em
]{proposition}[definition]{Proposition}
\newmdtheoremenv[
  backgroundcolor=lightgray!10,
  linecolor=black,
  innertopmargin=1em
]{lemma}[definition]{Lemma}
\newmdtheoremenv[
  backgroundcolor=lightgray!10,
  linecolor=black,
  innertopmargin=1em
]{corollary}[definition]{Corollary}


% Hide imports

\usepackage{scontents}
\newenvsc{imports}[store-env=imports,print-env=false]


% Labels of top-level sections

\renewcommand{\showlabelfont}{\scriptsize\ttfamily\color{gray}}


% Non-symbolic "\implies" and "\iff"

\renewcommand{\implies}{\quad \text{implies} \quad}
\renewcommand{\iff}{\quad \text{iff} \quad}

\usepackage{amssymb}
\usepackage{pgffor}

\newcommand{\id}{\textrm{id}}
\newcommand{\range}{\textrm{range}}
\newcommand{\const}[2]{\textrm{const}_{#1}^{#2}}
\newcommand{\symdiff}{\mathop{\triangle}}
\newcommand{\pow}{\mathcal{P}}
\newcommand{\onto}{\twoheadrightarrow}
\newcommand{\into}{\hookrightarrow}

% E.g. `\set{1, 2, 3, 4}` prints the numbers 1, 2, 3 and 4 as a comma
% separated list which is enclosed within curly braces.
\def\set#1{\ensuremath{
  \{\foreach[count=\i]\x in {#1}{
    \ifnum\i>1,\,\fi
    \x
  }\}
}}

\renewcommand{\succ}{\textrm{succ}}
\newcommand{\pred}{\textrm{pred}}
\newcommand{\V}{\textbf{V}}
\newcommand{\Ord}{\textbf{Ord}}
\newcommand{\Card}{\textbf{Card}}

\begin{document}
  \begin{imports}
    \begin{forthel}
      %[prove off][check off]
      [readtex \path{libraries/source/set-theory/zero.ftl.tex}]
      [readtex \path{libraries/source/set-theory/successor-ordinals.ftl.tex}]
      %[prove on][check on]
    \end{forthel}
  \end{imports}


  \section*{Natural Numbers}

  \begin{forthel}
    \begin{definition}[id=SET_THEORY_03_4310076227584000,printid]
      \[ \omega = \class{n \in \Ord | \classtext{$n \in X$ for every $X \subseteq \Ord$ such that $0 \in X$ and for all $x \in X$ we have $\succ(x) \in X$}}. \]
    \end{definition}

    Let a natural number stand for an element of $\omega$.
  \end{forthel}

  \begin{forthel}
    \begin{proposition}[id=SET_THEORY_03_3576717620805632,printid]
      $0 \in \omega$.
    \end{proposition}
  \end{forthel}

  \begin{forthel}
    \begin{proposition}[id=SET_THEORY_03_8807317141192704,printid]
      Let $n \in \omega$.
      Then $\succ(n) \in \omega$.
    \end{proposition}
  \end{forthel}

  \begin{forthel}
    \begin{proposition}[id=SET_THEORY_03_344585425387520,printid]
      Let $\Phi \subseteq \omega$.
      Assume that $0 \in \Phi$ and for every $x \in \Phi$ we have
      $\succ(x) \in \Phi$.
      Then $\Phi = \omega$.
    \end{proposition}
    \begin{proof}
      Suppose $\Phi \neq \omega$.
      Consider an element $n$ of $\omega$ that is not contained in $\Phi$.
      Take $\Phi' = \Phi \setminus \set{n}$.

      (1) $0 \in \Phi'$.
      Indeed $0 \in \Phi$ and $0 \neq n$.

      (2) For each $x \in \Phi'$ we have $\succ(x) \in \Phi'$. \\
      Proof.
        Let $x \in \Phi'$.
        Then $\succ(x) \in \Phi$.

        Let us show that $\succ(x) \neq n$.
          Assume $\succ(x) = n$.
          Then $x \notin \Phi$.
          Indeed $n \notin \Phi$ and if $x \in \Phi$ then
          $n = \succ(x) \in \Phi$.
          Contradiction.
        End.

        Thus $\succ(x) \in \Phi'$.
      Qed.

      Therefore every element of $\omega$ lies in $\Phi'$.
      Indeed $\Phi' \subseteq \Ord$.
      Consequently $n \in \Phi'$.
      Contradiction.
    \end{proof}
  \end{forthel}

  \begin{forthel}
    \begin{corollary}[id=SET_THEORY_03_4847727433220096,printid]
      $\omega$ is a set.
    \end{corollary}
    \begin{proof}
      Define $f(n) = \succ(n)$ for $n \in \omega$.
      Take a subset $X$ of $\omega$ that is inductive regarding $0$ and $f$.
      Indeed $f$ is a map from $\omega$ to $\omega$.
      Then we have $0 \in X$ and for each $n \in X$ we have $\succ(n) \in X$.
      Thus $X = \omega$.
      Therefore $\omega$ is a set.
    \end{proof}
  \end{forthel}

  \begin{forthel}
    \begin{proposition}[id=SET_THEORY_03_5885789275684864,printid]
      Let $n \in \omega$.
      Then $n = 0$ or $n = \succ(m)$ for some $m \in \omega$.
    \end{proposition}
    \begin{proof}
      Assume the contrary.
      Consider a $k \in \omega$ such that neither $k = 0$ nor $k = \succ(m)$ for
      some $m \in \omega$.
      Take a class $\omega'$ such that $\omega' = \omega \setminus \set{k}$. %!
      Then $\omega'$ is a set.

      (1) $0 \in \omega'$.
      Indeed $k \neq 0$.

      (2) For all $m \in \omega'$ we have $\succ(m) \in \omega'$. \\
      Proof.
        Let $m \in \omega'$.
        Then $\succ(m) \neq k$.
        Hence $\succ(m) \in \omega'$.
      Qed.

      Thus every element of $\omega$ is contained in $\omega'$.
      Therefore $k \in \omega'$.
      Contradiction.
    \end{proof}
  \end{forthel}

  \begin{forthel}
    \begin{proposition}[id=SET_THEORY_03_5057540872208384,printid]
      Every element of $\omega$ is an ordinal.
    \end{proposition}
  \end{forthel}
\end{document}
