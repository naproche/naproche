\documentclass{naproche-library}
\libinput{set-theory-preamble}
\begin{document}
\begin{smodule}{omega-is-least-limit-ordinal.ftl}
  \importmodule[libraries]{set-theory?omega-is-limit-ordinal.ftl}
  \importmodule[libraries]{set-theory?ordering.ftl}

  \section*{$\omega$ is the Least Limit Ordinal}

  \begin{proposition}[forthel,id=SET_THEORY_03_5517271459954688]
    Let $\lambda$ be a limit ordinal.
    Then \[ \omega \leq \lambda. \]
  \end{proposition}
  \begin{proof}[forthel]
    Assume the contrary.
    Then $\lambda < \omega$.
    Consequently $\lambda \in \omega$.
    Hence $\lambda = 0$ or $\lambda = \succ(n)$ for some $n \in \omega$.
    Thus $\lambda$ is not a limit ordinal.
    Contradiction.
  \end{proof}
\end{smodule}
\end{document}
