\documentclass{naproche-library}
\libinput{set-theory-preamble}
\begin{document}
\begin{smodule}{transfinite-induction.ftl}
  \importmodule[libraries]{set-theory?ordering.ftl}

  \section*{Transfinite Induction}

  \begin{theorem}[forthel,id=SET_THEORY_02_8493935460614144,printid]
    Let $\Phi$ be a class.
    Assume that for all ordinals $\alpha$ if $\Phi$ contains all $\beta \in \alpha$ then $\Phi$ contains $\alpha$.
    Then $\Phi$ contains every ordinal.
  \end{theorem}
  \begin{proof}[forthel]
    Define $B = \{ x \mid x$ is a set and if $x \in \Ord$ then $x \in \Phi \}$.

    Let us show that for all sets $x$ if $B$ contains every element of $x$ that is a set then $B$ contains $x$.
      Let $x$ be a set.
      Assume that every element of $x$ that is a set is contained in $B$.

      Case $x \notin \Ord$. Trivial.

      Case $x \in \Ord$.
        Then $\Phi$ contains all ordinals less than $x$.
        Hence $\Phi$ contains $x$.
        Thus $x \in B$.
      End.
    End.

    [prover vampire]
    Hence $B$ contains every set (by \printref{FOUNDATIONS_11_2812087589928960}).
    [prover eprover]
    Thus $\Phi$ contains every ordinal.
  \end{proof}

  \begin{theorem}[forthel,id=SET_THEORY_02_7892040431960064,printid]
    Let $\Phi$ be a class.
    
    (Initial case) Assume that $\Phi$ contains $0$.
    
    (Successor step) Assume that for all ordinals $\alpha$ if $\alpha \in \Phi$ then $\succ(\alpha) \in \Phi$.
    
    (Limit step) Assume that for all limit ordinals $\lambda$ if every $\alpha \in \lambda$ is contained in $\Phi$ then $\lambda \in \Phi$.
    
    Then $\Phi$ contains every ordinal.
  \end{theorem}
  \begin{proof}[forthel]
    Let us show that for all ordinals $\alpha$ if $\Phi$ contains all ordinals
    less than $\alpha$ then $\Phi$ contains $\alpha$.
      Let $\alpha$ be an ordinal.
      Then $\alpha = 0$ or $\alpha$ is a successor ordinal or $\alpha$ is a
      limit ordinal.
      Assume that $\Phi$ contains all $\beta \in \alpha$.

      Case $\alpha = 0$. Trivial.

      Case $\alpha$ is a successor ordinal.
        Take an ordinal $\beta$ such that $\alpha = \succ(\beta)$.
        Then $\beta \in \Phi$.
        Hence $\alpha \in \Phi$ (by successor step).
      End.

      Case $\alpha$ is a limit ordinal.
        Then $\beta \in \Phi$ for all ordinals $\beta$ less than $\alpha$.
        Hence $\alpha \in \Phi$ (by limit step).
      End.
    End.

    [prover vampire]
    Thus $\Phi$ contains every ordinal (by \printref{SET_THEORY_02_8493935460614144}).
    [prover eprover]
  \end{proof}
\end{smodule}
\end{document}
