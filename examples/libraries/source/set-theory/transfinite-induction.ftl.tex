\documentclass[10pt]{article}
\usepackage[english]{babel}
\usepackage{../../../meta-inf/lib/naproche}
% Top-level sections and proofs

\newtheoremstyle{foundations}% name of the style to be used
  {\dimexpr\ftlparskip+4pt\relax}% measure of space to leave above the theorem.
  % (We must add \ftlparskip since otherwise the name of the label would not fit
  % into the border drawn around the environment.)
  {4pt}% measure of space to leave below the theorem.
  {\normalfont}% name of font to use in the body of the theorem
  {0pt}% measure of space to indent
  {\bfseries}% name of head font
  {}% punctuation between head and body
  { }% space after theorem head; " " = normal inter-word space
  {\thmname{#1}\thmnumber{ #2}\thmnote{ (#3)}.}

\let\definition\relax
\let\enddefinition\relax
\let\signature\relax
\let\endsignature\relax
\let\axiom\relax
\let\endaxiom\relax
\let\theorem\relax
\let\endtheorem\relax
\let\proposition\relax
\let\endproposition\relax
\let\lemma\relax
\let\endlemma\relax
\let\corollary\relax
\let\endcorollary\relax

\theoremstyle{foundations}

\newmdtheoremenv[
  backgroundcolor=lightgray!10,
  linecolor=black,
  innertopmargin=1em
]{definition}{Definition}[section]
\newmdtheoremenv[
  backgroundcolor=lightgray!10,
  linecolor=black,
  innertopmargin=1em
]{signature}[definition]{Signature}
\newmdtheoremenv[
  backgroundcolor=lightgray!10,
  linecolor=black,
  innertopmargin=1em
]{axiom}[definition]{Axiom}
\newmdtheoremenv[
  backgroundcolor=lightgray!10,
  linecolor=black,
  innertopmargin=1em
]{theorem}[definition]{Theorem}
\newmdtheoremenv[
  backgroundcolor=lightgray!10,
  linecolor=black,
  innertopmargin=1em
]{proposition}[definition]{Proposition}
\newmdtheoremenv[
  backgroundcolor=lightgray!10,
  linecolor=black,
  innertopmargin=1em
]{lemma}[definition]{Lemma}
\newmdtheoremenv[
  backgroundcolor=lightgray!10,
  linecolor=black,
  innertopmargin=1em
]{corollary}[definition]{Corollary}


% Hide imports

\usepackage{scontents}
\newenvsc{imports}[store-env=imports,print-env=false]


% Labels of top-level sections

\renewcommand{\showlabelfont}{\scriptsize\ttfamily\color{gray}}


% Non-symbolic "\implies" and "\iff"

\renewcommand{\implies}{\quad \text{implies} \quad}
\renewcommand{\iff}{\quad \text{iff} \quad}

\usepackage{amssymb}
\usepackage{pgffor}

\newcommand{\id}{\textrm{id}}
\newcommand{\range}{\textrm{range}}
\newcommand{\const}[2]{\textrm{const}_{#1}^{#2}}
\newcommand{\symdiff}{\mathop{\triangle}}
\newcommand{\pow}{\mathcal{P}}
\newcommand{\onto}{\twoheadrightarrow}
\newcommand{\into}{\hookrightarrow}

% E.g. `\set{1, 2, 3, 4}` prints the numbers 1, 2, 3 and 4 as a comma
% separated list which is enclosed within curly braces.
\def\set#1{\ensuremath{
  \{\foreach[count=\i]\x in {#1}{
    \ifnum\i>1,\,\fi
    \x
  }\}
}}

\renewcommand{\succ}{\textrm{succ}}
\newcommand{\pred}{\textrm{pred}}
\newcommand{\V}{\textbf{V}}
\newcommand{\Ord}{\textbf{Ord}}
\newcommand{\Card}{\textbf{Card}}

\begin{document}
  \begin{imports}
    \begin{forthel}
      %[prove off][check off]
      [readtex \path{libraries/source/set-theory/ordering.ftl.tex}]
      %[prove on][check on]
    \end{forthel}
  \end{imports}


  \section*{Transfinite Induction}

  \begin{forthel}
    \begin{theorem}\printlabel{SET_THEORY_02_8493935460614144}
      Let $\Phi$ be a class.
      Assume that for all ordinals $\alpha$ if $\Phi$ contains all $\beta \in \alpha$ then $\Phi$ contains $\alpha$.
      Then $\Phi$ contains every ordinal.
    \end{theorem}
    \begin{proof}
      Define $B = \{ x \mid x$ is a set and if $x \in \Ord$ then $x \in \Phi \}$.

      Let us show that for all sets $x$ if $B$ contains every element of $x$ that is a set then $B$ contains $x$.
        Let $x$ be a set.
        Assume that every element of $x$ that is a set is contained in $B$.

        Case $x \notin \Ord$. Trivial.

        Case $x \in \Ord$.
          Then $\Phi$ contains all ordinals less than $x$.
          Hence $\Phi$ contains $x$.
          Thus $x \in B$.
        End.
      End.

      [prover vampire]
      Hence $B$ contains every set (by \printref{FOUNDATIONS_11_2812087589928960}).
      [prover eprover]
      Thus $\Phi$ contains every ordinal.
    \end{proof}
  \end{forthel}

  \begin{forthel}
    \begin{theorem}\printlabel{SET_THEORY_02_7892040431960064}
      Let $\Phi$ be a class.
      
      (Initial case) Assume that $\Phi$ contains $0$.
      
      (Successor step) Assume that for all ordinals $\alpha$ if $\alpha \in \Phi$ then $\succ(\alpha) \in \Phi$.
      
      (Limit step) Assume that for all limit ordinals $\lambda$ if every $\alpha \in \lambda$ is contained in $\Phi$ then $\lambda \in \Phi$.
      
      Then $\Phi$ contains every ordinal.
    \end{theorem}
    \begin{proof}
      Let us show that for all ordinals $\alpha$ if $\Phi$ contains all ordinals
      less than $\alpha$ then $\Phi$ contains $\alpha$.
        Let $\alpha$ be an ordinal.
        Then $\alpha = 0$ or $\alpha$ is a successor ordinal or $\alpha$ is a
        limit ordinal.
        Assume that $\Phi$ contains all $\beta \in \alpha$.

        Case $\alpha = 0$. Trivial.

        Case $\alpha$ is a successor ordinal.
          Take an ordinal $\beta$ such that $\alpha = \succ(\beta)$.
          Then $\beta \in \Phi$.
          Hence $\alpha \in \Phi$ (by successor step).
        End.

        Case $\alpha$ is a limit ordinal.
          Then $\beta \in \Phi$ for all ordinals $\beta$ less than $\alpha$.
          Hence $\alpha \in \Phi$ (by limit step).
        End.
      End.

      [prover vampire]
      Thus $\Phi$ contains every ordinal (by \printref{SET_THEORY_02_8493935460614144}).
      [prover eprover]
    \end{proof}
  \end{forthel}
\end{document}
