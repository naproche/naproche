\documentclass[10pt]{article}
\usepackage[english]{babel}
\usepackage{../../../meta-inf/lib/naproche}
% Top-level sections and proofs

\newtheoremstyle{foundations}% name of the style to be used
  {\dimexpr\ftlparskip+4pt\relax}% measure of space to leave above the theorem.
  % (We must add \ftlparskip since otherwise the name of the label would not fit
  % into the border drawn around the environment.)
  {4pt}% measure of space to leave below the theorem.
  {\normalfont}% name of font to use in the body of the theorem
  {0pt}% measure of space to indent
  {\bfseries}% name of head font
  {}% punctuation between head and body
  { }% space after theorem head; " " = normal inter-word space
  {\thmname{#1}\thmnumber{ #2}\thmnote{ (#3)}.}

\let\definition\relax
\let\enddefinition\relax
\let\signature\relax
\let\endsignature\relax
\let\axiom\relax
\let\endaxiom\relax
\let\theorem\relax
\let\endtheorem\relax
\let\proposition\relax
\let\endproposition\relax
\let\lemma\relax
\let\endlemma\relax
\let\corollary\relax
\let\endcorollary\relax

\theoremstyle{foundations}

\newmdtheoremenv[
  backgroundcolor=lightgray!10,
  linecolor=black,
  innertopmargin=1em
]{definition}{Definition}[section]
\newmdtheoremenv[
  backgroundcolor=lightgray!10,
  linecolor=black,
  innertopmargin=1em
]{signature}[definition]{Signature}
\newmdtheoremenv[
  backgroundcolor=lightgray!10,
  linecolor=black,
  innertopmargin=1em
]{axiom}[definition]{Axiom}
\newmdtheoremenv[
  backgroundcolor=lightgray!10,
  linecolor=black,
  innertopmargin=1em
]{theorem}[definition]{Theorem}
\newmdtheoremenv[
  backgroundcolor=lightgray!10,
  linecolor=black,
  innertopmargin=1em
]{proposition}[definition]{Proposition}
\newmdtheoremenv[
  backgroundcolor=lightgray!10,
  linecolor=black,
  innertopmargin=1em
]{lemma}[definition]{Lemma}
\newmdtheoremenv[
  backgroundcolor=lightgray!10,
  linecolor=black,
  innertopmargin=1em
]{corollary}[definition]{Corollary}


% Hide imports

\usepackage{scontents}
\newenvsc{imports}[store-env=imports,print-env=false]


% Labels of top-level sections

\renewcommand{\showlabelfont}{\scriptsize\ttfamily\color{gray}}


% Non-symbolic "\implies" and "\iff"

\renewcommand{\implies}{\quad \text{implies} \quad}
\renewcommand{\iff}{\quad \text{iff} \quad}

\usepackage{amssymb}
\usepackage{pgffor}

\newcommand{\id}{\textrm{id}}
\newcommand{\range}{\textrm{range}}
\newcommand{\const}[2]{\textrm{const}_{#1}^{#2}}
\newcommand{\symdiff}{\mathop{\triangle}}
\newcommand{\pow}{\mathcal{P}}
\newcommand{\onto}{\twoheadrightarrow}
\newcommand{\into}{\hookrightarrow}

% E.g. `\set{1, 2, 3, 4}` prints the numbers 1, 2, 3 and 4 as a comma
% separated list which is enclosed within curly braces.
\def\set#1{\ensuremath{
  \{\foreach[count=\i]\x in {#1}{
    \ifnum\i>1,\,\fi
    \x
  }\}
}}

\renewcommand{\succ}{\textrm{succ}}
\newcommand{\pred}{\textrm{pred}}
\newcommand{\V}{\textbf{V}}
\newcommand{\Ord}{\textbf{Ord}}
\newcommand{\Card}{\textbf{Card}}

\begin{document}
  \begin{imports}
    \begin{forthel}
      %[prove off][check off]
      [read \path{libraries/source/set-theory/limit-ordinals.ftl.tex}]
      %[prove on][check on]
    \end{forthel}
  \end{imports}


  \section*{The Standard Ordering of the Ordinals}

  \begin{forthel}
    \begin{definition}[id=SET_THEORY_02_6654252130762752,printid]
      Let $\alpha, \beta$ be ordinals.
      $\alpha$ is less than $\beta$ iff $\alpha \in \beta$.
    \end{definition}

    Let $\alpha < \beta$ stand for $\alpha$ is less than $\beta$.
    Let $\alpha \nless \beta$ stand for not $\alpha < \beta$.

    Let $\alpha$ is greater than $\beta$ stand for $\beta < \alpha$.
    Let $\alpha > \beta$ stand for $\beta < \alpha$.
    Let $\alpha \ngtr \beta$ stand for not $\alpha > \beta$.
  \end{forthel}

  \begin{forthel}
    \begin{definition}[id=SET_THEORY_02_2639956210089984,printid]
      Let $\alpha, \beta$ be ordinals.
      $\alpha$ is less than or equal to $\beta$ iff $\alpha < \beta$ or $\alpha = \beta$.
    \end{definition}

    Let $\alpha \leq \beta$ stand for $\alpha$ is less than or equal to $\beta$.
    Let $\alpha \nleq \beta$ stand for not $\alpha \leq \beta$.

    Let $\alpha$ is greater than or equal to $\beta$ stand for $\beta \leq \alpha$.
    Let $\alpha \geq \beta$ stand for $\beta \leq \alpha$.
    Let $\alpha \ngeq \beta$ stand for not $\alpha \geq \beta$.
  \end{forthel}

  \begin{forthel}
    \begin{proposition}[id=SET_THEORY_02_3089369577553920,printid]
      Let $\alpha, \beta$ be ordinals.
      If $\alpha \leq \beta$ then $\alpha \subseteq \beta$.
    \end{proposition}
    \begin{proof}
      Case $\alpha \leq \beta$.
        Then $\alpha < \beta$ or $\alpha = \beta$.
        Let $x \in \alpha$.
        If $\alpha < \beta$ then $x \in \alpha \in \beta$.
        Hence if $\alpha < \beta$ then $x \in \beta$.
        If $\alpha = \beta$ then $x \in \beta$.
        Thus $x \in \beta$.
      End.
    \end{proof}
  \end{forthel}

  \begin{forthel}
    \begin{proposition}[id=SET_THEORY_02_6229364135952384,printid]
      Let $\alpha$ be an ordinal.
      Then $\alpha \nless \alpha$.
    \end{proposition}
    \begin{proof}
      Assume $\alpha < \alpha$.
      Then $\alpha \in \alpha$.
      Contradiction.
    \end{proof}
  \end{forthel}

  \begin{forthel}
    \begin{proposition}[id=SET_THEORY_02_7098683017396224,printid]
      Let $\alpha, \beta, \gamma$ be ordinals.
      If $\alpha < \beta$ and $\beta < \gamma$ then $\alpha < \gamma$.
    \end{proposition}
    \begin{proof}
      Assume $\alpha < \beta$ and $\beta < \gamma$.
      Then $\alpha \in \beta \in \gamma$.
      Hence $\alpha \in \gamma$.
      Thus $\alpha < \gamma$.
    \end{proof}
  \end{forthel}

  \begin{forthel}
    \begin{proposition}[id=SET_THEORY_02_1718825707896832,printid]
      Let $\alpha, \beta$ be ordinals.
      Then $\alpha < \beta$ or $\alpha = \beta$ or $\alpha > \beta$.
    \end{proposition}
    \begin{proof}
      Assume the contrary.
      
      Define $A = \{ \alpha' \in \Ord \mid$ there exists an ordinal $\beta'$ such that neither $\alpha' < \beta'$ nor $\alpha' = \beta'$ nor $\alpha' > \beta' \}$.
      
      $A$ is nonempty.
      Hence we can take a $\alpha' \in A$ such that for no $\gamma \in A$ we have $\gamma < \alpha'$.
      
      Define $B = \{ \beta' \in \Ord \mid$ neither $\alpha' < \beta'$ nor $\alpha' = \beta'$ nor $\alpha' > \beta' \}$.
      
      $B$ is nonempty.
      Hence we can take a $\beta' \in B$ such that for no $\gamma \in B$ we have $\gamma < \beta'$.

      Let us show that $\alpha' \subseteq \beta'$.
        Let $a \in \alpha'$.
        Then $a < \beta'$ or $a = \beta'$ or $a > \beta'$.
        Indeed if neither $a < \beta'$ nor $a = \beta'$ nor $a > \beta'$ then
        $a \in A$.
        If $a = \beta'$ then $\beta' < \alpha'$.
        If $a > \beta'$ then $\beta' < \alpha'$.
        Hence $a < \beta'$.
        Thus $a \in \beta'$.
      End.

      Let us show that $\beta' \subseteq \alpha'$.
        Let $b \in \beta'$.
        Then $b < \alpha'$ or $b = \alpha'$ or $b > \alpha'$.
        If $b = \alpha'$ then $\alpha' < \beta'$.
        If $b > \alpha'$ then $\alpha' < \beta'$.
        Hence $b < \alpha'$.
        Thus $b \in \alpha'$.
      End.

      Hence $\alpha' = \beta'$.
      Contradiction.
    \end{proof}
  \end{forthel}

  \begin{forthel}
    \begin{proposition}[id=SET_THEORY_02_610496856195072,printid]
      Let $\alpha, \beta$ be ordinals.
      If $\alpha \subseteq \beta$ then $\alpha \leq \beta$.
    \end{proposition}
    \begin{proof}
      Assume $\alpha \subseteq \beta$.

      Case $\alpha = \beta$. Trivial.

      Case $\alpha \neq \beta$.
        Then $\alpha < \beta$ or $\alpha > \beta$.
        Assume $\alpha > \beta$.
        Then $\beta \in \alpha$.
        Hence $\beta \in \beta$.
        Contradiction.
      End.
    \end{proof}
  \end{forthel}

  \begin{forthel}
    \begin{proposition}[id=SET_THEORY_02_5689190964527104,printid]
      Let $\alpha$ be an ordinal.
      Then $\alpha < \succ(\alpha)$.
    \end{proposition}
  \end{forthel}

  \begin{forthel}
    \begin{proposition}[id=SET_THEORY_02_4064972025888768,printid]
      Let $\alpha, \beta$ be ordinals.
      If $\beta < \succ(\alpha)$ then $\beta \leq \alpha$.
    \end{proposition}
    \begin{proof}
      Assume $\beta < \succ(\alpha)$.
      Then $\beta \in \succ(\alpha) = \alpha \cup \set{\alpha}$.
      Hence $\beta \in \alpha$ or $\beta \in \set{\alpha}$.
      Thus $\beta < \alpha$ or $\beta = \alpha$.
      Therefore $\beta \leq \alpha$.
    \end{proof}
  \end{forthel}

  \begin{forthel}
    \begin{proposition}[id=SET_THEORY_02_8242798790705152,printid]
      Let $\alpha$ be an ordinal.
      There exists no ordinal $\beta$ such that $\alpha < \beta < \succ(\alpha)$.
    \end{proposition}
    \begin{proof}
      Assume the contrary.
      Consider an ordinal $\beta$ such that $\alpha < \beta < \succ(\alpha)$.
      Then $\beta < \alpha$ or $\beta = \alpha$.
      Hence $\alpha < \alpha$.
      Contradiction.
    \end{proof}
  \end{forthel}

  \begin{forthel}
    \begin{proposition}[id=SET_THEORY_02_4240355610329088,printid]
      Let $\alpha$ be an ordinal.
      There exists no ordinal $\beta$ such that $\alpha < \beta < \succ(\alpha)$.
    \end{proposition}
    \begin{proof}
      Assume the contrary.
      Choose an ordinal $\beta$ such that $\alpha < \beta < \succ(\alpha)$.
      Then $\alpha \in \beta \in \alpha \cup \set{\alpha}$.
      Hence $\beta \in \alpha$ or $\beta = \alpha$.
      Then $\alpha \in \alpha$.
      Contradiction.
    \end{proof}
  \end{forthel}

  \begin{forthel}
    \begin{proposition}[id=SET_THEORY_02_4659024620421120,printid]
      Let $\lambda$ be a limit ordinal and $\alpha \in \lambda$.
      Then $\lambda$ contains $\succ(\alpha)$.
    \end{proposition}
    \begin{proof}
      If $\succ(\alpha) \notin \lambda$ then $\alpha < \lambda < \succ(\alpha)$.
    \end{proof}
  \end{forthel}
\end{document}
