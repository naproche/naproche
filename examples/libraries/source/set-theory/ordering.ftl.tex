\documentclass{naproche-library}
\libinput{set-theory-preamble}
\begin{document}
\begin{smodule}{ordering.ftl}
  \importmodule[libraries]{set-theory?limit-ordinals.ftl}

  \section*{The Standard Ordering of the Ordinals}

  \begin{definition}[forthel,id=SET_THEORY_02_6654252130762752]
    Let $\alpha, \beta$ be ordinals.
    $\alpha$ is less than $\beta$ iff $\alpha \in \beta$.

    Let $\alpha < \beta$ stand for $\alpha$ is less than $\beta$.
    Let $\alpha \nless \beta$ stand for not $\alpha < \beta$.

    Let $\alpha$ is greater than $\beta$ stand for $\beta < \alpha$.
    Let $\alpha > \beta$ stand for $\beta < \alpha$.
    Let $\alpha \ngtr \beta$ stand for not $\alpha > \beta$.
  \end{definition}

  \begin{definition}[forthel,id=SET_THEORY_02_2639956210089984]
    Let $\alpha, \beta$ be ordinals.
    $\alpha$ is less than or equal to $\beta$ iff $\alpha < \beta$ or $\alpha = \beta$.

    Let $\alpha \leq \beta$ stand for $\alpha$ is less than or equal to $\beta$.
    Let $\alpha \nleq \beta$ stand for not $\alpha \leq \beta$.

    Let $\alpha$ is greater than or equal to $\beta$ stand for $\beta \leq \alpha$.
    Let $\alpha \geq \beta$ stand for $\beta \leq \alpha$.
    Let $\alpha \ngeq \beta$ stand for not $\alpha \geq \beta$.
  \end{definition}

  \begin{proposition}[forthel,id=SET_THEORY_02_3089369577553920]
    Let $\alpha, \beta$ be ordinals.
    If $\alpha \leq \beta$ then $\alpha \subseteq \beta$.
  \end{proposition}
  \begin{proof}[forthel]
    Case $\alpha \leq \beta$.
      Then $\alpha < \beta$ or $\alpha = \beta$.
      Let $x \in \alpha$.
      If $\alpha < \beta$ then $x \in \alpha \in \beta$.
      Hence if $\alpha < \beta$ then $x \in \beta$.
      If $\alpha = \beta$ then $x \in \beta$.
      Thus $x \in \beta$.
    End.
  \end{proof}

  \begin{proposition}[forthel,id=SET_THEORY_02_6229364135952384]
    Let $\alpha$ be an ordinal.
    Then $\alpha \nless \alpha$.
  \end{proposition}
  \begin{proof}[forthel]
    Assume $\alpha < \alpha$.
    Then $\alpha \in \alpha$.
    Contradiction.
  \end{proof}

  \begin{proposition}[forthel,id=SET_THEORY_02_7098683017396224]
    Let $\alpha, \beta, \gamma$ be ordinals.
    If $\alpha < \beta$ and $\beta < \gamma$ then $\alpha < \gamma$.
  \end{proposition}
  \begin{proof}[forthel]
    Assume $\alpha < \beta$ and $\beta < \gamma$.
    Then $\alpha \in \beta \in \gamma$.
    Hence $\alpha \in \gamma$.
    Thus $\alpha < \gamma$.
  \end{proof}

  \begin{proposition}[forthel,id=SET_THEORY_02_1718825707896832]
    Let $\alpha, \beta$ be ordinals.
    Then $\alpha < \beta$ or $\alpha = \beta$ or $\alpha > \beta$.
  \end{proposition}
  \begin{proof}[forthel]
    Assume the contrary.
    
    Define $A = \{ \alpha' \in \Ord \mid$ there exists an ordinal $\beta'$ such that neither $\alpha' < \beta'$ nor $\alpha' = \beta'$ nor $\alpha' > \beta' \}$.
    
    $A$ is nonempty.
    Hence we can take a $\alpha' \in A$ such that for no $\gamma \in A$ we have $\gamma < \alpha'$.
    
    Define $B = \{ \beta' \in \Ord \mid$ neither $\alpha' < \beta'$ nor $\alpha' = \beta'$ nor $\alpha' > \beta' \}$.
    
    $B$ is nonempty.
    Hence we can take a $\beta' \in B$ such that for no $\gamma \in B$ we have $\gamma < \beta'$.

    Let us show that $\alpha' \subseteq \beta'$.
      Let $a \in \alpha'$.
      Then $a < \beta'$ or $a = \beta'$ or $a > \beta'$.
      Indeed if neither $a < \beta'$ nor $a = \beta'$ nor $a > \beta'$ then
      $a \in A$.
      If $a = \beta'$ then $\beta' < \alpha'$.
      If $a > \beta'$ then $\beta' < \alpha'$.
      Hence $a < \beta'$.
      Thus $a \in \beta'$.
    End.

    Let us show that $\beta' \subseteq \alpha'$.
      Let $b \in \beta'$.
      Then $b < \alpha'$ or $b = \alpha'$ or $b > \alpha'$.
      If $b = \alpha'$ then $\alpha' < \beta'$.
      If $b > \alpha'$ then $\alpha' < \beta'$.
      Hence $b < \alpha'$.
      Thus $b \in \alpha'$.
    End.

    Hence $\alpha' = \beta'$.
    Contradiction.
  \end{proof}

  \begin{proposition}[forthel,id=SET_THEORY_02_610496856195072]
    Let $\alpha, \beta$ be ordinals.
    If $\alpha \subseteq \beta$ then $\alpha \leq \beta$.
  \end{proposition}
  \begin{proof}[forthel]
    Assume $\alpha \subseteq \beta$.

    Case $\alpha = \beta$. Trivial.

    Case $\alpha \neq \beta$.
      Then $\alpha < \beta$ or $\alpha > \beta$.
      Assume $\alpha > \beta$.
      Then $\beta \in \alpha$.
      Hence $\beta \in \beta$.
      Contradiction.
    End.
  \end{proof}

  \begin{proposition}[forthel,id=SET_THEORY_02_5689190964527104]
    Let $\alpha$ be an ordinal.
    Then $\alpha < \succ(\alpha)$.
  \end{proposition}

  \begin{proposition}[forthel,id=SET_THEORY_02_4064972025888768]
    Let $\alpha, \beta$ be ordinals.
    If $\beta < \succ(\alpha)$ then $\beta \leq \alpha$.
  \end{proposition}
  \begin{proof}[forthel]
    Assume $\beta < \succ(\alpha)$.
    Then $\beta \in \succ(\alpha) = \alpha \cup \set{\alpha}$.
    Hence $\beta \in \alpha$ or $\beta \in \set{\alpha}$.
    Thus $\beta < \alpha$ or $\beta = \alpha$.
    Therefore $\beta \leq \alpha$.
  \end{proof}

  \begin{proposition}[forthel,id=SET_THEORY_02_8242798790705152]
    Let $\alpha$ be an ordinal.
    There exists no ordinal $\beta$ such that $\alpha < \beta < \succ(\alpha)$.
  \end{proposition}
  \begin{proof}[forthel]
    Assume the contrary.
    Consider an ordinal $\beta$ such that $\alpha < \beta < \succ(\alpha)$.
    Then $\beta < \alpha$ or $\beta = \alpha$.
    Hence $\alpha < \alpha$.
    Contradiction.
  \end{proof}

  \begin{proposition}[forthel,id=SET_THEORY_02_4240355610329088]
    Let $\alpha$ be an ordinal.
    There exists no ordinal $\beta$ such that $\alpha < \beta < \succ(\alpha)$.
  \end{proposition}
  \begin{proof}[forthel]
    Assume the contrary.
    Choose an ordinal $\beta$ such that $\alpha < \beta < \succ(\alpha)$.
    Then $\alpha \in \beta \in \alpha \cup \set{\alpha}$.
    Hence $\beta \in \alpha$ or $\beta = \alpha$.
    Then $\alpha \in \alpha$.
    Contradiction.
  \end{proof}

  \begin{proposition}[forthel,id=SET_THEORY_02_4659024620421120]
    Let $\lambda$ be a limit ordinal and $\alpha \in \lambda$.
    Then $\lambda$ contains $\succ(\alpha)$.
  \end{proposition}
  \begin{proof}[forthel]
    If $\succ(\alpha) \notin \lambda$ then $\alpha < \lambda < \succ(\alpha)$.
  \end{proof}
\end{smodule}
\end{document}
