\documentclass[10pt]{article}
\usepackage[english]{babel}
\usepackage{../../../meta-inf/lib/naproche}
% Top-level sections and proofs

\newtheoremstyle{foundations}% name of the style to be used
  {\dimexpr\ftlparskip+4pt\relax}% measure of space to leave above the theorem.
  % (We must add \ftlparskip since otherwise the name of the label would not fit
  % into the border drawn around the environment.)
  {4pt}% measure of space to leave below the theorem.
  {\normalfont}% name of font to use in the body of the theorem
  {0pt}% measure of space to indent
  {\bfseries}% name of head font
  {}% punctuation between head and body
  { }% space after theorem head; " " = normal inter-word space
  {\thmname{#1}\thmnumber{ #2}\thmnote{ (#3)}.}

\let\definition\relax
\let\enddefinition\relax
\let\signature\relax
\let\endsignature\relax
\let\axiom\relax
\let\endaxiom\relax
\let\theorem\relax
\let\endtheorem\relax
\let\proposition\relax
\let\endproposition\relax
\let\lemma\relax
\let\endlemma\relax
\let\corollary\relax
\let\endcorollary\relax

\theoremstyle{foundations}

\newmdtheoremenv[
  backgroundcolor=lightgray!10,
  linecolor=black,
  innertopmargin=1em
]{definition}{Definition}[section]
\newmdtheoremenv[
  backgroundcolor=lightgray!10,
  linecolor=black,
  innertopmargin=1em
]{signature}[definition]{Signature}
\newmdtheoremenv[
  backgroundcolor=lightgray!10,
  linecolor=black,
  innertopmargin=1em
]{axiom}[definition]{Axiom}
\newmdtheoremenv[
  backgroundcolor=lightgray!10,
  linecolor=black,
  innertopmargin=1em
]{theorem}[definition]{Theorem}
\newmdtheoremenv[
  backgroundcolor=lightgray!10,
  linecolor=black,
  innertopmargin=1em
]{proposition}[definition]{Proposition}
\newmdtheoremenv[
  backgroundcolor=lightgray!10,
  linecolor=black,
  innertopmargin=1em
]{lemma}[definition]{Lemma}
\newmdtheoremenv[
  backgroundcolor=lightgray!10,
  linecolor=black,
  innertopmargin=1em
]{corollary}[definition]{Corollary}


% Hide imports

\usepackage{scontents}
\newenvsc{imports}[store-env=imports,print-env=false]


% Labels of top-level sections

\renewcommand{\showlabelfont}{\scriptsize\ttfamily\color{gray}}


% Non-symbolic "\implies" and "\iff"

\renewcommand{\implies}{\quad \text{implies} \quad}
\renewcommand{\iff}{\quad \text{iff} \quad}

\usepackage{amssymb}
\usepackage{pgffor}

\newcommand{\id}{\textrm{id}}
\newcommand{\range}{\textrm{range}}
\newcommand{\const}[2]{\textrm{const}_{#1}^{#2}}
\newcommand{\symdiff}{\mathop{\triangle}}
\newcommand{\pow}{\mathcal{P}}
\newcommand{\onto}{\twoheadrightarrow}
\newcommand{\into}{\hookrightarrow}

% E.g. `\set{1, 2, 3, 4}` prints the numbers 1, 2, 3 and 4 as a comma
% separated list which is enclosed within curly braces.
\def\set#1{\ensuremath{
  \{\foreach[count=\i]\x in {#1}{
    \ifnum\i>1,\,\fi
    \x
  }\}
}}

\renewcommand{\succ}{\textrm{succ}}
\newcommand{\pred}{\textrm{pred}}
\newcommand{\V}{\textbf{V}}
\newcommand{\Ord}{\textbf{Ord}}
\newcommand{\Card}{\textbf{Card}}

\begin{document}
  \begin{imports}
    \begin{forthel}
      %[prove off][check off]
      [readtex \path{libraries/source/set-theory/axioms/choice.ftl.tex}]
      [readtex \path{libraries/source/set-theory/ordering.ftl.tex}]
      [readtex \path{libraries/source/foundations/equinumerosity.ftl.tex}]
      %[prove on][check on]
    \end{forthel}
  \end{imports}


  \section*{Cardinal Numbers}

  \begin{forthel}
    \begin{definition}\printlabel{SET_THEORY_06_8286266038681600}
      Let $x$ be a set.
      The cardinality of $x$ is the ordinal $\kappa$ such that $\kappa$ is equinumerous to $x$ and every ordinal that is equinumerous to $x$ is greater than or equal to $\kappa$.
    \end{definition}

    Let $|x|$ stand for the cardinality of $x$.
  \end{forthel}

  \begin{forthel}
    \begin{definition}\printlabel{SET_THEORY_06_6818986081648640}
      A cardinal number is an ordinal $\kappa$ such that $\kappa = |x|$ for some
      set $x$.
    \end{definition}

    Let a cardinal stand for a cardinal number.
  \end{forthel}

  \begin{forthel}
    \begin{proposition}\printlabel{SET_THEORY_06_2820082336006144}
      Let $\kappa$ be a cardinal.
      Then $|\kappa| = \kappa$.
    \end{proposition}
    \begin{proof}
      $\kappa$ is an ordinal that is equinumerous to $\kappa$.
      Hence $|\kappa| \leq \kappa$.
      Consider a set $x$ such that $\kappa = |x|$.
      Then $|\kappa|$ is an ordinal that is equinumerous to $x$.
      Hence $\kappa \leq |\kappa|$.
      Thus $|\kappa| = \kappa$.
    \end{proof}
  \end{forthel}

  \begin{forthel}
    \begin{proposition}\printlabel{SET_THEORY_06_6920913721229312}
      Let $x, y$ be sets.
      Then $x$ and $y$ are equinumerous iff $|x| = |y|$.
    \end{proposition}
    \begin{proof}
      Case $x$ and $y$ are equinumerous.
        Take a bijection $f$ between $x$ and $y$.
        Consider a bijection $g$ between $y$ and $|y|$.
        Then $g \circ f$ is a bijection between $x$ and $|y|$ (by \printref{FOUNDATIONS_08_6435206693126144}).
        Hence $x$ and $|y|$ are equinumerous.
        Thus $|y| \geq |x|$.

        $f^{-1}$ is a bijection between $y$ and $x$.
        Consider a bijection $h$ between $x$ and $|x|$.
        Then $h \circ f^{-1}$ is a bijection between $y$ and $|x|$ (by \printref{FOUNDATIONS_08_6435206693126144}).
        Hence $y$ and $|x|$ are equinumerous.
        Thus $|x| \geq |y|$.

        Therefore $|x| = |y|$.
      End.

      Case $|x| = |y|$.
        Consider a bijection $f$ between $x$ and $|x|$ and a bijection $g$
        between $|y|$ and $y$.
        Then $g \circ f$ is a bijection between $x$ and $y$.
        Hence $x$ and $y$ are equinumerous.
      End.
    \end{proof}
  \end{forthel}
\end{document}
