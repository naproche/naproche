\documentclass[10pt]{article}
\usepackage[english]{babel}
\usepackage{../../../meta-inf/lib/naproche}
% Top-level sections and proofs

\newtheoremstyle{foundations}% name of the style to be used
  {\dimexpr\ftlparskip+4pt\relax}% measure of space to leave above the theorem.
  % (We must add \ftlparskip since otherwise the name of the label would not fit
  % into the border drawn around the environment.)
  {4pt}% measure of space to leave below the theorem.
  {\normalfont}% name of font to use in the body of the theorem
  {0pt}% measure of space to indent
  {\bfseries}% name of head font
  {}% punctuation between head and body
  { }% space after theorem head; " " = normal inter-word space
  {\thmname{#1}\thmnumber{ #2}\thmnote{ (#3)}.}

\let\definition\relax
\let\enddefinition\relax
\let\signature\relax
\let\endsignature\relax
\let\axiom\relax
\let\endaxiom\relax
\let\theorem\relax
\let\endtheorem\relax
\let\proposition\relax
\let\endproposition\relax
\let\lemma\relax
\let\endlemma\relax
\let\corollary\relax
\let\endcorollary\relax

\theoremstyle{foundations}

\newmdtheoremenv[
  backgroundcolor=lightgray!10,
  linecolor=black,
  innertopmargin=1em
]{definition}{Definition}[section]
\newmdtheoremenv[
  backgroundcolor=lightgray!10,
  linecolor=black,
  innertopmargin=1em
]{signature}[definition]{Signature}
\newmdtheoremenv[
  backgroundcolor=lightgray!10,
  linecolor=black,
  innertopmargin=1em
]{axiom}[definition]{Axiom}
\newmdtheoremenv[
  backgroundcolor=lightgray!10,
  linecolor=black,
  innertopmargin=1em
]{theorem}[definition]{Theorem}
\newmdtheoremenv[
  backgroundcolor=lightgray!10,
  linecolor=black,
  innertopmargin=1em
]{proposition}[definition]{Proposition}
\newmdtheoremenv[
  backgroundcolor=lightgray!10,
  linecolor=black,
  innertopmargin=1em
]{lemma}[definition]{Lemma}
\newmdtheoremenv[
  backgroundcolor=lightgray!10,
  linecolor=black,
  innertopmargin=1em
]{corollary}[definition]{Corollary}


% Hide imports

\usepackage{scontents}
\newenvsc{imports}[store-env=imports,print-env=false]


% Labels of top-level sections

\renewcommand{\showlabelfont}{\scriptsize\ttfamily\color{gray}}


% Non-symbolic "\implies" and "\iff"

\renewcommand{\implies}{\quad \text{implies} \quad}
\renewcommand{\iff}{\quad \text{iff} \quad}

\usepackage{amssymb}
\usepackage{pgffor}

\newcommand{\id}{\textrm{id}}
\newcommand{\range}{\textrm{range}}
\newcommand{\const}[2]{\textrm{const}_{#1}^{#2}}
\newcommand{\symdiff}{\mathop{\triangle}}
\newcommand{\pow}{\mathcal{P}}
\newcommand{\onto}{\twoheadrightarrow}
\newcommand{\into}{\hookrightarrow}

% E.g. `\set{1, 2, 3, 4}` prints the numbers 1, 2, 3 and 4 as a comma
% separated list which is enclosed within curly braces.
\def\set#1{\ensuremath{
  \{\foreach[count=\i]\x in {#1}{
    \ifnum\i>1,\,\fi
    \x
  }\}
}}

\renewcommand{\succ}{\textrm{succ}}
\newcommand{\pred}{\textrm{pred}}
\newcommand{\V}{\textbf{V}}
\newcommand{\Ord}{\textbf{Ord}}
\newcommand{\Card}{\textbf{Card}}

\begin{document}
  \begin{imports}
    \begin{forthel}
      %[prove off][check off]
      [read \path{libraries/source/foundations/proper-classes.ftl.tex}]
      [read \path{libraries/source/set-theory/finite-and-infinite-sets.ftl.tex}]
      [read \path{cantor.ftl.tex}]
      %[prove on][check on]
    \end{forthel}
  \end{imports}


  \section*{Infinite Cardinal Numbers}

  \begin{forthel}
    \begin{definition}[id=SET_THEORY_07_4231078585827328,printid]
      $\Card$ is the collection of all infinite cardinals.
    \end{definition}
  \end{forthel}

  \begin{forthel}
    \begin{proposition}[id=SET_THEORY_07_8189062544359424,printid]
      Let $\alpha$ be an infinite ordinal.
      Then $|\succ(\alpha)| = |\alpha|$.
    \end{proposition}
    \begin{proof}
      For any $\beta \in \succ(\alpha)$ we have
      $\beta < \omega$ or $\omega \leq \beta < \alpha$ or $\beta = \alpha$.
      Define \[ f(\beta) =
        \begin{cases}
          \succ(\beta)  & : \beta < \omega
          \\
          \beta         & : \omega \leq \beta < \alpha
          \\
          0             & : \beta = \alpha
        \end{cases} \]
      for $\beta \in \succ(\alpha)$.

      Then $f$ is a map from $\succ(\alpha)$ to $\alpha$.
      Indeed we can show that $f(\beta) \in \alpha$ for all
      $\beta \in \succ(\alpha)$. \\
      Proof.
        Let $\beta \in \succ(\alpha)$.

        Case $\beta < \omega$.
          Then $f(\beta)
            = \succ(\beta)
            < \omega
            \leq \alpha$.
        End.

        Case $\omega \leq \beta < \alpha$.
          Then $f(\beta)
            = \beta
            < \alpha$.
        End.

        Case $\beta = \alpha$.
          Then $f(\beta)
            = 0
            < \alpha$.
        End.
      Qed.

      $f$ is surjective onto $\alpha$.
      Indeed we can show that for any $\beta \in \alpha$ there exists a
      $\gamma \in \succ(\alpha)$ such that $\beta = f(\gamma)$. \\
      Proof.
        Let $\beta \in \alpha$.
        Then $\beta = 0$ or $0 < \beta < \omega$ or $\beta \geq \omega$.

        Case $\beta = 0$.
          Then $\beta = f(\alpha)$.
        End.

        Case $0 < \beta < \omega$.
          Take an ordinal $\beta'$ such that $\beta = \succ(\beta')$.
          Then $\beta' < \omega$.
          Hence $\beta = f(\beta')$.
        End.

        Case $\beta \geq \omega$.
          Then $\beta = f(\beta)$.
        End.
      Qed.

      $f$ is injective.
      Indeed we can show that for all $\beta, \gamma \in \succ(\alpha)$ if
      $\beta \neq \gamma$ then $f(\beta) \neq f(\gamma)$. \\
      Proof.
        Let $\beta, \gamma \in \succ(\alpha)$.
        Assume $\beta \neq \gamma$.

        Case $\beta < \omega$.
          If $\gamma = \alpha$ then
          $f(\beta)
            = \succ(\beta)
            \neq 0
            = f(\gamma)$.
          If $\omega \leq \gamma < \alpha$ then
          $f(\beta)
            = \succ(\beta)
            < \omega
            \leq \gamma
            = f(\gamma)$.
        End.

        Case $\omega \leq \beta < \alpha$.
          If $\gamma = \alpha$ then
          $f(\beta)
            = \beta
            \geq \omega
            > 0
            = f(\gamma)$.
          If $\gamma < \omega$ then
          $f(\beta)
            = \beta
            \geq \omega
            > \succ(\gamma)
            = f(\gamma)$.
        End.

        Case $\beta = \alpha$.
          If $\gamma < \omega$ then
          $f(\beta)
            = 0
            \neq \succ(\gamma)
            = f(\gamma)$.
          If $\omega \leq \gamma < \alpha$ then
          $f(\beta)
            = 0
            < \omega
            \leq \gamma
            = f(\gamma)$.
        End.
      Qed.

      Hence $f$ is a bijection between $\succ(\alpha)$ and $\alpha$.
      Therefore $\succ(\alpha)$ and $\alpha$ are equinumerous.
      Consequently $|\succ(\alpha)| = |\alpha|$.
    \end{proof}
  \end{forthel}

  \begin{forthel}
    \begin{proposition}[id=SET_THEORY_07_8700732632989696,printid]
      Every infinite cardinal is a limit ordinal.
    \end{proposition}
    \begin{proof}
      Let $\kappa$ be an infinite cardinal.
      Suppose that $\kappa$ is not a limit ordinal.
      $\kappa \neq 0$.
      Hence $\kappa$ is a successor ordinal.
      Indeed $\kappa$ is an ordinal.
      Thus we can take an ordinal $\alpha$ such that $\kappa = \succ(\alpha)$.
      We have $\alpha > \kappa \geq \omega$.
      Hence $|\succ(\alpha)| = |\alpha|$.
      Thus $\alpha < |\kappa|$ and $\kappa$ is equinumerous to $\kappa$.
      Contradiction.
    \end{proof}
  \end{forthel}
\end{document}
