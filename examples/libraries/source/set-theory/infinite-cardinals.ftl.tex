\documentclass{naproche-library}
\libinput{set-theory-preamble}
\begin{document}
\begin{smodule}{infinite-cardinals.ftl}
  \importmodule[libraries]{foundations?proper-classes.ftl}
  \importmodule[libraries]{set-theory?finite-and-infinite-sets.ftl}

  \symdef{Card}{\textbf{Card}}
  
  \begin{forthel}
    [read \path{cantor.ftl.tex}]
  \end{forthel}

  \section*{Infinite Cardinal Numbers}

  \begin{definition}[forthel,id=SET_THEORY_07_4231078585827328,printid]
    $\Card$ is the collection of all infinite cardinals.
  \end{definition}

  \begin{proposition}[forthel,id=SET_THEORY_07_8189062544359424,printid]
    Let $\alpha$ be an infinite ordinal.
    Then $|\succ(\alpha)| = |\alpha|$.
  \end{proposition}
  \begin{proof}[forthel]
    For any $\beta \in \succ(\alpha)$ we have
    $\beta < \omega$ or $\omega \leq \beta < \alpha$ or $\beta = \alpha$.
    Define \[ f(\beta) =
      \begin{cases}
        \succ(\beta)  & : \beta < \omega
        \\
        \beta         & : \omega \leq \beta < \alpha
        \\
        0             & : \beta = \alpha
      \end{cases} \]
    for $\beta \in \succ(\alpha)$.

    Then $f$ is a map from $\succ(\alpha)$ to $\alpha$.
    Indeed we can show that $f(\beta) \in \alpha$ for all
    $\beta \in \succ(\alpha)$. \\
    Proof.
      Let $\beta \in \succ(\alpha)$.

      Case $\beta < \omega$.
        Then $f(\beta)
          = \succ(\beta)
          < \omega
          \leq \alpha$.
      End.

      Case $\omega \leq \beta < \alpha$.
        Then $f(\beta)
          = \beta
          < \alpha$.
      End.

      Case $\beta = \alpha$.
        Then $f(\beta)
          = 0
          < \alpha$.
      End.
    Qed.

    $f$ is surjective onto $\alpha$.
    Indeed we can show that for any $\beta \in \alpha$ there exists a
    $\gamma \in \succ(\alpha)$ such that $\beta = f(\gamma)$. \\
    Proof.
      Let $\beta \in \alpha$.
      Then $\beta = 0$ or $0 < \beta < \omega$ or $\beta \geq \omega$.

      Case $\beta = 0$.
        Then $\beta = f(\alpha)$.
      End.

      Case $0 < \beta < \omega$.
        Take an ordinal $\beta'$ such that $\beta = \succ(\beta')$.
        Then $\beta' < \omega$.
        Hence $\beta = f(\beta')$.
      End.

      Case $\beta \geq \omega$.
        Then $\beta = f(\beta)$.
      End.
    Qed.

    $f$ is injective.
    Indeed we can show that for all $\beta, \gamma \in \succ(\alpha)$ if
    $\beta \neq \gamma$ then $f(\beta) \neq f(\gamma)$. \\
    Proof.
      Let $\beta, \gamma \in \succ(\alpha)$.
      Assume $\beta \neq \gamma$.

      Case $\beta < \omega$.
        If $\gamma = \alpha$ then
        $f(\beta)
          = \succ(\beta)
          \neq 0
          = f(\gamma)$.
        If $\omega \leq \gamma < \alpha$ then
        $f(\beta)
          = \succ(\beta)
          < \omega
          \leq \gamma
          = f(\gamma)$.
      End.

      Case $\omega \leq \beta < \alpha$.
        If $\gamma = \alpha$ then
        $f(\beta)
          = \beta
          \geq \omega
          > 0
          = f(\gamma)$.
        If $\gamma < \omega$ then
        $f(\beta)
          = \beta
          \geq \omega
          > \succ(\gamma)
          = f(\gamma)$.
      End.

      Case $\beta = \alpha$.
        If $\gamma < \omega$ then
        $f(\beta)
          = 0
          \neq \succ(\gamma)
          = f(\gamma)$.
        If $\omega \leq \gamma < \alpha$ then
        $f(\beta)
          = 0
          < \omega
          \leq \gamma
          = f(\gamma)$.
      End.
    Qed.

    Hence $f$ is a bijection between $\succ(\alpha)$ and $\alpha$.
    Therefore $\succ(\alpha)$ and $\alpha$ are equinumerous.
    Consequently $|\succ(\alpha)| = |\alpha|$.
  \end{proof}

  \begin{proposition}[forthel,id=SET_THEORY_07_8700732632989696,printid]
    Every infinite cardinal is a limit ordinal.
  \end{proposition}
  \begin{proof}[forthel]
    Let $\kappa$ be an infinite cardinal.
    Suppose that $\kappa$ is not a limit ordinal.
    $\kappa \neq 0$.
    Hence $\kappa$ is a successor ordinal.
    Indeed $\kappa$ is an ordinal.
    Thus we can take an ordinal $\alpha$ such that $\kappa = \succ(\alpha)$.
    We have $\alpha > \kappa \geq \omega$.
    Hence $|\succ(\alpha)| = |\alpha|$.
    Thus $\alpha < |\kappa|$ and $\kappa$ is equinumerous to $\kappa$.
    Contradiction.
  \end{proof}
\end{smodule}
\end{document}
