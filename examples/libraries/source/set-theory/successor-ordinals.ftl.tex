\documentclass{naproche-library}
\libinput{set-theory-preamble}
\begin{document}
\begin{smodule}{successor-ordinals.ftl}
  \importmodule[libraries]{set-theory?ordinals.ftl}

  \symdef{succ}{\textrm{succ}}
  \symdef{pred}{\textrm{pred}}

  \section*{Successor Ordinals}

  \begin{definition}[forthel,id=SET_THEORY_02_8166925802668032]
    Let $\alpha$ be an ordinal.
    $\succ(\alpha) = \alpha \cup \set{\alpha}$.
  \end{definition}

  \begin{proposition}[forthel,id=SET_THEORY_02_1624410224066560]
    Let $\alpha$ be an ordinal.
    Then $\succ(\alpha)$ is an ordinal.
  \end{proposition}
  \begin{proof}[forthel]
    (1) $\succ(\alpha)$ is transitive. \newline
    Proof.
      Let $x \in \succ(\alpha)$ and $y \in x$.
      Then $x \in \alpha$ or $x = \alpha$.
      Hence $y \in \alpha$.
      Thus $y \in \succ(\alpha)$.
    Qed.

    (2) Every element of $\succ(\alpha)$ is transitive. \newline
    Proof.
      Let $x \in \succ(\alpha)$.
      Then $x \in \alpha$ or $x = \alpha$.
      Hence $x$ is transitive.
      Indeed $\alpha$ is transitive and every element of $\alpha$ is transitive.
    Qed.
  \end{proof}

  \begin{definition}[forthel,id=SET_THEORY_02_7129712109289472]
    A successor ordinal is an ordinal $\alpha$ such that $\alpha = \succ(\beta)$ for some ordinal $\beta$.
  \end{definition}

  \begin{proposition}[forthel,id=SET_THEORY_02_8651096763400192]
    Let $\alpha, \beta$ be ordinals.
    If $\succ(\alpha) = \succ(\beta)$ then $\alpha = \beta$.
  \end{proposition}
  \begin{proof}[forthel]
    Assume $\succ(\alpha) = \succ(\beta)$.

    (1) $\alpha \subseteq \beta$. \newline
    Proof.
      Let $\gamma \in \alpha$.
      Then $\gamma \in \alpha \cup \set{\alpha}
        = \succ(\alpha)
        = \succ(\beta)
        = \beta \cup \set{\beta}$.
      Hence $\gamma \in \beta$ or $\gamma = \beta$.
      Assume $\gamma = \beta$.
      Then $\beta \in \alpha$.
      Hence $\beta
        = (\beta \cup \set{\beta}) \setminus \set{\gamma}
        = (\alpha \cup \set{\alpha}) \setminus \set{\gamma}
        = (\alpha \setminus \set{\gamma}) \cup \set{\alpha}$.
      Therefore $\alpha \in \beta$.
      Consequently $\alpha \in \beta \in \alpha$.
      Contradiction.
    Qed.

    (2) $\beta \subseteq \alpha$. \newline
    Proof.
      Let $\gamma \in \beta$.
      Then $\gamma \in \beta \cup \set{\beta}
        = \succ(\beta)
        = \succ(\alpha)
        = \alpha \cup \set{\alpha}$.
      Hence $\gamma \in \alpha$ or $\gamma = \alpha$.
      Assume $\gamma = \alpha$.
      Then $\alpha \in \beta$.
      Hence $\alpha
        = (\alpha \cup \set{\alpha}) \setminus \set{\gamma}
        = (\beta \cup \set{\beta}) \setminus \set{\gamma}
        = (\beta \setminus \set{\gamma}) \cup \set{\beta}$.
      Therefore $\beta \in \alpha$.
      Consequently $\beta \in \alpha \in \beta$.
      Contradiction.
    Qed.
  \end{proof}

  \begin{definition}[forthel,id=SET_THEORY_02_735071524880384]
    Let $\alpha$ be a successor ordinal.
    $\pred(\alpha)$ is the ordinal $\beta$ such that $\alpha = \succ(\beta)$.
  \end{definition}
\end{smodule}
\end{document}
