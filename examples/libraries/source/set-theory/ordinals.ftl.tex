\documentclass{naproche-library}
\libinput{set-theory-preamble}
\begin{document}
\begin{smodule}{ordinals.ftl}
  \importmodule[libraries]{set-theory?transitive-classes.ftl}

  \symdef{Ord}{\textbf{Ord}}

  \section*{Ordinal Numbers}

  \begin{definition}[forthel,id=SET_THEORY_02_229593678086144,printid]
    An ordinal number is a transitive set $\alpha$ such that every element of $\alpha$ is a transitive set.

    Let an ordinal stand for an ordinal number.
  \end{definition}

  \begin{definition}[forthel,id=SET_THEORY_02_5852994258075648,printid]
    $\Ord$ is the class of all ordinals.
  \end{definition}

  \begin{proposition}[forthel,id=SET_THEORY_02_2358097091756032,printid]
    Let $\alpha$ be an ordinal.
    Then every element of $\alpha$ is an ordinal.
  \end{proposition}
  \begin{proof}[forthel]
    Let $x$ be an element of $\alpha$.
    Then $x$ is transitive.

    Let us show that every element of $x$ is a subset of $x$.
      Let $y$ be an element of $x$.
      Then $y$ is a subset of $x$.
      Let $z$ be an element of $y$.
      Every element of $y$ is an element of $x$.
      Hence $z$ is an element of $x$.
    End.

    Thus every element of $x$ is transitive.
    Therefore $x$ is an ordinal.
  \end{proof}
\end{smodule}
\end{document}
