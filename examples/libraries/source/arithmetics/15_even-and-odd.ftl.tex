\documentclass[10pt]{article}
\usepackage[english]{babel}
\usepackage{../../../../lib/tex/naproche}
% Top-level sections and proofs

\newtheoremstyle{foundations}% name of the style to be used
  {\dimexpr\ftlparskip+4pt\relax}% measure of space to leave above the theorem.
  % (We must add \ftlparskip since otherwise the name of the label would not fit
  % into the border drawn around the environment.)
  {4pt}% measure of space to leave below the theorem.
  {\normalfont}% name of font to use in the body of the theorem
  {0pt}% measure of space to indent
  {\bfseries}% name of head font
  {}% punctuation between head and body
  { }% space after theorem head; " " = normal inter-word space
  {\thmname{#1}\thmnumber{ #2}\thmnote{ (#3)}.}

\let\definition\relax
\let\enddefinition\relax
\let\signature\relax
\let\endsignature\relax
\let\axiom\relax
\let\endaxiom\relax
\let\theorem\relax
\let\endtheorem\relax
\let\proposition\relax
\let\endproposition\relax
\let\lemma\relax
\let\endlemma\relax
\let\corollary\relax
\let\endcorollary\relax

\theoremstyle{foundations}

\newmdtheoremenv[
  backgroundcolor=lightgray!10,
  linecolor=black,
  innertopmargin=1em
]{definition}{Definition}[section]
\newmdtheoremenv[
  backgroundcolor=lightgray!10,
  linecolor=black,
  innertopmargin=1em
]{signature}[definition]{Signature}
\newmdtheoremenv[
  backgroundcolor=lightgray!10,
  linecolor=black,
  innertopmargin=1em
]{axiom}[definition]{Axiom}
\newmdtheoremenv[
  backgroundcolor=lightgray!10,
  linecolor=black,
  innertopmargin=1em
]{theorem}[definition]{Theorem}
\newmdtheoremenv[
  backgroundcolor=lightgray!10,
  linecolor=black,
  innertopmargin=1em
]{proposition}[definition]{Proposition}
\newmdtheoremenv[
  backgroundcolor=lightgray!10,
  linecolor=black,
  innertopmargin=1em
]{lemma}[definition]{Lemma}
\newmdtheoremenv[
  backgroundcolor=lightgray!10,
  linecolor=black,
  innertopmargin=1em
]{corollary}[definition]{Corollary}


% Hide imports

\usepackage{scontents}
\newenvsc{imports}[store-env=imports,print-env=false]


% Labels of top-level sections

\renewcommand{\showlabelfont}{\scriptsize\ttfamily\color{gray}}


% Non-symbolic "\implies" and "\iff"

\renewcommand{\implies}{\quad \text{implies} \quad}
\renewcommand{\iff}{\quad \text{iff} \quad}

\usepackage{amssymb}
\usepackage{pgffor}

\newcommand{\id}{\textrm{id}}
\newcommand{\range}{\textrm{range}}
\newcommand{\const}[2]{\textrm{const}_{#1}^{#2}}
\newcommand{\symdiff}{\mathop{\triangle}}
\newcommand{\pow}{\mathcal{P}}
\newcommand{\onto}{\twoheadrightarrow}
\newcommand{\into}{\hookrightarrow}

% E.g. `\set{1, 2, 3, 4}` prints the numbers 1, 2, 3 and 4 as a comma
% separated list which is enclosed within curly braces.
\def\set#1{\ensuremath{
  \{\foreach[count=\i]\x in {#1}{
    \ifnum\i>1,\,\fi
    \x
  }\}
}}

\useackage{amssymb}

\newcommand{\Nat}{\mathbb{N}}
\newcommand{\Prime}{\mathbb{P}}
\renewcommand{\succ}{\textrm{succ}}
\newcommand{\pred}{\textrm{pred}}
\newcommand{\add}{\textrm{add}}
\newcommand{\mul}{\textrm{mul}}
\renewcommand{\exp}{\textrm{exp}}
\newcommand{\fac}{\textrm{fac}}
\renewcommand{\div}{\mathrel{\textrm{div}}}
\renewcommand{\mod}{\mathrel{\textrm{mod}}}

\begin{document}
  \begin{imports}
    \begin{forthel}
      %[prove off][check off]
      [readtex \path{libraries/source/arithmetics/13_exponentiation-and-divisibility.ftl.tex}]
      %[prove on][check on]
    \end{forthel}
  \end{imports}


  \subsection{Definition}

  \begin{forthel}
    \begin{definition}\printlabel{ARITHMETIC_15_4521358965847512}
      Let $n$ be a natural number.
      $n$ is even iff $n$ is divisible by $2$.
    \end{definition}
  \end{forthel}

  \begin{forthel}
    \begin{definition}\printlabel{ARITHMETIC_15_1023652125874596}
      Let $n$ be a natural number.
      $n$ is odd iff $n$ is not divisible by $2$.
    \end{definition}
  \end{forthel}

  \begin{forthel}
    \begin{proposition}\printlabel{ARITHMETIC_15_0236985458752156}
      Let $n$ be a natural number.
      $n$ is even iff $n = 2 \cdot m$ for some natural number $m$.
    \end{proposition}
    \begin{proof}
      Case $n$ is even.
        Then $2$ divides $n$.
        Hence $n = 2 \cdot m$ for some natural number $m$.
      End.

      Case $n = 2 \cdot m$ for some natural number $m$.
        Then $2$ divides $n$.
        Hence $2$ is even.
      End.
    \end{proof}
  \end{forthel}

  \begin{forthel}
    \begin{proposition}\printlabel{ARITHMETIC_15_1023512547854265}
      Let $n$ be a natural number.
      $n$ is odd iff $n = (2 \cdot m) + 1$ for some natural number $m$.
    \end{proposition}
    \begin{proof}
      Case $n$ is odd.
        Define $P = \class{n' \in \Nat | \text{if $n'$ is odd then $n' = (2 \cdot m) + 1$ for some natural number $m$}}$.

        (BASE CASE) $P$ contains $0$.
        Indeed $0$ is not odd.

        (INDUCTION STEP) For all natural numbers $n'$ we have $n' \in P \implies n' + 1 \in P$. \\
        Proof.
          Let $n'$ be a natural number.
          Assume $n' \in P$.

          Let us show that if $n' + 1$ is odd then $(n' + 1) = (2 \cdot m) + 1$ for some natural number $m$.
            Assume that $n' + 1$ is odd.

            Case $n'$ is even.
              Take a natural number $m$ such that $n' = 2 \cdot m$.
              Then $n' + 1 = (2 \cdot m) + 1$.
            End.

            Case $n'$ is odd.
              Take a natural number $m$ such that $n' = (2 \cdot m) + 1$.
              Then $n' + 1 =
              ((2 \cdot m) + 1) + 1 =
              (2 \cdot m) + (1 + 1) =
              (2 \cdot m) + 2 =
              2 \cdot (m + 1)$.
              Hence $2$ divides $n'$.
              Thus $n'$ is even.
              Contradiction.
            End.
          End.
        Qed.

        Then $P$ contains every natural number (by \cref{ARITHMETIC_03_647949900054528}).
      End.

      Case $n = (2 \cdot m) + 1$ for some natural number $m$.
        Consider a natural number $m$ such that $n = (2 \cdot m) + 1$.
        Assume that $n$ is even.
        Then we can take a natural number $k$ such that $n = 2 \cdot k$.
        Then we have $2 \cdot k = (2 \cdot m) + 1$.
        Hence $2$ divides $(2 \cdot m) + 1$.
        Thus $2$ divides $1$.
        Indeed $2$ divides $2 \cdot m$.
        Contradiction.
      End.
    \end{proof}
  \end{forthel}

  \begin{forthel}
    \begin{proposition}\printlabel{ARITHMETIC_15_1023652154254789}
      Let $n$ be a natural number.
      $n$ is odd iff $n = (2 \cdot m) - 1$ for some positive natural number $m$.
    \end{proposition}
    \begin{proof}
      Case $n$ is odd.
        Consider a natural number $k$ such that $n = (2 \cdot k) + 1$.
        Take $m = k + 1$.
        Then $n =
        (2 \cdot k) + 1 =
        (2 \cdot (k + 0)) + 1 =
        (2 \cdot (k + (1 - 1))) + 1 =
        (2 \cdot ((k + 1) - 1)) + 1 =
        (2 \cdot (m - 1)) + 1 =
        ((2 \cdot m) - (2 \cdot 1)) + 1 =
        ((2 \cdot m) - 2) + 1 =
        (2 \cdot m) - 1$.
      End.

      Case $n = (2 \cdot m) - 1$ for some positive natural number $m$.
        Consider a natural number $m$ such that $n = (2 \cdot m) - 1$.
        Take $k = m - 1$.
        Then $n =
        (2 \cdot m) - 1 =
        (2 \cdot (m + 0)) - 1 =
        (2 \cdot (m + (1 - 1))) - 1 =
        (2 \cdot ((m + 1) - 1)) - 1 =
        (2 \cdot ((m - 1) + 1)) - 1 =
        (2 \cdot (k + 1)) - 1 =
        ((2 \cdot k) + (2 \cdot 1)) - 1 =
        ((2 \cdot k) + 2) - 1 =
        (2 \cdot k) + (2 - 1) =
        (2 \cdot k) + 1$.
        Hence $n$ is odd.
      End.
    \end{proof}
  \end{forthel}


  \subsection{Addition of even and odd numbers}

  \begin{forthel}
    \begin{proposition}\printlabel{ARITHMETIC_15_7845441256365256}
      Let $n, m$ be natural numbers.
      Assume that $n$ and $m$ are even.
      Then $n + m$ is even.
    \end{proposition}
    \begin{proof}
      Take natural numbers $k, l$ such that $n = 2 \cdot k$ and $m = 2 \cdot l$.
      Then $n + m =
      (2 \cdot k) + (2 \cdot l) =
      2 \cdot (k + l)$.
      Hence $n + m$ is even.
    \end{proof}
  \end{forthel}

  \begin{forthel}
    \begin{proposition}\printlabel{ARITHMETIC_15_1023655256985478}
      Let $n, m$ be natural numbers.
      Assume that $n$ is even and $m$ is odd.
      Then $n + m$ is odd.
    \end{proposition}
    \begin{proof}
      Take natural numbers $k, l$ such that $n = 2 \cdot k$ and $m = (2 \cdot l) + 1$.
      Then $n + m =
      (2 \cdot k) + ((2 \cdot l) + 1) =
      ((2 \cdot k) + (2 \cdot l)) + 1 =
      (2 \cdot (k + l)) + 1$.
      Hence $n + m$ is odd.
    \end{proof}
  \end{forthel}

  \begin{forthel}
    \begin{corollary}\printlabel{ARITHMETIC_15_0125412589658745}
      Let $n, m$ be natural numbers.
      Assume that $n$ is odd and $m$ is even.
      Then $n + m$ is odd.
    \end{corollary}
  \end{forthel}

  \begin{forthel}
    \begin{proposition}\printlabel{ARITHMETIC_15_1023659854785412}
      Let $n, m$ be natural numbers.
      Assume that $n$ and $m$ are odd.
      Then $n + m$ is even.
    \end{proposition}
    \begin{proof}
      Take natural numbers $k, l$ such that $n = (2 \cdot k) + 1$ and $m = (2 \cdot l) + 1$.
      Then $n + m =
      ((2 \cdot k) + 1) + ((2 \cdot l) + 1) =
      (((2 \cdot k) + 1) + (2 \cdot l)) + 1 =
      ((2 \cdot k) + (1 + (2 \cdot l))) + 1 =
      ((2 \cdot k) + ((2 \cdot l) + 1)) + 1 =
      (((2 \cdot k) + (2 \cdot l)) + 1) + 1 =
      ((2 \cdot k) + (2 \cdot l)) + (1 + 1) =
      ((2 \cdot k) + (2 \cdot l)) + 2 =
      (2 \cdot (k + l)) + 2 =
      2 \cdot ((k + l) + 1)$.
      Hence $n + m$ is even.
    \end{proof}
  \end{forthel}


  \subsection{Subtraction of even and odd numbers}

  \begin{forthel}
    \begin{proposition}\printlabel{ARITHMETIC_15_8748569852145203}
      Let $n, m$ be natural numbers such that $n \geq m$.
      Assume that $n, m$ are even.
      Then $n - m$ is even.
    \end{proposition}
    \begin{proof}
      Take natural numbers $k, l$ such that $n = 2 \cdot k$ and $m = 2 \cdot l$.
      Then $k \geq l$.
      Hence $n - m =
      (2 \cdot k) - (2 \cdot l) =
      2 \cdot (k - l)$.
      Thus $n - m$ is even.
    \end{proof}
  \end{forthel}

  \begin{forthel}
    \begin{proposition}\printlabel{ARITHMETIC_15_0125412036589958}
      Let $n, m$ be natural numbers such that $n \geq m$.
      Assume that $n$ is even and $m$ is odd.
      Then $n - m$ is odd.
    \end{proposition}
    \begin{proof}
      Take natural numbers $k, l$ such that $n = 2 \cdot k$ and $m = (2 \cdot l) + 1$.
      Then $k \geq l$ and $2 \cdot (k - l) \geq 1$.
      Hence $n - m =
      (2 \cdot k) - ((2 \cdot l) + 1) =
      ((2 \cdot k) - (2 \cdot l)) - 1 =
      (2 \cdot (k - l)) - 1$.
      Thus $n - m$ is odd.
    \end{proof}
  \end{forthel}

  \begin{forthel}
    \begin{corollary}\printlabel{ARITHMETIC_15_1021458745896523}
      Let $n, m$ be natural numbers such that $n \geq m$.
      Assume that $n$ is odd and $m$ is even.
      Then $n - m$ is odd.
    \end{corollary}
    \begin{proof}
      Take natural numbers $k, l$ such that $n = (2 \cdot k) + 1$ and $m = 2 \cdot l$.
      Then $k \geq l$.
      Hence $n - m =
      ((2 \cdot k) + 1) - (2 \cdot l) =
      (1 + (2 \cdot k)) - (2 \cdot l) =
      1 + ((2 \cdot k) - (2 \cdot l)) =
      ((2 \cdot k) - (2 \cdot l)) + 1 =
      (2 \cdot (k - l)) + 1$.
      Indeed $((2 \cdot k) - (2 \cdot l)) = 2 \cdot (k - l)$.
      Thus $n - m$ is odd.
    \end{proof}
  \end{forthel}

  \begin{forthel}
    \begin{proposition}\printlabel{ARITHMETIC_15_0125478854587412}
      Let $n, m$ be natural numbers such that $n \geq m$.
      Assume that $n, m$ are odd.
      Then $n - m$ is even.
    \end{proposition}
    \begin{proof}
      Take natural numbers $k, l$ such that $n = (2 \cdot k) + 1$ and $m = (2 \cdot l) + 1$.
      Then $k \geq l$.
      Indeed $2 \cdot k \geq 2 \cdot l$.
      Hence $1 + (2 \cdot k) \geq 2 \cdot l$.
      Thus $n - m =
      ((2 \cdot k) + 1) - ((2 \cdot l) + 1) =
      ((1 + (2 \cdot k)) - (2 \cdot l)) - 1 =
      (1 + ((2 \cdot k) - (2 \cdot l))) - 1 =
      (1 + (2 \cdot (k - l))) - 1 =
      ((2 \cdot (k - l)) + 1) - 1 =
      2 \cdot (k - l)$.
      Indeed $((2 \cdot k) - (2 \cdot l)) = 2 \cdot (k - l)$.
      Therefore $n - m$ is even.
    \end{proof}
  \end{forthel}


  \subsection{Multiplication of even and odd numbers}

  \begin{forthel}
    \begin{proposition}\printlabel{ARITHMETIC_15_0125698547589652}
      Let $n, m$ be natural numbers.
      Assume that $n$ is even or $n$ is even.
      Then $n \cdot m$ is even.
    \end{proposition}
    \begin{proof}
      Case $n$ is even.
        Take a natural number $k$ such that $n = 2 \cdot k$.
        Then $n \cdot m =
        (2 \cdot k) \cdot m =
        2 \cdot (k \cdot m)$.
        Hence $n \cdot m$ is even.
      End.

      Case $m$ is even.
        Take a natural number $l$ such that $m = 2 \cdot l$.
        Then $n \cdot m =
        n \cdot (2 \cdot l) =
        (n \cdot 2) \cdot l =
        (2 \cdot n) \cdot l =
        2 \cdot (n \cdot l)$.
        Hence $n \cdot m$ is even.
      End.
    \end{proof}
  \end{forthel}

  \begin{forthel}
    \begin{proposition}\printlabel{ARITHMETIC_15_0236596587452145}
      Let $n, m$ be natural numbers.
      Assume that $n$ and $m$ are odd.
      Then $n \cdot m$ is odd.
    \end{proposition}
    \begin{proof}
      Take natural numbers $k, l$ such that $n = (2 \cdot k) + 1$ and $m = (2 \cdot l) + 1$.
      Then $n \cdot m =
      ((2 \cdot k) + 1) \cdot m =
      ((2 \cdot k) \cdot m) + (1 \cdot m) =
      ((2 \cdot k) \cdot m) + m =
      (2 \cdot (k \cdot m)) + m$.
      $2 \cdot (k \cdot m)$ is even and $m$ is odd.
      Hence $(2 \cdot (k \cdot m)) + m$ is odd.
      Therefore $n \cdot m$ is odd.
    \end{proof}
  \end{forthel}


  \subsection{Exponentiation of even and odd numbers}

  \begin{forthel}
    \begin{proposition}\printlabel{ARITHMETIC_15_1023659658745214}
      Let $n, m$ be natural numbers such that $m > 0$.
      Assume that $n$ is even.
      Then $n^{m}$ is even.
    \end{proposition}
    \begin{proof}
      Take a natural number $k$ such that $n = 2 \cdot k$.
      Consider a natural number $m'$ such that $m = m' + 1$.
      Then $n^{m} =
      (2 \cdot k)^{m} =
      (2^{m} \cdot (k^{m})) =
      (2^{m' + 1} \cdot (k^{m})) =
      (2^{m'} \cdot 2) \cdot (k^{m}) =
      (2 \cdot 2^{m'}) \cdot (k^{m}) =
      2 \cdot (2^{m'} \cdot (k^{m}))$.
      Hence $n^{m}$ is even.
    \end{proof}
  \end{forthel}

  \begin{forthel}
    \begin{proposition}\printlabel{ARITHMETIC_15_0021200236556985}
      Let $n, m$ be natural numbers.
      Assume that $n$ is odd.
      Then $n^{m}$ is odd.
    \end{proposition}
    \begin{proof}
      Define $P = \class{m' \in \Nat | \text{$n^{m'}$ is odd}}$.

      (BASE CASE) $P$ contains $0$.
      Indeed $n^{0} = 1$ and $1$ is odd.

      (INDUCTION STEP) For all natural numbers $m'$ we have $m' \in P \implies m' + 1 \in P$. \\
      Proof.
        Let $m'$ be a natural number.
        Assume $m' \in P$.
        We have $n^{m' + 1} = n^{m'} \cdot n$.
        $n^{m'}$ is odd.
        Hence we can take a natural number $k$ such that $n^{m'} = (2 \cdot k) + 1$.
        Then $n^{m' + 1} =
        ((2 \cdot k) + 1) \cdot n =
        ((2 \cdot k) \cdot n) + (1 \cdot n) =
        ((2 \cdot k) \cdot n) + n =
        (2 \cdot (k \cdot n)) + n$.
        $2 \cdot (k \cdot n)$ is even and $n$ is odd.
        Thus $(2 \cdot (k \cdot n)) + n$ is odd.
        Therefore $n^{m' + 1}$ is odd.
        Consequently $m' + 1 \in P$.
      Qed.

      Hence $P$ contains every natural number (by \cref{ARITHMETIC_03_647949900054528}).
      Thus $n^{m}$ is odd.
    \end{proof}
  \end{forthel}
\end{document}