\documentclass[10pt]{article}
\usepackage[english]{babel}
\usepackage{../../../meta-inf/lib/naproche}
% Top-level sections and proofs

\newtheoremstyle{foundations}% name of the style to be used
  {\dimexpr\ftlparskip+4pt\relax}% measure of space to leave above the theorem.
  % (We must add \ftlparskip since otherwise the name of the label would not fit
  % into the border drawn around the environment.)
  {4pt}% measure of space to leave below the theorem.
  {\normalfont}% name of font to use in the body of the theorem
  {0pt}% measure of space to indent
  {\bfseries}% name of head font
  {}% punctuation between head and body
  { }% space after theorem head; " " = normal inter-word space
  {\thmname{#1}\thmnumber{ #2}\thmnote{ (#3)}.}

\let\definition\relax
\let\enddefinition\relax
\let\signature\relax
\let\endsignature\relax
\let\axiom\relax
\let\endaxiom\relax
\let\theorem\relax
\let\endtheorem\relax
\let\proposition\relax
\let\endproposition\relax
\let\lemma\relax
\let\endlemma\relax
\let\corollary\relax
\let\endcorollary\relax

\theoremstyle{foundations}

\newmdtheoremenv[
  backgroundcolor=lightgray!10,
  linecolor=black,
  innertopmargin=1em
]{definition}{Definition}[section]
\newmdtheoremenv[
  backgroundcolor=lightgray!10,
  linecolor=black,
  innertopmargin=1em
]{signature}[definition]{Signature}
\newmdtheoremenv[
  backgroundcolor=lightgray!10,
  linecolor=black,
  innertopmargin=1em
]{axiom}[definition]{Axiom}
\newmdtheoremenv[
  backgroundcolor=lightgray!10,
  linecolor=black,
  innertopmargin=1em
]{theorem}[definition]{Theorem}
\newmdtheoremenv[
  backgroundcolor=lightgray!10,
  linecolor=black,
  innertopmargin=1em
]{proposition}[definition]{Proposition}
\newmdtheoremenv[
  backgroundcolor=lightgray!10,
  linecolor=black,
  innertopmargin=1em
]{lemma}[definition]{Lemma}
\newmdtheoremenv[
  backgroundcolor=lightgray!10,
  linecolor=black,
  innertopmargin=1em
]{corollary}[definition]{Corollary}


% Hide imports

\usepackage{scontents}
\newenvsc{imports}[store-env=imports,print-env=false]


% Labels of top-level sections

\renewcommand{\showlabelfont}{\scriptsize\ttfamily\color{gray}}


% Non-symbolic "\implies" and "\iff"

\renewcommand{\implies}{\quad \text{implies} \quad}
\renewcommand{\iff}{\quad \text{iff} \quad}

\usepackage{amssymb}
\usepackage{pgffor}

\newcommand{\id}{\textrm{id}}
\newcommand{\range}{\textrm{range}}
\newcommand{\const}[2]{\textrm{const}_{#1}^{#2}}
\newcommand{\symdiff}{\mathop{\triangle}}
\newcommand{\pow}{\mathcal{P}}
\newcommand{\onto}{\twoheadrightarrow}
\newcommand{\into}{\hookrightarrow}

% E.g. `\set{1, 2, 3, 4}` prints the numbers 1, 2, 3 and 4 as a comma
% separated list which is enclosed within curly braces.
\def\set#1{\ensuremath{
  \{\foreach[count=\i]\x in {#1}{
    \ifnum\i>1,\,\fi
    \x
  }\}
}}

\useackage{amssymb}

\newcommand{\Nat}{\mathbb{N}}
\newcommand{\Prime}{\mathbb{P}}
\renewcommand{\succ}{\textrm{succ}}
\newcommand{\pred}{\textrm{pred}}
\newcommand{\add}{\textrm{add}}
\newcommand{\mul}{\textrm{mul}}
\renewcommand{\exp}{\textrm{exp}}
\newcommand{\fac}{\textrm{fac}}
\renewcommand{\div}{\mathrel{\textrm{div}}}
\renewcommand{\mod}{\mathrel{\textrm{mod}}}

\begin{document}
  \begin{imports}
    \begin{forthel}
      %[prove off][check off]
      [readtex \path{libraries/source/arithmetics/divisibility.ftl.tex}]
      %[prove on][check on]
    \end{forthel}
  \end{imports}


  \section*{Division}

  \begin{forthel}
    \begin{definition}[id=ARITHMETIC_14_2313654268297915,printid]
      Let $n$ be a natural number and $m$ be a nonzero divisor of $n$.
      $\frac{n}{m}$ is the natural number $k$ such that $k \cdot m = n$.
    \end{definition}

    Let the quotient of $n$ and $m$ stand for $\frac{n}{m}$.
  \end{forthel}

  \begin{forthel}
    \begin{proposition}[id=ARITHMETIC_14_0843793254698710,printid]
      Let $n$ be a natural number.
      $\frac{n}{1} = n$.
    \end{proposition}
    \begin{proof}
      We have $\frac{n}{1}
        = \frac{n}{1} \cdot 1
        = n$.
    \end{proof}
  \end{forthel}

  \begin{forthel}
    \begin{proposition}[id=ARITHMETIC_14_1254235698632545,printid]
      Let $n$ be a natural number and $m$ be a nonzero divisor of $n$.
      Then $\frac{n}{m} = 0$ iff $n = 0$.
    \end{proposition}
    \begin{proof}
      Case $\frac{n}{m} = 0$.
        Then $n
          = \frac{n}{m} \cdot m
          = 0 \cdot m
          = 0$.
      End.

      Case $n = 0$.
        Then $\frac{n}{m} \cdot m
          = n
          = 0$.
        Hence $\frac{n}{m} = 0$ or $m = 0$.
        $m$ is nonzero.
        Thus $\frac{n}{m} = 0$.
      End.
    \end{proof}
  \end{forthel}

  \begin{forthel}
    \begin{proposition}[id=ARITHMETIC_14_5137961454123875,printid]
      Let $n, m, k$ be natural numbers such that $k$ is nonzero.
      Assume $k \mid n,m$.
      Then \[\frac{n + m}{k} = \frac{n}{k} + \frac{m}{k}.\]
    \end{proposition}
    \begin{proof}
      We have $\frac{n + m}{k} \cdot k = n + m$ and $\frac{n}{k} \cdot k = n$ and $\frac{m}{k} \cdot k = m$.
      Hence
      \[  \frac{n + m}{k} \cdot k
          = \left( \frac{n}{k} \cdot k \right) + \left( \frac{m}{k} \cdot k \right)
          = \left( \frac{n}{k} + \frac{m}{k} \right) \cdot k. \]
      Thus $\frac{n + m}{k} = \frac{n}{k} + \frac{m}{k}$.
    \end{proof}
  \end{forthel}

  \begin{forthel}
    \begin{proposition}[id=ARITHMETIC_14_1203565412058488,printid]
      Let $n, m$ be natural numbers and $k$ be a nonzero divisor of $m$.
      Then \[\frac{n \cdot m}{k} = n \cdot \frac{m}{k}.\]
    \end{proposition}
    \begin{proof}
      We have $\frac{n \cdot m}{k} \cdot k = n \cdot m$ and $\frac{m}{k} \cdot k = m$.
      Hence
      \[  \frac{n \cdot m}{k} \cdot k
          = n \cdot \left( \frac{m}{k} \cdot k \right)
          = \left( n \cdot \frac{m}{k} \right) \cdot k. \]
      Thus $\frac{n \cdot m}{k} = n \cdot \frac{m}{k}$.
    \end{proof}
  \end{forthel}

  \begin{forthel}
    \begin{corollary}[id=ARITHMETIC_14_7985412544563256,printid]
      Let $n, m$ be natural numbers and $k$ be a nonzero divisor of $m$.
      Then \[\frac{n \cdot m}{k} = \frac{m}{k} \cdot n.\]
    \end{corollary}
  \end{forthel}

  \begin{forthel}
    \begin{corollary}[id=ARITHMETIC_14_5446124202158602,printid]
      Let $n, m$ be natural numbers and $k$ be a nonzero divisor of $n$.
      Then \[\frac{n \cdot m}{k} = \frac{n}{k} \cdot m.\]
    \end{corollary}
  \end{forthel}

  \begin{forthel}
    \begin{corollary}[id=ARITHMETIC_14_7751120023654896,printid]
      Let $n, m$ be natural numbers and $k$ be a nonzero divisor of $n$.
      Then \[\frac{n \cdot m}{k} = m \cdot \frac{n}{k}.\]
    \end{corollary}
  \end{forthel}

  \begin{forthel}
    \begin{proposition}[id=ARITHMETIC_14_0531254868745988,printid]
      Let $n, k$ be natural numbers such that $k$ be nonzero and $m$ be a nonzero divisor of $n$.
      Then \[\frac{n \cdot k}{m \cdot k} = \frac{n}{m}.\]
    \end{proposition}
    \begin{proof}
      We have $\frac{n \cdot k}{m \cdot k} \cdot (m \cdot k) = n \cdot k$.
      Hence
      \[  \left( \frac{n \cdot k}{m \cdot k} \cdot m \right) \cdot k
          = \frac{n \cdot k}{m \cdot k} \cdot (m \cdot k)
          = n \cdot k. \]
      Thus $\frac{n \cdot k}{m \cdot k} \cdot m = n$.
      Therefore $\frac{n}{m} = \frac{n \cdot k}{m \cdot k}$.
    \end{proof}
  \end{forthel}

  \begin{forthel}
    \begin{corollary}[id=ARITHMETIC_14_7954254458622034,printid]
      Let $n, k$ be natural numbers such that $k$ be nonzero and $m$ be a nonzero divisor of $n$.
      Then \[\frac{k \cdot n}{k \cdot m} = \frac{n}{m}.\]
    \end{corollary}
  \end{forthel}
\end{document}