\documentclass{naproche-library}
\libinput{arithmetics-preamble}
\begin{document}
\begin{smodule}{division.ftl}
  \importmodule[libraries]{arithmetics?divisibility.ftl}

  \section*{Division}

  \begin{definition}[forthel,id=ARITHMETIC_14_2313654268297915,printid]
    Let $n$ be a natural number and $m$ be a nonzero divisor of $n$.
    $\frac{n}{m}$ is the natural number $k$ such that $k \cdot m = n$.
  \end{definition}

  Let the quotient of $n$ and $m$ stand for $\frac{n}{m}$.

  \begin{proposition}[forthel,id=ARITHMETIC_14_0843793254698710,printid]
    Let $n$ be a natural number.
    $\frac{n}{1} = n$.
  \end{proposition}
  \begin{proof}[forthel]
    We have $\frac{n}{1}
      = \frac{n}{1} \cdot 1
      = n$.
  \end{proof}

  \begin{proposition}[forthel,id=ARITHMETIC_14_1254235698632545,printid]
    Let $n$ be a natural number and $m$ be a nonzero divisor of $n$.
    Then $\frac{n}{m} = 0$ iff $n = 0$.
  \end{proposition}
  \begin{proof}[forthel]
    Case $\frac{n}{m} = 0$.
      Then $n
        = \frac{n}{m} \cdot m
        = 0 \cdot m
        = 0$.
    End.

    Case $n = 0$.
      Then $\frac{n}{m} \cdot m
        = n
        = 0$.
      Hence $\frac{n}{m} = 0$ or $m = 0$.
      $m$ is nonzero.
      Thus $\frac{n}{m} = 0$.
    End.
  \end{proof}

  \begin{proposition}[forthel,id=ARITHMETIC_14_5137961454123875,printid]
    Let $n, m, k$ be natural numbers such that $k$ is nonzero.
    Assume $k \mid n,m$.
    Then \[\frac{n + m}{k} = \frac{n}{k} + \frac{m}{k}.\]
  \end{proposition}
  \begin{proof}[forthel]
    We have $\frac{n + m}{k} \cdot k = n + m$ and $\frac{n}{k} \cdot k = n$ and $\frac{m}{k} \cdot k = m$.
    Hence
    \[  \frac{n + m}{k} \cdot k
        = \left( \frac{n}{k} \cdot k \right) + \left( \frac{m}{k} \cdot k \right)
        = \left( \frac{n}{k} + \frac{m}{k} \right) \cdot k. \]
    Thus $\frac{n + m}{k} = \frac{n}{k} + \frac{m}{k}$.
  \end{proof}

  \begin{proposition}[forthel,id=ARITHMETIC_14_1203565412058488,printid]
    Let $n, m$ be natural numbers and $k$ be a nonzero divisor of $m$.
    Then \[\frac{n \cdot m}{k} = n \cdot \frac{m}{k}.\]
  \end{proposition}
  \begin{proof}[forthel]
    We have $\frac{n \cdot m}{k} \cdot k = n \cdot m$ and $\frac{m}{k} \cdot k = m$.
    Hence
    \[  \frac{n \cdot m}{k} \cdot k
        = n \cdot \left( \frac{m}{k} \cdot k \right)
        = \left( n \cdot \frac{m}{k} \right) \cdot k. \]
    Thus $\frac{n \cdot m}{k} = n \cdot \frac{m}{k}$.
  \end{proof}

  \begin{corollary}[forthel,id=ARITHMETIC_14_7985412544563256,printid]
    Let $n, m$ be natural numbers and $k$ be a nonzero divisor of $m$.
    Then \[\frac{n \cdot m}{k} = \frac{m}{k} \cdot n.\]
  \end{corollary}

  \begin{corollary}[forthel,id=ARITHMETIC_14_5446124202158602,printid]
    Let $n, m$ be natural numbers and $k$ be a nonzero divisor of $n$.
    Then \[\frac{n \cdot m}{k} = \frac{n}{k} \cdot m.\]
  \end{corollary}

  \begin{corollary}[forthel,id=ARITHMETIC_14_7751120023654896,printid]
    Let $n, m$ be natural numbers and $k$ be a nonzero divisor of $n$.
    Then \[\frac{n \cdot m}{k} = m \cdot \frac{n}{k}.\]
  \end{corollary}

  \begin{proposition}[forthel,id=ARITHMETIC_14_0531254868745988,printid]
    Let $n, k$ be natural numbers such that $k$ be nonzero and $m$ be a nonzero divisor of $n$.
    Then \[\frac{n \cdot k}{m \cdot k} = \frac{n}{m}.\]
  \end{proposition}
  \begin{proof}[forthel]
    We have $\frac{n \cdot k}{m \cdot k} \cdot (m \cdot k) = n \cdot k$.
    Hence
    \[  \left( \frac{n \cdot k}{m \cdot k} \cdot m \right) \cdot k
        = \frac{n \cdot k}{m \cdot k} \cdot (m \cdot k)
        = n \cdot k. \]
    Thus $\frac{n \cdot k}{m \cdot k} \cdot m = n$.
    Therefore $\frac{n}{m} = \frac{n \cdot k}{m \cdot k}$.
  \end{proof}

  \begin{corollary}[forthel,id=ARITHMETIC_14_7954254458622034,printid]
    Let $n, k$ be natural numbers such that $k$ be nonzero and $m$ be a nonzero divisor of $n$.
    Then \[\frac{k \cdot n}{k \cdot m} = \frac{n}{m}.\]
  \end{corollary}
\end{smodule}
\end{document}