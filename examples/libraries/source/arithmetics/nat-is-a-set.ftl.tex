\documentclass{naproche-library}
\libinput{arithmetics-preamble}
\begin{document}
\begin{smodule}{nat-is-a-set.ftl}
  \importmodule[libraries]{arithmetics?natural-numbers.ftl}
  \importmodule[libraries]{set-theory/axioms?infinity.ftl}

  \section*{The Natural Numbers Form a Set}

  \begin{proposition}[forthel,id=ARITHMETIC_07_4685510236547454,printid]
    $\Nat$ is a set.
  \end{proposition}
  \begin{proof}[forthel]
    Define $f(n) = n + 1$ for $n \in \Nat$.
    Then $f$ is a map from $\Nat$ to $\Nat$.
    Hence we can take a subset $X$ of $\Nat$ that is inductive regarding
    $0$ and $f$.
    Then $0 \in X$ and for all $n \in X$ we have $n + 1 \in X$.
    Hence $X$ contains every natural number.
    Thus we have $\Nat \subseteq X$ and $X \subseteq \Nat$.
    Therefore $\Nat = X$.
    Consequently $\Nat$ is a set.
  \end{proof}
\end{smodule}
\end{document}
