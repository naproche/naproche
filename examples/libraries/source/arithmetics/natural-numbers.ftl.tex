\documentclass{naproche-library}
\libinput{arithmetics-preamble}
\usepackage{amssymb}
\begin{document}
\begin{smodule}{natural-numbers.ftl}
  \importmodule[libraries]{foundations?classes.ftl}

  \symdef{Nat}{\mathbb N}

  \section*{Natural Numbers}

  \subsection*{The Language of Natural Number Arithmetic}

  \begin{signature}[forthel,id=ARITHMETIC_01_3074681254969344]
    A natural number is an object.
  \end{signature}

  \begin{definition}[forthel,id=ARITHMETIC_01_7367148418629632]
    $\Nat$ is the class of natural numbers.
  \end{definition}

  \begin{signature}[forthel,id=ARITHMETIC_01_1567933815848960]
    Let $n, m$ be natural numbers.
    $n + m$ is a natural number.

    Let the sum of $n$ and $m$ stand for $n + m$.
  \end{signature}

  \begin{signature}[forthel,id=ARITHMETIC_01_7633304715001856]
    $0$ is a natural number.

    Let zero stand for $0$.
    Let $n$ is nonzero stand for $n \neq 0$.
  \end{signature}

  \begin{signature}[forthel,id=ARITHMETIC_01_1200254566985232]
    $1$ is a natural number.

    Let one stand for $1$.
    Let the direct successor of $n$ stand for $n + 1$.
  \end{signature}

  \begin{definition}[forthel,id=ARITHMETIC_01_4584236572999680]
    $2 = 1 + 1$.

    Let two stand for $2$.
  \end{definition}

  \begin{definition}[forthel,id=ARITHMETIC_01_3836725109456896]
    $3 = 2 + 1$.

    Let three stand for $3$.
  \end{definition}

  \begin{definition}[forthel,id=ARITHMETIC_01_1709884968009728]
    $4 = 3 + 1$.

    Let four stand for $4$.
  \end{definition}

  \begin{definition}[forthel,id=ARITHMETIC_01_6734726333202432]
    $5 = 4 + 1$.

    Let five stand for $5$.
  \end{definition}

  \begin{definition}[forthel,id=ARITHMETIC_01_949139189792768]
    $6 = 5 + 1$.

    Let six stand for $6$.
  \end{definition}

  \begin{definition}[forthel,id=ARITHMETIC_01_7245471749767168]
    $7 = 6 + 1$.

    Let seven stand for $7$.
  \end{definition}

  \begin{definition}[forthel,id=ARITHMETIC_01_5658172888973312]
    $8 = 7 + 1$.

    Let eight stand for $8$.
  \end{definition}

  \begin{definition}[forthel,id=ARITHMETIC_01_7371844250238976]
    $9 = 8 + 1$.

    Let nine stand for $9$.
  \end{definition}


  \subsection*{The Axioms of Natural Number Arithmetic}

  \begin{axiom}[forthel,id=ARITHMETIC_01_3604163883696128]
    Let $n, m$ be natural numbers.
    If $n + 1 = m + 1$ then $n = m$.
  \end{axiom}

  \begin{axiom}[forthel,id=ARITHMETIC_01_4454289938317312]
    There exists no natural number $n$ such that $n + 1 = 0$.
  \end{axiom}

  \begin{axiom}[forthel,title=Induction,id=ARITHMETIC_01_4764664342773760]
    Let $\Phi$ be a class.
    Assume $0 \in \Phi$ and for all natural numbers $n$ if $n \in \Phi$ then
    $n + 1 \in \Phi$.
    Then $\Phi$ contains every natural number.
  \end{axiom}

  \begin{axiom}[forthel,id=ARITHMETIC_03_5962332515874540]
    Then $1 = 0 + 1$.
  \end{axiom}

  \begin{axiom}[forthel,id=ARITHMETIC_03_4827955356237824]
    Let $n$ be a natural number.
    Then $n + 0 = n$.
  \end{axiom}

  \begin{axiom}[forthel,id=ARITHMETIC_03_4427935386238924]
    Let $n, m$ be natural numbers.
    Then $n + (m + 1) = (n + m) + 1$.
  \end{axiom}


  \subsection*{Immediate Consequences of the Axioms}

  \begin{proposition}[forthel,id=ARITHMETIC_01_4966080109871104]
    Let $n$ be a natural number.
    Then $n = 0$ or $n = m + 1$ for some natural number $m$.
  \end{proposition}
  \begin{proof}[forthel]
    Define $\Phi = \{ n' \in \Nat \mid n' = 0$ or $n' = m' + 1$ for some natural number $m' \}$.
    $0 \in \Phi$ and for all $n' \in  \Phi$ we have $n' + 1 \in \Phi$.
    Hence every natural number is contained in $\Phi$.
    Thus $n = 0$ or $n = m + 1$ for some natural number $m$.
  \end{proof}

  \begin{proposition}[forthel,id=ARITHMETIC_01_5996049267163136]
    Let $n$ be a natural number.
    Then $n \neq n + 1$.
  \end{proposition}
  \begin{proof}[forthel]
    Define $\Phi = \{ n' \in \Nat \mid n' \neq n' + 1 \}$.

    (1) $0$ belongs to $\Phi$.

    (2) For all $n' \in \Phi$ we have $n' + 1 \in \Phi$. \newline
    Proof.
      Let $n' \in \Phi$.
      Then $n' \neq n' + 1$.
      If $n' + 1 = (n' + 1) + 1$ then $n' = n' + 1$.
      Thus it is wrong that $n' + 1 = (n' + 1) + 1$.
      Hence $n' + 1 \in \Phi$.
    Qed.

    Therefore every natural number is an element of $\Phi$.
    Consequently $n \neq n + 1$.
  \end{proof}


  \subsection*{Computation Laws for Addition}

  \subsubsection*{Associativity}

  \begin{proposition}[forthel,id=ARITHMETIC_03_3235893452210176]
    Let $n, m, k$ be natural numbers.
    Then $n + (m + k) = (n + m) + k$.
  \end{proposition}
  \begin{proof}[forthel]
    Define $\Phi = \{ k' \in \Nat \mid n + (m + k') = (n + m) + k' \}$.

    (1) $0$ is contained in $\Phi$.
    Indeed $n + (m + 0) = n + m = (n + m) + 0$.

    (2) For all $k' \in \Phi$ we have $k' + 1 \in \Phi$. \newline
    Proof.
      Let $k' \in \Phi$.
      Then $n + (m + k') = (n + m) + k'$.
      Hence
      \[  n + (m + (k' + 1))        \]
      \[    = n + ((m + k') + 1)    \]
      \[    = (n + (m + k')) + 1    \]
      \[    = ((n + m) + k') + 1    \]
      \[    = (n + m) + (k' + 1).   \]
      Thus $k' + 1 \in \Phi$.
    Qed.

    Thus every natural number is an element of $\Phi$.
    Therefore $n + (m + k) = (n + m) + k$.
  \end{proof}


  \subsubsection*{Commutativity}

  \begin{proposition}[forthel,id=ARITHMETIC_03_4029553232052224]
    Let $n, m$ be natural numbers.
    Then $n + m = m + n$.
  \end{proposition}
  \begin{proof}[forthel]
    Define $\Phi = \{ m' \in \Nat \mid n + m' = m' + n \}$.

    (1) $0$ is an element of $\Phi$. \newline
    Proof.
      Define $\Psi = \{ n' \in \Nat \mid n' + 0 = 0 + n' \}$.

      (1a) $0$ belongs to $\Psi$.

      (1b) For all $n' \in \Psi$ we have $n' + 1 \in \Psi$. \newline
      Proof.
        Let $n' \in \Psi$.
        Then $n' + 0 = 0 + n'$.
        Hence
        \[  (n' + 1) + 0        \]
        \[    = n' + 1          \]
        \[    = (n' + 0) + 1    \]
        \[    = (0 + n') + 1    \]
        \[    = 0 + (n' + 1).   \]
      Qed.

      Hence every natural number belongs to $\Psi$.
      Thus $n + 0 = 0 + n$.
      Therefore $0$ is an element of $\Phi$.
    Qed.

    Let us show that (2) $n + 1 = 1 + n$. \newline
    Proof.
      Define $\Theta = \{ n' \in \Nat \mid n' + 1 = 1 + n' \}$.

      (2a) $0$ is an element of $\Theta$.

      (2b) For all $n' \in \Theta$ we have $n' + 1 \in \Theta$. \newline
      Proof.
        Let $n' \in \Theta$.
        Then $n' + 1 = 1 + n'$.
        Hence
        \[  (n' + 1) + 1        \]
        \[    = (1 + n') + 1    \]
        \[    = 1 + (n' + 1).   \]
        Thus $n' + 1 \in \Theta$.
      Qed.

      Thus every natural number belongs to $\Theta$.
      Therefore $n + 1 = 1 + n$.
    Qed.

    (3) For all $m' \in \Phi$ we have $m' + 1 \in \Phi$. \newline
    Proof.
      Let $m' \in \Phi$.
      Then $n + m' = m' + n$.
      Hence
      \[  n + (m'  + 1)       \]
      \[    = (n + m') + 1    \]
      \[    = (m' + n) + 1    \]
      \[    = m' + (n + 1)    \]
      \[    = m' + (1 + n)    \]
      \[    = (m' + 1) + n.   \]
      Thus $m' + 1 \in \Phi$.
    Qed.

    Thus every natural number is an element of $\Phi$.
    Therefore $n + m = m + n$.
  \end{proof}


  \subsubsection*{Cancellation}

  \begin{proposition}[forthel,id=ARITHMETIC_03_3137702874578944]
    Let $n, m, k$ be natural numbers.
    If $n + k = m + k$ then $n = m$.
  \end{proposition}
  \begin{proof}[forthel]
    Define $\Phi = \{ k' \in \Nat \mid$ if $n + k' = m + k'$ then $n = m \}$.

    (1) $0$ is an element of $\Phi$.

    (2) For all $k' \in \Phi$ we have $k' + 1 \in \Phi$. \newline
    Proof.
      Let $k' \in \Phi$.
      Suppose $n + (k' + 1) = m + (k' + 1)$.
      Then $(n + k') + 1 = (m + k') + 1$.
      Hence $n + k' = m + k'$.
      Thus $n = m$.
    Qed.

    Therefore every natural number is an element of $\Phi$.
    Consequently if $n + k = m + k$ then $n = m$.
  \end{proof}

  \begin{corollary}[forthel,id=ARITHMETIC_03_8445946379632640]
    Let $n, m, k$ be natural numbers.
    If $k + n = k + m$ then $n = m$.
  \end{corollary}
  \begin{proof}[forthel]
    Assume $k + n = k + m$.
    We have $k + n = n + k$ and $k + m = m + k$.
    Hence $n + k = m + k$.
    Thus $n = m$.
  \end{proof}


  \subsubsection*{Zero Sums}

  \begin{proposition}[forthel,id=ARITHMETIC_03_3520602170195968]
    Let $n, m$ be natural numbers.
    If $n + m = 0$ then $n = 0$ and $m = 0$.
  \end{proposition}
  \begin{proof}[forthel]
    Assume $n + m = 0$.
    Suppose $n \neq 0$ or $m \neq 0$.
    Then we can take a $k \in \Nat$ such that $n = k + 1$ or $m = k + 1$.
    Hence there exists a natural number $l$ such that
    $n + m
      = l + (k + 1)
      = (l + k) + 1
      \neq 0$.
    Contradiction.
  \end{proof}
\end{smodule}
\end{document}
