\documentclass[10pt]{article}
\usepackage[english]{babel}
\usepackage{../../../../lib/tex/naproche}
% Top-level sections and proofs

\newtheoremstyle{foundations}% name of the style to be used
  {\dimexpr\ftlparskip+4pt\relax}% measure of space to leave above the theorem.
  % (We must add \ftlparskip since otherwise the name of the label would not fit
  % into the border drawn around the environment.)
  {4pt}% measure of space to leave below the theorem.
  {\normalfont}% name of font to use in the body of the theorem
  {0pt}% measure of space to indent
  {\bfseries}% name of head font
  {}% punctuation between head and body
  { }% space after theorem head; " " = normal inter-word space
  {\thmname{#1}\thmnumber{ #2}\thmnote{ (#3)}.}

\let\definition\relax
\let\enddefinition\relax
\let\signature\relax
\let\endsignature\relax
\let\axiom\relax
\let\endaxiom\relax
\let\theorem\relax
\let\endtheorem\relax
\let\proposition\relax
\let\endproposition\relax
\let\lemma\relax
\let\endlemma\relax
\let\corollary\relax
\let\endcorollary\relax

\theoremstyle{foundations}

\newmdtheoremenv[
  backgroundcolor=lightgray!10,
  linecolor=black,
  innertopmargin=1em
]{definition}{Definition}[section]
\newmdtheoremenv[
  backgroundcolor=lightgray!10,
  linecolor=black,
  innertopmargin=1em
]{signature}[definition]{Signature}
\newmdtheoremenv[
  backgroundcolor=lightgray!10,
  linecolor=black,
  innertopmargin=1em
]{axiom}[definition]{Axiom}
\newmdtheoremenv[
  backgroundcolor=lightgray!10,
  linecolor=black,
  innertopmargin=1em
]{theorem}[definition]{Theorem}
\newmdtheoremenv[
  backgroundcolor=lightgray!10,
  linecolor=black,
  innertopmargin=1em
]{proposition}[definition]{Proposition}
\newmdtheoremenv[
  backgroundcolor=lightgray!10,
  linecolor=black,
  innertopmargin=1em
]{lemma}[definition]{Lemma}
\newmdtheoremenv[
  backgroundcolor=lightgray!10,
  linecolor=black,
  innertopmargin=1em
]{corollary}[definition]{Corollary}


% Hide imports

\usepackage{scontents}
\newenvsc{imports}[store-env=imports,print-env=false]


% Labels of top-level sections

\renewcommand{\showlabelfont}{\scriptsize\ttfamily\color{gray}}


% Non-symbolic "\implies" and "\iff"

\renewcommand{\implies}{\quad \text{implies} \quad}
\renewcommand{\iff}{\quad \text{iff} \quad}

\usepackage{amssymb}
\usepackage{pgffor}

\newcommand{\id}{\textrm{id}}
\newcommand{\range}{\textrm{range}}
\newcommand{\const}[2]{\textrm{const}_{#1}^{#2}}
\newcommand{\symdiff}{\mathop{\triangle}}
\newcommand{\pow}{\mathcal{P}}
\newcommand{\onto}{\twoheadrightarrow}
\newcommand{\into}{\hookrightarrow}

% E.g. `\set{1, 2, 3, 4}` prints the numbers 1, 2, 3 and 4 as a comma
% separated list which is enclosed within curly braces.
\def\set#1{\ensuremath{
  \{\foreach[count=\i]\x in {#1}{
    \ifnum\i>1,\,\fi
    \x
  }\}
}}

\useackage{amssymb}

\newcommand{\Nat}{\mathbb{N}}
\newcommand{\Prime}{\mathbb{P}}
\renewcommand{\succ}{\textrm{succ}}
\newcommand{\pred}{\textrm{pred}}
\newcommand{\add}{\textrm{add}}
\newcommand{\mul}{\textrm{mul}}
\renewcommand{\exp}{\textrm{exp}}
\newcommand{\fac}{\textrm{fac}}
\renewcommand{\div}{\mathrel{\textrm{div}}}
\renewcommand{\mod}{\mathrel{\textrm{mod}}}

\begin{document}
  \begin{imports}
    \begin{forthel}
      %[prove off][check off]
      [readtex \path{libraries/source/arithmetics/01_natural-numbers.ftl.tex}]
      %[prove on][check on]
    \end{forthel}
  \end{imports}


  \section{Definition of Addition}

  \begin{forthel}
    \begin{signature}\printlabel{ARITHMETIC_03_4372222701469696}
      Let $n, m$ be natural numbers.
      $n + m$ is a natural number.
    \end{signature}

    Let the sum of $n$ and $m$ stand for $n + m$.
  \end{forthel}

  \begin{forthel}
    \begin{axiom}\printlabel{ARITHMETIC_03_4827955356237824}
      Let $n$ be a natural number.
      Then $n + 0 = n$.
    \end{axiom}
  \end{forthel}

  \begin{forthel}
    \begin{axiom}\printlabel{ARITHMETIC_03_4427935386238924}
      Let $n, m$ be natural numbers.
      Then $n + \succ(m) = \succ(n + m)$.
    \end{axiom}
  \end{forthel}

  \begin{forthel}
    \begin{lemma}\printlabel{ARITHMETIC_03_777009668030464}
      Let $n$ be a natural number.
      Then $\succ(n) = n + 1$.
    \end{lemma}
  \end{forthel}

  \begin{forthel}
    \begin{lemma}\printlabel{ARITHMETIC_03_1031280145727488}
      Let $n, m$ be natural numbers.
      Then $n + (m + 1) = (n + m) + 1$.
    \end{lemma}
  \end{forthel}


  \section{The Peano Axioms and Recursion, Revisited}

  \begin{forthel}
    \begin{proposition}\printlabel{ARITHMETIC_03_3170769680990208}
      Let $n, m$ be natural numbers.
      If $n + 1 = m + 1$ then $n = m$.
    \end{proposition}
  \end{forthel}

  \begin{forthel}
    \begin{proposition}\printlabel{ARITHMETIC_03_1101538491629568}
      Let $n$ be a natural number.
      Then $n + 1 \neq 0$.
    \end{proposition}
  \end{forthel}

  \begin{forthel}
    \begin{proposition}[Induction]\printlabel{ARITHMETIC_03_647949900054528}
      Let $A$ be a class.
      Assume $0 \in A$.
      Assume that for all $n \in \Nat$ if $n \in A$ then $n + 1 \in A$.
      Then $A$ contains every natural number.
    \end{proposition}
  \end{forthel}


  \section{Computation Laws}

  \subsection{Associativity}

  \begin{forthel}
    \begin{proposition}\printlabel{ARITHMETIC_03_3235893452210176}
      Let $n, m, k$ be natural numbers.
      Then \[ n + (m + k) = (n + m) + k. \]
    \end{proposition}
    \begin{proof}
      Define $\Phi = \{ k' \in \Nat \mid n + (m + k') = (n + m) + k' \}$.

      (1) $0$ is contained in $\Phi$.
      Indeed $n + (m + 0) = n + m = (n + m) + 0$.

      (2) For all $k' \in \Phi$ we have $k' + 1 \in \Phi$. \\
      Proof.
        Let $k' \in \Phi$.
        Then $n + (m + k') = (n + m) + k'$.
        Hence
        \[  n + (m + (k' + 1))        \]
        \[    = n + ((m + k') + 1)    \]
        \[    = (n + (m + k')) + 1    \]
        \[    = ((n + m) + k') + 1    \]
        \[    = (n + m) + (k' + 1).   \]
        Thus $k' + 1 \in \Phi$.
      Qed.

      Thus every natural number is an element of $\Phi$.
      Therefore $n + (m + k) = (n + m) + k$.
    \end{proof}
  \end{forthel}


  \subsection{Commutativity}

  \begin{forthel}
    \begin{proposition}\printlabel{ARITHMETIC_03_4029553232052224}
      Let $n, m$ be natural numbers.
      Then \[ n + m = m + n. \]
    \end{proposition}
    \begin{proof}
      Define $\Phi = \{ m' \in \Nat \mid n + m' = m' + n \}$.

      (1) $0$ is an element of $\Phi$. \\
      Proof.
        Define $\Psi = \{ n' \in \Nat \mid n' + 0 = 0 + n' \}$.

        (1a) $0$ belongs to $\Psi$.

        (1b) For all $n' \in \Psi$ we have $n' + 1 \in \Psi$. \\
        Proof.
          Let $n' \in \Psi$.
          Then $n' + 0 = 0 + n'$.
          Hence
          \[  (n' + 1) + 0        \]
          \[    = n' + 1          \]
          \[    = (n' + 0) + 1    \]
          \[    = (0 + n') + 1    \]
          \[    = 0 + (n' + 1).   \]
        Qed.

        Hence every natural number belongs to $\Psi$.
        Thus $n + 0 = 0 + n$.
        Therefore $0$ is an element of $\Phi$.
      Qed.

      Let us show that (2) $n + 1 = 1 + n$. \\
      Proof.
        Define $\Theta = \{ n' \in \Nat \mid n' + 1 = 1 + n' \}$.

        (2a) $0$ is an element of $\Theta$.

        (2b) For all $n' \in \Theta$ we have $n' + 1 \in \Theta$. \\
        Proof.
          Let $n' \in \Theta$.
          Then $n' + 1 = 1 + n'$.
          Hence
          \[  (n' + 1) + 1        \]
          \[    = (1 + n') + 1    \]
          \[    = 1 + (n' + 1).   \]
          Thus $n' + 1 \in \Theta$.
        Qed.

        Thus every natural number belongs to $\Theta$.
        Therefore $n + 1 = 1 + n$.
      Qed.

      (3) For all $m' \in \Phi$ we have $m' + 1 \in \Phi$. \\
      Proof.
        Let $m' \in \Phi$.
        Then $n + m' = m' + n$.
        Hence
        \[  n + (m'  + 1)       \]
        \[    = (n + m') + 1    \]
        \[    = (m' + n) + 1    \]
        \[    = m' + (n + 1)    \]
        \[    = m' + (1 + n)    \]
        \[    = (m' + 1) + n.   \]
        Thus $m' + 1 \in \Phi$.
      Qed.

      Thus every natural number is an element of $\Phi$.
      Therefore $n + m = m + n$.
    \end{proof}
  \end{forthel}


  \subsection{Cancellation}

  \begin{forthel}
    \begin{proposition}\printlabel{ARITHMETIC_03_3137702874578944}
      Let $n, m, k$ be natural numbers.
      Then \[ n + k = m + k \implies n = m. \]
    \end{proposition}
    \begin{proof}
      Define $\Phi = \{ k' \in \Nat \mid$ if $n + k' = m + k'$ then $n = m \}$.

      (1) $0$ is an element of $\Phi$.

      (2) For all $k' \in \Phi$ we have $k' + 1 \in \Phi$. \\
      Proof.
        Let $k' \in \Phi$.
        Suppose $n + (k' + 1) = m + (k' + 1)$.
        Then $(n + k') + 1 = (m + k') + 1$.
        Hence $n + k' = m + k'$.
        Thus $n = m$.
      Qed.

      Therefore every natural number is an element of $\Phi$.
      Consequently if $n + k = m + k$ then $n = m$.
    \end{proof}
  \end{forthel}

  \begin{forthel}
    \begin{corollary}\printlabel{ARITHMETIC_03_8445946379632640}
      Let $n, m, k$ be natural numbers.
      Then \[ k + n = k + m \implies n = m. \]
    \end{corollary}
    \begin{proof}
      Assume $k + n = k + m$.
      We have $k + n = n + k$ and $k + m = m + k$.
      Hence $n + k = m + k$.
      Thus $n = m$.
    \end{proof}
  \end{forthel}


  \subsection{Zero Sums}

  \begin{forthel}
    \begin{proposition}\printlabel{ARITHMETIC_03_3520602170195968}
      Let $n, m$ be natural numbers.
      If $n + m = 0$ then $n = 0$ and $m = 0$.
    \end{proposition}
    \begin{proof}
      Assume $n + m = 0$.
      Suppose $n \neq 0$ or $m \neq 0$.
      Then we can take a $k \in \Nat$ such that $n = k + 1$ or $m = k + 1$.
      Hence there exists a natural number $l$ such that
      $n + m
        = l + (k + 1)
        = (l + k) + 1
        \neq 0$.
      Contradiction.
    \end{proof}
  \end{forthel}
\end{document}
