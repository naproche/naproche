\documentclass{naproche-library}
\libinput{arithmetics-preamble}
\begin{document}
\begin{smodule}{subtraction.ftl}
  \importmodule[libraries]{arithmetics?ordering.ftl}

  \section*{Subtraction}

  \begin{definition}[forthel,id=ARITHMETIC_05_8878757276286976]
    Let $n, m$ be natural numbers such that $n \geq m$.
    $n - m$ is the natural number $k$ such that $n = m + k$.

    Let the difference of $n$ and $m$ stand for $n - m$.
  \end{definition}

  \begin{proposition}[forthel,id=ARITHMETIC_05_874271710642176]
    Let $n, m$ be natural numbers such that $n \geq m$.
    Then $n - m = 0$ iff $n = m$.
  \end{proposition}
  \begin{proof}[forthel]
    Case $n - m = 0$.
      Then $n
        = (n - m) + m
        = 0 + m
        = m$.
    End.

    Case $n = m$.
      We have $(n - m) + m
        = n
        = m
        = 0 + m$.
      Hence $n - m = 0$.
    End.
  \end{proof}

  \begin{corollary}[forthel,id=ARITHMETIC_05_8457713057005568]
    Let $n$ be a natural number.
    Then $n - n = 0$.
  \end{corollary}

  \begin{proposition}[forthel,id=ARITHMETIC_05_8518521570983936]
    Let $n$ be a natural number.
    Then $n - 0 = n$.
  \end{proposition}
  \begin{proof}[forthel]
    We have $n
      = (n - 0) + 0
      = n - 0$.
  \end{proof}

  \begin{proposition}[forthel,id=ARITHMETIC_05_4222566117933056]
    Let $n, m$ be natural numbers such that $n \geq m$.
    Then $n - m \leq n$.
  \end{proposition}
  \begin{proof}[forthel]
    We have $(n - m) + m = n$.
    Hence $n - m \leq n$.
  \end{proof}

  \begin{proposition}[forthel,id=ARITHMETIC_05_1269537257291776]
    Let $n, m$ be natural numbers such that $n \geq m$.
    If $0 \neq m < n$ then $n - m < n$.
  \end{proposition}
  \begin{proof}[forthel]
    Assume $0 \neq m < n$.
    Suppose $n - m \geq n$.
    We have $(n - m) + m = n$.
    Then $n + m
      = (n - m) + m
      = n
      = n + 0$.
    Hence $m = 0$.
    Contradiction.
  \end{proof}

  \begin{proposition}[forthel,id=ARITHMETIC_05_4767595811045376]
    Let $n, m, k$ be natural numbers such that $n \geq m$.
    Then $(n - m) + k = (n + k) - m$.
  \end{proposition}
  \begin{proof}[forthel]
    We have
    \[  ((n - m) + k) + m       \]
    \[    = ((n - m) + m) + k   \]
    \[    = n + k               \]
    \[    = ((n + k) - m) + m.  \]

    Hence $(n - m) + k = (n + k) - m$.
  \end{proof}

  \begin{corollary}[forthel,id=ARITHMETIC_05_7578468875239424]
    Let $n, m, k$ be natural numbers such that $n \geq m$.
    Then $k + (n - m) = (k + n) - m$.
  \end{corollary}

  \begin{proposition}[forthel,id=ARITHMETIC_05_7595909347016704]
    Let $n, m, k$ be natural numbers such that $n \geq  m + k$.
    Then $(n - m) - k = n - (m + k)$.
  \end{proposition}
  \begin{proof}[forthel]
    We have
    \[  ((n - m) - k) + (m + k)       \]
    \[    = (((n - m) - k) + k) + m   \]
    \[    = (n - m) + m               \]
    \[    = n                         \]
    \[    = (n - (m + k)) + (m + k).  \]

    Hence $(n - m) - k = n - (m + k)$.
  \end{proof}
\end{smodule}
\end{document}
