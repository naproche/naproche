\documentclass{naproche-library}
\libinput{arithmetics-preamble}
\begin{document}
\begin{smodule}{multiplication-and-ordering.ftl}
  \importmodule[libraries]{arithmetics?multiplication.ftl}
  \importmodule[libraries]{arithmetics?ordering.ftl}

  \section*{Multiplication and Ordering}

  \begin{proposition}[forthel,id=ARITHMETIC_06_8817333933965312,printid]
    Let $n, m, k$ be natural numbers.
    Assume $k \neq 0$.
    Then $n < m$ iff $n \cdot k < m \cdot k$.
  \end{proposition}
  \begin{proof}[forthel]
    Case $n \cdot k < m \cdot k$.
      Define $\Phi = \{ n' \in \Nat \mid$ if $n' \cdot k < m \cdot k$ then $n' < m \}$.

      (1) $\Phi$ contains $0$.

      (2) For all $n' \in \Phi$ we have $n' + 1 \in \Phi$. \\
      Proof.
        Let $n' \in \Phi$.

        Let us show that if $(n' + 1) \cdot k < m \cdot k$ then $n' + 1 < m$.
          Assume $(n' + 1) \cdot k < m \cdot k$.
          Then $(n' \cdot k) + k < m \cdot k$.
          Hence $n' \cdot k < m \cdot k$.
          Thus $n' < m$.
          Then $n' + 1 \leq m$.
          If $n' + 1 = m$ then $(n' + 1) \cdot k = m \cdot k$.
          Hence $n' + 1 < m$.
        End.
      Qed.

      Therefore every natural number is contained in $\Phi$ (by \printref{ARITHMETIC_01_4764664342773760}).
      Consequently $n < m$.
    End.

    Case $n < m$.
      Take a positive natural number $l$ such that $m = n + l$.
      Then $m \cdot k = (n + l) \cdot k = (n \cdot k) + (l \cdot k)$.
      $l \cdot k$ is positive.
      Hence $n \cdot k < m \cdot k$.
    End.
  \end{proof}

  \begin{corollary}[forthel,id=ARITHMETIC_06_5048640368279552,printid]
    Let $n, m, k$ be natural numbers.
    Assume $k \neq 0$.
    Then $n < m$ iff $k \cdot n < k \cdot m$.
  \end{corollary}
  \begin{proof}[forthel]
    The thesis (by \printref{ARITHMETIC_06_8817333933965312}, \printref{ARITHMETIC_06_1764759896588288}).
  \end{proof}

  \begin{proposition}[forthel,id=ARITHMETIC_06_1826268599287808,printid]
    Let $n, m, k$ be natural numbers.
    If $n, m > k$ then $n \cdot m > k$.
  \end{proposition}
  \begin{proof}[forthel]
    Define $\Phi = \{ n' \in \Nat \mid$ if $n', m > k$ then $n' \cdot m > k \}$.

    (1) $\Phi$ contains $0$.

    (2) For all $n' \in \Phi$ we have $n' + 1 \in \Phi$. \\
    Proof.
      Let $n' \in \Phi$.

      Let us show that if $n' + 1, m > k$ then $(n' + 1) \cdot m > k$.
        Assume $n' + 1, m > k$.
        Then $(n' + 1) \cdot m = (n' \cdot m) + m$.
        If $n' = 0$ then
        $(n' \cdot m) + m
          = 0 + m
          = m
          > k$.
        If $n' \neq 0$ then
        $(n' \cdot m) + m
          > m
          > k$.
        Indeed if $n' \neq 0$ then $n' \cdot m > 0$.
        Indeed $m > 0$.
        Hence $(n' + 1) \cdot m > k$.
      Qed.
    Qed.

    Thus every natural number is contained in $\Phi$ (by \printref{ARITHMETIC_01_4764664342773760}).
    Therefore if $n, m > k$ then $n \cdot m > k$.
  \end{proof}

  \begin{corollary}[forthel,id=ARITHMETIC_06_1751605544222720,printid]
    Let $n, m, k$ be natural numbers.
    If $n \leq m$ then $k \cdot n \leq k \cdot m$.
  \end{corollary}

  \begin{corollary}[forthel,id=ARITHMETIC_06_3965209318260736,printid]
    Let $n, m, k$ be natural numbers.
    Assume $k \neq 0$.
    If $k \cdot n \leq k \cdot m$ then $n \leq m$.
  \end{corollary}
  \begin{proof}[forthel]
    If $k \cdot n = k \cdot m$ then $n = m$ (by \printref{ARITHMETIC_06_8575191374364672}).
    If $k \cdot n < k \cdot m$ then $n < m$ (by \printref{ARITHMETIC_06_5048640368279552}).
  \end{proof}

  \begin{corollary}[forthel,id=ARITHMETIC_06_8946886668976128,printid]
    Let $n, m, k$ be natural numbers.
    If $n \leq m$ then $n \cdot k \leq m \cdot k$.
  \end{corollary}

  \begin{corollary}[forthel,id=ARITHMETIC_06_4374428949413888,printid]
    Let $n, m, k$ be natural numbers.
    Assume $k \neq 0$.
    If $n \cdot k \leq m \cdot k$ then $n \leq m$.
  \end{corollary}

  \begin{proposition}[forthel,id=ARITHMETIC_06_8813409145454592,printid]
    Let $n, m, k$ be natural numbers.
    Assume $m > 0$ and $k > 1$.
    Then $k \cdot m > m$.
  \end{proposition}
  \begin{proof}[forthel]
    Take a natural number $l$ such that $k = l + 2$.
    Then
    \[  k \cdot m                       \]
    \[    = (l + 2) \cdot m             \]
    \[    = (l \cdot m) + (2 \cdot m)   \]
    \[    = (l \cdot m) + (m + m)       \]
    \[    = ((l \cdot m) + m) + m       \]
    \[    = ((l + 1) \cdot m) + m       \]
    \[    \geq 1 + m                    \]
    \[    > m.                          \]
    Indeed $((l + 1) \cdot m) + m \geq 1 + m$.
  \end{proof}
\end{smodule}
\end{document}
