\documentclass[10pt]{article}
\usepackage[english]{babel}
\usepackage{../../../meta-inf/lib/naproche}
% Top-level sections and proofs

\newtheoremstyle{foundations}% name of the style to be used
  {\dimexpr\ftlparskip+4pt\relax}% measure of space to leave above the theorem.
  % (We must add \ftlparskip since otherwise the name of the label would not fit
  % into the border drawn around the environment.)
  {4pt}% measure of space to leave below the theorem.
  {\normalfont}% name of font to use in the body of the theorem
  {0pt}% measure of space to indent
  {\bfseries}% name of head font
  {}% punctuation between head and body
  { }% space after theorem head; " " = normal inter-word space
  {\thmname{#1}\thmnumber{ #2}\thmnote{ (#3)}.}

\let\definition\relax
\let\enddefinition\relax
\let\signature\relax
\let\endsignature\relax
\let\axiom\relax
\let\endaxiom\relax
\let\theorem\relax
\let\endtheorem\relax
\let\proposition\relax
\let\endproposition\relax
\let\lemma\relax
\let\endlemma\relax
\let\corollary\relax
\let\endcorollary\relax

\theoremstyle{foundations}

\newmdtheoremenv[
  backgroundcolor=lightgray!10,
  linecolor=black,
  innertopmargin=1em
]{definition}{Definition}[section]
\newmdtheoremenv[
  backgroundcolor=lightgray!10,
  linecolor=black,
  innertopmargin=1em
]{signature}[definition]{Signature}
\newmdtheoremenv[
  backgroundcolor=lightgray!10,
  linecolor=black,
  innertopmargin=1em
]{axiom}[definition]{Axiom}
\newmdtheoremenv[
  backgroundcolor=lightgray!10,
  linecolor=black,
  innertopmargin=1em
]{theorem}[definition]{Theorem}
\newmdtheoremenv[
  backgroundcolor=lightgray!10,
  linecolor=black,
  innertopmargin=1em
]{proposition}[definition]{Proposition}
\newmdtheoremenv[
  backgroundcolor=lightgray!10,
  linecolor=black,
  innertopmargin=1em
]{lemma}[definition]{Lemma}
\newmdtheoremenv[
  backgroundcolor=lightgray!10,
  linecolor=black,
  innertopmargin=1em
]{corollary}[definition]{Corollary}


% Hide imports

\usepackage{scontents}
\newenvsc{imports}[store-env=imports,print-env=false]


% Labels of top-level sections

\renewcommand{\showlabelfont}{\scriptsize\ttfamily\color{gray}}


% Non-symbolic "\implies" and "\iff"

\renewcommand{\implies}{\quad \text{implies} \quad}
\renewcommand{\iff}{\quad \text{iff} \quad}

\usepackage{amssymb}
\usepackage{pgffor}

\newcommand{\id}{\textrm{id}}
\newcommand{\range}{\textrm{range}}
\newcommand{\const}[2]{\textrm{const}_{#1}^{#2}}
\newcommand{\symdiff}{\mathop{\triangle}}
\newcommand{\pow}{\mathcal{P}}
\newcommand{\onto}{\twoheadrightarrow}
\newcommand{\into}{\hookrightarrow}

% E.g. `\set{1, 2, 3, 4}` prints the numbers 1, 2, 3 and 4 as a comma
% separated list which is enclosed within curly braces.
\def\set#1{\ensuremath{
  \{\foreach[count=\i]\x in {#1}{
    \ifnum\i>1,\,\fi
    \x
  }\}
}}

\useackage{amssymb}

\newcommand{\Nat}{\mathbb{N}}
\newcommand{\Prime}{\mathbb{P}}
\renewcommand{\succ}{\textrm{succ}}
\newcommand{\pred}{\textrm{pred}}
\newcommand{\add}{\textrm{add}}
\newcommand{\mul}{\textrm{mul}}
\renewcommand{\exp}{\textrm{exp}}
\newcommand{\fac}{\textrm{fac}}
\renewcommand{\div}{\mathrel{\textrm{div}}}
\renewcommand{\mod}{\mathrel{\textrm{mod}}}

\begin{document}
  \begin{imports}
    \begin{forthel}
      %[prove off][check off]
      [read \path{libraries/source/arithmetics/multiplication.ftl.tex}]
      [read \path{libraries/source/arithmetics/ordering.ftl.tex}]
      %[prove on][check on]
    \end{forthel}
  \end{imports}


  \section*{Multiplication and Ordering}

  \begin{forthel}
    \begin{proposition}[id=ARITHMETIC_06_8817333933965312,printid]
      Let $n, m, k$ be natural numbers.
      Assume $k \neq 0$.
      Then $n < m$ iff $n \cdot k < m \cdot k$.
    \end{proposition}
    \begin{proof}
      Case $n \cdot k < m \cdot k$.
        Define $\Phi = \{ n' \in \Nat \mid$ if $n' \cdot k < m \cdot k$ then $n' < m \}$.

        (1) $\Phi$ contains $0$.

        (2) For all $n' \in \Phi$ we have $n' + 1 \in \Phi$. \\
        Proof.
          Let $n' \in \Phi$.

          Let us show that if $(n' + 1) \cdot k < m \cdot k$ then $n' + 1 < m$.
            Assume $(n' + 1) \cdot k < m \cdot k$.
            Then $(n' \cdot k) + k < m \cdot k$.
            Hence $n' \cdot k < m \cdot k$.
            Thus $n' < m$.
            Then $n' + 1 \leq m$.
            If $n' + 1 = m$ then $(n' + 1) \cdot k = m \cdot k$.
            Hence $n' + 1 < m$.
          End.
        Qed.

        Therefore every natural number is contained in $\Phi$ (by \printref{ARITHMETIC_01_4764664342773760}).
        Consequently $n < m$.
      End.

      Case $n < m$.
        Take a positive natural number $l$ such that $m = n + l$.
        Then $m \cdot k = (n + l) \cdot k = (n \cdot k) + (l \cdot k)$.
        $l \cdot k$ is positive.
        Hence $n \cdot k < m \cdot k$.
      End.
    \end{proof}
  \end{forthel}

  \begin{forthel}
    \begin{corollary}[id=ARITHMETIC_06_5048640368279552,printid]
      Let $n, m, k$ be natural numbers.
      Assume $k \neq 0$.
      Then $n < m$ iff $k \cdot n < k \cdot m$.
    \end{corollary}
    \begin{proof}
      The thesis (by \printref{ARITHMETIC_06_8817333933965312}, \printref{ARITHMETIC_06_1764759896588288}).
    \end{proof}
  \end{forthel}

  \begin{forthel}
    \begin{proposition}[id=ARITHMETIC_06_1826268599287808,printid]
      Let $n, m, k$ be natural numbers.
      If $n, m > k$ then $n \cdot m > k$.
    \end{proposition}
    \begin{proof}
      Define $\Phi = \{ n' \in \Nat \mid$ if $n', m > k$ then $n' \cdot m > k \}$.

      (1) $\Phi$ contains $0$.

      (2) For all $n' \in \Phi$ we have $n' + 1 \in \Phi$. \\
      Proof.
        Let $n' \in \Phi$.

        Let us show that if $n' + 1, m > k$ then $(n' + 1) \cdot m > k$.
          Assume $n' + 1, m > k$.
          Then $(n' + 1) \cdot m = (n' \cdot m) + m$.
          If $n' = 0$ then
          $(n' \cdot m) + m
            = 0 + m
            = m
            > k$.
          If $n' \neq 0$ then
          $(n' \cdot m) + m
            > m
            > k$.
          Indeed if $n' \neq 0$ then $n' \cdot m > 0$.
          Indeed $m > 0$.
          Hence $(n' + 1) \cdot m > k$.
        Qed.
      Qed.

      Thus every natural number is contained in $\Phi$ (by \printref{ARITHMETIC_01_4764664342773760}).
      Therefore if $n, m > k$ then $n \cdot m > k$.
    \end{proof}
  \end{forthel}

  \begin{forthel}
    \begin{corollary}[id=ARITHMETIC_06_1751605544222720,printid]
      Let $n, m, k$ be natural numbers.
      If $n \leq m$ then $k \cdot n \leq k \cdot m$.
    \end{corollary}
  \end{forthel}

  \begin{forthel}
    \begin{corollary}[id=ARITHMETIC_06_3965209318260736,printid]
      Let $n, m, k$ be natural numbers.
      Assume $k \neq 0$.
      If $k \cdot n \leq k \cdot m$ then $n \leq m$.
    \end{corollary}
    \begin{proof}
      If $k \cdot n = k \cdot m$ then $n = m$ (by \printref{ARITHMETIC_06_8575191374364672}).
      If $k \cdot n < k \cdot m$ then $n < m$ (by \printref{ARITHMETIC_06_5048640368279552}).
    \end{proof}
  \end{forthel}

  \begin{forthel}
    \begin{corollary}[id=ARITHMETIC_06_8946886668976128,printid]
      Let $n, m, k$ be natural numbers.
      If $n \leq m$ then $n \cdot k \leq m \cdot k$.
    \end{corollary}
  \end{forthel}

  \begin{forthel}
    \begin{corollary}[id=ARITHMETIC_06_4374428949413888,printid]
      Let $n, m, k$ be natural numbers.
      Assume $k \neq 0$.
      If $n \cdot k \leq m \cdot k$ then $n \leq m$.
    \end{corollary}
  \end{forthel}

  \begin{forthel}
    \begin{proposition}[id=ARITHMETIC_06_8813409145454592,printid]
      Let $n, m, k$ be natural numbers.
      Assume $m > 0$ and $k > 1$.
      Then $k \cdot m > m$.
    \end{proposition}
    \begin{proof}
      Take a natural number $l$ such that $k = l + 2$.
      Then
      \[  k \cdot m                       \]
      \[    = (l + 2) \cdot m             \]
      \[    = (l \cdot m) + (2 \cdot m)   \]
      \[    = (l \cdot m) + (m + m)       \]
      \[    = ((l \cdot m) + m) + m       \]
      \[    = ((l + 1) \cdot m) + m       \]
      \[    \geq 1 + m                    \]
      \[    > m.                          \]
      Indeed $((l + 1) \cdot m) + m \geq 1 + m$.
    \end{proof}
  \end{forthel}
\end{document}
