\documentclass{naproche-library}
\libinput{arithmetics-preamble}
\begin{document}
\begin{smodule}{exponentiation-of-even-and-odd-numbers.ftl}
  \importmodule[libraries]{arithmetics?exponentiation.ftl}
  \importmodule[libraries]{arithmetics?even-and-odd-numbers.ftl}

  \section*{Exponentiation of Even and Odd Numbers}

  \begin{proposition}[forthel,id=ARITHMETIC_15_1023659658745214]
    Let $n, m$ be natural numbers such that $m > 0$.
    Assume that $n$ is even.
    Then $n^{m}$ is even.
  \end{proposition}
  \begin{proof}[forthel]
    Take a natural number $k$ such that $n = 2 \cdot k$.
    Consider a natural number $m'$ such that $m = m' + 1$.
    Then $n^{m}
      = (2 \cdot k)^{m}
      = (2^{m} \cdot (k^{m}))
      = (2^{m' + 1} \cdot (k^{m}))
      = (2^{m'} \cdot 2) \cdot (k^{m})
      = (2 \cdot 2^{m'}) \cdot (k^{m})
      = 2 \cdot (2^{m'} \cdot (k^{m}))$.
    Hence $n^{m}$ is even.
  \end{proof}

  \begin{proposition}[forthel,id=ARITHMETIC_15_0021200236556985]
    Let $n, m$ be natural numbers.
    Assume that $n$ is odd.
    Then $n^{m}$ is odd.
  \end{proposition}
  \begin{proof}[forthel]
    Define $\Phi = \{ m' \in \Nat \mid n^{m'}$ is odd $\}$.

    (1) $\Phi$ contains $0$.
    Indeed $n^{0} = 1$ and $1$ is odd.

    (2) For all $m' \in \Phi$ we have $m' + 1 \in \Phi$. \newline
    Proof.
      Let $m' \in \Phi$.
      We have $n^{m' + 1} = n^{m'} \cdot n$.
      $n^{m'}$ is odd.
      Hence we can take a natural number $k$ such that $n^{m'} = (2 \cdot k) + 1$.
      Then $n^{m' + 1}
        = ((2 \cdot k) + 1) \cdot n
        = ((2 \cdot k) \cdot n) + (1 \cdot n)
        = ((2 \cdot k) \cdot n) + n
        = (2 \cdot (k \cdot n)) + n$.
      $2 \cdot (k \cdot n)$ is even and $n$ is odd.
      Thus $(2 \cdot (k \cdot n)) + n$ is odd.
      Therefore $n^{m' + 1}$ is odd.
      Consequently $m' + 1 \in \Phi$.
    Qed.

    Hence $\Phi$ contains every natural number (by \printref{ARITHMETIC_01_4764664342773760}).
    Thus $n^{m}$ is odd.
  \end{proof}
\end{smodule}
\end{document}