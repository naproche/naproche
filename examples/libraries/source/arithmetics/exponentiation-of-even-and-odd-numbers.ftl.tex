\documentclass[10pt]{article}
\usepackage[english]{babel}
\usepackage{../../../meta-inf/lib/naproche}
% Top-level sections and proofs

\newtheoremstyle{foundations}% name of the style to be used
  {\dimexpr\ftlparskip+4pt\relax}% measure of space to leave above the theorem.
  % (We must add \ftlparskip since otherwise the name of the label would not fit
  % into the border drawn around the environment.)
  {4pt}% measure of space to leave below the theorem.
  {\normalfont}% name of font to use in the body of the theorem
  {0pt}% measure of space to indent
  {\bfseries}% name of head font
  {}% punctuation between head and body
  { }% space after theorem head; " " = normal inter-word space
  {\thmname{#1}\thmnumber{ #2}\thmnote{ (#3)}.}

\let\definition\relax
\let\enddefinition\relax
\let\signature\relax
\let\endsignature\relax
\let\axiom\relax
\let\endaxiom\relax
\let\theorem\relax
\let\endtheorem\relax
\let\proposition\relax
\let\endproposition\relax
\let\lemma\relax
\let\endlemma\relax
\let\corollary\relax
\let\endcorollary\relax

\theoremstyle{foundations}

\newmdtheoremenv[
  backgroundcolor=lightgray!10,
  linecolor=black,
  innertopmargin=1em
]{definition}{Definition}[section]
\newmdtheoremenv[
  backgroundcolor=lightgray!10,
  linecolor=black,
  innertopmargin=1em
]{signature}[definition]{Signature}
\newmdtheoremenv[
  backgroundcolor=lightgray!10,
  linecolor=black,
  innertopmargin=1em
]{axiom}[definition]{Axiom}
\newmdtheoremenv[
  backgroundcolor=lightgray!10,
  linecolor=black,
  innertopmargin=1em
]{theorem}[definition]{Theorem}
\newmdtheoremenv[
  backgroundcolor=lightgray!10,
  linecolor=black,
  innertopmargin=1em
]{proposition}[definition]{Proposition}
\newmdtheoremenv[
  backgroundcolor=lightgray!10,
  linecolor=black,
  innertopmargin=1em
]{lemma}[definition]{Lemma}
\newmdtheoremenv[
  backgroundcolor=lightgray!10,
  linecolor=black,
  innertopmargin=1em
]{corollary}[definition]{Corollary}


% Hide imports

\usepackage{scontents}
\newenvsc{imports}[store-env=imports,print-env=false]


% Labels of top-level sections

\renewcommand{\showlabelfont}{\scriptsize\ttfamily\color{gray}}


% Non-symbolic "\implies" and "\iff"

\renewcommand{\implies}{\quad \text{implies} \quad}
\renewcommand{\iff}{\quad \text{iff} \quad}

\usepackage{amssymb}
\usepackage{pgffor}

\newcommand{\id}{\textrm{id}}
\newcommand{\range}{\textrm{range}}
\newcommand{\const}[2]{\textrm{const}_{#1}^{#2}}
\newcommand{\symdiff}{\mathop{\triangle}}
\newcommand{\pow}{\mathcal{P}}
\newcommand{\onto}{\twoheadrightarrow}
\newcommand{\into}{\hookrightarrow}

% E.g. `\set{1, 2, 3, 4}` prints the numbers 1, 2, 3 and 4 as a comma
% separated list which is enclosed within curly braces.
\def\set#1{\ensuremath{
  \{\foreach[count=\i]\x in {#1}{
    \ifnum\i>1,\,\fi
    \x
  }\}
}}

\useackage{amssymb}

\newcommand{\Nat}{\mathbb{N}}
\newcommand{\Prime}{\mathbb{P}}
\renewcommand{\succ}{\textrm{succ}}
\newcommand{\pred}{\textrm{pred}}
\newcommand{\add}{\textrm{add}}
\newcommand{\mul}{\textrm{mul}}
\renewcommand{\exp}{\textrm{exp}}
\newcommand{\fac}{\textrm{fac}}
\renewcommand{\div}{\mathrel{\textrm{div}}}
\renewcommand{\mod}{\mathrel{\textrm{mod}}}

\begin{document}
  \begin{imports}
    \begin{forthel}
      %[prove off][check off]
      [read \path{libraries/source/arithmetics/exponentiation.ftl.tex}]
      [read \path{libraries/source/arithmetics/even-and-odd-numbers.ftl.tex}]
      %[prove on][check on]
    \end{forthel}
  \end{imports}


  \section*{Exponentiation of Even and Odd Numbers}

  \begin{forthel}
    \begin{proposition}[id=ARITHMETIC_15_1023659658745214,printid]
      Let $n, m$ be natural numbers such that $m > 0$.
      Assume that $n$ is even.
      Then $n^{m}$ is even.
    \end{proposition}
    \begin{proof}
      Take a natural number $k$ such that $n = 2 \cdot k$.
      Consider a natural number $m'$ such that $m = m' + 1$.
      Then $n^{m}
        = (2 \cdot k)^{m}
        = (2^{m} \cdot (k^{m}))
        = (2^{m' + 1} \cdot (k^{m}))
        = (2^{m'} \cdot 2) \cdot (k^{m})
        = (2 \cdot 2^{m'}) \cdot (k^{m})
        = 2 \cdot (2^{m'} \cdot (k^{m}))$.
      Hence $n^{m}$ is even.
    \end{proof}
  \end{forthel}

  \begin{forthel}
    \begin{proposition}[id=ARITHMETIC_15_0021200236556985,printid]
      Let $n, m$ be natural numbers.
      Assume that $n$ is odd.
      Then $n^{m}$ is odd.
    \end{proposition}
    \begin{proof}
      Define $\Phi = \{ m' \in \Nat \mid n^{m'}$ is odd $\}$.

      (1) $\Phi$ contains $0$.
      Indeed $n^{0} = 1$ and $1$ is odd.

      (2) For all $m' \in \Phi$ we have $m' + 1 \in \Phi$. \\
      Proof.
        Let $m' \in \Phi$.
        We have $n^{m' + 1} = n^{m'} \cdot n$.
        $n^{m'}$ is odd.
        Hence we can take a natural number $k$ such that $n^{m'} = (2 \cdot k) + 1$.
        Then $n^{m' + 1}
          = ((2 \cdot k) + 1) \cdot n
          = ((2 \cdot k) \cdot n) + (1 \cdot n)
          = ((2 \cdot k) \cdot n) + n
          = (2 \cdot (k \cdot n)) + n$.
        $2 \cdot (k \cdot n)$ is even and $n$ is odd.
        Thus $(2 \cdot (k \cdot n)) + n$ is odd.
        Therefore $n^{m' + 1}$ is odd.
        Consequently $m' + 1 \in \Phi$.
      Qed.

      Hence $\Phi$ contains every natural number (by \printref{ARITHMETIC_01_4764664342773760}).
      Thus $n^{m}$ is odd.
    \end{proof}
  \end{forthel}
\end{document}