\documentclass{naproche-library}
\libinput{arithmetics-preamble}
\begin{document}
\begin{smodule}{modular-arithmetics.ftl}
  \importmodule[libraries]{arithmetics?multiplication-and-ordering.ftl}

  \symdef{nil}{[]}
  \symdef{cons}{\mathop{::}}

  \section*{Modular Arithmetics}

  \subsection*{Quotients and Remainders}

  \begin{definition}[forthel,id=ARITHMETIC_08_8621463798022144,printid]
    Let $n, m$ be natural numbers such that $m \neq 0$.
    $n \div m$ is the natural number $q$ such that $n = (m \cdot q) + r$ for some natural number $r$ that is less than $m$.

    Let the quotient of $n$ over $m$ stand for $n \div m$.
  \end{definition}

  \begin{definition}[forthel,id=ARITHMETIC_08_3560980160184320,printid]
    Let $n, m$ be natural numbers such that $m \neq 0$.
    $n \mod m$ is the natural number $r$ such that $r < m$ and there exists a natural number $q$ such that $n = (m \cdot q) + r$.

    Let the remainder of $n$ over $m$ stand for $n \mod m$.
  \end{definition}


  \subsection*{Basic Properties}

  \begin{definition}[forthel,id=ARITHMETIC_08_5448561831444480,printid]
    Let $n, m, k$ be natural numbers such that $k \neq 0$.
    $n \equiv m \pmod{k}$ iff $n \mod k = m \mod k$.

    Let $n$ and $m$ are congruent modulo $k$ stand for $n \equiv m \pmod{k}$.
  \end{definition}

  \begin{proposition}[forthel,id=ARITHMETIC_08_3818318544764928,printid]
    Let $n, k$ be natural numbers such that $k \neq 0$.
    Then $n \equiv n \pmod{k}$.
  \end{proposition}
  \begin{proof}[forthel]
    We have $n \mod k = n \mod k$.
    Hence $n \equiv n \pmod{k}$.
  \end{proof}

  \begin{proposition}[forthel,id=ARITHMETIC_08_2337210737098752,printid]
    Let $n, m, k$ be natural numbers such that $k \neq 0$.
    If $n \equiv m \pmod{k}$ then $m \equiv n \pmod{k}$.
  \end{proposition}
  \begin{proof}[forthel]
    Assume $n \equiv m \pmod{k}$.
    Then $n \mod k = m \mod k$.
    Hence $m \mod k = n \mod k$.
    Thus $m \equiv n \pmod{k}$.
  \end{proof}

  \begin{proposition}[forthel,id=ARITHMETIC_08_7464329746055168,printid]
    Let $n, m, l, k$ be natural numbers such that $k \neq 0$.
    If $n \equiv m \pmod{k}$ and $m \equiv l \pmod{k}$ then $n \equiv l \pmod{k}$.
  \end{proposition}
  \begin{proof}[forthel]
    Assume $n \equiv m \pmod{k}$ and $m \equiv l \pmod{k}$.
    Then $n \mod k = m \mod k$ and $m \mod k = l \mod k$.
    Hence $n \mod k = l \mod k$.
    Thus $n \equiv l \pmod{k}$.
  \end{proof}

  \begin{proposition}[forthel,id=ARITHMETIC_08_2034122983735296,printid]
    Let $n, m, k$ be natural numbers such that $k \neq 0$.
    Assume $n \geq m$.
    Then $n \equiv m \pmod{k}$ iff $n = (k \cdot x) + m$ for some natural number $x$.
  \end{proposition}
  \begin{proof}[forthel]
    Case $n \equiv m \pmod{k}$.
      Then $n \mod k = m \mod k$.
      Take a natural number $r$ such that $r < k$ and $n \mod k = r = m \mod k$.
      Take a nonzero natural number $l$ such that $k = r + l$.
      Consider natural numbers $q,q'$ such that $n = (q \cdot k) + r$ and $m = (q' \cdot k) + r$.

      Then $q \geq q'$. \\
      Proof.
        Assume the contrary.
        Then $q < q'$.
        Hence $q \cdot k < q' \cdot k$.
        Thus $(q \cdot k) + r < (q' \cdot k) + r$ (by \printref{ARITHMETIC_04_7354062662008832}).
        Indeed $q \cdot k$ and $q' \cdot k$ are natural numbers.
        Therefore $n < m$.
        Contradiction.
      Qed.

      Take a natural number $x$ such that $q = q' + x$.

      Let us show that $n = (k \cdot x) + m$.
        We have
        \[  (k \cdot x) + m                       \]
        \[    = (k \cdot x) + ((q' \cdot k) + r)  \]
        \[    = ((k \cdot x) + (q' \cdot k)) + r  \]
        \[    = ((k \cdot x) + (k \cdot q')) + r  \]
        \[    = (k \cdot (q' + x)) + r            \]
        \[    = (k \cdot q) + r                   \]
        \[    = n.                                \]
      End.
    End.

    Case $n = (k \cdot x) + m$ for some natural number $x$.
      Consider a natural number $x$ such that $n = (k \cdot x) + m$.
      Take natural numbers $r, r'$ such that $n \mod k = r$ and $m \mod k = r'$.
      Then $r, r' < k$.
      Take natural numbers $q, q'$ such that $n = (k \cdot q) + r$ and $m = (k \cdot q') + r'$.
      Then
      \[  (k \cdot q) + r                         \]
      \[    = n                                   \]
      \[    = (k \cdot x) + m                     \]
      \[    = (k \cdot x) + ((k \cdot q') + r')   \]
      \[    = ((k \cdot x) + (k \cdot q')) + r'   \]
      \[    = (k \cdot (x + q')) + r'.            \]
      Hence $r = r'$.
      Thus $n \mod k = m \mod k$.
      Therefore $n \equiv m \pmod{k}$.
    End.
  \end{proof}

  \begin{proposition}[forthel,id=ARITHMETIC_08_2988318228742144,printid]
    Let $n, m, k, k'$ be natural numbers such that $k, k' \neq 0$.
    If $n \equiv m \pmod{k \cdot k'}$ then $n \equiv m \pmod{k}$.
  \end{proposition}
  \begin{proof}[forthel]
    Assume $n \equiv m \pmod{k \cdot k'}$.

    Case $n \geq m$.
      We can take a natural number $x$ such that $n = ((k \cdot k') \cdot x) + m$.
      Then $n = (k \cdot (k' \cdot x)) + m$.
      Hence $n \equiv m \pmod{k}$.
    End.

    Case $m \geq n$.
      We have $m \equiv n \pmod{k \cdot k'}$.
      Hence we can take a natural number $x$ such that $m = ((k \cdot k') \cdot x) + n$.
      Then $m = (k \cdot (k' \cdot x)) + n$.
      Thus $m \equiv n \pmod{k}$.
      Therefore $n \equiv m \pmod{k}$.
    End.
  \end{proof}

  \begin{corollary}[forthel,id=ARITHMETIC_08_5895145169879040,printid]
    Let $n, m, k, k'$ be natural numbers such that $k, k' \neq 0$.
    If $n \equiv m \pmod{k \cdot k'}$ then $n \equiv m \pmod{k'}$.
  \end{corollary}
  \begin{proof}[forthel]
    Assume $n \equiv m \pmod{k \cdot k'}$.
    Then $n \equiv m \pmod{k' \cdot k}$.
    Hence $n \equiv m \pmod{k'}$.
  \end{proof}

  \begin{proposition}[forthel,id=ARITHMETIC_08_5984712287846400,printid]
    Let $n, k$ be natural numbers such that $k \neq 0$.
    Then $n + k \equiv n \pmod{k}$.
  \end{proposition}
  \begin{proof}[forthel]
    Take $r = n \mod k$ and $r' = (n + k) \mod k$.
    Consider a $q \in \Nat$ such that $n = (k \cdot q) + r$ and $r < k$.
    Consider a $q' \in \Nat$ such that $n + k = (k \cdot q') + r'$ and
    $r' < k$.
    Then $(k \cdot q') + r'
      = n + k
      = ((k \cdot q) + r) + k
      = (k + (k \cdot q)) + r
      = (k \cdot (q + 1)) + r$.
    Hence $r = r'$.
    Consequently $n \mod k = (n + k) \mod k$.
    Thus $n + k \equiv n \pmod{k}$.
  \end{proof}
\end{smodule}
\end{document}
