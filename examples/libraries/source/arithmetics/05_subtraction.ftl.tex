\documentclass[10pt]{article}
\usepackage[english]{babel}
\usepackage{../../../../lib/tex/naproche}
% Top-level sections and proofs

\newtheoremstyle{foundations}% name of the style to be used
  {\dimexpr\ftlparskip+4pt\relax}% measure of space to leave above the theorem.
  % (We must add \ftlparskip since otherwise the name of the label would not fit
  % into the border drawn around the environment.)
  {4pt}% measure of space to leave below the theorem.
  {\normalfont}% name of font to use in the body of the theorem
  {0pt}% measure of space to indent
  {\bfseries}% name of head font
  {}% punctuation between head and body
  { }% space after theorem head; " " = normal inter-word space
  {\thmname{#1}\thmnumber{ #2}\thmnote{ (#3)}.}

\let\definition\relax
\let\enddefinition\relax
\let\signature\relax
\let\endsignature\relax
\let\axiom\relax
\let\endaxiom\relax
\let\theorem\relax
\let\endtheorem\relax
\let\proposition\relax
\let\endproposition\relax
\let\lemma\relax
\let\endlemma\relax
\let\corollary\relax
\let\endcorollary\relax

\theoremstyle{foundations}

\newmdtheoremenv[
  backgroundcolor=lightgray!10,
  linecolor=black,
  innertopmargin=1em
]{definition}{Definition}[section]
\newmdtheoremenv[
  backgroundcolor=lightgray!10,
  linecolor=black,
  innertopmargin=1em
]{signature}[definition]{Signature}
\newmdtheoremenv[
  backgroundcolor=lightgray!10,
  linecolor=black,
  innertopmargin=1em
]{axiom}[definition]{Axiom}
\newmdtheoremenv[
  backgroundcolor=lightgray!10,
  linecolor=black,
  innertopmargin=1em
]{theorem}[definition]{Theorem}
\newmdtheoremenv[
  backgroundcolor=lightgray!10,
  linecolor=black,
  innertopmargin=1em
]{proposition}[definition]{Proposition}
\newmdtheoremenv[
  backgroundcolor=lightgray!10,
  linecolor=black,
  innertopmargin=1em
]{lemma}[definition]{Lemma}
\newmdtheoremenv[
  backgroundcolor=lightgray!10,
  linecolor=black,
  innertopmargin=1em
]{corollary}[definition]{Corollary}


% Hide imports

\usepackage{scontents}
\newenvsc{imports}[store-env=imports,print-env=false]


% Labels of top-level sections

\renewcommand{\showlabelfont}{\scriptsize\ttfamily\color{gray}}


% Non-symbolic "\implies" and "\iff"

\renewcommand{\implies}{\quad \text{implies} \quad}
\renewcommand{\iff}{\quad \text{iff} \quad}

\usepackage{amssymb}
\usepackage{pgffor}

\newcommand{\id}{\textrm{id}}
\newcommand{\range}{\textrm{range}}
\newcommand{\const}[2]{\textrm{const}_{#1}^{#2}}
\newcommand{\symdiff}{\mathop{\triangle}}
\newcommand{\pow}{\mathcal{P}}
\newcommand{\onto}{\twoheadrightarrow}
\newcommand{\into}{\hookrightarrow}

% E.g. `\set{1, 2, 3, 4}` prints the numbers 1, 2, 3 and 4 as a comma
% separated list which is enclosed within curly braces.
\def\set#1{\ensuremath{
  \{\foreach[count=\i]\x in {#1}{
    \ifnum\i>1,\,\fi
    \x
  }\}
}}

\useackage{amssymb}

\newcommand{\Nat}{\mathbb{N}}
\newcommand{\Prime}{\mathbb{P}}
\renewcommand{\succ}{\textrm{succ}}
\newcommand{\pred}{\textrm{pred}}
\newcommand{\add}{\textrm{add}}
\newcommand{\mul}{\textrm{mul}}
\renewcommand{\exp}{\textrm{exp}}
\newcommand{\fac}{\textrm{fac}}
\renewcommand{\div}{\mathrel{\textrm{div}}}
\renewcommand{\mod}{\mathrel{\textrm{mod}}}

\begin{document}
  \begin{imports}
    \begin{forthel}
      %[prove off][check off]
      [readtex \path{libraries/source/arithmetics/04_ordering.ftl.tex}]
      %[prove on][check on]
    \end{forthel}
  \end{imports}


  \section{Subtraction}

  \begin{forthel}
    \begin{definition}\printlabel{ARITHMETIC_05_8878757276286976}
      Let $n, m$ be natural numbers such that $n \geq m$.
      $n - m$ is the natural number $k$ such that $n = m + k$.
    \end{definition}

    Let the difference of $n$ and $m$ stand for $n - m$.
  \end{forthel}

  \begin{forthel}
    \begin{proposition}\printlabel{ARITHMETIC_05_874271710642176}
      Let $n, m$ be natural numbers such that $n \geq m$.
      Then \[ n - m = 0 \iff n = m. \]
    \end{proposition}
    \begin{proof}
      Case $n - m = 0$.
        Then $n
          = (n - m) + m
          = 0 + m
          = m$.
      End.

      Case $n = m$.
        We have $(n - m) + m
          = n
          = m
          = 0 + m$.
        Hence $n - m = 0$.
      End.
    \end{proof}
  \end{forthel}

  \begin{forthel}
    \begin{corollary}\printlabel{ARITHMETIC_05_8457713057005568}
      Let $n$ be a natural number.
      Then \[ n - n = 0. \]
    \end{corollary}
  \end{forthel}

  \begin{forthel}
    \begin{proposition}\printlabel{ARITHMETIC_05_8518521570983936}
      Let $n$ be a natural number.
      Then \[ n - 0 = n. \]
    \end{proposition}
    \begin{proof}
      We have $n
        = (n - 0) + 0
        = n - 0$.
    \end{proof}
  \end{forthel}

  \begin{forthel}
    \begin{proposition}\printlabel{ARITHMETIC_05_4222566117933056}
      Let $n, m$ be natural numbers such that $n \geq m$.
      Then \[ n - m \leq n. \]
    \end{proposition}
    \begin{proof}
      We have $(n - m) + m = n$.
      Hence $n - m \leq n$.
    \end{proof}
  \end{forthel}

  \begin{forthel}
    \begin{proposition}\printlabel{ARITHMETIC_05_1269537257291776}
      Let $n, m$ be natural numbers such that $n \geq m$.
      Then \[ 0 \neq m < n \implies n - m < n. \]
    \end{proposition}
    \begin{proof}
      Assume $0 \neq m < n$.
      Suppose $n - m \geq n$.
      We have $(n - m) + m = n$.
      Then $n + m
        = (n - m) + m
        = n
        = n + 0$.
      Hence $m = 0$.
      Contradiction.
    \end{proof}
  \end{forthel}

  \begin{forthel}
    \begin{proposition}\printlabel{ARITHMETIC_05_4767595811045376}
      Let $n, m, k$ be natural numbers such that $n \geq m$.
      Then \[ (n - m) + k = (n + k) - m. \]
    \end{proposition}
    \begin{proof}
      We have
      \[  ((n - m) + k) + m       \]
      \[    = ((n - m) + m) + k   \]
      \[    = n + k               \]
      \[    = ((n + k) - m) + m.  \]

      Hence $(n - m) + k = (n + k) - m$.
    \end{proof}
  \end{forthel}

  \begin{forthel}
    \begin{corollary}\printlabel{ARITHMETIC_05_7578468875239424}
      Let $n, m, k$ be natural numbers such that $n \geq m$.
      Then \[ k + (n - m) = (k + n) - m. \]
    \end{corollary}
  \end{forthel}

  \begin{forthel}
    \begin{proposition}\printlabel{ARITHMETIC_05_7595909347016704}
      Let $n, m, k$ be natural numbers such that $n \geq  m + k$.
      Then \[ (n - m) - k = n - (m + k). \]
    \end{proposition}
    \begin{proof}
      We have
      \[  ((n - m) - k) + (m + k)       \]
      \[    = (((n - m) - k) + k) + m   \]
      \[    = (n - m) + m               \]
      \[    = n                         \]
      \[    = (n - (m + k)) + (m + k).  \]

      Hence $(n - m) - k = n - (m + k)$.
    \end{proof}
  \end{forthel}
\end{document}
