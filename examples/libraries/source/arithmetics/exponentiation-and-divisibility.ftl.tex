\documentclass[10pt]{article}
\usepackage[english]{babel}
\usepackage{../../../meta-inf/lib/naproche}
% Top-level sections and proofs

\newtheoremstyle{foundations}% name of the style to be used
  {\dimexpr\ftlparskip+4pt\relax}% measure of space to leave above the theorem.
  % (We must add \ftlparskip since otherwise the name of the label would not fit
  % into the border drawn around the environment.)
  {4pt}% measure of space to leave below the theorem.
  {\normalfont}% name of font to use in the body of the theorem
  {0pt}% measure of space to indent
  {\bfseries}% name of head font
  {}% punctuation between head and body
  { }% space after theorem head; " " = normal inter-word space
  {\thmname{#1}\thmnumber{ #2}\thmnote{ (#3)}.}

\let\definition\relax
\let\enddefinition\relax
\let\signature\relax
\let\endsignature\relax
\let\axiom\relax
\let\endaxiom\relax
\let\theorem\relax
\let\endtheorem\relax
\let\proposition\relax
\let\endproposition\relax
\let\lemma\relax
\let\endlemma\relax
\let\corollary\relax
\let\endcorollary\relax

\theoremstyle{foundations}

\newmdtheoremenv[
  backgroundcolor=lightgray!10,
  linecolor=black,
  innertopmargin=1em
]{definition}{Definition}[section]
\newmdtheoremenv[
  backgroundcolor=lightgray!10,
  linecolor=black,
  innertopmargin=1em
]{signature}[definition]{Signature}
\newmdtheoremenv[
  backgroundcolor=lightgray!10,
  linecolor=black,
  innertopmargin=1em
]{axiom}[definition]{Axiom}
\newmdtheoremenv[
  backgroundcolor=lightgray!10,
  linecolor=black,
  innertopmargin=1em
]{theorem}[definition]{Theorem}
\newmdtheoremenv[
  backgroundcolor=lightgray!10,
  linecolor=black,
  innertopmargin=1em
]{proposition}[definition]{Proposition}
\newmdtheoremenv[
  backgroundcolor=lightgray!10,
  linecolor=black,
  innertopmargin=1em
]{lemma}[definition]{Lemma}
\newmdtheoremenv[
  backgroundcolor=lightgray!10,
  linecolor=black,
  innertopmargin=1em
]{corollary}[definition]{Corollary}


% Hide imports

\usepackage{scontents}
\newenvsc{imports}[store-env=imports,print-env=false]


% Labels of top-level sections

\renewcommand{\showlabelfont}{\scriptsize\ttfamily\color{gray}}


% Non-symbolic "\implies" and "\iff"

\renewcommand{\implies}{\quad \text{implies} \quad}
\renewcommand{\iff}{\quad \text{iff} \quad}

\usepackage{amssymb}
\usepackage{pgffor}

\newcommand{\id}{\textrm{id}}
\newcommand{\range}{\textrm{range}}
\newcommand{\const}[2]{\textrm{const}_{#1}^{#2}}
\newcommand{\symdiff}{\mathop{\triangle}}
\newcommand{\pow}{\mathcal{P}}
\newcommand{\onto}{\twoheadrightarrow}
\newcommand{\into}{\hookrightarrow}

% E.g. `\set{1, 2, 3, 4}` prints the numbers 1, 2, 3 and 4 as a comma
% separated list which is enclosed within curly braces.
\def\set#1{\ensuremath{
  \{\foreach[count=\i]\x in {#1}{
    \ifnum\i>1,\,\fi
    \x
  }\}
}}

\useackage{amssymb}

\newcommand{\Nat}{\mathbb{N}}
\newcommand{\Prime}{\mathbb{P}}
\renewcommand{\succ}{\textrm{succ}}
\newcommand{\pred}{\textrm{pred}}
\newcommand{\add}{\textrm{add}}
\newcommand{\mul}{\textrm{mul}}
\renewcommand{\exp}{\textrm{exp}}
\newcommand{\fac}{\textrm{fac}}
\renewcommand{\div}{\mathrel{\textrm{div}}}
\renewcommand{\mod}{\mathrel{\textrm{mod}}}

\begin{document}
  \begin{imports}
    \begin{forthel}
      %[prove off][check off]
      [read \path{libraries/source/arithmetics/exponentiation-and-ordering.ftl.tex}]
      [read \path{libraries/source/arithmetics/divisibility.ftl.tex}]
      %[prove on][check on]
    \end{forthel}
  \end{imports}


  \section*{Exponentiation and Divisibility}

  \begin{forthel}
    \begin{proposition}[id=ARITHMETIC_13_8426075493236736,printid]
      Let $n, m, k$ be natural numbers such that $n, m \neq 0$ and $k > 1$.
      Then $k^{n} \mid k^{m}$ iff $n \leq m$.
    \end{proposition}
    \begin{proof}
      Case $k^{n} \mid k^{m}$.
        Assume $n > m$.
        Take a nonzero natural number $l$ such that $n = m + l$.
        Then $k^{n}
          = k^{m + l}
          = k^{m} \cdot k^{l}$.
        Hence $k^{m} \mid k^{n}$.
        Thus $k^{m} = k^{n}$.
        Therefore $m = n$.
        Contradiction.
      End.
  
      Case $n \leq m$.
        Take a natural number $l$ such that $m = n + l$.
        Then $k^{m}
          = k^{n + l}
          = k^{n} \cdot k^{l}$.
        Hence $k^{n} \mid k^{m}$.
      End.
    \end{proof}
  \end{forthel}
  
  \begin{forthel}
    \begin{proposition}[id=ARITHMETIC_13_797196163219456,printid]
      Let $n$ be a composite natural number.
      Then $n$ has a nontrivial divisor $m$ such that $m^{2} \leq n$.
    \end{proposition}
    \begin{proof}
      Define $A = \{ m \in \Nat \mid m$ is a nontrivial divisor of $n \}$.
      $A$ is nonempty.
      Hence we can take a $m \in A$ such that $m \leq l$ for all $l \in A$.
      Consider a natural number $k$ such that $m \cdot k = n$.
      Then $m \leq k$.
      Hence $m^{2} = m \cdot m \leq m \cdot k = n$.
      Therefore $m^{2} \leq n$.
    \end{proof}
  \end{forthel}
\end{document}