\documentclass{naproche-library}
\libinput{arithmetics-preamble}
\begin{document}
\begin{smodule}{factorial.ftl}
  \importmodule[libraries]{arithmetics?multiplication.ftl}

  \section*{Factorial}

  \begin{signature}[forthel,id=ARITHMETIC_12_0210357812531785]
    Let $n$ be a natural number.
    $n!$ is a natural number.
  \end{signature}

  \begin{axiom}[forthel,id=ARITHMETIC_12_6945210541236989]
    $0! = 1$.
  \end{axiom}

  \begin{axiom}[forthel,id=ARITHMETIC_12_2312015846987452]
    Let $n$ be a natural number.
    Then $(n + 1)! = n! \cdot (n + 1)$.
  \end{axiom}

  \begin{proposition}[forthel,id=ARITHMETIC_12_6123519005949952]
    Let $n$ be a natural number.
    Then $n! \neq 0$.
  \end{proposition}
  \begin{proof}[forthel]
    Define $\Phi = \{n' \in \Nat \mid n'! \neq 0 \}$.

    (1) $\Phi$ contains $0$.
    Indeed $0! = 1 \neq 0$.

    (2) For all $n' \in \Phi$ we have $n' + 1 \in \Phi$. \\
    Proof.
      Let $n' \in \Phi$.
      We have $(n' + 1)! = (n' + 1) \cdot n'!$.
      $n' + 1$ and $n'!$ are nonzero.
      Hence $(n' + 1)! \neq 0$.
    Qed.

    Thus $\Phi$ contains every natural number (by \printref{ARITHMETIC_01_4764664342773760}).
    Therefore $n! \neq 0$.
  \end{proof}
\end{smodule}
\end{document}