\documentclass[10pt]{article}
\usepackage[english]{babel}
\usepackage{../../../../lib/tex/naproche}
% Top-level sections and proofs

\newtheoremstyle{foundations}% name of the style to be used
  {\dimexpr\ftlparskip+4pt\relax}% measure of space to leave above the theorem.
  % (We must add \ftlparskip since otherwise the name of the label would not fit
  % into the border drawn around the environment.)
  {4pt}% measure of space to leave below the theorem.
  {\normalfont}% name of font to use in the body of the theorem
  {0pt}% measure of space to indent
  {\bfseries}% name of head font
  {}% punctuation between head and body
  { }% space after theorem head; " " = normal inter-word space
  {\thmname{#1}\thmnumber{ #2}\thmnote{ (#3)}.}

\let\definition\relax
\let\enddefinition\relax
\let\signature\relax
\let\endsignature\relax
\let\axiom\relax
\let\endaxiom\relax
\let\theorem\relax
\let\endtheorem\relax
\let\proposition\relax
\let\endproposition\relax
\let\lemma\relax
\let\endlemma\relax
\let\corollary\relax
\let\endcorollary\relax

\theoremstyle{foundations}

\newmdtheoremenv[
  backgroundcolor=lightgray!10,
  linecolor=black,
  innertopmargin=1em
]{definition}{Definition}[section]
\newmdtheoremenv[
  backgroundcolor=lightgray!10,
  linecolor=black,
  innertopmargin=1em
]{signature}[definition]{Signature}
\newmdtheoremenv[
  backgroundcolor=lightgray!10,
  linecolor=black,
  innertopmargin=1em
]{axiom}[definition]{Axiom}
\newmdtheoremenv[
  backgroundcolor=lightgray!10,
  linecolor=black,
  innertopmargin=1em
]{theorem}[definition]{Theorem}
\newmdtheoremenv[
  backgroundcolor=lightgray!10,
  linecolor=black,
  innertopmargin=1em
]{proposition}[definition]{Proposition}
\newmdtheoremenv[
  backgroundcolor=lightgray!10,
  linecolor=black,
  innertopmargin=1em
]{lemma}[definition]{Lemma}
\newmdtheoremenv[
  backgroundcolor=lightgray!10,
  linecolor=black,
  innertopmargin=1em
]{corollary}[definition]{Corollary}


% Hide imports

\usepackage{scontents}
\newenvsc{imports}[store-env=imports,print-env=false]


% Labels of top-level sections

\renewcommand{\showlabelfont}{\scriptsize\ttfamily\color{gray}}


% Non-symbolic "\implies" and "\iff"

\renewcommand{\implies}{\quad \text{implies} \quad}
\renewcommand{\iff}{\quad \text{iff} \quad}

\usepackage{amssymb}
\usepackage{pgffor}

\newcommand{\id}{\textrm{id}}
\newcommand{\range}{\textrm{range}}
\newcommand{\const}[2]{\textrm{const}_{#1}^{#2}}
\newcommand{\symdiff}{\mathop{\triangle}}
\newcommand{\pow}{\mathcal{P}}
\newcommand{\onto}{\twoheadrightarrow}
\newcommand{\into}{\hookrightarrow}

% E.g. `\set{1, 2, 3, 4}` prints the numbers 1, 2, 3 and 4 as a comma
% separated list which is enclosed within curly braces.
\def\set#1{\ensuremath{
  \{\foreach[count=\i]\x in {#1}{
    \ifnum\i>1,\,\fi
    \x
  }\}
}}

\useackage{amssymb}

\newcommand{\Nat}{\mathbb{N}}
\newcommand{\Prime}{\mathbb{P}}
\renewcommand{\succ}{\textrm{succ}}
\newcommand{\pred}{\textrm{pred}}
\newcommand{\add}{\textrm{add}}
\newcommand{\mul}{\textrm{mul}}
\renewcommand{\exp}{\textrm{exp}}
\newcommand{\fac}{\textrm{fac}}
\renewcommand{\div}{\mathrel{\textrm{div}}}
\renewcommand{\mod}{\mathrel{\textrm{mod}}}

\begin{document}
  \begin{imports}
    \begin{forthel}
      %[prove off][check off]
      [readtex \path{libraries/source/arithmetics/factorial.ftl.tex}]
      [readtex \path{libraries/source/arithmetics/exponentiation-and-ordering.ftl.tex}]
      %[prove on][check on]
    \end{forthel}
  \end{imports}


  \section*{Factorial and Exponentiation}

  \begin{forthel}
    \begin{proposition}\printlabel{ARITHMETIC_12_8113296594960384}
      Let $n$ be a natural number.
      Assume $n \geq 2$.
      Then $n^{n} > n!$.
    \end{proposition}
    \begin{proof}
      (a) Define $\Phi = \{ n' \in \Nat_{\geq 2} \mid n'^{n'} > n'! \}$.
  
      (1) $\Phi$ contains $2$.
  
      (2) For all $n' \in \Phi$ we have $n' + 1 \in \Phi$. \\
      Proof.
        Let $n' \in \Phi$.
        Then $n' \geq 2$.
  
        Take $X = (n' + 1)^{n'} \cdot (n' + 1)$ and $Y = n'^{n'} \cdot (n' + 1)$.
        Then $X > Y$. \\
        Proof.
          We have $n' + 1 > n'$ and $n' \neq 0$.
          Thus $(n' + 1)^{n'} > n'^{n'}$.
          $n' + 1$ is nonzero.
          Hence $(n' + 1)^{n'} \cdot (n' + 1) > n'^{n'} \cdot (n' + 1)$.
        Qed.
  
        Take $Z = n'! \cdot (n' + 1)$.
        Then $Y > Z$.
        Indeed $n'^{n'} > n'!$ and $n' + 1 \neq 0$.
  
        Hence $(n' + 1)^{n' + 1} = X > Y > Z = (n' + 1)!$.
        Thus $(n' + 1)^{n' + 1} > (n' + 1)!$.
      Qed.
  
      Therefore $\Phi$ contains every element of $\Nat_{\geq 2}$ (by \printref{ARITHMETIC_04_4976599269113856}).
      Consequently $n^{n} > n!$ (by a).
    \end{proof}
  \end{forthel}

  \begin{forthel}
    \begin{proposition}\printlabel{ARITHMETIC_12_5413271156817920}
      Let $n$ be a natural number.
      Assume $n \geq 4$.
      Then $n! > 2^{n}$.
    \end{proposition}
    \begin{proof}
      (a) Define $\Phi = \{ n' \in \Nat_{\geq 4} \mid n'! > 2^{n'} \}$.
  
      (1) $\Phi$ contains $4$. \\
      Proof.
        \[  4!                                  \]
        \[    = 4 \cdot (3 \cdot 2)             \]
        \[    = 2 \cdot (2 \cdot (3 \cdot 2))   \]
        \[    = 3 \cdot (2 \cdot (2 \cdot 2))   \]
        \[    > 2 \cdot (2 \cdot (2 \cdot 2))   \]
        \[    = 2^{4}.                          \]
      Qed.
  
      (2) For all $n' \in \Phi$ we have $n' + 1 \in \Phi$. \\
      Proof.
        Let $n' \in \Phi$.
        Then $n' \geq 4$ and $n'! > 2^{n'}$.

        Take $X = n'! \cdot (n' + 1)$ and $Y = 2^{n'} \cdot (n' + 1)$.
        Then $X > Y$.

        Take $Z = 2^{n'} \cdot 2$.
        Then $Y > Z$.
        Indeed $n' + 1 > 2$.

        Hence $(n' + 1)! = X > Y > Z = 2^{n' + 1}$.
        Thus $(n' + 1)! > 2^{n' + 1}$.
      Qed.
  
      Therefore $\Phi$ contains every element of $\Nat_{\geq 4}$ (by \printref{ARITHMETIC_04_4976599269113856}).
      Consequently $n! > 2^{n}$ (by a).
    \end{proof}
  \end{forthel}
\end{document}