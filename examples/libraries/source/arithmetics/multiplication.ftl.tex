\documentclass{naproche-library}
\libinput{arithmetics-preamble}
\begin{document}
\begin{smodule}{multiplication.ftl}
  \importmodule[libraries]{arithmetics?natural-numbers.ftl}

  \section*{Multiplication}

  \subsection*{Definition}

  \begin{signature}[forthel,id=ARITHMETIC_06_6626346484629504,printid]
    Let $n, m$ be natural numbers.
    $n \cdot m$ is a natural number.

    Let the product of $n$ and $m$ stand for $n \cdot m$.
  \end{signature}

  \begin{axiom}[forthel,id=ARITHMETIC_06_8941041092657152,printid]
    Let $n$ be a natural number.
    Then $n \cdot 0 = 0$.
  \end{axiom}

  \begin{axiom}[forthel,id=ARITHMETIC_06_2211275408932864,printid]
    Let $n, m$ be natural numbers.
    Then $n \cdot (m + 1) = (n \cdot m) + n$.
  \end{axiom}


  \subsection*{Computation Laws}

  \subsubsection*{Distributivity}

  \begin{proposition}[forthel,id=ARITHMETIC_06_9001524774567936,printid]
    Let $n, m, k$ be natural numbers.
    Then $n \cdot (m + k) = (n \cdot m) + (n \cdot k)$.
  \end{proposition}
  \begin{proof}[forthel]
    Define $\Phi = \{ k' \in \Nat \mid n \cdot (m + k') = (n \cdot m) + (n \cdot k') \}$.

    (1) $0$ is an element of $\Phi$.
    Indeed $n \cdot (m + 0)
      = n \cdot m
      = (n \cdot m) + 0
      = (n \cdot m) + (n \cdot 0)$.

    (2) For all $k' \in \Phi$ we have $k' + 1 \in \Phi$. \\
    Proof.
      Let $k'\in \Phi$.
      Then
      \[  n \cdot (m + (k' + 1))                  \]
      \[    = n \cdot ((m + k') + 1)              \]
      \[    = (n \cdot (m + k')) + n              \]
      \[    = ((n \cdot m) + (n \cdot k')) + n    \]
      \[    = (n \cdot m) + ((n \cdot k') + n)    \]
      \[    = (n \cdot m) + (n \cdot (k' + 1)).   \]
      Hence $n \cdot (m + (k' + 1)) = (n \cdot m) + (n \cdot (k' + 1))$.
      Thus $k' + 1 \in \Phi$.
    Qed.

    Thus every natural number is contained in $\Phi$.
    Therefore $n \cdot (m + k) = (n \cdot m) + (n \cdot k)$.
  \end{proof}

  \begin{proposition}[forthel,id=ARITHMETIC_06_5742967566368768,printid]
    Let $n, m, k$ be natural numbers.
    Then $(n + m) \cdot k = (n \cdot k) + (m \cdot k)$.
  \end{proposition}
  \begin{proof}[forthel]
    Define $\Phi = \{ k' \in \Nat \mid (n + m) \cdot k' = (n \cdot k') + (m \cdot k') \}$.

    (1) $0$ belongs to $\Phi$.
    Indeed $(n + m) \cdot 0
      = 0
      = 0 + 0
      = (n \cdot 0) + (m \cdot 0)$.

    (2) For all $k' \in \Phi$ we have $k' + 1 \in \Phi$. \\
    Proof.
      Let $k' \in \Phi$.
      Then
      \[  (n + m) \cdot (k' + 1)                        \]
      \[    = ((n + m) \cdot k') + (n + m)              \]
      \[    = ((n \cdot k') + (m \cdot k')) + (n + m)   \]
      \[    = (((n \cdot k') + (m \cdot k')) + n) + m   \]
      \[    = ((n \cdot k') + ((m \cdot k') + n)) + m   \]
      \[    = ((n \cdot k') + (n + (m \cdot k'))) + m   \]
      \[    = (((n \cdot k') + n) + (m \cdot k')) + m   \]
      \[    = ((n \cdot k') + n) + ((m \cdot k') + m)   \]
      \[    = (n \cdot (k' + 1)) + (m \cdot (k' + 1)).  \]
      Thus $(n + m) \cdot (k' + 1) = (n \cdot (k' + 1)) + (m \cdot (k' + 1))$.
    Qed.

    Thus every natural number is an element of $\Phi$.
    Therefore $(n + m) \cdot k = (n \cdot k) + (m \cdot k)$.
  \end{proof}


  \subsubsection*{Multiplication with $1$ and $2$}

  \begin{proposition}[forthel,id=ARITHMETIC_06_2910559821365248,printid]
    Let $n$ be a natural number.
    Then $n \cdot 1 = n$.
  \end{proposition}
  \begin{proof}[forthel]
    $n \cdot 1
      = n \cdot (0 + 1)
      = (n \cdot 0) + n
      = 0 + n
      = n$.
  \end{proof}

  \begin{corollary}[forthel,id=ARITHMETIC_06_5679541582299136,printid]
    Let $n$ be a natural number.
    Then $n \cdot 2 = n + n$.
  \end{corollary}
  \begin{proof}[forthel]
    $n \cdot 2
      = n \cdot (1 + 1)
      = (n \cdot 1) + n
      = n + n$.
  \end{proof}


  \subsubsection*{Associativity}

  \begin{proposition}[forthel,id=ARITHMETIC_06_347295585402880,printid]
    Let $n, m, k$ be natural numbers.
    Then $n \cdot (m \cdot k) = (n \cdot m) \cdot k$.
  \end{proposition}
  \begin{proof}[forthel]
    Define $\Phi = \{ k' \in \Nat \mid n \cdot (m \cdot k') = (n \cdot m) \cdot k' \}$.

    (1) $0$ is contained in $\Phi$.
    Indeed $n \cdot (m \cdot 0)
      = n \cdot 0
      = 0
      = (n \cdot m) \cdot 0$.

    (2) For all $k' \in \Phi$ we have $k' + 1 \in \Phi$. \\
    Proof.
      Let $k' \in \Phi$.
      Then
      \[  n \cdot (m \cdot (k' + 1))                          \]
      \[    = n \cdot ((m \cdot k') + m)                      \]
      \[    = (n \cdot (m \cdot k')) + (n \cdot m)            \]
      \[    = ((n \cdot m) \cdot k') + (n \cdot m)            \]
      \[    = ((n \cdot m) \cdot k') + ((n \cdot m) \cdot 1)  \]
      \[    = (n \cdot m) \cdot (k' + 1).                     \]
    Qed.

    Hence every natural number is contained in $\Phi$ (by \printref{ARITHMETIC_01_4764664342773760}).
    Thus $n \cdot (m \cdot k) = (n \cdot m) \cdot k$.
  \end{proof}


  \subsubsection*{Commutativity}

  \begin{proposition}[forthel,id=ARITHMETIC_06_1764759896588288,printid]
    Let $n, m$ be natural numbers.
    Then \[ n \cdot m = m \cdot n. \]
  \end{proposition}
  \begin{proof}[forthel]
    Define $\Phi = \{ m' \in \Nat \mid n \cdot m' = m' \cdot n \}$.

    (1) $0$ is contained in $\Phi$. \\
    Proof.
      Define $\Psi = \{ n' \in \Nat \mid n' \cdot 0 = 0 \cdot n' \}$.

      (1a) $0$ is contained in $\Psi$.

      (1b) For all $n' \in \Psi$ we have $n' + 1 \in \Psi$. \\
      Proof.
        Let $n' \in \Psi$.
        Then
        \[ (n' + 1) \cdot 0
          = 0
          = n' \cdot 0
          = 0 \cdot n'
          = (0 \cdot n') + 0
          = 0 \cdot (n' + 1). \]
      Qed.

      Hence every natural number is contained in $\Psi$ (by \printref{ARITHMETIC_01_4764664342773760}).
      Thus $n \cdot 0 = 0 \cdot n$.
    Qed.

    (2) $1$ belongs to $\Phi$. \\
    Proof.
      Define $\Theta = \{ n' \in \Nat \mid n' \cdot 1 = 1 \cdot n' \}$.

      (2a) $0$ is contained in $\Theta$.

      (2b) For all $n' \in \Theta$ we have $n' + 1 \in \Theta$. \\
      Proof.
        Let $n' \in \Theta$.
        Then
        \[  (n' + 1) \cdot 1        \]
        \[    = (n' \cdot 1) + 1    \]
        \[    = (1 \cdot n') + 1    \]
        \[    = 1 \cdot (n' + 1).   \]
      Qed.

      Thus every natural number is contained in $\Theta$ (by \printref{ARITHMETIC_01_4764664342773760}).
      Therefore $n \cdot 1 = 1 \cdot n$.
    Qed.

    (3) For all $m' \in \Phi$ we have $m' + 1 \in \Phi$. \\
    Proof.
      Let $m' \in \Phi$.
      Then
      \[  n \cdot (m' + 1)                \]
      \[    = (n \cdot m') + (n \cdot 1)  \]
      \[    = (m' \cdot n) + (1 \cdot n)  \]
      \[    = (1 \cdot n) + (m' \cdot n)  \]
      \[    = (1 + m') \cdot n            \]
      \[    = (m' + 1) \cdot n.           \]
      Indeed $((1 \cdot n) + (m' \cdot n)) = (1 + m') \cdot n$. %!
    Qed.

    Hence every natural number is contained in $\Phi$ (by \printref{ARITHMETIC_01_4764664342773760}).
    Thus $n \cdot m = m \cdot n$.
  \end{proof}


  \subsubsection*{Non-Existence of Zero-Divisors}

  \begin{proposition}[forthel,id=ARITHMETIC_06_3843962875936768,printid]
    Let $n, m$ be natural numbers such that $n \cdot m = 0$.
    Then $n = 0$ or $m = 0$.
  \end{proposition}
  \begin{proof}[forthel]
    Suppose $n, m \neq 0$.
    Take natural numbers $n', m'$ such that $n = (n' + 1)$ and $m = (m' + 1)$.
    Then
    \[  0                                     \]
    \[    = n \cdot m                         \]
    \[    = (n' + 1) \cdot (m' + 1)           \]
    \[    = ((n' + 1) \cdot m') + (n' + 1)    \]
    \[    = (((n' + 1) \cdot m') + n') + 1.   \]
    Indeed $(n' + 1) \cdot (m' + 1) = ((n' + 1) \cdot m') + (n' + 1)$.
    Hence $0 = k + 1$ for some natural number $k$.
    Contradiction.
  \end{proof}


  \subsubsection*{Cancellation}

  \begin{proposition}[forthel,id=ARITHMETIC_06_31055184658432,printid]
    Let $n, m, k$ be natural numbers.
    Assume $k \neq 0$.
    If $n \cdot k = m \cdot k$ then $n = m$.
  \end{proposition}
  \begin{proof}[forthel]
    Define $\Phi = \{ n' \in \Nat \mid$ for all $m' \in \Nat$ if $n' \cdot k = m' \cdot k$ and $k \neq 0$ then $n' = m' \}$.

    (1) $0$ is contained in $\Phi$. \\
    Proof.
      Let $m' \in \Nat$.
      Assume $0 \cdot k = m' \cdot k$ and $k \neq 0$.
      Then $m' \cdot k = 0$.
      Hence $m' = 0$ or $k = 0$.
      Thus $m' = 0$.
    Qed.

    (2) For all $n' \in \Phi$ we have $n' + 1 \in \Phi$. \\
    Proof.
      Let $n' \in \Phi$.

      Let us show that for all $m' \in \Nat$ if $(n' + 1) \cdot k = m' \cdot k$ and $k \neq 0$ then $n' + 1 = m'$.
        Let $m' \in \Nat$.
        Assume $(n' + 1) \cdot k = m' \cdot k$ and $k \neq 0$.

        Case $m' = 0$.
          Then $(n' + 1) \cdot k = 0$.
          Hence $n' + 1 = 0$.
          Contradiction.
        End.

        Case $m' \neq 0$.
          Take a natural number $l$ such that $m' = l + 1$.
          Then $(n' + 1) \cdot k = (l + 1) \cdot k$.
          Hence $(n' \cdot k) + k
            = (n' \cdot k) + (1 \cdot k)
            = (n' \cdot k) + k
            = (l + 1) \cdot k
            = (l \cdot k) + (1 \cdot k)
            = (l \cdot k) + k$.
          Thus $n' \cdot k = l \cdot k$ (by \printref{ARITHMETIC_03_3137702874578944}).
          Indeed $n' \cdot k$ and $l \cdot k$ are natural numbers.
          Then we have $n' = l$.
          Indeed if $n' \cdot k = l \cdot k$ and $k \neq 0$ then $n' = l$.
          Therefore $n' + 1 = l + 1 = m'$.
        End.
      End.

      Hence $n' + 1 \in \Phi$.
    Qed.

    Thus every natural number is contained in $\Phi$ (by \printref{ARITHMETIC_01_4764664342773760}).
    Therefore if $n \cdot k = m \cdot k$ then $n = m$.
  \end{proof}

  \begin{corollary}[forthel,id=ARITHMETIC_06_8575191374364672,printid]
    Let $n, m, k$ be natural numbers.
    Assume $k \neq 0$.
    If $k \cdot n = k \cdot m$ then $n = m$.
  \end{corollary}
  \begin{proof}[forthel]
    Assume $k \cdot n = k \cdot m$.
    We have $k \cdot n = n \cdot k$ and $k \cdot m = m \cdot k$.
    Hence $n \cdot k = m \cdot k$.
    Thus $n = m$ (by \printref{ARITHMETIC_06_31055184658432}).
  \end{proof}
\end{smodule}
\end{document}
