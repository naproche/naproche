\documentclass[10pt]{article}
\usepackage[english]{babel}
\usepackage{../../../../lib/tex/naproche}
% Top-level sections and proofs

\newtheoremstyle{foundations}% name of the style to be used
  {\dimexpr\ftlparskip+4pt\relax}% measure of space to leave above the theorem.
  % (We must add \ftlparskip since otherwise the name of the label would not fit
  % into the border drawn around the environment.)
  {4pt}% measure of space to leave below the theorem.
  {\normalfont}% name of font to use in the body of the theorem
  {0pt}% measure of space to indent
  {\bfseries}% name of head font
  {}% punctuation between head and body
  { }% space after theorem head; " " = normal inter-word space
  {\thmname{#1}\thmnumber{ #2}\thmnote{ (#3)}.}

\let\definition\relax
\let\enddefinition\relax
\let\signature\relax
\let\endsignature\relax
\let\axiom\relax
\let\endaxiom\relax
\let\theorem\relax
\let\endtheorem\relax
\let\proposition\relax
\let\endproposition\relax
\let\lemma\relax
\let\endlemma\relax
\let\corollary\relax
\let\endcorollary\relax

\theoremstyle{foundations}

\newmdtheoremenv[
  backgroundcolor=lightgray!10,
  linecolor=black,
  innertopmargin=1em
]{definition}{Definition}[section]
\newmdtheoremenv[
  backgroundcolor=lightgray!10,
  linecolor=black,
  innertopmargin=1em
]{signature}[definition]{Signature}
\newmdtheoremenv[
  backgroundcolor=lightgray!10,
  linecolor=black,
  innertopmargin=1em
]{axiom}[definition]{Axiom}
\newmdtheoremenv[
  backgroundcolor=lightgray!10,
  linecolor=black,
  innertopmargin=1em
]{theorem}[definition]{Theorem}
\newmdtheoremenv[
  backgroundcolor=lightgray!10,
  linecolor=black,
  innertopmargin=1em
]{proposition}[definition]{Proposition}
\newmdtheoremenv[
  backgroundcolor=lightgray!10,
  linecolor=black,
  innertopmargin=1em
]{lemma}[definition]{Lemma}
\newmdtheoremenv[
  backgroundcolor=lightgray!10,
  linecolor=black,
  innertopmargin=1em
]{corollary}[definition]{Corollary}


% Hide imports

\usepackage{scontents}
\newenvsc{imports}[store-env=imports,print-env=false]


% Labels of top-level sections

\renewcommand{\showlabelfont}{\scriptsize\ttfamily\color{gray}}


% Non-symbolic "\implies" and "\iff"

\renewcommand{\implies}{\quad \text{implies} \quad}
\renewcommand{\iff}{\quad \text{iff} \quad}

\usepackage{amssymb}
\usepackage{pgffor}

\newcommand{\id}{\textrm{id}}
\newcommand{\range}{\textrm{range}}
\newcommand{\const}[2]{\textrm{const}_{#1}^{#2}}
\newcommand{\symdiff}{\mathop{\triangle}}
\newcommand{\pow}{\mathcal{P}}
\newcommand{\onto}{\twoheadrightarrow}
\newcommand{\into}{\hookrightarrow}

% E.g. `\set{1, 2, 3, 4}` prints the numbers 1, 2, 3 and 4 as a comma
% separated list which is enclosed within curly braces.
\def\set#1{\ensuremath{
  \{\foreach[count=\i]\x in {#1}{
    \ifnum\i>1,\,\fi
    \x
  }\}
}}

\useackage{amssymb}

\newcommand{\Nat}{\mathbb{N}}
\newcommand{\Prime}{\mathbb{P}}
\renewcommand{\succ}{\textrm{succ}}
\newcommand{\pred}{\textrm{pred}}
\newcommand{\add}{\textrm{add}}
\newcommand{\mul}{\textrm{mul}}
\renewcommand{\exp}{\textrm{exp}}
\newcommand{\fac}{\textrm{fac}}
\renewcommand{\div}{\mathrel{\textrm{div}}}
\renewcommand{\mod}{\mathrel{\textrm{mod}}}

\begin{document}
  \begin{imports}
    \begin{forthel}
      %[prove off][check off]
      [readtex \path{libraries/source/arithmetics/natural-numbers.ftl.tex}]
      %[prove on][check on]
    \end{forthel}
  \end{imports}


  \section*{Multiplication}

  \subsection*{Definition}

  \begin{forthel}
    \begin{signature}\printlabel{ARITHMETIC_06_6626346484629504}
      Let $n, m$ be natural numbers.
      $n \cdot m$ is a natural number.
    \end{signature}

    Let the product of $n$ and $m$ stand for $n \cdot m$.
  \end{forthel}

  \begin{forthel}
    \begin{axiom}\printlabel{ARITHMETIC_06_8941041092657152}
      Let $n$ be a natural number.
      Then $n \cdot 0 = 0$.
    \end{axiom}
  \end{forthel}

  \begin{forthel}
    \begin{axiom}\printlabel{ARITHMETIC_06_2211275408932864}
      Let $n, m$ be natural numbers.
      Then $n \cdot (m + 1) = (n \cdot m) + n$.
    \end{axiom}
  \end{forthel}


  \subsection*{Computation Laws}

  \subsubsection*{Distributivity}

  \begin{forthel}
    \begin{proposition}\printlabel{ARITHMETIC_06_9001524774567936}
      Let $n, m, k$ be natural numbers.
      Then $n \cdot (m + k) = (n \cdot m) + (n \cdot k)$.
    \end{proposition}
    \begin{proof}
      Define $\Phi = \{ k' \in \Nat \mid n \cdot (m + k') = (n \cdot m) + (n \cdot k') \}$.

      (1) $0$ is an element of $\Phi$.
      Indeed $n \cdot (m + 0)
        = n \cdot m
        = (n \cdot m) + 0
        = (n \cdot m) + (n \cdot 0)$.

      (2) For all $k' \in \Phi$ we have $k' + 1 \in \Phi$. \\
      Proof.
        Let $k'\in \Phi$.
        Then
        \[  n \cdot (m + (k' + 1))                  \]
        \[    = n \cdot ((m + k') + 1)              \]
        \[    = (n \cdot (m + k')) + n              \]
        \[    = ((n \cdot m) + (n \cdot k')) + n    \]
        \[    = (n \cdot m) + ((n \cdot k') + n)    \]
        \[    = (n \cdot m) + (n \cdot (k' + 1)).   \]
        Hence $n \cdot (m + (k' + 1)) = (n \cdot m) + (n \cdot (k' + 1))$.
        Thus $k' + 1 \in \Phi$.
      Qed.

      Thus every natural number is contained in $\Phi$.
      Therefore $n \cdot (m + k) = (n \cdot m) + (n \cdot k)$.
    \end{proof}
  \end{forthel}

  \begin{forthel}
    \begin{proposition}\printlabel{ARITHMETIC_06_5742967566368768}
      Let $n, m, k$ be natural numbers.
      Then $(n + m) \cdot k = (n \cdot k) + (m \cdot k)$.
    \end{proposition}
    \begin{proof}
      Define $\Phi = \{ k' \in \Nat \mid (n + m) \cdot k' = (n \cdot k') + (m \cdot k') \}$.

      (1) $0$ belongs to $\Phi$.
      Indeed $(n + m) \cdot 0
        = 0
        = 0 + 0
        = (n \cdot 0) + (m \cdot 0)$.

      (2) For all $k' \in \Phi$ we have $k' + 1 \in \Phi$. \\
      Proof.
        Let $k' \in \Phi$.
        Then
        \[  (n + m) \cdot (k' + 1)                        \]
        \[    = ((n + m) \cdot k') + (n + m)              \]
        \[    = ((n \cdot k') + (m \cdot k')) + (n + m)   \]
        \[    = (((n \cdot k') + (m \cdot k')) + n) + m   \]
        \[    = ((n \cdot k') + ((m \cdot k') + n)) + m   \]
        \[    = ((n \cdot k') + (n + (m \cdot k'))) + m   \]
        \[    = (((n \cdot k') + n) + (m \cdot k')) + m   \]
        \[    = ((n \cdot k') + n) + ((m \cdot k') + m)   \]
        \[    = (n \cdot (k' + 1)) + (m \cdot (k' + 1)).  \]
        Thus $(n + m) \cdot (k' + 1) = (n \cdot (k' + 1)) + (m \cdot (k' + 1))$.
      Qed.

      Thus every natural number is an element of $\Phi$.
      Therefore $(n + m) \cdot k = (n \cdot k) + (m \cdot k)$.
    \end{proof}
  \end{forthel}


  \subsubsection*{Multiplication with $1$ and $2$}

  \begin{forthel}
    \begin{proposition}\printlabel{ARITHMETIC_06_2910559821365248}
      Let $n$ be a natural number.
      Then $n \cdot 1 = n$.
    \end{proposition}
    \begin{proof}
      $n \cdot 1
        = n \cdot (0 + 1)
        = (n \cdot 0) + n
        = 0 + n
        = n$.
    \end{proof}
  \end{forthel}

  \begin{forthel}
    \begin{corollary}\printlabel{ARITHMETIC_06_5679541582299136}
      Let $n$ be a natural number.
      Then $n \cdot 2 = n + n$.
    \end{corollary}
    \begin{proof}
      $n \cdot 2
        = n \cdot (1 + 1)
        = (n \cdot 1) + n
        = n + n$.
    \end{proof}
  \end{forthel}


  \subsubsection*{Associativity}

  \begin{forthel}
    \begin{proposition}\printlabel{ARITHMETIC_06_347295585402880}
      Let $n, m, k$ be natural numbers.
      Then $n \cdot (m \cdot k) = (n \cdot m) \cdot k$.
    \end{proposition}
    \begin{proof}
      Define $\Phi = \{ k' \in \Nat \mid n \cdot (m \cdot k') = (n \cdot m) \cdot k' \}$.

      (1) $0$ is contained in $\Phi$.
      Indeed $n \cdot (m \cdot 0)
        = n \cdot 0
        = 0
        = (n \cdot m) \cdot 0$.

      (2) For all $k' \in \Phi$ we have $k' + 1 \in \Phi$. \\
      Proof.
        Let $k' \in \Phi$.
        Then
        \[  n \cdot (m \cdot (k' + 1))                          \]
        \[    = n \cdot ((m \cdot k') + m)                      \]
        \[    = (n \cdot (m \cdot k')) + (n \cdot m)            \]
        \[    = ((n \cdot m) \cdot k') + (n \cdot m)            \]
        \[    = ((n \cdot m) \cdot k') + ((n \cdot m) \cdot 1)  \]
        \[    = (n \cdot m) \cdot (k' + 1).                     \]
      Qed.

      Hence every natural number is contained in $\Phi$ (by \printref{ARITHMETIC_01_4764664342773760}).
      Thus $n \cdot (m \cdot k) = (n \cdot m) \cdot k$.
    \end{proof}
  \end{forthel}


  \subsubsection*{Commutativity}

  \begin{forthel}
    \begin{proposition}\printlabel{ARITHMETIC_06_1764759896588288}
      Let $n, m$ be natural numbers.
      Then \[ n \cdot m = m \cdot n. \]
    \end{proposition}
    \begin{proof}
      Define $\Phi = \{ m' \in \Nat \mid n \cdot m' = m' \cdot n \}$.

      (1) $0$ is contained in $\Phi$. \\
      Proof.
        Define $\Psi = \{ n' \in \Nat \mid n' \cdot 0 = 0 \cdot n' \}$.

        (1a) $0$ is contained in $\Psi$.

        (1b) For all $n' \in \Psi$ we have $n' + 1 \in \Psi$. \\
        Proof.
          Let $n' \in \Psi$.
          Then
          \[ (n' + 1) \cdot 0
            = 0
            = n' \cdot 0
            = 0 \cdot n'
            = (0 \cdot n') + 0
            = 0 \cdot (n' + 1). \]
        Qed.

        Hence every natural number is contained in $\Psi$ (by \printref{ARITHMETIC_01_4764664342773760}).
        Thus $n \cdot 0 = 0 \cdot n$.
      Qed.

      (2) $1$ belongs to $\Phi$. \\
      Proof.
        Define $\Theta = \{ n' \in \Nat \mid n' \cdot 1 = 1 \cdot n' \}$.

        (2a) $0$ is contained in $\Theta$.

        (2b) For all $n' \in \Theta$ we have $n' + 1 \in \Theta$. \\
        Proof.
          Let $n' \in \Theta$.
          Then
          \[  (n' + 1) \cdot 1        \]
          \[    = (n' \cdot 1) + 1    \]
          \[    = (1 \cdot n') + 1    \]
          \[    = 1 \cdot (n' + 1).   \]
        Qed.

        Thus every natural number is contained in $\Theta$ (by \printref{ARITHMETIC_01_4764664342773760}).
        Therefore $n \cdot 1 = 1 \cdot n$.
      Qed.

      (3) For all $m' \in \Phi$ we have $m' + 1 \in \Phi$. \\
      Proof.
        Let $m' \in \Phi$.
        Then
        \[  n \cdot (m' + 1)                \]
        \[    = (n \cdot m') + (n \cdot 1)  \]
        \[    = (m' \cdot n) + (1 \cdot n)  \]
        \[    = (1 \cdot n) + (m' \cdot n)  \]
        \[    = (1 + m') \cdot n            \]
        \[    = (m' + 1) \cdot n.           \]
        Indeed $((1 \cdot n) + (m' \cdot n)) = (1 + m') \cdot n$. %!
      Qed.

      Hence every natural number is contained in $\Phi$ (by \printref{ARITHMETIC_01_4764664342773760}).
      Thus $n \cdot m = m \cdot n$.
    \end{proof}
  \end{forthel}


  \subsubsection*{Non-Existence of Zero-Divisors}

  \begin{forthel}
    \begin{proposition}\printlabel{ARITHMETIC_06_3843962875936768}
      Let $n, m$ be natural numbers such that $n \cdot m = 0$.
      Then $n = 0$ or $m = 0$.
    \end{proposition}
    \begin{proof}
      Suppose $n, m \neq 0$.
      Take natural numbers $n', m'$ such that $n = (n' + 1)$ and $m = (m' + 1)$.
      Then
      \[  0                                     \]
      \[    = n \cdot m                         \]
      \[    = (n' + 1) \cdot (m' + 1)           \]
      \[    = ((n' + 1) \cdot m') + (n' + 1)    \]
      \[    = (((n' + 1) \cdot m') + n') + 1.   \]
      Indeed $(n' + 1) \cdot (m' + 1) = ((n' + 1) \cdot m') + (n' + 1)$.
      Hence $0 = k + 1$ for some natural number $k$.
      Contradiction.
    \end{proof}
  \end{forthel}


  \subsubsection*{Cancellation}

  \begin{forthel}
    \begin{proposition}\printlabel{ARITHMETIC_06_31055184658432}
      Let $n, m, k$ be natural numbers.
      Assume $k \neq 0$.
      If $n \cdot k = m \cdot k$ then $n = m$.
    \end{proposition}
    \begin{proof}
      Define $\Phi = \{ n' \in \Nat \mid$ for all $m' \in \Nat$ if $n' \cdot k = m' \cdot k$ and $k \neq 0$ then $n' = m' \}$.

      (1) $0$ is contained in $\Phi$. \\
      Proof.
        Let $m' \in \Nat$.
        Assume $0 \cdot k = m' \cdot k$ and $k \neq 0$.
        Then $m' \cdot k = 0$.
        Hence $m' = 0$ or $k = 0$.
        Thus $m' = 0$.
      Qed.

      (2) For all $n' \in \Phi$ we have $n' + 1 \in \Phi$. \\
      Proof.
        Let $n' \in \Phi$.

        Let us show that for all $m' \in \Nat$ if $(n' + 1) \cdot k = m' \cdot k$ and $k \neq 0$ then $n' + 1 = m'$.
          Let $m' \in \Nat$.
          Assume $(n' + 1) \cdot k = m' \cdot k$ and $k \neq 0$.

          Case $m' = 0$.
            Then $(n' + 1) \cdot k = 0$.
            Hence $n' + 1 = 0$.
            Contradiction.
          End.

          Case $m' \neq 0$.
            Take a natural number $l$ such that $m' = l + 1$.
            Then $(n' + 1) \cdot k = (l + 1) \cdot k$.
            Hence $(n' \cdot k) + k
              = (n' \cdot k) + (1 \cdot k)
              = (n' \cdot k) + k
              = (l + 1) \cdot k
              = (l \cdot k) + (1 \cdot k)
              = (l \cdot k) + k$.
            Thus $n' \cdot k = l \cdot k$ (by \printref{ARITHMETIC_03_3137702874578944}).
            Indeed $n' \cdot k$ and $l \cdot k$ are natural numbers.
            Then we have $n' = l$.
            Indeed if $n' \cdot k = l \cdot k$ and $k \neq 0$ then $n' = l$.
            Therefore $n' + 1 = l + 1 = m'$.
          End.
        End.

        Hence $n' + 1 \in \Phi$.
      Qed.

      Thus every natural number is contained in $\Phi$ (by \printref{ARITHMETIC_01_4764664342773760}).
      Therefore if $n \cdot k = m \cdot k$ then $n = m$.
    \end{proof}
  \end{forthel}

  \begin{forthel}
    \begin{corollary}\printlabel{ARITHMETIC_06_8575191374364672}
      Let $n, m, k$ be natural numbers.
      Assume $k \neq 0$.
      If $k \cdot n = k \cdot m$ then $n = m$.
    \end{corollary}
    \begin{proof}
      Assume $k \cdot n = k \cdot m$.
      We have $k \cdot n = n \cdot k$ and $k \cdot m = m \cdot k$.
      Hence $n \cdot k = m \cdot k$.
      Thus $n = m$ (by \printref{ARITHMETIC_06_31055184658432}).
    \end{proof}
  \end{forthel}
\end{document}
