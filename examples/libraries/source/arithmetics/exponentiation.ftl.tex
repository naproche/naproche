\documentclass[10pt]{article}
\usepackage[english]{babel}
\usepackage{../../../meta-inf/lib/naproche}
% Top-level sections and proofs

\newtheoremstyle{foundations}% name of the style to be used
  {\dimexpr\ftlparskip+4pt\relax}% measure of space to leave above the theorem.
  % (We must add \ftlparskip since otherwise the name of the label would not fit
  % into the border drawn around the environment.)
  {4pt}% measure of space to leave below the theorem.
  {\normalfont}% name of font to use in the body of the theorem
  {0pt}% measure of space to indent
  {\bfseries}% name of head font
  {}% punctuation between head and body
  { }% space after theorem head; " " = normal inter-word space
  {\thmname{#1}\thmnumber{ #2}\thmnote{ (#3)}.}

\let\definition\relax
\let\enddefinition\relax
\let\signature\relax
\let\endsignature\relax
\let\axiom\relax
\let\endaxiom\relax
\let\theorem\relax
\let\endtheorem\relax
\let\proposition\relax
\let\endproposition\relax
\let\lemma\relax
\let\endlemma\relax
\let\corollary\relax
\let\endcorollary\relax

\theoremstyle{foundations}

\newmdtheoremenv[
  backgroundcolor=lightgray!10,
  linecolor=black,
  innertopmargin=1em
]{definition}{Definition}[section]
\newmdtheoremenv[
  backgroundcolor=lightgray!10,
  linecolor=black,
  innertopmargin=1em
]{signature}[definition]{Signature}
\newmdtheoremenv[
  backgroundcolor=lightgray!10,
  linecolor=black,
  innertopmargin=1em
]{axiom}[definition]{Axiom}
\newmdtheoremenv[
  backgroundcolor=lightgray!10,
  linecolor=black,
  innertopmargin=1em
]{theorem}[definition]{Theorem}
\newmdtheoremenv[
  backgroundcolor=lightgray!10,
  linecolor=black,
  innertopmargin=1em
]{proposition}[definition]{Proposition}
\newmdtheoremenv[
  backgroundcolor=lightgray!10,
  linecolor=black,
  innertopmargin=1em
]{lemma}[definition]{Lemma}
\newmdtheoremenv[
  backgroundcolor=lightgray!10,
  linecolor=black,
  innertopmargin=1em
]{corollary}[definition]{Corollary}


% Hide imports

\usepackage{scontents}
\newenvsc{imports}[store-env=imports,print-env=false]


% Labels of top-level sections

\renewcommand{\showlabelfont}{\scriptsize\ttfamily\color{gray}}


% Non-symbolic "\implies" and "\iff"

\renewcommand{\implies}{\quad \text{implies} \quad}
\renewcommand{\iff}{\quad \text{iff} \quad}

\usepackage{amssymb}
\usepackage{pgffor}

\newcommand{\id}{\textrm{id}}
\newcommand{\range}{\textrm{range}}
\newcommand{\const}[2]{\textrm{const}_{#1}^{#2}}
\newcommand{\symdiff}{\mathop{\triangle}}
\newcommand{\pow}{\mathcal{P}}
\newcommand{\onto}{\twoheadrightarrow}
\newcommand{\into}{\hookrightarrow}

% E.g. `\set{1, 2, 3, 4}` prints the numbers 1, 2, 3 and 4 as a comma
% separated list which is enclosed within curly braces.
\def\set#1{\ensuremath{
  \{\foreach[count=\i]\x in {#1}{
    \ifnum\i>1,\,\fi
    \x
  }\}
}}

\useackage{amssymb}

\newcommand{\Nat}{\mathbb{N}}
\newcommand{\Prime}{\mathbb{P}}
\renewcommand{\succ}{\textrm{succ}}
\newcommand{\pred}{\textrm{pred}}
\newcommand{\add}{\textrm{add}}
\newcommand{\mul}{\textrm{mul}}
\renewcommand{\exp}{\textrm{exp}}
\newcommand{\fac}{\textrm{fac}}
\renewcommand{\div}{\mathrel{\textrm{div}}}
\renewcommand{\mod}{\mathrel{\textrm{mod}}}

\begin{document}
  \begin{imports}
    \begin{forthel}
      %[prove off][check off]
      [readtex \path{libraries/source/arithmetics/multiplication.ftl.tex}]
      %[prove on][check on]
    \end{forthel}
  \end{imports}


  \section*{Exponentiation}

  \subsection*{Definition}

  \begin{forthel}
    \begin{signature}\printlabel{ARITHMETIC_09_3663815629602816}
      Let $n, m$ be natural numbers.
      $n^{m}$ is a natural number.
    \end{signature}
  \end{forthel}

  \begin{forthel}
    \begin{axiom}\printlabel{ARITHMETIC_09_5368818025103360}
      Let $n$ be a natural number.
      Then $n^{0} = 1$.
    \end{axiom}
  \end{forthel}

  \begin{forthel}
    \begin{axiom}\printlabel{ARITHMETIC_09_4140498660884480}
      Let $n, m$ be natural numbers.
      Then $n^{m + 1} = n^{m} \cdot n$.
    \end{axiom}
  \end{forthel}


  \subsection*{Computation Laws}

  \subsubsection*{Exponentiation with $0$, $1$ and $2$}

  \begin{forthel}
    \begin{proposition}\printlabel{ARITHMETIC_09_4673644676513792}
      Let $n$ be a natural number.
      Assume $n \neq 0$.
      Then $0^{n} = 0$.
    \end{proposition}
    \begin{proof}
      Take a natural number $m$ such that $n = m + 1$.
      Then $0^{n}
        = 0^{m + 1}
        = 0^{m} \cdot 0
        = 0$.
      Indeed $0^{m + 1} = 0^{m} \cdot 0$.
    \end{proof}
  \end{forthel}

  \begin{forthel}
    \begin{proposition}\printlabel{ARITHMETIC_09_7376849881530368}
      Let $n$ be a natural number.
      Then $1^{n} = 1$.
    \end{proposition}
    \begin{proof}
      Define $\Phi = \{ n' \in \Nat \mid 1^{n'} = 1 \}$.

      (1) $\Phi$ contains $0$.

      (2) For all $n' \in \Phi$ we have $n' + 1 \in \Phi$. \\
      Proof.
        Let $n' \in \Phi$.
        Then $1^{n' + 1}
          = 1^{n'} \cdot 1
          = 1 \cdot 1
          = 1$.
      Qed.

      Hence every natural number is contained in $\Phi$ (by \printref{ARITHMETIC_01_4764664342773760}).
      Thus $1^{n} = 1$.
    \end{proof}
  \end{forthel}

  \begin{forthel}
    \begin{proposition}\printlabel{ARITHMETIC_09_4975279749464064}
      Let $n$ be a natural number.
      Then $n^{1} = n$.
    \end{proposition}
    \begin{proof}
      We have $n^{1}
        = n^{0 + 1}
        = n^{0} \cdot n
        = 1 \cdot n
        = n$.
    \end{proof}
  \end{forthel}

  \begin{forthel}
    \begin{proposition}\printlabel{ARITHMETIC_09_8513812055457792}
      Let $n$ be a natural number.
      Then $n^{2} = n \cdot n$.
    \end{proposition}
    \begin{proof}
      We have $n^{2}
        = n^{1 + 1}
        = n^{1} \cdot n
        = n \cdot n$.
    \end{proof}
  \end{forthel}


  \subsubsection*{Sums as Exponents}

  \begin{forthel}
    \begin{proposition}\printlabel{ARITHMETIC_09_8152207530655744}
      Let $n, m, k$ be natural numbers.
      Then $k^{n + m} = k^{n} \cdot k^{m}$.
    \end{proposition}
    \begin{proof}
      Define $\Phi = \{ m' \in \Nat \mid k^{n + m'} = k^{n} \cdot k^{m'} \}$.

      (1) $\Phi$ contains $0$. \\
      Indeed $k^{n + 0}
        = k^{n}
        = k^{n} \cdot 1
        = k^{n} \cdot k^{0}$.

      (2) For all $m' \in \Phi$ we have $m' + 1 \in \Phi$. \\
      Proof.
        Let $m' \in \Phi$.
        Then
        \[  k^{n + (m' + 1)}                  \]
        \[    = k^{(n + m') + 1}              \]
        \[    = k^{n + m'} \cdot k            \]
        \[    = (k^{n} \cdot k^{m'}) \cdot k  \]
        \[    = k^{n} \cdot (k^{m'} \cdot k)  \]
        \[    = k^{n} \cdot k^{m' + 1}.       \]
      Qed.

      Hence every natural number is contained in $\Phi$ (by \printref{ARITHMETIC_01_4764664342773760}).
      Thus $k^{n + m} = k^{n} \cdot k^{m}$.
    \end{proof}
  \end{forthel}


  \subsubsection*{Products as Exponents}

  \begin{forthel}
    \begin{proposition}\printlabel{ARITHMETIC_09_7827956571308032}
      Let $n, m, k$ be natural numbers.
      Then $n^{m \cdot k} = (n^{m})^{k}$.
    \end{proposition}
    \begin{proof}
      Define $\Phi = \{ k' \in \Nat \mid n^{m \cdot k'} = (n^{m})^{k'} \}$.

      (1) $\Phi$ contains $0$.
      Indeed $(n^{m})^{0}
        = 1
        = n^{0}
        = n^{m \cdot 0}$.

      (2) For all $k' \in \Phi$ we have $k' + 1 \in \Phi$. \\
      Proof.
        Let $k' \in \Phi$.
        Then
        \[  (n^{m})^{k' + 1}                \]
        \[    = (n^{m})^{k'} \cdot n^{m}    \]
        \[    = n^{m \cdot k'} \cdot n^{m}  \]
        \[    = n^{(m \cdot k') + m}        \]
        \[    = n^{m \cdot (k' + 1)}.       \]
      Qed.

      Therefore every natural number is contained in $\Phi$ (by \printref{ARITHMETIC_01_4764664342773760}).
      Consequently $n^{m \cdot k} = (n^{m})^{k}$.
    \end{proof}
  \end{forthel}


  \subsubsection*{Products as Base}

  \begin{forthel}
    \begin{proposition}\printlabel{ARITHMETIC_09_2563032276271104}
      Let $n, m, k$ be natural numbers.
      Then $(n \cdot m)^{k} = n^{k} \cdot m^{k}$.
    \end{proposition}
    \begin{proof}
      Define $\Phi = \{ k' \in \Nat \mid (n \cdot m)^{k'} = n^{k'} \cdot m^{k'} \}$.

      (1) $\Phi$ contains $0$.
      Indeed $((n \cdot m)^{0})
        = 1
        = 1 \cdot 1
        = n^{0} \cdot m^{0}$. %!

      (2) For all $k' \in \Phi$ we have $k' + 1 \in \Phi$. \\
      Proof.
        Let $k' \in \Phi$.

        Let us show that $(n^{k'} \cdot m^{k'}) \cdot (n \cdot m) = (n^{k'} \cdot n) \cdot (m^{k'} \cdot m)$.
          \[  (n^{k'} \cdot m^{k'}) \cdot (n \cdot m)       \]
          \[    = ((n^{k'} \cdot m^{k'}) \cdot n) \cdot m   \]
          \[    = (n^{k'} \cdot (m^{k'} \cdot n)) \cdot m   \]
          \[    = (n^{k'} \cdot (n \cdot m^{k'})) \cdot m   \]
          \[    = ((n^{k'} \cdot n) \cdot m^{k'}) \cdot m   \]
          \[    = (n^{k'} \cdot n) \cdot (m^{k'} \cdot m).  \]
        Qed.

        Hence
        \[  (n \cdot m)^{k' + 1}                          \]
        \[    = (n \cdot m)^{k'} \cdot (n \cdot m)        \]
        \[    = (n^{k'} \cdot m^{k'}) \cdot (n \cdot m)   \]
        \[    = (n^{k'} \cdot n) \cdot (m^{k'} \cdot m)   \]
        \[    = n^{k' + 1} \cdot m^{k' + 1}.              \]
      Qed.

      Therefore every natural number is contained in $\Phi$ (by \printref{ARITHMETIC_01_4764664342773760}).
      Consequently $(n \cdot m)^{k} = n^{k} \cdot m^{k}$.
    \end{proof}
  \end{forthel}


  \subsubsection*{Zeroes of Exponentiation}

  \begin{forthel}
    \begin{proposition}\printlabel{ARITHMETIC_09_3860221447372800}
      Let $n, m$ be natural numbers.
      Then $n^{m} = 0$ iff $n = 0$ and $m \neq 0$.
    \end{proposition}
    \begin{proof}
      Case $n^{m} = 0$.
        Define $\Phi = \{ m' \in \Nat \mid$ if $n^{m'} = 0$ then $n = 0$ and $m' \neq 0 \}$.

        (1) $\Phi$ contains $0$.
        Indeed if $n^{0} = 0$ then we have a contradiction.

        (2) For all $m' \in \Phi$ we have $m' + 1 \in \Phi$. \\
        Proof.
          Let $m' \in \Phi$.

          Let us show that if $n^{m' + 1} = 0$ then $n = 0$ and $m' + 1 \neq 0$.
            Assume $n^{m' + 1} = 0$.
            Then $0 = n^{m' + 1} = n^{m'} \cdot n$.
            Hence $n^{m'} = 0$ or $n = 0$.
            We have $m' + 1 \neq 0$ and if $n^{m'} = 0$ then $n = 0$.
            Hence $n = 0$ and $m' + 1 \neq 0$.
          End.
        Qed.

        Thus every natural number is contained in $\Phi$ (by \printref{ARITHMETIC_01_4764664342773760}).
        Consequently $m \in \Phi$.
        Therefore $n = 0$ and $m \neq 0$.
      End.

      Case $n = 0$ and $m \neq 0$.
        Take a natural number $k$ such that $m = k + 1$.
        Then $n^{m}
          = n^{k + 1}
          = n^{k} \cdot n
          = 0^{k} \cdot 0
          = 0$.
      End.
    \end{proof}
  \end{forthel}
\end{document}
