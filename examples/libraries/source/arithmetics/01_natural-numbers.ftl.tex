\documentclass[10pt]{article}
\usepackage[english]{babel}
\usepackage{../../../../lib/tex/naproche}
% Top-level sections and proofs

\newtheoremstyle{foundations}% name of the style to be used
  {\dimexpr\ftlparskip+4pt\relax}% measure of space to leave above the theorem.
  % (We must add \ftlparskip since otherwise the name of the label would not fit
  % into the border drawn around the environment.)
  {4pt}% measure of space to leave below the theorem.
  {\normalfont}% name of font to use in the body of the theorem
  {0pt}% measure of space to indent
  {\bfseries}% name of head font
  {}% punctuation between head and body
  { }% space after theorem head; " " = normal inter-word space
  {\thmname{#1}\thmnumber{ #2}\thmnote{ (#3)}.}

\let\definition\relax
\let\enddefinition\relax
\let\signature\relax
\let\endsignature\relax
\let\axiom\relax
\let\endaxiom\relax
\let\theorem\relax
\let\endtheorem\relax
\let\proposition\relax
\let\endproposition\relax
\let\lemma\relax
\let\endlemma\relax
\let\corollary\relax
\let\endcorollary\relax

\theoremstyle{foundations}

\newmdtheoremenv[
  backgroundcolor=lightgray!10,
  linecolor=black,
  innertopmargin=1em
]{definition}{Definition}[section]
\newmdtheoremenv[
  backgroundcolor=lightgray!10,
  linecolor=black,
  innertopmargin=1em
]{signature}[definition]{Signature}
\newmdtheoremenv[
  backgroundcolor=lightgray!10,
  linecolor=black,
  innertopmargin=1em
]{axiom}[definition]{Axiom}
\newmdtheoremenv[
  backgroundcolor=lightgray!10,
  linecolor=black,
  innertopmargin=1em
]{theorem}[definition]{Theorem}
\newmdtheoremenv[
  backgroundcolor=lightgray!10,
  linecolor=black,
  innertopmargin=1em
]{proposition}[definition]{Proposition}
\newmdtheoremenv[
  backgroundcolor=lightgray!10,
  linecolor=black,
  innertopmargin=1em
]{lemma}[definition]{Lemma}
\newmdtheoremenv[
  backgroundcolor=lightgray!10,
  linecolor=black,
  innertopmargin=1em
]{corollary}[definition]{Corollary}


% Hide imports

\usepackage{scontents}
\newenvsc{imports}[store-env=imports,print-env=false]


% Labels of top-level sections

\renewcommand{\showlabelfont}{\scriptsize\ttfamily\color{gray}}


% Non-symbolic "\implies" and "\iff"

\renewcommand{\implies}{\quad \text{implies} \quad}
\renewcommand{\iff}{\quad \text{iff} \quad}

\usepackage{amssymb}
\usepackage{pgffor}

\newcommand{\id}{\textrm{id}}
\newcommand{\range}{\textrm{range}}
\newcommand{\const}[2]{\textrm{const}_{#1}^{#2}}
\newcommand{\symdiff}{\mathop{\triangle}}
\newcommand{\pow}{\mathcal{P}}
\newcommand{\onto}{\twoheadrightarrow}
\newcommand{\into}{\hookrightarrow}

% E.g. `\set{1, 2, 3, 4}` prints the numbers 1, 2, 3 and 4 as a comma
% separated list which is enclosed within curly braces.
\def\set#1{\ensuremath{
  \{\foreach[count=\i]\x in {#1}{
    \ifnum\i>1,\,\fi
    \x
  }\}
}}

\useackage{amssymb}

\newcommand{\Nat}{\mathbb{N}}
\newcommand{\Prime}{\mathbb{P}}
\renewcommand{\succ}{\textrm{succ}}
\newcommand{\pred}{\textrm{pred}}
\newcommand{\add}{\textrm{add}}
\newcommand{\mul}{\textrm{mul}}
\renewcommand{\exp}{\textrm{exp}}
\newcommand{\fac}{\textrm{fac}}
\renewcommand{\div}{\mathrel{\textrm{div}}}
\renewcommand{\mod}{\mathrel{\textrm{mod}}}

\begin{document}
  \begin{imports}
    \begin{forthel}
      %[prove off][check off]
      [readtex \path{libraries/source/foundations/10_sets.ftl.tex}]
      %[prove on][check on]
    \end{forthel}
  \end{imports}


  \section{The Language of Peano Arithmetic}

  \begin{forthel}
    \begin{signature}\printlabel{ARITHMETIC_01_3074681254969344}
      A natural number is an object.
    \end{signature}
  \end{forthel}

  \begin{forthel}
    \begin{definition}\printlabel{ARITHMETIC_01_7367148418629632}
      $\Nat$ is the class of natural numbers.
    \end{definition}
  \end{forthel}

  \begin{forthel}
    \begin{signature}\printlabel{ARITHMETIC_01_7633304715001856}
      $0$ is a natural number.
    \end{signature}

    Let zero stand for $0$.
    Let $n$ is nonzero stand for $n \neq 0$.
  \end{forthel}

  \begin{forthel}
    \begin{signature}\printlabel{ARITHMETIC_01_1567933815848960}
      Let $n$ be a natural number.
      $\succ(n)$ is a natural number.
    \end{signature}

    Let the direct successor of $n$ stand for $\succ(n)$.
  \end{forthel}


  \section{The Peano Axioms}

  \begin{forthel}
    \begin{axiom}\printlabel{ARITHMETIC_01_3604163883696128}
      Let $n, m$ be natural numbers.
      If $\succ(n) = \succ(m)$ then $n = m$.
    \end{axiom}
  \end{forthel}

  \begin{forthel}
    \begin{axiom}\printlabel{ARITHMETIC_01_4454289938317312}
      There exists no natural number $n$ such that $\succ(n) = 0$.
    \end{axiom}
  \end{forthel}

  \begin{forthel}
    \begin{axiom}\printlabel{ARITHMETIC_01_4764664342773760}
      Let $\Phi$ be a class.
      Assume $0 \in \Phi$ and for all natural numbers $n$ if $n \in \Phi$ then
      $\succ(n) \in \Phi$.
      Then $\Phi$ contains every natural number.
    \end{axiom}
  \end{forthel}


  \section{Immediate Consequences}

  \begin{forthel}
    \begin{proposition}\printlabel{ARITHMETIC_01_4966080109871104}
      Let $n$ be a natural number.
      Then $n = 0$ or $n = \succ(m)$ for some natural number $m$.
    \end{proposition}
    \begin{proof}
      Define $\Phi = \{ n' \in \Nat \mid n' = 0$ or $n' = \succ(m')$ for some
      natural number $m' \}$.
      $0 \in \Phi$ and for all $n' \in  \Phi$ we have $\succ(n') \in \Phi$.
      Hence every natural number is contained in $\Phi$.
      Thus $n = 0$ or $n = \succ(m)$ for some natural number $m$.
    \end{proof}
  \end{forthel}

  \begin{forthel}
    \begin{proposition}\printlabel{ARITHMETIC_01_5996049267163136}
      Let $n$ be a natural number.
      Then $n \neq \succ(n)$.
    \end{proposition}
    \begin{proof}
      Define $\Phi = \{ n' \in \Nat \mid n' \neq \succ(n') \}$.

      (1) $0$ belongs to $\Phi$.

      (2) For all $n' \in \Phi$ we have $\succ(n') \in \Phi$. \\
      Proof.
        Let $n' \in \Phi$.
        Then $n' \neq \succ(n')$.
        If $\succ(n') = \succ(\succ(n'))$ then $n' = \succ(n')$.
        Thus it is wrong that $\succ(n') = \succ(\succ(n'))$.
        Hence $\succ(n') \in \Phi$.
      Qed.

      Therefore every natural number is an element of $\Phi$.
      Consequently $n \neq \succ(n)$.
    \end{proof}
  \end{forthel}

  \begin{forthel}
    \begin{proposition}\printlabel{ARITHMETIC_01_6115694068367360}
      $\Nat$ is a set.
    \end{proposition}
    \begin{proof}
      Define $f(n) = \succ(n)$ for $n \in \Nat$.
      Then $f$ is a map from $\Nat$ to $\Nat$.
      Hence we can take a subset $X$ of $\Nat$ that is inductive regarding
      $0$ and $f$.
      Then $0 \in X$ and for all $n \in X$ we have $\succ(n) \in X$.
      Hence $X$ contains every natural number.
      Thus we have $\Nat \subseteq X$ and $X \subseteq \Nat$.
      Therefore $\Nat = X$.
      Consequently $\Nat$ is a set.
    \end{proof}
  \end{forthel}


  \section{Additional Constants}

  \begin{forthel}
    \begin{definition}\printlabel{ARITHMETIC_01_7540560137027584}
      $1 = \succ(0)$.
    \end{definition}

    Let one stand for $1$.
  \end{forthel}

  \begin{forthel}
    \begin{definition}\printlabel{ARITHMETIC_01_4584236572999680}
      $2 = \succ(1)$.
    \end{definition}

    Let two stand for $2$.
  \end{forthel}

  \begin{forthel}
    \begin{definition}\printlabel{ARITHMETIC_01_3836725109456896}
      $3 = \succ(2)$.
    \end{definition}

    Let three stand for $3$.
  \end{forthel}

  \begin{forthel}
    \begin{definition}\printlabel{ARITHMETIC_01_1709884968009728}
      $4 = \succ(3)$.
    \end{definition}

    Let four stand for $4$.
  \end{forthel}

  \begin{forthel}
    \begin{definition}\printlabel{ARITHMETIC_01_6734726333202432}
      $5 = \succ(4)$.
    \end{definition}

    Let five stand for $5$.
  \end{forthel}

  \begin{forthel}
    \begin{definition}\printlabel{ARITHMETIC_01_949139189792768}
      $6 = \succ(5)$.
    \end{definition}

    Let six stand for $6$.
  \end{forthel}

  \begin{forthel}
    \begin{definition}\printlabel{ARITHMETIC_01_7245471749767168}
      $7 = \succ(6)$.
    \end{definition}

    Let seven stand for $7$.
  \end{forthel}

  \begin{forthel}
    \begin{definition}\printlabel{ARITHMETIC_01_5658172888973312}
      $8 = \succ(7)$.
    \end{definition}

    Let eight stand for $8$.
  \end{forthel}

  \begin{forthel}
    \begin{definition}\printlabel{ARITHMETIC_01_7371844250238976}
      $9 = \succ(8)$.
    \end{definition}

    Let nine stand for $9$.
  \end{forthel}
\end{document}
