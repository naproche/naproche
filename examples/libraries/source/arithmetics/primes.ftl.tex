\documentclass[10pt]{article}
\usepackage[english]{babel}
\usepackage{../../../meta-inf/lib/naproche}
% Top-level sections and proofs

\newtheoremstyle{foundations}% name of the style to be used
  {\dimexpr\ftlparskip+4pt\relax}% measure of space to leave above the theorem.
  % (We must add \ftlparskip since otherwise the name of the label would not fit
  % into the border drawn around the environment.)
  {4pt}% measure of space to leave below the theorem.
  {\normalfont}% name of font to use in the body of the theorem
  {0pt}% measure of space to indent
  {\bfseries}% name of head font
  {}% punctuation between head and body
  { }% space after theorem head; " " = normal inter-word space
  {\thmname{#1}\thmnumber{ #2}\thmnote{ (#3)}.}

\let\definition\relax
\let\enddefinition\relax
\let\signature\relax
\let\endsignature\relax
\let\axiom\relax
\let\endaxiom\relax
\let\theorem\relax
\let\endtheorem\relax
\let\proposition\relax
\let\endproposition\relax
\let\lemma\relax
\let\endlemma\relax
\let\corollary\relax
\let\endcorollary\relax

\theoremstyle{foundations}

\newmdtheoremenv[
  backgroundcolor=lightgray!10,
  linecolor=black,
  innertopmargin=1em
]{definition}{Definition}[section]
\newmdtheoremenv[
  backgroundcolor=lightgray!10,
  linecolor=black,
  innertopmargin=1em
]{signature}[definition]{Signature}
\newmdtheoremenv[
  backgroundcolor=lightgray!10,
  linecolor=black,
  innertopmargin=1em
]{axiom}[definition]{Axiom}
\newmdtheoremenv[
  backgroundcolor=lightgray!10,
  linecolor=black,
  innertopmargin=1em
]{theorem}[definition]{Theorem}
\newmdtheoremenv[
  backgroundcolor=lightgray!10,
  linecolor=black,
  innertopmargin=1em
]{proposition}[definition]{Proposition}
\newmdtheoremenv[
  backgroundcolor=lightgray!10,
  linecolor=black,
  innertopmargin=1em
]{lemma}[definition]{Lemma}
\newmdtheoremenv[
  backgroundcolor=lightgray!10,
  linecolor=black,
  innertopmargin=1em
]{corollary}[definition]{Corollary}


% Hide imports

\usepackage{scontents}
\newenvsc{imports}[store-env=imports,print-env=false]


% Labels of top-level sections

\renewcommand{\showlabelfont}{\scriptsize\ttfamily\color{gray}}


% Non-symbolic "\implies" and "\iff"

\renewcommand{\implies}{\quad \text{implies} \quad}
\renewcommand{\iff}{\quad \text{iff} \quad}

\usepackage{amssymb}
\usepackage{pgffor}

\newcommand{\id}{\textrm{id}}
\newcommand{\range}{\textrm{range}}
\newcommand{\const}[2]{\textrm{const}_{#1}^{#2}}
\newcommand{\symdiff}{\mathop{\triangle}}
\newcommand{\pow}{\mathcal{P}}
\newcommand{\onto}{\twoheadrightarrow}
\newcommand{\into}{\hookrightarrow}

% E.g. `\set{1, 2, 3, 4}` prints the numbers 1, 2, 3 and 4 as a comma
% separated list which is enclosed within curly braces.
\def\set#1{\ensuremath{
  \{\foreach[count=\i]\x in {#1}{
    \ifnum\i>1,\,\fi
    \x
  }\}
}}

\useackage{amssymb}

\newcommand{\Nat}{\mathbb{N}}
\newcommand{\Prime}{\mathbb{P}}
\renewcommand{\succ}{\textrm{succ}}
\newcommand{\pred}{\textrm{pred}}
\newcommand{\add}{\textrm{add}}
\newcommand{\mul}{\textrm{mul}}
\renewcommand{\exp}{\textrm{exp}}
\newcommand{\fac}{\textrm{fac}}
\renewcommand{\div}{\mathrel{\textrm{div}}}
\renewcommand{\mod}{\mathrel{\textrm{mod}}}

\begin{document}
  \begin{imports}
    \begin{forthel}
      %[prove off][check off]
      [readtex \path{libraries/source/arithmetics/divisibility.ftl.tex}]
      [readtex \path{libraries/source/arithmetics/modular-arithmetics.ftl.tex}]
      %[prove on][check on]
    \end{forthel}
  \end{imports}


  \section*{Prime Numbers}

  \begin{forthel}
    \begin{definition}\printlabel{ARITHMETIC_10_5450464558579712}
      Let $n$ be a natural number.
      $n$ is prime iff $n > 1$ and $n$ has no nontrivial divisors.
    \end{definition}

    Let $n$ is compound stand for $n$ is not prime.
    Let a prime number stand for a natural number that is prime.
  \end{forthel}

  \begin{forthel}
    \begin{definition}\printlabel{ARITHMETIC_10_3834705971511296}
      $\Prime$ is the class of all prime numbers.
    \end{definition}
  \end{forthel}

  \begin{forthel}
    \begin{proposition}\printlabel{ARITHMETIC_10_7801379464675328}
      Let $n$ be a natural number such that $n > 1$.
      Then $n$ is prime iff every divisor of $n$ is a trivial divisor of $n$.
    \end{proposition}
  \end{forthel}

  \begin{forthel}
    \begin{proposition}\printlabel{ARITHMETIC_10_3606185106210816}
      Let $n$ be a natural number such that $n > 1$.
      Then $n$ has a prime divisor.
    \end{proposition}
    \begin{proof}
      Define $\Phi = \{ n' \in \Nat \mid$ if $n' > 1$ then $n'$ has a prime divisor $\}$.

      Let us show that for every $n' \in \Nat$ if $\Phi$ contains all
      predecessors of $n'$ then $\Phi$ contains $n'$.
        Let $n' \in \Nat$.
        Assume that $\Phi$ contains all predecessors of $n'$.
        We have $n' = 0$ or $n' = 1$ or $n'$ is prime or $n'$ is composite.

        Case $n' = 0$ or $n' = 1$. Trivial.

        Case $n'$ is prime. Obvious.

        Case $n'$ is composite.
          Take a nontrivial divisor $m$ of $n'$.
          Then $1 < m < n'$.
          $m$ is contained in $\Phi$.
          Hence we can take a prime divisor $p$ of $m$.
          Then we have $p \mid m \mid n'$.
          Thus $p \mid n'$.
          Therefore $p$ is a prime divisor of $n'$.
        End.
      End.

      Thus every natural number belongs to $\Phi$ (by \printref{ARITHMETIC_04_3609801697263616}).
    \end{proof}
  \end{forthel}

  \begin{forthel}
    \begin{definition}\printlabel{ARITHMETIC_10_463197419077632}
      Let $n, m$ be natural numbers.
      $n$ and $m$ are coprime iff for all nonzero natural numbers $k$ such that $k \mid n$ and $k \mid m$ we have $k = 1$.
    \end{definition}

    Let $n$ and $m$ are relatively prime stand for $n$ and $m$ are coprime.
    Let $n$ and $m$ are mutually prime stand for $n$ and $m$ are coprime.
    Let $n$ is prime to $m$ stand for $n$ and $m$ are coprime.
  \end{forthel}

  \begin{forthel}
    \begin{proposition}\printlabel{ARITHMETIC_10_5776394594287616}
      Let $n, m$ be natural numbers.
      $n$ and $m$ are coprime iff $n$ and $m$ have no common prime divisor.
    \end{proposition}
    \begin{proof}
      Case $n$ and $m$ are coprime.
        Let $p$ be a prime number such that $p \mid n$ and $p \mid m$.
        Then $p$ is nonzero and $p \neq 1$.
        Contradiction.
      End.

      Case $n$ and $m$ have no common prime divisor.
        Assume that $n$ and $m$ are not coprime.
        Let $k$ be a nonzero natural number such that $k \mid n$ and $k \mid m$.
        Assume that $k \neq 1$.
        Consider a prime divisor $p$ of $k$.
        Then $p \mid k \mid n,m$.
        Hence $p \mid n$ and $p \mid m$.
        Contradiction.
      End.
    \end{proof}
  \end{forthel}

  \begin{forthel}
    \begin{proposition}\printlabel{ARITHMETIC_10_7212152851005440}
      Let $n, m$ be natural numbers and $p$ be a prime number.
      If $p$ does not divide $n$ then $p$ and $n$ are coprime.
    \end{proposition}
    \begin{proof}
      Assume $p \nmid n$.
      Suppose that $p$ and $n$ are not coprime.
      Take a nonzero natural number $k$ such that $k \mid p$ and $k \mid n$.
      Then $k = p$.
      Hence $p \mid n$.
      Contradiction.
    \end{proof}
  \end{forthel}

  \begin{forthel}
    \begin{proposition}\printlabel{ARITHMETIC_10_8313676557713408}
      Let $n, m$ be natural numbers and $p$ be a prime number.
      If $p \mid n \cdot m$ then $p \mid n$ or $p \mid m$.
    \end{proposition}
    \begin{proof}
      Assume $p \mid n \cdot m$.

      Case $p \mid n$. Trivial.

      Case $p \nmid n$.
        Define $\Phi = \{ k \in \Nat \mid k \neq 0$ and $p \mid k \cdot m \}$.
        Then $p \in \Phi$ and $n \in \Phi$.
        Hence $\Phi$ contains some natural number.
        Thus we can take $a \in \Phi$ such that $a \leq k$ for all $k \in \Phi$.

        Let us show that $a$ divides all elements of $\Phi$.
          Let $k \in \Phi$.
          Take natural numbers $q, r$ such that $k = (a \cdot q) + r$ and $r < a$ (by \printref{ARITHMETIC_08_7743986617810944}).
          Indeed $a$ is nonzero.
          Then $k \cdot m
            = ((q \cdot a) + r) \cdot m
            = ((q \cdot a) \cdot m) + (r \cdot m)$.
          We have $p \mid k \cdot m$.
          Hence $p \mid ((q \cdot a) \cdot m) + (r \cdot m)$.

          We can show that $p \mid r \cdot m$.
            We have $p \mid a \cdot m$.
            Hence $p \mid (q \cdot a) \cdot m$.
            Indeed $((q \cdot a) \cdot m) = q \cdot (a \cdot m)$. %!
            Take $A = (q \cdot a) \cdot m$ and $B = r \cdot m$. %!
            Then $p \mid A + B$ and $p \mid A$.
            Thus $p \mid B$ (by \printref{ARITHMETIC_07_1076947887063040}).
            Indeed $p, A$ and $B$ are natural numbers.
            Consequently $p \mid r \cdot m$.
          End.

          Therefore $r = 0$.
          Indeed if $r \neq 0$ then $r$ is an element of $\Phi$ that is less than $a$.
          Hence $k = q \cdot a$.
          Thus $a$ divides $k$.
        End.

        Then we have $a \mid p$ and $a \mid n$.
        Hence $a = p$ or $a = 1$.
        Thus $a = 1$.
        Indeed if $a = p$ then $p \mid n$.
        Then $1 \in \Phi$.
        Therefore $p \mid 1 \cdot m = m$.
      End.
    \end{proof}
  \end{forthel}
\end{document}
