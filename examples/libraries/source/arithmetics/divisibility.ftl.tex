\documentclass{naproche-library}
\libinput{arithmetics-preamble}
\begin{document}
\begin{smodule}{divisibility.ftl}
  \importmodule[libraries]{arithmetics?multiplication-and-ordering.ftl}

  \section*{Divisibility}

  \begin{definition}[forthel,id=ARITHMETIC_07_4239998993825792,printid]
    Let $n, m$ be natural numbers.
    $n$ divides $m$ iff there exists a natural number $k$ such that $n \cdot k = m$.

    Let $m$ is divisible by $n$ stand for $n$ divides $m$.
    Let $n \mid m$ stand for $n$ divides $m$.
    Let $n \nmid m$ stand for $n$ does not divide $m$.
  \end{definition}

  \begin{lemma}[forthel,id=ARITHMETIC_07_1478855118290944,printid]
    Let $n, m$ be natural numbers.
    $n$ divides $m$ iff there exists a natural number $k$ such that $k \cdot n = m$.
  \end{lemma}

  \begin{definition}[forthel,id=ARITHMETIC_07_1311437490225152,printid]
    Let $n$ be a natural number.
    A factor of $n$ is a natural number that divides $n$.

    Let a divisor of $n$ stand for a factor of $n$.
  \end{definition}

  \begin{definition}[forthel,id=ARITHMETIC_10_5438991513944064,printid]
    Let $n$ be a natural number.
    A trivial divisor of $n$ is a divisor $m$ of $n$ such that $m = 1$ or $m = n$.
  \end{definition}

  \begin{definition}[forthel,id=ARITHMETIC_10_8768240253665280,printid]
    Let $n$ be a natural number.
    A nontrivial divisor of $n$ is a divisor $m$ of $n$ such that $m \neq 1$ and $m \neq n$.
  \end{definition}

  \begin{definition}[forthel,id=ARITHMETIC_10_8020087063707648,printid]
    Let $n$ be a natural number.
    $n$ is composite iff $n > 1$ and $n$ has a nontrivial divisor.
  \end{definition}

  \begin{proposition}[forthel,id=ARITHMETIC_07_2242720387039232,printid]
    Let $n$ be a natural number.
    Then $n \mid 0$.
  \end{proposition}
  \begin{proof}[forthel]
    We have $n \cdot 0 = 0$.
    Hence $n \mid 0$.
  \end{proof}

  \begin{proposition}[forthel,id=ARITHMETIC_07_8611150130315264,printid]
    Let $n$ be a natural number.
    If $0 \mid n$ then $n = 0$.
  \end{proposition}
  \begin{proof}[forthel]
    Assume $0 \mid n$.
    Consider a natural number $k$ such that $0 \cdot k = n$.
    Then $n = 0$.
  \end{proof}

  \begin{proposition}[forthel,id=ARITHMETIC_07_1259086070939648,printid]
    Let $n$ be a natural number.
    Then $1 \mid n$.
  \end{proposition}
  \begin{proof}[forthel]
    We have $1 \cdot n = n$.
    Hence $1 \mid n$.
  \end{proof}

  \begin{proposition}[forthel,id=ARITHMETIC_07_3944887330275328,printid]
    Let $n$ be a natural number.
    Then $n \mid n$.
  \end{proposition}
  \begin{proof}[forthel]
    We have $n \cdot 1 = n$.
    Hence $n \mid n$.
  \end{proof}

  \begin{proposition}[forthel,id=ARITHMETIC_07_6917446193643520,printid]
    Let $n$ be a natural number.
    If $n \mid 1$ then $n = 1$.
  \end{proposition}
  \begin{proof}[forthel]
    Assume $n \mid 1$.
    Take a natural number $k$ such that $n \cdot k = 1$.
    Suppose $n \neq 1$.
    Then $n < 1$ or $n > 1$.

    Case $n < 1$.
      Then $n = 0$.
      Hence $0
        = 0 \cdot k
        = n \cdot k
        = 1$.
      Contradiction.
    End.

    Case $n > 1$.
      We have $k \neq 0$.
      Indeed if $k = 0$ then
      $1
        = n \cdot k
        = n \cdot 0
        = 0$.
      Hence $k \geq 1$.
      Take a positive natural number $l$ such that $n = 1 + l$.
      Then $1
        < 1 + l
        = n
        = n \cdot 1
        \leq n \cdot k$.
      Hence $1 < n$.
      Contradiction.
    End.
  \end{proof}

  \begin{proposition}[forthel,id=ARITHMETIC_07_7463519983239168,printid]
    Let $n, m, k$ be natural numbers.
    If $n \mid m$ then $n \mid m \cdot k$.
  \end{proposition}
  \begin{proof}[forthel]
    Assume $n \mid m$.
    Take $l \in \Nat$ such that $n \cdot l = m$.
    Then $n \cdot (l \cdot k)
      = (n \cdot l) \cdot k
      = m \cdot k$.
    Hence $n \mid m \cdot k$.
  \end{proof}

  \begin{corollary}[forthel,id=ARITHMETIC_07_1588185794609152,printid]
    Let $n, m, k$ be natural numbers.
    If $n \mid m$ then $n \mid k \cdot m$.
  \end{corollary}

  \begin{proposition}[forthel,id=ARITHMETIC_07_7863858316181504,printid]
    Let $n, m, k$ be natural numbers.
    If $n \mid m \mid k$ then $n \mid k$.
  \end{proposition}
  \begin{proof}[forthel]
    Assume $n \mid m$ and $m \mid k$.
    Take natural numbers $l,l'$ such that $n \cdot l = m$ and $m \cdot l' = k$.
    Then $n \cdot (l \cdot l')
      = (n \cdot l) \cdot l'
      = m \cdot l'
      = k$.
    Hence $n \mid k$.
  \end{proof}

  \begin{proposition}[forthel,id=ARITHMETIC_07_4933275640397824,printid]
    Let $n, m$ be natural numbers such that $n \neq 0$.
    If $n \mid m$ and $m \mid n$ then $n = m$.
  \end{proposition}
  \begin{proof}[forthel]
    Assume $n \mid m$ and $m \mid n$.
    Take natural numbers $k,k'$ such that $n \cdot k = m$ and $m \cdot k' = n$.
    Then $n
      = m \cdot k'
      = (n \cdot k) \cdot k'
      = n \cdot (k \cdot k')$.
    Hence $k \cdot k' = 1$.
    Thus $k = 1 = k'$.
    Therefore $n = m$.
  \end{proof}

  \begin{proposition}[forthel,id=ARITHMETIC_07_1283495225720832,printid]
    Let $n, m, k$ be natural numbers.
    If $n \mid m$ then $k \cdot n \mid k \cdot m$.
  \end{proposition}
  \begin{proof}[forthel]
    Assume $n \mid m$.
    Take a natural number $l$ such that $n \cdot l = m$.
    Then $(k \cdot n) \cdot l
      = k \cdot (n \cdot l)
      = k \cdot m$.
    Hence $k \cdot n \mid k \cdot m$.
  \end{proof}

  \begin{proposition}[forthel,id=ARITHMETIC_07_6469492028735488,printid]
    Let $n, m, k$ be natural numbers.
    Assume $k \neq 0$.
    If $k \cdot n \mid k \cdot m$ then $n \mid m$.
  \end{proposition}
  \begin{proof}[forthel]
    Assume $k \cdot n \mid k \cdot m$.
    Take a natural number $l$ such that $(k \cdot n) \cdot l = k \cdot m$.
    Then $k \cdot (n \cdot l) = k \cdot m$.
    Hence $n \cdot l = m$ (by \printref{ARITHMETIC_06_8575191374364672}).
    Thus $n \mid m$.
  \end{proof}

  \begin{proposition}[forthel,id=ARITHMETIC_07_4700711333920768,printid]
    Let $n, m, k$ be natural numbers.
    If $k \mid n$ and $k \mid m$ then $k \mid (n' \cdot n) + (m' \cdot m)$
    for all natural numbers $n', m'$.
  \end{proposition}
  \begin{proof}[forthel]
    Assume $k \mid n$ and $k \mid m$.
    Let $n', m'$ be natural numbers.
    Take natural numbers $l,l'$ such that $k \cdot l = n$ and $k \cdot l' = m$.
    Then
    \[  k \cdot ((n' \cdot l) + (m' \cdot l'))                \]
    \[    = (k \cdot (n' \cdot l)) + (k \cdot (m' \cdot l'))  \]
    \[    = ((k \cdot n') \cdot l) + ((k \cdot m') \cdot l')  \]
    \[    = (n' \cdot (k \cdot l)) + (m' \cdot (k \cdot l'))  \]
    \[    = (n' \cdot n) + (m' \cdot m).                      \]
  \end{proof}

  \begin{corollary}[forthel,id=ARITHMETIC_07_1556786209357824,printid]
    Let $n, m, k$ be natural numbers.
    If $k \mid n$ and $k \mid m$ then $k \mid n + m$.
  \end{corollary}
  \begin{proof}[forthel]
    Assume $k \mid n$ and $k \mid m$.
    Take $n' = 1$ and $m' = 1$.
    Then $k \mid (n' \cdot n) + (m' \cdot m)$.
    $(n' \cdot n) + (m' \cdot m) = n + m$.
    Hence $k \mid n + m$.
  \end{proof}

  \begin{proposition}[forthel,id=ARITHMETIC_07_1076947887063040,printid]
    Let $n, m, k$ be natural numbers.
    If $k \mid n$ and $k \mid n + m$ then $k \mid m$.
  \end{proposition}
  \begin{proof}[forthel]
    Assume $k \mid n$ and $k \mid n + m$.

    Case $k = 0$. Obvious.

    Case $k \neq 0$.
      Take a natural number $l$ such that $n = k \cdot l$.
      Take a natural number $l'$ such that $n + m = k \cdot l'$.
      Then $(k \cdot l) + m = k \cdot l'$.
      We have $l' \geq l$.
      Indeed if $l' < l$ then
      $n + m
        = k \cdot l'
        < k \cdot l
        = n$ (by \printref{ARITHMETIC_06_5048640368279552}).
      Hence we can take a natural number $l''$ such that $l' = l + l''$.
      Then $(k \cdot l) + m
        = k \cdot l'
        = k \cdot (l + l'')
        = (k \cdot l) + (k \cdot l'')$.
      Indeed $k \cdot (l + l'') = (k \cdot l) + (k \cdot l'')$ (by \printref{ARITHMETIC_06_9001524774567936}).
      Thus $m = (k \cdot l'')$ (by \printref{ARITHMETIC_03_8445946379632640}).
      Indeed $k \cdot l$ and $k \cdot l''$ are natural numbers.
      Therefore $k \mid m$.
    End.
  \end{proof}

  \begin{proposition}[forthel,id=ARITHMETIC_07_2187144577679360,printid]
    Let $n, m$ be natural numbers such that $n, m \neq 0$.
    If $m \mid n$ then $m \leq n$.
  \end{proposition}
  \begin{proof}[forthel]
    Assume $m \mid n$.
    Take a natural number $k$ such that $m \cdot k = n$.
    If $k = 0$ then
    $n
      = m \cdot k
      = m \cdot 0
      = 0$.
    Thus $k \geq 1$.
    Assume $m > n$.
    Then $n
      = m \cdot k
      \geq m \cdot 1
      = m
      > n$.
    Hence $n > n$.
    Contradiction.
  \end{proof}
\end{smodule}
\end{document}
