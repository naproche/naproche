\documentclass[10pt]{article}
\usepackage[english]{babel}
\usepackage{../../../meta-inf/lib/naproche}
% Top-level sections and proofs

\newtheoremstyle{foundations}% name of the style to be used
  {\dimexpr\ftlparskip+4pt\relax}% measure of space to leave above the theorem.
  % (We must add \ftlparskip since otherwise the name of the label would not fit
  % into the border drawn around the environment.)
  {4pt}% measure of space to leave below the theorem.
  {\normalfont}% name of font to use in the body of the theorem
  {0pt}% measure of space to indent
  {\bfseries}% name of head font
  {}% punctuation between head and body
  { }% space after theorem head; " " = normal inter-word space
  {\thmname{#1}\thmnumber{ #2}\thmnote{ (#3)}.}

\let\definition\relax
\let\enddefinition\relax
\let\signature\relax
\let\endsignature\relax
\let\axiom\relax
\let\endaxiom\relax
\let\theorem\relax
\let\endtheorem\relax
\let\proposition\relax
\let\endproposition\relax
\let\lemma\relax
\let\endlemma\relax
\let\corollary\relax
\let\endcorollary\relax

\theoremstyle{foundations}

\newmdtheoremenv[
  backgroundcolor=lightgray!10,
  linecolor=black,
  innertopmargin=1em
]{definition}{Definition}[section]
\newmdtheoremenv[
  backgroundcolor=lightgray!10,
  linecolor=black,
  innertopmargin=1em
]{signature}[definition]{Signature}
\newmdtheoremenv[
  backgroundcolor=lightgray!10,
  linecolor=black,
  innertopmargin=1em
]{axiom}[definition]{Axiom}
\newmdtheoremenv[
  backgroundcolor=lightgray!10,
  linecolor=black,
  innertopmargin=1em
]{theorem}[definition]{Theorem}
\newmdtheoremenv[
  backgroundcolor=lightgray!10,
  linecolor=black,
  innertopmargin=1em
]{proposition}[definition]{Proposition}
\newmdtheoremenv[
  backgroundcolor=lightgray!10,
  linecolor=black,
  innertopmargin=1em
]{lemma}[definition]{Lemma}
\newmdtheoremenv[
  backgroundcolor=lightgray!10,
  linecolor=black,
  innertopmargin=1em
]{corollary}[definition]{Corollary}


% Hide imports

\usepackage{scontents}
\newenvsc{imports}[store-env=imports,print-env=false]


% Labels of top-level sections

\renewcommand{\showlabelfont}{\scriptsize\ttfamily\color{gray}}


% Non-symbolic "\implies" and "\iff"

\renewcommand{\implies}{\quad \text{implies} \quad}
\renewcommand{\iff}{\quad \text{iff} \quad}

\usepackage{amssymb}
\usepackage{pgffor}

\newcommand{\id}{\textrm{id}}
\newcommand{\range}{\textrm{range}}
\newcommand{\const}[2]{\textrm{const}_{#1}^{#2}}
\newcommand{\symdiff}{\mathop{\triangle}}
\newcommand{\pow}{\mathcal{P}}
\newcommand{\onto}{\twoheadrightarrow}
\newcommand{\into}{\hookrightarrow}

% E.g. `\set{1, 2, 3, 4}` prints the numbers 1, 2, 3 and 4 as a comma
% separated list which is enclosed within curly braces.
\def\set#1{\ensuremath{
  \{\foreach[count=\i]\x in {#1}{
    \ifnum\i>1,\,\fi
    \x
  }\}
}}

\useackage{amssymb}

\newcommand{\Nat}{\mathbb{N}}
\newcommand{\Prime}{\mathbb{P}}
\renewcommand{\succ}{\textrm{succ}}
\newcommand{\pred}{\textrm{pred}}
\newcommand{\add}{\textrm{add}}
\newcommand{\mul}{\textrm{mul}}
\renewcommand{\exp}{\textrm{exp}}
\newcommand{\fac}{\textrm{fac}}
\renewcommand{\div}{\mathrel{\textrm{div}}}
\renewcommand{\mod}{\mathrel{\textrm{mod}}}

\begin{document}
  \begin{imports}
    \begin{forthel}
      %[prove off][check off]
      [read \path{libraries/source/arithmetics/multiplication-and-ordering.ftl.tex}]
      %[prove on][check on]
    \end{forthel}
  \end{imports}


  \section*{Divisibility}

  \begin{forthel}
    \begin{definition}[id=ARITHMETIC_07_4239998993825792,printid]
      Let $n, m$ be natural numbers.
      $n$ divides $m$ iff there exists a natural number $k$ such that $n \cdot k = m$.
    \end{definition}

    Let $m$ is divisible by $n$ stand for $n$ divides $m$.
    Let $n \mid m$ stand for $n$ divides $m$.
    Let $n \nmid m$ stand for $n$ does not divide $m$.
  \end{forthel}

  \begin{forthel}
    \begin{lemma}[id=ARITHMETIC_07_1478855118290944,printid]
      Let $n, m$ be natural numbers.
      $n$ divides $m$ iff there exists a natural number $k$ such that $k \cdot n = m$.
    \end{lemma}
  \end{forthel}

  \begin{forthel}
    \begin{definition}[id=ARITHMETIC_07_1311437490225152,printid]
      Let $n$ be a natural number.
      A factor of $n$ is a natural number that divides $n$.
    \end{definition}

    Let a divisor of $n$ stand for a factor of $n$.
  \end{forthel}

  \begin{forthel}
    \begin{definition}[id=ARITHMETIC_10_5438991513944064,printid]
      Let $n$ be a natural number.
      A trivial divisor of $n$ is a divisor $m$ of $n$ such that $m = 1$ or $m = n$.
    \end{definition}
  \end{forthel}

  \begin{forthel}
    \begin{definition}[id=ARITHMETIC_10_8768240253665280,printid]
      Let $n$ be a natural number.
      A nontrivial divisor of $n$ is a divisor $m$ of $n$ such that $m \neq 1$ and $m \neq n$.
    \end{definition}
  \end{forthel}

  \begin{forthel}
    \begin{definition}[id=ARITHMETIC_10_8020087063707648,printid]
      Let $n$ be a natural number.
      $n$ is composite iff $n > 1$ and $n$ has a nontrivial divisor.
    \end{definition}
  \end{forthel}

  \begin{forthel}
    \begin{proposition}[id=ARITHMETIC_07_2242720387039232,printid]
      Let $n$ be a natural number.
      Then $n \mid 0$.
    \end{proposition}
    \begin{proof}
      We have $n \cdot 0 = 0$.
      Hence $n \mid 0$.
    \end{proof}
  \end{forthel}

  \begin{forthel}
    \begin{proposition}[id=ARITHMETIC_07_8611150130315264,printid]
      Let $n$ be a natural number.
      If $0 \mid n$ then $n = 0$.
    \end{proposition}
    \begin{proof}
      Assume $0 \mid n$.
      Consider a natural number $k$ such that $0 \cdot k = n$.
      Then $n = 0$.
    \end{proof}
  \end{forthel}

  \begin{forthel}
    \begin{proposition}[id=ARITHMETIC_07_1259086070939648,printid]
      Let $n$ be a natural number.
      Then $1 \mid n$.
    \end{proposition}
    \begin{proof}
      We have $1 \cdot n = n$.
      Hence $1 \mid n$.
    \end{proof}
  \end{forthel}

  \begin{forthel}
    \begin{proposition}[id=ARITHMETIC_07_3944887330275328,printid]
      Let $n$ be a natural number.
      Then $n \mid n$.
    \end{proposition}
    \begin{proof}
      We have $n \cdot 1 = n$.
      Hence $n \mid n$.
    \end{proof}
  \end{forthel}

  \begin{forthel}
    \begin{proposition}[id=ARITHMETIC_07_6917446193643520,printid]
      Let $n$ be a natural number.
      If $n \mid 1$ then $n = 1$.
    \end{proposition}
    \begin{proof}
      Assume $n \mid 1$.
      Take a natural number $k$ such that $n \cdot k = 1$.
      Suppose $n \neq 1$.
      Then $n < 1$ or $n > 1$.

      Case $n < 1$.
        Then $n = 0$.
        Hence $0
          = 0 \cdot k
          = n \cdot k
          = 1$.
        Contradiction.
      End.

      Case $n > 1$.
        We have $k \neq 0$.
        Indeed if $k = 0$ then
        $1
          = n \cdot k
          = n \cdot 0
          = 0$.
        Hence $k \geq 1$.
        Take a positive natural number $l$ such that $n = 1 + l$.
        Then $1
          < 1 + l
          = n
          = n \cdot 1
          \leq n \cdot k$.
        Hence $1 < n$.
        Contradiction.
      End.
    \end{proof}
  \end{forthel}

  \begin{forthel}
    \begin{proposition}[id=ARITHMETIC_07_7463519983239168,printid]
      Let $n, m, k$ be natural numbers.
      If $n \mid m$ then $n \mid m \cdot k$.
    \end{proposition}
    \begin{proof}
      Assume $n \mid m$.
      Take $l \in \Nat$ such that $n \cdot l = m$.
      Then $n \cdot (l \cdot k)
        = (n \cdot l) \cdot k
        = m \cdot k$.
      Hence $n \mid m \cdot k$.
    \end{proof}
  \end{forthel}

  \begin{forthel}
    \begin{corollary}[id=ARITHMETIC_07_1588185794609152,printid]
      Let $n, m, k$ be natural numbers.
      If $n \mid m$ then $n \mid k \cdot m$.
    \end{corollary}
  \end{forthel}

  \begin{forthel}
    \begin{proposition}[id=ARITHMETIC_07_7863858316181504,printid]
      Let $n, m, k$ be natural numbers.
      If $n \mid m \mid k$ then $n \mid k$.
    \end{proposition}
    \begin{proof}
      Assume $n \mid m$ and $m \mid k$.
      Take natural numbers $l,l'$ such that $n \cdot l = m$ and $m \cdot l' = k$.
      Then $n \cdot (l \cdot l')
        = (n \cdot l) \cdot l'
        = m \cdot l'
        = k$.
      Hence $n \mid k$.
    \end{proof}
  \end{forthel}

  \begin{forthel}
    \begin{proposition}[id=ARITHMETIC_07_4933275640397824,printid]
      Let $n, m$ be natural numbers such that $n \neq 0$.
      If $n \mid m$ and $m \mid n$ then $n = m$.
    \end{proposition}
    \begin{proof}
      Assume $n \mid m$ and $m \mid n$.
      Take natural numbers $k,k'$ such that $n \cdot k = m$ and $m \cdot k' = n$.
      Then $n
        = m \cdot k'
        = (n \cdot k) \cdot k'
        = n \cdot (k \cdot k')$.
      Hence $k \cdot k' = 1$.
      Thus $k = 1 = k'$.
      Therefore $n = m$.
    \end{proof}
  \end{forthel}

  \begin{forthel}
    \begin{proposition}[id=ARITHMETIC_07_1283495225720832,printid]
      Let $n, m, k$ be natural numbers.
      If $n \mid m$ then $k \cdot n \mid k \cdot m$.
    \end{proposition}
    \begin{proof}
      Assume $n \mid m$.
      Take a natural number $l$ such that $n \cdot l = m$.
      Then $(k \cdot n) \cdot l
        = k \cdot (n \cdot l)
        = k \cdot m$.
      Hence $k \cdot n \mid k \cdot m$.
    \end{proof}
  \end{forthel}

  \begin{forthel}
    \begin{proposition}[id=ARITHMETIC_07_6469492028735488,printid]
      Let $n, m, k$ be natural numbers.
      Assume $k \neq 0$.
      If $k \cdot n \mid k \cdot m$ then $n \mid m$.
    \end{proposition}
    \begin{proof}
      Assume $k \cdot n \mid k \cdot m$.
      Take a natural number $l$ such that $(k \cdot n) \cdot l = k \cdot m$.
      Then $k \cdot (n \cdot l) = k \cdot m$.
      Hence $n \cdot l = m$ (by \printref{ARITHMETIC_06_8575191374364672}).
      Thus $n \mid m$.
    \end{proof}
  \end{forthel}

  \begin{forthel}
    \begin{proposition}[id=ARITHMETIC_07_4700711333920768,printid]
      Let $n, m, k$ be natural numbers.
      If $k \mid n$ and $k \mid m$ then $k \mid (n' \cdot n) + (m' \cdot m)$
      for all natural numbers $n', m'$.
    \end{proposition}
    \begin{proof}
      Assume $k \mid n$ and $k \mid m$.
      Let $n', m'$ be natural numbers.
      Take natural numbers $l,l'$ such that $k \cdot l = n$ and $k \cdot l' = m$.
      Then
      \[  k \cdot ((n' \cdot l) + (m' \cdot l'))                \]
      \[    = (k \cdot (n' \cdot l)) + (k \cdot (m' \cdot l'))  \]
      \[    = ((k \cdot n') \cdot l) + ((k \cdot m') \cdot l')  \]
      \[    = (n' \cdot (k \cdot l)) + (m' \cdot (k \cdot l'))  \]
      \[    = (n' \cdot n) + (m' \cdot m).                      \]
    \end{proof}
  \end{forthel}

  \begin{forthel}
    \begin{corollary}[id=ARITHMETIC_07_1556786209357824,printid]
      Let $n, m, k$ be natural numbers.
      If $k \mid n$ and $k \mid m$ then $k \mid n + m$.
    \end{corollary}
    \begin{proof}
      Assume $k \mid n$ and $k \mid m$.
      Take $n' = 1$ and $m' = 1$.
      Then $k \mid (n' \cdot n) + (m' \cdot m)$.
      $(n' \cdot n) + (m' \cdot m) = n + m$.
      Hence $k \mid n + m$.
    \end{proof}
  \end{forthel}

  \begin{forthel}
    \begin{proposition}[id=ARITHMETIC_07_1076947887063040,printid]
      Let $n, m, k$ be natural numbers.
      If $k \mid n$ and $k \mid n + m$ then $k \mid m$.
    \end{proposition}
    \begin{proof}
      Assume $k \mid n$ and $k \mid n + m$.

      Case $k = 0$. Obvious.

      Case $k \neq 0$.
        Take a natural number $l$ such that $n = k \cdot l$.
        Take a natural number $l'$ such that $n + m = k \cdot l'$.
        Then $(k \cdot l) + m = k \cdot l'$.
        We have $l' \geq l$.
        Indeed if $l' < l$ then
        $n + m
          = k \cdot l'
          < k \cdot l
          = n$ (by \printref{ARITHMETIC_06_5048640368279552}).
        Hence we can take a natural number $l''$ such that $l' = l + l''$.
        Then $(k \cdot l) + m
          = k \cdot l'
          = k \cdot (l + l'')
          = (k \cdot l) + (k \cdot l'')$.
        Indeed $k \cdot (l + l'') = (k \cdot l) + (k \cdot l'')$ (by \printref{ARITHMETIC_06_9001524774567936}).
        Thus $m = (k \cdot l'')$ (by \printref{ARITHMETIC_03_8445946379632640}).
        Indeed $k \cdot l$ and $k \cdot l''$ are natural numbers.
        Therefore $k \mid m$.
      End.
    \end{proof}
  \end{forthel}

  \begin{forthel}
    \begin{proposition}[id=ARITHMETIC_07_2187144577679360,printid]
      Let $n, m$ be natural numbers such that $n, m \neq 0$.
      If $m \mid n$ then $m \leq n$.
    \end{proposition}
    \begin{proof}
      Assume $m \mid n$.
      Take a natural number $k$ such that $m \cdot k = n$.
      If $k = 0$ then
      $n
        = m \cdot k
        = m \cdot 0
        = 0$.
      Thus $k \geq 1$.
      Assume $m > n$.
      Then $n
        = m \cdot k
        \geq m \cdot 1
        = m
        > n$.
      Hence $n > n$.
      Contradiction.
    \end{proof}
  \end{forthel}
\end{document}
