\documentclass{naproche-library}
\libinput{arithmetics-preamble}
\begin{document}
\begin{smodule}{exponentiation-and-ordering.ftl}
  \importmodule[libraries]{arithmetics?multiplication-and-ordering.ftl}
  \importmodule[libraries]{arithmetics?exponentiation.ftl}

  \section*{Exponentiation and Ordering}

  \begin{proposition}[forthel,id=ARITHMETIC_09_3373702288769024]
    Let $n, m, k$ be natural numbers.
    Assume $k \neq 0$.
    Then $n < m$ iff $n^{k} < m^{k}$.
  \end{proposition}
  \begin{proof}[forthel]
    Case $n < m$.
      Define $\Phi = \{ k' \in \Nat \mid$ if $k' > 1$ then $n^{k'} < m^{k'} \}$.

      (1) $\Phi$ contains $0$.

      (2) $\Phi$ contains $1$.

      (3) $\Phi$ contains $2$. \newline
      Proof.
        Case $n = 0$ or $m = 0$. Obvious.

        Case $n, m \neq 0$.
          Then $n \cdot n
            < n \cdot m
            < m \cdot m$.
          Hence $n^{2}
            = n \cdot n
            < n \cdot m
            < m \cdot m
            = m^{2}$.
        End.
      Qed.

      (4) For all $k' \in \Phi$ we have $k' + 1 \in \Phi$. \newline
      Proof.
        Let $k' \in \Phi$.

        Let us show that if $k' + 1 > 1$ then $n^{k' + 1} < m^{k' + 1}$.
          Assume $k' + 1 > 1$.
          Then $n^{k'} < m^{k'}$.
          Indeed $k' \neq 0$ and $if k' = 1$ then $n^{k'} < m^{k'}$.

          Case $k' \leq 1$.
            Then $k' = 0$ or $k' = 1$.
            Hence $k' + 1 = 1$ or $k' + 1 = 2$.
            Thus $k' + 1 \in \Phi$.
            Therefore $n^{k' + 1} < m^{k' + 1}$.
          End.

          Case $k' > 1$.
            Case $n = 0$.
              Then $m \neq 0$.
              Hence $n^{k' + 1}
                = 0
                < m^{k'} \cdot m
                = m^{k' + 1}$.
              Thus $n^{k' + 1} < m^{k' + 1}$.
            End.

            Case $n \neq 0$.
              Then $n^{k'} \cdot n
                < m^{k'} \cdot n
                < m^{k'} \cdot m$.
              Indeed $n^{k'} < m^{k'} \neq 0$.
              Take $A = n^{k' + 1}$ and $B = m^{k' + 1}$. %!
              Then $A
                = n^{k' + 1}
                = n^{k'} \cdot n
                < m^{k'} \cdot n
                < m^{k'} \cdot m
                = m^{k' + 1}
                = B$.
              Take $X = m^{k'} \cdot n$ and $Y = m^{k'} \cdot m$.
              Then $A < X < Y = B$.
              Hence $A < B$.
              Thus $n^{k' + 1} < m^{k' + 1}$.
            End.
          End.

          Hence $n^{k' + 1} < m^{k' + 1}$.
          Indeed $k' \leq 1$ or $k' > 1$.
        End.

        Thus $k' + 1 \in \Phi$.
      Qed.

      Therefore every natural number is contained in $\Phi$ (by \printref{ARITHMETIC_01_4764664342773760}).
      Consequently $n^{k} < m^{k}$.
    End.

    Case $n^{k} < m^{k}$.
      Define $\Psi = \{ k' \in \Nat \mid$ if $n \geq m$ then $n^{k'} \geq m^{k'} \}$.

      (1) $\Psi$ contains $0$.

      (2) For all $k' \in \Psi$ we have $k' + 1 \in \Psi$. \newline
      Proof.
        Let $k' \in \Psi$.

        Let us show that if $n \geq m$ then $n^{k' + 1} \geq m^{k' + 1}$.
          Assume $n \geq m$.
          Then $n^{k'} \geq m^{k'}$.
          Hence $n^{k'} \cdot n \geq m^{k'} \cdot n \geq m^{k'} \cdot m$.
          Take $A = n^{k' + 1}$ and $B = m^{k' + 1}$. %!
          Thus $A
            = n^{k' + 1}
            = n^{k'} \cdot n
            \geq m^{k'} \cdot n
            \geq m^{k'} \cdot m
            = m^{k' + 1}
            = B$.
          Therefore $n^{k' + 1} = A \geq B = m^{k' + 1}$.
        End.

        Hence $k' + 1 \in \Psi$.
      Qed.

      Thus every natural number is contained in $\Psi$ (by \printref{ARITHMETIC_01_4764664342773760}).
      Therefore if $n \geq m$ then $n^{k} \geq m^{k}$.
      Consequently $n < m$.
    End.
  \end{proof}

  \begin{corollary}[forthel,id=ARITHMETIC_09_2797602550579200]
    Let $n, m, k$ be natural numbers.
    Assume $k \neq 0$.
    Then \[ n^{k} = m^{k} \implies n = m. \]
  \end{corollary}
  \begin{proof}[forthel]
    Assume $n^{k} = m^{k}$.
    Suppose $n \neq m$.
    Then $n < m$ or $m < n$.
    If $n < m$ then $n^{k} < m^{k}$.
    If $m < n$ then $m^{k} < n^{k}$.
    Thus $n^{k} \neq m^{k}$.
    Contradiction.
  \end{proof}

  \begin{corollary}[forthel,id=ARITHMETIC_09_6875081963732992]
    Let $n, m, k$ be natural numbers.
    Assume $k \neq 0$.
    Then \[ n^{k} \leq m^{k} \iff n \leq m. \]
  \end{corollary}
  \begin{proof}[forthel]
    If $n^{k} < m^{k}$ then $n < m$.
    If $n^{k} = m^{k}$ then $n = m$.

    If $n < m$ then $n^{k} < m^{k}$.
    If $n = m$ then $n^{k} = m^{k}$.
  \end{proof}

  \begin{proposition}[forthel,id=ARITHMETIC_09_3349764703780864]
    Let $n, m, k$ be natural numbers.
    Assume $k > 1$.
    Then \[ n < m \iff k^{n} < k^{m}. \]
  \end{proposition}
  \begin{proof}[forthel]
    Case $n < m$.
      Define $\Phi = \{ m' \in \Nat \mid$ if $n < m'$ then $k^{n} < k^{m'} \}$.

      (1) $\Phi$ contains $0$.

      (2) For all $m' \in \Phi$ we have $m' + 1 \in \Phi$. \newline
      Proof.
        Let $m' \in \Phi$.

        Let us show that if $n < m' + 1$ then $k^{n} < k^{m' + 1}$.
          Assume $n < m' + 1$.
          Then $n \leq m'$.
          We have $k^{m'} \cdot 1 < k^{m'} \cdot k$.
          Indeed $k^{m'} \neq 0$.

          Case $n = m'$.
            Take $A = k^{n}$ and $B = k^{m' + 1}$. %!
            Then $A
              = k^{n}
              = k^{m'}
              < k^{m'} \cdot k
              = k^{m' + 1}
              = B$.
            Hence $k^{n} = A < B = k^{m' + 1}$.
          End.

          Case $n < m'$.
            Take $A = k^{n}$ and $B = k^{m' + 1}$. %!
            Then $A
              = k^{n}
              < k^{m'}
              < k^{m'} \cdot k
              = k^{m' + 1}
              = B$.
            Hence $k^{n} = A < B = k^{m' + 1}$.
          End.
        Qed.
      Qed.

      Hence every natural number is contained in $\Phi$ (by \printref{ARITHMETIC_01_4764664342773760}).
      Thus $k^{n} < k^{m}$.
    End.

    Case $k^{n} < k^{m}$.
      Define $\Psi = \{ n' \in \Nat \mid$ if $n' \geq m$ then
      $k^{n'} \geq k^{m} \}$.

      (1) $0$ is contained in $\Psi$.

      (2) For all $n' \in \Psi$ we have $n' + 1 \in \Psi$. \newline
      Proof.
        Let $n' \in \Psi$.

        Let us show that if $n' + 1 \geq m$ then $k^{n' + 1} \geq k^{m}$.
          Assume $n' + 1 \geq m$.

          Case $n' + 1 = m$. Obvious.

          Case $n' + 1 > m$.
            Then $n' \geq m$.
            Hence $k^{n'} \geq k^{m}$.
            We have $k^{n'} \cdot 1 \leq k^{n'} \cdot k$.
            Indeed $1 \leq k$ and $k^{n'} \neq 0$.
            Take $A = k^{m}$ and $B = k^{n' + 1}$. %!
            Then $A
              = k^{m}
              \leq k^{n'}
              = k^{n'} \cdot 1
              \leq k^{n'} \cdot k
              = k^{n' + 1}
              = B$.
            Hence $k^{m} = A \leq B = k^{n' + 1}$.
          End.
        Qed.
      Qed.

      Thus every natural number is contained in $\Psi$ (by \printref{ARITHMETIC_01_4764664342773760}).
      Therefore if $n \geq m$ then $k^{n} \geq k^{m}$.
      Consequently $n < m$.
    End.
  \end{proof}

  \begin{corollary}[forthel,id=ARITHMETIC_09_6780506905509888]
    Let $n, m, k$ be natural numbers.
    Assume $k > 1$.
    If $k^{n} = k^{m}$ then $n = m$.
  \end{corollary}
  \begin{proof}[forthel]
    Assume $k^{n} = k^{m}$.
    Suppose $n \neq m$.
    Then $n < m$ or $m < n$.
    If $n < m$ then $k^{n} < k^{m}$.
    If $m < n$ then $k^{m} < k^{n}$.
    Thus $k^{n} \neq k^{m}$.
    Contradiction.
  \end{proof}

  \begin{corollary}[forthel,id=ARITHMETIC_09_2876620253691904]
    Let $n, m, k$ be natural numbers.
    Assume $k > 1$.
    Then $n \leq m$ iff $k^{n} \leq k^{m}$.
  \end{corollary}
  \begin{proof}[forthel]
    If $n \leq m$ then $k^{n} \leq k^{m}$.

    If $k^{n} = k^{m}$ then $n = m$.
    If $k^{n} < k^{m}$ then $n < m$.
  \end{proof}

  \begin{proposition}[forthel,id=ARITHMETIC_09_6984104377581568]
    Let $n$ be a natural number.
    Then $(n + 1)^{2} = (n^{2} + (2 \cdot n)) + 1$.
  \end{proposition}
  \begin{proof}[forthel]
    We have
    \[  (n + 1)^{2}                       \]
    \[    = (n + 1) \cdot (n + 1)         \]
    \[    = ((n + 1) \cdot n) + (n + 1)   \]
    \[    = ((n \cdot n) + n) + (n + 1)   \]
    \[    = (n^{2} + n) + (n + 1)         \]
    \[    = ((n^{2} + n) + n) + 1         \]
    \[    = (n^{2} + (n + n)) + 1         \]
    \[    = (n^{2} + (2 \cdot n)) + 1.    \]
  \end{proof}

  \begin{proposition}[forthel,id=ARITHMETIC_09_134060414337024]
    Let $n$ be a natural number.
    Assume $n \geq 3$.
    Then $n^{2} > (2 \cdot n) + 1$.
  \end{proposition}
  \begin{proof}[forthel]
    (a) Define $\Phi = \{ n' \in \Nat_{\geq 3} \mid n'^{2} > (2 \cdot n') + 1 \}$.

    (1) $\Phi$ contains $3$.
    Indeed $3^{2} > (2 \cdot 3) + 1$.

    (2) For all $n' \in \Phi$ we have $n' + 1 \in \Phi$. \newline
    Proof.
      Let $n' \in \Phi$.
      Then $n' \geq 3$.

      Take $V = (n'^{2} + (2 \cdot n')) + 1$ and $W = (((2 \cdot n') + 1) + (2 \cdot n')) + 1$.
      Then $V > W$. \newline
      Proof.
        We have $n'^{2} > (2 \cdot n') + 1$.
        Hence $n'^{2} + (2 \cdot n') > ((2 \cdot n') + 1) + (2 \cdot n')$ (by \printref{ARITHMETIC_04_7354062662008832}).
      Qed.

      Take $X = ((2 \cdot n') + (2 \cdot n')) + 1$.
      Then $W > X$. \newline
      Proof.
        We have $((2 \cdot n') + 1) + (2 \cdot n') > (2 \cdot n') + (2 \cdot n')$.
        Indeed $(2 \cdot n') + 1 > 2 \cdot n'$.
      Qed.

      Take $Y = (2 \cdot (n' + n')) + 1$ and $Z = (2 \cdot (n' + 1)) + 1$.
      Then $Y > Z$. \newline
      Proof.
        We have $n' + n' > n' + 1$ and $2 \neq 0$.
        Thus $2 \cdot (n' + n') > 2 \cdot (n' + 1)$ (by \printref{ARITHMETIC_06_5048640368279552}).
        Indeed $n' + n'$ and $n' + 1$ are natural numbers.
      Qed.

      Then $(n' + 1)^{2} = V > W > X = Y > Z = (2 \cdot (n' + 1)) + 1$.
      Hence $(n' + 1)^{2} > (2 \cdot (n' + 1)) + 1$.
    Qed.

    Therefore $\Phi$ contains every element of $\Nat_{\geq 3}$ (by \printref{ARITHMETIC_04_4976599269113856}, 1, 2).
    Consequently $n^{2} > (2 \cdot n) + 1$ (by a).
  \end{proof}

  \begin{proposition}[forthel,id=ARITHMETIC_09_1097950367318016]
    Let $n$ be a natural number.
    Assume $n \geq 5$.
    Then $2^{n} > n^{2}$.
  \end{proposition}
  \begin{proof}[forthel]
    (a) Define $\Phi = \{ n' \in \Nat_{\geq 5} \mid 2^{n'} > n'^{2} \}$.

    (1) $\Phi$ contains $5$.
    Indeed we can show that $2^{5} > 5^{2}$.
      We have $2^{5}
        = 2 \cdot (2 \cdot (2 \cdot (2 \cdot 2)))
        = 8 \cdot 4
        = (5 + 3) \cdot 4
        = (5 \cdot 4) + (3 \cdot 4)
        = (5 \cdot 4) + ((5 \cdot 2) + 2)
        = (5 \cdot (4 + 2)) + 2
        = (5 \cdot 6) + 2
        > 5 \cdot 6
        > 5 \cdot 5
        = 5^{2}$.
      Indeed $((5 \cdot 4) + ((5 \cdot 2) + 2)) = (5 \cdot (4 + 2)) + 2$. %!
    End.

    (2) For all $n' \in \Phi$ we have $n' + 1 \in \Phi$. \newline
    Proof.
      Let $n' \in \Phi$.
      Then $n' \geq 5$ and $2^{n'} > n'^{2}$.

      Take $V = 2^{n'} \cdot 2$ and $W = n'^{2} \cdot 2$.
      Then $V > W$.
      Indeed $2 \neq 0$.

      Take $X = n'^{2} + n'^{2}$.
      Then $W = X$.

      Take $Y = n'^{2} + ((2 \cdot n') + 1)$.
      Then $X > Y$.
      Indeed $n'^{2} > (2 \cdot n') + 1$.

      Take $Z = (n' + 1)^{2}$.
      Then $Y = Z$.

      Then $2^{n' + 1} = V > W = X > Y = Z = (n' + 1)^{2}$.
      Hence $2^{n' + 1} > (n' + 1)^{2}$.
    Qed.

    Therefore $\Phi$ contains every element of $\Nat_{\geq 5}$ (by \printref{ARITHMETIC_04_4976599269113856}).
    Consequently $2^{n} > n^{2}$ (by a).
  \end{proof}
\end{smodule}
\end{document}
