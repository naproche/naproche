\documentclass[10pt]{article}
\usepackage[english]{babel}
\usepackage{../../../meta-inf/lib/naproche}
% Top-level sections and proofs

\newtheoremstyle{foundations}% name of the style to be used
  {\dimexpr\ftlparskip+4pt\relax}% measure of space to leave above the theorem.
  % (We must add \ftlparskip since otherwise the name of the label would not fit
  % into the border drawn around the environment.)
  {4pt}% measure of space to leave below the theorem.
  {\normalfont}% name of font to use in the body of the theorem
  {0pt}% measure of space to indent
  {\bfseries}% name of head font
  {}% punctuation between head and body
  { }% space after theorem head; " " = normal inter-word space
  {\thmname{#1}\thmnumber{ #2}\thmnote{ (#3)}.}

\let\definition\relax
\let\enddefinition\relax
\let\signature\relax
\let\endsignature\relax
\let\axiom\relax
\let\endaxiom\relax
\let\theorem\relax
\let\endtheorem\relax
\let\proposition\relax
\let\endproposition\relax
\let\lemma\relax
\let\endlemma\relax
\let\corollary\relax
\let\endcorollary\relax

\theoremstyle{foundations}

\newmdtheoremenv[
  backgroundcolor=lightgray!10,
  linecolor=black,
  innertopmargin=1em
]{definition}{Definition}[section]
\newmdtheoremenv[
  backgroundcolor=lightgray!10,
  linecolor=black,
  innertopmargin=1em
]{signature}[definition]{Signature}
\newmdtheoremenv[
  backgroundcolor=lightgray!10,
  linecolor=black,
  innertopmargin=1em
]{axiom}[definition]{Axiom}
\newmdtheoremenv[
  backgroundcolor=lightgray!10,
  linecolor=black,
  innertopmargin=1em
]{theorem}[definition]{Theorem}
\newmdtheoremenv[
  backgroundcolor=lightgray!10,
  linecolor=black,
  innertopmargin=1em
]{proposition}[definition]{Proposition}
\newmdtheoremenv[
  backgroundcolor=lightgray!10,
  linecolor=black,
  innertopmargin=1em
]{lemma}[definition]{Lemma}
\newmdtheoremenv[
  backgroundcolor=lightgray!10,
  linecolor=black,
  innertopmargin=1em
]{corollary}[definition]{Corollary}


% Hide imports

\usepackage{scontents}
\newenvsc{imports}[store-env=imports,print-env=false]


% Labels of top-level sections

\renewcommand{\showlabelfont}{\scriptsize\ttfamily\color{gray}}


% Non-symbolic "\implies" and "\iff"

\renewcommand{\implies}{\quad \text{implies} \quad}
\renewcommand{\iff}{\quad \text{iff} \quad}

\usepackage{amssymb}
\usepackage{pgffor}

\newcommand{\id}{\textrm{id}}
\newcommand{\range}{\textrm{range}}
\newcommand{\const}[2]{\textrm{const}_{#1}^{#2}}
\newcommand{\symdiff}{\mathop{\triangle}}
\newcommand{\pow}{\mathcal{P}}
\newcommand{\onto}{\twoheadrightarrow}
\newcommand{\into}{\hookrightarrow}

% E.g. `\set{1, 2, 3, 4}` prints the numbers 1, 2, 3 and 4 as a comma
% separated list which is enclosed within curly braces.
\def\set#1{\ensuremath{
  \{\foreach[count=\i]\x in {#1}{
    \ifnum\i>1,\,\fi
    \x
  }\}
}}

\useackage{amssymb}

\newcommand{\Nat}{\mathbb{N}}
\newcommand{\Prime}{\mathbb{P}}
\renewcommand{\succ}{\textrm{succ}}
\newcommand{\pred}{\textrm{pred}}
\newcommand{\add}{\textrm{add}}
\newcommand{\mul}{\textrm{mul}}
\renewcommand{\exp}{\textrm{exp}}
\newcommand{\fac}{\textrm{fac}}
\renewcommand{\div}{\mathrel{\textrm{div}}}
\renewcommand{\mod}{\mathrel{\textrm{mod}}}

\begin{document}
  \begin{imports}
    \begin{forthel}
      %[prove off][check off]
      [read \path{libraries/source/arithmetics/multiplication-and-ordering.ftl.tex}]
      [read \path{libraries/source/arithmetics/exponentiation.ftl.tex}]
      %[prove on][check on]
    \end{forthel}
  \end{imports}


  \section*{Exponentiation and Ordering}

  \begin{forthel}
    \begin{proposition}[id=ARITHMETIC_09_3373702288769024,printid]
      Let $n, m, k$ be natural numbers.
      Assume $k \neq 0$.
      Then $n < m$ iff $n^{k} < m^{k}$.
    \end{proposition}
    \begin{proof}
      Case $n < m$.
        Define $\Phi = \{ k' \in \Nat \mid$ if $k' > 1$ then $n^{k'} < m^{k'} \}$.

        (1) $\Phi$ contains $0$.

        (2) $\Phi$ contains $1$.

        (3) $\Phi$ contains $2$. \\
        Proof.
          Case $n = 0$ or $m = 0$. Obvious.

          Case $n, m \neq 0$.
            Then $n \cdot n
              < n \cdot m
              < m \cdot m$.
            Hence $n^{2}
              = n \cdot n
              < n \cdot m
              < m \cdot m
              = m^{2}$.
          End.
        Qed.

        (4) For all $k' \in \Phi$ we have $k' + 1 \in \Phi$. \\
        Proof.
          Let $k' \in \Phi$.

          Let us show that if $k' + 1 > 1$ then $n^{k' + 1} < m^{k' + 1}$.
            Assume $k' + 1 > 1$.
            Then $n^{k'} < m^{k'}$.
            Indeed $k' \neq 0$ and $if k' = 1$ then $n^{k'} < m^{k'}$.

            Case $k' \leq 1$.
              Then $k' = 0$ or $k' = 1$.
              Hence $k' + 1 = 1$ or $k' + 1 = 2$.
              Thus $k' + 1 \in \Phi$.
              Therefore $n^{k' + 1} < m^{k' + 1}$.
            End.

            Case $k' > 1$.
              Case $n = 0$.
                Then $m \neq 0$.
                Hence $n^{k' + 1}
                  = 0
                  < m^{k'} \cdot m
                  = m^{k' + 1}$.
                Thus $n^{k' + 1} < m^{k' + 1}$.
              End.

              Case $n \neq 0$.
                Then $n^{k'} \cdot n
                  < m^{k'} \cdot n
                  < m^{k'} \cdot m$.
                Indeed $n^{k'} < m^{k'} \neq 0$.
                Take $A = n^{k' + 1}$ and $B = m^{k' + 1}$. %!
                Then $A
                  = n^{k' + 1}
                  = n^{k'} \cdot n
                  < m^{k'} \cdot n
                  < m^{k'} \cdot m
                  = m^{k' + 1}
                  = B$.
                Take $X = m^{k'} \cdot n$ and $Y = m^{k'} \cdot m$.
                Then $A < X < Y = B$.
                Hence $A < B$.
                Thus $n^{k' + 1} < m^{k' + 1}$.
              End.
            End.

            Hence $n^{k' + 1} < m^{k' + 1}$.
            Indeed $k' \leq 1$ or $k' > 1$.
          End.

          Thus $k' + 1 \in \Phi$.
        Qed.

        Therefore every natural number is contained in $\Phi$ (by \printref{ARITHMETIC_01_4764664342773760}).
        Consequently $n^{k} < m^{k}$.
      End.

      Case $n^{k} < m^{k}$.
        Define $\Psi = \{ k' \in \Nat \mid$ if $n \geq m$ then $n^{k'} \geq m^{k'} \}$.

        (1) $\Psi$ contains $0$.

        (2) For all $k' \in \Psi$ we have $k' + 1 \in \Psi$. \\
        Proof.
          Let $k' \in \Psi$.

          Let us show that if $n \geq m$ then $n^{k' + 1} \geq m^{k' + 1}$.
            Assume $n \geq m$.
            Then $n^{k'} \geq m^{k'}$.
            Hence $n^{k'} \cdot n \geq m^{k'} \cdot n \geq m^{k'} \cdot m$.
            Take $A = n^{k' + 1}$ and $B = m^{k' + 1}$. %!
            Thus $A
              = n^{k' + 1}
              = n^{k'} \cdot n
              \geq m^{k'} \cdot n
              \geq m^{k'} \cdot m
              = m^{k' + 1}
              = B$.
            Therefore $n^{k' + 1} = A \geq B = m^{k' + 1}$.
          End.

          Hence $k' + 1 \in \Psi$.
        Qed.

        Thus every natural number is contained in $\Psi$ (by \printref{ARITHMETIC_01_4764664342773760}).
        Therefore if $n \geq m$ then $n^{k} \geq m^{k}$.
        Consequently $n < m$.
      End.
    \end{proof}
  \end{forthel}

  \begin{forthel}
    \begin{corollary}[id=ARITHMETIC_09_2797602550579200,printid]
      Let $n, m, k$ be natural numbers.
      Assume $k \neq 0$.
      Then \[ n^{k} = m^{k} \implies n = m. \]
    \end{corollary}
    \begin{proof}
      Assume $n^{k} = m^{k}$.
      Suppose $n \neq m$.
      Then $n < m$ or $m < n$.
      If $n < m$ then $n^{k} < m^{k}$.
      If $m < n$ then $m^{k} < n^{k}$.
      Thus $n^{k} \neq m^{k}$.
      Contradiction.
    \end{proof}
  \end{forthel}

  \begin{forthel}
    \begin{corollary}[id=ARITHMETIC_09_6875081963732992,printid]
      Let $n, m, k$ be natural numbers.
      Assume $k \neq 0$.
      Then \[ n^{k} \leq m^{k} \iff n \leq m. \]
    \end{corollary}
    \begin{proof}
      If $n^{k} < m^{k}$ then $n < m$.
      If $n^{k} = m^{k}$ then $n = m$.

      If $n < m$ then $n^{k} < m^{k}$.
      If $n = m$ then $n^{k} = m^{k}$.
    \end{proof}
  \end{forthel}

  \begin{forthel}
    \begin{proposition}[id=ARITHMETIC_09_3349764703780864,printid]
      Let $n, m, k$ be natural numbers.
      Assume $k > 1$.
      Then \[ n < m \iff k^{n} < k^{m}. \]
    \end{proposition}
    \begin{proof}
      Case $n < m$.
        Define $\Phi = \{ m' \in \Nat \mid$ if $n < m'$ then $k^{n} < k^{m'} \}$.

        (1) $\Phi$ contains $0$.

        (2) For all $m' \in \Phi$ we have $m' + 1 \in \Phi$. \\
        Proof.
          Let $m' \in \Phi$.

          Let us show that if $n < m' + 1$ then $k^{n} < k^{m' + 1}$.
            Assume $n < m' + 1$.
            Then $n \leq m'$.
            We have $k^{m'} \cdot 1 < k^{m'} \cdot k$.
            Indeed $k^{m'} \neq 0$.

            Case $n = m'$.
              Take $A = k^{n}$ and $B = k^{m' + 1}$. %!
              Then $A
                = k^{n}
                = k^{m'}
                < k^{m'} \cdot k
                = k^{m' + 1}
                = B$.
              Hence $k^{n} = A < B = k^{m' + 1}$.
            End.

            Case $n < m'$.
              Take $A = k^{n}$ and $B = k^{m' + 1}$. %!
              Then $A
                = k^{n}
                < k^{m'}
                < k^{m'} \cdot k
                = k^{m' + 1}
                = B$.
              Hence $k^{n} = A < B = k^{m' + 1}$.
            End.
          Qed.
        Qed.

        Hence every natural number is contained in $\Phi$ (by \printref{ARITHMETIC_01_4764664342773760}).
        Thus $k^{n} < k^{m}$.
      End.

      Case $k^{n} < k^{m}$.
        Define $\Psi = \{ n' \in \Nat \mid$ if $n' \geq m$ then
        $k^{n'} \geq k^{m} \}$.

        (1) $0$ is contained in $\Psi$.

        (2) For all $n' \in \Psi$ we have $n' + 1 \in \Psi$. \\
        Proof.
          Let $n' \in \Psi$.

          Let us show that if $n' + 1 \geq m$ then $k^{n' + 1} \geq k^{m}$.
            Assume $n' + 1 \geq m$.

            Case $n' + 1 = m$. Obvious.

            Case $n' + 1 > m$.
              Then $n' \geq m$.
              Hence $k^{n'} \geq k^{m}$.
              We have $k^{n'} \cdot 1 \leq k^{n'} \cdot k$.
              Indeed $1 \leq k$ and $k^{n'} \neq 0$.
              Take $A = k^{m}$ and $B = k^{n' + 1}$. %!
              Then $A
                = k^{m}
                \leq k^{n'}
                = k^{n'} \cdot 1
                \leq k^{n'} \cdot k
                = k^{n' + 1}
                = B$.
              Hence $k^{m} = A \leq B = k^{n' + 1}$.
            End.
          Qed.
        Qed.

        Thus every natural number is contained in $\Psi$ (by \printref{ARITHMETIC_01_4764664342773760}).
        Therefore if $n \geq m$ then $k^{n} \geq k^{m}$.
        Consequently $n < m$.
      End.
    \end{proof}
  \end{forthel}

  \begin{forthel}
    \begin{corollary}[id=ARITHMETIC_09_6780506905509888,printid]
      Let $n, m, k$ be natural numbers.
      Assume $k > 1$.
      If $k^{n} = k^{m}$ then $n = m$.
    \end{corollary}
    \begin{proof}
      Assume $k^{n} = k^{m}$.
      Suppose $n \neq m$.
      Then $n < m$ or $m < n$.
      If $n < m$ then $k^{n} < k^{m}$.
      If $m < n$ then $k^{m} < k^{n}$.
      Thus $k^{n} \neq k^{m}$.
      Contradiction.
    \end{proof}
  \end{forthel}

  \begin{forthel}
    \begin{corollary}[id=ARITHMETIC_09_2876620253691904,printid]
      Let $n, m, k$ be natural numbers.
      Assume $k > 1$.
      Then $n \leq m$ iff $k^{n} \leq k^{m}$.
    \end{corollary}
    \begin{proof}
      If $n \leq m$ then $k^{n} \leq k^{m}$.

      If $k^{n} = k^{m}$ then $n = m$.
      If $k^{n} < k^{m}$ then $n < m$.
    \end{proof}
  \end{forthel}

  \begin{forthel}
    \begin{proposition}[id=ARITHMETIC_09_6984104377581568,printid]
      Let $n$ be a natural number.
      Then $(n + 1)^{2} = (n^{2} + (2 \cdot n)) + 1$.
    \end{proposition}
    \begin{proof}
      We have
      \[  (n + 1)^{2}                       \]
      \[    = (n + 1) \cdot (n + 1)         \]
      \[    = ((n + 1) \cdot n) + (n + 1)   \]
      \[    = ((n \cdot n) + n) + (n + 1)   \]
      \[    = (n^{2} + n) + (n + 1)         \]
      \[    = ((n^{2} + n) + n) + 1         \]
      \[    = (n^{2} + (n + n)) + 1         \]
      \[    = (n^{2} + (2 \cdot n)) + 1.    \]
    \end{proof}
  \end{forthel}

  \begin{forthel}
    \begin{proposition}[id=ARITHMETIC_09_134060414337024,printid]
      Let $n$ be a natural number.
      Assume $n \geq 3$.
      Then $n^{2} > (2 \cdot n) + 1$.
    \end{proposition}
    \begin{proof}
      (a) Define $\Phi = \{ n' \in \Nat_{\geq 3} \mid n'^{2} > (2 \cdot n') + 1 \}$.

      (1) $\Phi$ contains $3$.
      Indeed $3^{2} > (2 \cdot 3) + 1$.

      (2) For all $n' \in \Phi$ we have $n' + 1 \in \Phi$. \\
      Proof.
        Let $n' \in \Phi$.
        Then $n' \geq 3$.

        Take $V = (n'^{2} + (2 \cdot n')) + 1$ and $W = (((2 \cdot n') + 1) + (2 \cdot n')) + 1$.
        Then $V > W$. \\
        Proof.
          We have $n'^{2} > (2 \cdot n') + 1$.
          Hence $n'^{2} + (2 \cdot n') > ((2 \cdot n') + 1) + (2 \cdot n')$ (by \printref{ARITHMETIC_04_7354062662008832}).
        Qed.

        Take $X = ((2 \cdot n') + (2 \cdot n')) + 1$.
        Then $W > X$. \\
        Proof.
          We have $((2 \cdot n') + 1) + (2 \cdot n') > (2 \cdot n') + (2 \cdot n')$.
          Indeed $(2 \cdot n') + 1 > 2 \cdot n'$.
        Qed.

        Take $Y = (2 \cdot (n' + n')) + 1$ and $Z = (2 \cdot (n' + 1)) + 1$.
        Then $Y > Z$. \\
        Proof.
          We have $n' + n' > n' + 1$ and $2 \neq 0$.
          Thus $2 \cdot (n' + n') > 2 \cdot (n' + 1)$ (by \printref{ARITHMETIC_06_5048640368279552}).
          Indeed $n' + n'$ and $n' + 1$ are natural numbers.
        Qed.

        Then $(n' + 1)^{2} = V > W > X = Y > Z = (2 \cdot (n' + 1)) + 1$.
        Hence $(n' + 1)^{2} > (2 \cdot (n' + 1)) + 1$.
      Qed.

      Therefore $\Phi$ contains every element of $\Nat_{\geq 3}$ (by \printref{ARITHMETIC_04_4976599269113856}, 1, 2).
      Consequently $n^{2} > (2 \cdot n) + 1$ (by a).
    \end{proof}
  \end{forthel}

  \begin{forthel}
    \begin{proposition}[id=ARITHMETIC_09_1097950367318016,printid]
      Let $n$ be a natural number.
      Assume $n \geq 5$.
      Then $2^{n} > n^{2}$.
    \end{proposition}
    \begin{proof}
      (a) Define $\Phi = \{ n' \in \Nat_{\geq 5} \mid 2^{n'} > n'^{2} \}$.
  
      (1) $\Phi$ contains $5$.
      Indeed we can show that $2^{5} > 5^{2}$.
        We have $2^{5}
          = 2 \cdot (2 \cdot (2 \cdot (2 \cdot 2)))
          = 8 \cdot 4
          = (5 + 3) \cdot 4
          = (5 \cdot 4) + (3 \cdot 4)
          = (5 \cdot 4) + ((5 \cdot 2) + 2)
          = (5 \cdot (4 + 2)) + 2
          = (5 \cdot 6) + 2
          > 5 \cdot 6
          > 5 \cdot 5
          = 5^{2}$.
        Indeed $((5 \cdot 4) + ((5 \cdot 2) + 2)) = (5 \cdot (4 + 2)) + 2$. %!
      End.

      (2) For all $n' \in \Phi$ we have $n' + 1 \in \Phi$. \\
      Proof.
        Let $n' \in \Phi$.
        Then $n' \geq 5$ and $2^{n'} > n'^{2}$.
  
        Take $V = 2^{n'} \cdot 2$ and $W = n'^{2} \cdot 2$.
        Then $V > W$.
        Indeed $2 \neq 0$.
  
        Take $X = n'^{2} + n'^{2}$.
        Then $W = X$.
  
        Take $Y = n'^{2} + ((2 \cdot n') + 1)$.
        Then $X > Y$.
        Indeed $n'^{2} > (2 \cdot n') + 1$.
  
        Take $Z = (n' + 1)^{2}$.
        Then $Y = Z$.
  
        Then $2^{n' + 1} = V > W = X > Y = Z = (n' + 1)^{2}$.
        Hence $2^{n' + 1} > (n' + 1)^{2}$.
      Qed.
  
      Therefore $\Phi$ contains every element of $\Nat_{\geq 5}$ (by \printref{ARITHMETIC_04_4976599269113856}).
      Consequently $2^{n} > n^{2}$ (by a).
    \end{proof}
  \end{forthel}
\end{document}
