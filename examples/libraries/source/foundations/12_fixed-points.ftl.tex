\documentclass[10pt]{article}
\usepackage[english]{babel}
\usepackage{../../../../lib/tex/naproche}
% Top-level sections and proofs

\newtheoremstyle{foundations}% name of the style to be used
  {\dimexpr\ftlparskip+4pt\relax}% measure of space to leave above the theorem.
  % (We must add \ftlparskip since otherwise the name of the label would not fit
  % into the border drawn around the environment.)
  {4pt}% measure of space to leave below the theorem.
  {\normalfont}% name of font to use in the body of the theorem
  {0pt}% measure of space to indent
  {\bfseries}% name of head font
  {}% punctuation between head and body
  { }% space after theorem head; " " = normal inter-word space
  {\thmname{#1}\thmnumber{ #2}\thmnote{ (#3)}.}

\let\definition\relax
\let\enddefinition\relax
\let\signature\relax
\let\endsignature\relax
\let\axiom\relax
\let\endaxiom\relax
\let\theorem\relax
\let\endtheorem\relax
\let\proposition\relax
\let\endproposition\relax
\let\lemma\relax
\let\endlemma\relax
\let\corollary\relax
\let\endcorollary\relax

\theoremstyle{foundations}

\newmdtheoremenv[
  backgroundcolor=lightgray!10,
  linecolor=black,
  innertopmargin=1em
]{definition}{Definition}[section]
\newmdtheoremenv[
  backgroundcolor=lightgray!10,
  linecolor=black,
  innertopmargin=1em
]{signature}[definition]{Signature}
\newmdtheoremenv[
  backgroundcolor=lightgray!10,
  linecolor=black,
  innertopmargin=1em
]{axiom}[definition]{Axiom}
\newmdtheoremenv[
  backgroundcolor=lightgray!10,
  linecolor=black,
  innertopmargin=1em
]{theorem}[definition]{Theorem}
\newmdtheoremenv[
  backgroundcolor=lightgray!10,
  linecolor=black,
  innertopmargin=1em
]{proposition}[definition]{Proposition}
\newmdtheoremenv[
  backgroundcolor=lightgray!10,
  linecolor=black,
  innertopmargin=1em
]{lemma}[definition]{Lemma}
\newmdtheoremenv[
  backgroundcolor=lightgray!10,
  linecolor=black,
  innertopmargin=1em
]{corollary}[definition]{Corollary}


% Hide imports

\usepackage{scontents}
\newenvsc{imports}[store-env=imports,print-env=false]


% Labels of top-level sections

\renewcommand{\showlabelfont}{\scriptsize\ttfamily\color{gray}}


% Non-symbolic "\implies" and "\iff"

\renewcommand{\implies}{\quad \text{implies} \quad}
\renewcommand{\iff}{\quad \text{iff} \quad}

\usepackage{amssymb}
\usepackage{pgffor}

\newcommand{\id}{\textrm{id}}
\newcommand{\range}{\textrm{range}}
\newcommand{\const}[2]{\textrm{const}_{#1}^{#2}}
\newcommand{\symdiff}{\mathop{\triangle}}
\newcommand{\pow}{\mathcal{P}}
\newcommand{\onto}{\twoheadrightarrow}
\newcommand{\into}{\hookrightarrow}

% E.g. `\set{1, 2, 3, 4}` prints the numbers 1, 2, 3 and 4 as a comma
% separated list which is enclosed within curly braces.
\def\set#1{\ensuremath{
  \{\foreach[count=\i]\x in {#1}{
    \ifnum\i>1,\,\fi
    \x
  }\}
}}

\begin{document}
  \begin{imports}
    \begin{forthel}
      %[prove off][check off]

      [readtex \path{libraries/source/foundations/10_sets.ftl.tex}]

      %[prove on][check on]
    \end{forthel}
  \end{imports}


  \section{Fixed Points}

  \begin{forthel}
    \begin{definition}\printlabel{FOUNDATIONS_12_2177076576649216}
      Let $f$ be a map.
      A fixed point of $f$ is an element $x$ of $\dom(f)$ such that $f(x) = x$.
    \end{definition}
  \end{forthel}

  \begin{forthel}
    \begin{definition}\printlabel{FOUNDATIONS_12_1394550966845440}
      A map between systems of sets is a map from some system of sets to some
      system of sets.
    \end{definition}
  \end{forthel}

  \begin{forthel}
    \begin{definition}\printlabel{FOUNDATIONS_12_3290499861446656}
      Let $f$ be a map between systems of sets.
      $f$ is subset preserving iff for all $x, y \in \dom(f)$
      \[ x \subseteq y \implies f(x) \subseteq f(y). \]
    \end{definition}
  \end{forthel}

  \begin{forthel}
    \begin{theorem}[Knaster-Tarski Axiom]\printlabel{FOUNDATIONS_12_8420450166112256}
      Let $x$ be a set.
      Let $f$ be a subset preserving map from $\pow(x)$ to $\pow(x)$.
      Then $f$ has a fixed point.
    \end{theorem}
    \begin{proof}
      (1) Define $A = \{ y \mid y \subseteq x$ and $y \subseteq f(y) \}$.
      Then $A$ is a subset of $\pow(x)$.
      We have $\bigcup A \in \pow(x)$.

      Let us show that (2) $\bigcup A \subseteq f(\bigcup A)$.
        Let $u \in \bigcup A$.
        Take $y \in A$ such that $u \in y$.
        Then $u \in f(y)$.
        We have $y \subseteq \bigcup A$.
        Hence $f(y) \subseteq f(\bigcup A)$.
        Thus $f(y) \subseteq f(\bigcup A)$.
        Therefore $u \in f(\bigcup A)$.
      End.

      Then $f(\bigcup A) \in A$ (by 1).
      Indeed $f(\bigcup A) \subseteq x$.
      (3) Hence $f(\bigcup A) \subseteq \bigcup A$.
      Indeed every element of $f(\bigcup A)$ is an element of some element of
      $A$.

      Thus $f(\bigcup A) = \bigcup A$ (by 2, 3).
    \end{proof}
  \end{forthel}
\end{document}
