\documentclass[10pt]{article}
\usepackage[english]{babel}
\usepackage{../../../../lib/tex/naproche}
% Top-level sections and proofs

\newtheoremstyle{foundations}% name of the style to be used
  {\dimexpr\ftlparskip+4pt\relax}% measure of space to leave above the theorem.
  % (We must add \ftlparskip since otherwise the name of the label would not fit
  % into the border drawn around the environment.)
  {4pt}% measure of space to leave below the theorem.
  {\normalfont}% name of font to use in the body of the theorem
  {0pt}% measure of space to indent
  {\bfseries}% name of head font
  {}% punctuation between head and body
  { }% space after theorem head; " " = normal inter-word space
  {\thmname{#1}\thmnumber{ #2}\thmnote{ (#3)}.}

\let\definition\relax
\let\enddefinition\relax
\let\signature\relax
\let\endsignature\relax
\let\axiom\relax
\let\endaxiom\relax
\let\theorem\relax
\let\endtheorem\relax
\let\proposition\relax
\let\endproposition\relax
\let\lemma\relax
\let\endlemma\relax
\let\corollary\relax
\let\endcorollary\relax

\theoremstyle{foundations}

\newmdtheoremenv[
  backgroundcolor=lightgray!10,
  linecolor=black,
  innertopmargin=1em
]{definition}{Definition}[section]
\newmdtheoremenv[
  backgroundcolor=lightgray!10,
  linecolor=black,
  innertopmargin=1em
]{signature}[definition]{Signature}
\newmdtheoremenv[
  backgroundcolor=lightgray!10,
  linecolor=black,
  innertopmargin=1em
]{axiom}[definition]{Axiom}
\newmdtheoremenv[
  backgroundcolor=lightgray!10,
  linecolor=black,
  innertopmargin=1em
]{theorem}[definition]{Theorem}
\newmdtheoremenv[
  backgroundcolor=lightgray!10,
  linecolor=black,
  innertopmargin=1em
]{proposition}[definition]{Proposition}
\newmdtheoremenv[
  backgroundcolor=lightgray!10,
  linecolor=black,
  innertopmargin=1em
]{lemma}[definition]{Lemma}
\newmdtheoremenv[
  backgroundcolor=lightgray!10,
  linecolor=black,
  innertopmargin=1em
]{corollary}[definition]{Corollary}


% Hide imports

\usepackage{scontents}
\newenvsc{imports}[store-env=imports,print-env=false]


% Labels of top-level sections

\renewcommand{\showlabelfont}{\scriptsize\ttfamily\color{gray}}


% Non-symbolic "\implies" and "\iff"

\renewcommand{\implies}{\quad \text{implies} \quad}
\renewcommand{\iff}{\quad \text{iff} \quad}

\usepackage{amssymb}
\usepackage{pgffor}

\newcommand{\id}{\textrm{id}}
\newcommand{\range}{\textrm{range}}
\newcommand{\const}[2]{\textrm{const}_{#1}^{#2}}
\newcommand{\symdiff}{\mathop{\triangle}}
\newcommand{\pow}{\mathcal{P}}
\newcommand{\onto}{\twoheadrightarrow}
\newcommand{\into}{\hookrightarrow}

% E.g. `\set{1, 2, 3, 4}` prints the numbers 1, 2, 3 and 4 as a comma
% separated list which is enclosed within curly braces.
\def\set#1{\ensuremath{
  \{\foreach[count=\i]\x in {#1}{
    \ifnum\i>1,\,\fi
    \x
  }\}
}}

\begin{document}
  \begin{imports}
    \begin{forthel}
      %[prove off][check off]
      [readtex \path{libraries/source/foundations/01_classes.ftl.tex}]
      %[prove on][check on]
    \end{forthel}
  \end{imports}


  \section{Commutativity of Union and Intersection}

  \begin{forthel}
    \begin{proposition}\printlabel{FOUNDATIONS_02_8446177632583680}
      Let $A, B$ be classes.
      Then \[ A \cup B = B \cup A. \]
    \end{proposition}
    \begin{proof}
      Let us show that $A \cup B \subseteq B \cup A$.
        Let $x \in A \cup B$.
        Then $x \in A$ or $x \in B$.
        Hence $x \in B$ or $x \in A$.
        Thus $x \in B \cup A$.
      End.

      Let us show that $B \cup A \subseteq A \cup B$.
        Let $x \in B \cup A$.
        Then $x \in B$ or $x \in A$.
        Hence $x \in A$ or $x \in B$.
        Thus $x \in A \cup B$.
      End.
    \end{proof}
  \end{forthel}

  \begin{forthel}
    \begin{proposition}\printlabel{FOUNDATIONS_02_7565102251245568}
      Let $A, B$ be classes.
      Then \[ A \cap B = B \cap A. \]
    \end{proposition}
    \begin{proof}
      Let us show that $A \cap B \subseteq B \cap A$.
        Let $x \in A \cap B$.
        Then $x \in A$ and $x \in B$.
        Hence $x \in B$ and $x \in A$.
        Thus $x \in B \cap A$.
      End.

      Let us show that $B \cap A \subseteq A \cap B$.
        Let $x \in B \cap A$.
        Then $x \in B$ and $x \in A$.
        Hence $x \in A$ and $x \in B$.
        Thus $x \in A \cap B$.
      End.
    \end{proof}
  \end{forthel}


  \section{Associativity of Union and Intersection}

  \begin{forthel}
    \begin{proposition}\printlabel{FOUNDATIONS_02_3854032263184384}
      Let $A, B, C$ be classes.
      Then \[ (A \cup B) \cup C = A \cup (B \cup C). \]
    \end{proposition}
    \begin{proof}
      Let us show that $((A \cup B) \cup C) \subseteq A \cup (B \cup C)$. %!
        Let $x \in (A \cup B) \cup C$.
        Then $x \in A \cup B$ or $x \in C$.
        Hence $x \in A$ or $x \in B$ or $x \in C$.
        Thus $x \in A$ or $x \in (B \cup C)$.
        Therefore $x \in A \cup (B \cup C)$.
      End.

      Let us show that $A \cup (B \cup C) \subseteq (A \cup B) \cup C$.
        Let $x \in A \cup (B \cup C)$.
        Then $x \in A$ or $x \in B \cup C$.
        Hence $x \in A$ or $x \in B$ or $x \in C$.
        Thus $x \in A \cup B$ or $x \in C$.
        Therefore $x \in (A \cup B) \cup C$.
      End.
    \end{proof}
  \end{forthel}

  \begin{forthel}
    \begin{proposition}\printlabel{FOUNDATIONS_02_906751977193472}
      Let $A, B, C$ be classes.
      Then \[ (A \cap B) \cap C = A \cap (B \cap C). \]
    \end{proposition}
    \begin{proof}
      Let us show that $((A \cap B) \cap C) \subseteq A \cap (B \cap C)$. %!
        Let $x \in (A \cap B) \cap C$.
        Then $x \in A \cap B$ and $x \in C$.
        Hence $x \in A$ and $x \in B$ and $x \in C$.
        Thus $x \in A$ and $x \in (B \cap C)$.
        Therefore$ x \in A \cap (B \cap C)$.
      End.

      Let us show that $A \cap (B \cap C) \subseteq (A \cap B) \cap C$.
        Let $x \in A \cap (B \cap C)$.
        Then $x \in A$ and $x \in B \cap C$.
        Hence $x \in A$ and $x \in B$ and $x \in C$.
        Thus $x \in A \cap B$ and $x \in C$.
        Therefore $x \in (A \cap B) \cap C$.
      End.
    \end{proof}
  \end{forthel}


  \section{Distributivity of Union and Intersection}

  \begin{forthel}
    \begin{proposition}\printlabel{FOUNDATIONS_02_371139087958016}
      Let $A, B, C$ be classes.
      Then \[ A \cap (B \cup C) = (A \cap B) \cup (A \cap C). \]
    \end{proposition}
    \begin{proof}
      Let us show that $A \cap (B \cup C) \subseteq (A \cap B) \cup (A \cap C)$.
        Let $x \in A \cap (B \cup C)$.
        Then $x \in A$ and $x \in B \cup C$.
        Hence $x \in A$ and ($x \in B$ or $x \in C$).
        Thus ($x \in A$ and $x \in B$) or ($x \in A$ and $x \in C$).
        Therefore $x \in A \cap B$ or $x \in A \cap C$.
        Hence $x \in (A \cap B) \cup (A \cap C)$.
      End.

      Let us show that $((A \cap B) \cup (A \cap C)) \subseteq A \cap (B \cup C)$. %!
        Let $x \in (A \cap B) \cup (A \cap C)$.
        Then $x \in A \cap B$ or $x \in A \cap C$.
        Hence ($x \in A$ and $x \in B$) or ($x \in A$ and $x \in C$).
        Thus $x \in A$ and ($x \in B$ or $x \in C$).
        Therefore $x \in A$ and $x \in B \cup C$.
        Hence$ x \in A \cap (B \cup C)$.
      End.
    \end{proof}
  \end{forthel}

  \begin{forthel}
    \begin{proposition}\printlabel{FOUNDATIONS_02_5937390721957888}
      Let $A, B, C$ be classes.
      Then \[ A \cup (B \cap C) = (A \cup B) \cap (A \cup C). \]
    \end{proposition}
    \begin{proof}
      Let us show that $A \cup (B \cap C) \subseteq (A \cup B) \cap (A \cup C)$.
        Let $x \in A \cup (B \cap C)$.
        Then $x \in A$ or $x \in B \cap C$.
        Hence $x \in A$ or ($x \in B$ and $x \in C$).
        Thus ($x \in A$ or $x \in B$) and ($x \in$ A or $x \in C$).
        Therefore $x \in A \cup B$ and $x \in A \cup C$.
        Hence $x \in (A \cup B) \cap (A \cup C)$.
      End.

      Let us show that $((A \cup B) \cap (A \cup C)) \subseteq A \cup (B \cap C)$. %!
        Let $x \in (A \cup B) \cap (A \cup C)$.
        Then $x \in A \cup B$ and $x \in A \cup C$.
        Hence ($x \in A$ or $x \in B$) and ($x \in A$ or $x \in C$).
        Thus $x \in A$ or ($x \in B$ and $x \in C$).
        Therefore $x \in A$ or $x \in B \cap C$.
        Hence $x \in A \cup (B \cap C)$.
      End.
    \end{proof}
  \end{forthel}


  \section{Idempocy Laws for Union and Intersection}

  \begin{forthel}
    \begin{proposition}\printlabel{FOUNDATIONS_02_2096996737351680}
      Let $A$ be a class.
      Then \[ A \cup A = A. \]
    \end{proposition}
    \begin{proof}
      $A \cup A = \{x \mid x \in A$ or $x \in A \}$.
      Hence $A \cup A = \{ x \mid x \in A \}$.
      Thus $A \cup A = A$.
    \end{proof}
  \end{forthel}

  \begin{forthel}
    \begin{proposition}\printlabel{FOUNDATIONS_02_4053144145231872}
      Let $A$ be a class.
      Then \[ A \cap A = A. \]
    \end{proposition}
    \begin{proof}
      $A \cap A = \{ x \mid x \in A$ and $x \in A \}$.
      Hence $A \cap A = \{ x \mid x \in A \}$.
      Thus $A \cap A = A$.
    \end{proof}
  \end{forthel}


  \section{Distributivity of Complement}

  \begin{forthel}
    \begin{proposition}\printlabel{FOUNDATIONS_02_5296031436636160}
      Let $A, B, C$ be classes.
      Then \[ A \setminus (B \cap C) = (A \setminus B) \cup (A \setminus C). \]
    \end{proposition}
    \begin{proof}
      Let us show that $A \setminus (B \cap C) \subseteq
      (A \setminus B) \cup (A \setminus C)$.
        Let $x \in A \setminus (B \cap C)$.
        Then $x \in A$ and $x \notin B \cap C$.
        Hence it is wrong that ($x \in B$ and $x \in C$).
        Thus $x \notin B$ or $x \notin C$.
        Therefore $x \in A$ and ($x \notin B$ or $x \notin C$).
        Then ($x \in A$ and $x \notin B$) or ($x \in A$ and $x \notin C$).
        Hence $x \in A \setminus B$ or $x \in A \setminus C$.
        Thus $x \in (A \setminus B) \cup (A \setminus C)$.
      End.

      Let us show that $((A \setminus B) \cup (A \setminus C)) \subseteq
      A \setminus (B \cap C)$. %!
        Let $x \in (A \setminus B) \cup (A \setminus C)$.
        Then $x \in A \setminus B$ or $x \in A \setminus C$.
        Hence ($x \in A$ and $x \notin B$) or ($x \in A$ and $x \notin C$).
        Thus $x \in A$ and ($x \notin B$ or $x \notin C$).
        Therefore $x \in A$ and not ($x \in B$ and $x \in C$).
        Then $x \in A$ and not $x \in B \cap C$.
        Hence $x \in A \setminus (B \cap C)$.
      End.
    \end{proof}
  \end{forthel}

  \begin{forthel}
    \begin{proposition}\printlabel{FOUNDATIONS_02_2909554153095168}
      Let $A, B, C$ be classes.
      Then \[ A \setminus (B \cup C) = (A \setminus B) \cap (A \setminus C). \]
    \end{proposition}
    \begin{proof}
      Let us show that $A \setminus (B \cup C) \subseteq
      (A \setminus B) \cap (A \setminus C)$.
        Let $x \in A \setminus (B \cup C)$.
        Then $x \in A$ and $x \notin B \cup C$.
        Hence it is wrong that ($x \in B$ or $x \in C$).
        Thus $x \notin B$ and $x \notin C$.
        Therefore $x \in A$ and ($x \notin B$ and $x \notin C$).
        Then ($x \in A$ and $x \notin B$) and ($x \in A$ and $x \notin C$).
        Hence $x \in A \setminus B$ and $x \in A \setminus C$.
        Thus $x \in (A \setminus B) \cap (A \setminus C)$.
      End.

      Let us show that $((A \setminus B) \cap (A \setminus C)) \subseteq
      A \setminus (B \cup C)$. %!
        Let $x \in (A \setminus B) \cap (A \setminus C)$.
        Then $x \in A \setminus B$ and $x \in A \setminus C$.
        Hence ($x \in A$ and $x \notin B$) and ($x \in A$ and $x \notin C$).
        Thus $x \in A$ and ($x \notin B$ and $x \notin C$).
        Therefore $x \in A$ and not ($x \in B$ or $x \in C$).
        Then $x \in A$ and not $x \in B \cup C$.
        Hence $x \in A \setminus (B \cup C)$.
      End.
    \end{proof}
  \end{forthel}


  \section{Subclass Laws}

  \begin{forthel}
    \begin{proposition}\printlabel{FOUNDATIONS_02_3793981508943872}
      Let $A, B$ be classes.
      Then \[ A \subseteq A \cup B. \]
    \end{proposition}
    \begin{proof}
      Let $x \in A$.
      Then $x \in A$ or $x \in B$.
      Hence $x \in A \cup B$.
    \end{proof}
  \end{forthel}

  \begin{forthel}
    \begin{proposition}\printlabel{FOUNDATIONS_02_1591517646946304}
      Let $A, B$ be classes.
      Then \[ A \cap B \subseteq A. \]
    \end{proposition}
    \begin{proof}
      Let $x \in A \cap B$.
      Then $x \in A$ and $x \in B$.
      Hence $x \in A$.
    \end{proof}
  \end{forthel}

  \begin{forthel}
    \begin{proposition}\printlabel{FOUNDATIONS_02_6657236858306560}
      Let $A, B$ be classes.
      Then \[ A \subseteq B \iff A \cup B = B. \]
    \end{proposition}
    \begin{proof}
      Case $A \subseteq B$.

        Let us show that $A \cup B \subseteq B$.
          Let $x \in A \cup B$.
          Then $x \in A$ or $x \in B$.
          If $x \in A$ then $x \in B$.
          Hence $x \in B$.
        End.

        Let us show that $B \subseteq A \cup B$.
          Let $x \in B$.
          Then $x \in A$ or $x \in B$.
          Hence $x \in A \cup B$.
        End.
      End.

      Case $A \cup B = B$.
        Let $x \in A$.
        Then $x \in A$ or $x \in B$.
        Hence $x \in A \cup B = B$.
      End.
    \end{proof}
  \end{forthel}

  \begin{forthel}
    \begin{proposition}\printlabel{FOUNDATIONS_02_2356449346846720}
      Let $A, B$ be classes.
      Then \[ A \subseteq B \iff A \cap B = A. \]
    \end{proposition}
    \begin{proof}
      Case $A \subseteq B$.

        Let us show that $A \cap B \subseteq A$.
          Let $x \in A \cap B$.
          Then $x \in A$ and $x \in B$.
          Hence $x \in A$.
        End.

        Let us show that $A \subseteq A \cap B$.
          Let $x \in A$.
          Then $x \in B$.
          Hence $x \in A$ and $x \in B$.
          Thus $x \in A \cap B$.
        End.
      End.

      Case $A \cap B = A$.
        Let $x \in A$.
        Then $x \in A \cap B$.
        Hence $x \in A$ and $x \in B$.
        Thus $x \in B$.
      End.
    \end{proof}
  \end{forthel}


  \section{Complement Laws}

  \begin{forthel}
    \begin{proposition}\printlabel{FOUNDATIONS_02_7433299337150464}
      Let $A$ be a class.
      Then \[ A \setminus A = \emptyset. \]
    \end{proposition}
    \begin{proof}
      $A \setminus A$ has no elements.
      Indeed $A \setminus A = \{ x \mid x \in A$ and $x \notin A \}$.
      Hence the thesis.
    \end{proof}
  \end{forthel}

  \begin{forthel}
    \begin{proposition}\printlabel{FOUNDATIONS_02_3783696985358336}
      Let $A$ be a class.
      Then \[ A \setminus \emptyset = A. \]
    \end{proposition}
    \begin{proof}
      $A \setminus \emptyset = \{ x \mid x \in A$ and $x \notin \emptyset \}$.
      No element is an element of $\emptyset$.
      Hence $A \setminus \emptyset = \{ x \mid x \in A \}$.
      Then we have the thesis.
    \end{proof}
  \end{forthel}

  \begin{forthel}
    \begin{proposition}\printlabel{FOUNDATIONS_02_7083929257377792}
      Let $A, B$ be classes.
      Then \[ A \setminus (A \setminus B) = A \cap B. \]
    \end{proposition}
    \begin{proof}
      Let us show that $A \setminus (A \setminus B) \subseteq A \cap B$.
        Let $x \in A \setminus (A \setminus B)$.
        Then $x \in A$ and $x \notin A \setminus B$.
        Hence $x \notin A$ or $x \in B$.
        Thus $x \in B$.
        Therefore $x \in A \cap B$.
      End.

      Let us show that $A \cap B \subseteq A \setminus (A \setminus B)$.
        Let $x \in A \cap B$.
        Then $x \in A$ and $x \in B$.
        Hence $x \notin A$ or $x \in B$.
        Thus $x \notin A \setminus B$.
        Therefore $x \in A \setminus (A \setminus B)$.
      End.
    \end{proof}
  \end{forthel}

  \begin{forthel}
    \begin{proposition}\printlabel{FOUNDATIONS_02_4938646769631232}
      Let $A, B$ be classes.
      Then \[ B \subseteq A \iff A \setminus (A \setminus B) = B. \]
    \end{proposition}
    \begin{proof}
      Case $B \subseteq A$. Obvious.

      Case $A \setminus (A \setminus B) = B$.
        Then every element of $B$ is an element of $A \setminus
        (A \setminus B)$.
        Thus every element of $B$ is an element of $A$.
        Then we have the thesis.
      End.
    \end{proof}
  \end{forthel}

  \begin{forthel}
    \begin{proposition}\printlabel{FOUNDATIONS_02_5811954316738560}
      Let $A, B, C$ be classes.
      Then \[ A \cap (B \setminus C) = (A \cap B) \setminus (A \cap C). \]
    \end{proposition}
    \begin{proof}
      Let us show that $A \cap (B \setminus C) \subseteq
      (A \cap B) \setminus (A \cap C)$.
        Let $x \in A \cap (B \setminus C)$.
        Then $x \in A$ and $x \in B \setminus C$.
        Hence $x \in A$ and $x \in B$.
        Thus $x \in A \cap B$ and $x \notin C$.
        Therefore $x \notin A \cap C$.
        Then we have $x \in (A \cap B) \setminus (A \cap C)$.
      End.

      Let us show that $((A \cap B) \setminus (A \cap C)) \subseteq
      A \cap (B \setminus C)$. %!
        Let $x \in (A \cap B) \setminus (A \cap C)$.
        Then $x \in A$ and $x \in B$.
        $x \notin A \cap C$.
        Hence $x \notin C$.
        Thus $x \in B \setminus C$.
        Therefore $x \in A \cap (B \setminus C)$.
      End.
    \end{proof}
  \end{forthel}
\end{document}
