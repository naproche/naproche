\documentclass{naproche-library}
\libinput{foundations-preamble}
\begin{document}
\begin{smodule}{closure-under-finite-intersections.ftl}
  \importmodule[libraries]{foundations?systems-of-sets.ftl}
  \importmodule[libraries]{foundations?finite-and-infinite-classes.ftl}

  \section*{Closure Under Finite Intersections}

  \begin{definition}[forthel,id=FOUNDATIONS_14_4297814324543488]
    Let $X$ be a system of sets.
    $X$ is closed under finite intersections iff $\bigcap U \in X$ for every nonempty finite subclass $U$ of $X$.
  \end{definition}

  \begin{proposition}[forthel,id=FOUNDATIONS_17_1405012582334464]
    Let $X$ be a system of sets.
    $X$ is closed under finite intersections iff $U \cap V \in X$ for every $U, V \in X$.
  \end{proposition}
  \begin{proof}[forthel]
    Case $X$ is closed under finite intersections.
      Let $U, V \in X$.
      Then $\set{U, V}$ is a nonempty finite subclass of $X$.
      Hence $U \cap V = \bigcap \set{U, V} \in X$.
    End.

    Case $U \cap V \in X$ for every $U, V \in X$.
      Define $\Phi = \{ n \in \Nat \mid \bigcap U \in X$ for every nonempty subclass $U$ of $X$ that has $n$ elements $\}$.

      (1) $\Phi$ contains $0$.

      (2) For every $n \in \Phi$ we have $n + 1 \in \Phi$. \newline
      Proof.
        Let $n \in \Phi$.
        Then $\bigcap U \in X$ for every nonempty subclass $U$ of $X$ that has $n$ elements.

        Let us show that $\bigcap U \in X$ for every nonempty subclass $U$ of $X$ that has $n + 1$ elements.

          Case $n = 0$. Obvious.

          Case $n \neq 0$.
            Let $U$ be a nonempty subclass of $X$ such that $U$ has $n + 1$ elements.
            Take a bijection $f$ between $\{1, \dots, n + 1 \}$ and $U$.
            We have $\{ 1, \dots, n + 1 \} = \{ 1, \dots, n \} \cup \set{n + 1}$.
            Take $V = f[\{ 1, \dots, n \}]$.
            We have $\{ 1, \dots, n \} \subseteq \{ 1, \dots, n + 1 \}$.

            Let us show that $V \subseteq U$.
              Let $x \in V$.
              Take $k \in \{ 1, \dots, n \}$ such that $x = f(k)$.
              Indeed we can show that there exists a $k \in \{ 1, \dots, n \}$ such that $x = f(k)$.
                Assume the contrary.
                Then $x \neq f(k)$ for all $k \in \{ 1, \dots, n \}$.
                Hence $x \notin f[\{ 1, \dots, n \}] = V$.
                Contradiction.
              End.
              Hence $x \in U$.
              Indeed $x \in f[\{ 1, \dots, n + 1 \}]$.
            End.

            $V$ is nonempty.
            Indeed $f(1) \in f[\{ 1, \dots, n \}]$.
            Indeed $1 \in \{ 1, \dots, n \}$.
            Hence $f \restriction \{ 1, \dots, n \}$ is a bijection between $\{ 1, \dots, n \}$ and $V$ (by \cref{FOUNDATIONS_08_647446231252992}).
            Thus $V$ has $n$ elements.
            Consequently $\bigcap V \in X$.

            Let us show that $U = V \cup \set{f(n + 1)}$. \newline
              (1) $f[A \cup B] = f[A] \cup f[B]$ for all $A, B \subseteq \dom(f)$.

              (2) $f[\set{a}] = \set{f(a)}$ for all $a \in \dom(f)$.

              Hence $U
                = f[\dom(f)]
                = f[\{ 1, \dots, n + 1 \}]
                = f[\{ 1, \dots, n \} \cup \set{n + 1}]
                = f[\{ 1, \dots, n \}] \cup f[\set{n + 1}]
                = f[\{ 1, \dots, n \}] \cup \set{f(n + 1)}
                = V \cup \set{f(n + 1)}$.
              Indeed $n + 1 \in \dom(f)$ and $\{ 1, \dots, n \}, \set{1 + n} \subseteq \dom(f)$.
            End.

            Let us show that $\bigcap (A \cup B) = (\bigcap A) \cap (\bigcap B)$ for any nonempty systems of sets $A, B$.
              Let $A, B$ be nonempty systems of sets.
              $\bigcap (A \cup B) \subseteq (\bigcap A) \cap (\bigcap B)$.
              $((\bigcap A) \cap (\bigcap B)) \subseteq \bigcap (A \cup B)$. %!
            End.

            Let us show that $f(n + 1)$ and $f(k)$ are sets for every $k \in \{ 1, \dots, n \}$.
              Let $k \in \{ 1, \dots, n \}$.
              Then $f(k), f(n + 1) \in U$.
              Hence $f(k)$ and $f(n + 1)$ are sets.
            End.

            Hence $V$ and $\set{f(n + 1)}$ are nonempty systems of sets.
            Thus $\bigcap U
              = \bigcap (V \cup \set{f(n + 1)})
              = (\bigcap V) \cap (\bigcap \set{f(n + 1)})
              = (\bigcap V) \cap f(n + 1)
              \in X$.
          End.
        End.
      Qed.

      Therefore $\Phi$ contains every natural number (by \printref{ARITHMETIC_01_4764664342773760}).
      Thus $\bigcap U \in X$ for every nonempty finite subclass $U$ of $X$.
      Consequently $X$ is closed under finite intersections.
    End.
  \end{proof}
\end{smodule}
\end{document}
