\documentclass[10pt]{article}
\usepackage[english]{babel}
\usepackage{../../../meta-inf/lib/naproche}
% Top-level sections and proofs

\newtheoremstyle{foundations}% name of the style to be used
  {\dimexpr\ftlparskip+4pt\relax}% measure of space to leave above the theorem.
  % (We must add \ftlparskip since otherwise the name of the label would not fit
  % into the border drawn around the environment.)
  {4pt}% measure of space to leave below the theorem.
  {\normalfont}% name of font to use in the body of the theorem
  {0pt}% measure of space to indent
  {\bfseries}% name of head font
  {}% punctuation between head and body
  { }% space after theorem head; " " = normal inter-word space
  {\thmname{#1}\thmnumber{ #2}\thmnote{ (#3)}.}

\let\definition\relax
\let\enddefinition\relax
\let\signature\relax
\let\endsignature\relax
\let\axiom\relax
\let\endaxiom\relax
\let\theorem\relax
\let\endtheorem\relax
\let\proposition\relax
\let\endproposition\relax
\let\lemma\relax
\let\endlemma\relax
\let\corollary\relax
\let\endcorollary\relax

\theoremstyle{foundations}

\newmdtheoremenv[
  backgroundcolor=lightgray!10,
  linecolor=black,
  innertopmargin=1em
]{definition}{Definition}[section]
\newmdtheoremenv[
  backgroundcolor=lightgray!10,
  linecolor=black,
  innertopmargin=1em
]{signature}[definition]{Signature}
\newmdtheoremenv[
  backgroundcolor=lightgray!10,
  linecolor=black,
  innertopmargin=1em
]{axiom}[definition]{Axiom}
\newmdtheoremenv[
  backgroundcolor=lightgray!10,
  linecolor=black,
  innertopmargin=1em
]{theorem}[definition]{Theorem}
\newmdtheoremenv[
  backgroundcolor=lightgray!10,
  linecolor=black,
  innertopmargin=1em
]{proposition}[definition]{Proposition}
\newmdtheoremenv[
  backgroundcolor=lightgray!10,
  linecolor=black,
  innertopmargin=1em
]{lemma}[definition]{Lemma}
\newmdtheoremenv[
  backgroundcolor=lightgray!10,
  linecolor=black,
  innertopmargin=1em
]{corollary}[definition]{Corollary}


% Hide imports

\usepackage{scontents}
\newenvsc{imports}[store-env=imports,print-env=false]


% Labels of top-level sections

\renewcommand{\showlabelfont}{\scriptsize\ttfamily\color{gray}}


% Non-symbolic "\implies" and "\iff"

\renewcommand{\implies}{\quad \text{implies} \quad}
\renewcommand{\iff}{\quad \text{iff} \quad}

\usepackage{amssymb}
\usepackage{pgffor}

\newcommand{\id}{\textrm{id}}
\newcommand{\range}{\textrm{range}}
\newcommand{\const}[2]{\textrm{const}_{#1}^{#2}}
\newcommand{\symdiff}{\mathop{\triangle}}
\newcommand{\pow}{\mathcal{P}}
\newcommand{\onto}{\twoheadrightarrow}
\newcommand{\into}{\hookrightarrow}

% E.g. `\set{1, 2, 3, 4}` prints the numbers 1, 2, 3 and 4 as a comma
% separated list which is enclosed within curly braces.
\def\set#1{\ensuremath{
  \{\foreach[count=\i]\x in {#1}{
    \ifnum\i>1,\,\fi
    \x
  }\}
}}

\useackage{amssymb}

\newcommand{\Nat}{\mathbb{N}}
\newcommand{\Prime}{\mathbb{P}}
\renewcommand{\succ}{\textrm{succ}}
\newcommand{\pred}{\textrm{pred}}
\newcommand{\add}{\textrm{add}}
\newcommand{\mul}{\textrm{mul}}
\renewcommand{\exp}{\textrm{exp}}
\newcommand{\fac}{\textrm{fac}}
\renewcommand{\div}{\mathrel{\textrm{div}}}
\renewcommand{\mod}{\mathrel{\textrm{mod}}}

\begin{document}
  \begin{imports}
    \begin{forthel}
      %[prove off][check off]
      [readtex \path{libraries/source/foundations/systems-of-sets.ftl.tex}]
      [readtex \path{libraries/source/foundations/finite-and-infinite-classes.ftl.tex}]
      %[prove on][check on]
    \end{forthel}
  \end{imports}


  \section*{Closure Under Finite Intersections}

  \begin{forthel}
    \begin{definition}\printlabel{FOUNDATIONS_14_4297814324543488}
      Let $X$ be a system of sets.
      $X$ is closed under finite intersections iff $\bigcap U \in X$ for every nonempty finite subclass $U$ of $X$.
    \end{definition}
  \end{forthel}

  \begin{forthel}
    \begin{proposition}\printlabel{FOUNDATIONS_17_1405012582334464}
      Let $X$ be a system of sets.
      $X$ is closed under finite intersections iff $U \cap V \in X$ for every $U, V \in X$.
    \end{proposition}
    \begin{proof}
      Case $X$ is closed under finite intersections.
        Let $U, V \in X$.
        Then $\set{U, V}$ is a nonempty finite subclass of $X$.
        Hence $U \cap V = \bigcap \set{U, V} \in X$.
      End.
  
      Case $U \cap V \in X$ for every $U, V \in X$.
        Define $\Phi = \{ n \in \Nat \mid \bigcap U \in X$ for every nonempty subclass $U$ of $X$ that has $n$ elements $\}$.
  
        (1) $\Phi$ contains $0$.
  
        (2) For every $n \in \Phi$ we have $n + 1 \in \Phi$. \\
        Proof.
          Let $n \in \Phi$.
          Then $\bigcap U \in X$ for every nonempty subclass $U$ of $X$ that has $n$ elements.
  
          Let us show that $\bigcap U \in X$ for every nonempty subclass $U$ of $X$ that has $n + 1$ elements.
  
            Case $n = 0$. Obvious.
  
            Case $n \neq 0$.
              Let $U$ be a nonempty subclass of $X$ such that $U$ has $n + 1$ elements.
              Take a bijection $f$ between $\{1, \dots, n + 1 \}$ and $U$.
              We have $\{ 1, \dots, n + 1 \} = \{ 1, \dots, n \} \cup \set{n + 1}$.
              Take $V = f[\{ 1, \dots, n \}]$.
              We have $\{ 1, \dots, n \} \subseteq \{ 1, \dots, n + 1 \}$.
  
              Let us show that $V \subseteq U$.
                Let $x \in V$.
                Take $k \in \{ 1, \dots, n \}$ such that $x = f(k)$.
                Indeed we can show that there exists a $k \in \{ 1, \dots, n \}$ such that $x = f(k)$.
                  Assume the contrary.
                  Then $x \neq f(k)$ for all $k \in \{ 1, \dots, n \}$.
                  Hence $x \notin f[\{ 1, \dots, n \}] = V$.
                  Contradiction.
                End.
                Hence $x \in U$.
                Indeed $x \in f[\{ 1, \dots, n + 1 \}]$.
              End.

              $V$ is nonempty.
              Indeed $f(1) \in f[\{ 1, \dots, n \}]$.
              Indeed $1 \in \{ 1, \dots, n \}$.
              Hence $f \restriction \{ 1, \dots, n \}$ is a bijection between $\{ 1, \dots, n \}$ and $V$ (by \cref{FOUNDATIONS_08_647446231252992}).
              Thus $V$ has $n$ elements.
              Consequently $\bigcap V \in X$.

              Let us show that $U = V \cup \set{f(n + 1)}$. \\
                (1) $f[A \cup B] = f[A] \cup f[B]$ for all $A, B \subseteq \dom(f)$.

                (2) $f[\set{a}] = \set{f(a)}$ for all $a \in \dom(f)$.

                Hence $U
                  = f[\dom(f)]
                  = f[\{ 1, \dots, n + 1 \}]
                  = f[\{ 1, \dots, n \} \cup \set{n + 1}]
                  = f[\{ 1, \dots, n \}] \cup f[\set{n + 1}]
                  = f[\{ 1, \dots, n \}] \cup \set{f(n + 1)}
                  = V \cup \set{f(n + 1)}$.
                Indeed $n + 1 \in \dom(f)$ and $\{ 1, \dots, n \}, \set{1 + n} \subseteq \dom(f)$.
              End.
  
              Let us show that $\bigcap (A \cup B) = (\bigcap A) \cap (\bigcap B)$ for any nonempty systems of sets $A, B$.
                Let $A, B$ be nonempty systems of sets.
                $\bigcap (A \cup B) \subseteq (\bigcap A) \cap (\bigcap B)$.
                $((\bigcap A) \cap (\bigcap B)) \subseteq \bigcap (A \cup B)$. %!
              End.
  
              $f(n + 1)$ and $f(k)$ are sets for every $k \in \{ 1, \dots, n \}$.
              Indeed $f(k) \in U$ for every $k \in \{ 1, \dots, n \}$.
              Hence $V$ and $\set{f(n + 1)}$ are nonempty systems of sets.
              Thus $\bigcap U
                = \bigcap (V \cup \set{f(n + 1)})
                = (\bigcap V) \cap (\bigcap \set{f(n + 1)})
                = (\bigcap V) \cap f(n + 1)
                \in X$.
            End.
          End.
        Qed.

        Therefore $\Phi$ contains every natural number (by \printref{ARITHMETIC_01_4764664342773760}).
        Thus $\bigcap U \in X$ for every nonempty finite subclass $U$ of $X$.
        Consequently $X$ is closed under finite intersections.
      End.
    \end{proof}
  \end{forthel}
\end{document}
