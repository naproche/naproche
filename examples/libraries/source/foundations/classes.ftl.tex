\documentclass[10pt]{article}
\usepackage[english]{babel}
\usepackage{../../../../lib/tex/naproche}
% Top-level sections and proofs

\newtheoremstyle{foundations}% name of the style to be used
  {\dimexpr\ftlparskip+4pt\relax}% measure of space to leave above the theorem.
  % (We must add \ftlparskip since otherwise the name of the label would not fit
  % into the border drawn around the environment.)
  {4pt}% measure of space to leave below the theorem.
  {\normalfont}% name of font to use in the body of the theorem
  {0pt}% measure of space to indent
  {\bfseries}% name of head font
  {}% punctuation between head and body
  { }% space after theorem head; " " = normal inter-word space
  {\thmname{#1}\thmnumber{ #2}\thmnote{ (#3)}.}

\let\definition\relax
\let\enddefinition\relax
\let\signature\relax
\let\endsignature\relax
\let\axiom\relax
\let\endaxiom\relax
\let\theorem\relax
\let\endtheorem\relax
\let\proposition\relax
\let\endproposition\relax
\let\lemma\relax
\let\endlemma\relax
\let\corollary\relax
\let\endcorollary\relax

\theoremstyle{foundations}

\newmdtheoremenv[
  backgroundcolor=lightgray!10,
  linecolor=black,
  innertopmargin=1em
]{definition}{Definition}[section]
\newmdtheoremenv[
  backgroundcolor=lightgray!10,
  linecolor=black,
  innertopmargin=1em
]{signature}[definition]{Signature}
\newmdtheoremenv[
  backgroundcolor=lightgray!10,
  linecolor=black,
  innertopmargin=1em
]{axiom}[definition]{Axiom}
\newmdtheoremenv[
  backgroundcolor=lightgray!10,
  linecolor=black,
  innertopmargin=1em
]{theorem}[definition]{Theorem}
\newmdtheoremenv[
  backgroundcolor=lightgray!10,
  linecolor=black,
  innertopmargin=1em
]{proposition}[definition]{Proposition}
\newmdtheoremenv[
  backgroundcolor=lightgray!10,
  linecolor=black,
  innertopmargin=1em
]{lemma}[definition]{Lemma}
\newmdtheoremenv[
  backgroundcolor=lightgray!10,
  linecolor=black,
  innertopmargin=1em
]{corollary}[definition]{Corollary}


% Hide imports

\usepackage{scontents}
\newenvsc{imports}[store-env=imports,print-env=false]


% Labels of top-level sections

\renewcommand{\showlabelfont}{\scriptsize\ttfamily\color{gray}}


% Non-symbolic "\implies" and "\iff"

\renewcommand{\implies}{\quad \text{implies} \quad}
\renewcommand{\iff}{\quad \text{iff} \quad}

\usepackage{amssymb}
\usepackage{pgffor}

\newcommand{\id}{\textrm{id}}
\newcommand{\range}{\textrm{range}}
\newcommand{\const}[2]{\textrm{const}_{#1}^{#2}}
\newcommand{\symdiff}{\mathop{\triangle}}
\newcommand{\pow}{\mathcal{P}}
\newcommand{\onto}{\twoheadrightarrow}
\newcommand{\into}{\hookrightarrow}

% E.g. `\set{1, 2, 3, 4}` prints the numbers 1, 2, 3 and 4 as a comma
% separated list which is enclosed within curly braces.
\def\set#1{\ensuremath{
  \{\foreach[count=\i]\x in {#1}{
    \ifnum\i>1,\,\fi
    \x
  }\}
}}

\begin{document}
  \begin{imports}
    \begin{forthel}
      [readtex \path{meta-inf/source/vocabulary.ftl.tex}]
      [readtex \path{meta-inf/source/macros.ftl.tex}]
    \end{forthel}
  \end{imports}


  \section*{Classes}

  \subsection*{Sub- and Superclasses}

  \begin{forthel}
    \begin{definition}\printlabel{FOUNDATIONS_01_3275578358628352}
      Let $A$ be a class.
      A subclass of $A$ is a class $B$ such that every element of $B$ is an
      element of $A$.
    \end{definition}

    Let $B \subseteq A$ stand for $B$ is a subclass of $A$.
    Let $B \subset A$ stand for $B \subseteq A$.

    Let a superclass of $B$ stand for a class $A$ such that $B \subseteq A$.
    Let $B \supseteq A$ stand for $B$ is a superclass of $A$.
    Let $B \supset A$ stand for $B \subseteq A$.

    Let a proper subclass of $A$ stand for a subclass $B$ of $A$ such that $B \neq A$.
    Let $B \subsetneq A$ stand for $B$ is a proper subclass of $A$.

    Let a proper superclass of $B$ stand for a superclass $A$ of $B$ such that $A \neq B$.
    Let $B \supsetneq A$ stand for $B$ is a proper superclass of $A$.

    Let $A$ includes $B$ stand for $B \subseteq A$.
    Let $B$ is included in $A$ stand for $B \subseteq A$.
  \end{forthel}

  \begin{forthel}
    \begin{proposition}\printlabel{FOUNDATIONS_01_5994555614691328}
      Let $A$ be a class.
      Then $A \subseteq A$.
    \end{proposition}
    \begin{proof}
      Every element of $A$ is contained in $A$.
      Therefore $A \subseteq A$.
    \end{proof}
  \end{forthel}

  \begin{forthel}
    \begin{proposition}\printlabel{FOUNDATIONS_01_3939677545431040}
      Let $A, B, C$ be classes.
      If $A \subseteq B$ and $B \subseteq C$ then $A \subseteq C$.
    \end{proposition}
    \begin{proof}
      Assume $A \subseteq B$ and $B \subseteq C$.
      Then every element of $A$ is contained in $B$ and every element of $B$ is contained in $C$.
      Hence every element of $A$ is contained in $C$.
      Thus $A \subseteq C$.
    \end{proof}
  \end{forthel}

  \begin{forthel}
    \begin{proposition}\printlabel{FOUNDATIONS_01_7159957847801856}
      Let $A, B$ be classes.
      If $A \subseteq B$ and $B \subseteq A$ then $A = B$.
    \end{proposition}
    \begin{proof}
      Assume $A \subseteq B$ and $B \subseteq A$.
      Then every element of $A$ is contained in $B$ and every element of $B$ is contained in $A$.
      Hence $A = B$.
    \end{proof}
  \end{forthel}


  \subsection*{The Empty Class}

  \begin{forthel}
    \begin{definition}\printlabel{FOUNDATIONS_01_6252477624090624}
      Let $A$ be a class.
      $A$ is empty iff $A$ has no elements.
    \end{definition}

    Let $A$ is nonempty stand for $A$ is not empty.
  \end{forthel}

  \begin{forthel}
    \begin{definition}\printlabel{FOUNDATIONS_01_7939928493129728}
      $\emptyset = \{ x \mid x \neq x \}$.
    \end{definition}
  \end{forthel}

  \begin{forthel}
    \begin{proposition}\printlabel{FOUNDATIONS_01_2263153161273344}
      Let $A$ be a class.
      $A$ is empty iff $A = \emptyset$.
    \end{proposition}
    \begin{proof}
      We can show that $\emptyset$ is empty.
      Indeed any element $x$ of $\emptyset$ is nonequal to $x$.
      Hence if $A = \emptyset$ then $A$ is empty.
      If $A$ is empty then $A$ and $\emptyset$ have no elements.
      Hence if $A$ is empty then $A \subseteq \emptyset$ and $\emptyset \subseteq A$.
      Thus if $A$ is empty then $A = \emptyset$.
    \end{proof}
  \end{forthel}

  \begin{forthel}
    \begin{corollary}\printlabel{FOUNDATIONS_01_1495141426659328}
      $\emptyset$ is empty.
    \end{corollary}
  \end{forthel}

  \begin{forthel}
    \begin{corollary}\printlabel{FOUNDATIONS_01_6931785090859008}
      Let $A$ be a class.
      Then $\emptyset \subseteq A$.
    \end{corollary}
    \begin{proof}
      $\emptyset$ has no elements.
      Hence every element of $\emptyset$ is contained in $A$.
    \end{proof}
  \end{forthel}


  \subsection*{Unordered Pairs}

  \begin{forthel}
    \begin{definition}\printlabel{FOUNDATIONS_01_3471035364016128}
      Let $a, b$ be objects.
      The unordered pair of $a$ and $b$ is $\{ x \mid x = a$ or $x = b \}$.
    \end{definition}

    Let $\set{a, b}$ stand for the unordered pair of $a$ and $b$.
  \end{forthel}

  \begin{forthel}
    \begin{definition}\printlabel{FOUNDATIONS_01_605432672419840}
      An unordered pair is a class $A$ such that $A = \set{a, b}$ for some distinct objects $a, b$.
    \end{definition}
  \end{forthel}

  \begin{forthel}
    \begin{definition}\printlabel{FOUNDATIONS_01_1160414603771904}
      Let $a$ be an object.
      The singleton class of $a$ is $\{ x \mid x = a \}$.
    \end{definition}

    Let $\set{a}$ stand for the singleton class of $a$.
  \end{forthel}

  \begin{forthel}
    \begin{definition}\printlabel{FOUNDATIONS_01_6786618161627136}
      A singleton class is a class $A$ such that $A = \set{a}$ for some object $a$.
    \end{definition}
  \end{forthel}

  \begin{forthel}
    \begin{proposition}\printlabel{FOUNDATIONS_01_6125259604361216}
      Let $a, a', b, b'$ be objects.
      Assume $\set{a, b} = \set{a', b'}$.
      Then ($a = a'$ and $b = b'$) or ($a = b'$ and $b = a'$).
    \end{proposition}
    \begin{proof}
      We have $a = a'$ or $a = b'$.
      If $a = a'$ then $b = b'$.
      If $a = b'$ then $b = a'$.
      Hence ($a = a'$ and $b = b'$) or ($a = b'$ and $b = a'$).
    \end{proof}
  \end{forthel}

  \begin{forthel}
    \begin{corollary}\printlabel{FOUNDATIONS_01_6954678910713856}
      Let $a, a'$ be objects.
      If $\set{a} = \set{a'}$ then $a = a'$.
    \end{corollary}
  \end{forthel}

  \begin{forthel}
    \begin{definition}
      Let $A$ be a class.
      A unique element of $A$ is an element $a$ of $A$ such that for each $x \in A$ we have $x = a$.
    \end{definition}
  \end{forthel}

  \begin{forthel}
    \begin{proposition}
      Let $A$ be a class.
      Then $A$ has a unique element iff $A = \set{a}$ for some object $a$.
    \end{proposition}
  \end{forthel}


  \subsection*{Unions, Intersections, Complements}

  \begin{forthel}
    \begin{definition}\printlabel{FOUNDATIONS_01_2159753924968448}
      Let $A, B$ be classes.
      The union of $A$ and $B$ is $\{ x \mid x \in A$ or $x \in B \}$.
    \end{definition}

    Let $A \cup B$ stand for the union of $A$ and $B$.
  \end{forthel}

  \begin{forthel}
    \begin{definition}\printlabel{FOUNDATIONS_01_5744033011859456}
      Let $A, B$ be classes.
      The intersection of $A$ and $B$ is $\{ x \mid x \in A$ and $x \in B \}$.
    \end{definition}

    Let $A \cap B$ stand for the intersection of $A$ and $B$.
  \end{forthel}

  \begin{forthel}
    \begin{definition}\printlabel{FOUNDATIONS_01_7620345041256448}
      Let $A, B$ be classes.
      The complement of $B$ in $A$ is $\{ x \mid x \in A$ and $x \notin B \}$.
    \end{definition}

    Let $A \setminus B$ stand for the complement of $B$ in $A$.
  \end{forthel}


  \subsection*{Disjoint Classes}

  \begin{forthel}
    \begin{definition}\printlabel{FOUNDATIONS_01_4981913324355584}
      Let $A, B$ be classes.
      $A$ and $B$ are disjoint iff $A$ and $B$ have no common elements.
    \end{definition}
  \end{forthel}

  \begin{forthel}
    \begin{proposition}\printlabel{FOUNDATIONS_01_1211191546347520}
      Let $A, B$ be classes.
      Then $A$ and $B$ are disjoint iff $A \cap B$ is empty.
    \end{proposition}
  \end{forthel}
\end{document}
