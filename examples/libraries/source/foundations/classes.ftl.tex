\documentclass{naproche-library}
\libinput{foundations-preamble}
\begin{document}
\begin{smodule}{classes.ftl}
  \importmodule[meta-inf]{vocabulary.ftl}
  \importmodule[meta-inf]{macros.ftl}
  
  \symdef{set}[args=a]{\{\argsep{#1},\}}

  \section*{Classes}

  \subsection*{Sub- and Superclasses}

  \begin{definition}[forthel,id=FOUNDATIONS_01_3275578358628352,printid]
    Let $A$ be a class.
    A subclass of $A$ is a class $B$ such that every element of $B$ is an
    element of $A$.

    Let $B \subseteq A$ stand for $B$ is a subclass of $A$.
    Let $B \subset A$ stand for $B \subseteq A$.

    Let a superclass of $B$ stand for a class $A$ such that $B \subseteq A$.
    Let $B \supseteq A$ stand for $B$ is a superclass of $A$.
    Let $B \supset A$ stand for $B \subseteq A$.

    Let a proper subclass of $A$ stand for a subclass $B$ of $A$ such that $B \neq A$.
    Let $B \subsetneq A$ stand for $B$ is a proper subclass of $A$.

    Let a proper superclass of $B$ stand for a superclass $A$ of $B$ such that $A \neq B$.
    Let $B \supsetneq A$ stand for $B$ is a proper superclass of $A$.

    Let $A$ includes $B$ stand for $B \subseteq A$.
    Let $B$ is included in $A$ stand for $B \subseteq A$.
  \end{definition}

  \begin{proposition}[forthel,id=FOUNDATIONS_01_5994555614691328,printid]
    Let $A$ be a class.
    Then $A \subseteq A$.
  \end{proposition}
  \begin{proof}[forthel]
    Every element of $A$ is contained in $A$.
    Therefore $A \subseteq A$.
  \end{proof}

  \begin{proposition}[forthel,id=FOUNDATIONS_01_3939677545431040,printid]
    Let $A, B, C$ be classes.
    If $A \subseteq B$ and $B \subseteq C$ then $A \subseteq C$.
  \end{proposition}
  \begin{proof}[forthel]
    Assume $A \subseteq B$ and $B \subseteq C$.
    Then every element of $A$ is contained in $B$ and every element of $B$ is contained in $C$.
    Hence every element of $A$ is contained in $C$.
    Thus $A \subseteq C$.
  \end{proof}

  \begin{proposition}[forthel,id=FOUNDATIONS_01_7159957847801856,printid]
    Let $A, B$ be classes.
    If $A \subseteq B$ and $B \subseteq A$ then $A = B$.
  \end{proposition}
  \begin{proof}[forthel]
    Assume $A \subseteq B$ and $B \subseteq A$.
    Then every element of $A$ is contained in $B$ and every element of $B$ is contained in $A$.
    Hence $A = B$.
  \end{proof}


  \subsection*{The Empty Class}

  \begin{definition}[forthel,id=FOUNDATIONS_01_6252477624090624,printid]
    Let $A$ be a class.
    $A$ is empty iff $A$ has no elements.

    Let $A$ is nonempty stand for $A$ is not empty.
  \end{definition}

  \begin{definition}[forthel,id=FOUNDATIONS_01_7939928493129728,printid]
    $\emptyset = \{ x \mid x \neq x \}$.
  \end{definition}

  \begin{proposition}[forthel,id=FOUNDATIONS_01_2263153161273344,printid]
    Let $A$ be a class.
    $A$ is empty iff $A = \emptyset$.
  \end{proposition}
  \begin{proof}[forthel]
    We can show that $\emptyset$ is empty.
    Indeed any element $x$ of $\emptyset$ is nonequal to $x$.
    Hence if $A = \emptyset$ then $A$ is empty.
    If $A$ is empty then $A$ and $\emptyset$ have no elements.
    Hence if $A$ is empty then $A \subseteq \emptyset$ and $\emptyset \subseteq A$.
    Thus if $A$ is empty then $A = \emptyset$.
  \end{proof}

  \begin{corollary}[forthel,id=FOUNDATIONS_01_1495141426659328,printid]
    $\emptyset$ is empty.
  \end{corollary}

  \begin{corollary}[forthel,id=FOUNDATIONS_01_6931785090859008,printid]
    Let $A$ be a class.
    Then $\emptyset \subseteq A$.
  \end{corollary}
  \begin{proof}[forthel]
    $\emptyset$ has no elements.
    Hence every element of $\emptyset$ is contained in $A$.
  \end{proof}


  \subsection*{Unordered Pairs}

  \begin{definition}[forthel,id=FOUNDATIONS_01_3471035364016128,printid]
    Let $a, b$ be objects.
    The unordered pair of $a$ and $b$ is $\{ x \mid x = a$ or $x = b \}$.

    Let $\set{a, b}$ stand for the unordered pair of $a$ and $b$.
  \end{definition}

  \begin{definition}[forthel,id=FOUNDATIONS_01_605432672419840,printid]
    An unordered pair is a class $A$ such that $A = \set{a, b}$ for some distinct objects $a, b$.
  \end{definition}

  \begin{definition}[forthel,id=FOUNDATIONS_01_1160414603771904,printid]
    Let $a$ be an object.
    The singleton class of $a$ is $\{ x \mid x = a \}$.

    Let $\set{a}$ stand for the singleton class of $a$.
  \end{definition}

  \begin{definition}[forthel,id=FOUNDATIONS_01_6786618161627136,printid]
    A singleton class is a class $A$ such that $A = \set{a}$ for some object $a$.
  \end{definition}

  \begin{proposition}[forthel,id=FOUNDATIONS_01_6125259604361216,printid]
    Let $a, a', b, b'$ be objects.
    Assume $\set{a, b} = \set{a', b'}$.
    Then ($a = a'$ and $b = b'$) or ($a = b'$ and $b = a'$).
  \end{proposition}
  \begin{proof}[forthel]
    We have $a = a'$ or $a = b'$.
    If $a = a'$ then $b = b'$.
    If $a = b'$ then $b = a'$.
    Hence ($a = a'$ and $b = b'$) or ($a = b'$ and $b = a'$).
  \end{proof}

  \begin{corollary}[forthel,id=FOUNDATIONS_01_6954678910713856,printid]
    Let $a, a'$ be objects.
    If $\set{a} = \set{a'}$ then $a = a'$.
  \end{corollary}

  \begin{definition}[forthel,id=FOUNDATIONS_01_29867480293464351,printid]
    Let $A$ be a class.
    A unique element of $A$ is an element $a$ of $A$ such that for each $x \in A$ we have $x = a$.
  \end{definition}

  \begin{proposition}[forthel,id=FOUNDATIONS_01_0166348647163481,printid]
    Let $A$ be a class.
    Then $A$ has a unique element iff $A = \set{a}$ for some object $a$.
  \end{proposition}


  \subsection*{Unions, Intersections, Complements}

  \begin{definition}[forthel,id=FOUNDATIONS_01_2159753924968448,printid]
    Let $A, B$ be classes.
    The union of $A$ and $B$ is $\{ x \mid x \in A$ or $x \in B \}$.

    Let $A \cup B$ stand for the union of $A$ and $B$.
  \end{definition}

  \begin{definition}[forthel,id=FOUNDATIONS_01_5744033011859456,printid]
    Let $A, B$ be classes.
    The intersection of $A$ and $B$ is $\{ x \mid x \in A$ and $x \in B \}$.

    Let $A \cap B$ stand for the intersection of $A$ and $B$.
  \end{definition}

  \begin{definition}[forthel,id=FOUNDATIONS_01_7620345041256448,printid]
    Let $A, B$ be classes.
    The complement of $B$ in $A$ is $\{ x \mid x \in A$ and $x \notin B \}$.

    Let $A \setminus B$ stand for the complement of $B$ in $A$.
  \end{definition}


  \subsection*{Disjoint Classes}

  \begin{definition}[forthel,id=FOUNDATIONS_01_4981913324355584,printid]
    Let $A, B$ be classes.
    $A$ and $B$ are disjoint iff $A$ and $B$ have no common elements.
  \end{definition}

  \begin{proposition}[forthel,id=FOUNDATIONS_01_1211191546347520,printid]
    Let $A, B$ be classes.
    Then $A$ and $B$ are disjoint iff $A \cap B$ is empty.
  \end{proposition}
\end{smodule}
\end{document}
