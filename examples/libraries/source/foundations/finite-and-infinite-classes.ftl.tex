\documentclass{naproche-library}
\libinput{foundations-preamble}
\begin{document}
\begin{smodule}{finite-and-infinite-classes.ftl}
  \importmodule[libraries]{foundations?segments-of-natural-numbers.ftl}
  \importmodule[libraries]{foundations?equinumerosity.ftl}

  \section*{Finite and Infinite Classes}

  \subsection*{Finite Classes}

  \begin{definition}[forthel,id=FOUNDATIONS_14_3512046897512410]
    Let $X$ be a class and $n$ be a natural number.
    $X$ has $n$ elements iff $X$ is equinumerous to $\{ 1, \dots, n \}$.
  \end{definition}

  \begin{definition}[forthel,id=FOUNDATIONS_14_3694156977274880]
    Let $X$ be a class.
    $X$ is finite iff there exists a natural number $n$ such that $X$ has $n$ elements.
  \end{definition}

  \begin{proposition}[forthel,id=FOUNDATIONS_14_3929085203972096]
    Let $X, Y$ be classes.
    If $X$ is finite and $Y$ is equinumerous to $X$ then $Y$ is finite.
  \end{proposition}
  \begin{proof}[forthel]
    Assume that $X$ is finite and $Y$ is equinumerous to $X$.
    Take a natural number $n$ and a bijection $f$ between $\{ 1, \dots, n \}$ and $X$ and a bijection $g$ between $X$ and $Y$.
    Then $g \circ f$ is a bijection between $\{ 1, \dots, n \}$ and $Y$ (by \printref{FOUNDATIONS_08_6435206693126144}).
    Indeed $X, Y$ are classes.
    Hence $Y$ is finite.
  \end{proof}

  \begin{proposition}[forthel,id=FOUNDATIONS_14_5132547854597502]
    Let $X$ be a class.
    $X$ has $0$ elements iff $X = \emptyset$.
  \end{proposition}

  \begin{proposition}[forthel,id=FOUNDATIONS_14_6812054297034125]
    Let $X$ be a class.
    $X$ has $1$ element iff $X = \set{a}$ for some object $a$.
  \end{proposition}
  \begin{proof}[forthel]
    Case $X$ has $1$ element.
      Take a bijection $f$ between $\{ 1, \dots, 1 \}$ and $X$.
      We have $\{ 1, \dots, 1 \} = \set{1}$.
      Hence $X = \set{f(1)}$.
    End.

    Case $X = \set{a}$ for some object $a$.
      Consider an object $a$ such that $X = \set{a}$.
      Define $f(x) = 1$ for $x \in \set{a}$.
    Then $f$ is a bijection between $\set{a}$ and $\{ 1, \dots, 1 \}$.
    End.
  \end{proof}

  \begin{proposition}[forthel,id=FOUNDATIONS_14_3468912675458910]
    Let $X$ be a class.
    $X$ has $2$ elements iff $X = \set{a, b}$ for some distinct objects $a, b$.
  \end{proposition}
  \begin{proof}[forthel]
    Case $X$ has $2$ elements.
      Take a bijection $f$ between $\{ 1, \dots, 2 \}$ and $X$.
      We have $\{ 1, \dots, 2 \} = \set{1, 2}$.
      Hence $X = \set{f(1), f(2)}$.
      We have $f(1) \neq f(2)$.
    End.

    Case $X = \set{a, b}$ for some distinct objects $a, b$.
      Consider distinct objects $a, b$ such that $X = \set{a, b}$.
      Define \[f(x) =
        \begin{cases}
          1 & : x = a \\
          2 & x = b
        \end{cases}\]
      for $x \in \set{a, b}$.
      Then $f$ is a bijection between $\set{a, b}$ and $\{ 1, \dots, 2 \}$.
      Indeed $f$ is injective and surjective onto $\{ 1, \dots, 2 \}$.
    End.
  \end{proof}

  \begin{proposition}[forthel,id=FOUNDATIONS_14_0615204230800975]
    Let $n$ be a natural number and $X$ be a class that has $n$ elements and $a$ be an object such that $a \notin X$.
    Then $X \cup \set{a}$ has $n + 1$ elements.
  \end{proposition}
  \begin{proof}[forthel]
    Take a bijection $f$ between $X$ and $\{ 1, \dots, n \}$.
    Define \[g(x) =
      \begin{cases}
        f(x)  & : x \in X \\
        n + 1 & : x = a
      \end{cases}\]
    for $x \in X \cup \set{a}$.

    (1) $g$ is a map from $X \cup \set{a}$ to $\{ 1, \dots, n + 1 \}$. \newline
    Indeed we can show that $g(x) \in \{ 1, \dots, n + 1 \}$ for all $x \in X \cup \set{a}$.
      Let $x \in X \cup \set{a}$.
      If $x \in X$ then $g(x) \in \{ 1, \dots, n \}$.
      If $x = a$ then $g(x) = n + 1$.
    End.

    (2) $g$ is injective. \newline
    Proof.
      Let $x, y \in \dom(g)$.
      Assume $x \neq y$.
      
      Case $x, y \in X$. Obvious.

      Case $x \in X$ and $y = a$. Obvious.

      Case $x = a$ and $y \in X$. Obvious.
    Qed.

    (3) $g$ is surjective onto $\{ 1, \dots, n + 1 \}$.
    Indeed we can show that for every $k \in \{1, \dots, n + 1 \}$ there exists an $x \in \dom(g)$ such that $k = g(x)$. \newline
    Proof.
      Let $k \in \{ 1, \dots, n + 1 \}$.

      Case $k \leq n$.
        Then $k \in \{ 1, \dots, n \}$.
        Hence we can take a $x \in X$ such that $k = f(x)$.
      End.

      Case $k = n + 1$.
        Then $k = g(a)$.
      End.
    Qed.

    Hence $g$ is a bijection between $X \cup \set{a}$ and $\{ 1, \dots, n + 1 \}$.
  \end{proof}


  \subsection*{Infinite Classes}

  \begin{definition}[forthel,id=FOUNDATIONS_14_6612510618681344]
    Let $X$ be a class.
    $X$ is infinite iff $X$ is not finite.
  \end{definition}

  \begin{proposition}[forthel,id=FOUNDATIONS_14_5814530911240192]
    Let $X, Y$ be classes.
    If $X$ is infinite and $Y$ is equinumerous to $X$ then $Y$ is infinite.
  \end{proposition}
  \begin{proof}[forthel]
    Assume that $Y$ is equinumerous to $X$.
    If $Y$ is finite then $X$ is finite.
    Hence if $X$ is infinite then $Y$ is infinite.
  \end{proof}
\end{smodule}
\end{document}
