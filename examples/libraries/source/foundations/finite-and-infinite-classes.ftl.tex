\documentclass[10pt]{article}
\usepackage[english]{babel}
\usepackage{../../../../lib/tex/naproche}
% Top-level sections and proofs

\newtheoremstyle{foundations}% name of the style to be used
  {\dimexpr\ftlparskip+4pt\relax}% measure of space to leave above the theorem.
  % (We must add \ftlparskip since otherwise the name of the label would not fit
  % into the border drawn around the environment.)
  {4pt}% measure of space to leave below the theorem.
  {\normalfont}% name of font to use in the body of the theorem
  {0pt}% measure of space to indent
  {\bfseries}% name of head font
  {}% punctuation between head and body
  { }% space after theorem head; " " = normal inter-word space
  {\thmname{#1}\thmnumber{ #2}\thmnote{ (#3)}.}

\let\definition\relax
\let\enddefinition\relax
\let\signature\relax
\let\endsignature\relax
\let\axiom\relax
\let\endaxiom\relax
\let\theorem\relax
\let\endtheorem\relax
\let\proposition\relax
\let\endproposition\relax
\let\lemma\relax
\let\endlemma\relax
\let\corollary\relax
\let\endcorollary\relax

\theoremstyle{foundations}

\newmdtheoremenv[
  backgroundcolor=lightgray!10,
  linecolor=black,
  innertopmargin=1em
]{definition}{Definition}[section]
\newmdtheoremenv[
  backgroundcolor=lightgray!10,
  linecolor=black,
  innertopmargin=1em
]{signature}[definition]{Signature}
\newmdtheoremenv[
  backgroundcolor=lightgray!10,
  linecolor=black,
  innertopmargin=1em
]{axiom}[definition]{Axiom}
\newmdtheoremenv[
  backgroundcolor=lightgray!10,
  linecolor=black,
  innertopmargin=1em
]{theorem}[definition]{Theorem}
\newmdtheoremenv[
  backgroundcolor=lightgray!10,
  linecolor=black,
  innertopmargin=1em
]{proposition}[definition]{Proposition}
\newmdtheoremenv[
  backgroundcolor=lightgray!10,
  linecolor=black,
  innertopmargin=1em
]{lemma}[definition]{Lemma}
\newmdtheoremenv[
  backgroundcolor=lightgray!10,
  linecolor=black,
  innertopmargin=1em
]{corollary}[definition]{Corollary}


% Hide imports

\usepackage{scontents}
\newenvsc{imports}[store-env=imports,print-env=false]


% Labels of top-level sections

\renewcommand{\showlabelfont}{\scriptsize\ttfamily\color{gray}}


% Non-symbolic "\implies" and "\iff"

\renewcommand{\implies}{\quad \text{implies} \quad}
\renewcommand{\iff}{\quad \text{iff} \quad}

\usepackage{amssymb}
\usepackage{pgffor}

\newcommand{\id}{\textrm{id}}
\newcommand{\range}{\textrm{range}}
\newcommand{\const}[2]{\textrm{const}_{#1}^{#2}}
\newcommand{\symdiff}{\mathop{\triangle}}
\newcommand{\pow}{\mathcal{P}}
\newcommand{\onto}{\twoheadrightarrow}
\newcommand{\into}{\hookrightarrow}

% E.g. `\set{1, 2, 3, 4}` prints the numbers 1, 2, 3 and 4 as a comma
% separated list which is enclosed within curly braces.
\def\set#1{\ensuremath{
  \{\foreach[count=\i]\x in {#1}{
    \ifnum\i>1,\,\fi
    \x
  }\}
}}

\useackage{amssymb}

\newcommand{\Nat}{\mathbb{N}}
\newcommand{\Prime}{\mathbb{P}}
\renewcommand{\succ}{\textrm{succ}}
\newcommand{\pred}{\textrm{pred}}
\newcommand{\add}{\textrm{add}}
\newcommand{\mul}{\textrm{mul}}
\renewcommand{\exp}{\textrm{exp}}
\newcommand{\fac}{\textrm{fac}}
\renewcommand{\div}{\mathrel{\textrm{div}}}
\renewcommand{\mod}{\mathrel{\textrm{mod}}}

\begin{document}
  \begin{imports}
    \begin{forthel}
      %[prove off][check off]
      [readtex \path{libraries/source/foundations/segments-of-natural-numbers.ftl.tex}]
      [readtex \path{libraries/source/foundations/equinumerosity.ftl.tex}]
      %[prove on][check on]
    \end{forthel}
  \end{imports}


  \section*{Finite and Infinite Classes}

  \subsection*{Finite Classes}

  \begin{forthel}
    \begin{definition}\printlabel{FOUNDATIONS_14_3512046897512410}
      Let $X$ be a class and $n$ be a natural number.
      $X$ has $n$ elements iff $X$ is equinumerous to $\{ 1, \dots, n \}$.
    \end{definition}
  \end{forthel}

  \begin{forthel}
    \begin{definition}\printlabel{FOUNDATIONS_14_3694156977274880}
      Let $X$ be a class.
      $X$ is finite iff there exists a natural number $n$ such that $X$ has $n$ elements.
    \end{definition}
  \end{forthel}

  \begin{forthel}
    \begin{proposition}\printlabel{FOUNDATIONS_14_3929085203972096}
      Let $X, Y$ be classes.
      If $X$ is finite and $Y$ is equinumerous to $X$ then $Y$ is finite.
    \end{proposition}
    \begin{proof}
      Assume that $X$ is finite and $Y$ is equinumerous to $X$.
      Take a natural number $n$ and a bijection $f$ between $\{ 1, \dots, n \}$ and $X$ and a bijection $g$ between $X$ and $Y$.
      Then $g \circ f$ is a bijection between $\{ 1, \dots, n \}$ and $Y$ (by \printref{FOUNDATIONS_08_6435206693126144}).
      Indeed $X, Y$ are classes.
      Hence $Y$ is finite.
    \end{proof}
  \end{forthel}

  \begin{forthel}
    \begin{proposition}\printlabel{FOUNDATIONS_14_5132547854597502}
      Let $X$ be a class.
      $X$ has $0$ elements iff $X = \emptyset$.
    \end{proposition}
  \end{forthel}

  \begin{forthel}
    \begin{proposition}\printlabel{FOUNDATIONS_14_6812054297034125}
      Let $X$ be a class.
      $X$ has $1$ element iff $X = \set{a}$ for some object $a$.
    \end{proposition}
    \begin{proof}
      Case $X$ has $1$ element.
        Take a bijection $f$ between $\{ 1, \dots, 1 \}$ and $X$.
        We have $\{ 1, \dots, 1 \} = \set{1}$.
        Hence $X = \set{f(1)}$.
      End.

      Case $X = \set{a}$ for some object $a$.
        Consider an object $a$ such that $X = \set{a}$.
        Define $f(x) = 1$ for $x \in \set{a}$.
      Then $f$ is a bijection between $\set{a}$ and $\{ 1, \dots, 1 \}$.
      End.
    \end{proof}
  \end{forthel}

  \begin{forthel}
    \begin{proposition}\printlabel{FOUNDATIONS_14_3468912675458910}
      Let $X$ be a class.
      $X$ has $2$ elements iff $X = \set{a, b}$ for some distinct objects $a, b$.
    \end{proposition}
    \begin{proof}
      Case $X$ has $2$ elements.
        Take a bijection $f$ between $\{ 1, \dots, 2 \}$ and $X$.
        We have $\{ 1, \dots, 2 \} = \set{1, 2}$.
        Hence $X = \set{f(1), f(2)}$.
        We have $f(1) \neq f(2)$.
      End.

      Case $X = \set{a, b}$ for some distinct objects $a, b$.
        Consider distinct objects $a, b$ such that $X = \set{a, b}$.
        Define \[f(x) =
          \begin{cases}
            1 & : x = a \\
            2 & x = b
          \end{cases}\]
        for $x \in \set{a, b}$.
        Then $f$ is a bijection between $\set{a, b}$ and $\{ 1, \dots, 2 \}$.
        Indeed $f$ is injective and surjective onto $\{ 1, \dots, 2 \}$.
      End.
    \end{proof}
  \end{forthel}

  \begin{forthel}
    \begin{proposition}\printlabel{FOUNDATIONS_14_0615204230800975}
      Let $n$ be a natural number and $X$ be a class that has $n$ elements and $a$ be an object such that $a \notin X$.
      Then $X \cup \set{a}$ has $n + 1$ elements.
    \end{proposition}
    \begin{proof}
      Take a bijection $f$ between $X$ and $\{ 1, \dots, n \}$.
      Define \[g(x) =
        \begin{cases}
          f(x)  & : x \in X \\
          n + 1 & : x = a
        \end{cases}\]
      for $x \in X \cup \set{a}$.

      (1) $g$ is a map from $X \cup \set{a}$ to $\{ 1, \dots, n + 1 \}$. \\
      Indeed we can show that $g(x) \in \{ 1, \dots, n + 1 \}$ for all $x \in X \cup \set{a}$.
        Let $x \in X \cup \set{a}$.
        If $x \in X$ then $g(x) \in \{ 1, \dots, n \}$.
        If $x = a$ then $g(x) = n + 1$.
      End.

      (2) $g$ is injective.

      (3) $g$ is surjective onto $\{ 1, \dots, n + 1 \}$. \\
      Proof.
        Let $k \in \{ 1, \dots, n + 1 \}$.

        Case $k \leq n$.
          Then $k \in \{ 1, \dots, n \}$.
          Hence we can take a $x \in X$ such that $k = f(x)$.
        End.

        Case $k = n + 1$.
          Then $k = g(a)$.
        End.
      Qed.

      Hence $g$ is a bijection between $X \cup \set{a}$ and $\{ 1, \dots, n + 1 \}$.
    \end{proof}
  \end{forthel}


  \subsection*{Infinite Classes}

  \begin{forthel}
    \begin{definition}\printlabel{FOUNDATIONS_14_6612510618681344}
      Let $X$ be a class.
      $X$ is infinite iff $X$ is not finite.
    \end{definition}
  \end{forthel}

  \begin{forthel}
    \begin{proposition}\printlabel{FOUNDATIONS_14_5814530911240192}
      Let $X, Y$ be classes.
      If $X$ is infinite and $Y$ is equinumerous to $X$ then $Y$ is infinite.
    \end{proposition}
    \begin{proof}
      Assume that $Y$ is equinumerous to $X$.
      If $Y$ is finite then $X$ is finite.
      Hence if $X$ is infinite then $Y$ is infinite.
    \end{proof}
  \end{forthel}
\end{document}
