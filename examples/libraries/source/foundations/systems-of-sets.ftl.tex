\documentclass[10pt]{article}
\usepackage[english]{babel}
\usepackage{../../../meta-inf/lib/naproche}
% Top-level sections and proofs

\newtheoremstyle{foundations}% name of the style to be used
  {\dimexpr\ftlparskip+4pt\relax}% measure of space to leave above the theorem.
  % (We must add \ftlparskip since otherwise the name of the label would not fit
  % into the border drawn around the environment.)
  {4pt}% measure of space to leave below the theorem.
  {\normalfont}% name of font to use in the body of the theorem
  {0pt}% measure of space to indent
  {\bfseries}% name of head font
  {}% punctuation between head and body
  { }% space after theorem head; " " = normal inter-word space
  {\thmname{#1}\thmnumber{ #2}\thmnote{ (#3)}.}

\let\definition\relax
\let\enddefinition\relax
\let\signature\relax
\let\endsignature\relax
\let\axiom\relax
\let\endaxiom\relax
\let\theorem\relax
\let\endtheorem\relax
\let\proposition\relax
\let\endproposition\relax
\let\lemma\relax
\let\endlemma\relax
\let\corollary\relax
\let\endcorollary\relax

\theoremstyle{foundations}

\newmdtheoremenv[
  backgroundcolor=lightgray!10,
  linecolor=black,
  innertopmargin=1em
]{definition}{Definition}[section]
\newmdtheoremenv[
  backgroundcolor=lightgray!10,
  linecolor=black,
  innertopmargin=1em
]{signature}[definition]{Signature}
\newmdtheoremenv[
  backgroundcolor=lightgray!10,
  linecolor=black,
  innertopmargin=1em
]{axiom}[definition]{Axiom}
\newmdtheoremenv[
  backgroundcolor=lightgray!10,
  linecolor=black,
  innertopmargin=1em
]{theorem}[definition]{Theorem}
\newmdtheoremenv[
  backgroundcolor=lightgray!10,
  linecolor=black,
  innertopmargin=1em
]{proposition}[definition]{Proposition}
\newmdtheoremenv[
  backgroundcolor=lightgray!10,
  linecolor=black,
  innertopmargin=1em
]{lemma}[definition]{Lemma}
\newmdtheoremenv[
  backgroundcolor=lightgray!10,
  linecolor=black,
  innertopmargin=1em
]{corollary}[definition]{Corollary}


% Hide imports

\usepackage{scontents}
\newenvsc{imports}[store-env=imports,print-env=false]


% Labels of top-level sections

\renewcommand{\showlabelfont}{\scriptsize\ttfamily\color{gray}}


% Non-symbolic "\implies" and "\iff"

\renewcommand{\implies}{\quad \text{implies} \quad}
\renewcommand{\iff}{\quad \text{iff} \quad}

\usepackage{amssymb}
\usepackage{pgffor}

\newcommand{\id}{\textrm{id}}
\newcommand{\range}{\textrm{range}}
\newcommand{\const}[2]{\textrm{const}_{#1}^{#2}}
\newcommand{\symdiff}{\mathop{\triangle}}
\newcommand{\pow}{\mathcal{P}}
\newcommand{\onto}{\twoheadrightarrow}
\newcommand{\into}{\hookrightarrow}

% E.g. `\set{1, 2, 3, 4}` prints the numbers 1, 2, 3 and 4 as a comma
% separated list which is enclosed within curly braces.
\def\set#1{\ensuremath{
  \{\foreach[count=\i]\x in {#1}{
    \ifnum\i>1,\,\fi
    \x
  }\}
}}

\begin{document}
  \begin{imports}
    \begin{forthel}
      %[prove off][check off]
      [readtex \path{libraries/source/foundations/sub-and-supersets.ftl.tex}]
      %[prove on][check on]
    \end{forthel}
  \end{imports}


  \section*{Systems of Sets}

  \begin{forthel}
    \begin{definition}\printlabel{FOUNDATIONS_10_5805323570905088}
      A system of sets is a class $X$ such that every element of $X$ is a set.
    \end{definition}
  \end{forthel}

  \begin{forthel}
    \begin{definition}\printlabel{FOUNDATIONS_10_1631952387964928}
      A system of nonempty sets is a class $X$ such that every element of $X$ is a nonempty set.
    \end{definition}
  \end{forthel}

  \begin{forthel}
    \begin{definition}\printlabel{FOUNDATIONS_10_943381479948288}
      Let $A$ be a class.
      A system of subsets of $A$ is a class $X$ such that every element of $X$ is a subset of $A$.
    \end{definition}
  \end{forthel}

  \begin{forthel}
    \begin{proposition}\printlabel{FOUNDATIONS_10_8268633648136192}
      Let $A$ be a class.
      Then $\emptyset$ is a system of subsets of $A$.
    \end{proposition}
  \end{forthel}

  \begin{forthel}
    \begin{proposition}\printlabel{FOUNDATIONS_10_7546016869908480}
      Let $A$ be a class.
      Then $\pow(A)$ is a system of subsets of $A$.
    \end{proposition}
  \end{forthel}

  \begin{forthel}
    \begin{proposition}
      Let $X, Y$ be systems of sets.
      Then $X \cup Y$ is a system of sets.
    \end{proposition}
  \end{forthel}

  \begin{forthel}
    \begin{proposition}
      Let $X, Y$ be systems of sets.
      Then $X \cap Y$ is a system of sets.
    \end{proposition}
  \end{forthel}

  \begin{forthel}
    \begin{proposition}
      Let $X, Y$ be systems of sets.
      Then $X \setminus Y$ is a system of sets.
    \end{proposition}
  \end{forthel}


  \subsection*{Unions Over Systems of Sets}

  \begin{forthel}
    \begin{definition}\printlabel{FOUNDATIONS_10_541772562300928}
      Let $X$ be a system of sets.
      The union over $X$ is $\{ a \mid a \in x$ for some $x \in X \}$.
    \end{definition}

    Let $\bigcup X$ stand for the union over $X$.
  \end{forthel}

  \begin{forthel}
    \begin{proposition}\printlabel{FOUNDATIONS_10_4872701241982976}
      $\bigcup \emptyset = \emptyset$.
    \end{proposition}
    \begin{proof}
      $\bigcup \emptyset = \{ a \mid a \in x$ for some $x \in \emptyset \}$.
      $\emptyset$ has no elements.
      Hence there is no object $a$ such that $a \in x$ for some $x \in \emptyset$.
      Thus $\bigcup \emptyset = \emptyset$.
    \end{proof}
  \end{forthel}

  \begin{forthel}
    \begin{proposition}\printlabel{FOUNDATIONS_10_2559541585641472}
      Let $x, y$ be sets.
      Then $\bigcup \set{x, y} = x \cup y$.
    \end{proposition}
    \begin{proof}
      Let us show that $\bigcup \set{x, y} \subseteq x \cup y$.
        Let $a \in \bigcup \set{x, y}$.
        Then $a$ is contained in some element of $\set{x, y}$.
        Hence $a \in x$ or $a \in y$.
        Thus $a \in x \cup y$.
      End.

      Let us show that $x \cup y \subseteq \bigcup \set{x, y}$.
        Let $a \in x \cup y$.
        Then $a \in x$ or $a \in y$.
        Hence $a$ is contained in some element of $\set{x, y}$.
        Therefore $a \in \bigcup \set{x, y}$.
      End.
    \end{proof}
  \end{forthel}

  \begin{forthel}
    \begin{corollary}\printlabel{FOUNDATIONS_10_2157223832715264}
      Let $x$ be a set.
      Then $\bigcup \set{x} = x$.
    \end{corollary}
  \end{forthel}


  \subsection*{Intersections Over Systems of Sets}

  \begin{forthel}
    \begin{definition}\printlabel{FOUNDATIONS_10_2659345095458816}
      Let $X$ be a system of sets.
      The intersection over $X$ is $\{ a \mid a \in x$ for all $x \in X \}$.
    \end{definition}

    Let $\bigcap X$ stand for the intersection over $X$.
  \end{forthel}

  \begin{forthel}
    \begin{proposition}\printlabel{FOUNDATIONS_10_2809770322952192}
      $\bigcap \emptyset$ is the class of all objects.
    \end{proposition}
    \begin{proof}
      Define $V = \{ x \mid x$ is an object $\}$.
      We have $\bigcap \emptyset \subseteq V$.
      Indeed every element of $\bigcap \emptyset$ is an object.

      Let us show that $V \subseteq \bigcap \emptyset$.
        Let $a \in V$.
        Then $a$ is an object.
        For every $x \in \emptyset$ we have $a \in x$.
        Indeed $\emptyset$ has no elements.
        Thus $a \in \bigcap \emptyset$.
      End.
    \end{proof}
  \end{forthel}

  \begin{forthel}
    \begin{proposition}\printlabel{FOUNDATIONS_10_7851827447988224}
      Let $x, y$ be sets.
      Then $\bigcap \set{x, y} = x \cap y$.
    \end{proposition}
    \begin{proof}
      Let us show that $\bigcap \set{x, y} \subseteq x \cap y$.
        Let $a \in \bigcap \set{x, y}$.
        Then $a$ is contained in every element of $\set{x, y}$.
        Hence $a \in x$ and $a \in y$.
        Thus $a \in x \cap y$.
      End.

      Let us show that $x \cap y \subseteq \bigcap \set{x, y}$.
        Let $a \in x \cap y$.
        Then $a \in x$ and $a \in y$.
        Hence $a$ is contained in every element of $\set{x, y}$.
        Therefore $a \in \bigcap \set{x, y}$.
      End.
    \end{proof}
  \end{forthel}

  \begin{forthel}
    \begin{corollary}\printlabel{FOUNDATIONS_10_7239895674257408}
      Let $x$ be a set.
      Then $\bigcap \set{x} = x$.
    \end{corollary}
  \end{forthel}
\end{document}
