\documentclass{naproche-library}
\libinput{foundations-preamble}
\begin{document}
\begin{smodule}{computation-laws-for-products.ftl}
  \importmodule[libraries]{foundations?computation-laws-for-classes.ftl}
  \importmodule[libraries]{foundations?pairs-and-products.ftl}

  \section*{Computation Laws for Cartesian Products}

  \subsection*{Subclass Laws}

  \begin{proposition}[forthel,id=FOUNDATIONS_05_5719644021194752,printid]
    Let $A, B, C$ be classes.
    IF $A \subseteq B$ then $A \times C \subseteq B \times C$.
  \end{proposition}
  \begin{proof}[forthel]
    Assume $A \subseteq B$.
    Let $x \in A \times C$.
    Take $a \in A$ and $c \in C$ such that $x = (a, c)$.
    Then $a \in B$.
    Hence $(a, c) \in B \times C$.
  \end{proof}

  \begin{proposition}[forthel,id=FOUNDATIONS_05_4888282951319552,printid]
    Let $A, A', B, B'$ be classes.
    Assume that $A$ and $A'$ are nonempty.
    Then $(A \times A') \subseteq (B \times B')$ iff $A \subseteq B$ and $A' \subseteq B'$.
  \end{proposition}
  \begin{proof}[forthel]
    Case $(A \times A') \subseteq (B \times B')$.
      Let us show that for all $a \in A$ and all $a' \in A'$ we have $a \in B$ and $a' \in B'$.
        Let $a \in A$ and $a' \in A'$.
        Then $(a, a') \in A \times A'$.
        Hence $(a, a') \in B \times B'$.
        Thus $a \in B$ and $a' \in B'$.
      End.
    End.

    Case $A \subseteq B$ and $A' \subseteq B'$.
      Let $x \in A \times A'$.
      Take $a \in A$ and $a' \in A'$ such that $x = (a, a')$.
      Then $a \in B$ and $a' \in B'$.
      Hence $(a, a') \in B \times B'$.
    End.
  \end{proof}


  \subsection*{Distributivity of Product and Union}

  \begin{proposition}[forthel,id=FOUNDATIONS_05_8849658323402752,printid]
    Let $A, B, C$ be classes.
    Then $(A \cup B) \times C = (A \times C) \cup (B \times C)$.
  \end{proposition}
  \begin{proof}[forthel]
    Let us show that $((A \cup B) \times C) \subseteq (A \times C) \cup (B \times C).$ %!
      Let $x \in (A \cup B) \times C$.
      Take $y \in A \cup B$ and $c \in C$ such that $x = (y, c)$.
      Then $y \in A$ or $y \in B$.
      If $y \in A$ then $x \in A \times C$ and if $y \in B$ then $x \in B \times C$.
      Hence $x \in A \times C$ or $x \in B \times C$.
      Thus $x \in (A \times C) \cup (B \times C)$.
    End.

    Let us show that $((A \times C) \cup (B \times C)) \subseteq (A \cup B) \times C$. %!
      Let $x \in (A \times C) \cup (B \times C)$.
      Then $x \in A \times C$ or $x \in B \times C$.
      Take objects $y, c$ such that $x = (y, c)$.
      Then ($y \in A$ or $y \in B$) and $c \in C$.
      Hence $y \in A \cup B$.
      Thus $x \in (A \cup B) \times C$.
    End.
  \end{proof}

  \begin{proposition}[forthel,id=FOUNDATIONS_05_476526841692160,printid]
    Let $A, B, C$ be classes.
    Then $A \times (B \cup C) = (A \times B) \cup (A \times C)$.
  \end{proposition}
  \begin{proof}[forthel]
    Let us show that $A \times (B \cup C) \subseteq (A \times B) \cup (A \times C)$.
      Let $x \in A \times (B \cup C)$.
      Take $a \in A$ and $y \in B \cup C$ such that $x = (a, y)$.
      Then $y \in B$ or $y \in C$.
      Hence $x \in A \times B$ or $x \in A \times C$.
      Indeed if $y \in B$ then $x \in A \times B$ and if $y \in C$ then $x \in A \times C$.
      Thus $x \in (A \times B) \cup (A \times C)$.
    End.

    Let us show that $((A \times B) \cup (A \times C)) \subseteq A \times (B \cup C)$.
      Let $x \in (A \times B) \cup (A \times C)$.
      Then $x \in A \times B$ or $x \in A \times C$.
      Take objects $a, y$ such that $x = (a, y)$.
      Then $a \in A$ and ($y \in B$ or $y \in C$).
      Hence $x \in A \times (B \cup C)$.
    End.
  \end{proof}


  \subsection*{Distributivity of Product and Intersection}

  \begin{proposition}[forthel,id=FOUNDATIONS_05_1249567930580992,printid]
    Let $A, B, C$ be classes.
    Then $(A \cap B) \times C = (A \times C) \cap (B \times C)$.
  \end{proposition}
  \begin{proof}[forthel]
    Let us show that $((A \cap B) \times C) \subseteq (A \times C) \cap (B \times C)$. %!
      Let $x \in (A \cap B) \times C$.
      Take $y \in A \cap B$ and $c \in C$ such that $x = (y, c)$.
      Then $y \in A$ and $y \in B$.
      Hence $x \in A \times C$ and $x \in B \times C$.
      Thus $x \in (A \times C) \cap (B \times C)$.
    End.

    Let us show that $((A \times C) \cap (B \times C)) \subseteq (A \cap B) \times C$. %!
      Let $x \in (A \times C) \cap (B \times C)$.
      Then $x \in A \times C$ and $x \in B \times C$.
      Take objects $y, z$ such that $x = (y, z)$.
      Then ($y \in A$ and $y \in B$) and $z \in C$.
      Hence $y \in A \cap B$.
      Thus $x \in (A \cap B) \times C$.
    End.
  \end{proof}

  \begin{proposition}[forthel,id=FOUNDATIONS_05_954964241285120,printid]
    Let $A, B, C$ be classes.
    Then $A \times (B \cap C) = (A \times B) \cap (A \times C)$.
  \end{proposition}
  \begin{proof}[forthel]
    Let us show that $A \times (B \cap C) \subseteq (A \times B) \cap (A \times C)$.
      Let $x \in A \times (B \cap C)$.
      Take $a \in A$ and $b \in B \cap C$ such that $x = (a, b)$.
      Then $b \in B$ and $b \in C$.
      Hence $x \in A \times B$ and $x \in A \times C$.
      Thus $x \in (A \times B) \cap (A \times C)$.
    End.

    Let us show that $((A \times B) \cap (A \times C)) \subseteq A \times (B \cap C)$. %!
      Let $x \in (A \times B) \cap (A \times C)$.
      Then $x \in A \times B$ and $x \in A \times C$.
      Take objects $y, z$ such that $x = (y, z)$.
      Then $y \in A$ and ($z \in B$ and $z \in C$).
      Hence $x \in A \times (B \cap C)$.
    End.
  \end{proof}


  \subsection*{Distributivity of Product and Complement}

  \begin{proposition}[forthel,id=FOUNDATIONS_05_6495329908162560,printid]
    Let $A, B, C$ be classes.
    Then $(A \setminus B) \times C = (A \times C) \setminus (B \times C)$.
  \end{proposition}
  \begin{proof}[forthel]
    Let us show that $((A \setminus B) \times C) \subseteq (A \times C) \setminus (B \times C).$ %!
      Let $x \in (A \setminus B) \times C$.
      Take $a \in A \setminus B$ and $c \in C$ such that $x = (a, c)$.
      Then $a \in A$ and $a \notin B$.
      Hence $x \in A \times C$ and $x \notin B \times C$.
      Thus $x \in (A \times C) \setminus (B \times C)$.
    End.

    Let us show that $((A \times C) \setminus (B \times C)) \subseteq (A \setminus B) \times C$. %!
      Let $x \in (A \times C) \setminus (B \times C)$.
      Then $x \in A \times C$ and $x \notin B \times C$.
      Take $a \in A$ and $c \in C$ such that $x = (a, c)$.
      Then $a \notin B$.
      Indeed if $a \in B$ then $x \in B \times C$.
      Hence $a \in A \setminus B$.
      Thus $x \in (A \setminus B) \times C$.
    End.
  \end{proof}

  \begin{proposition}[forthel,id=FOUNDATIONS_05_3195639422779392,printid]
    Let $A, B, C$ be classes.
    Then $A \times (B \setminus C) = (A \times B) \setminus (A \times C)$.
  \end{proposition}
  \begin{proof}[forthel]
    Let us show that $A \times (B \setminus C) \subseteq (A \times B) \setminus (A \times C)$.
      Let $x \in A \times (B \setminus C)$.
      Take $a \in A$ and $b \in B \setminus C$ such that $x = (a, b)$.
      Then $b \in B$ and $b \notin C$.
      Hence $x \in A \times B$ and $x \notin A \times C$.
      Thus $x \in (A \times B) \setminus (A \times C)$.
    End.

    Let us show that $((A \times B) \setminus (A \times C)) \subseteq A \times (B \setminus C)$. %!
      Let $x \in (A \times B) \setminus (A \times C)$.
      Then $x \in A \times B$ and $x \notin A \times C$.
      Take objects $a, b$ such that $x = (a, b)$.
      Then $a \in A$ and ($b \in B$ and $b \notin C$).
      Hence $x \in A \times (B \setminus C)$.
    End.
  \end{proof}


  \subsection*{Equality Law}

  \begin{proposition}[forthel,id=FOUNDATIONS_05_2677218429894656,printid]
    Let $A, A', B, B'$ be classes.
    Assume that $A$ and $A'$ are nonempty or $B$ and $B'$ are nonempty.
    Then $(A \times A') = (B \times B')$ iff $A = B$ and $A' = B'$.
  \end{proposition}
  \begin{proof}[forthel]
    Case $A \times A' = B \times B'$.
      Then $A$ and $A'$ are nonempty iff $B$ and $B'$ are nonempty.

      Let us show that for all $a \in A$ and all $a' \in A'$ we have $a \in B$ and $a' \in B'$.
        Let $a \in A$ and $a' \in A'$.
        Then $(a, a') \in A \times A'$.
        Hence we can take $x \in B \times B'$ such that $x = (a, a')$.
        Thus $a \in B$ and $a' \in B'$.
      End.

      Therefore $A \subseteq B$ and $A' \subseteq B'$.
      Indeed $A$ and $A'$ are nonempty.

      Let us show that for all $b \in B$ and all $b' \in B'$ we have $b \in A$ and $b' \in A'$.
        Let $b \in B$ and $b' \in B'$.
        Then $(b, b') \in B \times B'$.
        Hence we can take $x \in A \times A'$ such that $x = (b, b')$.
        Thus $(b, b') \in A \times A'$.
      End.

      Therefore $B \subseteq A$ and $B' \subseteq A'$.
      Indeed $B$ and $B'$ are nonempty.
    End.

    Case $A = B$ and $A' = B'$. Trivial.
  \end{proof}


  \subsection*{Intersections of Products}

  \begin{proposition}[forthel,id=FOUNDATIONS_05_4154592050806784,printid]
    Let $A, A', B, B'$ be classes.
    Then $(A \times B) \cap (A' \times B') = (A \cap A') \times (B \cap B')$.
  \end{proposition}
  \begin{proof}[forthel]
    Let us show that $((A \times B) \cap (A' \times B')) \subseteq (A \cap A') \times (B \cap B')$. %!
      Let $x \in (A \times B) \cap (A' \times B')$.
      Then $x \in A \times B$ and $x \in A' \times B'$.
      Take objects $a, b$ such that $x = (a, b)$.
      Then $a \in A, A'$ and $b \in B, B'$.
      Hence $a \in A \cap A'$ and $b \in B \cap B'$.
      Thus $x \in (A \cap A') \times (B \cap B')$.
    End.

    Let us show that $(A \cap A') \times (B \cap B') \subseteq (A \times B) \cap (A' \times B')$.
      Let $x \in (A \cap A') \times (B \cap B')$.
      Take elements $a, b$ such that $x = (a, b)$.
      Then $a \in A \cap A'$ and $b \in B \cap B'$.
      Hence $a \in A, A'$ and $b \in B, B'$.
      Thus $x \in A \times B$ and $x \in A' \times B'$.
      Therefore $x \in (A \times B) \cap (A' \times B')$.
    End.
  \end{proof}


  \subsection*{Unions of Products}

  \begin{proposition}[forthel,id=FOUNDATIONS_05_7090174334861312,printid]
    Let $A, A', B, B'$ be classes.
    Then $(A \times B) \cup (A' \times B') \subseteq (A \cup A') \times (B \cup B')$.
  \end{proposition}
  \begin{proof}[forthel]
    Let $x \in (A \times B) \cup (A' \times B')$.
    Then $x \in A \times B$ or $x \in A' \times B'$.
    Take objects $a, b$ such that $x = (a, b)$.
    Then ($a \in A$ or $a \in A'$) and ($b \in B$ or $b \in B'$).
    Hence $a \in A \cup A'$ and $b \in B \cup B'$.
    Thus $x \in (A \cup A') \times (B \cup B')$.
  \end{proof}


  \subsection*{Complements of Products}

  \begin{proposition}[forthel,id=FOUNDATIONS_05_5552125989879808,printid]
    Let $A, A', B, B'$ be classes.
    Then $(A \times B) \setminus (A' \times B') = (A \times (B \setminus B')) \cup ((A \setminus A') \times B)$.
  \end{proposition}
  \begin{proof}[forthel]
    Let us show that $((A \times B) \setminus (A' \times B')) \subseteq (A \times (B \setminus B')) \cup ((A \setminus A') \times B)$.
      Let $x \in (A \times B) \setminus (A' \times B')$.
      Then $x \in A \times B$ and $x \notin A' \times B'$.
      Take $a \in A$ and $b \in B$ such that $x = (a, b)$.
      Then it is wrong that $a \in A'$ and $b \in B'$.
      Hence $a \notin A'$ or $b \notin B'$.
      Thus $a \in A \setminus A'$ or $b \in B \setminus B'$.
      Therefore $x \in A \times (B \setminus B')$ or $x \in (A \setminus A') \times B$.
      Hence we have $x \in (A \times (B \setminus B')) \cup ((A \setminus A') \times B)$.
    End.

    Let us show that $(A \times (B \setminus B')) \cup ((A \setminus A') \times B) \subseteq (A \times B) \setminus (A' \times B')$.
      Let $x \in (A \times (B \setminus B')) \cup ((A \setminus A') \times B)$.
      Then $x \in (A \times (B \setminus B'))$ or $x \in ((A \setminus A') \times B)$.
      Take elements $a,b$ such that $x = (a, b)$ (by \printref{FOUNDATIONS_04_2877806274936832}).
      Then ($a \in A$ and $b \in B \setminus B'$) or ($a \in A \setminus A'$ and $b \in B$).

      Case $a \in A$ and $b \in B \setminus B'$.
        Then $a \in A$ and $b \in B$.
        Hence $x \in A \times B$.
        We have $b \notin B'$.
        Thus $x \notin A' \times B'$.
        Therefore $x \in (A \times B) \setminus (A' \times B')$.
      End.

      Case $a \in A \setminus A'$ and $b \in B$.
        Then $a \in A$ and $b \in B$.
        Hence $x \in A \times B$.
        We have $a \notin A'$.
        Thus $x \notin A' \times B'$.
        Therefore $x \in (A \times B) \setminus (A' \times B')$.
      End.
    End.
  \end{proof}
\end{smodule}
\end{document}
