\documentclass[10pt]{article}
\usepackage[english]{babel}
\usepackage{../../../meta-inf/lib/naproche}
% Top-level sections and proofs

\newtheoremstyle{foundations}% name of the style to be used
  {\dimexpr\ftlparskip+4pt\relax}% measure of space to leave above the theorem.
  % (We must add \ftlparskip since otherwise the name of the label would not fit
  % into the border drawn around the environment.)
  {4pt}% measure of space to leave below the theorem.
  {\normalfont}% name of font to use in the body of the theorem
  {0pt}% measure of space to indent
  {\bfseries}% name of head font
  {}% punctuation between head and body
  { }% space after theorem head; " " = normal inter-word space
  {\thmname{#1}\thmnumber{ #2}\thmnote{ (#3)}.}

\let\definition\relax
\let\enddefinition\relax
\let\signature\relax
\let\endsignature\relax
\let\axiom\relax
\let\endaxiom\relax
\let\theorem\relax
\let\endtheorem\relax
\let\proposition\relax
\let\endproposition\relax
\let\lemma\relax
\let\endlemma\relax
\let\corollary\relax
\let\endcorollary\relax

\theoremstyle{foundations}

\newmdtheoremenv[
  backgroundcolor=lightgray!10,
  linecolor=black,
  innertopmargin=1em
]{definition}{Definition}[section]
\newmdtheoremenv[
  backgroundcolor=lightgray!10,
  linecolor=black,
  innertopmargin=1em
]{signature}[definition]{Signature}
\newmdtheoremenv[
  backgroundcolor=lightgray!10,
  linecolor=black,
  innertopmargin=1em
]{axiom}[definition]{Axiom}
\newmdtheoremenv[
  backgroundcolor=lightgray!10,
  linecolor=black,
  innertopmargin=1em
]{theorem}[definition]{Theorem}
\newmdtheoremenv[
  backgroundcolor=lightgray!10,
  linecolor=black,
  innertopmargin=1em
]{proposition}[definition]{Proposition}
\newmdtheoremenv[
  backgroundcolor=lightgray!10,
  linecolor=black,
  innertopmargin=1em
]{lemma}[definition]{Lemma}
\newmdtheoremenv[
  backgroundcolor=lightgray!10,
  linecolor=black,
  innertopmargin=1em
]{corollary}[definition]{Corollary}


% Hide imports

\usepackage{scontents}
\newenvsc{imports}[store-env=imports,print-env=false]


% Labels of top-level sections

\renewcommand{\showlabelfont}{\scriptsize\ttfamily\color{gray}}


% Non-symbolic "\implies" and "\iff"

\renewcommand{\implies}{\quad \text{implies} \quad}
\renewcommand{\iff}{\quad \text{iff} \quad}

\usepackage{amssymb}
\usepackage{pgffor}

\newcommand{\id}{\textrm{id}}
\newcommand{\range}{\textrm{range}}
\newcommand{\const}[2]{\textrm{const}_{#1}^{#2}}
\newcommand{\symdiff}{\mathop{\triangle}}
\newcommand{\pow}{\mathcal{P}}
\newcommand{\onto}{\twoheadrightarrow}
\newcommand{\into}{\hookrightarrow}

% E.g. `\set{1, 2, 3, 4}` prints the numbers 1, 2, 3 and 4 as a comma
% separated list which is enclosed within curly braces.
\def\set#1{\ensuremath{
  \{\foreach[count=\i]\x in {#1}{
    \ifnum\i>1,\,\fi
    \x
  }\}
}}

\begin{document}
  \begin{imports}
    \begin{forthel}
      %[prove off][check off]
      [readtex \path{libraries/source/foundations/classes.ftl.tex}]
      %[prove on][check on]
    \end{forthel}
  \end{imports}


  \section*{Maps}

  \subsection*{Range}

  \begin{forthel}
    \begin{definition}\printlabel{FOUNDATIONS_06_4284980337311744}
      Let $f$ be a map.
      A value of $f$ is an object $b$ such that $b = f(a)$ for some $a \in \dom(f)$.
    \end{definition}
  \end{forthel}

  \begin{forthel}
    \begin{definition}\printlabel{FOUNDATIONS_06_1938831225913344}
      Let $f$ be a map.
      The range of $f$ is $\{ f(a) \mid a \in \dom(f) \}$.
    \end{definition}

    Let $\range(f)$ stand for the range of $f$.
  \end{forthel}

  \begin{forthel}
    \begin{proposition}\printlabel{FOUNDATIONS_06_6386349418479616}
      Let $f$ be a map and $b$ be an object.
      $b$ is a value of $f$ iff $b \in \range(f)$.
    \end{proposition}
    \begin{proof}
      Case $b$ is a value of $f$.
        Take $a \in \dom(f)$ such that $b = f(a)$.
        $b$ is an object.
        Hence $b \in \range(f)$.
      End.

      Case $b \in \range(f)$.
        Then $b$ is an object such that $b = f(a)$ for some $a \in \dom(f)$.
        Hence $b$ is a value of $f$.
      End.
    \end{proof}
  \end{forthel}


  \subsection*{Identity Map}

  \begin{forthel}
    \begin{definition}\printlabel{FOUNDATIONS_06_1920902360989696}
      Let $A$ be a class.
      $\id_{A}$ is the map defined on $A$ such that $\id_{A}(a) = a$ for all $a \in A$.
    \end{definition}

    Let the identity map on $A$ stand for $\id_{A}$.
  \end{forthel}


  \subsection*{Composition}

  \begin{forthel}
    \begin{definition}\printlabel{FOUNDATIONS_06_7605717729017856}
      Let $f, g$ be maps.
      Assume $\range(f) \subseteq \dom(g)$.
      $g \circ f$ is the map defined on $\dom(f)$ such that $(g \circ f)(a) = g(f(a))$ for all $a \in \dom(f)$.
    \end{definition}

    Let the composition of $g$ and $f$ stand for $g \circ f$.
  \end{forthel}


  \subsection*{Restriction}

  \begin{forthel}
    \begin{definition}\printlabel{FOUNDATIONS_06_7095412741636096}
      Let $f$ be a map and $X \subseteq \dom(f)$.
      $f \restriction X$ is the map defined on $X$ such that $(f \restriction X)(a) = f(a)$ for all $a \in X$.
    \end{definition}

    Let the restriction of $f$ to $X$ stand for $f \restriction X$.
  \end{forthel}

  \begin{forthel}
    \begin{proposition}\printlabel{FOUNDATIONS_06_2170189258948608}
      Let $A$ be a class and $X \subseteq A$.
      Then $\id_{A} \restriction X = \id_{X}$.
    \end{proposition}
  \end{forthel}


  \subsection*{Image and Preimage}

  \begin{forthel}
    \begin{definition}\printlabel{FOUNDATIONS_06_3038237683613696}
      Let $f$ be a map and $A$ be a class.
      The image of $A$ under $f$ is $\{ f(a) \mid a \in \dom(f) \cap A \}$.
    \end{definition}

    Let the direct image of $A$ under $f$ stand for the image of $A$ under $f$.
    Let $f_{*}(A)$ stand for the image of $A$ under $f$.

    Let $f[A]$ stand for $f_{*}(A)$.
  \end{forthel}

  \begin{forthel}
    \begin{definition}\printlabel{FOUNDATIONS_06_4563167805964288}
      Let $f$ be a map and $B$ be a class.
      The preimage of $B$ under $f$ is $\{ a \in \dom(f) \mid f(a) \in B \}$.
    \end{definition}

    Let the inverse image of $B$ under $f$ stand for the preimage of $B$ under $f$.
    Let $f^{*}(B)$ stand for the preimage of $B$ under $f$.
  \end{forthel}


  \subsection*{Maps Between Classes}

  \begin{forthel}
    \begin{definition}\printlabel{FOUNDATIONS_06_6934038600220672}
      Let $A$ be a class.
      A map of $A$ is a map $f$ such that $\dom(f) = A$.
    \end{definition}

    Let a function of $A$ stand for a map of $A$ that is a function.
  \end{forthel}

  \begin{forthel}
    \begin{definition}\printlabel{FOUNDATIONS_06_7725375157174272}
      Let $B$ be a class.
      A map to $B$ is a map $f$ such that $f(a) \in B$ for each $a \in \dom(f)$.
    \end{definition}

    Let a function to $B$ stand for a map to $B$ that is a function.
  \end{forthel}

  \begin{forthel}
    \begin{definition}\printlabel{FOUNDATIONS_06_2823507398361088}
      Let $A, B$ be classes.
      A map from $A$ to $B$ is a map $f$ such that $\dom(f) = A$ and $f(a) \in B$ for each $a \in A$.
    \end{definition}

    Let $f: A \to B$ stand for $f$ is a map from $A$ to $B$.

    Let a function from $A$ to $B$ stand for a map from $A$ to $B$ that is a function.
  \end{forthel}

  \begin{forthel}
    \begin{definition}\printlabel{FOUNDATIONS_06_3390734908522496}
      Let $A$ be a class.
      A map on $A$ is a map from $A$ to $A$.
    \end{definition}

    Let a function on $A$ stand for a map on $A$ that is a function.
  \end{forthel}

  \begin{forthel}
    \begin{proposition}\printlabel{FOUNDATIONS_06_3312973569327104}
      Let $A, B$ be classes and $f, g : A \to B$.
      Assume that $f(a) = g(a)$ for all $a \in A$.
      Then $f = g$.
    \end{proposition}
  \end{forthel}

  \begin{forthel}
    \begin{proposition}
      Let $A, B$ be classes and $f$ be a map of $A$.
      Assume that $f(a) \in B$ for all $a \in A$.
      Then $f$ is a map from $A$ to $B$ iff $\range(f) \subseteq B$.
    \end{proposition}
  \end{forthel}

  \begin{forthel}
    \begin{proposition}\printlabel{FOUNDATIONS_06_5104361690628096}
      Let $A$ be a class.
      Then $\id_{A}$ is a map on $A$.
    \end{proposition}
  \end{forthel}

  \begin{forthel}
    \begin{proposition}\printlabel{FOUNDATIONS_06_1706446651654144}
      Let $A, B, C$ be classes and $f : A \to B$ and $g : B \to C$.
      Then $g \circ f : A \to C$.
    \end{proposition}
  \end{forthel}

  \begin{forthel}
    \begin{proposition}\printlabel{FOUNDATIONS_06_4078561256275968}
      Let $A, B$ be classes and $f : A \to B$ and $X \subseteq A$.
      Then $f \restriction X : X \to B$.
    \end{proposition}
  \end{forthel}

  \begin{forthel}
    \begin{proposition}\printlabel{FOUNDATIONS_06_3964401904254976}
      Let $A, B$ be classes and $f : A \to B$.
      Then $f \circ \id_{A} = f = \id_{B} \circ f$.
    \end{proposition}
    \begin{proof}
      $A$ is the domain of $f \circ \id_{A}$ and the domain of $f$ and the domain of $\id_{B} \circ f$.
      We have $(f \circ \id_{A})(a)
        = f(\id_{A}(a))
        = f(a)
        = \id_{B}(f(a))
        = (\id_{B} \circ f)(a)$
      for all $a \in A$.
      Hence $f \circ \id_{A}
        = f
        = \id_{B} \circ f$.
    \end{proof}
  \end{forthel}

  \begin{forthel}
    \begin{proposition}\printlabel{FOUNDATIONS_06_3118771061391360}
      Let $A$ be a class and $X \subseteq A$.
      Then $\id_{A} \restriction X = \id_{X}$.
    \end{proposition}
    \begin{proof}
      We have $\dom(\id_{A} \restriction X)
        = X
        = \dom(\id_{X})$.
      $(\id_{A} \restriction X)(a)
        = \id_{A}(a)
        = a
        = \id_{X}(a)$
      for all $a \in X$.
      Hence $\id_{A} \restriction X = \id_{X}$.
    \end{proof}
  \end{forthel}

  \begin{forthel}
    \begin{proposition}\printlabel{FOUNDATIONS_06_6866147389472768}
      Let $A, B, C, D$ be classes and $f : A \to B$ and $g : B \to C$ and $h : C \to D$.
      Then $h \circ (g \circ f) = (h \circ g) \circ f$.
    \end{proposition}
    \begin{proof}
      $h \circ (g \circ f)$ and $(h \circ g) \circ f$ are maps from $A$ to $D$.

      Let us show that $(h \circ (g \circ f))(a) = ((h \circ g) \circ f)(a)$ for all $a \in A$.
        Let $a \in A$.
        Then $(h \circ (g \circ f))(a)
          = h((g \circ f)(a))
          = h(g(f(a)))
          = (h \circ g)(f(a))
          = ((h \circ g) \circ f)(a)$.
      End.

      Hence $h \circ (g \circ f) = (h \circ g) \circ f$.
    \end{proof}
  \end{forthel}


  \subsection*{Classes of Functions}

  \begin{forthel}
    \begin{definition}\printlabel{FOUNDATIONS_06_5119110467813376}
      Let $A, B$ be classes.
      $[A \to B]$ is the class of all functions from $A$ to $B$.
    \end{definition}
  \end{forthel}


  \subsection*{Fixed Points}

  \begin{forthel}
    \begin{definition}\printlabel{FOUNDATIONS_06_2177076576649216}
      Let $f$ be a map.
      A fixed point of $f$ is an element $x$ of $\dom(f)$ such that $f(x) = x$.
    \end{definition}
  \end{forthel}
\end{document}
