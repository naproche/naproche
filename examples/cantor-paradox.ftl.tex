\documentclass{article}
\usepackage{naproche}
\libinput[meta-inf]{example-preamble}
\libinput[libraries]{foundations-preamble}

\title{Cantor's Paradoxes}
\author{\Naproche formalization: \\[0.5em]Marcel Schütz}
\date{2024}

\begin{document}
  \pagenumbering{gobble}
  \maketitle

  \usemodule[libraries]{set-theory?infinite-cardinals.ftl}
  \usemodule[libraries]{foundations?universal-class.ftl}

  \begin{forthel}
    [read \path{burali-forti-paradox.ftl.tex}]
  \end{forthel}

  \noindent \emph{Cantor's First Paradox}, discovered by Georg Cantor in 1897,
  states that the collection $\Card$ of all (infinite) cardinal numbers is a
  proper class \cite[chapter 156]{Cantor1991}.

  \begin{forthel}
    \begin{theorem*}[title=Cantor's First Paradox,id=cantor_paradox_1]
      $\Card$ is a proper class.
    \end{theorem*}
    \begin{proof}
      Suppose that $\Card$ is a set.
      Then $\bigcup \Card$ is a set.

      Let us show that $\bigcup \Card$ contains every ordinal.
        Let $\alpha$ be an ordinal.
        Choose an infinite ordinal $\beta$ such that $\beta \geq \alpha$.
        Choose a cardinal $\kappa$ greater than $\beta$.
        Then $\alpha \in \kappa \in \Card$.
        Hence $\alpha \in \bigcup \Card$.
      End.

      Therefore $\Ord \subseteq \bigcup \Card$.
      Thus $\Ord$ is a set.
      Contradiction.
    \end{proof}
  \end{forthel}

  \emph{Cantor's Second Paradox} denotes the observation that the collection
  $\V$ of all set is a proper class.
  It was shown by Georg Cantor in 1899 via his famous theorem stating that the
  cardinality of any set is strictly smaller than the cardinality of its 
  powerset \cite[chapter 163]{Cantor1991}.

  \begin{forthel}
    \begin{theorem*}[title=Cantor's Second Paradox,id=cantor_paradox_2]
      $\V$ is a proper class.
    \end{theorem*}
    \begin{proof}
      Assume the contrary.
      Then $\pow(\V)$ is a set.
      Hence $\pow(\V)$ is a subset of $\V$.
      Thus $|\pow(\V)| \leq |\V|$.
      Contradiction.
      Indeed $|x| < |\pow(x)|$ for any set $x$.
    \end{proof}
  \end{forthel}

  \printbibliography
  \printlicense[CcByNcSa]{2024}{Marcel Schütz}
\end{document}
