\documentclass{article}
\usepackage{naproche}
\libinput[meta-inf]{example-preamble}
\libinput[libraries]{foundations-preamble}

\title{The Knaster-Tarski fixed point theorem and \\
the Cantor-Schröder-Bernstein Theorem}
\author{\Naproche formalization:\\[0.5em]Alexander Holz, Marcel Schütz}
\date{2024}

\begin{document}
  \maketitle

  \usemodule[libraries]{foundations?equinumerosity.ftl}
  \usemodule[libraries]{foundations?maps-and-systems-of-sets.ftl}
  \usemodule[libraries]{set-theory?zf.ftl}

  \noindent This is a formalization of the \textit{Knaster-Tarski Fixed Point
  Theorem} (i.e. of the assertion that every subset-preserving map has a fixed
  point) and of the \textit{Cantor-Schröder-Bernstein Theorem} (i.e. of the
  fact that two sets are equinumerous iff they can be embedded into each other),
  where the proof of the latter theorem is based on the former one, as in
  \cite{Schroeder2012}.


  \section*{The Knaster-Tarski Fixed Point Theorem}

  \begin{forthel}
    \begin{theorem*}[title=Knaster-Tarski,id=knaster_tarski]
      Let $x$ be a set.
      Let $h$ be a map from $\pow(x)$ to $\pow(x)$ that preserves subsets.
      Then $h$ has a fixed point.
    \end{theorem*}
    \begin{proof}
      (1) Define $A = \{y \in \pow(x) \mid y \subseteq h(y)\}$.
      Then $A$ is a subset of $\pow(x)$.
      We have $\bigcup A \in \pow(x)$.

      Let us show that (2) $\bigcup A \subseteq h(\bigcup A)$.
        Let $u \in \bigcup A$.
        Take $y \in A$ such that $u \in y$.
        Then $u \in h(y)$.
        We have $y \subseteq \bigcup A$.
        Hence $h(y) \subseteq h(\bigcup A)$.
        Indeed $h$ is a subset preserving map between systems of sets and $y, \bigcup A \in \dom(h)$.
        Thus $h(y) \subseteq h(\bigcup A)$.
        Therefore $u \in h(\bigcup A)$.
      End.

      Then $h(\bigcup A) \in A$ (by 1).
      Indeed $h(\bigcup A) \subseteq x$.
      (3) Hence $h(\bigcup A) \subseteq \bigcup A$.
      Indeed every element of $h(\bigcup A)$ is an element of some element of
      $A$.

      Thus $h(\bigcup A) = \bigcup A$ (by 2, 3).
    \end{proof}
  \end{forthel}


  \section*{The Cantor-Schröder-Bernstein Theorem}

  \begin{forthel}
    \begin{theorem*}[title=Cantor-Schröder-Bernstein,id=cantor_schroeder_bernstein]
      Let $x,y$ be sets.
      $x$ and $y$ are equinumerous iff there exists a injective map from $x$ to $y$ and there exists an injective map from $y$ to $x$.
    \end{theorem*}
    \begin{proof}
      Case $x$ and $y$ are equinumerous.
        Take a bijection $f$ between $x$ and $y$.
        Then $f^{-1}$ is a bijection between $y$ and $x$.
        Hence $f$ is an injective map from $x$ to $y$ and $f^{-1}$ is an
        injective map from $y$ to $x$.
      End.

      Case there exists an injective map from $x$ to $y$ and there exists an injective map from $y$ to $x$.
        Take an injective map $f$ from $x$ to $y$.
        Take an injective map $g$ from $y$ to $x$.
        We have $y \setminus f[a] \subseteq y$ for any $a \in \pow(x)$.

        (1) Define $h(a) = x \setminus g[y \setminus f[a]]$ for $a \in \pow(x)$.

        $h$ is a map from $\pow(x)$ to $\pow(x)$.
        Indeed $h(a)$ is a subset of $x$ for each subset $a$ of $x$.

        Let us show that $h$ preserves subsets.
          Let $u, v$ be subsets of $x$.
          Assume $u \subseteq v$.
          Then $f[u] \subseteq f[v]$.
          Hence $y \setminus f[v] \subseteq y \setminus f[u]$.
          Thus $g[y \setminus f[v]] \subseteq g[y \setminus f[u]]$.
          Indeed $y \setminus f[v]$ and $y \setminus f[u]$ are subsets of $y$.
          Therefore $x \setminus g[y \setminus f[u]] \subseteq x \setminus g[y \setminus f[v]]$.
          Consequently $h[u] \subseteq h[v]$.
        End.

        Hence we can take a fixed point $c$ of $h$ (by \nameref{knaster_tarski}).

        (2) Define $F(u) = f(u)$ for $u \in c$.

        We have $c = h(c)$ iff $x \setminus c = g[y \setminus f[c]]$.
        $g^{-1}$ is a bijection between $\range(g)$ and $y$.
        Thus $x \setminus c = g[y \setminus f[c]] \subseteq \range(g)$.
        Therefore $x \setminus c$ is a subset of $\dom(g^{-1})$.

        (3) Define $G(u) = g^{-1}(u)$ for $u \in x \setminus c$.

        $F$ is a bijection between $c$ and $\range(F)$.
        $G$ is a bijection between $x \setminus c$ and $\range(G)$.

        Define \[ H(u) =
          \begin{cases}
            F(u) & : u \in c \\
            G(u) & : u \notin c
          \end{cases} \]
        for $u \in x$.

        Let us show that $H$ is a map to $y$.
          $\dom(H)$ is a set and every value of $H$ is an object.
          Hence $H$ is a map.

          Let us show that every value of $H$ lies in $y$.
            Let $v$ be a value of $H$.
            Take $u \in x$ such that $H(u) = v$.
            If $u \in c$ then $v = H(u) = F(u) = f(u) \in y$.
            If $u \notin c$ then $v = H(u) = G(u) = g^{-1}(u) \in y$.
          End.
        End.

        (4) $H$ is surjective onto $y$.
        Indeed we can show that every element of $y$ is a value of $H$.
          Let $v \in y$.

          Case $v \in f[c]$.
            Take $u \in c$ such that $f(u) = v$.
            Then $F(u) = v$.
          End.

          Case $v \notin f[c]$.
            Then $v \in y \setminus f[c]$.
            Hence $g(v) \in g[y \setminus f[c]]$.
            Thus $g(v) \in x \setminus h(c)$.
            We have $g(v) \in x \setminus c$.
            Therefore we can take $u \in x \setminus c$ such that $G(u) = v$.
            Then $v = H(u)$.
          End.
        End.

        (5) $H$ is injective.
        Indeed we can show that for all $u, v \in x$ if $u \neq v$ then $H(u) \neq H(v)$.
          Let $u,v \in x$.
          Assume $u \neq v$.

          Case $u,v \in c$.
            Then $H(u) = F(u)$ and $H(v) = F(v)$.
            We have $F(u) \neq F(v)$.
            Hence $H(u) \neq H(v)$.
          End.

          Case $u,v \notin c$.
            Then $H(u) = G(u)$ and $H(v) = G(v)$.
            We have $G(u) \neq G(v)$.
            Hence $H(u) \neq H(v)$.
          End.

          Case $u \in c$ and $v \notin c$.
            Then $H(u) = F(u)$ and $H(v) = G(v)$.
            Hence $v \in g[y \setminus F[c]]$.
            We have $G(v) = g^{-1}(v) \in y \setminus F[c]$.
            Indeed $v = g(v')$ for some $v' \in y \setminus F[c]$.
            Thus $G(v) \neq F(u)$.
          End.

          Case $u \notin c$ and $v \in c$.
            Then $H(u) = G(u)$ and $H(v) = F(v)$.
            Hence $u \in g[y \setminus F[c]]$.
            We have $G(u) = g^{-1}(u) \in y \setminus F[c]$.
            Indeed $u = g(u')$ for some $u' \in y \setminus F[c]$.
            Thus $G(u) \neq F(v)$.
          End.
        End.

        Consequently $H$ is a bijection between $x$ and $y$ (by 4, 5).
        Therefore $x$ and $y$ are equinumerous.
      End.
    \end{proof}
  \end{forthel}

  \printbibliography
  \printlicense[CcByNcSa]{2024}{Marcel Schütz}
\end{document}
