\documentclass{article}
\usepackage[utf8]{inputenc}
\usepackage[english]{babel}
\usepackage{meta-inf/lib/naproche}
\usepackage{amssymb}
\usepackage{pgffor}

\newcommand{\id}{\textrm{id}}
\newcommand{\range}{\textrm{range}}
\newcommand{\const}[2]{\textrm{const}_{#1}^{#2}}
\newcommand{\symdiff}{\mathop{\triangle}}
\newcommand{\pow}{\mathcal{P}}
\newcommand{\onto}{\twoheadrightarrow}
\newcommand{\into}{\hookrightarrow}

% E.g. `\set{1, 2, 3, 4}` prints the numbers 1, 2, 3 and 4 as a comma
% separated list which is enclosed within curly braces.
\def\set#1{\ensuremath{
  \{\foreach[count=\i]\x in {#1}{
    \ifnum\i>1,\,\fi
    \x
  }\}
}}

\renewcommand{\succ}{\textrm{succ}}
\newcommand{\pred}{\textrm{pred}}
\newcommand{\V}{\textbf{V}}
\newcommand{\Ord}{\textbf{Ord}}
\newcommand{\Card}{\textbf{Card}}


\usepackage[backend=bibtex]{biblatex}
\usepackage{csquotes}
\addbibresource{meta-inf/lib/bibliography}

\usepackage[
  type=CC,
  modifier=by-nc-sa,
  version=4.0,
]{doclicense}

\title{The Drinker Paradox}
\author{\Naproche formalization:\\[0.5em]Marcel Schütz}
\date{2024}

\begin{document}
  \pagenumbering{gobble}
  \maketitle

  \begin{imports}
    \begin{forthel}
      %[prove off][check off]
      [readtex \path{meta-inf/source/everyday-ontology.ftl.tex}]
      %[prove on][check on]
    \end{forthel}
  \end{imports}

  \noindent The \emph{Drinker Paradox} is a principle of classical predicate 
  logic popularised by the logician Raymond Smullyan in his 1978 book
  \textit{What Is the Name of this Book?} \cite{Smullyan1978} which can be 
  stated as:

  \begin{quotation}
    \noindent There is someone in the pub such that, if he is drinking then 
    everyone in the pub is drinking.
  \end{quotation}

  \begin{forthel}
    \begin{signature*}
      The pub is an object.
    \end{signature*}
    
    \begin{theorem*}[title=Drinker Paradox,id=drinker_paradox]
      Assume that there is a person inside the pub.
      Then there is a person $P$ inside the pub such that if $P$ is drinking then every person inside the pub is drinking.
    \end{theorem*}
    \begin{proof}[by case analysis]
      Case every person inside the pub is drinking.
        Choose a person $P$ inside the pub.
        Then $P$ is drinking and every person inside the pub is drinking.
        Hence if $P$ is drinking then every person inside the pub is drinking.
      End.
    
      Case there is a person inside the pub that is not drinking.
        Consider a person $P$ inside the pub that is not drinking.
        Then if $P$ is drinking then every person inside the pub is drinking.
      End.
    \end{proof}
  \end{forthel}

  \printbibliography

  \section*{Copyright}
  \doclicenseThis
\end{document}
