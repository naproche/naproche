\documentclass{article}

\usepackage[utf8]{inputenc}
\usepackage[english]{babel}
\usepackage{amssymb, mathtools}

\usepackage{../lib/tex/naproche}

\title{König's Theorem}
\author{Steffen Frerix and Peter Koepke\\
University of Bonn}
\date{2018 - 2021}

\newcommand{\card}[1]{\operatorname{card}(#1)}
\newcommand{\Naproche}{$\mathbb{N}$aproche}

\begin{document}

  \maketitle

König's Theorem is an important set-theoretical result about the
arithmetic of cardinals. It was proved by Julius König in 1905
(König, Julius: Zum Kontinuumsproblem, Mathematische Annalen 60 (1905), 177–180).
The proof is reminiscent of Cantor's diagonal argument for
proving that $\kappa < 2^\kappa$.

The formalization begins with a minimal axiomatic setup on sets, functions
and cardinals, sufficient for the proof of the theorem.

On an older laptop
with an Intel Pentium N3710 processor \Naproche{} takes
around 35 seconds to check this text.

\section{ForTheL Setup}

\begin{forthel}
    [synonym cardinal/-s]
    [synonym sequence/-s]
    [read ZFC.ftl]
\end{forthel}


\section{Sets and Functions}

\begin{forthel}
    Let $M,N$ denote sets.

    \begin{definition}
     A subset of $M$ is a set $N$ such that every element
     of $N$ is an element of $M$.
    \end{definition}

    \begin{definition}
      $M \setminus N = \{x \in M | x$ is not an element of $N\}$.
    \end{definition}

    Let $f$ denote a function.
    Let the domain of $f$ stand for $\dom(f)$.

    \begin{definition}
      Assume $M$ is a subset of the domain of $f$.
      $f[M] = \{f(x) | x$ is an element of $M\}$.
    \end{definition}

    Let the image of $f$ stand for $f[\dom(f)]$.

    \begin{axiom}
      Assume $M$ is a subset of the domain of $f$.
      Then $f[M]$ is a set.
    \end{axiom}

    \begin{axiom}
      $f$ is setsized.
    \end{axiom}
\end{forthel}

  \section{Cardinals}

\begin{forthel}
    \begin{signature}
      A cardinal is a set.
    \end{signature}

    Let $A,B,C$ denote cardinals.
    \begin{signature}
      $A < B$ is an atom.
    \end{signature}

    Let $A$ is less than $B$ stand for $A < B$.
    Let $A \leq B$ stand for $A = B$ or $A < B$.

    \begin{axiom}
      $A < B < C \implies A < C$.
    \end{axiom}

    \begin{axiom}
      $A < B$ or $B < A$ or $B = A$.
    \end{axiom}

    \begin{signature}
      The cardinality of $M$ is a cardinal.
    \end{signature}

    Let $\card{M}$ denote the cardinality of $M$.

    \begin{axiom}
      $\card{A} = A$.
    \end{axiom}

    \begin{axiom}[ImageCard]
      Assume $M$ is a subset of $Dom(f)$. $\card{f[M]} \leq \card{M}$.
    \end{axiom}

    \begin{axiom}
      Assume the cardinality of $N$ is less than the cardinality of $M$.
Then $M \setminus N$ has an element.
    \end{axiom}

    \begin{axiom}[SurjExi]
      Assume $\card{M} \leq \card{N}$. Assume $M$ has an element. There exists a
      function $f$ such that $N$ is the domain of $f$ and $M$ is the image of $f$.
    \end{axiom}

\end{forthel}

\section{Sums and Products of cardinals}

\newcommand{\SumSet}[2]{{\dot{\bigcup}}_{i \in #2} {#1}_{i}}
\newcommand{\Sum}[2]{{\sum}_{i \in #2} {#1}_{i}}
\newcommand{\ProdSet}[2]{{\bigtimes}_{i \in #2} {#1}_{i}}
\newcommand{\Prod}[2]{{\prod}_{i \in #2} {#1}_{i}}

  \begin{forthel}
    Let $D$ denote a set.

    \begin{definition}
      A sequence of cardinals on $D$ is a function $\kappa$
      such that $\dom(\kappa) = D$
      and $\kappa(i)$ is a cardinal for every element $i$ of $D$.
    \end{definition}

Let $f_{i}$ stand for $f(i)$.

    \begin{signature}
      Let $\kappa$ be a sequence of cardinals on $D$.
     $\SumSet{\kappa}{D}$ is a set.
    \end{signature}

    \begin{axiom}[SumDef]
      Let $\kappa$ be a sequence of cardinals on $D$. $\SumSet{\kappa}{D} =
      \{(n,i) | i$ is an element of $D$ and $n$ is an element
      of $\kappa_{i}\}$.
    \end{axiom}

    \begin{axiom}
      Let $\kappa$ be a sequence of cardinals on $D$.
      Then $\SumSet{\kappa}{D}$ is a set.
    \end{axiom}

    \begin{definition}
      Let $\kappa$ be a sequence of cardinals on $D$.
      $\Sum{\kappa}{D} = \card{\SumSet{\kappa}{D}}$.
    \end{definition}

    \begin{signature}
      Let $\kappa$ be a sequence of cardinals on $D$.
      $\ProdSet{\kappa}{D}$ is a set.
    \end{signature}

    \begin{axiom}[ProdDef]
      Let $\kappa$ be a sequence of cardinals on $D$. $\ProdSet{\kappa}{D} =
      \{$ function $f | \dom(f) = D \wedge (f(i)$ is an element of
      $\kappa_{i}$ for every
      element $i$ of $D)\}$.
    \end{axiom}

    \begin{axiom}
      Let $\kappa$ be a sequence of cardinals on $D$.
      Then $\ProdSet{\kappa}{D}$ is a set.
    \end{axiom}

    \begin{definition}
      Let $\kappa$ be a sequence of cardinals on $D$.
      $\Prod{\kappa}{D} = \card{\ProdSet{\kappa}{D}}$.
    \end{definition}
\end{forthel}

König's Theorem requires some form of the axiom of choice.
Currently choice is built into Naproche by the {\em choose}
construct in function definitions.

The axiom of choice is also required to show that products
of non-empty factors are themselves non-empty:

\begin{forthel}

    \begin{lemma}[Choice]
      Let $\lambda$ be a sequence of cardinals on $D$.
      Assume that $\lambda_{i}$
     has an element for every element $i$ of $D$.
     Then $\ProdSet{\lambda}{D}$ has an element.
    \end{lemma}
    \begin{proof}
      Define $f(i) =$ choose an element $v$ of
      $\lambda_{i}$ in $v$ for $i$ in $D$.
     Then $f$ is an element of $\ProdSet{\lambda}{D}$.
    \end{proof}

\end{forthel}


\section{König's theorem}

\begin{forthel}
    \begin{theorem}
      Let $\kappa, \lambda$ be sequences of cardinals on $D$.
     Assume that for every element $i$ of $D$
     $\kappa_{i} < \lambda_{i}$.
     Then $$\Sum{\kappa}{D} < \Prod{\lambda}{D}.$$
    \end{theorem}
    \begin{proof}
      Proof by contradiction. Assume the contrary.
      Then $$\Prod{\lambda}{D} \leq \Sum{\kappa}{D}.$$
      Take a function $G$ such that $\SumSet{\kappa}{D}$ is the domain of $G$
      and $\ProdSet{\lambda}{D}$ is the image of $G$.
      Indeed $\ProdSet{\lambda}{D}$ has an element.

      Define

      $\Delta(i) =
     \{G((n,i))(i) | n$ is an element of $\kappa_{i}\}$

      for $i$ in $D$.

      For every element $f$ of $\ProdSet{\lambda}{D}$
      for every element $i$ of $D$
      $f(i)$ is an element of $\lambda_{i}$.
      For every element $i$ of $D$ $\lambda_{i}$ is a set.
      For every element $i$ of $D$ for every element $d$ of $\Delta(i)$
      we have $d\in \lambda_{i}$.
      For every element $i$ of $D$ $\Delta(i)$ is a set.

     (1) For every element $i$ of $D$
      $\card{\Delta(i)} < \lambda_{i}$.

      Proof.
        Let $i$ be an element of $D$.
        Define $F(n) = G((n,i))(i)$ for $n$ in $\kappa_{i}$.
        Then $F[\kappa(i)] = \Delta(i)$.
      end.

      Define
      $f(i) =$ choose an element $v$ of
      $\lambda_{i} \setminus \Delta(i)$ in
      $v$ for $i$ in $D$. Then $f$ is an element of
      $\ProdSet{\lambda}{D}$. Take an element $j$ of $D$ and an
      element $m$ of $\kappa_{j}$ such that $G((m,j)) = f$.
      $G((m,j))(j)$ is an element of $\Delta(j)$ and $f(j)$
      is not an element of $\Delta(j)$. Contradiction.
    \end{proof}
  \end{forthel}




\end{document}
