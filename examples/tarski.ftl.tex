\documentclass{article}

\usepackage[utf8]{inputenc}
\usepackage[english]{babel}
\usepackage{../lib/tex/naproche}

\title{Knaster-Tarski fixed point theorem}
\author{\Naproche{} Formalization: Andrei Paskevich, Steffen Frerix,\\
Erik Sturzenhecker, and Peter Koepke}
\date{2013 -- 2021}

\begin{document}
  \pagenumbering{gobble}

  \maketitle

  The Knaster-Tarski theorem is a result from lattice theory about fixed points
  of monotone functions.
  Bronis\l aw Knaster and Alfred Tarski established it in 1928 for the special
  case of power set lattices (Knaster, B. with Tarski, A.: Un th\' eor\`eme sur
  les fonctions d'ensembles, Ann. Soc. Polon. Math. vol. 6 (1928), 133-134).
  This more general result was stated by Tarski in 1955 (Tarski, A.: A
  lattice-theoretical fixpoint theorem and its applications, Pacific Journal of
  Mathematics vol. 5 (1955), no. 2, 285-309).

  Our formalization checks in about 45 seconds on a modest laptop.

  \section{Preliminaries}
  We first import a small $\mathbb{N}$aproche formalization of an ontology with
  sets, functions, and objects:

  \begin{forthel}
    [readtex preliminaries.ftl.tex]
  \end{forthel}

  \section{Partial Orders}
  We work with a ``global'' order relation $\leq$.

  \begin{forthel}
    Let $x,y,z,u,v,w$ denote elements. Let $S,T$ denote sets.

    \begin{signature}
      $x \leq y$ is a relation.
    \end{signature}

    \begin{axiom}
      $x \leq x$.
    \end{axiom}

    \begin{axiom}
      $x \leq y \leq x \implies x = y$.
    \end{axiom}

    \begin{axiom}
      $x \leq y \leq z \implies x \leq z$.
    \end{axiom}

    \begin{definition}
      Let $S$ be a subset of $T$.
      A lower bound of $S$ in $T$ is an element $u$ of $T$ such that $u \leq x$
      for every $x \in S$.
    \end{definition}

    \begin{definition}
      Let $S$ be a subset of $T$.
      An upper bound of $S$ in $T$ is an element $u$ of $T$ such that $x \leq u$
      for every $x \in S$.
    \end{definition}

    \begin{definition}
      Let $S$ be a subset of $T$.
      An infimum of $S$ in $T$ is an element $u$ of $T$ such that $u$ is a lower
      bound of $S$ in $T$ and for every lower bound $v$ of $S$ in $T$ we have
      $v \leq u$.
    \end{definition}

    \begin{definition}
      Let $S$ be a subset of $T$.
      A supremum of $S$ in $T$ is an element $u$ of $T$ such that $u$ is a upper
      bound of $S$ in $T$ and for every upper bound $v$ of $S$ in $T$ we have
      $u \leq v$.
    \end{definition}

    \begin{lemma}
      Let $S$ be a subset of $T$. Let $u,v$ be suprema of $S$ in $T$.
      Then $u = v$.
    \end{lemma}

    \begin{lemma}
      Let $S$ be a subset of $T$. Let $u,v$ be infima of $S$ in $T$.
      Then $u = v$.
    \end{lemma}

    \begin{definition}
      A complete lattice is a set $S$ such that every subset of $S$ has an
      infimum in $S$ and a supremum in $S$.
    \end{definition}
  \end{forthel}

  \section{Functions}

  \begin{forthel}
    Let $f$ stand for a function.

    \begin{definition}
      A fixed point of $f$ is an element $x$ of $\dom(f)$ such that $f(x) = x$.
    \end{definition}

    \begin{definition}
      $f$ is monotone iff for all $x,y \in \dom(f)$
      $x \leq y \implies f(x) \leq f(y)$.
    \end{definition}
  \end{forthel}

  \section{Knaster-Tarski theorem}

  \begin{forthel}
    Let $f$ stand for a function.

    \begin{theorem}[KnasterTarski]
      Let $U$ be a complete lattice.
      Let $f:U \rightarrow U$ and $f$ be monotone.
      Let $S$ be the class of fixed points of $f$.
      Then $S$ is a complete lattice.
    \end{theorem}
    \begin{proof}
      Let $T$ be a subset of $S$.
      Let us show that $T$ has a supremum in $S$.  Define

        \[ P = \{x \in U \mid \text{$f(x) \leq x$ and $x$ is an upper bound of
        $T$ in $U$} \}. \]

        Take an infimum $p$ of $P$ in $U$.
        $f(p)$ is a lower bound of $P$ in $U$ and an upper bound of $T$ in $U$.
        Hence $p$ is a fixed point of $f$ and a supremum of $T$ in $S$.
      qed.

      Let us show that $T$ has an infimum in $S$. Define

        \[ Q = \{x \in U \mid \text{$x \leq f(x)$ and $x$ is a lower bound of
        $T$ in $U$} \}. \]

        Take a supremum $q$ of $Q$ in $U$.
        $f(q)$ is an upper bound of $Q$ in $U$ and a lower bound of $T$ in $U$.
        Hence $q$ is a fixed point of $f$.
        $q$ is an infimum of $T$ in $S$.
      qed.
    \end{proof}
  \end{forthel}

\end{document}
