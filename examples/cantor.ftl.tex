\documentclass{article}

\usepackage[utf8]{inputenc}
\usepackage[english]{babel}
\usepackage{xurl}
\usepackage{../lib/tex/naproche}
\usepackage{../lib/tex/basicnotions}

\title{Cantor's theorem}
\author{}
\date{}

\begin{document}
  \pagenumbering{gobble}

  \maketitle

  This formalization of Cantor's Theorem is based on the following files:

  \begin{itemize}
    \item \url{$NAPROCHE/examples/basic-notions/sets-and-functions/sections/01_sets/02_powerset.ftl.tex}
    \item \url{$NAPROCHE/examples/basic-notions/sets-and-functions/sections/02_functions/01_functions.ftl.tex}
  \end{itemize}

  \typeout{
    \begin{forthel}
      [readtex basic-notions/sets-and-functions/sections/01_sets/02_powerset.ftl.tex]
      [readtex basic-notions/sets-and-functions/sections/02_functions/01_functions.ftl.tex]
    \end{forthel}
  }

  \begin{forthel}
    \begin{theorem}[Cantor]
      Let $M$ be a set.
      No function of $M$ surjects onto the powerset of $M$.
    \end{theorem}
    \begin{proof}
      Proof by contradiction.
      Assume the contrary.
      Take a function $f$ from $M$ onto $\pow(M)$.
      The value of $f$ at any element of $M$ is a set.
      Define $N = \class{x \in M | \text{$x$ is not an element of $f(x)$}}$.
      $N$ is a subset of $M$.
      Take an element $z$ of $M$ such that $f(z) = N$.
      Then \[ z \in N \iff z \notin f(z) = N. \]
      Contradiction.
    \end{proof}
  \end{forthel}
\end{document}
