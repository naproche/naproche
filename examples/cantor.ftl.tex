\documentclass{article}
\usepackage{naproche}
\libinput[meta-inf]{example-preamble}
\libinput[libraries]{foundations-preamble}

\newcommand\pow{\mathcal P}
\newcommand\card[1]{|#1|}

\title{Cantor's Theorem}
\author{\Naproche formalization: \vspace{0.5em} \\
Marcel Schütz}
\date{2024}

\begin{document}
  \maketitle

  \noindent This is a formalization of \emph{Cantor's Theorem}
  \cite{Cantor1891}, i.e. of the assertion that any set $x$ is strictly smaller
  than its powerset $\pow(x)$
  In an elementary version, this ``strictly smaller than'' relation can be
  formalized via the notion of surjective maps:

  \inputref[libraries]{foundations/cantor.ftl.tex}

  Using the notion of the cardinality $\card{x}$ of a set $x$, the
  ``strictly smaller than'' relation can alternatively be formalized as follows:

  \inputref[libraries]{set-theory/cantor.ftl.tex}

  \printbibliography
  \printlicense[CcByNcSa]{2024}{Marcel Schütz}
\end{document}
