\documentclass{article}
\usepackage{naproche}
\libinput[meta-inf]{example-preamble}
\libinput[libraries]{foundations-preamble}

\newcommand\Ord{\textbf{Ord}}

\title{Zermelo's Well-ordering Theorem}
\author{\Naproche formalization:\\[0.5em]Marcel Schütz}
\date{2024}

\begin{document}
  \maketitle

  \noindent This is a formalization of \textit{Zermelo's Well-ordering Theorem},
  i.e. of the assertion that under the assumption of the axiom of choice every
  set is equinumerous to some ordinal number, where an ordinal number is
  regarded as a transitive set whose elements are transitive sets as well.
  The proof of this theorem presented here is oriented on \cite{Koepke2018}.

  In the following, for any class $A$, we write $A^{< \infty}$ to
  denote the collection of all maps $f : \alpha \to A$ for some ordinal
  $\alpha$.
  Moreover, for any map $G : A^{< \infty} \to A$ we say that a map $F : \Ord \to
  A$, where $\Ord$ denotes the class of all ordinals, is recursive regarding $G$
  if $F(\alpha) = G(F \restriction \alpha)$ for all $\alpha \in \Ord$.

  \inputref[libraries]{set-theory/zermelo.ftl.tex}

  \printbibliography
  \printlicense[CcByNcSa]{2024}{Marcel Schütz}
\end{document}
