\documentclass{article}
\usepackage[english]{babel}
\usepackage{meta-inf/lib/naproche}
\usepackage{amssymb}
\usepackage{pgffor}

\newcommand{\id}{\textrm{id}}
\newcommand{\range}{\textrm{range}}
\newcommand{\const}[2]{\textrm{const}_{#1}^{#2}}
\newcommand{\symdiff}{\mathop{\triangle}}
\newcommand{\pow}{\mathcal{P}}
\newcommand{\onto}{\twoheadrightarrow}
\newcommand{\into}{\hookrightarrow}

% E.g. `\set{1, 2, 3, 4}` prints the numbers 1, 2, 3 and 4 as a comma
% separated list which is enclosed within curly braces.
\def\set#1{\ensuremath{
  \{\foreach[count=\i]\x in {#1}{
    \ifnum\i>1,\,\fi
    \x
  }\}
}}

\renewcommand{\succ}{\textrm{succ}}
\newcommand{\pred}{\textrm{pred}}
\newcommand{\V}{\textbf{V}}
\newcommand{\Ord}{\textbf{Ord}}
\newcommand{\Card}{\textbf{Card}}


\usepackage[backend=bibtex]{biblatex}
\usepackage{csquotes}
\addbibresource{meta-inf/lib/bibliography}

\usepackage[
  type=CC,
  modifier=by-nc-sa,
  version=4.0,
]{doclicense}


\title{Zermelo's Well-ordering Theorem}
\author{\Naproche formalization: Marcel Schütz}
\date{2024}

\begin{document}
  \maketitle

  \noindent This is a formalization of \textit{Zermelo's Well-ordering Theorem},
  i.e. of the assertion that under the assumption of the axiom of choice every
  set is equinumerous to some ordinal number, where an ordinal number is
  regarded as a transitive set whose elements are transitive sets as well.
  The proof of this theorem presented here is oriented on \cite{Koepke2018}.

  \begin{imports}
    \begin{forthel}
      %[prove off][check off]
      [readtex \path{libraries/source/foundations/equinumerosity.ftl.tex}]
      [readtex \path{libraries/source/foundations/choice-maps.ftl.tex}]
      [readtex \path{libraries/source/foundations/proper-classes.ftl.tex}]
      [readtex \path{transfinite-recursion.ftl.tex}]
      %[prove on][check on]
    \end{forthel}
  \end{imports}

  \begin{forthel}
    \begin{axiom*}[Axiom of Choice]
      Let $X$ be a system of nonempty sets.
      Then there exists a choice map for $X$.
    \end{axiom*}
  \end{forthel}

  \noindent In the following, for any class $A$, we write $A^{< \infty}$ to
  denote the collection of all maps $f : \alpha \to A$ for some ordinal
  $\alpha$.
  Moreover, for any map $G : A^{< \infty} \to A$ we say that a map $F : \Ord \to
  A$, where $\Ord$ denotes the class of all ordinals, is recursive regarding $G$
  if $F(\alpha) = G(F \restriction \alpha)$ for all $\alpha \in \Ord$.

  \begin{forthel}
    \begin{theorem*}[Zermelo's Well-Ordering Theorem]\label{zermelo}
      Every set is equinumerous to some ordinal.
    \end{theorem*}
    \begin{proof}
      Let $x$ be a set.
      Consider a choice map $g$ for $\pow(x) \setminus \set{\emptyset}$.
      Take $A = x \cup \set{x}$.
      Every $F \in A^{< \infty}$ is a map from some ordinal to $A$.
      For any $F \in A^{< \infty}$ if $x \setminus \range(F) \neq \emptyset$ then $x \setminus \range(F) \in \dom(g)$.
      Indeed $x \setminus \range(F)$ is a subset of $x$ for any $F \in A^{< \infty}$.
      Define \[ G(F) =
        \begin{cases}
          g(x \setminus \range(F))
          & : x \setminus \range(F) \neq \emptyset
          \\
          x
          & : x \setminus \range(F) = \emptyset
        \end{cases} \]
      for $F \in A^{< \infty}$.
      We can show that for any $F \in A^{< \infty}$ if $x \setminus \range(F) \neq \emptyset$ then $G(F) \in x \setminus \range(F)$.
        Let $F \in A^{< \infty}$.
        Assume $x \setminus \range(F) \neq \emptyset$.  
        Then $G(F) \in x \setminus \range(F)$.
      End.
      $G$ is a map from $A^{< \infty}$ to $A$.
      Indeed we can show that for any $F \in A^{< \infty}$ we have $G(F) \in A$.
        Let $F \in A^{< \infty}$.
        If $x \setminus \range(F) \neq \emptyset$ then $G(F) \in x \setminus \range(F)$.
        If $x \setminus \range(F) = \emptyset$ then $G(F) = x$.
        Hence $G(F) \in A$.
      End.
      Hence we can take a map $F$ from $\Ord$ to $A$ that is recursive regarding $G$.
      For any ordinal $\alpha$ we have $F \restriction \alpha \in A^{< \infty}$.

      For any $\alpha \in \Ord$ we have
      \[ x \setminus F[\alpha] \neq \emptyset \implies F(\alpha) \in x \setminus F[\alpha] \]
      and
      \[ x \setminus F[\alpha] = \emptyset \implies F(\alpha) = x. \]
      Proof.
        Let $\alpha \in \Ord$.
        We have $F[\alpha] = \{ F(\beta) \mid \beta \in \alpha \}$.
        Hence $F[\alpha] = \{ G(F \restriction \beta) \mid \beta \in \alpha \}$.
        We have $\range(F \restriction \alpha) = \{ F(\beta) \mid \beta \in \alpha \}$.
        Thus $\range(F \restriction \alpha) = F[\alpha]$.

        Case $x \setminus F[\alpha] \neq \emptyset$.
          Then $x \setminus \range(F \restriction \alpha) \neq \emptyset$.
          Hence $F(\alpha)
            = G(F \restriction \alpha)
            \in x \setminus \range(F \restriction \alpha)
            = x \setminus F[\alpha]$.
        End.

        Case $x \setminus F[\alpha] \neq \emptyset$.
          Then $x \setminus \range(F \restriction \alpha) = \emptyset$.
          Hence $F(\alpha)
            = G(F \restriction \alpha)
            = x$.
        End.
      Qed.

      (1) For any ordinals $\alpha, \beta$ such that $\alpha < \beta$ and $F(\beta) \neq x$ we have $F(\alpha), F(\beta) \in x$ and $F(\alpha) \neq F(\beta)$. \\
      Proof.
        Let $\alpha, \beta \in \Ord$.
        Assume $\alpha < \beta$ and $F(\beta) \neq x$.
        Then $x \setminus F[\beta] \neq \emptyset$.
        (a) Hence $F(\beta) \in x \setminus F[\beta]$.
        We have $F[\alpha] \subseteq F[\beta]$.
        Thus $x \setminus F[\alpha] \neq \emptyset$.
        (b) Therefore $F(\alpha) \in x \setminus F[\alpha]$.
        Consequently $F(\alpha), F(\beta) \in x$ (by a, b).
        We have $F(\alpha) \in F[\beta]$ and $F(\beta) \notin F[\beta]$.
        Thus $F(\alpha) \neq F(\beta)$.
      Qed.

      (2) There exists an ordinal $\alpha$ such that $F(\alpha) = x$. \\
      Proof.
        Assume the contrary.
        Then $F$ is a map from $\Ord$ to $x$.

        Let us show that $F$ is injective.
          Let $\alpha, \beta \in \Ord$.
          Assume $\alpha \neq \beta$.
          Then $\alpha < \beta$ or $\beta < \alpha$.
          Hence $F(\alpha) \neq F(\beta)$ (by 1).
          Indeed $F(\alpha), F(\beta) \neq x$.
        End.

        Thus $F$ is an injective map from some proper class to some set.
        Contradiction.
      Qed.

      Define $\Phi = \{ \alpha \in \Ord \mid F(\alpha) = x \}$.
      $\Phi$ is nonempty.
      Let us show that we can take an $\alpha \in \Phi$ such that $\alpha \leq \beta$ for all $\beta \in \Phi$.
        Take an $\alpha \in \Phi$ such that for no $\beta \in \Phi$ we have $\beta < \alpha$ (by \printref{FOUNDATIONS_10_9532145874510315}).
        Then $\alpha \leq \beta$ for all $\beta \in \Phi$.
      End.
      Take $f = F \restriction \alpha$.
      Then $f$ is a map from $\alpha$ to $x$.
      Indeed for no $\beta \in \alpha$ we have $F(\beta) = x$.

      (3) $f$ is surjective onto $x$. \\
      Proof.
        $x \setminus F[\alpha] = \emptyset$.
        Hence $\range(f)
          = f[\alpha]
          = F[\alpha]
          = x$.
      Qed.

      (4) $f$ is injective. \\
      Proof.
        Let $\beta, \gamma \in \alpha$.
        Assume $\beta \neq \gamma$.
        We have $f(\beta), f(\gamma) \neq x$.
        Hence $f(\beta) \neq f(\gamma)$ (by 1).
        Indeed $\beta < \gamma$ or $\gamma < \beta$.
      Qed.

      Therefore $f$ is a bijection between $\alpha$ and $x$.
      Consequently $x$ and $\alpha$ are equinumerous.
    \end{proof}
  \end{forthel}

  \printbibliography

  \section*{Copyright}
  \doclicenseThis
\end{document}
