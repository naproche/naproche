\documentclass{article}
\usepackage{naproche}
\usepackage{comment}
\libinput{preamble}
\begin{document}
\begin{smodule}{macros.ftl}
\STEXexport{
  \let\oldin\in
  \let\in\relax
  \let\notin\relax
  \let\oldneq\neq
  \let\neq\relax
  \let\dom\relax
}
\symdef{in}[op=\oldin]{\;\oldin\;}
\symdef{notin}[op=\not\oldin]{\;\not\oldin\;}
\symdef{eq}[op={=}]{\;=\;}
\symdef{neq}[op=\oldneq]{\;\oldneq\;}
\symdef{dom}[args=1,op=\mathrm{dom}]{\comp{\mathrm{dom}}\dobrackets{#1}}
\symdef{pair}[args=2,op=(\cdot,\cdot)]{\comp(#1\comp,\,#2\comp)}
\symdef{apply}[args=ia]{#1\dobrackets{\argsep{#2}{\comp,\,}}}

\begin{comment}
\begin{forthel}
  Let $x\eq y$ stand for $x=y$.
  Let $\pair{x}{y}$ stand for $(x,y)$.
  Let $\dom{f}$ stand for $\dom(f)$.
  Let $\apply{f}{x}$ stand for $f(x)$.
  Let $f(x,y)$ stand for $f((x,y))$.
\end{forthel}
\end{comment}

\section{Macros}

This is a collection of commonly used synonyms for notions hard-coded into
\Naproche. It is intended to be imported to other ForTheL files to provide
these synonyms for them, rather that being considered as a stand-alone file.

\subsection{Equality}

\begin{forthel}
  Let $x$ and $y$ agree stand for $x\eq y$.

  Let $x$ agrees with $y$ stand for $x\eq y$.

  Let $x$ and $y$ are distinct stand for $x\neq y$.
\end{forthel}

\subsection{Elementhood Relation}

\begin{forthel}
  Let a member of $X$ stand for an element of $X$.

  Let $x$ belongs to $X$ stand for $x$ is an element of $X$.

  Let $X$ contains $x$ stand for $x$ is an element of $X$.

  Let $x$ is contained in $X$ stand for $x$ is an element of $X$.

  Let $x$ lies in $X$ stand for $x$ is an element of $X$.

  Let $x$ is in $X$ stand for $x$ is an element of $X$.
\end{forthel}

\subsection{Domain}

\begin{forthel}
  Let the domain of $f$ stand for $\dom{f}$.

  Let $f$ is defined on $X$ stand for $\dom{f}\eq X$.
\end{forthel}

\subsection{Map Application}

\begin{forthel}
  Let the value of $f$ at $x$ stand for $\apply{f}{x}$.

  Let $\apply{f}{x,y}$ stand for $\apply{f}{\pair{x}{y}}$.
\end{forthel}
\end{smodule}
\end{document}
