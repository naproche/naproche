\documentclass{article}
\usepackage{naproche}
\libinput[meta-inf]{example-preamble}

\title{The Barber Paradox}
\author{\Naproche formalization:\\[0.5em]Marcel Schütz}
\date{2024}

\begin{document}
  \pagenumbering{gobble}
  \maketitle

  \usemodule[meta-inf]{everyday-ontology.ftl}

  \noindent The \emph{Barber Paradox} is a puzzle illustrating Russell's
  Paradox, formulated by Bertrand Russell in 1919 \cite[p. 355]{Russell1919}:

  \begin{quotation}
    \noindent You can define the barber as ``one who shaves all those, and 
    those only, who do not shave themselves''.
    The question is, does the barber shave himself?
  \end{quotation}
  
  \noindent In \Naproche we can formalize this paradox as follows.

  \begin{forthel}
    \begin{signature*}
      The barber is a man such that for any man $M$ the barber shaves $M$ iff $M$ does not shave himself.
    \end{signature*}
    
    \begin{theorem*}[title=Barber Paradox,id=barber_paradox]
      The barber shaves himself iff the barber does not shave himself.
    \end{theorem*}
  \end{forthel}

  \printbibliography

  \section*{Copyright}
  \doclicenseThis
\end{document}
