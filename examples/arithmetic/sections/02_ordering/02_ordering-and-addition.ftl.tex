\documentclass[../../arithmetic.tex]{subfiles}

\begin{document}
  \section{Ordering and addition}

  \begin{forthel}
    [readtex \path{arithmetic/sections/02_ordering/01_ordering.ftl.tex}]
  \end{forthel}

  \begin{forthel}
    Let $k, l, m, n$ denote natural numbers.
  \end{forthel}

  \noindent In this section we will briefly study the behaviour of the ordering
  with respect to addition.

  \begin{forthel}
    \begin{proposition}\label{Arithmetic_02_02_179654}
      We have \[ n < m \iff n + k < m + k. \]
    \end{proposition}
    \begin{proof}
      Case $n < m$.
        Take a positive natural number $l$ such that $m = n + l$.
        Then $m + k = (n + l) + k = (n + k) + l$.
        Hence $n + k < m + k$.
      End.

      Case $n + k < m + k$.
        Take a positive natural number $l$ such that $m + k = (n + k) + l$.
        $(n + k) + l = n + (k + l) = n + (l + k) = (n + l) + k$.
        Hence $m + k = (n + l) + k$.
        Thus $m = n + l$.
        Therefore $n < m$.
      End.
    \end{proof}


    \begin{corollary}\label{Arithmetic_02_02_316437}
      We have \[ n < m \iff k + n < k + m. \]
    \end{corollary}
    \begin{proof}
      We have $k + n = n + k$ and $k + m = m + k$.
      Hence $k + n < k + m$ iff $n + k < m + k$.
    \end{proof}


    \begin{corollary}\label{Arithmetic_02_02_143631}
      $n \leq m$ iff $k + n \leq k + m$.
    \end{corollary}

    \begin{corollary}\label{Arithmetic_02_02_598206}
      $n \leq m$ iff $n + k \leq m + k$.
    \end{corollary}
  \end{forthel}
\end{document}
