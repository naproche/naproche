\documentclass[../arithmetic.tex]{subfiles}

\begin{document}
  \chapter{Multiplication}\label{chapter:multiplication}

  \filename{arithmetic/sections/05_multiplication.ftl.tex}

  \begin{forthel}
    %[prove off][check off]

    [readtex \path{arithmetic/sections/05_subtraction.ftl.tex}]

    %[prove on][check on]
  \end{forthel}


  \section{Definition of multiplication}

  \begin{forthel}
    \begin{lemma}\printlabel{ARITHMETIC_06_7897906468093952}
      There exists a $\varphi : \Nat \times \Nat \to \Nat$ such
      that for all $n \in \Nat$ we have $\varphi(n, 0) = 0$ and
      $\varphi(n, m \plus 1) = \varphi(n,m) \plus n$ for any $m \in \Nat$.
    \end{lemma}
    \begin{proof}
      Take $A = \funspace{\Nat}{\Nat}$.
      Define $a(n) = 0$ for $n \in \Nat$.
      Then $A$ is a set and $a \in A$.

      [skipfail on] % Wrong proof task %!!
      Define $f(g) = \fun n \in \Nat. g(n) \plus n$ for $g \in A$.
      [skipfail off]

      Then $f : A \to A$.
      Indeed $f(g)$ is a map from $\Nat$ to $\Nat$ for any $g \in A$.
      Consider a $\psi : \Nat \to A$ such that $\psi$ is recursively defined by
      $a$ and $f$ (by \cref{ARITHMETIC_02_2489427471368192}).
      For any objects $n, m$ we have $(n,m) \in \Nat \times \Nat$ iff
      $n, m \in \Nat$.
      Define $\varphi(n,m) = \psi(m)(n)$ for $(n,m) \in \Nat \times \Nat$.
      Then $\varphi$ is a map from $\Nat \times \Nat$ to $\Nat$.
      Indeed $\varphi(n,m) \in \Nat$ for all $n, m \in \Nat$.

      (1) For all $n \in \Nat$ we have $\varphi(n,0) = 0$. \\
      Proof.
        Let $n \in \Nat$.
        Then $\varphi(n,0)
          = \psi(0)(n)
          = a(0)
          = 0$.
      Qed.

      (2) For all $n, m \in \Nat$ we have $\varphi(n, m \plus 1) =
      \varphi(n,m) \plus n$. \\
      Proof.
        Let $n, m \in \Nat$.
        Then $\varphi(n, m \plus 1)
          = \psi(m \plus 1)(n)
          = f(\psi(m))(n)
          = \psi(m)(n) \plus n
          = \varphi(n,m) \plus n$.
      Qed.
    \end{proof}
  \end{forthel}

  \begin{forthel}
    \begin{lemma}\printlabel{ARITHMETIC_06_2076592937369600}
      Let $\varphi, \varphi' : \Nat \times \Nat \to \Nat$.
      Assume that for all $n \in \Nat$ we have $\varphi(n, 0) = 0$ and
      $\varphi(n, m \plus 1) = \varphi(n,m) \plus n$ for any $m \in \Nat$.
      Assume that for all $n \in \Nat$ we have $\varphi'(n, 0) = 0$ and
      $\varphi'(n, m \plus 1) = \varphi'(n,m) \plus n$ for any $m \in \Nat$.
      Then $\varphi = \varphi'$.
    \end{lemma}
    \begin{proof}
      Define $\Phi = \{ m \in \Nat \mid \varphi(n,m) = \varphi'(n,m)$ for
      all $n \in \Nat \}$.

      (1) $0 \in \Phi$.
      Indeed $\varphi(n,0) = 0 = \varphi'(n,0)$ for all $n \in \Nat$.

      (2) For all $m \in \Phi$ we have $m \plus 1 \in \Phi$. \\
      Proof.
        Let $m \in \Phi$.
        Then $\varphi(n,m) = \varphi'(n,m)$ for all $n \in \Nat$.
        Hence $\varphi(n, m \plus 1)
          = \varphi(n,m) \plus n
          = \varphi'(n,m) \plus n
          = \varphi'(n, m \plus 1)$
        for all $n \in \Nat$.
      Qed.

      Thus $\Phi$ contains every natural number.
      Therefore $\varphi(n,m) = \varphi'(n,m)$ for all $n, m \in \Nat$.
    \end{proof}
  \end{forthel}

  \begin{forthel}
    \begin{definition}\printlabel{ARITHMETIC_06_6626346484629504}
      $\mul$ is the map from $\Nat \times \Nat$ to $\Nat$ such that for all
      $n \in \Nat$ we have $\mul(n, 0) = 0$ and $\mul(n, m \plus 1) =
      \mul(n, m) \plus n$ for any $m \in \Nat$.
    \end{definition}

    Let $n \cdot m$ stand for $\mul(n, m)$.
    Let the product of $n$ and $m$ stand for $n \cdot m$.
  \end{forthel}

  \begin{forthel}
    \begin{lemma}\printlabel{ARITHMETIC_06_1682857820946432}
      Let $n, m$ be natural numbers.
      Then $(n,m) \in \dom(\mul)$.
    \end{lemma}
  \end{forthel}

  \begin{forthel}
    \begin{lemma}\printlabel{ARITHMETIC_06_8420678923452416}
      Let $n, m$ be natural numbers.
      Then $n \cdot m$ is a natural number.
    \end{lemma}
  \end{forthel}

  \begin{forthel}
    \begin{lemma}\printlabel{ARITHMETIC_06_8941041092657152}
      Let $n$ be a natural number.
      Then $n \cdot 0 = 0$.
    \end{lemma}
  \end{forthel}

  \begin{forthel}
    \begin{lemma}\printlabel{ARITHMETIC_06_2211275408932864}
      Let $n, m$ be natural numbers.
      Then $n \cdot (m \plus 1) = (n \cdot m) \plus n$.
    \end{lemma}
  \end{forthel}


  \section{Computation laws}

  \subsection*{Distributivity}

  \begin{forthel}
    \begin{proposition}\printlabel{ARITHMETIC_06_9001524774567936}
      Let $n, m, k$ be natural numbers.
      Then \[ n \cdot (m \plus k) = (n \cdot m) \plus (n \cdot k). \]
    \end{proposition}
    \begin{proof}
      Define $\Phi = \{ k' \in \Nat \mid n \cdot (m \plus k') =
      (n \cdot m) \plus (n \cdot k') \}$.

      (1) $0$ is an element of $\Phi$.
      Indeed $n \cdot (m \plus 0)
        = n \cdot m
        = (n \cdot m) \plus 0
        = (n \cdot m) \plus (n \cdot 0)$.

      (2) For all $k' \in \Phi$ we have $k' \plus 1 \in \Phi$. \\
      Proof.
        Let $k'\in \Phi$.
        Then
        \[  n \cdot (m \plus (k' \plus 1))                  \]
        \[    = n \cdot ((m \plus k') \plus 1)              \]
        \[    = (n \cdot (m \plus k')) \plus n              \]
        \[    = ((n \cdot m) \plus (n \cdot k')) \plus n    \]
        \[    = (n \cdot m) \plus ((n \cdot k') \plus n)    \]
        \[    = (n \cdot m) \plus (n \cdot (k' \plus 1)).   \]
        Hence $n \cdot (m \plus (k' \plus 1)) = (n \cdot m) \plus (n \cdot (k' \plus 1))$.
        Thus $k' \plus 1 \in \Phi$.
      Qed.

      Thus every natural number is contained in $\Phi$.
      Therefore $n \cdot (m \plus k) = (n \cdot m) \plus (n \cdot k)$.
    \end{proof}
  \end{forthel}

  \begin{forthel}
    \begin{proposition}\printlabel{ARITHMETIC_06_5742967566368768}
      Let $n, m, k$ be natural numbers.
      Then \[ (n \plus m) \cdot k = (n \cdot k) \plus (m \cdot k). \]
    \end{proposition}
    \begin{proof}
      Define $\Phi = \{ k' \in \Nat \mid (n \plus m) \cdot k' =
      (n \cdot k') \plus (m \cdot k') \}$.

      (1) $0$ belongs to $\Phi$.
      Indeed $(n \plus m) \cdot 0
        = 0
        = 0 \plus 0
        = (n \cdot 0) \plus (m \cdot 0)$.

      (2) For all $k' \in \Phi$ we have $k' \plus 1 \in \Phi$. \\
      Proof.
        Let $k' \in \Phi$.
        Then
        \[  (n \plus m) \cdot (k' \plus 1)                        \]
        \[    = ((n \plus m) \cdot k') \plus (n \plus m)              \]
        \[    = ((n \cdot k') \plus (m \cdot k')) \plus (n \plus m)   \]
        \[    = (((n \cdot k') \plus (m \cdot k')) \plus n) \plus m   \]
        \[    = ((n \cdot k') \plus ((m \cdot k') \plus n)) \plus m   \]
        \[    = ((n \cdot k') \plus (n \plus (m \cdot k'))) \plus m   \]
        \[    = (((n \cdot k') \plus n) \plus (m \cdot k')) \plus m   \]
        \[    = ((n \cdot k') \plus n) \plus ((m \cdot k') \plus m)   \]
        \[    = (n \cdot (k' \plus 1)) \plus (m \cdot (k' \plus 1)).  \]
        Thus $(n \plus m) \cdot (k' \plus 1) = (n \cdot (k' \plus 1)) \plus (m \cdot (k' \plus 1))$.
      Qed.

      Thus every natural number is an element of $\Phi$.
      Therefore $(n \plus m) \cdot k = (n \cdot k) \plus (m \cdot k)$.
    \end{proof}
  \end{forthel}


  \subsection*{Multiplication with $1$ and $2$}

  \begin{forthel}
    \begin{proposition}\printlabel{ARITHMETIC_06_2910559821365248}
      Let $n$ be a natural number.
      Then \[ n \cdot 1 = n. \]
    \end{proposition}
    \begin{proof}
      $n \cdot 1
        = n \cdot (0 \plus 1)
        = (n \cdot 0) \plus n
        = 0 \plus n
        = n$.
    \end{proof}
  \end{forthel}

  \begin{forthel}
    \begin{corollary}\printlabel{ARITHMETIC_06_5679541582299136}
      Let $n$ be a natural number.
      Then \[ n \cdot 2 = n \plus n. \]
    \end{corollary}
    \begin{proof}
      $n \cdot 2
        = n \cdot (1 \plus 1)
        = (n \cdot 1) \plus n
        = n \plus n$.
    \end{proof}
  \end{forthel}


  \subsection*{Associativity}

  \begin{forthel}
    \begin{proposition}\printlabel{ARITHMETIC_06_347295585402880}
      Let $n, m, k$ be natural numbers.
      Then \[ n \cdot (m \cdot k) = (n \cdot m) \cdot k. \]
    \end{proposition}
    \begin{proof}
      Define $\Phi = \{ k' \in \Nat \mid n \cdot (m \cdot k') =
      (n \cdot m) \cdot k' \}$.

      (1) $0$ is contained in $\Phi$.
      Indeed $n \cdot (m \cdot 0)
        = n \cdot 0
        = 0
        = (n \cdot m) \cdot 0$.

      (2) For all $k' \in \Phi$ we have $k' \plus 1 \in \Phi$. \\
      Proof.
        Let $k' \in \Phi$.
        Then
        \[  n \cdot (m \cdot (k' \plus 1))                          \]
        \[    = n \cdot ((m \cdot k') \plus m)                      \]
        \[    = (n \cdot (m \cdot k')) \plus (n \cdot m)            \]
        \[    = ((n \cdot m) \cdot k') \plus (n \cdot m)            \]
        \[    = ((n \cdot m) \cdot k') \plus ((n \cdot m) \cdot 1)  \]
        \[    = (n \cdot m) \cdot (k' \plus 1).                     \]
      Qed.

      Hence every natural number is contained in $\Phi$.
      Thus $n \cdot (m \cdot k) = (n \cdot m) \cdot k$.
    \end{proof}
  \end{forthel}


  \subsection*{Commutativity}

  \begin{forthel}
    \begin{proposition}\printlabel{ARITHMETIC_06_1764759896588288}
      Let $n, m$ be natural numbers.
      Then \[ n \cdot m = m \cdot n. \]
    \end{proposition}
    \begin{proof}
      Define $\Phi = \{ m' \in \Nat \mid n \cdot m' = m' \cdot n \}$.

      (1) $0$ is contained in $\Phi$. \\
      Proof.
        Define $\Psi = \{ n' \in \Nat \mid n' \cdot 0 = 0 \cdot n' \}$.

        (1a) $0$ is contained in $\Psi$.

        (1b) For all $n' \in \Psi$ we have $n' \plus 1 \in \Psi$. \\
        Proof.
          Let $n' \in \Psi$.
          Then
          \[ (n' \plus 1) \cdot 0
            = 0
            = n' \cdot 0
            = 0 \cdot n'
            = (0 \cdot n') \plus 0
            = 0 \cdot (n' \plus 1). \]
        Qed.

        Hence every natural number is contained in $\Psi$.
        Thus $n \cdot 0 = 0 \cdot n$.
      Qed.

      (2) $1$ belongs to $\Phi$. \\
      Proof.
        Define $\Theta = \{ n' \in \Nat \mid n' \cdot 1 = 1 \cdot n' \}$.

        (2a) $0$ is contained in $\Theta$.

        (2b) For all $n' \in \Theta$ we have $n' \plus 1 \in \Theta$. \\
        Proof.
          Let $n' \in \Theta$.
          Then
          \[  (n' \plus 1) \cdot 1        \]
          \[    = (n' \cdot 1) \plus 1    \]
          \[    = (1 \cdot n') \plus 1    \]
          \[    = 1 \cdot (n' \plus 1).   \]
        Qed.

        Thus every natural number is contained in $\Theta$.
        Therefore $n \cdot 1 = 1 \cdot n$.
      Qed.

      (3) For all $m' \in \Phi$ we have $m' \plus 1 \in \Phi$. \\
      Proof.
        Let $m' \in \Phi$.
        Then
        \[  n \cdot (m' \plus 1)                \]
        \[    = (n \cdot m') \plus (n \cdot 1)  \]
        \[    = (m' \cdot n) \plus (1 \cdot n)  \]
        \[    = (1 \cdot n) \plus (m' \cdot n)  \]
        \[    = (1 \plus m') \cdot n            \]
        \[    = (m' \plus 1) \cdot n.           \]
        Indeed $((1 \cdot n) \plus (m' \cdot n)) = (1 \plus m') \cdot n$. %!
      Qed.

      Hence every natural number is contained in $\Phi$.
      Thus $n \cdot m = m \cdot n$.
    \end{proof}
  \end{forthel}


  \subsection*{Non-existence of zero-divisors}

  \begin{forthel}
    \begin{proposition}\printlabel{ARITHMETIC_06_3843962875936768}
      Let $n, m$ be natural numbers such that $n \cdot m = 0$.
      Then $n = 0$ or $m = 0$.
    \end{proposition}
    \begin{proof}
      Suppose $n, m \neq 0$.
      Take natural numbers $n', m'$ such that $n = (n' \plus 1)$ and $m = (m' \plus 1)$.
      Then
      \[  0                                     \]
      \[    = n \cdot m                         \]
      \[    = (n' \plus 1) \cdot (m' \plus 1)           \]
      \[    = ((n' \plus 1) \cdot m') \plus (n' \plus 1)    \]
      \[    = (((n' \plus 1) \cdot m') \plus n') \plus 1.   \]
      Hence $0 = k \plus 1$ for some natural number $k$.
      Contradiction.
    \end{proof}
  \end{forthel}


  \subsection*{Cancellation}

  \begin{forthel}
    \begin{proposition}\printlabel{ARITHMETIC_06_31055184658432}
      Let $n, m, k$ be natural numbers.
      Assume $k \neq 0$.
      Then \[ n \cdot k = m \cdot k \implies n = m. \]
    \end{proposition}
    \begin{proof}
      Define $\Phi = \{ n' \in \Nat \mid$ for all $m' \in \Nat$ if
      $n' \cdot k = m' \cdot k$ and $k \neq 0$ then $n' = m' \}$.

      (1) $0$ is contained in $\Phi$. \\
      Proof.
        Let $m' \in \Nat$.
        Assume $0 \cdot k = m' \cdot k$ and $k \neq 0$.
        Then $m' \cdot k = 0$.
        Hence $m' = 0$ or $k = 0$.
        Thus $m' = 0$.
      Qed.

      (2) For all $n' \in \Phi$ we have $n' \plus 1 \in \Phi$. \\
      Proof.
        Let $n' \in \Phi$.

        Let us show that for all $m' \in \Nat$ if $(n' \plus 1) \cdot k =
        m' \cdot k$ and $k \neq 0$ then $n' \plus 1 = m'$.
          Let $m' \in \Nat$.
          Assume $(n' \plus 1) \cdot k = m' \cdot k$ and $k \neq 0$.

          Case $m' = 0$.
            Then $(n' \plus 1) \cdot k = 0$.
            Hence $n' \plus 1 = 0$.
            Contradiction.
          End.

          Case $m' \neq 0$.
            Take a natural number $l$ such that $m' = l \plus 1$.
            Then $(n' \plus 1) \cdot k = (l \plus 1) \cdot k$.
            Hence $(n' \cdot k) \plus k
              = (n' \cdot k) \plus (1 \cdot k)
              = (n' \cdot k) \plus k
              = (l \plus 1) \cdot k
              = (l \cdot k) \plus (1 \cdot k)
              = (l \cdot k) \plus k$.
            Thus $n' \cdot k = l \cdot k$.
            Then we have $n' = l$.
            Indeed if $n' \cdot k = l \cdot k$ and $k \neq 0$ then $n' = l$.
            Therefore $n' \plus 1 = l \plus 1 = m'$.
          End.
        End.

        [prover vampire]
        Hence $n' \plus 1 \in \Phi$.
      Qed.

      Thus every natural number is contained in $\Phi$.
      Therefore if $n \cdot k = m \cdot k$ then $n = m$.
    \end{proof}
  \end{forthel}

  \begin{forthel}
    \begin{corollary}\printlabel{ARITHMETIC_06_8575191374364672}
      Let $n, m, k$ be natural numbers.
      Assume $k \neq 0$.
      Then \[ k \cdot n = k \cdot m \implies n = m. \]
    \end{corollary}
    \begin{proof}
      Assume $k \cdot n = k \cdot m$.
      We have $k \cdot n = n \cdot k$ and $k \cdot m = m \cdot k$.
      Hence $n \cdot k = m \cdot k$.
      Thus $n = m$ (by \cref{ARITHMETIC_06_31055184658432}).
    \end{proof}
  \end{forthel}


  \section{Ordering and multiplication}

  \begin{forthel}
    \begin{proposition}\printlabel{ARITHMETIC_06_8817333933965312}
      Let $n, m, k$ be natural numbers.
      Assume $k \neq 0$.
      Then \[ n \less m \iff n \cdot k \less m \cdot k. \]
    \end{proposition}
    \begin{proof}
      Case $n \cdot k \less m \cdot k$.
        Define $\Phi = \{ n' \in \Nat \mid$ if $n' \cdot k \less m \cdot k$ then
        $n' \less m \}$.

        (1) $\Phi$ contains $0$.

        (2) For all $n' \in \Phi$ we have $n' \plus 1 \in \Phi$. \\
        Proof.
          Let $n' \in \Phi$.

          Let us show that if $(n' \plus 1) \cdot k \less m \cdot k$ then $n' \plus 1 \less m$.
            Assume $(n' \plus 1) \cdot k \less m \cdot k$.
            Then $(n' \cdot k) \plus k \less m \cdot k$.
            Hence $n' \cdot k \less m \cdot k$.
            Thus $n' \less m$.
            Then $n' \plus 1 \leq m$.
            If $n' \plus 1 = m$ then $(n' \plus 1) \cdot k = m \cdot k$.
            Hence $n' \plus 1 \less m$.
          End.
        Qed.

        Therefore every natural number is contained in $\Phi$.
        Consequently $n \less m$.
      End.

      Case $n \less m$.
        Take a positive natural number $l$ such that $m = n \plus l$.
        Then $m \cdot k = (n \plus l) \cdot k = (n \cdot k) \plus (l \cdot k)$.
        $l \cdot k$ is positive.
        Hence $n \cdot k \less m \cdot k$.
      End.
    \end{proof}
  \end{forthel}

  \begin{forthel}
    \begin{corollary}\printlabel{ARITHMETIC_06_5048640368279552}
      Let $n, m, k$ be natural numbers.
      Assume $k \neq 0$.
      Then \[ n \less m \iff k \cdot n \less k \cdot m. \]
    \end{corollary}
    \begin{proof}
      We have $k \cdot n = n \cdot k$ and $k \cdot m = m \cdot k$.
      Hence $k \cdot n \less k \cdot m$ iff $n \cdot k \less m \cdot k$.
    \end{proof}
  \end{forthel}

  \begin{forthel}
    \begin{proposition}\printlabel{ARITHMETIC_06_1826268599287808}
      Let $n, m, k$ be natural numbers.
      Then \[ n, m \gtr k \implies n \cdot m \gtr k. \]
    \end{proposition}
    \begin{proof}
      Define $\Phi = \{ n' \in \Nat \mid$ if $n', m \gtr k$ then $n' \cdot m \gtr k \}$.

      (1) $\Phi$ contains $0$.

      (2) For all $n' \in \Phi$ we have $n' \plus 1 \in \Phi$. \\
      Proof.
        Let $n' \in \Phi$.

        Let us show that if $n' \plus 1, m \gtr k$ then $(n' \plus 1) \cdot m \gtr k$.
          Assume $n' \plus 1, m \gtr k$.
          Then $(n' \plus 1) \cdot m = (n' \cdot m) \plus m$.
          If $n' = 0$ then
          $(n' \cdot m) \plus m
            = 0 \plus m
            = m
            \gtr k$.
          If $n' \neq 0$ then
          $(n' \cdot m) \plus m
            \gtr m
            \gtr k$.
          Indeed if $n' \neq 0$ then $n' \cdot m \gtr 0$.
          Indeed $m \gtr 0$.
          Hence $(n' \plus 1) \cdot m \gtr k$.
        Qed.
      Qed.

      Thus every natural number is contained in $\Phi$.
      Therefore if $n, m \gtr k$ then $n \cdot m \gtr k$.
    \end{proof}
  \end{forthel}

  \begin{forthel}
    \begin{corollary}\printlabel{ARITHMETIC_06_1751605544222720}
      Let $n, m, k$ be natural numbers.
      Then \[ n \leq m \implies k \cdot n \leq k \cdot m. \]
    \end{corollary}
  \end{forthel}

  \begin{forthel}
    \begin{corollary}\printlabel{ARITHMETIC_06_3965209318260736}
      Let $n, m, k$ be natural numbers.
      Assume $k \neq 0$.
      Then \[ k \cdot n \leq k \cdot m \implies n \leq m. \]
    \end{corollary}
  \end{forthel}

  \begin{forthel}
    \begin{corollary}\printlabel{ARITHMETIC_06_8946886668976128}
      Let $n, m, k$ be natural numbers.
      Then \[ n \leq m \implies n \cdot k \leq m \cdot k. \]
    \end{corollary}
  \end{forthel}

  \begin{forthel}
    \begin{corollary}\printlabel{ARITHMETIC_06_4374428949413888}
      Let $n, m, k$ be natural numbers.
      Assume $k \neq 0$.
      Then \[ n \cdot k \leq m \cdot k \implies n \leq m. \]
    \end{corollary}
  \end{forthel}

  \begin{forthel}
    \begin{proposition}\printlabel{ARITHMETIC_06_8813409145454592}
      Let $n, m, k$ be natural numbers.
      Assume $m \gtr 0$ and $k \gtr 1$.
      Then $k \cdot m \gtr m$.
    \end{proposition}
    \begin{proof}
      Take a natural number $l$ such that $k = l \plus 2$.
      Then
      \[  k \cdot m                       \]
      \[    = (l \plus 2) \cdot m             \]
      \[    = (l \cdot m) \plus (2 \cdot m)   \]
      \[    = (l \cdot m) \plus (m \plus m)       \]
      \[    = ((l \cdot m) \plus m) \plus m       \]
      \[    = ((l \plus 1) \cdot m) \plus m       \]
      \[    \geq 1 \plus m                    \]
      \[    \gtr m.                          \]
      Indeed $((l \plus 1) \cdot m) \plus m \geq 1 \plus m$.
    \end{proof}
  \end{forthel}


  \section{Multiplication and subtraction}

  \begin{forthel}
    \begin{proposition}\printlabel{ARITHMETIC_06_5458841930039296}
      Let $n, m, k$ be natural numbers such that $n \geq m$.
      Then \[ (n \minus m) \cdot k = (n \cdot k) \minus (m \cdot k). \]
    \end{proposition}
    \begin{proof}
      We have
      \[  ((n \minus m) \cdot k) \plus (m \cdot k)                 \]
      \[    = ((n \minus m) \plus m) \cdot k                       \]
      \[    = n \cdot k                                   \]
      \[    = ((n \cdot k) \minus (m \cdot k)) \plus (m \cdot k).  \]

      Hence $(n \minus m) \cdot k = (n \cdot k) \minus (m \cdot k)$.
    \end{proof}
  \end{forthel}

  \begin{forthel}
    \begin{corollary}\printlabel{ARITHMETIC_06_8461123277815808}
      Let $n, m, k$ be natural numbers such that $n \geq m$.
      Then \[ k \cdot (n \minus m) = (k \cdot n) \minus (k \cdot m). \]
    \end{corollary}
  \end{forthel}
\end{document}
