\documentclass[../../arithmetic.tex]{subfiles}

\begin{document}
  \section{Even and odd numbers}

  \begin{forthel}
    [readtex \path{arithmetic/sections/02_ordering/07_subtraction.ftl.tex}]
  \end{forthel}

  \begin{forthel}
    [readtex \path{arithmetic/sections/03_divisibility/01_divisibility.ftl.tex}]
  \end{forthel}

  \begin{forthel}
    Let $k,l,m,n$ denote natural numbers.
  \end{forthel}


  \subsection{Definition}

  \begin{forthel}
    \begin{definition}
      $n$ is even iff $n$ is divisible by $2$.
    \end{definition}

    \begin{definition}
      $n$ is odd iff $n$ is not divisible by $2$.
    \end{definition}

    \begin{proposition}\label{Arithmetic_03_07_592766}
      $n$ is even iff $n = 2 \cdot m$ for some natural number $m$.
    \end{proposition}
    \begin{proof}
      Case $n$ is even.
        Then $2$ divides $n$.
        Hence $n = 2 \cdot m$ for some natural number $m$.
      End.

      Case $n = 2 \cdot m$ for some natural number $m$.
        Then $2$ divides $n$.
        Hence $2$ is even.
      End.
    \end{proof}

    \begin{proposition}\label{Arithmetic_03_07_275487}
      $n$ is odd iff $n = (2 \cdot m) + 1$ for some natural number $m$.
    \end{proposition}
    \begin{proof}
      Case $n$ is odd.
        Define $P = \class{n' \in \Nat | \text{if $n'$ is odd then $n' = (2 \cdot m) + 1$ for some natural number $m$}}$.

        (BASE CASE) $P$ contains $0$.
        Indeed $0$ is not odd.

        (INDUCTION STEP) For all natural numbers $n'$ we have $n' \in P \implies n' + 1 \in P$. \\
        Proof.
          Let $n'$ be a natural number.
          Assume $n' \in P$.

          Let us show that if $n' + 1$ is odd then $(n' + 1) = (2 \cdot m) + 1$ for some natural number $m$.
            Assume that $n' + 1$ is odd.

            Case $n'$ is even.
              Take a natural number $m$ such that $n' = 2 \cdot m$.
              Then $n' + 1 = (2 \cdot m) + 1$.
            End.

            Case $n'$ is odd.
              Take a natural number $m$ such that $n' = (2 \cdot m) + 1$.
              Then $n' + 1 =
              ((2 \cdot m) + 1) + 1 =
              (2 \cdot m) + (1 + 1) =
              (2 \cdot m) + 2 =
              2 \cdot (m + 1)$.
              Hence $2$ divides $n'$.
              Thus $n'$ is even.
              Contradiction.
            End.
          End.
        Qed.

        Then $P$ contains every natural number.
      End.

      Case $n = (2 \cdot m) + 1$ for some natural number $m$.
        Consider a natural number $m$ such that $n = (2 \cdot m) + 1$.
        Assume that $n$ is even.
        Then we can take a natural number $k$ such that $n = 2 \cdot k$.
        Then we have $2 \cdot k = (2 \cdot m) + 1$.
        Hence $2$ divides $(2 \cdot m) + 1$.
        Thus $2$ divides $1$ (by \ref{Arithmetic_03_01_695362}).
        Indeed $2$ divides $2 \cdot m$.
        Contradiction.
      End.
    \end{proof}

    \begin{proposition}\label{Arithmetic_03_07_164145}
      $n$ is odd iff $n = (2 \cdot m) - 1$ for some positive natural number $m$.
    \end{proposition}
    \begin{proof}
      Case $n$ is odd.
        Consider a natural number $k$ such that $n = (2 \cdot k) + 1$.
        Take $m = k + 1$.
        Then $n =
        (2 \cdot k) + 1 =
        (2 \cdot (k + 0)) + 1 =
        (2 \cdot (k + (1 - 1))) + 1 =
        (2 \cdot ((k + 1) - 1)) + 1 =
        (2 \cdot (m - 1)) + 1 =
        ((2 \cdot m) - (2 \cdot 1)) + 1 =
        ((2 \cdot m) - 2) + 1 =
        (2 \cdot m) - 1$.
      End.

      Case $n = (2 \cdot m) - 1$ for some positive natural number $m$.
        Consider a natural number $m$ such that $n = (2 \cdot m) - 1$.
        Take $k = m - 1$.
        Then $n =
        (2 \cdot m) - 1 =
        (2 \cdot (m + 0)) - 1 =
        (2 \cdot (m + (1 - 1))) - 1 =
        (2 \cdot ((m + 1) - 1)) - 1 =
        (2 \cdot ((m - 1) + 1)) - 1 =
        (2 \cdot (k + 1)) - 1 =
        ((2 \cdot k) + (2 \cdot 1)) - 1 =
        ((2 \cdot k) + 2) - 1 =
        (2 \cdot k) + (2 - 1) =
        (2 \cdot k) + 1$.
        Hence $n$ is odd.
      End.
    \end{proof}
  \end{forthel}


  \subsection{Addition of even and odd numbers}

  \begin{forthel}
    \begin{proposition}\label{Arithmetic_03_07_501002}
      Let $n, m$ be even.
      Then $n + m$ is even.
    \end{proposition}
    \begin{proof}
      Take natural numbers $k, l$ such that $n = 2 \cdot k$ and $m = 2 \cdot l$.
      Then $n + m =
      (2 \cdot k) + (2 \cdot l) =
      2 \cdot (k + l)$.
      Hence $n + m$ is even.
    \end{proof}

    \begin{proposition}\label{Arithmetic_03_07_434363}
      Let $n$ be even and $m$ be odd.
      Then $n + m$ is odd.
    \end{proposition}
    \begin{proof}
      Take natural numbers $k, l$ such that $n = 2 \cdot k$ and $m = (2 \cdot l) + 1$.
      Then $n + m =
      (2 \cdot k) + ((2 \cdot l) + 1) =
      ((2 \cdot k) + (2 \cdot l)) + 1 =
      (2 \cdot (k + l)) + 1$.
      Hence $n + m$ is odd.
    \end{proof}

    \begin{corollary}\label{Arithmetic_03_07_763076}
      Let $n$ be odd and $m$ be even.
      Then $n + m$ is odd.
    \end{corollary}

    \begin{proposition}\label{Arithmetic_03_07_147414}
      Let $n, m$ be odd.
      Then $n + m$ is even.
    \end{proposition}
    \begin{proof}
      Take natural numbers $k, l$ such that $n = (2 \cdot k) + 1$ and $m = (2 \cdot l) + 1$.
      Then $n + m =
      ((2 \cdot k) + 1) + ((2 \cdot l) + 1) =
      (((2 \cdot k) + 1) + (2 \cdot l)) + 1 =
      ((2 \cdot k) + (1 + (2 \cdot l))) + 1 =
      ((2 \cdot k) + ((2 \cdot l) + 1)) + 1 =
      (((2 \cdot k) + (2 \cdot l)) + 1) + 1 =
      ((2 \cdot k) + (2 \cdot l)) + (1 + 1) =
      ((2 \cdot k) + (2 \cdot l)) + 2 =
      (2 \cdot (k + l)) + 2 =
      2 \cdot ((k + l) + 1)$.
      Hence $n + m$ is even.
    \end{proof}
  \end{forthel}


  \subsection{Subtraction of even and odd numbers}

  \begin{forthel}
    \begin{proposition}\label{Arithmetic_03_07_575710}
      Assume $n \geq m$.
      Let $n, m$ be even.
      Then $n - m$ is even.
    \end{proposition}
    \begin{proof}
      Take natural numbers $k, l$ such that $n = 2 \cdot k$ and $m = 2 \cdot l$.
      Then $k \geq l$.
      Hence $n - m =
      (2 \cdot k) - (2 \cdot l) =
      2 \cdot (k - l)$.
      Thus $n - m$ is even.
    \end{proof}

    \begin{proposition}\label{Arithmetic_03_07_458980}
      Assume $n \geq m$.
      Let $n$ be even and $m$ be odd.
      Then $n - m$ is odd.
    \end{proposition}
    \begin{proof}
      Take natural numbers $k, l$ such that $n = 2 \cdot k$ and $m = (2 \cdot l) + 1$.
      Then $k \geq l$.
      Hence $n - m =
      (2 \cdot k) - ((2 \cdot l) + 1) =
      ((2 \cdot k) - (2 \cdot l)) - 1 =
      (2 \cdot (k - l)) - 1$.
      Thus $n - m$ is odd.
    \end{proof}

    \begin{corollary}\label{Arithmetic_03_07_728276}
      Assume $n \geq m$.
      Let $n$ be odd and $m$ be even.
      Then $n - m$ is odd.
    \end{corollary}
    \begin{proof}
      Take natural numbers $k, l$ such that $n = (2 \cdot k) + 1$ and $m = 2 \cdot l$.
      Then $k \geq l$.
      Hence $n - m =
      ((2 \cdot k) + 1) - (2 \cdot l) =
      (1 + (2 \cdot k)) - (2 \cdot l) =
      1 + ((2 \cdot k) - (2 \cdot l)) =
      ((2 \cdot k) - (2 \cdot l)) + 1 =
      (2 \cdot (k - l)) + 1$.
      Indeed $((2 \cdot k) - (2 \cdot l)) = 2 \cdot (k - l)$.
      Thus $n - m$ is odd.
    \end{proof}

    \begin{proposition}\label{Arithmetic_03_07_105734}
      Assume $n \geq m$.
      Let $n, m$ be odd.
      Then $n - m$ is even.
    \end{proposition}
    \begin{proof}
      Take natural numbers $k, l$ such that $n = (2 \cdot k) + 1$ and $m = (2 \cdot l) + 1$.
      Then $k \geq l$.
      Hence $n - m =
      ((2 \cdot k) + 1) - ((2 \cdot l) + 1) =
      ((1 + (2 \cdot k)) - (2 \cdot l)) - 1 =
      (1 + ((2 \cdot k) - (2 \cdot l))) - 1 =
      (1 + (2 \cdot (k - l))) - 1 =
      ((2 \cdot (k - l)) + 1) - 1 =
      2 \cdot (k - l)$.
      Thus $n - m$ is even.
    \end{proof}
  \end{forthel}


  \subsection{Multiplication of even and odd numbers}

  \begin{forthel}
    \begin{proposition}\label{Arithmetic_03_07_673515}
      Let $n$ be even or $n$ be even.
      Then $n \cdot m$ is even.
    \end{proposition}
    \begin{proof}
      Case $n$ is even.
        Take a natural number $k$ such that $n = 2 \cdot k$.
        Then $n \cdot m =
        (2 \cdot k) \cdot m =
        2 \cdot (k \cdot m)$.
        Hence $n \cdot m$ is even.
      End.

      Case $m$ is even.
        Take a natural number $l$ such that $m = 2 \cdot l$.
        Then $n \cdot m =
        n \cdot (2 \cdot l) =
        (n \cdot 2) \cdot l =
        (2 \cdot n) \cdot l =
        2 \cdot (n \cdot l)$.
        Hence $n \cdot m$ is even.
      End.
    \end{proof}

    \begin{proposition}\label{Arithmetic_03_07_349450}
      Let $n, m$ be odd.
      Then $n \cdot m$ is odd.
    \end{proposition}
    \begin{proof}
      Take natural numbers $k, l$ such that $n = (2 \cdot k) + 1$ and $m = (2 \cdot l) + 1$.
      Then $n \cdot m =
      ((2 \cdot k) + 1) \cdot m =
      ((2 \cdot k) \cdot m) + (1 \cdot m) =
      ((2 \cdot k) \cdot m) + m =
      (2 \cdot (k \cdot m)) + m$.
      $2 \cdot (k \cdot m)$ is even and $m$ is odd.
      Hence $(2 \cdot (k \cdot m)) + m$ is odd.
      Therefore $n \cdot m$ is odd.
    \end{proof}
  \end{forthel}


  \subsection{Exponentiation of even and odd numbers}

  \begin{forthel}
    \begin{proposition}\label{Arithmetic_03_07_192104}
      Assume $m > 0$.
      Let $n$ be even.
      Then $n^{m}$ is even.
    \end{proposition}
    \begin{proof}
      Take a natural number $k$ such that $n = 2 \cdot k$.
      Consider a natural number $m'$ such that $m = m' + 1$.
      Then $n^{m} =
      (2 \cdot k)^{m} =
      (2^{m} \cdot (k^{m})) =
      (2^{m' + 1} \cdot (k^{m})) =
      (2^{m'} \cdot 2) \cdot (k^{m}) =
      (2 \cdot 2^{m'}) \cdot (k^{m}) =
      2 \cdot (2^{m'} \cdot (k^{m}))$.
      Hence $n^{m}$ is even.
    \end{proof}

    \begin{proposition}\label{Arithmetic_03_07_330205}
      Let $n$ be odd.
      Then $n^{m}$ is odd.
    \end{proposition}
    \begin{proof}
      Define $P = \class{m' \in \Nat | \text{$n^{m'}$ is odd}}$.

      (BASE CASE) $P$ contains $0$.
      Indeed $n^{0} = 1$ and $1$ is odd.

      (INDUCTION STEP) For all natural numbers $m'$ we have $m' \in P \implies m' + 1 \in P$. \\
      Proof.
        Let $m'$ be a natural number.
        Assume $m' \in P$.
        We have $n^{m' + 1} = n^{m'} \cdot n$.
        $n^{m'}$ is odd.
        Hence we can take a natural number $k$ such that $n^{m'} = (2 \cdot k) + 1$.
        Then $n^{m' + 1} =
        ((2 \cdot k) + 1) \cdot n =
        ((2 \cdot k) \cdot n) + n =
        (2 \cdot (k \cdot n)) + n$.
        $2 \cdot (k \cdot n)$ is even and $n$ is odd.
        Thus $(2 \cdot (k \cdot n)) + n$ is odd.
        Therefore $n^{m' + 1}$ is odd.
        Consequently $m' + 1 \in P$.
      Qed.

      Hence $P$ contains every natural number.
      Thus $n^{m}$ is odd.
    \end{proof}
  \end{forthel}
\end{document}
