\documentclass[../arithmetic.tex]{subfiles}

\begin{document}
  \chapter{Addition}\label{chapter:addition}

  \filename{arithmetic/sections/03_addition.ftl.tex}

  \begin{forthel}
    %[prove off][check off]

    [readtex \path{arithmetic/sections/02_recursion.ftl.tex}]

    %[prove on][check on]
  \end{forthel}


  \section{Definition of addition}

  \begin{forthel}
    \begin{lemma}\printlabel{ARITHMETIC_03_722195546374144}
      There exists a $\varphi : \Nat \times \Nat \to \Nat$ such
      that for all $n \in \Nat$ we have $\varphi(n,0) = n$ and
      $\varphi(n,\succ{m}) = \succ{\varphi(n,m)}$ for all $m \in \Nat$.
    \end{lemma}
    \begin{proof}
      Take $A = \funspace{\Nat}{\Nat}$.
      Define $a(n) = n$ for $n \in \Nat$.
      Then $A$ is a set and $a \in A$.

      [skipfail on] % Wrong proof task %!!
      Define $f(g) = \fun n \in \Nat. \succ{g(n)}$ for $g \in A$.
      [skipfail off]

      Then $f : A \to A$.
      Indeed $f(g)$ is a map from $\Nat$ to $\Nat$ for any $g \in A$.
      Consider a $\psi : \Nat \to A$ such that $\psi$ is recursively defined by
      $a$ and $f$ (by \cref{ARITHMETIC_02_2489427471368192}).
      Define $\varphi(n,m) = \psi(m)(n)$ for $(n,m) \in \Nat \times \Nat$.
      Then $\varphi$ is a map from $\Nat \times \Nat$ to $\Nat$.

      (1) For all $n \in \Nat$ we have $\varphi(n,0) = n$. \\
      Proof.
        Let $n \in \Nat$.
        Then $\varphi(n,0)
          = \psi(0)(n)
          = a(n)
          = n$.
      Qed.

      (2) For all $n, m \in \Nat$ we have $\varphi(n, \succ{m}) =
      \succ{\varphi(n,m)}$. \\
      Proof.
        Let $n, m \in \Nat$.
        Then $\varphi(n, \succ{m})
          = \psi(\succ{m})(n)
          = f(\psi(m))(n)
          = \succ{\psi(m)(n)}
          = \succ{\varphi(n,m)}$.
      Qed.
    \end{proof}
  \end{forthel}

  \begin{forthel}
    \begin{lemma}\printlabel{ARITHMETIC_04_2637025605844992}
      Let $\varphi, \varphi' : \Nat \times \Nat \to \Nat$.
      Assume that for all $n \in \Nat$ we have $\varphi(n,0) = n$ and
      $\varphi(n,\succ{m}) = \succ{\varphi(n,m)}$ for all $m \in \Nat$.
      Assume that for all $n \in \Nat$ we have $\varphi'(n,0) = n$ and
      $\varphi'(n,\succ{m}) = \succ{\varphi'(n,m)}$ for all $m \in \Nat$.
      Then $\varphi = \varphi'$.
    \end{lemma}
    \begin{proof}
      Define $\Phi = \{ m \in \Nat \mid \varphi(n,m) = \varphi'(n,m)$ for
      all $n \in \Nat \}$.

      (1) $0 \in \Phi$.
      Indeed $\varphi(n,0) = n = \varphi'(n,0)$ for all $n \in \Nat$.

      (2) For all $m \in \Phi$ we have $\succ{m} \in \Phi$. \\
      Proof.
        Let $m \in \Phi$.
        Then $\varphi(n,m) = \varphi'(n,m)$ for all $n \in \Nat$.
        Hence $\varphi(n, \succ{m})
          = \succ{\varphi(n,m)}
          = \succ{\varphi'(n,m)}
          = \varphi(n, \succ{m})$
        for all $n \in \Nat$.
      Qed.

      Thus $\Phi$ contains every natural number.
      Therefore $\varphi(n,m) = \varphi'(n,m)$ for all $n, m \in \Nat$.
    \end{proof}
  \end{forthel}

  \begin{forthel}
    \begin{definition}\printlabel{ARITHMETIC_03_4372222701469696}
      $\add$ is the map from $\Nat \times \Nat$ to $\Nat$ such that for all
      $n \in \Nat$ we have $\add(n,0) = n$ and $\add(n,\succ{m}) =
      \succ{\add(n,m)}$ for all $m \in \Nat$.
    \end{definition}

    Let $n \plus m$ stand for $\add(n,m)$.
    Let the sum of $n$ and $m$ stand for $n \plus m$.
  \end{forthel}

  \begin{forthel}
    \begin{lemma}\printlabel{ARITHMETIC_03_3886414804549632}
      Let $n, m$ be natural numbers.
      Then $(n,m) \in \dom(\add)$.
    \end{lemma}
  \end{forthel}

  \begin{forthel}
    \begin{lemma}\printlabel{ARITHMETIC_03_5964925614686208}
      Let $n, m$ be natural numbers.
      Then $n \plus m$ is a natural number.
    \end{lemma}
  \end{forthel}

  \begin{forthel}
    \begin{lemma}\printlabel{ARITHMETIC_03_777009668030464}
      Let $n$ be a natural number.
      Then $\succ{n} = n \plus 1$.
    \end{lemma}
  \end{forthel}

  \begin{forthel}
    \begin{lemma}\printlabel{ARITHMETIC_03_4827955356237824}
      Let $n$ be a natural number.
      Then $n \plus 0 = n$.
    \end{lemma}
  \end{forthel}

  \begin{forthel}
    \begin{lemma}\printlabel{ARITHMETIC_03_1031280145727488}
      Let $n, m$ be natural numbers.
      Then $n \plus (m \plus 1) = (n \plus m) \plus 1$.
    \end{lemma}
  \end{forthel}


  \section{The Peano axioms and recursion, revisited}

  \begin{forthel}
    \begin{proposition}\printlabel{ARITHMETIC_03_3170769680990208}
      Let $n, m$ be natural numbers.
      If $n \plus 1 = m \plus 1$ then $n = m$.
    \end{proposition}
  \end{forthel}

  \begin{forthel}
    \begin{proposition}\printlabel{ARITHMETIC_03_1101538491629568}
      Let $n$ be a natural number.
      Then $n \plus 1 \neq 0$.
    \end{proposition}
  \end{forthel}

  \begin{forthel}
    \begin{proposition}[Induction]\printlabel{ARITHMETIC_03_647949900054528}
      Let $A$ be a class.
      Assume $0 \in A$.
      Assume that for all $n \in \Nat$ if $n \in A$ then $n \plus 1 \in A$.
      Then $A$ contains every natural number.
    \end{proposition}
  \end{forthel}

  \begin{forthel}
    \begin{proposition}
      Let $a$ be an object and $f$ be a map.
      Let $\varphi$ be a map from $\Nat$ to $\dom(f)$.
      $\varphi$ is recursively defined by $a$ and $f$ iff $\varphi(0) = a$ and
      $\varphi(n \plus 1) = f(\varphi(n))$ for every $n \in \Nat$.
    \end{proposition}
  \end{forthel}


  \section{Computation laws}

  \subsection*{Associativity}

  \begin{forthel}
    \begin{proposition}\printlabel{ARITHMETIC_03_3235893452210176}
      Let $n, m, k$ be natural numbers.
      Then \[ n \plus (m \plus k) = (n \plus m) \plus k. \]
    \end{proposition}
    \begin{proof}
      Define $\Phi = \{ k' \in \Nat \mid n \plus (m \plus k') = (n \plus m) \plus k' \}$.

      (1) $0$ is contained in $\Phi$.
      Indeed $n \plus (m \plus 0) = n \plus m = (n \plus m) \plus 0$.

      (2) For all $k' \in \Phi$ we have $k' \plus 1 \in \Phi$. \\
      Proof.
        Let $k' \in \Phi$.
        Then $n \plus (m \plus k') = (n \plus m) \plus k'$.
        Hence
        \[  n \plus (m \plus (k' \plus 1))        \]
        \[    = n \plus ((m \plus k') \plus 1)    \]
        \[    = (n \plus (m \plus k')) \plus 1    \]
        \[    = ((n \plus m) \plus k') \plus 1    \]
        \[    = (n \plus m) \plus (k' \plus 1).   \]
        Thus $k' \plus 1 \in \Phi$.
      Qed.

      Thus every natural number is an element of $\Phi$.
      Therefore $n \plus (m \plus k) = (n \plus m) \plus k$.
    \end{proof}
  \end{forthel}


  \subsection*{Commutativity}

  \begin{forthel}
    \begin{proposition}\printlabel{ARITHMETIC_03_4029553232052224}
      Let $n, m$ be natural numbers.
      Then \[ n \plus m = m \plus n. \]
    \end{proposition}
    \begin{proof}
      Define $\Phi = \{ m' \in \Nat \mid n \plus m' = m' \plus n \}$.

      (1) $0$ is an element of $\Phi$. \\
      Proof.
        Define $\Psi = \{ n' \in \Nat \mid n' \plus 0 = 0 \plus n' \}$.

        (1a) $0$ belongs to $\Psi$.

        (1b) For all $n' \in \Psi$ we have $n' \plus 1 \in \Psi$. \\
        Proof.
          Let $n' \in \Psi$.
          Then $n' \plus 0 = 0 \plus n'$.
          Hence
          \[  (n' \plus 1) \plus 0        \]
          \[    = n' \plus 1          \]
          \[    = (n' \plus 0) \plus 1    \]
          \[    = (0 \plus n') \plus 1    \]
          \[    = 0 \plus (n' \plus 1).   \]
        Qed.

        Hence every natural number belongs to $\Psi$.
        Thus $n \plus 0 = 0 \plus n$.
        Therefore $0$ is an element of $\Phi$.
      Qed.

      Let us show that (2) $n \plus 1 = 1 \plus n$. \\
      Proof.
        Define $\Theta = \{ n' \in \Nat \mid n' \plus 1 = 1 \plus n' \}$.

        (2a) $0$ is an element of $\Theta$.

        (2b) For all $n' \in \Theta$ we have $n' \plus 1 \in \Theta$. \\
        Proof.
          Let $n' \in \Theta$.
          Then $n' \plus 1 = 1 \plus n'$.
          Hence
          \[  (n' \plus 1) \plus 1        \]
          \[    = (1 \plus n') \plus 1    \]
          \[    = 1 \plus (n' \plus 1).   \]
          Thus $n' \plus 1 \in \Theta$.
        Qed.

        Thus every natural number belongs to $\Theta$.
        Therefore $n \plus 1 = 1 \plus n$.
      Qed.

      (3) For all $m' \in \Phi$ we have $m' \plus 1 \in \Phi$. \\
      Proof.
        Let $m' \in \Phi$.
        Then $n \plus m' = m' \plus n$.
        Hence
        \[  n \plus (m'  \plus 1)       \]
        \[    = (n \plus m') \plus 1    \]
        \[    = (m' \plus n) \plus 1    \]
        \[    = m' \plus (n \plus 1)    \]
        \[    = m' \plus (1 \plus n)    \]
        \[    = (m' \plus 1) \plus n.   \]
        Thus $m' \plus 1 \in \Phi$.
      Qed.

      Thus every natural number is an element of $\Phi$.
      Therefore $n \plus m = m \plus n$.
    \end{proof}
  \end{forthel}


  \subsection*{Cancellation}

  \begin{forthel}
    \begin{proposition}\printlabel{ARITHMETIC_03_3137702874578944}
      Let $n, m, k$ be natural numbers.
      Then \[ n \plus k = m \plus k \implies n = m. \]
    \end{proposition}
    \begin{proof}
      Define $\Phi = \{ k' \in \Nat \mid$ if $n \plus k' = m \plus k'$ then $n = m \}$.

      (1) $0$ is an element of $\Phi$.

      (2) For all $k' \in \Phi$ we have $k' \plus 1 \in \Phi$. \\
      Proof.
        Let $k' \in \Phi$.
        Suppose $n \plus (k' \plus 1) = m \plus (k' \plus 1)$.
        Then $(n \plus k') \plus 1 = (m \plus k') \plus 1$.
        Hence $n \plus k' = m \plus k'$.
        Thus $n = m$.
      Qed.

      Therefore every natural number is an element of $\Phi$.
      Consequently if $n \plus k = m \plus k$ then $n = m$.
    \end{proof}
  \end{forthel}

  \begin{forthel}
    \begin{corollary}\printlabel{ARITHMETIC_03_8445946379632640}
      Let $n, m, k$ be natural numbers.
      Then \[ k \plus n = k \plus m \implies n = m. \]
    \end{corollary}
    \begin{proof}
      Assume $k \plus n = k \plus m$.
      We have $k \plus n = n \plus k$ and $k \plus m = m \plus k$.
      Hence $n \plus k = m \plus k$.
      Thus $n = m$.
    \end{proof}
  \end{forthel}


  \subsection*{Zero sums}

  \begin{forthel}
    \begin{proposition}\printlabel{ARITHMETIC_03_3520602170195968}
      Let $n, m$ be natural numbers.
      If $n \plus m = 0$ then $n = 0$ and $m = 0$.
    \end{proposition}
    \begin{proof}
      Assume $n \plus m = 0$.
      Suppose $n \neq 0$ or $m \neq 0$.
      Then we can take a $k \in \Nat$ such that $n = k \plus 1$ or $m = k \plus 1$.
      Hence there exists a natural number $l$ such that
      $n \plus m
        = l \plus (k \plus 1)
        = (l \plus k) \plus 1
        \neq 0$.
      Contradiction.
    \end{proof}
  \end{forthel}
\end{document}
