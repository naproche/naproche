\documentclass[../arithmetic.tex]{subfiles}

\begin{document}
  \chapter{Subtraction}\label{chapter:subtraction}

  \filename{arithmetic/sections/05_subtraction.ftl.tex}

  \begin{forthel}
    %[prove off][check off]

    [readtex \path{arithmetic/sections/04_ordering.ftl.tex}]

    %[prove on][check on]
  \end{forthel}

  \begin{forthel}
    \begin{definition}\printlabel{ARITHMETIC_05_8878757276286976}
      Let $n, m$ be natural numbers such that $n \geq m$.
      $n \minus m$ is the natural number $k$ such that $n = m \plus k$.
    \end{definition}

    Let the difference of $n$ and $m$ stand for $n \minus m$.
  \end{forthel}

  \begin{forthel}
    \begin{proposition}\printlabel{ARITHMETIC_05_874271710642176}
      Let $n, m$ be natural numbers such that $n \geq m$.
      Then \[ n \minus m = 0 \iff n = m. \]
    \end{proposition}
    \begin{proof}
      Case $n \minus m = 0$.
        Then $n
          = (n \minus m) \plus m
          = 0 \plus m
          = m$.
      End.

      Case $n = m$.
        We have $(n \minus m) \plus m
          = n
          = m
          = 0 \plus m$.
        Hence $n \minus m = 0$.
      End.
    \end{proof}
  \end{forthel}

  \begin{forthel}
    \begin{corollary}\printlabel{ARITHMETIC_05_8457713057005568}
      Let $n$ be a natural number.
      Then \[ n \minus n = 0. \]
    \end{corollary}
  \end{forthel}

  \begin{forthel}
    \begin{proposition}\printlabel{ARITHMETIC_05_8518521570983936}
      Let $n$ be a natural number.
      Then \[ n \minus 0 = n. \]
    \end{proposition}
    \begin{proof}
      We have $n
        = (n \minus 0) \plus 0
        = n \minus 0$.
    \end{proof}
  \end{forthel}

  \begin{forthel}
    \begin{proposition}\printlabel{ARITHMETIC_05_4222566117933056}
      Let $n, m$ be natural numbers such that $n \geq m$.
      Then \[ n \minus m \leq n. \]
    \end{proposition}
    \begin{proof}
      We have $(n \minus m) \plus m = n$.
      Hence $n \minus m \leq n$.
    \end{proof}
  \end{forthel}

  \begin{forthel}
    \begin{proposition}\printlabel{ARITHMETIC_05_1269537257291776}
      Let $n, m$ be natural numbers such that $n \geq m$.
      Then \[ 0 \neq m \less n \implies n \minus m \less n. \]
    \end{proposition}
    \begin{proof}
      Assume $0 \neq m \less n$.
      Suppose $n \minus m \geq n$.
      We have $(n \minus m) \plus m = n$.
      Then $n \plus m
        = (n \minus m) \plus m
        = n
        = n \plus 0$.
      Hence $m = 0$.
      Contradiction.
    \end{proof}
  \end{forthel}

  \begin{forthel}
    \begin{proposition}\printlabel{ARITHMETIC_05_4767595811045376}
      Let $n, m, k$ be natural numbers such that $n \geq m$.
      Then \[ (n \minus m) \plus k = (n \plus k) \minus m. \]
    \end{proposition}
    \begin{proof}
      We have
      \[  ((n \minus m) \plus k) \plus m       \]
      \[    = ((n \minus m) \plus m) \plus k   \]
      \[    = n \plus k               \]
      \[    = ((n \plus k) \minus m) \plus m.  \]

      Hence $(n \minus m) \plus k = (n \plus k) \minus m$.
    \end{proof}
  \end{forthel}

  \begin{forthel}
    \begin{corollary}\printlabel{ARITHMETIC_05_7578468875239424}
      Let $n, m, k$ be natural numbers such that $n \geq m$.
      Then \[ k \plus (n \minus m) = (k \plus n) \minus m. \]
    \end{corollary}
  \end{forthel}

  \begin{forthel}
    \begin{proposition}\printlabel{ARITHMETIC_05_7595909347016704}
      Let $n, m, k$ be natural numbers such that $n \geq  m \plus k$.
      Then \[ (n \minus m) \minus k = n \minus (m \plus k). \]
    \end{proposition}
    \begin{proof}
      We have
      \[  ((n \minus m) \minus k) \plus (m \plus k)       \]
      \[    = (((n \minus m) \minus k) \plus k) \plus m   \]
      \[    = (n \minus m) \plus m               \]
      \[    = n                         \]
      \[    = (n \minus (m \plus k)) \plus (m \plus k).  \]

      Hence $(n \minus m) \minus k = n \minus (m \plus k)$.
    \end{proof}
  \end{forthel}
\end{document}
