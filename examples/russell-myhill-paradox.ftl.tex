\documentclass{article}
\usepackage[utf8]{inputenc}
\usepackage[english]{babel}
\usepackage{../lib/tex/naproche}
\usepackage{amssymb}
\usepackage{pgffor}

\newcommand{\id}{\textrm{id}}
\newcommand{\range}{\textrm{range}}
\newcommand{\const}[2]{\textrm{const}_{#1}^{#2}}
\newcommand{\symdiff}{\mathop{\triangle}}
\newcommand{\pow}{\mathcal{P}}
\newcommand{\onto}{\twoheadrightarrow}
\newcommand{\into}{\hookrightarrow}

% E.g. `\set{1, 2, 3, 4}` prints the numbers 1, 2, 3 and 4 as a comma
% separated list which is enclosed within curly braces.
\def\set#1{\ensuremath{
  \{\foreach[count=\i]\x in {#1}{
    \ifnum\i>1,\,\fi
    \x
  }\}
}}

\renewcommand{\succ}{\textrm{succ}}
\newcommand{\pred}{\textrm{pred}}
\newcommand{\V}{\textbf{V}}
\newcommand{\Ord}{\textbf{Ord}}
\newcommand{\Card}{\textbf{Card}}


\usepackage[backend=bibtex]{biblatex}
\usepackage{csquotes}
\addbibresource{meta-inf/lib/bibliography}

\usepackage[
  type=CC,
  modifier=by-nc-sa,
  version=4.0,
]{doclicense}

\title{The Russell-Myhill Paradox}
\author{\Naproche formalization: \vspace{0.5em} \\
Marcel Schütz}
\date{2024}

\begin{document}
  \pagenumbering{gobble}
  \maketitle

  \begin{imports}
    \begin{forthel}
      %[prove off][check off]
      [readtex \path{meta-inf/source/everyday-ontology.ftl.tex}]
      %[prove on][check on]
    \end{forthel}
  \end{imports}

  \noindent The \emph{Russell-Myhill Paradox} is a paradox discovered by 
  Bertrand Russell which states that two systems of propositions with the same 
  \emph{logical product} are not necessarily equal.
  It was discussed in Russell's 1903 \emph{Principles of Mathematics}
  \cite[Appendix B]{Russell1903} and rediscovered in 1958 by John Myhill
  \cite{Myhill1958}.

  \begin{forthel}
    \begin{signature*}
      Let $P$ be a system of propositions.
      The logical product of $P$ is a proposition $p$ such that $p$ is true iff every element of $P$ is true.
    \end{signature*}
    
    Let $\sqcap P$ denote the logical product of $P$.
    
    \begin{theorem*}[Russell-Myhill Paradox]\label{russell_myhill_paradox}
      It is wrong that
      \[ \sqcap P = \sqcap Q \implies P = Q \]
      for all systems of propositions $P, Q$.
    \end{theorem*}
    \begin{proof}
      Assume the contrary.
      Define $Q = \{q \mid$ there exists a system of propositions $P$ such that $q = \sqcap P$ and $q \notin P\}$.
      Consider $q = \sqcap Q$.
      Then for any system of propositions $P$ such that $\sqcap q = P$ we have $P = Q$.
      Hence $q \in Q$ iff $q \notin Q$.
      Contradiction.
    \end{proof}
  \end{forthel}

  \printbibliography

  \section*{Copyright}
  \doclicenseThis
\end{document}
