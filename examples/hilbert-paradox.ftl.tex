\documentclass{article}
\usepackage[utf8]{inputenc}
\usepackage[english]{babel}
\usepackage{meta-inf/lib/naproche}
\usepackage{amssymb}
\usepackage{pgffor}

\newcommand{\id}{\textrm{id}}
\newcommand{\range}{\textrm{range}}
\newcommand{\const}[2]{\textrm{const}_{#1}^{#2}}
\newcommand{\symdiff}{\mathop{\triangle}}
\newcommand{\pow}{\mathcal{P}}
\newcommand{\onto}{\twoheadrightarrow}
\newcommand{\into}{\hookrightarrow}

% E.g. `\set{1, 2, 3, 4}` prints the numbers 1, 2, 3 and 4 as a comma
% separated list which is enclosed within curly braces.
\def\set#1{\ensuremath{
  \{\foreach[count=\i]\x in {#1}{
    \ifnum\i>1,\,\fi
    \x
  }\}
}}

\renewcommand{\succ}{\textrm{succ}}
\newcommand{\pred}{\textrm{pred}}
\newcommand{\V}{\textbf{V}}
\newcommand{\Ord}{\textbf{Ord}}
\newcommand{\Card}{\textbf{Card}}


\usepackage[backend=bibtex]{biblatex}
\usepackage{csquotes}
\addbibresource{meta-inf/lib/bibliography}

\title{Hilbert's Paradox}
\author{\Naproche formalization: \vspace{0.5em} \\
Marcel Schütz}
\date{2024}

\begin{document}
  \pagenumbering{gobble}
  \maketitle

  \begin{imports}
    \begin{forthel}
      %[prove off][check off]
      [readtex \path{libraries/source/set-theory/zf.ftl.tex}]
      [readtex \path{libraries/source/foundations/universal-class.ftl.tex}]
      [readtex \path{libraries/source/foundations/proper-classes.ftl.tex}]
      [readtex \path{libraries/source/foundations/closure-under-arbitrary-unions.ftl.tex}]
      [readtex \path{libraries/source/foundations/closure-under-powersets.ftl.tex}]
      %[prove on][check on]
    \end{forthel}
  \end{imports}

  \noindent \emph{Hilbert's Paradox}, discovered around 1900 by David Hilbert, 
  demonstates that there cannot exists a set that is closed under powerset and 
  under union of arbitrary subset \cite{PeckhausKahl2002}.

  \begin{forthel}
    \begin{theorem*}[Hilbert's Paradox]\label{hilbert_paradox}
      There exists no system of sets that is closed under powersets and closed under unions.
    \end{theorem*}
    \begin{proof}
      Assume the contrary.
      Consider a system of sets $S$ that is closed under powersets and closed under unions.
      We have $S \subseteq S$.
      Hence $\bigcup S \in S$.
      Thus $\pow(\bigcup S) \in S$.
      Contradiction.
    \end{proof}
  \end{forthel}

  \noindent Using Hilbert's Paradox it can further be shown that there exists 
  no universal set.

  \begin{forthel}
    \begin{corollary*}\label{hilbert_paradox_corollary}
      $\V$ is a proper class.
    \end{corollary*}
    \begin{proof}
      Assume the contrary.
      Then $\V$ is closed under powersets and closed under unions.
      Contradiction (by \nameref{hilbert_paradox}).
    \end{proof}
  \end{forthel}

  \printbibliography
\end{document}
