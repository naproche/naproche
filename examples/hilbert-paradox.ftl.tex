\documentclass{article}
\usepackage{naproche}
\libinput[meta-inf]{example-preamble}
\libinput[libraries]{set-theory-preamble}

\newcommand\V{\mathbb V}

\title{Hilbert's Paradox}
\author{\Naproche formalization:\\[0.5em]Marcel Schütz}
\date{2024}

\begin{document}
  \pagenumbering{gobble}
  \maketitle

  \usemodule[libraries]{foundations?universal-class.ftl}

  \noindent \emph{Hilbert's Paradox}, discovered around 1900 by David Hilbert, 
  demonstates that there cannot exists a set that is closed under powersets and 
  under unions of arbitrary subsets \cite{PeckhausKahl2002}.

  \inputref[libraries]{set-theory/hilbert-paradox.ftl.tex}

  \noindent Using Hilbert's Paradox it can further be shown that there exists 
  no universal set, i.e. that the class $\V$ of all sets is a proper class:

  \begin{corollary*}[forthel,id=hilbert_paradox_corollary]
    $\V$ is a proper class.
  \end{corollary*}
  \begin{proof}[forthel]
    Assume the contrary.
    Then $\V$ is closed under powersets and closed under unions.
    Contradiction (by \nameref{hilbert_paradox}).
  \end{proof}

  \printbibliography
  \printlicense[CcByNcSa]{2024}{Marcel Schütz}
\end{document}
