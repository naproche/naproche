\documentclass{article}
\usepackage{naproche}
\libinput[meta-inf]{example-preamble}
\libinput[libraries]{foundations-preamble}

\title{Hilbert's Paradox}
\author{\Naproche formalization:\\[0.5em]Marcel Schütz}
\date{2024}

\begin{document}
  \pagenumbering{gobble}
  \maketitle

  \usemodule[libraries]{set-theory?zf.ftl}
  \usemodule[libraries]{foundations?universal-class.ftl}
  \usemodule[libraries]{foundations?proper-classes.ftl}
  \usemodule[libraries]{foundations?closure-under-arbitrary-unions.ftl}
  \usemodule[libraries]{foundations?closure-under-powersets.ftl}

  \noindent \emph{Hilbert's Paradox}, discovered around 1900 by David Hilbert, 
  demonstates that there cannot exists a set that is closed under powerset and 
  under union of arbitrary subset \cite{PeckhausKahl2002}.

  \begin{forthel}
    \begin{theorem*}[title=Hilbert's Paradox,id=hilbert_paradox]
      There exists no system of sets that is closed under powersets and closed under unions.
    \end{theorem*}
    \begin{proof}
      Assume the contrary.
      Consider a system of sets $S$ that is closed under powersets and closed under unions.
      We have $S \subseteq S$.
      Hence $\bigcup S \in S$.
      Thus $\pow(\bigcup S) \in S$.
      Contradiction.
    \end{proof}
  \end{forthel}

  \noindent Using Hilbert's Paradox it can further be shown that there exists 
  no universal set.

  \begin{forthel}
    \begin{corollary*}[id=hilbert_paradox_corollary]
      $\V$ is a proper class.
    \end{corollary*}
    \begin{proof}
      Assume the contrary.
      Then $\V$ is closed under powersets and closed under unions.
      Contradiction (by \nameref{hilbert_paradox}).
    \end{proof}
  \end{forthel}

  \printbibliography
  \printlicense[CcByNcSa]{2024}{Marcel Schütz}
\end{document}
