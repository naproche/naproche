\documentclass{article}
\usepackage[english]{babel}
\usepackage{../lib/tex/naproche}
\usepackage{amssymb}
\usepackage{pgffor}

\newcommand{\id}{\textrm{id}}
\newcommand{\range}{\textrm{range}}
\newcommand{\const}[2]{\textrm{const}_{#1}^{#2}}
\newcommand{\symdiff}{\mathop{\triangle}}
\newcommand{\pow}{\mathcal{P}}
\newcommand{\onto}{\twoheadrightarrow}
\newcommand{\into}{\hookrightarrow}

% E.g. `\set{1, 2, 3, 4}` prints the numbers 1, 2, 3 and 4 as a comma
% separated list which is enclosed within curly braces.
\def\set#1{\ensuremath{
  \{\foreach[count=\i]\x in {#1}{
    \ifnum\i>1,\,\fi
    \x
  }\}
}}

\renewcommand{\succ}{\textrm{succ}}
\newcommand{\pred}{\textrm{pred}}
\newcommand{\V}{\textbf{V}}
\newcommand{\Ord}{\textbf{Ord}}
\newcommand{\Card}{\textbf{Card}}


\usepackage[backend=bibtex]{biblatex}
\usepackage{csquotes}
\addbibresource{meta-inf/lib/bibliography}

\newcommand{\surjects}{\twoheadrightarrow}

\title{Regularity of successor cardinals}
\author{\Naproche formalization: \vspace{0.5em} \\
Steffen Frerix (2018), \\
Adrian De Lon (2021)}
\date{}

\begin{document}
  \maketitle

  \noindent This is a formalization of a theorem of Felix Hausdorff stating that
  successor cardinals are always regular.

  \begin{imports}
    \begin{forthel}
      %[prove off][check off]
      [readtex \path{libraries/source/set-theory/cardinals-and-maps.ftl.tex}]
      [readtex \path{libraries/source/set-theory/successor-cardinals.ftl.tex}]
      [readtex \path{libraries/source/set-theory/regular-cardinals.ftl.tex}]
      [readtex \path{libraries/source/foundations/constant-maps.ftl.tex}]
      %[prove on][check on]
    \end{forthel}
  \end{imports}

  In lack of a formalization in \Naproche of a proof of the fact that
  $|X \times X| = |X|$ for every set $X$ which is required by the subsequent
  proof of Hausdorff's Theorem, we add an axiom that ensures that this fact
  holds in our theory:

  \begin{forthel}
    \begin{axiom*}
      $|X \times X| = |X|$ for every set $X$.
    \end{axiom*}
  \end{forthel}

  The following result appears in \cite[p.~443]{Hausdorff1908},
  where Hausdorff mentions that the proof of the regularity of successor
  cardinals is \textit{``ganz einfach''} (\textit{``very simple''})
  and can be skipped.

  \begin{forthel}
    \begin{theorem*}[Hausdorff]
      Let $\kappa$ be a cardinal.
      Then $\kappa^+$ is regular.
    \end{theorem*}
    \begin{proof}[by contradiction]
      Assume the contrary.
      Take a cofinal subset $x$ of $\kappa^+$ such that $|x| \neq \kappa^+$.
      Then $|x| \leq \kappa$.
      Take a surjective map $f$ from $\kappa$ onto $x$ (by \printref{SET_THEORY_06_192336220913664}).
      Indeed $x$ and $\kappa$ are nonempty and $|\kappa| = \kappa$.
      Then $f(\xi) \in \kappa^+$ for all $\xi \in \kappa$.

      Let us show that for all $z \in \kappa^+$ if $z$ is nonempty then there exists a surjective map from $\kappa$ onto $z$.
        Let $z \in \kappa^+$.
        Assume that $z$ is nonempty.
        $\kappa$ is nonempty.
        Hence the thesis (by \printref{SET_THEORY_06_192336220913664}).
        Indeed $|\kappa| \geq |z|$.
      End.

      Define \[ g(z) =
        \begin{cases}
          \text{choose $h : \kappa \onto z$ in $h$}
          & : \text{$z$ has an element}
          \\
          \text{$\const{\kappa}{0}$}
          & : \text{$z$ has no element}
        \end{cases}
      \] for $z$ in $\kappa^+$.

      Let us show that for all $\xi, \zeta \in \kappa$ $g(f(\xi))$ is a map such that $\zeta \in \dom(g(f(\xi)))$.
        Let $\xi, \zeta \in \kappa$.
        If $f(\xi)$ has an element then $g(f(\xi))$ is a surjective map from $\kappa$ onto $f(\xi)$.
        If $f(\xi)$ has no element then $g(f(\xi)) = \const{\kappa}{0}$.
        Hence $\dom(g(f(\xi))) = \kappa$.
        Therefore $\zeta \in \dom(g(f(\xi)))$.
      End.

      For all objects $\xi, \zeta$ we have $\xi, \zeta \in \kappa$ iff $(\xi, \zeta) \in \kappa \times \kappa$.
      Define $h(\xi,\zeta) = g(f(\xi))(\zeta)$ for $(\xi,\zeta) \in \kappa \times \kappa$.

      Let us show that $h$ is surjective onto $\kappa^+$.

        Every element of $\kappa^+$ is an element of $h[\kappa \times \kappa]$. \\
        Proof.
          Let $n$ be an element of $\kappa^+$.
          Take an element $\xi$ of $\kappa$ such that $n < f(\xi)$.
          Take an element $\zeta$ of $\kappa$ such that $g(f(\xi))(\zeta) = n$.
          Indeed $g(f(\xi))$ is a surjective map from $\kappa$ onto $f(\xi)$.
          Then $n = h(\xi,\zeta)$.
        End.

        Every element of $h[\kappa \times \kappa]$ is an element of $\kappa^+$. \\
        Proof.
          Let $n$ be an element of $h[\kappa \times \kappa]$.
          We can take elements $a, b$ of $\kappa$ such that $n = h(a,b)$.
          Then $n = g(f(a))(b)$.
          Indeed $(a,b) \in \kappa \times \kappa$ and $h(a,b) = g(f(a))(b)$.
          $f(a)$ is an element of $\kappa^+$.
          Every element of $f(a)$ is an element of $\kappa^+$.

          Case $f(a)$ has an element.
            Then $g(f(a))$ is a surjective map from $\kappa$ onto $f(a)$.
            Hence $n \in f(a) \in \kappa^+$.
            Thus $n \in \kappa^+$.
          End.

          Case $f(a)$ has no element.
            Then $g(f(a)) = \const{\kappa}{0}$.
            Hence $n$ is the empty set.
            Thus $n \in \kappa^+$.
          End.
        End.

        Hence $\range(h) = h[\kappa \times \kappa] = \kappa^+$.
      End.

      Therefore $|\kappa^+| \leq |\kappa \times \kappa|$ (by \printref{SET_THEORY_06_192336220913664}).
      Indeed $\kappa \times \kappa$ and $\kappa^+$ are nonempty sets and $h$ is a surjective map from $\kappa \times \kappa$ to $\kappa^+$.
      Consequently $\kappa^+ \leq \kappa$.
      Contradiction.
    \end{proof}
  \end{forthel}

  \printbibliography
\end{document}
