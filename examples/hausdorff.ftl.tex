\documentclass{article}
\usepackage[english]{babel}
\usepackage{xr}
\usepackage{../lib/tex/naproche}
\usepackage{amssymb}
\usepackage{pgffor}

\newcommand{\id}{\textrm{id}}
\newcommand{\range}{\textrm{range}}
\newcommand{\const}[2]{\textrm{const}_{#1}^{#2}}
\newcommand{\symdiff}{\mathop{\triangle}}
\newcommand{\pow}{\mathcal{P}}
\newcommand{\onto}{\twoheadrightarrow}
\newcommand{\into}{\hookrightarrow}

% E.g. `\set{1, 2, 3, 4}` prints the numbers 1, 2, 3 and 4 as a comma
% separated list which is enclosed within curly braces.
\def\set#1{\ensuremath{
  \{\foreach[count=\i]\x in {#1}{
    \ifnum\i>1,\,\fi
    \x
  }\}
}}

\renewcommand{\succ}{\textrm{succ}}
\newcommand{\pred}{\textrm{pred}}
\newcommand{\V}{\textbf{V}}
\newcommand{\Ord}{\textbf{Ord}}
\newcommand{\Card}{\textbf{Card}}


\usepackage[backend=bibtex]{biblatex}
\usepackage{csquotes}
\addbibresource{meta-inf/lib/bibliography}

\externaldocument{set-theory/set-theory}

\newcommand{\surjects}{\twoheadrightarrow}

\title{Regularity of successor cardinals}
\author{\Naproche formalization: \vspace{0.5em} \\
Steffen Frerix (2018), \\
Adrian De Lon (2021)}
\date{}

\begin{document}
  \maketitle

  \noindent This is a formalization of a theorem of Felix Hausdorff stating that
  successor cardinals are always regular.
  On mid-range hardware \Naproche needs approximately 1:30 Minutes to verify it,
  plus approximately 25 minutes to verify the library files it depends on.


  \section{Preliminaries}

  \begin{forthel}
    %[prove off][check off]

    [readtex \path{libraries/source/set-theory/06_cardinals.ftl.tex}]

    %[prove on][check on]
  \end{forthel}

  \begin{forthel}
    Let $X$ denote a set.
    Let $\kappa$ denote a cardinal.

    \begin{axiom*}
      $|X \times X| = |X|$.
    \end{axiom*}

    \begin{signature*}
      $\kappa^{+}$ is a cardinal such that $\kappa < \kappa^{+}$ and there is no
      cardinal $\nu$ such that $\kappa < \nu < \kappa^{+}$.
    \end{signature*}

    \begin{axiom*}
      $|\alpha| \leq \kappa$ for every element $\alpha$ of $\kappa^{+}$.
    \end{axiom*}

    \begin{definition*}
      The constant zero on $X$ is the function $f$ such that $\dom(f) = X$ and
      $f(x) = 0$ for every $x \in X$.
    \end{definition*}

    Let $0^{X}$ stand for the constant zero on $X$.
  \end{forthel}


  \section{Cofinality and regular cardinals}

  \begin{forthel}
    Let $\kappa$ denote a cardinal.

    \begin{definition*}[Cofinality]
      Let $Y$ be a subset of $\kappa$.
      $Y$ is cofinal in $\kappa$ iff for every element $x$ of $\kappa$ there
      exists an element $y$ of $Y$ such that $x < y$.
    \end{definition*}

    Let a cofinal subset of $\kappa$ stand for a subset of $\kappa$ that is
    cofinal in $\kappa$.

    \begin{definition*}
      $\kappa$ is regular iff $|x| = \kappa$ for every cofinal subset $x$ of
      $\kappa$.
    \end{definition*}
  \end{forthel}


  \section{Hausdorff's theorem}

  The following result appears in \cite[p.~443]{Hausdorff1908},
  where Hausdorff mentions that the proof is
  \textit{``ganz einfach''} (\textit{``very simple''}) and can be skipped.

  \begin{forthel}
    Let $\kappa$ denote a cardinal.

    \begin{theorem*}[Hausdorff]
      $\kappa^{+}$ is regular.
    \end{theorem*}
    \begin{proof}[by contradiction]
      Assume the contrary.
      Take a cofinal subset $x$ of $\kappa^{+}$ such that $|x| \neq \kappa^{+}$.
      Then $|x| \leq \kappa$.
      Take a surjective map $f$ from $\kappa$ onto $x$
      (by \ref{SET_THEORY_06_192336220913664}).
      Indeed $x$ and $\kappa$ are nonempty and $|\kappa| = \kappa$.
      Then $f(\xi) \in \kappa^{+}$ for all $\xi \in \kappa$.
      For all $z \in \kappa^{+}$ if $z$ has an element then there exists a
      surjective map $h$ from $\kappa$ onto $z$.
      Indeed $\kappa$ is nonempty.

      Define \[ g(z) =
        \begin{cases}
          \text{choose $h : \kappa \onto z$ in $h$}
          & : \text{$z$ has an element}
          \\
          \text{$0^{\kappa}$}
          & : \text{$z$ has no element}
        \end{cases}
      \] for $z$ in $\kappa^{+}$.

      Let us show that for all $\xi, \zeta \in \kappa$ $g(f(\xi))$ is a map such
      that $\zeta \in \dom(g(f(\xi)))$.
        Let $\xi, \zeta \in \kappa$.
        If $f(\xi)$ has an element then $g(f(\xi))$ is a surjective map from
        $\kappa$ onto $f(\xi)$.
        If $f(\xi)$ has no element then $g(f(\xi)) = 0^{\kappa}$.
        Hence $\dom(g(f(\xi))) = \kappa$.
        Therefore $\zeta \in \dom(g(f(\xi)))$.
      End.

      For all objects $\xi, \zeta$ we have $\xi, \zeta \in \kappa$ iff
      $(\xi, \zeta) \in \kappa \times \kappa$.
      Define $h(\xi,\zeta) = g(f(\xi))(\zeta)$ for $(\xi,\zeta) \in \kappa
      \times \kappa$.

      Let us show that $h$ is surjective onto $\kappa^{+}$.

        Every element of $\kappa^{+}$ is an element of $h[\kappa \times
        \kappa]$. \\
        Proof.
          Let $n$ be an element of $\kappa^{+}$.
          Take an element $\xi$ of $\kappa$ such that $n < f(\xi)$.
          Take an element $\zeta$ of $\kappa$ such that $g(f(\xi))(\zeta) = n$.
          Indeed $g(f(\xi))$ is a surjective map from $\kappa$ onto $f(\xi)$.
          Then $n = h(\xi,\zeta)$.
        End.

        Every element of $h[\kappa \times \kappa]$ is an element of
        $\kappa^{+}$. \\
        Proof.
          Let $n$ be an element of $h[\kappa \times \kappa]$.
          We can take elements $a, b$ of $\kappa$ such that $n = h(a,b)$.
          Then $n = g(f(a))(b)$.
          $f(a)$ is an element of $\kappa^{+}$.
          Every element of $f(a)$ is an element of $\kappa^{+}$.

          Case $f(a)$ has an element.
            Then $g(f(a))$ is a surjective map from $\kappa$ onto $f(a)$.
            Hence $n \in f(a) \in \kappa^{+}$.
            Thus $n \in \kappa^{+}$.
          End.

          Case $f(a)$ has no element.
            Then $g(f(a)) = 0^{\kappa}$.
            Hence $n$ is the empty set.
            Thus $n \in \kappa^{+}$.
          End.
        End.

        Hence $\range(h) = h[\kappa \times \kappa] = \kappa^{+}$.
      End.

      Therefore $|\kappa^{+}| \leq |\kappa \times \kappa|$
      (by \ref{SET_THEORY_06_192336220913664}).
      Indeed $\kappa \times \kappa$ and $\kappa^{+}$ are nonempty sets and $h$
      is a surjective map from $\kappa \times \kappa$ to $\kappa^{+}$.
      Consequently $\kappa^{+} \leq \kappa$.
      Contradiction.
    \end{proof}
  \end{forthel}

  \printbibliography
\end{document}
