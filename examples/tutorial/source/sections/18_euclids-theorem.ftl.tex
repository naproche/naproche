\documentclass{stex}
\libinput{preamble}
\begin{document}

\section{Euclid's Theorem}

Now everything is in place for the proof that there
are infinitely many prime numbers.
\begin{forthel}

\begin{signature} $\Primes$ is the class of prime natural numbers.
\end{signature}

% \begin{lemma} Contradiction. \end{lemma}


\begin{theorem}[Euclid]
$\Primes$ is infinite.
\end{theorem}
\begin{proof}
Assume that $r$ is a natural number and
$p$ is a sequence of length $r$ and
$\Finset{p}{1}{r}$ is a subclass of $\Primes$.

(1) $p_{i}$ is a nonzero natural number for every $i \in \dom(p)$.

Consider $n=\Prod{p}{1}{r}+1$.
$\Prod{p}{1}{r}$ is nonzero.
Hence $n$ is nontrivial. Take a prime divisor $q$ of $n$.

Let us show that $q \neq p_{i}$ for all natural numbers $i$ such that
$1 \leq i \leq r$.

Proof by contradiction.
Assume the contrary.
Take a natural number $i$ such that $1 \leq i \leq r$ and $q=p_{i}$.
$q$ is a divisor of $n$ and $q$ is a divisor of $\Prod{p}{1}{r}$
(by Factorproperty,1).
Thus $q$ divides $1$. Contradiction. qed.

Hence $\Finset{p}{1}{r}$ is not the class of prime natural numbers.
\end{proof}
\end{forthel}

\end{document}
