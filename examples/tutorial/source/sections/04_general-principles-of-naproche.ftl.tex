\documentclass{stex}
\libinput{preamble}
\begin{document}

\section{General Principles of \Naproche}

\subsection{Natural Language Processing}

\Naproche interprets ForTheL input texts in first-order logic
and also proposes proof methods to be used by the internal reasoner
of \Naproche{} or by the external ATP.
ForTheL leverages a number of natural language
mechanisms to capture formal content in a compact, user-friendly
and natural way. This corresponds to usual natural language
features, where the phrase ``white horse that belongs
to Mary'' with its adjective, noun and relative sentence
corresponds to a first-order statement like
$$horse(x) \wedge white(x) \wedge property-of(x,Mary)$$
with a (hidden) variable $x$, predicates $horse()$,
$white()$, and $property-of( , )$, and a constant $Mary$.
\Naproche{} extracts this formal context whilst reading
the input sentence by sentence. Previous sentences provide the context
of already introduced language components, in which the
new sentence is to be interpreted.

Indeed the ``white horse'' can be formalized in ForTheL. We use
the ASCII format, since we are mainly developing a language.

\begin{verbatim}
Signature. A human being is a notion.
Signature. Mary is a human being.
Signature. An animal is a notion.
Signature. A horse is an animal.

Let A denote an animal. Let H denote a human being.
Signature. A is white is an atom.
Signature. A belongs to H is an atom.
\end{verbatim}
From these premisses we can show the following triviality:
\begin{verbatim}
Theorem. Every white horse that belongs to Mary
is white.
\end{verbatim}

\begin{exercise}
Inspect this natural language formalization: what is the
meaning of ``notion'' and ``atom''. Why can't you replace
``Every'' by ``Some'' in the Theorem. What is the meaning
of the error message for the theorem ``Mary is white''?
Change the example so that horses and Mary can all be
potentially white.
\end{exercise}

\begin{exercise}
Formalize the classical syllogism
\begin{quotation}
All men are mortal.\\
Socrates is a man.\\
Therefore, Socrates is mortal.
\end{quotation}
in \Naproche.
\end{exercise}


\subsection{Axiomatic Approach}

The \Naproche{} system
comes with a minimal set of in-built mathematical notions.
Usually one has to explicitly extend the first-order
language through Signature and Definition commands and through
Axioms. Then Lemmas and Theorems can be postulated and proved
with familiar proof structures. In the following this procedure
is explained along a standard proof of the infinitude of
prime numbers:
\begin{itemize}
\item
set up a language and axioms for natural number arithmetic;
\item
define divisibility and prime natural numbers;
\item
introduce some set theory so that one can define
finite sets, sequences and products.
\end{itemize}
Finally, a checked natural language proof of Euclid's theorem
can be carried out in this axiomatic setup.

\end{document}
