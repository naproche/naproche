\documentclass{stex}
\libinput{preamble}
\begin{document}

\section{Division}

Prime (natural) numbers are defined via divisibility.

\begin{forthel}

\begin{definition}
  $n$ divides $m$ iff for some $l$ $m = n * l$.
\end{definition}

Let $x | y$ denote $x$ divides $y$.
Let a divisor of $x$ denote a natural number
that divides $x$.

\begin{lemma} Assume $l | m | n$. Then $l | n$.
\end{lemma}

\end{forthel}

The definition is similar to the definition of $\leq$.
Note, however, the possible syntactic variations:
``there exists a natural number $l$ such that
$m = n * l$''; it is also possible
to put the quantifier after the property:
``$n$ divides $m$ iff $m = n * l$ for some $l$''.

Natural language has many mechanisms for putting
information into sentences in a compact, un-formalistic way.
Un-formalistic means, e.g., that natural language
does not normally use brackets (...) in speech.
``The syntax and semantics of the ForTheL language'' by Andrei
Paskevich is still a good guide to most ForTheL constructs, although
the language has been extended since then.

\end{document}
