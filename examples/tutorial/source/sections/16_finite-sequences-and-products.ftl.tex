\documentclass{stex}
\libinput{preamble}
\begin{document}

\section{Finite Sequences and Products}

We shall use natural numbers and bounded intervals
of natural numbers to ``count'' the elements of sets.

\begin{forthel}
\begin{definition} $\Seq{m}{n} = \{ i \in \mathbb{N} |
m \leq i \leq n \}$.
\end{definition}
\end{forthel}

This translates to
\begin{verbatim}
forall v0 ((HeadTerm :: v0 = \Seq{m}{n}) iff ((aClass(v0) and
(aSet(\mathbb{N}) implies aSet(v0))) and
forall v1 (aElementOf(v1,v0) iff ((aElementOf(v1,\mathbb{N})
and (m\leq v1 and v1\leq n)) and aObject(v1)))))
\end{verbatim}

Note that the second line of the translation contains an instance
of the axiom of separation: if the class that bounds
the variable $i$ is a set then
the resulting class is a set. Thus:

\begin{forthel}
\begin{lemma} $\Seq{m}{n}$ is a set.
\end{lemma}
\end{forthel}

The notation $\Seq{m}{n}$ indicates the notational and
typographical potential
for \Naproche{} texts if one uses appropriate \LaTeX macros.

$\Seq{m}{n}$ is printed using the macro

\verb+\newcommand{\Seq}[2]{\{#1,\dots,#2\}}+

The macro is used in the source code of the above definition:

\begin{verbatim}
\begin{definition} $\Seq{m}{n} = \{ i \in \mathbb{N} |
m \leq i \leq n \}$.
\end{definition}
\end{verbatim}

\Naproche{} reads this as the introduction of a binary function
pattern \verb+\Seq{m}{n}+ as one can see from the first-order translation.

In this way, intuitive and customary $\dots$-notations which may seem
vague can be given an unambiguous meaning and can be used
in \Naproche{} formalizations.

These techniques are used in further definitions and notations for finite sequences.

\begin{forthel}

\begin{definition} A sequence of length $n$ is a
function $F$ such that $\dom(F) = \Seq{1}{n}$.
\end{definition}

\end{forthel}

The members $F(i)$ of a sequence $F$ are often
written in an indexed notation $F_{i}$.
This is just an alias at the parsing level:

\begin{forthel}
Let $F_{i}$ stand for $F(i)$.

\begin{definition} Let $F$ be a sequence of length $n$.
$\Finset{F}{1}{n} = \{ F_{i} | i \in \dom(F)\}$.
\end{definition}
\end{forthel}

This definition uses the \LaTeX{}-macro

\verb+\newcommand{\Finset}[3]{\{#1_{#2},\dots,#1_{#3}\}}+

Dot notation is also used for iterations of all sorts.
For Euclid's theorem we shall want to consider products
of finitely many prime numbers. So we postulate
axiomatically:

\begin{forthel}

\begin{signature} Let $F$ be a sequence of length $n$
such that $\Finset{F}{1}{n} \subseteq \mathbb{N}$.
$\Prod{F}{1}{n}$ is a natural number.
\end{signature}

\begin{axiom}[Factorproperty] Let $F$ be a sequence of length $n$
such that $F(i)$ is a nonzero natural number for every $i \in \dom(F)$.
Then $\Prod{F}{1}{n}$ is nonzero and
$F(i)$ divides $\Prod{F}{1}{n}$ for every $i \in \dom(F)$.
\end{axiom}


\end{forthel}

Note that we can name toplevel sections by single words like ``Factorproperty''
or numbers. These can be referenced later in the form ``(by Factorproperty)''.

\end{document}
