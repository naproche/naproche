\documentclass{stex}
\libinput{preamble}
\begin{document}

\section{ForTheL and \LaTeX}

ForTheL is intended to be faithful to the mathematical
language as found in mathematical books and papers.
These are usually typeset in \LaTeX which in particular
allows the rendering of complex symbolism. This motivates the
integration of ForTheL into \LaTeX. Currently, \Naproche{}
accepts two dialects of the Formula Theory Language:
an ASCII-based one indicated by the .ftl file ending, as used above,
and a LaTeX-oriented version with a .ftl.tex file ending.
The latter can be immediately processed by \LaTeX{} for
high-quality mathematical typesetting.

The development of \Naproche{} is directed towards the \LaTeX-format,
also because typesetting contains semantic information
that may be exploited in the proof checking process.
In \LaTeX, e.g., the variable $a$ is distinguished from the word ``a''
by an italic font. One may also introduce custom \LaTeX{} environments
to signal specific processing of parts of texts.
Translating \verb+.ftl+-files to \verb+.ftl.tex+ is usually
straightforward. ForTheL-constructs like \verb+Definition. ...+
or \verb+Theorem. ...+ obviously correspond to definition or
theorem environments in \LaTeX.

In this tutorial, we mainly use the \verb+.ftl.tex+-format.
The ASCII-format will, however, be supported for some time
since it allows for experimentation with \Naproche{} without the
effort of correct \LaTeX{} syntax. Whereas finished \Naproche{}
formalizations should be written in \LaTeX{} format and be accompanied
by pdf-printouts, the ASCII format might be particularly important
in didactical situations which emphasize logical and mathematical content.

\subsection{Natural Numbers in .ftl.tex}

We introduce the language of arithmetic in the \LaTeX{} dialect:
start a new file with the \LaTeX-ending \path{.ftl.tex}, containing:

\begin{verbatim}
\begin{forthel}

[synonym number/numbers]

\begin{signature}  A natural number is a mathematical object.
\end{signature}

Let $m,n$ denote natural numbers.
\begin{signature} $0$ is a natural number.
\end{signature}

Let $x$ is nonzero stand for $x \neq 0$.

\begin{signature} $1$ is a nonzero natural number.
\end{signature}

\begin{signature} $m + n$ is a natural number.
\end{signature}

\begin{signature} $m * n$ is a natural number.
\end{signature}
\end{forthel}
\end{verbatim}

This file which will also be the beginning of our formalization
of Euclid's theorem on the infinitude of primes.

At the same time it is a \LaTeX{} file which can by typeset as a
\LaTeX{} document in the context of an adequate \LaTeX{} preamble

\begin{verbatim}
\documentclass[11pt]{article}
\usepackage{amssymb}
\usepackage{exercises}
\usepackage{url}
\usepackage{../lib/tex/naproche}

\begin{document}
.....
.....
\end{document}
\end{verbatim}

The \LaTeX style file \path{../lib/tex/naproche} defines the
forthel environment so that ForTheL content, i.e., the
logical core of a text, is typeset against a light grey background:

\begin{greybox}

[synonym number/numbers]

\begin{signature}  A natural number is a mathematical object.
\end{signature}

Let $m,n$ denote natural numbers.

\begin{signature} $0$ is a natural number.
\end{signature}

Let $x$ is nonzero stand for $x \neq 0$.

\begin{signature} $1$ is a nonzero natural number.
\end{signature}

\begin{signature} $m + n$ is a natural number.
\end{signature}

\begin{signature} $m * n$ is a natural number.
\end{signature}
\end{greybox}

\begin{exercise}
Convert the initial formalization of arithmetic to the \LaTeX{} format.
\end{exercise}

\begin{exercise}
Experiment with the \LaTeX{} formalization to see whether greek letters,
letters in other fonts like bold, fracture, ..., other symbols,
possibly with decorations
like $\pi^+$ or $\tilde{\otimes}$ can be used as variables or
operation symbols.
\end{exercise}

\subsection{Literate Formalizations}
Since only content in
\begin{quotation}
\verb_\_\verb_begin{forthel} ... _\verb_\_\verb_end{forthel}_
\end{quotation}
\noindent environments is passed to \Naproche{} one can write arbitrary material
outside those environments. This allows to insert arbitrary comments,
including sectioning commands or even pictures, into
\verb+.ftl.tex+ formalizations.

As was mentiioned before, the \LaTeX{} source \verb+TUTORIAL.ftl.tex+ of this document
is a proper ForTheL text which proof-checks successfully in
\Naproche{}. One can experiment with this text by loading it into the Isabelle-PIDE
and editing the
forthel environments. It may be convenient to
deactivate some of these environments by replacing the outer
\verb_\_\verb_begin_ and \verb_\_\verb_end_ by
\verb_begin_ and \verb_end_.

\begin{exercise}
Augment the \LaTeX formalization of arithmetic so that it
prettyprints with a title, abstract, some comments, and a
small bibliography whilst still being a correct ForTheL formalization.
\end{exercise}

\begin{exercise}
Write a formalization of the real numbers as a ring in a language
that has $+,\cdot,-,0,1$. You may use the ASCII format (observe that
in that format, lines starting with a \verb+#+ are interpreted
as comment).
\end{exercise}

\end{document}
