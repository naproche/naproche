\documentclass{stex}
\libinput{preamble}
\begin{document}

\section{Quick Start: $1 + 2 = 3$}

Let us prove our first theorem in Isabelle/Naproche:

\begin{itemize}
\item Download and install Isabelle2022 from
\path{https://isabelle.in.tum.de/} (this may take several minutes).

\item Start Isabelle jEdit which may again some time and
start editing a new file \path{temp.ftl}; \path{.ftl} files are recognized
as ForTheL files and are continuously checked by Isabelle/Naproche.

\item Copy the following bit of number theory into \path{temp.ftl}:
\begin{verbatim}
[synonym number/numbers]
Signature. A natural number is a mathematical object.
Signature. 0 is a natural number.
Signature. 1 is a natural number.
Signature. Assume that k,l are natural numbers.
k + l is a natural number.
Definition. 2 = 1 + 1.
Definition. 3 = 2 + 1.
Axiom. Assume that k,l,m are natural numbers.
(k+l)+m=k+(l+m).
Theorem. 1 + 2 = 3.
\end{verbatim}

\item The Output window of Isabelle/Naproche displays the checking
process, ending in \verb+verification successful+ and
some statistics.
Further feedback is given by coloured highlighting of the buffer and by
ballons when hovering over sentences with the mouse pointer.

\item The theorem $1+2=3$ has been verified!
\end{itemize}

\subsection{Explanation of the ForTheL Text}
The above text consists of a sequence of ForTheL ``commands'' that steer its
parsing and logical processing. The ForTheL language is designed
in a way that appears natural to human mathematicians although it is a
completely formal language like a programming language.

\begin{itemize}

\item \verb+[synonym number/numbers]+ is a linguistic command which
identifies the identifiers \verb+nummber+ and \verb+nummbers+ in the
subsequent text. This \textit{allows} to use grammatically correct singular
or plural forms. On the other hand \Naproche{} does not enforce grammatical
correctness.

\item \verb+Signature+ commands define and extend the mathematical
``vocabulary'' of the text. The command
\\\verb+Signature. A natural number is a mathematical object.+\\
has the following effects:
\begin{itemize}
\item The noun phrase ``natural number'' is added to
the vocabulary of the parser; by the previous \verb+[synonym ...]+
command one can also write ``natural numbers'' instead.
\item Internally, a unary predicate symbol
\verb+aNaturalNumber+ is generated, to be used by the first-order
logical processing. One can see this symbol by hovering the mouse
over the command: a pop-up balloon will show a translation to a
first-order formula which contains the subformula
\verb+aNaturalNumber(v0)+ with a variable \verb+v0+ inserted.
\item The first-order translation\\
\verb+forall v0 ((HeadTerm :: aNaturalNumber(v0))+\\
\verb+implies aObject(v0))+\\
is added to internal representation of the text. Note that
\Naproche{} has an inbuilt predicate \verb+aObject+ to encompass
all mathematical objects to be used in a text. This predicate is
addressed by the phrases ``object'' or ``mathematical object''.
The first-order
translation expresses that the collection of natural numbers is
a subcollection of the collection of all mathematical objects.
\item The newly introduced predicate is tagged by \verb+HeadTerm+,
to be used in further processing.
\end{itemize}

\item \verb+Signature. 0 is a natural number.+ extends the
language by a new constant symbol \verb+0+ which is intended to be
a natural number. This can be seen in the translation:
\verb+forall v0+\\
\verb+((HeadTerm :: v0 = 0) implies aNaturalNumber(v0))+.

\item \verb/Signature. Assume that k,l are natural numbers./\\
\verb/k + l is a natural number./\\
generates an
internal binary function symbol \verb/+/. The translation
shows that the (universally quantified) variables \verb/k/ and
\verb/l/ are natural numbers, and that the resulting term
\verb/k+l/ is a natural number. So the \verb/+/ operation is
of type $\mathbb{N} \times \mathbb{N} \rightarrow \mathbb{N}$, where
$\mathbb{N}$ is the standard notation for the class / set / type of
natural numbers.

\item The command \verb/Definition. 2 = 1 + 1./ also extends the
language. A new constant symbol \verb/2/ is added together with
an obvious definitional equivalence
\verb/forall v0 ((HeadTerm :: v0 = 2) iff v0 = 1+1)/.

\item The only axiom of our small number theory is added by the
command \\
\verb/Axiom. Assume that k,l,m are natural numbers./\\
\verb/(k+l)+m=k+(l+m)./\\
The first-order translation
includes the guards\\
\verb/((aNaturalNumber(k) and aNaturalNumber(l))/\\
\verb/and aNaturalNumber(m))/\\
for the universally quantified variables, as explained above.

\item Finally
\verb/Theorem. 1 + 2 = 3./ requests that the statement $1+2=3$ be
proved. The system sends the previous first-order
translations as ``hypotheses'' and the thesis ``$1+2=3$'' as
``conjecture'' to the external automated theorem prover (ATP) in
the first-order format TPTP:
\end{itemize}
\begin{footnotesize}
\begin{verbatim}
fof(m_,hypothesis,( ! [W0] : (aNaturalNumber(W0) => aObject(W0)))).
fof(m_,hypothesis,aNaturalNumber(sz0)).
fof(m_,hypothesis,aNaturalNumber(sz1)).
fof(m_,hypothesis,( ! [W0] : ( ! [W1] : ((aNaturalNumber(W0) &
      aNaturalNumber(W1)) => aNaturalNumber(sdtpldt(W0,W1)))))).
fof(m_,hypothesis,((sz2 = sdtpldt(sz1,sz1)) & ( ! [W0] :
      ((W0 = sdtpldt(sz1,sz1)) => (W0 = sz2))))).
fof(m_,hypothesis,((sz3 = sdtpldt(sz2,sz1)) & ( ! [W0] :
      ((W0 = sdtpldt(sz2,sz1)) => (W0 = sz3))))).
fof(m_,hypothesis,( ! [W0] : ( ! [W1] : ( ! [W2] :
      (((aNaturalNumber(W0) & aNaturalNumber(W1)) & aNaturalNumber(W2))
      => (sdtpldt(sdtpldt(W0,W1),W2) = sdtpldt(W0,sdtpldt(W1,W2)))))))).
fof(m__,conjecture,(sdtpldt(sz1,sz2) = sz3)).
\end{verbatim}
\end{footnotesize}
\begin{itemize}
\item \Naproche{} uses internal reasoning and the eprover ATP to
derive the conjecture
along the equality chain:
$$1 + 2 = 1 + (1 + 1) = (1 + 1) + 1 = 2 + 1 = 3,$$
as one could see from the eprover proof output.
\end{itemize}

\subsection{Exercises}

\begin{exercise}
Extend the above number theory by further constants and prove
equalities like $2 + 3 = 5$. What about $2 + 2 = 5$? What about
$0+0$? Can
you prove that $2 + 2 \neq 5$? (In the ftl-language, $\neq$ is expressed
by the ASCII combination \verb+!=+.)
\end{exercise}

\begin{exercise}
One can define the order relation $\leq$ in our number theory
by\\
\verb/Definition. k \leq l iff there exists a natural number m/\\
\verb/such that k + m = l./\\
Which axioms for partial or linear orders can now be proved about
\verb+\leq+?
\end{exercise}

\begin{exercise} (Pretyping of variables) In our number theory,
the variables \verb+k,l,m+ in the definition of $+$ or in the
associativity axiom are typed by explicit assumptions. Instead one
can ``pretype'' variables until the end of a file by a command of
the form\\
\verb+Let k,l,m denote natural numbers.+\\
Eliminate the type assumptions by pretyping.
What happens, if a variable does not have a previous
type assumption?
\end{exercise}

\begin{exercise} Experiment with your \verb+.ftl+ texts.
What happens if one exchanges \verb+Definition+ by
\verb+Signature+ and vice versa?
\end{exercise}

\end{document}
