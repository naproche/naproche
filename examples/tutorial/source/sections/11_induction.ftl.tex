\documentclass{stex}
\libinput{preamble}
\begin{document}

\section{Induction}

\Naproche{} has inherited an elegant treatment of induction
from the SAD system. \Naproche{} has a special binary relation
symbol $\prec$ for a universal inductive relation: if
the property $P$ is \textit{inductive} along $\prec$, i.e., if at any
point $m$ property $P$ is inherited at $m$ provided all
$\prec$-predecessors of $m$ satisfy $P$, then $P$ holds everywhere.
More formally the general induction principle for the relation
$\prec$ reads:
\begin{quotation}
To prove $\forall k \phi(k)$, it suffices to prove
inductivity along $\prec$:
$$\forall v_0 (\forall v_1 (v_1 \prec v_0 \rightarrow \phi(v_1))
\rightarrow \phi(v_0).$$
\end{quotation}

This modification of proof tasks
is immediately carried out by the parser when it comes across the
keyword ``proof by induction''.

Induction along the relation $<$ on natural numbers is introduced
axiomatically by embedding the $<$-relation into $\prec$:

\begin{forthel}
\begin{axiom} If $n < m$ then $n \prec m$.
\end{axiom}
\end{forthel}

Let us consider a proof by induction that $\leq$ is
a ``discrete'' order with nothing strictly between
$0$ and $1$.

\begin{forthel}
\begin{lemma} For every $n$ $n = 0$ or $1 \leq n$.
\end{lemma}
\begin{proof}[by induction on $n$]
\end{proof}
\end{forthel}

The automatic translation of the statement of the lemma is
\begin{verbatim}
forall v0 (aNaturalNumber(v0) implies
((InductionHypothesis ::
forall v1 (aNaturalNumber(v1) implies (iLess(v1,v0) implies
(v1 = 0 or 1\leq v1))))
implies (v0 = 0 or 1\leq v0)))
\end{verbatim}

The formula expresses that the property ``$n = 0$ or $1 \leq n$''
is \textit{inductive}. Proving the universal formula
``For every $n$ $n = 0$ or $1 \leq n$'' has been reduced to
proving inductivity. In this case the inductivity task can be
carried out automatically by
eprover.

Remark: It seems difficult to prove this inductivity by a short intuitive
proof. Apparently the above axiomatization is not standard and should
be replaced by an approach where the natural numbers are generated from
$0$ by a $+1$-operation, expressed by using $n \prec n+1$ as
induction axiom.

With the previous lemma we can show:

\begin{forthel}
\begin{lemma} If $m \neq 0$ then $n \leq n * m$.
\end{lemma}
\end{forthel}

\end{document}
