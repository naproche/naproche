\documentclass{stex}
\libinput{preamble}
\begin{document}

\section{The Natural Order - Defining \\Relations and Functions}

Definitions extend the first-order language by defined symbols
as in the following examples concerning the ordering
of the natural numbers. A definition corresponds to a
Signature command in which a symbol is introduced plus
an Axiom containing the defining property.

\begin{forthel}

\begin{definition} $m \leq n$ iff
there exists a natural number $l$ such that
$m + l = n$.
\end{definition}

Let $m < n$ stand for $m \leq n$ and $m \neq n$.

\begin{definition} Assume that $n \leq m$.
$m - n$ is a natural number $l$ such that $n + l  = m$.
\end{definition}

\end{forthel}

The first definition defines the binary relation $\leq$ by an
``iff'' equivalence. This is followed by a purely
syntactic definition of $<$. $m < n$ is simply an abbreviation
for another formula. Such abbreviations and aliases are already expanded
during parsing, possibly recursively. The third definition
defines the binary difference function $-$.

\begin{exercise}
Introduce aliases for numbers and arithmetic operations
so that one can prove:
``The sum of one and three is equal to five''.
\end{exercise}

\begin{exercise}
Introduce the square $m^2$ of a natural number and
prove the binomial formula for $(m+n)^2$.
\end{exercise}

\subsection{Axiomatic Content in Definitions}

Definitions of functions and constants usually contain
implicit postulates corresponding to the existence
and uniquess-properties of function values and constants. In
the above definition of the difference ($-$), the condition for
$l$ should be
satisfiable by a unique natural number. This is however \textit{not}
checked by \Naproche, so that the well-definedness of the
function is the user's responsibility. If the function
definition were non-unique we could have a contradictory
system of assumptions. Consider, e.g., the wrong definition

\begin{definition} Assume that $n \leq m$.
$m - n$ is a natural number $l$ such that $n = m$.
\end{definition}

The first-order translation would be

\begin{verbatim}
(aNaturalNumber(m) and aNaturalNumber(n))
n\leq m
forall v0 ((HeadTerm :: v0 = m-n)
   iff (aNaturalNumber(v0) and n = m))
\end{verbatim}

Every number fits the defining equivalence provided that $m = n$.
But then $0 = 0 - 0 = 1$, contradiction.

With relation definitions, such problems do not arise.

\end{document}
