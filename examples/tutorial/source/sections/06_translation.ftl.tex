\documentclass{stex}
\libinput{preamble}
\begin{document}

\section{On the Translation from Natural to First-Order Language}

Above we have introduced
the functions $+$ and $*$ of addition and multiplication
(of natural numbers)
and the constants $0$ and $1$. Domains are captured by
unary relations. The type ``natural number''
of ordinary mathematical discourse is modeled by the
internal unary relation symbol
\verb+aNaturalNumber+, and the arithmetic functions and
quantifiers are
restricted to the extension of the unary relation symbol.
So the (weak) type system of ordinary mathematical language
is modeled by a system of first-order predicates. These
types do not follow any strict ``type theory'' with specific
mathematical laws but they are powerful enough
to organize the universe of mathematics.

\subsection{First-Order Translation}
The ForTheL code for introducing the type, or rather
notion, of natural numbers, the constants $0$ and $1$
and the operations of $+$ and $*$ has the following
first-order translation
which can be found in the output
window of jEdit or hovering the mouse
over the sentence:
\begin{small}
\begin{verbatim}
1. forall v0 ((HeadTerm :: aNaturalNumber(v0)) implies
     aObject(v0))
2. forall v0 ((HeadTerm :: v0 = 0) implies aNaturalNumber(v0))
3. forall v0 ((HeadTerm :: v0 = 1) implies
     (aNaturalNumber(v0) and not v0 = 0))
4. (aNaturalNumber(m) and aNaturalNumber(n))
5. forall v0 ((HeadTerm :: v0 = m+n) implies aNaturalNumber(v0))
6. (aNaturalNumber(m) and aNaturalNumber(n))
7. forall v0 ((HeadTerm :: v0 = m*n) implies aNaturalNumber(v0))
\end{verbatim}
\end{small}
In these formulas we see the newly introduced first-order symbols:

\verb_aNaturalNumber(v0), 0, 1, +, *_.

The first-order translations follow a certain idiom which
is favourable for the overall processing. Formula 1
exhibits
the new symbol marked by the tag \verb+HeadTerm+. Similarly
formula 2 emphasizes the symbol \verb+0+ which would not have been
the case in the equivalent \verb+aNaturalNumber(0)+.
Note that 5 and 7 both have the premises

\verb+(aNaturalNumber(m) and aNaturalNumber(n))+

\noindent for the two arguments of the operations.

The correctness of mathematical texts includes the principle of
\textit{ontological correctness} which in this case means that
such premises have to
be proved before the operations can reasonably be applied within
a proof. \Naproche{} checks texts sentence by sentence.
The first check is for the indicated ontological correctness
of the sentence. Only if that is successful the logical correctness
will be checked if the sentence is an assertion of a mathematical fact;
Assumptions, or statements in signature or definition commands will
be treated differently.

\begin{exercise}
Extract the first-order translation of one of your formalizations.
You can put the command \verb+[dump on]+ into your ForTheL text
to increase the amount
of information in the output buffer of Isabelle.
\end{exercise}

\subsection{Some ForTheL Commands and Keywords}

Let us now go through the natural language phrases used to
reach this translation. New first-order symbols are spawned by
Signature commands. The new notion comes before the keyword ``is''
after which the new notion is classified as a new type (``is a notion'')
or as a member of of an existing type (``is a natural number'').

The phrase before ``is'' is read as a new language pattern that
the parser learns. A pattern has some word tokens,
like ``natural'',
``number'', or some symbolic tokens, like
``$0$'', ``$1$'', ``$+$'', ``$*$''.
In between those tokens a pattern may have holes for the insertion
of terms, which in the Signature command are indicated by previously
introduced variables, like ``$m$'' or ``$n$''. These were introduced in
the parser command ``Let $m,n$ denote natural numbers.''
Thereafter, $m$ and $n$ are variables which are ``pretyped'' to be
natural numbers. With that,
\begin{signature} $m + n$ is a natural number.
\end{signature}
has the ``double translation''
\begin{verbatim}
(aNaturalNumber(m) and aNaturalNumber(n))
forall v0 ((HeadTerm :: v0 = m+n) implies aNaturalNumber(v0))
\end{verbatim}
where the first (or more) formulas are premises and the last contains the
newly introduced symbol.

We can also qualify the typing on the right-hand side of the
``is'' keyword
by first-order formulae. In our example, we have introduced a pattern for
a first-order formula by the parser command
``Let $x$ is nonzero stand for $x \neq 0$.''
This formula is then applied
as an adjective in the next Signature command
\begin{signature} $1$ is a nonzero natural number.
\end{signature}
Note that some natural language processing is also taking place:
``nonzero'' is introduced within the phrase ``$x$ is nonzero''
in an
adjective position. So in the Signature command, ``nonzero''
can be
used as an adjective which modifies ``natural number''.
The first-order
effect of this is a conjunction
\begin{verbatim}
3. forall v0 ((HeadTerm :: v0 = 1) implies
     (aNaturalNumber(v0) and not v0 = 0))
\end{verbatim}
The equality ``$=$'' and inequality ``$\neq$'' are predefined
phrases with corresponding first-order symbols.

\subsection{``Grammar''}

Note that we have also used``linguistic'' commands:
the command \verb+[synonym number/numbers]+
identifies the tokens ``number'' and ``numbers'', providing the
plural form. The command can be abbreviated to
\verb+[synonym number/-s]+. This is a simple linguistic
``hack'' which allows grammatically correct forms. But it
also allows wrong ones, and it is up to the user to make the right
choices.

\end{document}
