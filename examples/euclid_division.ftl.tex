\documentclass{article}
\usepackage{naproche}
\libinput[meta-inf]{example-preamble}
\libinput[libraries]{foundations-preamble}
\libinput[libraries]{arithmetics-preamble}

\title{Euclid's Division Theorem}
\author{\Naproche formalization:\\[0.5em]Marcel Schütz}
\date{2024}

\begin{document}
  \maketitle

  \usemodule[libraries]{arithmetics?multiplication-and-ordering.ftl}
  \usemodule[libraries]{arithmetics?subtraction.ftl}

  \noindent This is a formalization of Euclid's Division Theorem
  (within Peano Arithmetic) which states
  that for every pair of natural numbers $n$ and $m$ with $m \neq 0$ there
  exist unique natural numbers $q$ and $r$ such that
  \[n = mq + r\]
  with $r < m$  (cf. \cite[p. 17]{Burton2005}).

  \begin{forthel}
    \begin{theorem}[title=Euclid's Division Theorem: Existence,id=euclid_existence]
      Let $n, m$ be natural numbers such that $m \neq 0$.
      Then there exist natural numbers $q, r$ such that
      \[n = (m \cdot q) + r\]
      and $r < m$.
    \end{theorem}
    \begin{proof}
      (a) Define $\Phi = \{ n' \in \Nat \mid$ there exist natural numbers $q, r$ such that $r < m$ and $n' = (m \cdot q) + r \}$.

      (1) $\Phi$ contains $0$.
      Proof.
        Take $q = 0$ and $r = 0$.
        Then $r < m$ and $0 = (m \cdot q) + r$.
        Hence $0 \in \Phi$.
      Qed.

      (2) For all $n' \in \Phi$ we have $n' + 1 \in \Phi$.
      Proof.
        Let $n' \in \Phi$.

        Let us show that there exist natural numbers $q, r$ such that $r < m$ and $n' + 1 = (m \cdot q) + r$.
          Take natural numbers $q', r'$ such that $r' < m$ and $n' = (m \cdot q') + r'$ (by a).
          We have $r' + 1 < m$ or $r' + 1 = m$.

          Case $r' + 1 < m$.
            Take $q = q' + 0$ and $r = r' + 1$. %!
            Then $r < m$ and $n' + 1
              = ((q' \cdot m) + r') + 1
              = (q' \cdot m) + (r' + 1)
              = (q \cdot m) + r$.
          End.

          Case $r' + 1 = m$.
            Take  $q = q' + 1$ and $r = 0$.
            Then $r < m$ and
            $n' + 1
              = ((q' \cdot m) + r') + 1
              = (q' \cdot m) + (r' + 1)
              = (q' \cdot m) + m
              = (q' \cdot m) + (1 \cdot m)
              = (q' + 1) \cdot m
              = (q \cdot m) + r$.
          End.
        End.

        Hence $n' + 1 \in \Phi$.
      Qed.

      Then $\Phi$ contains every natural number (by \printref{ARITHMETIC_01_4764664342773760}).
      Thus there exist natural numbers $q, r$ such that $n = (m \cdot q) + r$ and $r < m$ (by a).
    \end{proof}
  \end{forthel}

  \begin{forthel}
    \begin{theorem}[title=Euclid's Division Theorem: Uniqueness,id=euclid_uniqueness]
      Let $n, m$ be natural numbers such that $m \neq 0$.
      Let $q, r$ be natural numbers such that
      \[n = (m \cdot q) + r\]
      and $r < m$.
      Let $q', r'$ be natural numbers such that
      \[n = (m \cdot q') + r'\]
      and $r' < m$.
      Then $q = q'$ and $r = r'$.
    \end{theorem}
    \begin{proof}
      We have $(m \cdot q) + r = (m \cdot q') + r'$.

      Case $q \geq q'$ and $r \geq r'$.
        (1) $((m \cdot q) + r) - r' = (m \cdot q) + (r - r')$ (by \printref{ARITHMETIC_05_7578468875239424}).
        (2) $((m \cdot q') + r') - r'
          = (m \cdot q') + (r' - r')
          = m \cdot q'$.
        Hence $(m \cdot q) + (r - r') = m \cdot q'$.
        Thus $((m \cdot q) - (m \cdot q')) + (r - r') = 0$.
        Consequently $(m \cdot q) - (m \cdot q') = 0$ and $r - r' = 0$.
        If $(m \cdot q) - (m \cdot q') = 0$ then $q - q' = 0$.
        Therefore $q - q' = 0$ and $r - r' = 0$.
        Thus we have $q = q'$ and $r = r'$.
      End.

      Case $q \geq q'$ and $r < r'$.
        Take $q'' = q - q'$ and $r'' = r' - r$.
        Then $(m \cdot (q' + q'')) + r = (m \cdot q') + (r + r'')$.
        We have $(m \cdot q') + (r + r'')
          = (m \cdot q') + (r'' + r)
          = ((m \cdot q') + r'') + r$.
        Hence $(m \cdot (q' + q'')) + r = ((m \cdot q') + r'') + r$.
        Thus $m \cdot (q' + q'') = (m \cdot q') + r''$ (by \printref{ARITHMETIC_03_3137702874578944}).
        We have $m \cdot (q' + q'') = (m \cdot q') + (m \cdot q'')$.
        Hence $(m \cdot q') + (m \cdot q'') = (m \cdot q') + r''$.
        Thus $m \cdot q'' = r''$ (by \printref{ARITHMETIC_03_8445946379632640}).
        Then we have $m \cdot q'' < m \cdot 1$.
        Indeed $m \cdot q''
          = r''
          \leq r'
          < m
          = m \cdot 1$.
        Therefore $q'' < 1$ (by \printref{ARITHMETIC_06_5048640368279552}).
        Consequently $q - q' = q'' = 0$.
        Hence $q = q'$.
        Thus $(m \cdot q) + r = (m \cdot q) + r'$.
        Therefore $r = r'$.
      End.

      Case $q < q'$ and $r \geq r'$.
        Take $q'' = q' - q$ and $r'' = r - r'$.
        Then $(m \cdot q) + (r' + r'') = (m \cdot (q + q'')) + r'$.
        We have $(m \cdot q) + (r' + r'')
          = (m \cdot q) + (r'' + r')
          = ((m \cdot q) + r'') + r'$.
        Hence $((m \cdot q) + r'') + r' = (m \cdot (q + q'')) + r'$.
        Thus $(m \cdot q) + r'' = m \cdot (q + q'')$ (by \printref{ARITHMETIC_03_3137702874578944}).
        We have $m \cdot (q + q'') = (m \cdot q) + (m \cdot q'')$.
        Hence $(m \cdot q) + r'' = (m \cdot q) + (m \cdot q'')$.
        Thus $r'' = m \cdot q''$ (by \printref{ARITHMETIC_03_8445946379632640}).
        Then we have $m \cdot q'' < m \cdot 1$.
        Indeed $m \cdot q''
          = r''
          \leq r
          < m
          = m \cdot 1$.
        Therefore $q'' < 1$ (by \printref{ARITHMETIC_06_5048640368279552}).
        Consequently $q' - q = q'' = 0$.
        Hence $q' = q$.
        Thus $(m \cdot q) + r = (m \cdot q) + r'$.
        Therefore $r = r'$.
      End.

      Case $q < q'$ and $r < r'$.
        (1) $((m \cdot q') + r') - r = (m \cdot q') + (r' - r)$ (by \printref{ARITHMETIC_05_7578468875239424}).
        (2) $((m \cdot q) + r) - r
          = (m \cdot q) + (r - r)
          = m \cdot q$.
        Hence $(m \cdot q') + (r' - r) = m \cdot q$.
        Thus $((m \cdot q') - (m \cdot q)) + (r' - r)
          = ((m \cdot q') + (r' - r)) - (m \cdot q)
          = 0$.
        Consequently $(m \cdot q') - (m \cdot q) = 0$ and $r' - r = 0$.
        If $(m \cdot q') - (m \cdot q) = 0$ then $q' - q = 0$.
        Therefore $q' - q = 0$ and $r' - r = 0$.
        Thus we have $q' = q$ and $r' = r$.
      End.
    \end{proof}
  \end{forthel}

  \printbibliography
  \printlicense[CcByNcSa]{2024}{Marcel Schütz}
\end{document}
