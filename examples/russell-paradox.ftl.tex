\documentclass{article}
\usepackage{naproche}
\libinput[meta-inf]{example-preamble}
\libinput[libraries]{foundations-preamble}

\title{Russell's Paradox}
\author{\Naproche formalization:\\[0.5em]Marcel Schütz}
\date{2024}

\begin{document}
  \pagenumbering{gobble}
  \maketitle

  \noindent \emph{Russell's Paradox} is a set-theoretical paradox discovered by
  Bertrand Russell around 1902 \cite[chapter XV]{Frege1980} which shows that 
  there exist statements $\varphi$ whose extension
  $\{x\mid\varphi(x)\}$ cannot constitute a set -- or in other words:
  Not every class is a set.

  \begin{forthel}
    \begin{theorem*}[title=Russell's Paradox,id=russell_paradox]
      It is wrong that every class is a set.
    \end{theorem*}
    \begin{proof}
      Assume the contrary.
      Define $R = \{x \mid x$ is a set such that $x \notin x\}$.
      Then $R$ is a set.
      Hence $R \in R$ iff $R \notin R$.
      Contradiction.
    \end{proof}
  \end{forthel}

  \printbibliography

  \section*{Copyright}
  \doclicenseThis
\end{document}
