\documentclass{article}
\usepackage[utf8]{inputenc}
\usepackage[english]{babel}
\usepackage{../lib/tex/naproche}
\usepackage{amssymb}
\usepackage{pgffor}

\newcommand{\id}{\textrm{id}}
\newcommand{\range}{\textrm{range}}
\newcommand{\const}[2]{\textrm{const}_{#1}^{#2}}
\newcommand{\symdiff}{\mathop{\triangle}}
\newcommand{\pow}{\mathcal{P}}
\newcommand{\onto}{\twoheadrightarrow}
\newcommand{\into}{\hookrightarrow}

% E.g. `\set{1, 2, 3, 4}` prints the numbers 1, 2, 3 and 4 as a comma
% separated list which is enclosed within curly braces.
\def\set#1{\ensuremath{
  \{\foreach[count=\i]\x in {#1}{
    \ifnum\i>1,\,\fi
    \x
  }\}
}}

\renewcommand{\succ}{\textrm{succ}}
\newcommand{\pred}{\textrm{pred}}
\newcommand{\V}{\textbf{V}}
\newcommand{\Ord}{\textbf{Ord}}
\newcommand{\Card}{\textbf{Card}}


\usepackage[backend=bibtex]{biblatex}
\usepackage{csquotes}
\addbibresource{meta-inf/lib/bibliography}

\title{Russell's Paradox}
\author{\Naproche formalization: \vspace{0.5em} \\
Marcel Schütz}
\date{2024}

\begin{document}
  \pagenumbering{gobble}
  \maketitle

  \noindent \emph{Russell's Paradox} is a set-theoretical paradox discovered by
  Bertrand Russell around 1902 \cite[chapter XV]{Frege1980} which shows that 
  there exist statements $\varphi$ whose extension
  $\{x\mid\varphi(x)\}$ cannot constitute a set -- or in other words:
  Not every class is a set.

  \begin{forthel}
    \begin{theorem*}[Russell's Paradox]\label{russell_paradox}
      It is wrong that every class is a set.
    \end{theorem*}
    \begin{proof}
      Assume the contrary.
      Define $R = \{x \mid x$ is a set such that $x \notin x\}$.
      Then $R$ is a set.
      Hence $R \in R$ iff $R \notin R$.
      Contradiction.
    \end{proof}
  \end{forthel}

  \printbibliography
\end{document}
